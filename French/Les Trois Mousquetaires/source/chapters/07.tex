%!TeX root=../musketeersfr.tex 

\chapter{L'Intérieur Des Mousquetaires}

\lettrine{L}{orsque} d'Artagnan fut hors du Louvre, et qu'il consulta ses amis sur l'emploi qu'il devait faire de sa part des quarante pistoles, Athos lui conseilla de commander un bon repas à la Pomme de Pin, Porthos de prendre un laquais, et Aramis de se faire une maîtresse convenable. 

Le repas fut exécuté le jour même, et le laquais y servit à table. Le repas avait été commandé par Athos, et le laquais fourni par Porthos. C'était un Picard que le glorieux mousquetaire avait embauché le jour même et à cette occasion sur le pont de la Tournelle, pendant qu'il faisait des ronds en crachant dans l'eau. 

Porthos avait prétendu que cette occupation était la preuve d'une organisation réfléchie et contemplative, et il l'avait emmené sans autre recommandation. La grande mine de ce gentilhomme, pour le compte duquel il se crut engagé, avait séduit Planchet --- c'était le nom du Picard ---; il y eut chez lui un léger désappointement lorsqu'il vit que la place était déjà prise par un confrère nommé Mousqueton, et lorsque Porthos lui eut signifié que son état de maison, quoi que grand, ne comportait pas deux domestiques, et qu'il lui fallait entrer au service de d'Artagnan. Cependant, lorsqu'il assista au dîner que donnait son maître et qu'il vit celui-ci tirer en payant une poignée d'or de sa poche, il crut sa fortune faite et remercia le Ciel d'être tombé en la possession d'un pareil Crésus; il persévéra dans cette opinion jusqu'après le festin, des reliefs duquel il répara de longues abstinences. Mais en faisant, le soir, le lit de son maître, les chimères de Planchet s'évanouirent. Le lit était le seul de l'appartement, qui se composait d'une antichambre et d'une chambre à coucher. Planchet coucha dans l'antichambre sur une couverture tirée du lit de d'Artagnan, et dont d'Artagnan se passa depuis. 

Athos, de son côté, avait un valet qu'il avait dressé à son service d'une façon toute particulière, et que l'on appelait Grimaud. Il était fort silencieux, ce digne seigneur. Nous parlons d'Athos, bien entendu. Depuis cinq ou six ans qu'il vivait dans la plus profonde intimité avec ses compagnons Porthos et Aramis, ceux-ci se rappelaient l'avoir vu sourire souvent, mais jamais ils ne l'avaient entendu rire. Ses paroles étaient brèves et expressives, disant toujours ce qu'elles voulaient dire, rien de plus: pas d'enjolivements, pas de broderies, pas d'arabesques. Sa conversation était un fait sans aucun épisode. 

Quoique Athos eût à peine trente ans et fût d'une grande beauté de corps et d'esprit, personne ne lui connaissait de maîtresse. Jamais il ne parlait de femmes. Seulement il n'empêchait pas qu'on en parlât devant lui, quoiqu'il fût facile de voir que ce genre de conversation, auquel il ne se mêlait que par des mots amers et des aperçus misanthropiques, lui était parfaitement désagréable. Sa réserve, sa sauvagerie et son mutisme en faisaient presque un vieillard; il avait donc, pour ne point déroger à ses habitudes, habitué Grimaud à lui obéir sur un simple geste ou sur un simple mouvement des lèvres. Il ne lui parlait que dans des circonstances suprêmes. 

Quelquefois Grimaud, qui craignait son maître comme le feu, tout en ayant pour sa personne un grand attachement et pour son génie une grande vénération, croyait avoir parfaitement compris ce qu'il désirait, s'élançait pour exécuter l'ordre reçu, et faisait précisément le contraire. Alors Athos haussait les épaules et, sans se mettre en colère, rossait Grimaud. Ces jours-là, il parlait un peu. 

Porthos, comme on a pu le voir, avait un caractère tout opposé à celui d'Athos: non seulement il parlait beaucoup, mais il parlait haut; peu lui importait au reste, il faut lui rendre cette justice, qu'on l'écoutât ou non; il parlait pour le plaisir de parler et pour le plaisir de s'entendre; il parlait de toutes choses excepté de sciences, excipant à cet endroit de la haine invétérée que depuis son enfance il portait, disait-il, aux savants. Il avait moins grand air qu'Athos, et le sentiment de son infériorité à ce sujet l'avait, dans le commencement de leur liaison, rendu souvent injuste pour ce gentilhomme, qu'il s'était alors efforcé de dépasser par ses splendides toilettes. Mais, avec sa simple casaque de mousquetaire et rien que par la façon dont il rejetait la tête en arrière et avançait le pied, Athos prenait à l'instant même la place qui lui était due et reléguait le fastueux Porthos au second rang. Porthos s'en consolait en remplissant l'antichambre de M. de Tréville et les corps de garde du Louvre du bruit de ses bonnes fortunes, dont Athos ne parlait jamais, et pour le moment, après avoir passé de la noblesse de robe à la noblesse d'épée, de la robine à la baronne, il n'était question de rien de moins pour Porthos que d'une princesse étrangère qui lui voulait un bien énorme. 

Un vieux proverbe dit: «Tel maître, tel valet.» Passons donc du valet d'Athos au valet de Porthos, de Grimaud à Mousqueton. 

Mousqueton était un Normand dont son maître avait changé le nom pacifique de Boniface en celui infiniment plus sonore et plus belliqueux de Mousqueton. Il était entré au service de Porthos à la condition qu'il serait habillé et logé seulement, mais d'une façon magnifique; il ne réclamait que deux heures par jour pour les consacrer à une industrie qui devait suffire à pourvoir à ses autres besoins. Porthos avait accepté le marché; la chose lui allait à merveille. Il faisait tailler à Mousqueton des pourpoints dans ses vieux habits et dans ses manteaux de rechange, et, grâce à un tailleur fort intelligent qui lui remettait ses hardes à neuf en les retournant, et dont la femme était soupçonnée de vouloir faire descendre Porthos de ses habitudes aristocratiques, Mousqueton faisait à la suite de son maître fort bonne figure. 

Quant à Aramis, dont nous croyons avoir suffisamment exposé le caractère, caractère du reste que, comme celui de ses compagnons, nous pourrons suivre dans son développement, son laquais s'appelait Bazin. Grâce à l'espérance qu'avait son maître d'entrer un jour dans les ordres, il était toujours vêtu de noir, comme doit l'être le serviteur d'un homme d'Église. C'était un Berrichon de trente-cinq à quarante ans, doux, paisible, grassouillet, occupant à lire de pieux ouvrages les loisirs que lui laissait son maître, faisant à la rigueur pour deux un dîner de peu de plats, mais excellent. Au reste, muet, aveugle, sourd et d'une fidélité à toute épreuve. 

Maintenant que nous connaissons, superficiellement du moins, les maîtres et les valets, passons aux demeures occupées par chacun d'eux. 

Athos habitait rue Férou, à deux pas du Luxembourg; son appartement se composait de deux petites chambres, fort proprement meublées, dans une maison garnie dont l'hôtesse encore jeune et véritablement encore belle lui faisait inutilement les doux yeux. Quelques fragments d'une grande splendeur passée éclataient çà et là aux murailles de ce modeste logement: c'était une épée, par exemple, richement damasquinée, qui remontait pour la façon à l'époque de François I\ier\, et dont la poignée seule, incrustée de pierres précieuses, pouvait valoir deux cents pistoles, et que cependant, dans ses moments de plus grande détresse, Athos n'avait jamais consenti à engager ni à vendre. Cette épée avait longtemps fait l'ambition de Porthos. Porthos aurait donné dix années de sa vie pour posséder cette épée. 

Un jour qu'il avait rendez-vous avec une duchesse, il essaya même de l'emprunter à Athos. Athos, sans rien dire, vida ses poches, ramassa tous ses bijoux: bourses, aiguillettes et chaînes d'or, il offrit tout à Porthos; mais quant à l'épée, lui dit-il, elle était scellée à sa place et ne devait la quitter que lorsque son maître quitterait lui-même son logement. Outre son épée, il y avait encore un portrait représentant un seigneur du temps de Henri III vêtu avec la plus grande élégance, et qui portait l'ordre du Saint-Esprit, et ce portrait avait avec Athos certaines ressemblances de lignes, certaines similitudes de famille, qui indiquaient que ce grand seigneur, chevalier des ordres du roi, était son ancêtre. 

Enfin, un coffre de magnifique orfèvrerie, aux mêmes armes que l'épée et le portrait, faisait un milieu de cheminée qui jurait effroyablement avec le reste de la garniture. Athos portait toujours la clef de ce coffre sur lui. Mais un jour il l'avait ouvert devant Porthos, et Porthos avait pu s'assurer que ce coffre ne contenait que des lettres et des papiers: des lettres d'amour et des papiers de famille, sans doute. 

Porthos habitait un appartement très vaste et d'une très somptueuse apparence, rue du Vieux-Colombier. Chaque fois qu'il passait avec quelque ami devant ses fenêtres, à l'une desquelles Mousqueton se tenait toujours en grande livrée, Porthos levait la tête et la main, et disait: \textit{Voilà ma demeure!} Mais jamais on ne le trouvait chez lui, jamais il n'invitait personne à y monter, et nul ne pouvait se faire une idée de ce que cette somptueuse apparence renfermait de richesses réelles. 

Quant à Aramis, il habitait un petit logement composé d'un boudoir, d'une salle à manger et d'une chambre à coucher, laquelle chambre, située comme le reste de l'appartement au rez-de-chaussée, donnait sur un petit jardin frais, vert, ombreux et impénétrable aux yeux du voisinage. 

Quant à d'Artagnan, nous savons comment il était logé, et nous avons déjà fait connaissance avec son laquais, maître Planchet. 

D'Artagnan, qui était fort curieux de sa nature, comme sont les gens, du reste, qui ont le génie de l'intrigue, fit tous ses efforts pour savoir ce qu'étaient au juste Athos, Porthos et Aramis; car, sous ces noms de guerre, chacun des jeunes gens cachait son nom de gentilhomme, Athos surtout, qui sentait son grand seigneur d'une lieue. Il s'adressa donc à Porthos pour avoir des renseignements sur Athos et Aramis, et à Aramis pour connaître Porthos. 

Malheureusement, Porthos lui-même ne savait de la vie de son silencieux camarade que ce qui en avait transpiré. On disait qu'il avait eu de grands malheurs dans ses affaires amoureuses, et qu'une affreuse trahison avait empoisonné à jamais la vie de ce galant homme. Quelle était cette trahison? Tout le monde l'ignorait. 

Quant à Porthos, excepté son véritable nom, que M. de Tréville savait seul, ainsi que celui de ses deux camarades, sa vie était facile à connaître. Vaniteux et indiscret, on voyait à travers lui comme à travers un cristal. La seule chose qui eût pu égarer l'investigateur eût été que l'on eût cru tout le bien qu'il disait de lui. 

Quant à Aramis, tout en ayant l'air de n'avoir aucun secret, c'était un garçon tout confit de mystères, répondant peu aux questions qu'on lui faisait sur les autres, et éludant celles que l'on faisait sur lui-même. Un jour, d'Artagnan, après l'avoir longtemps interrogé sur Porthos et en avoir appris ce bruit qui courait de la bonne fortune du mousquetaire avec une princesse, voulut savoir aussi à quoi s'en tenir sur les aventures amoureuses de son interlocuteur. 

«Et vous, mon cher compagnon, lui dit-il, vous qui parlez des baronnes, des comtesses et des princesses des autres? 

\speak  Pardon, interrompit Aramis, j'ai parlé parce que Porthos en parle lui-même, parce qu'il a crié toutes ces belles choses devant moi. Mais croyez bien, mon cher monsieur d'Artagnan, que si je les tenais d'une autre source ou qu'il me les eût confiées, il n'y aurait pas eu de confesseur plus discret que moi. 

\speak  Je n'en doute pas, reprit d'Artagnan; mais enfin, il me semble que vous-même vous êtes assez familier avec les armoiries, témoin certain mouchoir brodé auquel je dois l'honneur de votre connaissance.» 

Aramis, cette fois, ne se fâcha point, mais il prit son air le plus modeste et répondit affectueusement: 

«Mon cher, n'oubliez pas que je veux être Église, et que je fuis toutes les occasions mondaines. Ce mouchoir que vous avez vu ne m'avait point été confié, mais il avait été oublié chez moi par un de mes amis. J'ai dû le recueillir pour ne pas les compromettre, lui et la dame qu'il aime. Quant à moi, je n'ai point et ne veux point avoir de maîtresse, suivant en cela l'exemple très judicieux d'Athos, qui n'en a pas plus que moi. 

\speak  Mais, que diable! vous n'êtes pas abbé, puisque vous êtes mousquetaire. 

\speak  Mousquetaire par intérim, mon cher, comme dit le cardinal, mousquetaire contre mon gré, mais homme Église dans le cœur, croyez-moi. Athos et Porthos m'ont fourré là-dedans pour m'occuper: j'ai eu, au moment d'être ordonné, une petite difficulté avec\dots Mais cela ne vous intéresse guère, et je vous prends un temps précieux. 

\speak  Point du tout, cela m'intéresse fort, s'écria d'Artagnan, et je n'ai pour le moment absolument rien à faire. 

\speak  Oui, mais moi j'ai mon bréviaire à dire, répondit Aramis, puis quelques vers à composer que m'a demandés Mme d'Aiguillon; ensuite je dois passer rue Saint-Honoré afin d'acheter du rouge pour Mme de Chevreuse. Vous voyez, mon cher ami, que si rien ne vous presse, je suis très pressé, moi.» 

Et Aramis tendit affectueusement la main à son compagnon, et prit congé de lui. 

D'Artagnan ne put, quelque peine qu'il se donnât, en savoir davantage sur ses trois nouveaux amis. Il prit donc son parti de croire dans le présent tout ce qu'on disait de leur passé, espérant des révélations plus sûres et plus étendues de l'avenir. En attendant, il considéra Athos comme un Achille, Porthos comme un Ajax, et Aramis comme un Joseph. 

Au reste, la vie des quatre jeunes gens était joyeuse: Athos jouait, et toujours malheureusement. Cependant il n'empruntait jamais un sou à ses amis, quoique sa bourse fût sans cesse à leur service, et lorsqu'il avait joué sur parole, il faisait toujours réveiller son créancier à six heures du matin pour lui payer sa dette de la veille. 

Porthos avait des fougues: ces jours-là, s'il gagnait, on le voyait insolent et splendide; s'il perdait, il disparaissait complètement pendant quelques jours, après lesquels il reparaissait le visage blême et la mine allongée, mais avec de l'argent dans ses poches. 

Quant à Aramis, il ne jouait jamais. C'était bien le plus mauvais mousquetaire et le plus méchant convive qui se pût voir\dots Il avait toujours besoin de travailler. Quelquefois au milieu d'un dîner, quand chacun, dans l'entraînement du vin et dans la chaleur de la conversation, croyait que l'on en avait encore pour deux ou trois heures à rester à table, Aramis regardait sa montre, se levait avec un gracieux sourire et prenait congé de la société, pour aller, disait-il, consulter un casuiste avec lequel il avait rendez-vous. D'autres fois, il retournait à son logis pour écrire une thèse, et priait ses amis de ne pas le distraire. 

Cependant Athos souriait de ce charmant sourire mélancolique, si bien séant à sa noble figure, et Porthos buvait en jurant qu'Aramis ne serait jamais qu'un curé de village. 

Planchet, le valet de d'Artagnan, supporta noblement la bonne fortune; il recevait trente sous par jour, et pendant un mois il revenait au logis gai comme pinson et affable envers son maître. Quand le vent de l'adversité commença à souffler sur le ménage de la rue des Fossoyeurs, c'est-à-dire quand les quarante pistoles du roi Louis XIII furent mangées ou à peu près, il commença des plaintes qu'Athos trouva nauséabondes, Porthos indécentes, et Aramis ridicules. Athos conseilla donc à d'Artagnan de congédier le drôle, Porthos voulait qu'on le bâtonnât auparavant, et Aramis prétendit qu'un maître ne devait entendre que les compliments qu'on fait de lui. 

«Cela vous est bien aisé à dire, reprit d'Artagnan: à vous, Athos, qui vivez muet avec Grimaud, qui lui défendez de parler, et qui, par conséquent, n'avez jamais de mauvaises paroles avec lui; à vous, Porthos, qui menez un train magnifique et qui êtes un dieu pour votre valet Mousqueton; à vous enfin, Aramis, qui, toujours distrait par vos études théologiques, inspirez un profond respect à votre serviteur Bazin, homme doux et religieux; mais moi qui suis sans consistance et sans ressources, moi qui ne suis pas mousquetaire ni même garde, moi, que ferai-je pour inspirer de l'affection, de la terreur ou du respect à Planchet? 

\speak  La chose est grave, répondirent les trois amis, c'est une affaire d'intérieur; il en est des valets comme des femmes, il faut les mettre tout de suite sur le pied où l'on désire qu'ils restent. Réfléchissez donc.» 

D'Artagnan réfléchit et se résolut à rouer Planchet par provision, ce qui fut exécuté avec la conscience que d'Artagnan mettait en toutes choses; puis, après l'avoir bien rossé, il lui défendit de quitter son service sans sa permission. «Car, ajouta-t-il, l'avenir ne peut me faire faute; j'attends inévitablement des temps meilleurs. Ta fortune est donc faite si tu restes près de moi, et je suis trop bon maître pour te faire manquer ta fortune en t'accordant le congé que tu me demandes.» 

Cette manière d'agir donna beaucoup de respect aux mousquetaires pour la politique de d'Artagnan. Planchet fut également saisi d'admiration et ne parla plus de s'en aller. 

La vie des quatre jeunes gens était devenue commune; d'Artagnan, qui n'avait aucune habitude, puisqu'il arrivait de sa province et tombait au milieu d'un monde tout nouveau pour lui, prit aussitôt les habitudes de ses amis. 

On se levait vers huit heures en hiver, vers six heures en été, et l'on allait prendre le mot d'ordre et l'air des affaires chez M. de Tréville. D'Artagnan, bien qu'il ne fût pas mousquetaire, en faisait le service avec une ponctualité touchante: il était toujours de garde, parce qu'il tenait toujours compagnie à celui de ses trois amis qui montait la sienne. On le connaissait à l'hôtel des mousquetaires, et chacun le tenait pour un bon camarade; M. de Tréville, qui l'avait apprécié du premier coup d'œil, et qui lui portait une véritable affection, ne cessait de le recommander au roi. 

De leur côté, les trois mousquetaires aimaient fort leur jeune camarade. L'amitié qui unissait ces quatre hommes, et le besoin de se voir trois ou quatre fois par jour, soit pour duel, soit pour affaires, soit pour plaisir, les faisaient sans cesse courir l'un après l'autre comme des ombres; et l'on rencontrait toujours les inséparables se cherchant du Luxembourg à la place Saint-Sulpice, ou de la rue du Vieux-Colombier au Luxembourg. 

En attendant, les promesses de M. de Tréville allaient leur train. Un beau jour, le roi commanda à M. le chevalier des Essarts de prendre d'Artagnan comme cadet dans sa compagnie des gardes. D'Artagnan endossa en soupirant cet habit, qu'il eût voulu, au prix de dix années de son existence, troquer contre la casaque de mousquetaire. Mais M. de Tréville promit cette faveur après un noviciat de deux ans, noviciat qui pouvait être abrégé au reste, si l'occasion se présentait pour d'Artagnan de rendre quelque service au roi ou de faire quelque action d'éclat. D'Artagnan se retira sur cette promesse et, dès le lendemain, commença son service. 

Alors ce fut le tour d'Athos, de Porthos et d'Aramis de monter la garde avec d'Artagnan quand il était de garde. La compagnie de M. le chevalier des Essarts prit ainsi quatre hommes au lieu d'un, le jour où elle prit d'Artagnan.