%!TeX root=../musketeersfr.tex 

\chapter{Fatalité}

\lettrine{C}{ependant} Milady, ivre de colère, rugissant sur le pont du bâtiment comme une lionne qu'on embarque, avait été tentée de se jeter à la mer pour regagner la côte, car elle ne pouvait se faire à l'idée qu'elle avait été insultée par d'Artagnan, menacée par Athos, et qu'elle quittait la France sans se venger d'eux. Bientôt, cette idée était devenue pour elle tellement insupportable, qu'au risque de ce qui pouvait arriver de terrible pour elle-même, elle avait supplié le capitaine de la jeter sur la côte; mais le capitaine, pressé d'échapper à sa fausse position, placé entre les croiseurs français et anglais, comme la chauve-souris entre les rats et les oiseaux, avait grande hâte de regagner l'Angleterre, et refusa obstinément d'obéir à ce qu'il prenait pour un caprice de femme, promettant à sa passagère, qui au reste lui était particulièrement recommandée par le cardinal, de la jeter, si la mer et les Français le permettaient, dans un des ports de la Bretagne, soit à Lorient, soit à Brest; mais en attendant, le vent était contraire, la mer mauvaise, on louvoyait et l'on courait des bordées. Neuf jours après la sortie de la Charente, Milady, toute pâle de ses chagrins et de sa rage, voyait apparaître seulement les côtes bleuâtres du Finistère. 

Elle calcula que pour traverser ce coin de la France et revenir près du cardinal il lui fallait au moins trois jours; ajoutez un jour pour le débarquement et cela faisait quatre; ajoutez ces quatre jours aux neuf autres, c'était treize jours de perdus, treize jours pendant lesquels tant d'événements importants se pouvaient passer à Londres. Elle songea que sans aucun doute le cardinal serait furieux de son retour, et que par conséquent il serait plus disposé à écouter les plaintes qu'on porterait contre elle que les accusations qu'elle porterait contre les autres. Elle laissa donc passer Lorient et Brest sans insister près du capitaine, qui, de son côté, se garda bien de lui donner l'éveil. Milady continua donc sa route, et le jour même où Planchet s'embarquait de Portsmouth pour la France, la messagère de son Éminence entrait triomphante dans le port. 

Toute la ville était agitée d'un mouvement extraordinaire: --- quatre grands vaisseaux récemment achevés venaient d'être lancés à la mer; --- debout sur la jetée, chamarré d'or, éblouissant, selon son habitude de diamants et de pierreries, le feutre orné d'une plume blanche qui retombait sur son épaule, on voyait Buckingham entouré d'un état-major presque aussi brillant que lui. 

C'était une de ces belles et rares journées d'hiver où l'Angleterre se souvient qu'il y a un soleil. L'astre pâli, mais cependant splendide encore, se couchait à l'horizon, empourprant à la fois le ciel et la mer de bandes de feu et jetant sur les tours et les vieilles maisons de la ville un dernier rayon d'or qui faisait étinceler les vitres comme le reflet d'un incendie. Milady, en respirant cet air de l'Océan plus vif et plus balsamique à l'approche de la terre, en contemplant toute la puissance de ces préparatifs qu'elle était chargée de détruire, toute la puissance de cette armée qu'elle devait combattre à elle seule --- elle femme --- avec quelques sacs d'or, se compara mentalement à Judith, la terrible Juive, lorsqu'elle pénétra dans le camp des Assyriens et qu'elle vit la masse énorme de chars, de chevaux, d'hommes et d'armes qu'un geste de sa main devait dissiper comme un nuage de fumée. 

On entra dans la rade; mais comme on s'apprêtait à y jeter l'ancre, un petit cutter formidablement armé s'approcha du bâtiment marchand, se donnant comme garde-côte, et fit mettre à la mer son canot, qui se dirigea vers l'échelle. Ce canot renfermait un officier, un contremaître et huit rameurs; l'officier seul monta à bord, où il fut reçu avec toute la déférence qu'inspire l'uniforme. 

L'officier s'entretint quelques instants avec le patron, lui fit lire un papier dont il était porteur, et, sur l'ordre du capitaine marchand, tout l'équipage du bâtiment, matelots et passagers, fut appelé sur le pont. 

Lorsque cette espèce d'appel fut fait, l'officier s'enquit tout haut du point de départ du brik, de sa route, de ses atterrissements, et à toutes les questions le capitaine satisfit sans hésitation et sans difficulté. Alors l'officier commença de passer la revue de toutes les personnes les unes après les autres, et, s'arrêtant à Milady, la considéra avec un grand soin, mais sans lui adresser une seule parole. 

Puis il revint au capitaine, lui dit encore quelques mots; et, comme si c'eût été à lui désormais que le bâtiment dût obéir, il commanda une manoeuvre que l'équipage exécuta aussitôt. Alors le bâtiment se remit en route, toujours escorté du petit cutter, qui voguait bord à bord avec lui, menaçant son flanc de la bouche de ses six canons tandis que la barque suivait dans le sillage du navire, faible point près de l'énorme masse. 

Pendant l'examen que l'officier avait fait de Milady, Milady, comme on le pense bien, l'avait de son côté dévoré du regard. Mais, quelque habitude que cette femme aux yeux de flamme eût de lire dans le cœur de ceux dont elle avait besoin de deviner les secrets, elle trouva cette fois un visage d'une impassibilité telle qu'aucune découverte ne suivit son investigation. L'officier qui s'était arrêté devant elle et qui l'avait silencieusement étudiée avec tant de soin pouvait être âgé de vingt-cinq à vingt-six ans, était blanc de visage avec des yeux bleu clair un peu enfoncés; sa bouche, fine et bien dessinée, demeurait immobile dans ses lignes correctes; son menton, vigoureusement accusé, dénotait cette force de volonté qui, dans le type vulgaire britannique, n'est ordinairement que de l'entêtement; un front un peu fuyant, comme il convient aux poètes, aux enthousiastes et aux soldats, était à peine ombragé d'une chevelure courte et clairsemée, qui, comme la barbe qui couvrait le bas de son visage, était d'une belle couleur châtain foncé. 

Lorsqu'on entra dans le port, il faisait déjà nuit. La brume épaississait encore l'obscurité et formait autour des fanaux et des lanternes des jetées un cercle pareil à celui qui entoure la lune quand le temps menace de devenir pluvieux. L'air qu'on respirait était triste, humide et froid. 

Milady, cette femme si forte, se sentait frissonner malgré elle. 

L'officier se fit indiquer les paquets de Milady, fit porter son bagage dans le canot; et lorsque cette opération fut faite, il l'invita à y descendre elle-même en lui tendant sa main. 

Milady regarda cet homme et hésita. 

«Qui êtes-vous, monsieur, demanda-t-elle, qui avez la bonté de vous occuper si particulièrement de moi? 

\speak  Vous devez le voir, madame, à mon uniforme; je suis officier de la marine anglaise, répondit le jeune homme. 

\speak  Mais enfin, est-ce l'habitude que les officiers de la marine anglaise se mettent aux ordres de leurs compatriotes lorsqu'ils abordent dans un port de la Grande-Bretagne, et poussent la galanterie jusqu'à les conduire à terre? 

\speak  Oui, Milady, c'est l'habitude, non point par galanterie, mais par prudence, qu'en temps de guerre les étrangers soient conduits à une hôtellerie désignée, afin que jusqu'à parfaite information sur eux ils restent sous la surveillance du gouvernement.» 

Ces mots furent prononcés avec la politesse la plus exacte et le calme le plus parfait. Cependant ils n'eurent point le don de convaincre Milady. 

«Mais je ne suis pas étrangère, monsieur, dit-elle avec l'accent le plus pur qui ait jamais retenti de Portsmouth à Manchester, je me nomme Lady Clarick, et cette mesure\dots 

\speak  Cette mesure est générale, Milady, et vous tenteriez inutilement de vous y soustraire. 

\speak  Je vous suivrai donc, monsieur.» 

Et acceptant la main de l'officier, elle commença de descendre l'échelle au bas de laquelle l'attendait le canot. L'officier la suivit; un grand manteau était étendu à la poupe, l'officier la fit asseoir sur le manteau et s'assit près d'elle. 

«Nagez», dit-il aux matelots. 

Les huit rames retombèrent dans la mer, ne formant qu'un seul bruit, ne frappant qu'un seul coup, et le canot sembla voler sur la surface de l'eau. 

Au bout de cinq minutes on touchait à terre. 

L'officier sauta sur le quai et offrit la main à Milady. 

Une voiture attendait. 

«Cette voiture est-elle pour nous? demanda Milady. 

\speak  Oui, madame, répondit l'officier. 

\speak  L'hôtellerie est donc bien loin? 

\speak  À l'autre bout de la ville. 

\speak  Allons», dit Milady. 

Et elle monta résolument dans la voiture. 

L'officier veilla à ce que les paquets fussent soigneusement attachés derrière la caisse, et cette opération terminée, prit sa place près de Milady et referma la portière. 

Aussitôt, sans qu'aucun ordre fût donné et sans qu'on eût besoin de lui indiquer sa destination, le cocher partit au galop et s'enfonça dans les rues de la ville. 

Une réception si étrange devait être pour Milady une ample matière à réflexion; aussi, voyant que le jeune officier ne paraissait nullement disposé à lier conversation, elle s'accouda dans un angle de la voiture et passa les unes après les autres en revue toutes les suppositions qui se présentaient à son esprit. 

Cependant, au bout d'un quart d'heure, étonnée de la longueur du chemin, elle se pencha vers la portière pour voir où on la conduisait. On n'apercevait plus de maisons; des arbres apparaissaient dans les ténèbres comme de grands fantômes noirs courant les uns après les autres. 

Milady frissonna. 

«Mais nous ne sommes plus dans la ville, monsieur», dit-elle. 

Le jeune officier garda le silence. 

«Je n'irai pas plus loin, si vous ne me dites pas où vous me conduisez; je vous en préviens, monsieur!» 

Cette menace n'obtint aucune réponse. 

«Oh! c'est trop fort! s'écria Milady, au secours! au secours!» 

Pas une voix ne répondit à la sienne, la voiture continua de rouler avec rapidité; l'officier semblait une statue. 

Milady regarda l'officier avec une de ces expressions terribles, particulières à son visage et qui manquaient si rarement leur effet; la colère faisait étinceler ses yeux dans l'ombre. 

Le jeune homme resta impassible. 

Milady voulut ouvrir la portière et se précipiter. 

«Prenez garde, madame, dit froidement le jeune homme, vous vous tuerez en sautant.» 

Milady se rassit écumante; l'officier se pencha, la regarda à son tour et parut surpris de voir cette figure, si belle naguère, bouleversée par la rage et devenue presque hideuse. L'astucieuse créature comprit qu'elle se perdait en laissant voir ainsi dans son âme; elle rasséréna ses traits, et d'une voix gémissante: 

«Au nom du Ciel, monsieur! dites-moi si c'est à vous, si c'est à votre gouvernement, si c'est à un ennemi que je dois attribuer la violence que l'on me fait? 

\speak  On ne vous fait aucune violence, madame, et ce qui vous arrive est le résultat d'une mesure toute simple que nous sommes forcés de prendre avec tous ceux qui débarquent en Angleterre. 

\speak  Alors vous ne me connaissez pas, monsieur? 

\speak  C'est la première fois que j'ai l'honneur de vous voir. 

\speak  Et, sur votre honneur, vous n'avez aucun sujet de haine contre moi? 

\speak  Aucun, je vous le jure.» 

II y avait tant de sérénité, de sang-froid, de douceur même dans la voix du jeune homme, que Milady fut rassurée. 

Enfin, après une heure de marche à peu près, la voiture s'arrêta devant une grille de fer qui fermait un chemin creux conduisant à un château sévère de forme, massif et isolé. Alors, comme les roues tournaient sur un sable fin, Milady entendit un vaste mugissement, qu'elle reconnut pour le bruit de la mer qui vient se briser sur une côte escarpée. 

La voiture passa sous deux voûtes, et enfin s'arrêta dans une cour sombre et carrée; presque aussitôt la portière de la voiture s'ouvrit, le jeune homme sauta légèrement à terre et présenta sa main à Milady, qui s'appuya dessus, et descendit à son tour avec assez de calme. 

«Toujours est-il, dit Milady en regardant autour d'elle et en ramenant ses yeux sur le jeune officier avec le plus gracieux sourire, que je suis prisonnière; mais ce ne sera pas pour longtemps, j'en suis sûre, ajouta-t-elle, ma conscience et votre politesse, monsieur, m'en sont garants.» 

Si flatteur que fût le compliment, l'officier ne répondit rien; mais, tirant de sa ceinture un petit sifflet d'argent pareil à celui dont se servent les contremaîtres sur les bâtiments de guerre, il siffla trois fois, sur trois modulations différentes: alors plusieurs hommes parurent, dételèrent les chevaux fumants et emmenèrent la voiture sous une remise. 

Puis l'officier, toujours avec la même politesse calme, invita sa prisonnière à entrer dans la maison. Celle-ci, toujours avec son même visage souriant, lui prit le bras, et entra avec lui sous une porte basse et cintrée qui, par une voûte éclairée seulement au fond, conduisait à un escalier de pierre tournant autour d'une arête de pierre; puis on s'arrêta devant une porte massive qui, après l'introduction dans la serrure d'une clef que le jeune homme portait sur lui, roula lourdement sur ses gonds et donna ouverture à la chambre destinée à Milady. 

D'un seul regard, la prisonnière embrassa l'appartement dans ses moindres détails. 

C'était une chambre dont l'ameublement était à la fois bien propre pour une prison et bien sévère pour une habitation d'homme libre; cependant, des barreaux aux fenêtres et des verrous extérieurs à la porte décidaient le procès en faveur de la prison. 

Un instant toute la force d'âme de cette créature, trempée cependant aux sources les plus vigoureuses, l'abandonna; elle tomba sur un fauteuil, croisant les bras, baissant la tête, et s'attendant à chaque instant à voir entrer un juge pour l'interroger. 

Mais personne n'entra, que deux ou trois soldats de marine qui apportèrent les malles et les caisses, les déposèrent dans un coin et se retirèrent sans rien dire. 

L'officier présidait à tous ces détails avec le même calme que Milady lui avait constamment vu, ne prononçant pas une parole lui-même, et se faisant obéir d'un geste de sa main ou d'un coup de son sifflet. 

On eût dit qu'entre cet homme et ses inférieurs la langue parlée n'existait pas ou devenait inutile. 

Enfin Milady n'y put tenir plus longtemps, elle rompit le silence: 

«Au nom du Ciel, monsieur! s'écria-t-elle, que veut dire tout ce qui se passe? Fixez mes irrésolutions; j'ai du courage pour tout danger que je prévois, pour tout malheur que je comprends. Où suis-je et que suis-je ici? suis-je libre, pourquoi ces barreaux et ces portes? suis-je prisonnière, quel crime ai-je commis? 

\speak  Vous êtes ici dans l'appartement qui vous est destiné, madame. J'ai reçu l'ordre d'aller vous prendre en mer et de vous conduire en ce château: cet ordre, je l'ai accompli, je crois, avec toute la rigidité d'un soldat, mais aussi avec toute la courtoisie d'un gentilhomme. Là se termine, du moins jusqu'à présent, la charge que j'avais à remplir près de vous, le reste regarde une autre personne. 

\speak  Et cette autre personne, quelle est-elle? demanda Milady; ne pouvez-vous me dire son nom?\dots» 

En ce moment on entendit par les escaliers un grand bruit d'éperons; quelques voix passèrent et s'éteignirent, et le bruit d'un pas isolé se rapprocha de la porte. 

«Cette personne, la voici, madame», dit l'officier en démasquant le passage, et en se rangeant dans l'attitude du respect et de la soumission. 

En même temps, la porte s'ouvrit; un homme parut sur le seuil. 

Il était sans chapeau, portait l'épée au côté, et froissait un mouchoir entre ses doigts. 

Milady crut reconnaître cette ombre dans l'ombre, elle s'appuya d'une main sur le bras de son fauteuil, et avança la tête comme pour aller au-devant d'une certitude. 

Alors l'étranger s'avança lentement; et, à mesure qu'il s'avançait en entrant dans le cercle de lumière projeté par la lampe, Milady se reculait involontairement. 

Puis, lorsqu'elle n'eut plus aucun doute: 

«Eh quoi! mon frère! s'écria-t-elle au comble de la stupeur, c'est vous? 

\speak  Oui, belle dame! répondit Lord de Winter en faisant un salut moitié courtois, moitié ironique, moi-même. 

\speak  Mais alors, ce château? 

\speak  Est à moi. 

\speak  Cette chambre? 

\speak  C'est la vôtre. 

\speak  Je suis donc votre prisonnière? 

\speak  À peu près. 

\speak  Mais c'est un affreux abus de la force! 

\speak  Pas de grands mots; asseyons-nous, et causons tranquillement, comme il convient de faire entre un frère et une soeur.» 

Puis, se retournant vers la porte, et voyant que le jeune officier attendait ses derniers ordres: 

«C'est bien, dit-il, je vous remercie; maintenant, laissez-nous, monsieur Felton.» 