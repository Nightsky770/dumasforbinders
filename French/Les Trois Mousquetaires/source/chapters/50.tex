%!TeX root=../musketeersfr.tex 

\chapter{Causerie D'Un Frère Avec Sa Sœur}

\lettrine{P}{endant} le temps que Lord de Winter mit à fermer la porte, à pousser un volet et à approcher un siège du fauteuil de sa belle-sœur, Milady, rêveuse, plongea son regard dans les profondeurs de la possibilité, et découvrit toute la trame qu'elle n'avait pas même pu entrevoir, tant qu'elle ignorait en quelles mains elle était tombée. Elle connaissait son beau-frère pour un bon gentilhomme, franc-chasseur, joueur intrépide, entreprenant près des femmes, mais d'une force inférieure à la sienne à l'endroit de l'intrigue. Comment avait-il pu découvrir son arrivée? la faire saisir? Pourquoi la retenait-il? 

Athos lui avait bien dit quelques mots qui prouvaient que la conversation qu'elle avait eue avec le cardinal était tombée dans des oreilles étrangères; mais elle ne pouvait admettre qu'il eût pu creuser une contre-mine si prompte et si hardie. 

Elle craignit bien plutôt que ses précédentes opérations en Angleterre n'eussent été découvertes. Buckingham pouvait avoir deviné que c'était elle qui avait coupé les deux ferrets, et se venger de cette petite trahison; mais Buckingham était incapable de se porter à aucun excès contre une femme, surtout si cette femme était censée avoir agi par un sentiment de jalousie. 

Cette supposition lui parut la plus probable; il lui sembla qu'on voulait se venger du passé, et non aller au-devant de l'avenir. Toutefois, et en tout cas, elle s'applaudit d'être tombée entre les mains de son beau-frère, dont elle comptait avoir bon marché, plutôt qu'entre celles d'un ennemi direct et intelligent. 

«Oui, causons, mon frère, dit-elle avec une espèce d'enjouement, décidée qu'elle était à tirer de la conversation, malgré toute la dissimulation que pourrait y apporter Lord de Winter, les éclaircissements dont elle avait besoin pour régler sa conduite à venir. 

\speak  Vous vous êtes donc décidée à revenir en Angleterre, dit Lord de Winter, malgré la résolution que vous m'aviez si souvent manifestée à Paris de ne jamais remettre les pieds sur le territoire de la Grande-Bretagne?» 

Milady répondit à une question par une autre question. 

«Avant tout, dit-elle, apprenez-moi donc comment vous m'avez fait guetter assez sévèrement pour être d'avance prévenu non seulement de mon arrivée, mais encore du jour, de l'heure et du port où j'arrivais.» 

Lord de Winter adopta la même tactique que Milady, pensant que, puisque sa belle-sœur l'employait, ce devait être la bonne. 

«Mais, dites-moi vous-même, ma chère sœur, reprit-il, ce que vous venez faire en Angleterre. 

\speak  Mais je viens vous voir, reprit Milady, sans savoir combien elle aggravait, par cette réponse, les soupçons qu'avait fait naître dans l'esprit de son beau-frère la lettre de d'Artagnan, et voulant seulement capter la bienveillance de son auditeur par un mensonge. 

\speak  Ah! me voir? dit sournoisement Lord de Winter. 

\speak  Sans doute, vous voir. Qu'y a-t-il d'étonnant à cela? 

\speak  Et vous n'avez pas, en venant en Angleterre, d'autre but que de me voir? 

\speak  Non. 

\speak  Ainsi, c'est pour moi seul que vous vous êtes donné la peine de traverser la Manche? 

\speak  Pour vous seul. 

\speak  Peste! quelle tendresse, ma sœur! 

\speak  Mais ne suis-je pas votre plus proche parente? demanda Milady du ton de la plus touchante naïveté. 

\speak  Et même ma seule héritière, n'est-ce pas?» dit à son tour Lord de Winter, en fixant ses yeux sur ceux de Milady. 

Quelque puissance qu'elle eût sur elle-même, Milady ne put s'empêcher de tressaillir, et comme, en prononçant les dernières paroles qu'il avait dites, Lord de Winter avait posé la main sur le bras de sa sœur, ce tressaillement ne lui échappa point. 

En effet, le coup était direct et profond. La première idée qui vint à l'esprit de Milady fut qu'elle avait été trahie par Ketty, et que celle-ci avait raconté au baron cette aversion intéressée dont elle avait imprudemment laissé échapper des marques devant sa suivante; elle se rappela aussi la sortie furieuse et imprudente qu'elle avait faite contre d'Artagnan, lorsqu'il avait sauvé la vie de son beau-frère. 

«Je ne comprends pas, Milord, dit-elle pour gagner du temps et faire parler son adversaire. Que voulez-vous dire? et y a-t-il quelque sens inconnu caché sous vos paroles? 

\speak  Oh! mon Dieu, non, dit Lord de Winter avec une apparente bonhomie; vous avez le désir de me voir, et vous venez en Angleterre. J'apprends ce désir, ou plutôt je me doute que vous l'éprouvez, et afin de vous épargner tous les ennuis d'une arrivée nocturne dans un port, toutes les fatigues d'un débarquement, j'envoie un de mes officiers au-devant de vous; je mets une voiture à ses ordres, et il vous amène ici dans ce château, dont je suis gouverneur, où je viens tous les jours, et où, pour que notre double désir de nous voir soit satisfait, je vous fais préparer une chambre. Qu'y a-t-il dans tout ce que je dis là de plus étonnant que dans ce que vous m'avez dit? 

\speak  Non, ce que je trouve d'étonnant, c'est que vous ayez été prévenu de mon arrivée. 

\speak  C'est cependant la chose la plus simple, ma chère sœur: n'avez-vous pas vu que le capitaine de votre petit bâtiment avait, en entrant dans la rade, envoyé en avant et afin d'obtenir son entrée dans le port, un petit canot porteur de son livre de loch et de son registre d'équipage? Je suis commandant du port, on m'a apporté ce livre, j'y ai reconnu votre nom. Mon cœur m'a dit ce que vient de me confier votre bouche, c'est-à-dire dans quel but vous vous exposiez aux dangers d'une mer si périlleuse ou tout au moins si fatigante en ce moment, et j'ai envoyé mon cutter au-devant de vous. Vous savez le reste.» 

Milady comprit que Lord de Winter mentait et n'en fut que plus effrayée. 

«Mon frère, continua-t-elle, n'est-ce pas Milord Buckingham que je vis sur la jetée, le soir, en arrivant? 

\speak  Lui-même. Ah! je comprends que sa vue vous ait frappée, reprit Lord de Winter: vous venez d'un pays où l'on doit beaucoup s'occuper de lui, et je sais que ses armements contre la France préoccupent fort votre ami le cardinal. 

\speak  Mon ami le cardinal! s'écria Milady, voyant que, sur ce point comme sur l'autre, Lord de Winter paraissait instruit de tout. 

\speak  N'est-il donc point votre ami? reprit négligemment le baron; ah! pardon, je le croyais; mais nous reviendrons à Milord duc plus tard, ne nous écartons point du tour sentimental que la conversation avait pris: vous veniez, disiez-vous, pour me voir? 

\speak  Oui. 

\speak  Eh bien, je vous ai répondu que vous seriez servie à souhait et que nous nous verrions tous les jours. 

\speak  Dois-je donc demeurer éternellement ici? demanda Milady avec un certain effroi. 

\speak  Vous trouveriez-vous mal logée, ma sœur? demandez ce qui vous manque, et je m'empresserai de vous le faire donner. 

\speak  Mais je n'ai ni mes femmes ni mes gens\dots 

\speak  Vous aurez tout cela, madame; dites-moi sur quel pied votre premier mari avait monté votre maison; quoique je ne sois que votre beau-frère, je vous la monterai sur un pied pareil. 

\speak  Mon premier mari! s'écria Milady en regardant Lord de Winter avec des yeux effarés. 

\speak  Oui, votre mari français; je ne parle pas de mon frère. Au reste, si vous l'avez oublié, comme il vit encore, je pourrais lui écrire et il me ferait passer des renseignements à ce sujet.» 

Une sueur froide perla sur le front de Milady. 

«Vous raillez, dit-elle d'une voix sourde. 

\speak  En ai-je l'air? demanda le baron en se relevant et en faisant un pas en arrière. 

\speak  Ou plutôt vous m'insultez, continua-t-elle en pressant de ses mains crispées les deux bras du fauteuil et en se soulevant sur ses poignets. 

\speak  Vous insulter, moi! dit Lord de Winter avec mépris; en vérité, madame, croyez-vous que ce soit possible? 

\speak  En vérité, monsieur, dit Milady, vous êtes ou ivre ou insensé; sortez et envoyez-moi une femme. 

\speak  Des femmes sont bien indiscrètes, ma sœur! ne pourrais-je pas vous servir de suivante? de cette façon tous nos secrets resteraient en famille. 

\speak  Insolent! s'écria Milady, et, comme mue par un ressort, elle bondit sur le baron, qui l'attendait avec impassibilité, mais une main cependant sur la garde de son épée. 

\speak  Eh! eh! dit-il, je sais que vous avez l'habitude d'assassiner les gens, mais je me défendrai, moi, je vous en préviens, fût-ce contre vous. 

\speak  Oh! vous avez raison, dit Milady, et vous me faites l'effet d'être assez lâche pour porter la main sur une femme. 

\speak  Peut-être que oui, d'ailleurs j'aurais mon excuse: ma main ne serait pas la première main d'homme qui se serait posée sur vous, j'imagine.» 

Et le baron indiqua d'un geste lent et accusateur l'épaule gauche de Milady, qu'il toucha presque du doigt. 

Milady poussa un rugissement sourd, et se recula jusque dans l'angle de la chambre, comme une panthère qui veut s'acculer pour s'élancer. 

«Oh! rugissez tant que vous voudrez, s'écria Lord de Winter, mais n'essayez pas de mordre, car, je vous en préviens, la chose tournerait à votre préjudice: il n'y a pas ici de procureurs qui règlent d'avance les successions, il n'y a pas de chevalier errant qui vienne me chercher querelle pour la belle dame que je retiens prisonnière; mais je tiens tout prêts des juges qui disposeront d'une femme assez éhontée pour venir se glisser, bigame, dans le lit de Lord de Winter, mon frère aîné, et ces juges, je vous en préviens, vous enverront à un bourreau qui vous fera les deux épaules pareilles.» 

Les yeux de Milady lançaient de tels éclairs, que quoiqu'il fût homme et armé devant une femme désarmée il sentit le froid de la peur se glisser jusqu'au fond de son âme; il n'en continua pas moins, mais avec une fureur croissante: 

«Oui, je comprends, après avoir hérité de mon frère, il vous eût été doux d'hériter de moi; mais, sachez-le d'avance, vous pouvez me tuer ou me faire tuer, mes précautions sont prises, pas un penny de ce que je possède ne passera dans vos mains. N'êtes-vous pas déjà assez riche, vous qui possédez près d'un million, et ne pouviez-vous vous arrêter dans votre route fatale, si vous ne faisiez le mal que pour la jouissance infinie et suprême de le faire? Oh! tenez, je vous le dis, si la mémoire de mon frère ne m'était sacrée, vous iriez pourrir dans un cachot d'État ou rassasier à Tyburn la curiosité des matelots; je me tairai, mais vous, supportez tranquillement votre captivité; dans quinze ou vingt jours je pars pour La Rochelle avec l'armée; mais la veille de mon départ, un vaisseau viendra vous prendre, que je verrai partir et qui vous conduira dans nos colonies du Sud; et, soyez tranquille, je vous adjoindrai un compagnon qui vous brûlera la cervelle à la première tentative que vous risquerez pour revenir en Angleterre ou sur le continent.» 

Milady écoutait avec une attention qui dilatait ses yeux enflammés. 

«Oui, mais à cette heure, continua Lord de Winter, vous demeurerez dans ce château: les murailles en sont épaisses, les portes en sont fortes, les barreaux en sont solides; d'ailleurs votre fenêtre donne à pic sur la mer: les hommes de mon équipage, qui me sont dévoués à la vie et à la mort, montent la garde autour de cet appartement, et surveillent tous les passages qui conduisent à la cour; puis arrivée à la cour, il vous resterait encore trois grilles à traverser. La consigne est précise: un pas, un geste, un mot qui simule une évasion, et l'on fait feu sur vous; si l'on vous tue, la justice anglaise m'aura, je l'espère, quelque obligation de lui avoir épargné de la besogne. Ah! vos traits reprennent leur calme, votre visage retrouve son assurance: Quinze jours, vingt jours dites-vous, bah! d'ici là, j'ai l'esprit inventif, il me viendra quelque idée; j'ai l'esprit infernal, et je trouverai quelque victime. D'ici à quinze jours, vous dites-vous, je serai hors d'ici. Ah! ah! essayez!» 

Milady se voyant devinée s'enfonça les ongles dans la chair pour dompter tout mouvement qui eût pu donner à sa physionomie une signification quelconque, autre que celle de l'angoisse. 

Lord de Winter continua: 

«L'officier qui commande seul ici en mon absence, vous l'avez vu, donc vous le connaissez déjà, sait, comme vous voyez, observer une consigne, car vous n'êtes pas, je vous connais, venue de Portsmouth ici sans avoir essayé de le faire parler. Qu'en dites-vous? une statue de marbre eût-elle été plus impassible et plus muette? Vous avez déjà essayé le pouvoir de vos séductions sur bien des hommes, et malheureusement vous avez toujours réussi; mais essayez sur celui-là, pardieu! si vous en venez à bout, je vous déclare le démon lui-même.» 

Il alla vers la porte et l'ouvrit brusquement. 

«Qu'on appelle M. Felton, dit-il. Attendez encore un instant, et je vais vous recommander à lui.» 

Il se fit entre ces deux personnages un silence étrange, pendant lequel on entendit le bruit d'un pas lent et régulier qui se rapprochait; bientôt, dans l'ombre du corridor, on vit se dessiner une forme humaine, et le jeune lieutenant avec lequel nous avons déjà fait connaissance s'arrêta sur le seuil, attendant les ordres du baron. 

«Entrez, mon cher John, dit Lord de Winter, entrez et fermez la porte.» 

Le jeune officier entra. 

«Maintenant, dit le baron, regardez cette femme: elle est jeune, elle est belle, elle a toutes les séductions de la terre, eh bien, c'est un monstre qui, à vingt-cinq ans, s'est rendu coupable d'autant de crimes que vous pouvez en lire en un an dans les archives de nos tribunaux; sa voix prévient en sa faveur, sa beauté sert d'appât aux victimes, son corps même paye ce qu'elle a promis, c'est une justice à lui rendre; elle essayera de vous séduire, peut-être même essayera-t-elle de vous tuer. Je vous ai tiré de la misère, Felton, je vous ai fait nommer lieutenant, je vous ai sauvé la vie une fois, vous savez à quelle occasion; je suis pour vous non seulement un protecteur, mais un ami; non seulement un bienfaiteur, mais un père; cette femme est revenue en Angleterre afin de conspirer contre ma vie; je tiens ce serpent entre mes mains; eh bien, je vous fais appeler et vous dis: Ami Felton, John, mon enfant, garde-moi et surtout garde-toi de cette femme; jure sur ton salut de la conserver pour le châtiment qu'elle a mérité. John Felton, je me fie à ta parole; John Felton, je crois à ta loyauté. 

\speak  Milord, dit le jeune officier en chargeant son regard pur de toute la haine qu'il put trouver dans son cœur, Milord, je vous jure qu'il sera fait comme vous désirez.» 

Milady reçut ce regard en victime résignée: il était impossible de voir une expression plus soumise et plus douce que celle qui régnait alors sur son beau visage. À peine si Lord de Winter lui-même reconnut la tigresse qu'un instant auparavant il s'apprêtait à combattre. 

«Elle ne sortira jamais de cette chambre, entendez-vous, John, continua le baron; elle ne correspondra avec personne, elle ne parlera qu'à vous, si toutefois vous voulez bien lui faire l'honneur de lui adresser la parole. 

\speak  Il suffit, Milord, j'ai juré. 

\speak  Et maintenant, madame, tâchez de faire la paix avec Dieu, car vous êtes jugée par les hommes.» 

Milady laissa tomber sa tête comme si elle se fût sentie écrasée par ce jugement. Lord de Winter sortit en faisant un geste à Felton, qui sortit derrière lui et ferma la porte. 

Un instant après on entendait dans le corridor le pas pesant d'un soldat de marine qui faisait sentinelle, sa hache à la ceinture et son mousquet à la main. 

Milady demeura pendant quelques minutes dans la même position, car elle songea qu'on l'examinait peut-être par la serrure; puis lentement elle releva sa tête, qui avait repris une expression formidable de menace et de défi, courut écouter à la porte, regarda par la fenêtre, et revenant s'enterrer dans un vaste fauteuil, elle songea.