%!TeX root=../musketeersfr.tex 

\chapter{L'Homme De Meung}

\lettrine{C}{e} rassemblement était produit non point par l'attente d'un homme qu'on devait pendre, mais par la contemplation d'un pendu. 

\zz
La voiture, arrêtée un instant, reprit donc sa marche, traversa la foule, continua son chemin, enfila la rue Saint-Honoré, tourna la rue des Bons-Enfants et s'arrêta devant une porte basse. 

La porte s'ouvrit, deux gardes reçurent dans leurs bras Bonacieux, soutenu par l'exempt; on le poussa dans une allée, on lui fit monter un escalier, et on le déposa dans une antichambre. 

Tous ces mouvements s'étaient opérés pour lui d'une façon machinale. 

Il avait marché comme on marche en rêve; il avait entrevu les objets à travers un brouillard; ses oreilles avaient perçu des sons sans les comprendre; on eût pu l'exécuter dans ce moment qu'il n'eût pas fait un geste pour entreprendre sa défense, qu'il n'eût pas poussé un cri pour implorer la pitié. 

Il resta donc ainsi sur la banquette, le dos appuyé au mur et les bras pendants, à l'endroit même où les gardes l'avaient déposé. 

Cependant, comme, en regardant autour de lui, il ne voyait aucun objet menaçant, comme rien n'indiquait qu'il courût un danger réel, comme la banquette était convenablement rembourrée, comme la muraille était recouverte d'un beau cuir de Cordoue, comme de grands rideaux de damas rouge flottaient devant la fenêtre, retenus par des embrasses d'or, il comprit peu à peu que sa frayeur était exagérée, et il commença de remuer la tête à droite et à gauche et de bas en haut. 

À ce mouvement, auquel personne ne s'opposa, il reprit un peu de courage et se risqua à ramener une jambe, puis l'autre; enfin, en s'aidant de ses deux mains, il se souleva sur sa banquette et se trouva sur ses pieds. 

En ce moment, un officier de bonne mine ouvrit une portière, continua d'échanger encore quelques paroles avec une personne qui se trouvait dans la pièce voisine, et se retournant vers le prisonnier: 

«C'est vous qui vous nommez Bonacieux? dit-il. 

\speak  Oui, monsieur l'officier, balbutia le mercier, plus mort que vif, pour vous servir. 

\speak  Entrez», dit l'officier. 

Et il s'effaça pour que le mercier pût passer. Celui-ci obéit sans réplique, et entra dans la chambre où il paraissait être attendu. 

C'était un grand cabinet, aux murailles garnies d'armes offensives et défensives, clos et étouffé, et dans lequel il y avait déjà du feu, quoique l'on fût à peine à la fin du mois de septembre. Une table carrée, couverte de livres et de papiers sur lesquels était déroulé un plan immense de la ville de La Rochelle, tenait le milieu de l'appartement. 

Debout devant la cheminée était un homme de moyenne taille, à la mine haute et fière, aux yeux perçants, au front large, à la figure amaigrie qu'allongeait encore une royale surmontée d'une paire de moustaches. Quoique cet homme eût trente-six à trente-sept ans à peine, cheveux, moustache et royale s'en allaient grisonnant. Cet homme, moins l'épée, avait toute la mine d'un homme de guerre, et ses bottes de buffle encore légèrement couvertes de poussière indiquaient qu'il avait monté à cheval dans la journée. 

Cet homme, c'était Armand-Jean Duplessis, cardinal de Richelieu, non point tel qu'on nous le représente, cassé comme un vieillard, souffrant comme un martyr, le corps brisé, la voix éteinte, enterré dans un grand fauteuil comme dans une tombe anticipée, ne vivant plus que par la force de son génie, et ne soutenant plus la lutte avec l'Europe que par l'éternelle application de sa pensée, mais tel qu'il était réellement à cette époque, c'est-à-dire adroit et galant cavalier, faible de corps déjà, mais soutenu par cette puissance morale qui a fait de lui un des hommes les plus extraordinaires qui aient existé; se préparant enfin, après avoir soutenu le duc de Nevers dans son duché de Mantoue, après avoir pris Nîmes, Castres et Uzès, à chasser les Anglais de l'île de Ré et à faire le siège de La Rochelle. 

À la première vue, rien ne dénotait donc le cardinal, et il était impossible à ceux-là qui ne connaissaient point son visage de deviner devant qui ils se trouvaient. 

Le pauvre mercier demeura debout à la porte, tandis que les yeux du personnage que nous venons de décrire se fixaient sur lui, et semblaient vouloir pénétrer jusqu'au fond du passé. 

«C'est là ce Bonacieux? demanda-t-il après un moment de silence. 

\speak  Oui, Monseigneur, reprit l'officier. 

\speak  C'est bien, donnez-moi ces papiers et laissez-nous.» 

L'officier prit sur la table les papiers désignés, les remit à celui qui les demandait, s'inclina jusqu'à terre, et sortit. 

Bonacieux reconnut dans ces papiers ses interrogatoires de la Bastille. De temps en temps, l'homme de la cheminée levait les yeux de dessus les écritures, et les plongeait comme deux poignards jusqu'au fond du cœur du pauvre mercier. 

Au bout de dix minutes de lecture et dix secondes d'examen, le cardinal était fixé. 

«Cette tête-là n'a jamais conspiré», murmura-t-il; mais n'importe, voyons toujours. 

\speak  Vous êtes accusé de haute trahison, dit lentement le cardinal. 

\speak  C'est ce qu'on m'a déjà appris, Monseigneur, s'écria Bonacieux, donnant à son interrogateur le titre qu'il avait entendu l'officier lui donner; mais je vous jure que je n'en savais rien.» 

Le cardinal réprima un sourire. 

«Vous avez conspiré avec votre femme, avec Mme de Chevreuse et avec Milord duc de Buckingham. 

\speak  En effet, Monseigneur, répondit le mercier, je l'ai entendue prononcer tous ces noms-là. 

\speak  Et à quelle occasion? 

\speak  Elle disait que le cardinal de Richelieu avait attiré le duc de Buckingham à Paris pour le perdre et pour perdre la reine avec lui. 

\speak  Elle disait cela? s'écria le cardinal avec violence. 

\speak  Oui, Monseigneur; mais moi je lui ai dit qu'elle avait tort de tenir de pareils propos, et que Son Éminence était incapable\dots 

\speak  Taisez-vous, vous êtes un imbécile, reprit le cardinal. 

\speak  C'est justement ce que ma femme m'a répondu, Monseigneur. 

\speak  Savez-vous qui a enlevé votre femme? 

\speak  Non, Monseigneur. 

\speak  Vous avez des soupçons, cependant? 

\speak  Oui, Monseigneur; mais ces soupçons ont paru contrarier M. le commissaire, et je ne les ai plus. 

\speak  Votre femme s'est échappée, le saviez-vous? 

\speak  Non, Monseigneur, je l'ai appris depuis que je suis en prison, et toujours par l'entremise de M. le commissaire, un homme bien aimable!» 

Le cardinal réprima un second sourire. 

«Alors vous ignorez ce que votre femme est devenue depuis sa fuite? 

\speak  Absolument, Monseigneur; mais elle a dû rentrer au Louvre. 

\speak  À une heure du matin elle n'y était pas rentrée encore. 

\speak  Ah! mon Dieu! mais qu'est-elle devenue alors? 

\speak  On le saura, soyez tranquille; on ne cache rien au cardinal; le cardinal sait tout. 

\speak  En ce cas, Monseigneur, est-ce que vous croyez que le cardinal consentira à me dire ce qu'est devenue ma femme? 

\speak  Peut-être; mais il faut d'abord que vous avouiez tout ce que vous savez relativement aux relations de votre femme avec Mme de Chevreuse. 

\speak  Mais, Monseigneur, je n'en sais rien; je ne l'ai jamais vue. 

\speak  Quand vous alliez chercher votre femme au Louvre, revenait-elle directement chez vous? 

\speak  Presque jamais: elle avait affaire à des marchands de toile, chez lesquels je la conduisais. 

\speak  Et combien y en avait-il de marchands de toile? 

\speak  Deux, Monseigneur. 

\speak  Où demeurent-ils? 

\speak  Un, rue de Vaugirard; l'autre, rue de La Harpe. 

\speak  Entriez-vous chez eux avec elle? 

\speak  Jamais, Monseigneur; je l'attendais à la porte. 

\speak  Et quel prétexte vous donnait-elle pour entrer ainsi toute seule? 

\speak  Elle ne m'en donnait pas; elle me disait d'attendre, et j'attendais. 

\speak  Vous êtes un mari complaisant, mon cher monsieur Bonacieux!» dit le cardinal. 

«Il m'appelle son cher monsieur! dit en lui-même le mercier. Peste! les affaires vont bien!» 

«Reconnaîtriez-vous ces portes? 

\speak  Oui. 

\speak  Savez-vous les numéros? 

\speak  Oui. 

\speak  Quels sont-ils? 

\speak  N° 25, dans la rue de Vaugirard; n° 75, dans la rue de La Harpe. 

\speak  C'est bien», dit le cardinal. 

À ces mots, il prit une sonnette d'argent, et sonna; l'officier rentra. 

«Allez, dit-il à demi-voix, me chercher Rochefort; et qu'il vienne à l'instant même, s'il est rentré. 

\speak  Le comte est là, dit l'officier, il demande instamment à parler à Votre Éminence!» 

«À Votre Éminence! murmura Bonacieux, qui savait que tel était le titre qu'on donnait d'ordinaire à M. le cardinal;\dots à Votre Éminence!» 

«Qu'il vienne alors, qu'il vienne!» dit vivement Richelieu. 

L'officier s'élança hors de l'appartement, avec cette rapidité que mettaient d'ordinaire tous les serviteurs du cardinal à lui obéir. 

«À Votre Éminence!» murmurait Bonacieux en roulant des yeux égarés. 

Cinq secondes ne s'étaient pas écoulées depuis la disparition de l'officier, que la porte s'ouvrit et qu'un nouveau personnage entra. 

«C'est lui, s'écria Bonacieux. 

\speak  Qui lui? demanda le cardinal. 

\speak  Celui qui m'a enlevé ma femme.» 

Le cardinal sonna une seconde fois. L'officier reparut. 

«Remettez cet homme aux mains de ses deux gardes, et qu'il attende que je le rappelle devant moi. 

\speak  Non, Monseigneur! non, ce n'est pas lui! s'écria Bonacieux; non, je m'étais trompé: c'est un autre qui ne lui ressemble pas du tout! Monsieur est un honnête homme. 

\speak  Emmenez cet imbécile!» dit le cardinal. 

L'officier prit Bonacieux sous le bras, et le reconduisit dans l'antichambre où il trouva ses deux gardes. 

Le nouveau personnage qu'on venait d'introduire suivit des yeux avec impatience Bonacieux jusqu'à ce qu'il fût sorti, et dès que la porte se fut refermée sur lui: 

«Ils se sont vus, dit-il en s'approchant vivement du cardinal. 

\speak  Qui? demanda Son Éminence. 

\speak  Elle et lui. 

\speak  La reine et le duc? s'écria Richelieu. 

\speak  Oui. 

\speak  Et où cela? 

\speak  Au Louvre. 

\speak  Vous en êtes sûr? 

\speak  Parfaitement sûr. 

\speak  Qui vous l'a dit? 

\speak  Mme de Lannoy, qui est toute à Votre Éminence, comme vous le savez. 

\speak  Pourquoi ne l'a-t-elle pas dit plus tôt? 

\speak  Soit hasard, soit défiance, la reine a fait coucher Mme de Fargis dans sa chambre, et l'a gardée toute la journée. 

\speak  C'est bien, nous sommes battus. Tâchons de prendre notre revanche. 

\speak  Je vous y aiderai de toute mon âme, Monseigneur, soyez tranquille. 

\speak  Comment cela s'est-il passé? 

\speak  À minuit et demi, la reine était avec ses femmes\dots 

\speak  Où cela? 

\speak  Dans sa chambre à coucher\dots 

\speak  Bien. 

\speak  Lorsqu'on est venu lui remettre un mouchoir de la part de sa dame de lingerie\dots 

\speak  Après? 

\speak  Aussitôt la reine a manifesté une grande émotion, et, malgré le rouge dont elle avait le visage couvert, elle a pâli. 

\speak  Après! après! 

\speak  Cependant, elle s'est levée, et d'une voix altérée: «Mesdames, a-t-elle dit, attendez-moi dix minutes, puis je reviens.» Et elle a ouvert la porte de son alcôve, puis elle est sortie. 

\speak  Pourquoi Mme de Lannoy n'est-elle pas venue vous prévenir à l'instant même? 

\speak  Rien n'était bien certain encore; d'ailleurs, la reine avait dit: «Mesdames, attendez-moi»; et elle n'osait désobéir à la reine. 

\speak  Et combien de temps la reine est-elle restée hors de la chambre? 

\speak  Trois quarts d'heure. 

\speak  Aucune de ses femmes ne l'accompagnait? 

\speak  Doña Estefania seulement. 

\speak  Et elle est rentrée ensuite? 

\speak  Oui, mais pour prendre un petit coffret de bois de rose à son chiffre, et sortir aussitôt. 

\speak  Et quand elle est rentrée, plus tard, a-t-elle rapporté le coffret? 

\speak  Non. 

\speak  Mme de Lannoy savait-elle ce qu'il y avait dans ce coffret? 

\speak  Oui: les ferrets en diamants que Sa Majesté a donnés à la reine. 

\speak  Et elle est rentrée sans ce coffret? 

\speak  Oui. 

\speak  L'opinion de Mme de Lannoy est qu'elle les a remis alors à Buckingham? 

\speak  Elle en est sûre. 

\speak  Comment cela? 

\speak  Pendant la journée, Mme de Lannoy, en sa qualité de dame d'atour de la reine, a cherché ce coffret, a paru inquiète de ne pas le trouver et a fini par en demander des nouvelles à la reine. 

\speak  Et alors, la reine\dots? 

\speak  La reine est devenue fort rouge et a répondu qu'ayant brisé la veille un de ses ferrets, elle l'avait envoyé raccommoder chez son orfèvre. 

\speak  Il faut y passer et s'assurer si la chose est vraie ou non. 

\speak  J'y suis passé. 

\speak  Eh bien, l'orfèvre? 

\speak  L'orfèvre n'a entendu parler de rien. 

\speak  Bien! bien! Rochefort, tout n'est pas perdu, et peut-être\dots peut-être tout est-il pour le mieux! 

\speak  Le fait est que je ne doute pas que le génie de Votre Éminence\dots 

\speak  Ne répare les bêtises de mon agent, n'est-ce pas? 

\speak  C'est justement ce que j'allais dire, si Votre Éminence m'avait laissé achever ma phrase. 

\speak  Maintenant, savez-vous où se cachaient la duchesse de Chevreuse et le duc de Buckingham? 

\speak  Non, Monseigneur, mes gens n'ont pu rien me dire de positif là-dessus. 

\speak  Je le sais, moi. 

\speak  Vous, Monseigneur? 

\speak  Oui, ou du moins je m'en doute. Ils se tenaient, l'un rue de Vaugirard, n° 25, et l'autre rue de La Harpe, n° 75. 

\speak  Votre Éminence veut-elle que je les fasse arrêter tous deux? 

\speak  Il sera trop tard, ils seront partis. 

\speak  N'importe, on peut s'en assurer. 

\speak  Prenez dix hommes de mes gardes, et fouillez les deux maisons. 

\speak  J'y vais, Monseigneur.» 

Et Rochefort s'élança hors de l'appartement. 

Le cardinal, resté seul, réfléchit un instant et sonna une troisième fois. 

Le même officier reparut. 

«Faites entrer le prisonnier», dit le cardinal. 

Maître Bonacieux fut introduit de nouveau, et, sur un signe du cardinal, l'officier se retira. 

«Vous m'avez trompé, dit sévèrement le cardinal. 

\speak  Moi, s'écria Bonacieux, moi, tromper Votre Éminence! 

\speak  Votre femme, en allant rue de Vaugirard et rue de La Harpe, n'allait pas chez des marchands de toile. 

\speak  Et où allait-elle, juste Dieu? 

\speak  Elle allait chez la duchesse de Chevreuse et chez le duc de Buckingham. 

\speak  Oui, dit Bonacieux rappelant tous ses souvenirs; oui, c'est cela, Votre Éminence a raison. J'ai dit plusieurs fois à ma femme qu'il était étonnant que des marchands de toile demeurassent dans des maisons pareilles, dans des maisons qui n'avaient pas d'enseignes, et chaque fois ma femme s'est mise à rire. Ah! Monseigneur, continua Bonacieux en se jetant aux pieds de l'Éminence, ah! que vous êtes bien le cardinal, le grand cardinal, l'homme de génie que tout le monde révère.» 

Le cardinal, tout médiocre qu'était le triomphe remporté sur un être aussi vulgaire que l'était Bonacieux, n'en jouit pas moins un instant; puis, presque aussitôt, comme si une nouvelle pensée se présentait à son esprit, un sourire plissa ses lèvres, et tendant la main au mercier: 

«Relevez-vous, mon ami, lui dit-il, vous êtes un brave homme. 

\speak  Le cardinal m'a touché la main! j'ai touché la main du grand homme! s'écria Bonacieux; le grand homme m'a appelé son ami! 

\speak  Oui, mon ami; oui! dit le cardinal avec ce ton paterne qu'il savait prendre quelquefois, mais qui ne trompait que les gens qui ne le connaissaient pas; et comme on vous a soupçonné injustement, eh bien, il vous faut une indemnité: tenez! prenez ce sac de cent pistoles, et pardonnez-moi. 

\speak  Que je vous pardonne, Monseigneur! dit Bonacieux hésitant à prendre le sac, craignant sans doute que ce prétendu don ne fût qu'une plaisanterie. Mais vous étiez bien libre de me faire arrêter, vous êtes bien libre de me faire torturer, vous êtes bien libre de me faire pendre: vous êtes le maître, et je n'aurais pas eu le plus petit mot à dire. Vous pardonner, Monseigneur! Allons donc, vous n'y pensez pas! 

\speak  Ah! mon cher monsieur Bonacieux! vous y mettez de la générosité, je le vois, et je vous en remercie. Ainsi donc, vous prenez ce sac, et vous vous en allez sans être trop mécontent? 

\speak  Je m'en vais enchanté, Monseigneur. 

\speak  Adieu donc, ou plutôt à revoir, car j'espère que nous nous reverrons. 

\speak  Tant que Monseigneur voudra, et je suis bien aux ordres de Son Éminence. 

\speak  Ce sera souvent, soyez tranquille, car j'ai trouvé un charme extrême à votre conversation. 

\speak  Oh! Monseigneur! 

\speak  Au revoir, monsieur Bonacieux, au revoir. 

Et le cardinal lui fit un signe de la main, auquel Bonacieux répondit en s'inclinant jusqu'à terre; puis il sortit à reculons, et quand il fut dans l'antichambre, le cardinal l'entendit qui, dans son enthousiasme, criait à tue-tête: «Vive Monseigneur! vive Son Éminence! vive le grand cardinal!» Le cardinal écouta en souriant cette brillante manifestation des sentiments enthousiastes de maître Bonacieux; puis, quand les cris de Bonacieux se furent perdus dans l'éloignement: 

«Bien, dit-il, voici désormais un homme qui se fera tuer pour moi.» 

Et le cardinal se mit à examiner avec la plus grande attention la carte de La Rochelle qui, ainsi que nous l'avons dit, était étendue sur son bureau, traçant avec un crayon la ligne où devait passer la fameuse digue qui, dix-huit mois plus tard, fermait le port de la cité assiégée. 

Comme il en était au plus profond de ses méditations stratégiques, la porte se rouvrit, et Rochefort rentra. 

«Eh bien? dit vivement le cardinal en se levant avec une promptitude qui prouvait le degré d'importance qu'il attachait à la commission dont il avait chargé le comte. 

\speak  Eh bien, dit celui-ci, une jeune femme de vingt-six à vingt-huit ans et un homme de trente-cinq à quarante ans ont logé effectivement, l'un quatre jours et l'autre cinq, dans les maisons indiquées par Votre Éminence: mais la femme est partie cette nuit, et l'homme ce matin. 

\speak  C'étaient eux! s'écria le cardinal, qui regardait à la pendule; et maintenant, continua-t-il, il est trop tard pour faire courir après: la duchesse est à Tours, et le duc à Boulogne. C'est à Londres qu'il faut les rejoindre. 

\speak  Quels sont les ordres de Votre Éminence? 

\speak  Pas un mot de ce qui s'est passé; que la reine reste dans une sécurité parfaite; qu'elle ignore que nous savons son secret; qu'elle croie que nous sommes à la recherche d'une conspiration quelconque. Envoyez-moi le garde des sceaux Séguier. 

\speak  Et cet homme, qu'en a fait Votre Éminence? 

\speak  Quel homme? demanda le cardinal. 

\speak  Ce Bonacieux? 

\speak  J'en ai fait tout ce qu'on pouvait en faire. J'en ai fait l'espion de sa femme.» 

Le comte de Rochefort s'inclina en homme qui reconnaît la grande supériorité du maître, et se retira. 

Resté seul, le cardinal s'assit de nouveau, écrivit une lettre qu'il cacheta de son sceau particulier, puis il sonna. L'officier entra pour la quatrième fois. 

«Faites-moi venir Vitray, dit-il, et dites-lui de s'apprêter pour un voyage.» 

Un instant après, l'homme qu'il avait demandé était debout devant lui, tout botté et tout éperonné. 

«Vitray, dit-il, vous allez partir tout courant pour Londres. Vous ne vous arrêterez pas un instant en route. Vous remettrez cette lettre à Milady. Voici un bon de deux cents pistoles, passez chez mon trésorier et faites-vous payer. Il y en a autant à toucher si vous êtes ici de retour dans six jours et si vous avez bien fait ma commission.» 

Le messager, sans répondre un seul mot, s'inclina, prit la lettre, le bon de deux cents pistoles, et sortit. 

Voici ce que contenait la lettre:  «Milady, 

«Trouvez-vous au premier bal où se trouvera le duc de Buckingham. Il aura à son pourpoint douze ferrets de diamants, approchez-vous de lui et coupez-en deux. 

«Aussitôt que ces ferrets seront en votre possession, prévenez-moi.» 