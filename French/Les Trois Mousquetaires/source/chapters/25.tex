%!TeX root=../musketeersfr.tex 

\chapter{Porthos}

\lettrine{A}{u} lieu de rentrer chez lui directement, d'Artagnan mit pied à terre à la porte de M. de Tréville, et monta rapidement l'escalier. Cette fois, il était décidé à lui raconter tout ce qui venait de se passer. Sans doute il lui donnerait de bons conseils dans toute cette affaire; puis, comme M. de Tréville voyait presque journellement la reine, il pourrait peut-être tirer de Sa Majesté quelque renseignement sur la pauvre femme à qui l'on faisait sans doute payer son dévouement à sa maîtresse. 

M. de Tréville écouta le récit du jeune homme avec une gravité qui prouvait qu'il voyait autre chose, dans toute cette aventure, qu'une intrigue d'amour; puis, quand d'Artagnan eut achevé: 

«Hum! dit-il, tout ceci sent Son Éminence d'une lieue. 

\speak  Mais, que faire? dit d'Artagnan. 

\speak  Rien, absolument rien, à cette heure, que quitter Paris, comme je vous l'ai dit, le plus tôt possible. Je verrai la reine, je lui raconterai les détails de la disparition de cette pauvre femme, qu'elle ignore sans doute; ces détails la guideront de son côté, et, à votre retour, peut-être aurai-je quelque bonne nouvelle à vous dire. Reposez vous en sur moi.» 

D'Artagnan savait que, quoique Gascon, M. de Tréville n'avait pas l'habitude de promettre, et que lorsque par hasard il promettait, il tenait plus qu'il n'avait promis. Il le salua donc, plein de reconnaissance pour le passé et pour l'avenir, et le digne capitaine, qui de son côté éprouvait un vif intérêt pour ce jeune homme si brave et si résolu, lui serra affectueusement la main en lui souhaitant un bon voyage. 

Décidé à mettre les conseils de M. de Tréville en pratique à l'instant même, d'Artagnan s'achemina vers la rue des Fossoyeurs, afin de veiller à la confection de son portemanteau. En s'approchant de sa maison, il reconnut M. Bonacieux en costume du matin, debout sur le seuil de sa porte. Tout ce que lui avait dit, la veille, le prudent Planchet sur le caractère sinistre de son hôte revint alors à l'esprit de d'Artagnan, qui le regarda plus attentivement qu'il n'avait fait encore. En effet, outre cette pâleur jaunâtre et maladive qui indique l'infiltration de la bile dans le sang et qui pouvait d'ailleurs n'être qu'accidentelle, d'Artagnan remarqua quelque chose de sournoisement perfide dans l'habitude des rides de sa face. Un fripon ne rit pas de la même façon qu'un honnête homme, un hypocrite ne pleure pas les mêmes larmes qu'un homme de bonne foi. Toute fausseté est un masque, et si bien fait que soit le masque, on arrive toujours, avec un peu d'attention, à le distinguer du visage. 

Il sembla donc à d'Artagnan que M. Bonacieux portait un masque, et même que ce masque était des plus désagréables à voir. 

En conséquence il allait, vaincu par sa répugnance pour cet homme, passer devant lui sans lui parler, quand, ainsi que la veille, M. Bonacieux l'interpella. 

«Eh bien, jeune homme, lui dit-il, il paraît que nous faisons de grasses nuits? Sept heures du matin, peste! Il me semble que vous retournez tant soit peu les habitudes reçues, et que vous rentrez à l'heure où les autres sortent. 

\speak  On ne vous fera pas le même reproche, maître Bonacieux, dit le jeune homme, et vous êtes le modèle des gens rangés. Il est vrai que lorsque l'on possède une jeune et jolie femme, on n'a pas besoin de courir après le bonheur: c'est le bonheur qui vient vous trouver; n'est-ce pas, monsieur Bonacieux?» 

Bonacieux devint pâle comme la mort et grimaça un sourire. 

«Ah! ah! dit Bonacieux, vous êtes un plaisant compagnon. Mais où diable avez-vous été courir cette nuit, mon jeune maître? Il paraît qu'il ne faisait pas bon dans les chemins de traverse.» 

D'Artagnan baissa les yeux vers ses bottes toutes couvertes de boue; mais dans ce mouvement ses regards se portèrent en même temps sur les souliers et les bas du mercier; on eût dit qu'on les avait trempés dans le même bourbier; les uns et les autres étaient maculés de taches absolument pareilles. 

Alors une idée subite traversa l'esprit de d'Artagnan. Ce petit homme gros, court, grisonnant, cette espèce de laquais vêtu d'un habit sombre, traité sans considération par les gens d'épée qui composaient l'escorte, c'était Bonacieux lui-même. Le mari avait présidé à l'enlèvement de sa femme. 

Il prit à d'Artagnan une terrible envie de sauter à la gorge du mercier et de l'étrangler; mais, nous l'avons dit, c'était un garçon fort prudent, et il se contint. Cependant la révolution qui s'était faite sur son visage était si visible, que Bonacieux en fut effrayé et essaya de reculer d'un pas; mais justement il se trouvait devant le battant de la porte, qui était fermée, et l'obstacle qu'il rencontra le força de se tenir à la même place. 

«Ah çà! mais vous qui plaisantez, mon brave homme, dit d'Artagnan, il me semble que si mes bottes ont besoin d'un coup d'éponge, vos bas et vos souliers réclament aussi un coup de brosse. Est-ce que de votre côté vous auriez couru la prétantaine, maître Bonacieux? Ah! diable, ceci ne serait point pardonnable à un homme de votre âge et qui, de plus, a une jeune et jolie femme comme la vôtre. 

\speak  Oh! mon Dieu, non, dit Bonacieux; mais hier j'ai été à Saint-Mandé pour prendre des renseignements sur une servante dont je ne puis absolument me passer, et comme les chemins étaient mauvais, j'en ai rapporté toute cette fange, que je n'ai pas encore eu le temps de faire disparaître.» 

Le lieu que désignait Bonacieux comme celui qui avait été le but de sa course fut une nouvelle preuve à l'appui des soupçons qu'avait conçus d'Artagnan. Bonacieux avait dit Saint-Mandé, parce que Saint-Mandé est le point absolument opposé à Saint-Cloud. 

Cette probabilité lui fut une première consolation. Si Bonacieux savait où était sa femme, on pourrait toujours, en employant des moyens extrêmes, forcer le mercier à desserrer les dents et à laisser échapper son secret. Il s'agissait seulement de changer cette probabilité en certitude. 

«Pardon, mon cher monsieur Bonacieux, si j'en use avec vous sans façon, dit d'Artagnan; mais rien n'altère comme de ne pas dormir, j'ai donc une soif d'enragé; permettez-moi de prendre un verre d'eau chez vous; vous le savez, cela ne se refuse pas entre voisins.» 

Et sans attendre la permission de son hôte, d'Artagnan entra vivement dans la maison, et jeta un coup d'œil rapide sur le lit. Le lit n'était pas défait. Bonacieux ne s'était pas couché. Il rentrait donc seulement il y avait une heure ou deux; il avait accompagné sa femme jusqu'à l'endroit où on l'avait conduite, ou tout au moins jusqu'au premier relais. 

«Merci, maître Bonacieux, dit d'Artagnan en vidant son verre, voilà tout ce que je voulais de vous. Maintenant je rentre chez moi, je vais faire brosser mes bottes par Planchet, et quand il aura fini, je vous l'enverrai si vous voulez pour brosser vos souliers.» 

Et il quitta le mercier tout ébahi de ce singulier adieu et se demandant s'il ne s'était pas enferré lui-même. 

Sur le haut de l'escalier il trouva Planchet tout effaré. 

«Ah! monsieur, s'écria Planchet dès qu'il eut aperçu son maître, en voilà bien d'une autre, et il me tardait bien que vous rentrassiez. 

\speak  Qu'y a-t-il donc? demanda d'Artagnan. 

\speak  Oh! je vous le donne en cent, monsieur, je vous le donne en mille de deviner la visite que j'ai reçue pour vous en votre absence. 

\speak  Quand cela? 

\speak  Il y a une demi-heure, tandis que vous étiez chez M. de Tréville. 

\speak  Et qui donc est venu? Voyons, parle. 

\speak  M. de Cavois. 

\speak  M. de Cavois? 

\speak  En personne. 

\speak  Le capitaine des gardes de Son Éminence? 

\speak  Lui-même. 

\speak  Il venait m'arrêter? 

\speak  Je m'en suis douté, monsieur, et cela malgré son air patelin. 

\speak  Il avait l'air patelin, dis-tu? 

\speak  C'est-à-dire qu'il était tout miel, monsieur. 

\speak  Vraiment? 

\speak  Il venait, disait-il, de la part de Son Éminence, qui vous voulait beaucoup de bien, vous prier de le suivre au Palais-Royal. 

\speak  Et tu lui as répondu? 

\speak  Que la chose était impossible, attendu que vous étiez hors de la maison, comme il le pouvait voir. 

\speak  Alors qu'a-t-il dit? 

\speak  Que vous ne manquiez pas de passer chez lui dans la journée; puis il a ajouté tout bas: «Dis à ton maître que Son Éminence est parfaitement disposée pour lui, et que sa fortune dépend peut-être de cette entrevue.» 

\speak  Le piège est assez maladroit pour le cardinal, reprit en souriant le jeune homme. 

\speak  Aussi, je l'ai vu, le piège, et j'ai répondu que vous seriez désespéré à votre retour. 

\speak  Où est-il allé? a demandé M. de Cavois. À Troyes en Champagne, ai-je répondu. Et quand est-il parti? 

\speak  Hier soir.» 

\speak  Planchet, mon ami, interrompit d'Artagnan, tu es véritablement un homme précieux. 

\speak  Vous comprenez, monsieur, j'ai pensé qu'il serait toujours temps, si vous désirez voir M. de Cavois, de me démentir en disant que vous n'étiez point parti; ce serait moi, dans ce cas, qui aurais fait le mensonge, et comme je ne suis pas gentilhomme, moi, je puis mentir. 

\speak  Rassure-toi, Planchet, tu conserveras ta réputation d'homme véridique: dans un quart d'heure nous partons. 

\speak  C'est le conseil que j'allais donner à monsieur; et où allons-nous, sans être trop curieux? 

\speak  Pardieu! du côté opposé à celui vers lequel tu as dit que j'étais allé. D'ailleurs, n'as-tu pas autant de hâte d'avoir des nouvelles de Grimaud, de Mousqueton et de Bazin que j'en ai, moi, de savoir ce que sont devenus Athos, Porthos et Aramis? 

\speak  Si fait, monsieur, dit Planchet, et je partirai quand vous voudrez; l'air de la province vaut mieux pour nous, à ce que je crois, en ce moment, que l'air de Paris. Ainsi donc\dots 

\speak  Ainsi donc, fais notre paquet, Planchet, et partons; moi, je m'en vais devant, les mains dans mes poches, pour qu'on ne se doute de rien. Tu me rejoindras à l'hôtel des Gardes. À propos, Planchet, je crois que tu as raison à l'endroit de notre hôte, et que c'est décidément une affreuse canaille. 

\speak  Ah! croyez-moi, monsieur, quand je vous dis quelque chose; je suis physionomiste, moi, allez!» 

D'Artagnan descendit le premier, comme la chose avait été convenue; puis, pour n'avoir rien à se reprocher, il se dirigea une dernière fois vers la demeure de ses trois amis: on n'avait reçu aucune nouvelle d'eux, seulement une lettre toute parfumée et d'une écriture élégante et menue était arrivée pour Aramis. D'Artagnan s'en chargea. Dix minutes après, Planchet le rejoignait dans les écuries de l'hôtel des Gardes. D'Artagnan, pour qu'il n'y eût pas de temps perdu, avait déjà sellé son cheval lui-même. 

«C'est bien, dit-il à Planchet, lorsque celui-ci eut joint le portemanteau à l'équipement; maintenant selle les trois autres, et partons. 

\speak  Croyez-vous que nous irons plus vite avec chacun deux chevaux? demanda Planchet avec son air narquois. 

\speak  Non, monsieur le mauvais plaisant, répondit d'Artagnan, mais avec nos quatre chevaux nous pourrons ramener nos trois amis, si toutefois nous les retrouvons vivants. 

\speak  Ce qui serait une grande chance, répondit Planchet, mais enfin il ne faut pas désespérer de la miséricorde de Dieu. 

\speak  Amen», dit d'Artagnan en enfourchant son cheval. 

Et tous deux sortirent de l'hôtel des Gardes, s'éloignèrent chacun par un bout de la rue, l'un devant quitter Paris par la barrière de la Villette et l'autre par la barrière de Montmartre, pour se rejoindre au-delà de Saint-Denis, manoeuvre stratégique qui, ayant été exécutée avec une égale ponctualité, fut couronnée des plus heureux résultats. D'Artagnan et Planchet entrèrent ensemble à Pierrefitte. 

Planchet était plus courageux, il faut le dire, le jour que la nuit. 

Cependant sa prudence naturelle ne l'abandonnait pas un seul instant; il n'avait oublié aucun des incidents du premier voyage, et il tenait pour ennemis tous ceux qu'il rencontrait sur la route. Il en résultait qu'il avait sans cesse le chapeau à la main, ce qui lui valait de sévères mercuriales de la part de d'Artagnan, qui craignait que, grâce à cet excès de politesse, on ne le prît pour le valet d'un homme de peu. 

Cependant, soit qu'effectivement les passants fussent touchés de l'urbanité de Planchet, soit que cette fois personne ne fût aposté sur la route du jeune homme, nos deux voyageurs arrivèrent à Chantilly sans accident aucun et descendirent à l'hôtel du Grand Saint Martin, le même dans lequel ils s'étaient arrêtés lors de leur premier voyage. 

L'hôte, en voyant un jeune homme suivi d'un laquais et de deux chevaux de main, s'avança respectueusement sur le seuil de la porte. Or, comme il avait déjà fait onze lieues, d'Artagnan jugea à propos de s'arrêter, que Porthos fût ou ne fût pas dans l'hôtel. Puis peut-être n'était-il pas prudent de s'informer du premier coup de ce qu'était devenu le mousquetaire. Il résulta de ces réflexions que d'Artagnan, sans demander aucune nouvelle de qui que ce fût, descendit, recommanda les chevaux à son laquais, entra dans une petite chambre destinée à recevoir ceux qui désiraient être seuls, et demanda à son hôte une bouteille de son meilleur vin et un déjeuner aussi bon que possible, demande qui corrobora encore la bonne opinion que l'aubergiste avait prise de son voyageur à la première vue. 

Aussi d'Artagnan fut-il servi avec une célérité miraculeuse. 

Le régiment des gardes se recrutait parmi les premiers gentilshommes du royaume, et d'Artagnan, suivi d'un laquais et voyageant avec quatre chevaux magnifiques, ne pouvait, malgré la simplicité de son uniforme, manquer de faire sensation. L'hôte voulut le servir lui-même; ce que voyant, d'Artagnan fit apporter deux verres et entama la conversation suivante: 

«Ma foi, mon cher hôte, dit d'Artagnan en remplissant les deux verres, je vous ai demandé de votre meilleur vin et si vous m'avez trompé, vous allez être puni par où vous avez péché, attendu que, comme je déteste boire seul, vous allez boire avec moi. Prenez donc ce verre, et buvons. À quoi boirons-nous, voyons, pour ne blesser aucune susceptibilité? Buvons à la prospérité de votre établissement! 

\speak  Votre Seigneurie me fait honneur, dit l'hôte, et je la remercie bien sincèrement de son bon souhait. 

\speak  Mais ne vous y trompez pas, dit d'Artagnan, il y a plus d'égoïsme peut-être que vous ne le pensez dans mon toast: il n'y a que les établissements qui prospèrent dans lesquels on soit bien reçu; dans les hôtels qui périclitent, tout va à la débandade, et le voyageur est victime des embarras de son hôte; or, moi qui voyage beaucoup et surtout sur cette route, je voudrais voir tous les aubergistes faire fortune. 

\speak  En effet, dit l'hôte, il me semble que ce n'est pas la première fois que j'ai l'honneur de voir monsieur. 

\speak  Bah? je suis passé dix fois peut-être à Chantilly, et sur les dix fois je me suis arrêté au moins trois ou quatre fois chez vous. Tenez, j'y étais encore il y a dix ou douze jours à peu près; je faisais la conduite à des amis, à des mousquetaires, à telle enseigne que l'un d'eux s'est pris de dispute avec un étranger, un inconnu, un homme qui lui a cherché je ne sais quelle querelle. 

\speak  Ah! oui vraiment! dit l'hôte, et je me le rappelle parfaitement. N'est-ce pas de M. Porthos que Votre Seigneurie veut me parler? 

\speak  C'est justement le nom de mon compagnon de voyage. 

«Mon Dieu! mon cher hôte, dites-moi, lui serait-il arrivé malheur? 

\speak  Mais Votre Seigneurie a dû remarquer qu'il n'a pas pu continuer sa route. 

\speak  En effet, il nous avait promis de nous rejoindre, et nous ne l'avons pas revu. 

\speak  Il nous a fait l'honneur de rester ici. 

\speak  Comment! il vous a fait l'honneur de rester ici? 

\speak  Oui, monsieur, dans cet hôtel; nous sommes même bien inquiets. 

\speak  Et de quoi? 

\speak  De certaines dépenses qu'il a faites. 

\speak  Eh bien, mais les dépenses qu'il a faites, il les paiera. 

\speak  Ah! monsieur, vous me mettez véritablement du baume dans le sang! Nous avons fait de fort grandes avances, et ce matin encore le chirurgien nous déclarait que si M. Porthos ne le payait pas, c'était à moi qu'il s'en prendrait, attendu que c'était moi qui l'avais envoyé chercher. 

\speak  Mais Porthos est donc blessé? 

\speak  Je ne saurais vous le dire, monsieur. 

\speak  Comment, vous ne sauriez me le dire? vous devriez cependant être mieux informé que personne. 

\speak  Oui, mais dans notre état nous ne disons pas tout ce que nous savons, monsieur, surtout quand on nous a prévenus que nos oreilles répondraient pour notre langue. 

\speak  Eh bien, puis-je voir Porthos? 

\speak  Certainement, monsieur. Prenez l'escalier, montez au premier et frappez au n° 1. Seulement, prévenez que c'est vous. 

\speak  Comment! que je prévienne que c'est moi? 

\speak  Oui, car il pourrait vous arriver malheur. 

\speak  Et quel malheur voulez-vous qu'il m'arrive? 

\speak  M. Porthos peut vous prendre pour quelqu'un de la maison et, dans un mouvement de colère, vous passer son épée à travers le corps ou vous brûler la cervelle. 

\speak  Que lui avez-vous donc fait? 

\speak  Nous lui avons demandé de l'argent. 

\speak  Ah! diable, je comprends cela; c'est une demande que Porthos reçoit très mal quand il n'est pas en fonds; mais je sais qu'il devait y être. 

\speak  C'est ce que nous avions pensé aussi, monsieur; comme la maison est fort régulière et que nous faisons nos comptes toutes les semaines, au bout de huit jours nous lui avons présenté notre note; mais il paraît que nous sommes tombés dans un mauvais moment, car, au premier mot que nous avons prononcé sur la chose, il nous a envoyés à tous les diables; il est vrai qu'il avait joué la veille. 

\speak  Comment, il avait joué la veille! et avec qui? 

\speak  Oh! mon Dieu, qui sait cela? avec un seigneur qui passait et auquel il avait fait proposer une partie de lansquenet. 

\speak  C'est cela, le malheureux aura tout perdu. 

\speak  Jusqu'à son cheval, monsieur, car lorsque l'étranger a été pour partir, nous nous sommes aperçus que son laquais sellait le cheval de M. Porthos. Alors nous lui en avons fait l'observation, mais il nous a répondu que nous nous mêlions de ce qui ne nous regardait pas et que ce cheval était à lui. Nous avons aussitôt fait prévenir M. Porthos de ce qui se passait, mais il nous à fait dire que nous étions des faquins de douter de la parole d'un gentilhomme, et que, puisque celui-là avait dit que le cheval était à lui, il fallait bien que cela fût. 

\speak  Je le reconnais bien là, murmura d'Artagnan. 

\speak  Alors, continua l'hôte, je lui fis répondre que du moment où nous paraissions destinés à ne pas nous entendre à l'endroit du paiement, j'espérais qu'il aurait au moins la bonté d'accorder la faveur de sa pratique à mon confrère le maître de l'Aigle d'Or; mais M. Porthos me répondit que mon hôtel étant le meilleur, il désirait y rester. 

«Cette réponse était trop flatteuse pour que j'insistasse sur son départ. Je me bornai donc à le prier de me rendre sa chambre, qui est la plus belle de l'hôtel, et de se contenter d'un joli petit cabinet au troisième. Mais à ceci M. Porthos répondit que, comme il attendait d'un moment à l'autre sa maîtresse, qui était une des plus grandes dames de la cour, je devais comprendre que la chambre qu'il me faisait l'honneur d'habiter chez moi était encore bien médiocre pour une pareille personne. 

«Cependant, tout en reconnaissant la vérité de ce qu'il disait, je crus devoir insister; mais, sans même se donner la peine d'entrer en discussion avec moi, il prit son pistolet, le mit sur sa table de nuit et déclara qu'au premier mot qu'on lui dirait d'un déménagement quelconque à l'extérieur ou à l'intérieur, il brûlerait la cervelle à celui qui serait assez imprudent pour se mêler d'une chose qui ne regardait que lui. Aussi, depuis ce temps-là, monsieur, personne n'entre plus dans sa chambre, si ce n'est son domestique. 

\speak  Mousqueton est donc ici? 

\speak  Oui, monsieur; cinq jours après son départ, il est revenu de fort mauvaise humeur de son côté; il paraît que lui aussi a eu du désagrément dans son voyage. Malheureusement, il est plus ingambe que son maître, ce qui fait que pour son maître il met tout sens dessus dessous, attendu que, comme il pense qu'on pourrait lui refuser ce qu'il demande, il prend tout ce dont il a besoin sans demander. 

\speak  Le fait est, répondit d'Artagnan, que j'ai toujours remarqué dans Mousqueton un dévouement et une intelligence très supérieurs. 

\speak  Cela est possible, monsieur; mais supposez qu'il m'arrive seulement quatre fois par an de me trouver en contact avec une intelligence et un dévouement semblables, et je suis un homme ruiné. 

\speak  Non, car Porthos vous paiera. 

\speak  Hum! fit l'hôtelier d'un ton de doute. 

\speak  C'est le favori d'une très grande dame qui ne le laissera pas dans l'embarras pour une misère comme celle qu'il vous doit. 

\speak  Si j'ose dire ce que je crois là-dessus\dots 

\speak  Ce que vous croyez? 

\speak  Je dirai plus: ce que je sais. 

\speak  Ce que vous savez? 

\speak  Et même ce dont je suis sûr. 

\speak  Et de quoi êtes-vous sûr, voyons? 

\speak  Je dirai que je connais cette grande dame. 

\speak  Vous? 

\speak  Oui, moi. 

\speak  Et comment la connaissez-vous? 

\speak  Oh! monsieur, si je croyais pouvoir me fier à votre discrétion\dots 

\speak  Parlez, et foi de gentilhomme, vous n'aurez pas à vous repentir de votre confiance. 

\speak  Eh bien, monsieur, vous concevez, l'inquiétude fait faire bien des choses. 

\speak  Qu'avez-vous fait? 

\speak  Oh! d'ailleurs, rien qui ne soit dans le droit d'un créancier. 

\speak  Enfin? 

\speak  M. Porthos nous a remis un billet pour cette duchesse, en nous recommandant de le jeter à la poste. Son domestique n'était pas encore arrivé. Comme il ne pouvait pas quitter sa chambre, il fallait bien qu'il nous chargeât de ses commissions. 

\speak  Ensuite? 

\speak  Au lieu de mettre la lettre à la poste, ce qui n'est jamais bien sûr, j'ai profité de l'occasion de l'un de mes garçons qui allait à Paris, et je lui ai ordonné de la remettre à cette duchesse elle-même. C'était remplir les intentions de M. Porthos, qui nous avait si fort recommandé cette lettre, n'est-ce pas? 

\speak  À peu près. 

\speak  Eh bien, monsieur, savez-vous ce que c'est que cette grande dame? 

\speak  Non; j'en ai entendu parler à Porthos, voilà tout. 

\speak  Savez-vous ce que c'est que cette prétendue duchesse? 

\speak  Je vous le répète, je ne la connais pas. 

\speak  C'est une vieille procureuse au Châtelet, monsieur, nommée Mme Coquenard, laquelle a au moins cinquante ans, et se donne encore des airs d'être jalouse. Cela me paraissait aussi fort singulier, une princesse qui demeure rue aux Ours. 

\speak  Comment savez-vous cela? 

\speak  Parce qu'elle s'est mise dans une grande colère en recevant la lettre, disant que M. Porthos était un volage, et que c'était encore pour quelque femme qu'il avait reçu ce coup d'épée. 

\speak  Mais il a donc reçu un coup d'épée? 

\speak  Ah! mon Dieu! qu'ai-je dit là? 

\speak  Vous avez dit que Porthos avait reçu un coup d'épée. 

\speak  Oui; mais il m'avait si fort défendu de le dire! 

\speak  Pourquoi cela? 

\speak  Dame! monsieur, parce qu'il s'était vanté de perforer cet étranger avec lequel vous l'avez laisse en dispute, et que c'est cet étranger, au contraire, qui, malgré toutes ses rodomontades, l'a couché sur le carreau. Or, comme M. Porthos est un homme fort glorieux, excepté envers la duchesse, qu'il avait cru intéresser en lui faisant le récit de son aventure, il ne veut avouer à personne que c'est un coup d'épée qu'il a reçu. 

\speak  Ainsi c'est donc un coup d'épée qui le retient dans son lit? 

\speak  Et un maître coup d'épée, je vous l'assure. Il faut que votre ami ait l'âme chevillée dans le corps. 

\speak  Vous étiez donc là? 

\speak  Monsieur, je les avais suivis par curiosité, de sorte que j'ai vu le combat sans que les combattants me vissent. 

\speak  Et comment cela s'est-il passé? 

\speak  Oh! la chose n'a pas été longue, je vous en réponds. Ils se sont mis en garde; l'étranger a fait une feinte et s'est fendu; tout cela si rapidement, que lorsque M. Porthos est arrivé à la parade, il avait déjà trois pouces de fer dans la poitrine. Il est tombé en arrière. L'étranger lui a mis aussitôt la pointe de son épée à la gorge; et M. Porthos, se voyant à la merci de son adversaire, s'est avoué vaincu. Sur quoi, l'étranger lui a demandé son nom et apprenant qu'il s'appelait M. Porthos, et non M. d'Artagnan, lui a offert son bras, l'a ramené à l'hôtel, est monté à cheval et a disparu. 

\speak  Ainsi c'est à M. d'Artagnan qu'en voulait cet étranger? 

\speak  Il paraît que oui. 

\speak  Et savez-vous ce qu'il est devenu? 

\speak  Non; je ne l'avais jamais vu jusqu'à ce moment et nous ne l'avons pas revu depuis. 

\speak  Très bien; je sais ce que je voulais savoir. Maintenant, vous dites que la chambre de Porthos est au premier, n° 1? 

\speak  Oui, monsieur, la plus belle de l'auberge; une chambre que j'aurais déjà eu dix fois l'occasion de louer. 

\speak  Bah! tranquillisez vous, dit d'Artagnan en riant; Porthos vous paiera avec l'argent de la duchesse Coquenard. 

\speak  Oh! monsieur, procureuse ou duchesse, si elle lâchait les cordons de sa bourse, ce ne serait rien; mais elle a positivement répondu qu'elle était lasse des exigences et des infidélités de M. Porthos, et qu'elle ne lui enverrait pas un denier. 

\speak  Et avez-vous rendu cette réponse à votre hôte? 

\speak  Nous nous en sommes bien gardés: il aurait vu de quelle manière nous avions fait la commission. 

\speak  Si bien qu'il attend toujours son argent? 

\speak  Oh! mon Dieu, oui! Hier encore, il a écrit; mais, cette fois, c'est son domestique qui a mis la lettre à la poste. 

\speak  Et vous dites que la procureuse est vieille et laide. 

\speak  Cinquante ans au moins, monsieur, et pas belle du tout, à ce qu'a dit Pathaud. 

\speak  En ce cas, soyez tranquille, elle se laissera attendrir; d'ailleurs Porthos ne peut pas vous devoir grand-chose. 

\speak  Comment, pas grand-chose! Une vingtaine de pistoles déjà, sans compter le médecin. Oh! il ne se refuse rien, allez! on voit qu'il est habitué à bien vivre. 

\speak  Eh bien, si sa maîtresse l'abandonne, il trouvera des amis, je vous le certifie. Ainsi, mon cher hôte, n'ayez aucune inquiétude, et continuez d'avoir pour lui tous les soins qu'exige son état. 

\speak  Monsieur m'a promis de ne pas parler de la procureuse et de ne pas dire un mot de la blessure. 

\speak  C'est chose convenue; vous avez ma parole. 

\speak  Oh! c'est qu'il me tuerait, voyez-vous! 

\speak  N'ayez pas peur; il n'est pas si diable qu'il en a l'air. 

En disant ces mots, d'Artagnan monta l'escalier, laissant son hôte un peu plus rassuré à l'endroit de deux choses auxquelles il paraissait beaucoup tenir: sa créance et sa vie. 

Au haut de l'escalier, sur la porte la plus apparente du corridor était tracé, à l'encre noire, un n° 1 gigantesque; d'Artagnan frappa un coup, et, sur l'invitation de passer outre qui lui vint de l'intérieur, il entra. 

Porthos était couché, et faisait une partie de lansquenet avec Mousqueton, pour s'entretenir la main, tandis qu'une broche chargée de perdrix tournait devant le feu, et qu'à chaque coin d'une grande cheminée bouillaient sur deux réchauds deux casseroles, d'où s'exhalait une double odeur de gibelotte et de matelote qui réjouissait l'odorat. En outre, le haut d'un secrétaire et le marbre d'une commode étaient couverts de bouteilles vides. 

À la vue de son ami, Porthos jeta un grand cri de joie; et Mousqueton, se levant respectueusement, lui céda la place et s'en alla donner un coup d'œil aux deux casseroles, dont il paraissait avoir l'inspection particulière. 

«Ah! pardieu! c'est vous, dit Porthos à d'Artagnan, soyez le bienvenu, et excusez-moi si je ne vais pas au-devant de vous. Mais, ajouta-t-il en regardant d'Artagnan avec une certaine inquiétude, vous savez ce qui m'est arrivé? 

\speak  Non. 

\speak  L'hôte ne vous a rien dit? 

\speak  J'ai demandé après vous, et je suis monté tout droit.» 

\speak  Porthos parut respirer plus librement. 

«Et que vous est-il donc arrivé, mon cher Porthos? continua d'Artagnan. 

\speak  Il m'est arrivé qu'en me fendant sur mon adversaire, à qui j'avais déjà allongé trois coups d'épée, et avec lequel je voulais en finir d'un quatrième, mon pied a porté sur une pierre, et je me suis foulé le genou. 

\speak  Vraiment? 

\speak  D'honneur! Heureusement pour le maraud, car je ne l'aurais laissé que mort sur la place, je vous en réponds. 

\speak  Et qu'est-il devenu? 

\speak  Oh! je n'en sais rien; il en a eu assez, et il est parti sans demander son reste; mais vous, mon cher d'Artagnan, que vous est-il arrivé? 

\speak  De sorte, continua d'Artagnan, que cette foulure, mon cher Porthos, vous retient au lit? 

\speak  Ah! mon Dieu, oui, voilà tout; du reste, dans quelques jours je serai sur pied. 

\speak  Pourquoi alors ne vous êtes-vous pas fait transporter à Paris? Vous devez vous ennuyer cruellement ici. 

\speak  C'était mon intention; mais, mon cher ami, il faut que je vous avoue une chose. 

\speak  Laquelle? 

\speak  C'est que, comme je m'ennuyais cruellement, ainsi que vous le dites, et que j'avais dans ma poche les soixante-quinze pistoles que vous m'aviez distribuées j'ai, pour me distraire, fait monter près de moi un gentilhomme qui était de passage, et auquel j'ai proposé de faire une partie de dés. Il a accepté, et, ma foi, mes soixante-quinze pistoles sont passées de ma poche dans la sienne, sans compter mon cheval, qu'il a encore emporté par dessus le marché. Mais vous, mon cher d'Artagnan? 

\speak  Que voulez-vous, mon cher Porthos, on ne peut pas être privilégié de toutes façons, dit d'Artagnan; vous savez le proverbe: “Malheureux au jeu, heureux en amour.” Vous êtes trop heureux en amour pour que le jeu ne se venge pas; mais que vous importent, à vous, les revers de la fortune! n'avez-vous pas, heureux coquin que vous êtes, n'avez-vous pas votre duchesse, qui ne peut manquer de vous venir en aide? 

\speak  Eh bien, voyez, mon cher d'Artagnan, comme je joue de guignon, répondit Porthos de l'air le plus dégagé du monde! je lui ai écrit de m'envoyer quelque cinquante louis dont j'avais absolument besoin, vu la position où je me trouvais\dots 

\speak  Eh bien? 

\speak  Eh bien, il faut qu'elle soit dans ses terres, car elle ne m'a pas répondu. 

\speak  Vraiment? 

\speak  Non. Aussi je lui ai adressé hier une seconde épître plus pressante encore que la première; mais vous voilà, mon très cher, parlons de vous. Je commençais, je vous l'avoue, à être dans une certaine inquiétude sur votre compte. 

\speak  Mais votre hôte se conduit bien envers vous, à ce qu'il paraît, mon cher Porthos, dit d'Artagnan, montrant au malade les casseroles pleines et les bouteilles vides. 

\speak  Couci-couci! répondit Porthos. Il y a déjà trois ou quatre jours que l'impertinent m'a monté son compte, et que je les ai mis à la porte, son compte et lui; de sorte que je suis ici comme une façon de vainqueur, comme une manière de conquérant. Aussi, vous le voyez, craignant toujours d'être forcé dans la position, je suis armé jusqu'aux dents. 

\speak  Cependant, dit en riant d'Artagnan, il me semble que de temps en temps vous faites des sorties.» 

Et il montrait du doigt les bouteilles et les casseroles. 

«Non, pas moi, malheureusement! dit Porthos. Cette misérable foulure me retient au lit, mais Mousqueton bat la campagne, et il rapporte des vivres. Mousqueton, mon ami, continua Porthos, vous voyez qu'il nous arrive du renfort, il nous faudra un supplément de victuailles. 

\speak  Mousqueton, dit d'Artagnan, il faudra que vous me rendiez un service. 

\speak  Lequel, monsieur? 

\speak  C'est de donner votre recette à Planchet; je pourrais me trouver assiégé à mon tour, et je ne serais pas fâché qu'il me fît jouir des mêmes avantages dont vous gratifiez votre maître. 

\speak  Eh! mon Dieu! monsieur, dit Mousqueton d'un air modeste, rien de plus facile. Il s'agit d'être adroit, voilà tout. J'ai été élevé à la campagne, et mon père, dans ses moments perdus, était quelque peu braconnier. 

\speak  Et le reste du temps, que faisait-il? 

\speak  Monsieur, il pratiquait une industrie que j'ai toujours trouvée assez heureuse. 

\speak  Laquelle? 

\speak  Comme c'était au temps des guerres des catholiques et des huguenots, et qu'il voyait les catholiques exterminer les huguenots, et les huguenots exterminer les catholiques, le tout au nom de la religion, il s'était fait une croyance mixte, ce qui lui permettait d'être tantôt catholique, tantôt huguenot. Or il se promenait habituellement, son escopette sur l'épaule, derrière les haies qui bordent les chemins, et quand il voyait venir un catholique seul, la religion protestante l'emportait aussitôt dans son esprit. Il abaissait son escopette dans la direction du voyageur; puis, lorsqu'il était à dix pas de lui, il entamait un dialogue qui finissait presque toujours par l'abandon que le voyageur faisait de sa bourse pour sauver sa vie. Il va sans dire que lorsqu'il voyait venir un huguenot, il se sentait pris d'un zèle catholique si ardent, qu'il ne comprenait pas comment, un quart d'heure auparavant, il avait pu avoir des doutes sur la supériorité de notre sainte religion. Car, moi, monsieur, je suis catholique, mon père, fidèle à ses principes, ayant fait mon frère aîné huguenot. 

\speak  Et comment a fini ce digne homme? demanda d'Artagnan. 

\speak  Oh! de la façon la plus malheureuse, monsieur. Un jour, il s'était trouvé pris dans un chemin creux entre un huguenot et un catholique à qui il avait déjà eu affaire, et qui le reconnurent tous deux; de sorte qu'ils se réunirent contre lui et le pendirent à un arbre; puis ils vinrent se vanter de la belle équipée qu'ils avaient faite dans le cabaret du premier village, où nous étions à boire, mon frère et moi. 

\speak  Et que fîtes-vous? dit d'Artagnan. 

\speak  Nous les laissâmes dire, reprit Mousqueton. Puis comme, en sortant de ce cabaret, ils prenaient chacun une route opposée, mon frère alla s'embusquer sur le chemin du catholique, et moi sur celui du protestant. Deux heures après, tout était fini, nous leur avions fait à chacun son affaire, tout en admirant la prévoyance de notre pauvre père qui avait pris la précaution de nous élever chacun dans une religion différente. 

\speak  En effet, comme vous le dites, Mousqueton, votre père me paraît avoir été un gaillard fort intelligent. Et vous dites donc que, dans ses moments perdus, le brave homme était braconnier? 

\speak  Oui, monsieur, et c'est lui qui m'a appris à nouer un collet et à placer une ligne de fond. Il en résulte que lorsque j'ai vu que notre gredin d'hôte nous nourrissait d'un tas de grosses viandes bonnes pour des manants, et qui n'allaient point à deux estomacs aussi débilités que les nôtres, je me suis remis quelque peu à mon ancien métier. Tout en me promenant dans le bois de M. le Prince, j'ai tendu des collets dans les passées; tout en me couchant au bord des pièces d'eau de Son Altesse, j'ai glissé des lignes dans les étangs. De sorte que maintenant, grâce à Dieu, nous ne manquons pas, comme monsieur peut s'en assurer, de perdrix et de lapins, de carpes et d'anguilles, tous aliments légers et sains, convenables pour des malades. 

\speak  Mais le vin, dit d'Artagnan, qui fournit le vin? c'est votre hôte? 

\speak  C'est-à-dire, oui et non. 

\speak  Comment, oui et non? 

\speak  Il le fournit, il est vrai, mais il ignore qu'il a cet honneur. 

\speak  Expliquez-vous, Mousqueton, votre conversation est pleine de choses instructives. 

\speak  Voici, monsieur. Le hasard a fait que j'ai rencontré dans mes pérégrinations un Espagnol qui avait vu beaucoup de pays, et entre autres le Nouveau Monde. 

\speak  Quel rapport le Nouveau Monde peut-il avoir avec les bouteilles qui sont sur ce secrétaire et sur cette commode? 

\speak  Patience, monsieur, chaque chose viendra à son tour. 

\speak  C'est juste, Mousqueton; je m'en rapporte à vous, et j'écoute. 

\speak  Cet Espagnol avait à son service un laquais qui l'avait accompagné dans son voyage au Mexique. Ce laquais était mon compatriote, de sorte que nous nous liâmes d'autant plus rapidement qu'il y avait entre nous de grands rapports de caractère. Nous aimions tous deux la chasse par-dessus tout, de sorte qu'il me racontait comment, dans les plaines de pampas, les naturels du pays chassent le tigre et les taureaux avec de simples noeuds coulants qu'ils jettent au cou de ces terribles animaux. D'abord, je ne voulais pas croire qu'on pût en arriver à ce degré d'adresse, de jeter à vingt ou trente pas l'extrémité d'une corde où l'on veut; mais devant la preuve il fallait bien reconnaître la vérité du récit. Mon ami plaçait une bouteille à trente pas, et à chaque coup il lui prenait le goulot dans un noeud coulant. Je me livrai à cet exercice, et comme la nature m'a doué de quelques facultés, aujourd'hui je jette le lasso aussi bien qu'aucun homme du monde. Eh bien, comprenez-vous? Notre hôte a une cave très bien garnie, mais dont la clef ne le quitte pas; seulement, cette cave a un soupirail. Or, par ce soupirail, je jette le lasso; et comme je sais maintenant où est le bon coin, j'y puise. Voici, monsieur, comment le Nouveau Monde se trouve être en rapport avec les bouteilles qui sont sur cette commode et sur ce secrétaire. Maintenant, voulez-vous goûter notre vin, et, sans prévention, vous nous direz ce que vous en pensez. 

\speak  Merci, mon ami, merci; malheureusement, je viens de déjeuner. 

\speak  Eh bien, dit Porthos, mets la table, Mousqueton, et tandis que nous déjeunerons, nous, d'Artagnan nous racontera ce qu'il est devenu lui-même, depuis dix jours qu'il nous a quittés. 

\speak  Volontiers», dit d'Artagnan. 

Tandis que Porthos et Mousqueton déjeunaient avec des appétits de convalescents et cette cordialité de frères qui rapproche les hommes dans le malheur, d'Artagnan raconta comment Aramis blessé avait été forcé de s'arrêter à Crèvecœur, comment il avait laissé Athos se débattre à Amiens entre les mains de quatre hommes qui l'accusaient d'être un faux-monnayeur, et comment, lui, d'Artagnan, avait été forcé de passer sur le ventre du comte de Wardes pour arriver jusqu'en Angleterre. 

Mais là s'arrêta la confidence de d'Artagnan; il annonça seulement qu'à son retour de la Grande-Bretagne il avait ramené quatre chevaux magnifiques, dont un pour lui et un autre pour chacun de ses compagnons, puis il termina en annonçant à Porthos que celui qui lui était destiné était déjà installé dans l'écurie de l'hôtel. 

En ce moment Planchet entra; il prévenait son maître que les chevaux étaient suffisamment reposés, et qu'il serait possible d'aller coucher à Clermont. 

Comme d'Artagnan était à peu près rassuré sur Porthos, et qu'il lui tardait d'avoir des nouvelles de ses deux autres amis, il tendit la main au malade, et le prévint qu'il allait se mettre en route pour continuer ses recherches. Au reste, comme il comptait revenir par la même route, si, dans sept à huit jours, Porthos était encore à l'hôtel du Grand Saint Martin, il le reprendrait en passant. 

Porthos répondit que, selon toute probabilité, sa foulure ne lui permettrait pas de s'éloigner d'ici là. D'ailleurs il fallait qu'il restât à Chantilly pour attendre une réponse de sa duchesse. 

D'Artagnan lui souhaita cette réponse prompte et bonne; et après avoir recommandé de nouveau Porthos à Mousqueton, et payé sa dépense à l'hôte, il se remit en route avec Planchet, déjà débarrassé d'un de ses chevaux de main.