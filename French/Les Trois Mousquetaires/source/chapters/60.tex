%!TeX root=../musketeersfr.tex 

\chapter{En France}

\lettrine{L}{a} première crainte du roi d'Angleterre, Charles I\ier\, en apprenant cette mort, fut qu'une si terrible nouvelle ne décourageât les Rochelois; il essaya, dit Richelieu dans ses Mémoires, de la leur cacher le plus longtemps possible, faisant fermer les ports par tout son royaume, et prenant soigneusement garde qu'aucun vaisseau ne sortit jusqu'à ce que l'armée que Buckingham apprêtait fût partie, se chargeant, à défaut de Buckingham, de surveiller lui-même le départ. 

Il poussa même la sévérité de cet ordre jusqu'à retenir en Angleterre l'ambassadeur de Danemark, qui avait pris congé, et l'ambassadeur ordinaire de Hollande, qui devait ramener dans le port de Flessingue les navires des Indes que Charles I\ier\ avait fait restituer aux Provinces-Unies. 

Mais comme il ne songea à donner cet ordre que cinq heures après l'événement, c'est-à-dire à deux heures de l'après-midi, deux navires étaient déjà sortis du port: l'un emmenant, comme nous le savons, Milady, laquelle, se doutant déjà de l'événement, fut encore confirmée dans cette croyance en voyant le pavillon noir se déployer au mât du vaisseau amiral. 

Quant au second bâtiment, nous dirons plus tard qui il portait et comment il partit. 

Pendant ce temps, du reste, rien de nouveau au camp de La Rochelle; seulement le roi, qui s'ennuyait fort, comme toujours, mais peut-être encore un peu plus au camp qu'ailleurs, résolut d'aller incognito passer les fêtes de Saint-Louis à Saint-Germain, et demanda au cardinal de lui faire préparer une escorte de vingt mousquetaires seulement. Le cardinal, que l'ennui du roi gagnait quelquefois, accorda avec grand plaisir ce congé à son royal lieutenant, lequel promit d'être de retour vers le 15 septembre. 

M. de Tréville, prévenu par Son Éminence, fit son portemanteau, et comme, sans en savoir la cause, il savait le vif désir et même l'impérieux besoin que ses amis avaient de revenir à Paris, il va sans dire qu'il les désigna pour faire partie de l'escorte. 

Les quatre jeunes gens surent la nouvelle un quart d'heure après M. de Tréville, car ils furent les premiers à qui il la communiqua. Ce fut alors que d'Artagnan apprécia la faveur que lui avait accordée le cardinal en le faisant enfin passer aux mousquetaires; sans cette circonstance, il était forcé de rester au camp tandis que ses compagnons partaient. 

On verra plus tard que cette impatience de remonter vers Paris avait pour cause le danger que devait courir Mme Bonacieux en se rencontrant au couvent de Béthune avec Milady, son ennemie mortelle. Aussi, comme nous l'avons dit, Aramis avait écrit immédiatement à Marie Michon, cette lingère de Tours qui avait de si belles connaissances, pour qu'elle obtînt que la reine donnât l'autorisation à Mme Bonacieux de sortir du couvent et de se retirer soit en Lorraine, soit en Belgique. La réponse ne s'était pas fait attendre, et, huit ou dix jours après, Aramis avait reçu cette lettre: 


\begin{mail}{}{Mon cher cousin,}
	
Voici l'autorisation de ma sœur à retirer notre petite servante du couvent de Béthune, dont vous pensez que l'air est mauvais pour elle. Ma sœur vous envoie cette autorisation avec grand plaisir, car elle aime fort cette petite fille, à laquelle elle se réserve d'être utile plus tard.
\closeletter[Je vous embrasse.]{Marie Michon}
\end{mail}

À cette lettre était jointe une autorisation ainsi conçue:

\begin{mail}{Au Louvre, le 10 août 1628.}{}
La supérieure du couvent de Béthune remettra aux mains de la personne qui lui remettra ce billet la novice qui était entrée dans son couvent sous ma recommandation et sous mon patronage.\closeletter{Anne}
\end{mail}


On comprend combien ces relations de parenté entre Aramis et une lingère qui appelait la reine sa sœur avaient égayé la verve des jeunes gens; mais Aramis, après avoir rougi deux ou trois fois jusqu'au blanc des yeux aux grosses plaisanteries de Porthos, avait prié ses amis de ne plus revenir sur ce sujet, déclarant que s'il lui en était dit encore un seul mot, il n'emploierait plus sa cousine comme intermédiaire dans ces sortes d'affaires. 

Il ne fut donc plus question de Marie Michon entre les quatre mousquetaires, qui d'ailleurs avaient ce qu'ils voulaient: l'ordre de tirer Mme Bonacieux du couvent des Carmélites de Béthune. Il est vrai que cet ordre ne leur servirait pas à grand-chose tant qu'ils seraient au camp de La Rochelle, c'est-à-dire à l'autre bout de la France; aussi d'Artagnan allait-il demander un congé à M. de Tréville, en lui confiant tout bonnement l'importance de son départ, lorsque cette nouvelle lui fut transmise, ainsi qu'à ses trois compagnons, que le roi allait partir pour Paris avec une escorte de vingt mousquetaires, et qu'ils faisaient partie de l'escorte. 

La joie fut grande. On envoya les valets devant avec les bagages, et l'on partit le 16 au matin. 

Le cardinal reconduisit Sa Majesté de Surgères à Mauzé, et là, le roi et son ministre prirent congé l'un de l'autre avec de grandes démonstrations d'amitié. 

Cependant le roi, qui cherchait de la distraction, tout en cheminant le plus vite qu'il lui était possible, car il désirait être arrivé à Paris pour le 23, s'arrêtait de temps en temps pour voler la pie, passe-temps dont le goût lui avait autrefois été inspiré par de Luynes, et pour lequel il avait toujours conservé une grande prédilection. Sur les vingt mousquetaires, seize, lorsque la chose arrivait, se réjouissaient fort de ce bon temps; mais quatre maugréaient de leur mieux. D'Artagnan surtout avait des bourdonnements perpétuels dans les oreilles, ce que Porthos expliquait ainsi: 

«Une très grande dame m'a appris que cela veut dire que l'on parle de vous quelque part.» 

Enfin l'escorte traversa Paris le 23, dans la nuit; le roi remercia M. de Tréville, et lui permit de distribuer des congés pour quatre jours, à la condition que pas un des favorisés ne paraîtrait dans un lieu public, sous peine de la Bastille. 

Les quatre premiers congés accordés, comme on le pense bien, furent à nos quatre amis. Il y a plus, Athos obtint de M. de Tréville six jours au lieu de quatre et fit mettre dans ces six jours deux nuits de plus, car ils partirent le 24, à cinq heures du soir, et par complaisance encore, M. de Tréville postdata le congé du 25 au matin. 

«Eh, mon Dieu, disait d'Artagnan, qui, comme on le sait, ne doutait jamais de rien, il me semble que nous faisons bien de l'embarras pour une chose bien simple: en deux jours, et en crevant deux ou trois chevaux (peu m'importe: j'ai de l'argent), je suis à Béthune, je remets la lettre de la reine à la supérieure, et je ramène le cher trésor que je vais chercher, non pas en Lorraine, non pas en Belgique, mais à Paris, où il sera mieux caché, surtout tant que M. le cardinal sera à La Rochelle. Puis, une fois de retour de la campagne, eh bien, moitié par la protection de sa cousine, moitié en faveur de ce que nous avons fait personnellement pour elle, nous obtiendrons de la reine ce que nous voudrons. Restez donc ici, ne vous épuisez pas de fatigue inutilement; moi et Planchet, c'est tout ce qu'il faut pour une expédition aussi simple.» 

À ceci Athos répondit tranquillement: 

«Nous aussi, nous avons de l'argent; car je n'ai pas encore bu tout à fait le reste du diamant, et Porthos et Aramis ne l'ont pas tout à fait mangé. Nous crèverons donc aussi bien quatre chevaux qu'un. Mais songez, d'Artagnan, ajouta-t-il d'une voix si sombre que son accent donna le frisson au jeune homme, songez que Béthune est une ville où le cardinal a donné rendez-vous à une femme qui, partout où elle va, mène le malheur après elle. Si vous n'aviez affaire qu'à quatre hommes, d'Artagnan, je vous laisserais aller seul; vous avez affaire à cette femme, allons-y quatre, et plaise à Dieu qu'avec nos quatre valets nous soyons en nombre suffisant! 

\speak  Vous m'épouvantez, Athos, s'écria d'Artagnan; que craignez-vous donc, mon Dieu? 

\speak  Tout!» répondit Athos. 

D'Artagnan examina les visages de ses compagnons, qui, comme celui d'Athos, portaient l'empreinte d'une inquiétude profonde, et l'on continua la route au plus grand pas des chevaux, mais sans ajouter une seule parole. 

Le 25 au soir, comme ils entraient à Arras, et comme d'Artagnan venait de mettre pied à terre à l'auberge de la Herse d'Or pour boire un verre de vin, un cavalier sortit de la cour de la poste, où il venait de relayer, prenant au grand galop, et avec un cheval frais, le chemin de Paris. Au moment où il passait de la grande porte dans la rue, le vent entrouvrit le manteau dont il était enveloppé, quoiqu'on fût au mois d'août, et enleva son chapeau, que le voyageur retint de sa main, au moment où il avait déjà quitté sa tête, et l'enfonça vivement sur ses yeux. 

D'Artagnan, qui avait les yeux fixés sur cet homme, devint fort pâle et laissa tomber son verre. 

«Qu'avez-vous, monsieur? dit Planchet\dots Oh! là, accourez, messieurs, voilà mon maître qui se trouve mal!» 

Les trois amis accoururent et trouvèrent d'Artagnan qui, au lieu de se trouver mal, courait à son cheval. Ils l'arrêtèrent sur le seuil de la porte. 

«Eh bien, où diable vas-tu donc ainsi? lui cria Athos. 

\speak  C'est lui! s'écria d'Artagnan, pâle de colère et la sueur sur le front, c'est lui! laissez-moi le rejoindre! 

\speak  Mais qui, lui? demanda Athos. 

\speak  Lui, cet homme! 

\speak  Quel homme? 

\speak  Cet homme maudit, mon mauvais génie, que j'ai toujours vu lorsque j'étais menacé de quelque malheur: celui qui accompagnait l'horrible femme lorsque je la rencontrai pour la première fois, celui que je cherchais quand j'ai provoqué Athos, celui que j'ai vu le matin du jour où Mme Bonacieux a été enlevée! l'homme de Meung enfin! je l'ai vu, c'est lui! Je l'ai reconnu quand le vent a entrouvert son manteau. 

\speak  Diable! dit Athos rêveur. 

\speak  En selle, messieurs, en selle; poursuivons-le, et nous le rattraperons. 

\speak  Mon cher, dit Aramis, songez qu'il va du côté opposé à celui où nous allons; qu'il a un cheval frais et que nos chevaux sont fatigués; que par conséquent nous crèverons nos chevaux sans même avoir la chance de le rejoindre. Laissons l'homme, d'Artagnan, sauvons la femme. 

\speak  Eh! monsieur! s'écria un garçon d'écurie courant après l'inconnu, eh! monsieur, voilà un papier qui s'est échappé de votre chapeau! Eh! monsieur! eh! 

\speak  Mon ami, dit d'Artagnan, une demi-pistole pour ce papier! 

\speak  Ma foi, monsieur, avec grand plaisir! le voici! 

Le garçon d'écurie, enchanté de la bonne journée qu'il avait faite, rentra dans la cour de l'hôtel: d'Artagnan déplia le papier. 

«Eh bien? demandèrent ses amis en l'entourant. 

\speak  Rien qu'un mot! dit d'Artagnan. 

\speak  Oui, dit Aramis, mais ce nom est un nom de ville ou de village. 

\speak «\textit{Armentières}», lut Porthos. Armentières, je ne connais pas cela! 

\speak  Et ce nom de ville ou de village est écrit de sa main! s'écria Athos. 

\speak  Allons, allons, gardons soigneusement ce papier, dit d'Artagnan, peut-être n'ai-je pas perdu ma dernière pistole. À cheval, mes amis, à cheval!» 

Et les quatre compagnons s'élancèrent au galop sur la route de Béthune.