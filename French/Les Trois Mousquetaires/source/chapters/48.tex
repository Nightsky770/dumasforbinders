%!TeX root=../musketeersfr.tex 

\chapter{Affaire De Famille}

\lettrine{A}{thos} avait trouvé le mot: \textit{affaire de famille}. Une affaire de famille n'était point soumise à l'investigation du cardinal; une affaire de famille ne regardait personne; on pouvait s'occuper devant tout le monde d'une affaire de famille. 

Ainsi, Athos avait trouvé le mot: affaire de famille. 

Aramis avait trouvé l'idée: les laquais. 

Porthos avait trouvé le moyen: le diamant. 

D'Artagnan seul n'avait rien trouvé, lui ordinairement le plus inventif des quatre; mais il faut dire aussi que le nom seul de Milady le paralysait. 

Ah! si; nous nous trompons: il avait trouvé un acheteur pour le diamant. 

Le déjeuner chez M. de Tréville fut d'une gaieté charmante. D'Artagnan avait déjà son uniforme; comme il était à peu près de la même taille qu'Aramis, et qu'Aramis, largement payé, comme on se le rappelle, par le libraire qui lui avait acheté son poème, avait fait tout en double, il avait cédé à son ami un équipement complet. 

D'Artagnan eût été au comble de ses voeux, s'il n'eût point vu pointer Milady, comme un nuage sombre à l'horizon. 

Après déjeuner, on convint qu'on se réunirait le soir au logis d'Athos, et que là on terminerait l'affaire. 

D'Artagnan passa la journée à montrer son habit de mousquetaire dans toutes les rues du camp. 

Le soir, à l'heure dite, les quatre amis se réunirent: il ne restait plus que trois choses à décider: 

Ce qu'on écrirait au frère de Milady; 

Ce qu'on écrirait à la personne adroite de Tours; 

Et quels seraient les laquais qui porteraient les lettres. 

Chacun offrait le sien: Athos parlait de la discrétion de Grimaud, qui ne parlait que lorsque son maître lui décousait la bouche; Porthos vantait la force de Mousqueton, qui était de taille à rosser quatre hommes de complexion ordinaire; Aramis, confiant dans l'adresse de Bazin, faisait un éloge pompeux de son candidat; enfin, d'Artagnan avait foi entière dans la bravoure de Planchet, et rappelait de quelle façon il s'était conduit dans l'affaire épineuse de Boulogne. 

Ces quatre vertus disputèrent longtemps le prix, et donnèrent lieu à de magnifiques discours, que nous ne rapporterons pas ici, de peur qu'ils ne fassent longueur. 

«Malheureusement, dit Athos, il faudrait que celui qu'on enverra possédât en lui seul les quatre qualités réunies. 

\speak  Mais où rencontrer un pareil laquais? 

\speak  Introuvable! dit Athos; je le sais bien: prenez donc Grimaud. 

\speak  Prenez Mousqueton. 

\speak  Prenez Bazin. 

\speak  Prenez Planchet; Planchet est brave et adroit: c'est déjà deux qualités sur quatre. 

\speak  Messieurs, dit Aramis, le principal n'est pas de savoir lequel de nos quatre laquais est le plus discret, le plus fort, le plus adroit ou le plus brave; le principal est de savoir lequel aime le plus l'argent. 

\speak  Ce que dit Aramis est plein de sens, reprit Athos; il faut spéculer sur les défauts des gens et non sur leurs vertus: Monsieur l'abbé, vous êtes un grand moraliste! 

\speak  Sans doute, répliqua Aramis; car non seulement nous avons besoin d'être bien servis pour réussir, mais encore pour ne pas échouer; car, en cas d'échec, il y va de la tête, non pas pour les laquais\dots 

\speak  Plus bas, Aramis! dit Athos. 

\speak  C'est juste, non pas pour les laquais, reprit Aramis, mais pour le maître, et même pour les maîtres! Nos valets nous sont-ils assez dévoués pour risquer leur vie pour nous? Non. 

\speak  Ma foi, dit d'Artagnan, je répondrais presque de Planchet, moi. 

\speak  Eh bien, mon cher ami, ajoutez à son dévouement naturel une bonne somme qui lui donne quelque aisance, et alors, au lieu d'en répondre une fois, répondez-en deux. 

\speak  Eh! bon Dieu! vous serez trompés tout de même, dit Athos, qui était optimiste quand il s'agissait des choses, et pessimiste quand il s'agissait des hommes. Ils promettront tout pour avoir de l'argent, et en chemin la peur les empêchera d'agir. Une fois pris, on les serrera; serrés, ils avoueront. Que diable! nous ne sommes pas des enfants! Pour aller en Angleterre (Athos baissa la voix), il faut traverser toute la France, semée d'espions et de créatures du cardinal; il faut une passe pour s'embarquer; il faut savoir l'anglais pour demander son chemin à Londres. Tenez, je vois la chose bien difficile. 

\speak  Mais point du tout, dit d'Artagnan, qui tenait fort à ce que la chose s'accomplît; je la vois facile, au contraire, moi. Il va sans dire, parbleu! que si l'on écrit à Lord de Winter des choses par-dessus les maisons, des horreurs du cardinal\dots 

\speak  Plus bas! dit Athos. 

\speak  Des intrigues et des secrets d'état, continua d'Artagnan en se conformant à la recommandation, il va sans dire que nous serons tous roués vifs; mais, pour Dieu, n'oubliez pas, comme vous l'avez dit vous-même, Athos, que nous lui écrivons pour affaire de famille; que nous lui écrivons à cette seule fin qu'il mette Milady, dès son arrivée à Londres, hors d'état de nous nuire. Je lui écrirai donc une lettre à peu près en ces termes: 

\speak  Voyons, dit Aramis, en prenant par avance un visage de critique. 

\speak «Monsieur et cher ami\dots» 

\speak  Ah! oui; cher ami, à un Anglais, interrompit Athos; bien commencé! bravo, d'Artagnan! Rien qu'avec ce mot-là vous serez écartelé, au lieu d'être roué vif. 

\speak  Eh bien, soit; je dirai donc, monsieur, tout court. 

\speak  Vous pouvez même dire, Milord, reprit Athos, qui tenait fort aux convenances. 

\speak »Milord, vous souvient-il du petit enclos aux chèvres du Luxembourg?» 

\speak  Bon! le Luxembourg à présent! On croira que c'est une allusion à la reine mère! Voilà qui est ingénieux, dit Athos. 

\speak  Eh bien, nous mettrons tout simplement: «Milord, vous souvient-il de certain petit enclos où l'on vous sauva la vie?» 

\speak  Mon cher d'Artagnan, dit Athos, vous ne serez jamais qu'un fort mauvais rédacteur: «Où l'on vous sauva la vie!» Fi donc! ce n'est pas digne. On ne rappelle pas ces services-là à un galant homme. Bienfait reproché, offense faite. 

\speak  Ah! mon cher, dit d'Artagnan, vous êtes insupportable, et s'il faut écrire sous votre censure, ma foi, j'y renonce. 

\speak  Et vous faites bien. Maniez le mousquet et l'épée, mon cher, vous vous tirez galamment des deux exercices; mais passez la plume à M. l'abbé, cela le regarde. 

\speak  Ah! oui, au fait, dit Porthos, passez la plume à Aramis, qui écrit des thèses en latin, lui. 

\speak  Eh bien, soit dit d'Artagnan, rédigez-nous cette note, Aramis; mais, de par notre Saint-Père le pape! tenez-vous serré, car je vous épluche à mon tour, je vous en préviens. 

\speak  Je ne demande pas mieux, dit Aramis avec cette naïve confiance que tout poète a en lui-même; mais qu'on me mette au courant: j'ai bien ouï dire, de-ci de-là, que cette belle-sœur était une coquine, j'en ai même acquis la preuve en écoutant sa conversation avec le cardinal. 

\speak  Plus bas donc, sacrebleu! dit Athos. 

\speak  Mais, continua Aramis, le détail m'échappe. 

\speak  Et à moi aussi», dit Porthos. 

D'Artagnan et Athos se regardèrent quelque temps en silence. Enfin Athos, après s'être recueilli, et en devenant plus pâle encore qu'il n'était de coutume, fit un signe d'adhésion, d'Artagnan comprit qu'il pouvait parler. 

«Eh bien, voici ce qu'il y a à dire, reprit d'Artagnan: Milord, votre belle-sœur est une scélérate, qui a voulu vous faire tuer pour hériter de vous. Mais elle ne pouvait épouser votre frère, étant déjà mariée en France, et ayant été\dots» 

D'Artagnan s'arrêta comme s'il cherchait le mot, en regardant Athos. 

«Chassée par son mari, dit Athos. 

\speak  Parce qu'elle avait été marquée, continua d'Artagnan. 

\speak  Bah! s'écria Porthos, impossible! elle a voulu faire tuer son beau-frère? 

\speak  Oui. 

\speak  Elle était mariée? demanda Aramis. 

\speak  Oui. 

\speak  Et son mari s'est aperçu qu'elle avait une fleur de lis sur l'épaule? s'écria Porthos. 

\speak  Oui.» 

Ces trois oui avaient été dits par Athos, chacun avec une intonation plus sombre. 

«Et qui l'a vue, cette fleur de lis? demanda Aramis. 

\speak  D'Artagnan et moi, ou plutôt, pour observer l'ordre chronologique, moi et d'Artagnan, répondit Athos. 

\speak  Et le mari de cette affreuse créature vit encore? dit Aramis. 

\speak  Il vit encore. 

\speak  Vous en êtes sûr? 

\speak  J'en suis sûr.» 

Il y eut un instant de froid silence, pendant lequel chacun se sentit impressionné selon sa nature. 

«Cette fois, reprit Athos, interrompant le premier le silence, d'Artagnan nous a donné un excellent programme, et c'est cela qu'il faut écrire d'abord. 

\speak  Diable! vous avez raison, Athos, reprit Aramis, et la rédaction est épineuse. M. le chancelier lui-même serait embarrassé pour rédiger une épître de cette force, et cependant M. le chancelier rédige très agréablement un procès-verbal. N'importe! taisez-vous, j'écris.» 

Aramis en effet prit la plume, réfléchit quelques instants, se mit à écrire huit ou dix lignes d'une charmante petite écriture de femme, puis, d'une voix douce et lente, comme si chaque mot eût été scrupuleusement pesé, il lut ce qui suit: 
\begin{mail}{}{Milord,}
La personne qui vous écrit ces quelques lignes a eu l'honneur de croiser l'épée avec vous dans un petit enclos de la rue d'Enfer. Comme vous avez bien voulu, depuis, vous dire plusieurs fois l'ami de cette personne, elle vous doit de reconnaître cette amitié par un bon avis. Deux fois vous avez failli être victime d'une proche parente que vous croyez votre héritière, parce que vous ignorez qu'avant de contracter mariage en Angleterre, elle était déjà mariée en France. Mais, la troisième fois, qui est celle-ci, vous pouvez y succomber. Votre parente est partie de La Rochelle pour l'Angleterre pendant la nuit. Surveillez son arrivée car elle a de grands et terribles projets. Si vous tenez absolument à savoir ce dont elle est capable, lisez son passé sur son épaule gauche.
\end{mail}

«Eh bien, voilà qui est à merveille, dit Athos, et vous avez une plume de secrétaire d'état, mon cher Aramis. Lord de Winter fera bonne garde maintenant, si toutefois l'avis lui arrive; et tombât-il aux mains de Son Éminence elle-même, nous ne saurions être compromis. Mais comme le valet qui partira pourrait nous faire accroire qu'il a été à Londres et s'arrêter à Châtelleraut, ne lui donnons avec la lettre que la moitié de la somme en lui promettant l'autre moitié en échange de la réponse. Avez-vous le diamant? continua Athos. 

«J'ai mieux que cela, j'ai la somme.» 

Et d'Artagnan jeta le sac sur la table: au son de l'or, Aramis leva les yeux. Porthos tressaillit; quant à Athos, il resta impassible. 

«Combien dans ce petit sac? dit-il. 

\speak  Sept mille livres en louis de douze francs. 

\speak  Sept mille livres! s'écria Porthos, ce mauvais petit diamant valait sept mille livres? 

\speak  Il paraît, dit Athos, puisque les voilà; je ne présume pas que notre ami d'Artagnan y ait mis du sien. 

\speak  Mais, messieurs, dans tout cela, dit d'Artagnan, nous ne pensons pas à la reine. Soignons un peu la santé de son cher Buckingham. C'est le moins que nous lui devions. 

\speak  C'est juste, dit Athos, mais ceci regarde Aramis. 

\speak  Eh bien, répondit celui-ci en rougissant, que faut-il que je fasse? 

\speak  Mais, répliqua Athos, c'est tout simple: rédiger une seconde lettre pour cette adroite personne qui habite Tours.» 

Aramis reprit la plume, se mit à réfléchir de nouveau, et écrivit les lignes suivantes, qu'il soumit à l'instant même à l'approbation de ses amis: 

«Ma chère cousine\dots» 

«Ah! dit Athos, cette personne adroite est votre parente! 

\speak  Cousine germaine, dit Aramis. 

\speak  Va donc pour cousine!» 

Aramis continua: 

«Ma chère cousine, Son Éminence le cardinal, que Dieu conserve pour le bonheur de la France et la confusion des ennemis du royaume, est sur le point d'en finir avec les rebelles hérétiques de La Rochelle: il est probable que le secours de la flotte anglaise n'arrivera pas même en vue de la place; j'oserai même dire que je suis certain que M. de Buckingham sera empêché de partir par quelque grand événement. Son Éminence est le plus illustre politique des temps passés, du temps présent et probablement des temps à venir. Il éteindrait le soleil si le soleil le gênait. Donnez ces heureuses nouvelles à votre sœur, ma chère cousine. J'ai rêvé que cet Anglais maudit était mort. Je ne puis me rappeler si c'était par le fer ou par le poison; seulement ce dont je suis sûr, c'est que j'ai rêvé qu'il était mort, et, vous le savez, mes rêves ne me trompent jamais. Assurez-vous donc de me voir revenir bientôt.» 

«À merveille! s'écria Athos, vous êtes le roi des poètes; mon cher Aramis, vous parlez comme l'Apocalypse et vous êtes vrai comme l'évangile. Il ne vous reste maintenant que l'adresse à mettre sur cette lettre. 

\speak  C'est bien facile», dit Aramis. 

Il plia coquettement la lettre, la reprit et écrivit: 

«À Mademoiselle Marie Michon, lingère à Tours. 

Les trois amis se regardèrent en riant: ils étaient pris. 

«Maintenant, dit Aramis, vous comprenez, messieurs, que Bazin seul peut porter cette lettre à Tours; ma cousine ne connaît que Bazin et n'a confiance qu'en lui: tout autre ferait échouer l'affaire. D'ailleurs Bazin est ambitieux et savant; Bazin a lu l'histoire, messieurs, il sait que Sixte Quint est devenu pape après avoir gardé les pourceaux; eh bien, comme il compte se mettre d'église en même temps que moi, il ne désespère pas à son tour de devenir pape ou tout au moins cardinal: vous comprenez qu'un homme qui a de pareilles visées ne se laissera pas prendre, ou, s'il est pris, subira le martyre plutôt que de parler. 

\speak  Bien, bien, dit d'Artagnan, je vous passe de grand cœur Bazin; mais passez-moi Planchet: Milady l'a fait jeter à la porte, certain jour, avec force coups de bâton; or Planchet a bonne mémoire, et, je vous en réponds, s'il peut supposer une vengeance possible, il se fera plutôt échiner que d'y renoncer. Si vos affaires de Tours sont vos affaires, Aramis, celles de Londres sont les miennes. Je prie donc qu'on choisisse Planchet, lequel d'ailleurs a déjà été à Londres avec moi et sait dire très correctement: London, \textit{sir, if you please} et \textit{my master} lord d'Artagnan; avec cela soyez tranquilles, il fera son chemin en allant et en revenant. 

\speak  En ce cas, dit Athos, il faut que Planchet reçoive sept cents livres pour aller et sept cents livres pour revenir, et Bazin, trois cents livres pour aller et trois cents livres pour revenir; cela réduira la somme à cinq mille livres; nous prendrons mille livres chacun pour les employer comme bon nous semblera, et nous laisserons un fond de mille livres que gardera l'abbé pour les cas extraordinaires ou les besoins communs. Cela vous va-t-il? 

\speak  Mon cher Athos, dit Aramis, vous parlez comme Nestor, qui était, comme chacun sait, le plus sage des Grecs. 

\speak  Eh bien, c'est dit, reprit Athos, Planchet et Bazin partiront; à tout prendre, je ne suis pas fâché de conserver Grimaud: il est accoutumé à mes façons et j'y tiens; la journée d'hier a déjà dû l'ébranler, ce voyage le perdrait.» 

On fit venir Planchet, et on lui donna des instructions; il avait été prévenu déjà par d'Artagnan, qui, du premier coup, lui avait annoncé la gloire, ensuite l'argent, puis le danger. 

«Je porterai la lettre dans le parement de mon habit, dit Planchet, et je l'avalerai si l'on me prend. 

\speak  Mais alors tu ne pourras pas faire la commission, dit d'Artagnan. 

\speak  Vous m'en donnerez ce soir une copie que je saurai par cœur demain.» 

D'Artagnan regarda ses amis comme pour leur dire: 

«Eh bien, que vous avais-je promis?» 

«Maintenant, continua-t-il en s'adressant à Planchet, tu as huit jours pour arriver près de Lord de Winter, tu as huit autres jours pour revenir ici, en tout seize jours; si le seizième jour de ton départ, à huit heures du soir, tu n'es pas arrivé, pas d'argent, fût-il huit heures cinq minutes. 

Alors, monsieur, dit Planchet, achetez-moi une montre. 

Prends celle-ci, dit Athos, en lui donnant la sienne avec une insouciante générosité, et sois brave garçon. Songe que, si tu parles, si tu bavardes, si tu flânes, tu fais couper le cou à ton maître, qui a si grande confiance dans ta fidélité qu'il nous a répondu de toi. Mais songe aussi que s'il arrive, par ta faute, malheur à d'Artagnan, je te retrouverai partout, et ce sera pour t'ouvrir le ventre. 

\speak  Oh! monsieur! dit Planchet, humilié du soupçon et surtout effrayé de l'air calme du mousquetaire. 

\speak  Et moi, dit Porthos en roulant ses gros yeux, songe que je t'écorche vif. 

\speak  Ah! monsieur! 

\speak  Et moi, continua Aramis de sa voix douce et mélodieuse, songe que je te brûle à petit feu comme un sauvage. 

\speak  Ah! monsieur!» 

Et Planchet se mit à pleurer; nous n'oserions dire si ce fut de terreur, à cause des menaces qui lui étaient faites, ou d'attendrissement de voir quatre amis si étroitement unis. 

D'Artagnan lui prit la main, et l'embrassa. 

«Vois-tu, Planchet, lui dit-il, ces messieurs te disent tout cela par tendresse pour moi, mais au fond ils t'aiment. 

\speak  Ah! monsieur! dit Planchet, ou je réussirai, ou l'on me coupera en quatre; me coupât-on en quatre, soyez convaincu qu'il n'y a pas un morceau qui parlera.» 

Il fut décidé que Planchet partirait le lendemain à huit heures du matin, afin, comme il l'avait dit, qu'il pût, pendant la nuit, apprendre la lettre par cœur. Il gagna juste douze heures à cet arrangement; il devait être revenu le seizième jour, à huit heures du soir. 

Le matin, au moment où il allait monter à cheval, d'Artagnan, qui se sentait au fond du cœur un faible pour le duc, prit Planchet à part. 

«Écoute, lui dit-il, quand tu auras remis la lettre à Lord de Winter et qu'il l'aura lue, tu lui diras encore: “Veillez sur Sa Grâce Lord Buckingham, car on veut l'assassiner.” Mais ceci, Planchet, vois-tu, c'est si grave et si important, que je n'ai pas même voulu avouer à mes amis que je te confierais ce secret, et que pour une commission de capitaine je ne voudrais pas te l'écrire. 

\speak  Soyez tranquille, monsieur, dit Planchet, vous verrez si l'on peut compter sur moi. 

Et monté sur un excellent cheval, qu'il devait quitter à vingt lieues de là pour prendre la poste, Planchet partit au galop, le cœur un peu serré par la triple promesse que lui avaient faite les mousquetaires, mais du reste dans les meilleures dispositions du monde. 

Bazin partit le lendemain matin pour Tours, et eut huit jours pour faire sa commission. 

Les quatre amis, pendant toute la durée de ces deux absences, avaient, comme on le comprend bien, plus que jamais l'œil au guet, le nez au vent et l'oreille aux écoutes. Leurs journées se passaient à essayer de surprendre ce qu'on disait, à guetter les allures du cardinal et à flairer les courriers qui arrivaient. Plus d'une fois un tremblement insurmontable les prit, lorsqu'on les appela pour quelque service inattendu. Ils avaient d'ailleurs à se garder pour leur propre sûreté; Milady était un fantôme qui, lorsqu'il était apparu une fois aux gens, ne les laissait pas dormir tranquillement. 

Le matin du huitième jour, Bazin, frais comme toujours et souriant selon son habitude, entra dans le cabaret de Parpaillot, comme les quatre amis étaient en train de déjeuner, en disant, selon la convention arrêtée: 

«Monsieur Aramis, voici la réponse de votre cousine.» 

Les quatre amis échangèrent un coup d'œil joyeux: la moitié de la besogne était faite; il est vrai que c'était la plus courte et la plus facile. 

Aramis prit, en rougissant malgré lui, la lettre, qui était d'une écriture grossière et sans orthographe. 

«Bon Dieu! s'écria-t-il en riant, décidément j'en désespère; jamais cette pauvre Michon n'écrira comme M. de Voiture. 

\speak  Qu'est-ce que cela feut dire, cette baufre Migeon? demanda le Suisse, qui était en train de causer avec les quatre amis quand la lettre était arrivée. 

\speak  Oh! mon Dieu! moins que rien, dit Aramis, une petite lingère charmante que j'aimais fort et à qui j'ai demandé quelques lignes de sa main en manière de souvenir. 

\speak  Dutieu! dit le Suisse; zi zella il être auzi grante tame que son l'égridure, fous l'être en ponne fordune, mon gamarate! 

Aramis lut la lettre et la passa à Athos. 

«Voyez donc ce qu'elle m'écrit, Athos», dit-il. 

Athos jeta un coup d'œil sur l'épître, et, pour faire évanouir tous les soupçons qui auraient pu naître, lut tout haut: 

\begin{mail}{}{Mon cousin,} 
	Ma sœur et moi devinons très bien les rêves, et nous en avons même une peur affreuse; mais du vôtre, on pourra dire, je l'espère, tout songe est mensonge. Adieu! portez-vous bien, et faites que de temps en temps nous entendions parler de vous. 
	
	\closeletter{Aglaé Michon.}
\end{mail}

«Et de quel rêve parle-t-elle? demanda le dragon, qui s'était approché pendant la lecture. 

\speak  Foui, te quel rêfe? dit le Suisse. 

\speak  Eh! pardieu! dit Aramis, c'est tout simple, d'un rêve que j'ai fait et que je lui ai raconté. 

\speak  Oh! foui, par Tieu! c'être tout simple de ragonter son rêfe; mais moi je ne rêfe jamais. 

\speak  Vous êtes fort heureux, dit Athos en se levant, et je voudrais bien pouvoir en dire autant que vous! 

\speak  Chamais! reprit le Suisse, enchanté qu'un homme comme Athos lui enviât quelque chose, chamais! chamais!» 

D'Artagnan, voyant qu'Athos se levait, en fit autant, prit son bras, et sortit. 

Porthos et Aramis restèrent pour faire face aux quolibets du dragon et du Suisse. 

Quant à Bazin, il s'alla coucher sur une botte de paille; et comme il avait plus d'imagination que le Suisse, il rêva que M. Aramis, devenu pape, le coiffait d'un chapeau de cardinal. 

Mais, comme nous l'avons dit, Bazin n'avait, par son heureux retour, enlevé qu'une partie de l'inquiétude qui aiguillonnait les quatre amis. Les jours de l'attente sont longs, et d'Artagnan surtout aurait parié que les jours avaient maintenant quarante-huit heures. Il oubliait les lenteurs obligées de la navigation, il s'exagérait la puissance de Milady. Il prêtait à cette femme, qui lui apparaissait pareille à un démon, des auxiliaires surnaturels comme elle; il s'imaginait, au moindre bruit, qu'on venait l'arrêter, et qu'on ramenait Planchet pour le confronter avec lui et ses amis. Il y a plus: sa confiance autrefois si grande dans le digne Picard, diminuait de jour en jour. Cette inquiétude était si grande, qu'elle gagnait Porthos et Aramis. Il n'y avait qu'Athos qui demeurât impassible, comme si aucun danger ne s'agitait autour de lui, et qu'il respirât son atmosphère quotidienne. 

Le seizième jour surtout, ces signes d'agitation étaient si visibles chez d'Artagnan et ses deux amis, qu'ils ne pouvaient rester en place, et qu'ils erraient comme des ombres sur le chemin par lequel devait revenir Planchet. 

«Vraiment, leur disait Athos, vous n'êtes pas des hommes, mais des enfants, pour qu'une femme vous fasse si grand-peur! Et de quoi s'agit-il, après tout? D'être emprisonnés! Eh bien, mais on nous tirera de prison: on en a bien retiré Mme Bonacieux. D'être décapités? Mais tous les jours, dans la tranchée, nous allons joyeusement nous exposer à pis que cela, car un boulet peut nous casser la jambe, et je suis convaincu qu'un chirurgien nous fait plus souffrir en nous coupant la cuisse qu'un bourreau en nous coupant la tête. Demeurez donc tranquilles; dans deux heures, dans quatre, dans six heures, au plus tard, Planchet sera ici: il a promis d'y être, et moi j'ai très grande foi aux promesses de Planchet, qui m'a l'air d'un fort brave garçon. 

\speak  Mais s'il n'arrive pas? dit d'Artagnan. 

\speak  Eh bien, s'il n'arrive pas, c'est qu'il aura été retardé, voilà tout. Il peut être tombé de cheval, il peut avoir fait une cabriole par-dessus le pont, il peut avoir couru si vite qu'il en ait attrapé une fluxion de poitrine. Eh! messieurs! faisons donc la part des événements. La vie est un chapelet de petites misères que le philosophe égrène en riant. Soyez philosophes comme moi, messieurs, mettez-vous à table et buvons; rien ne fait paraître l'avenir couleur de rose comme de le regarder à travers un verre de chambertin. 

\speak  C'est fort bien, répondit d'Artagnan; mais je suis las d'avoir à craindre, en buvant frais, que le vin ne sorte de la cave de Milady. 

\speak  Vous êtes bien difficile, dit Athos, une si belle femme! 

\speak  Une femme de marque!» dit Porthos avec son gros rire. 

Athos tressaillit, passa la main sur son front pour en essuyer la sueur, et se leva à son tour avec un mouvement nerveux qu'il ne put réprimer. 

Le jour s'écoula cependant, et le soir vint plus lentement, mais enfin il vint; les buvettes s'emplirent de chalands; Athos, qui avait empoché sa part du diamant, ne quittait plus le Parpaillot. Il avait trouvé dans M. de Busigny, qui, au reste, leur avait donné un dîner magnifique, un \textit{partner} digne de lui. Ils jouaient donc ensemble, comme d'habitude, quand sept heures sonnèrent: on entendit passer les patrouilles qui allaient doubler les postes; à sept heures et demie la retraite sonna. 

«Nous sommes perdus, dit d'Artagnan à l'oreille d'Athos. 

\speak  Vous voulez dire que nous avons perdu, dit tranquillement Athos en tirant quatre pistoles de sa poche et en les jetant sur la table. Allons, messieurs, continua-t-il, on bat la retraite, allons nous coucher.» 

Et Athos sortit du Parpaillot suivi de d'Artagnan. Aramis venait derrière donnant le bras à Porthos. Aramis mâchonnait des vers, et Porthos s'arrachait de temps en temps quelques poils de moustache en signe de désespoir. 

Mais voilà que tout à coup, dans l'obscurité, une ombre se dessine, dont la forme est familière à d'Artagnan, et qu'une voix bien connue lui dit: 

«Monsieur, je vous apporte votre manteau, car il fait frais ce soir. 

\speak  Planchet! s'écria d'Artagnan, ivre de joie. 

\speak  Planchet! répétèrent Porthos et Aramis. 

\speak  Eh bien, oui, Planchet, dit Athos, qu'y a-t-il d'étonnant à cela? Il avait promis d'être de retour à huit heures, et voilà les huit heures qui sonnent. Bravo! Planchet, vous êtes un garçon de parole, et si jamais vous quittez votre maître, je vous garde une place à mon service. 

\speak  Oh! non, jamais, dit Planchet, jamais je ne quitterai M. d'Artagnan.» 

En même temps d'Artagnan sentit que Planchet lui glissait un billet dans la main. 

D'Artagnan avait grande envie d'embrasser Planchet au retour comme il l'avait embrassé au départ; mais il eut peur que cette marque d'effusion, donnée à son laquais en pleine rue, ne parût extraordinaire à quelque passant, et il se contint. 

«J'ai le billet, dit-il à Athos et à ses amis. 

\speak  C'est bien, dit Athos, entrons chez nous, et nous le lirons. 

Le billet brûlait la main de d'Artagnan: il voulait hâter le pas; mais Athos lui prit le bras et le passa sous le sien, et force fut au jeune homme de régler sa course sur celle de son ami. 

Enfin on entra dans la tente, on alluma une lampe, et tandis que Planchet se tenait sur la porte pour que les quatre amis ne fussent pas surpris, d'Artagnan, d'une main tremblante, brisa le cachet et ouvrit la lettre tant attendue. 

Elle contenait une demi-ligne, d'une écriture toute britannique et d'une concision toute spartiate: 

«\textit{Thank you, be easy.}» 

Ce qui voulait dire: 

«Merci, soyez tranquille.» 

Athos prit la lettre des mains de d'Artagnan, l'approcha de la lampe, y mit le feu, et ne la lâcha point qu'elle ne fût réduite en cendres. 

Puis appelant Planchet: 

«Maintenant, mon garçon, lui dit-il, tu peux réclamer tes sept cents livres, mais tu ne risquais pas grand-chose avec un billet comme celui-là. 

\speak  Ce n'est pas faute que j'aie inventé bien des moyens de le serrer, dit Planchet. 

\speak  Eh bien, dit d'Artagnan, conte-nous cela. 

\speak  Dame! c'est bien long, monsieur. 

\speak  Tu as raison, Planchet, dit Athos; d'ailleurs la retraite est battue, et nous serions remarqués en gardant de la lumière plus longtemps que les autres. 

\speak  Soit, dit d'Artagnan, couchons-nous. Dors bien, Planchet! 

\speak  Ma foi, monsieur! ce sera la première fois depuis seize jours. 

\speak  Et moi aussi! dit d'Artagnan. 

\speak  Et moi aussi! répéta Porthos. 

\speak  Et moi aussi! répéta Aramis. 

\speak  Eh bien, voulez-vous que je vous avoue la vérité? et moi aussi!» dit Athos.