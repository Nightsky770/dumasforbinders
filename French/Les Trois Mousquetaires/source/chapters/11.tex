%!TeX root=../musketeersfr.tex 

\chapter{L'Intrigue Se Noue}

\lettrine{S}{a} visite faite à M. de Tréville, d'Artagnan prit, tout pensif, le plus long pour rentrer chez lui. 

\zz
À quoi pensait d'Artagnan, qu'il s'écartait ainsi de sa route, regardant les étoiles du ciel, et tantôt soupirant tantôt souriant? 

Il pensait à Mme Bonacieux. Pour un apprenti mousquetaire, la jeune femme était presque une idéalité amoureuse. Jolie, mystérieuse, initiée à presque tous les secrets de cour, qui reflétaient tant de charmante gravité sur ses traits gracieux, elle était soupçonnée de n'être pas insensible, ce qui est un attrait irrésistible pour les amants novices; de plus, d'Artagnan l'avait délivrée des mains de ces démons qui voulaient la fouiller et la maltraiter, et cet important service avait établi entre elle et lui un de ces sentiments de reconnaissance qui prennent si facilement un plus tendre caractère. 

D'Artagnan se voyait déjà, tant les rêves marchent vite sur les ailes de l'imagination, accosté par un messager de la jeune femme qui lui remettait quelque billet de rendez-vous, une chaîne d'or ou un diamant. Nous avons dit que les jeunes cavaliers recevaient sans honte de leur roi; ajoutons qu'en ce temps de facile morale, ils n'avaient pas plus de vergogne à l'endroit de leurs maîtresses, et que celles-ci leur laissaient presque toujours de précieux et durables souvenirs, comme si elles eussent essayé de conquérir la fragilité de leurs sentiments par la solidité de leurs dons. 

On faisait alors son chemin par les femmes, sans en rougir. Celles qui n'étaient que belles donnaient leur beauté, et de là vient sans doute le proverbe, que la plus belle fille du monde ne peut donner que ce qu'elle a. Celles qui étaient riches donnaient en outre une partie de leur argent, et l'on pourrait citer bon nombre de héros de cette galante époque qui n'eussent gagné ni leurs éperons d'abord, ni leurs batailles ensuite, sans la bourse plus ou moins garnie que leur maîtresse attachait à l'arçon de leur selle. 

D'Artagnan ne possédait rien; l'hésitation du provincial, vernis léger, fleur éphémère, duvet de la pêche, s'était évaporée au vent des conseils peu orthodoxes que les trois mousquetaires donnaient à leur ami. D'Artagnan, suivant l'étrange coutume du temps, se regardait à Paris comme en campagne, et cela ni plus ni moins que dans les Flandres: l'Espagnol là-bas, la femme ici. C'était partout un ennemi à combattre, des contributions à frapper. 

Mais, disons-le, pour le moment d'Artagnan était mû d'un sentiment plus noble et plus désintéressé. Le mercier lui avait dit qu'il était riche; le jeune homme avait pu deviner qu'avec un niais comme l'était M. Bonacieux, ce devait être la femme qui tenait la clef de la bourse. Mais tout cela n'avait influé en rien sur le sentiment produit par la vue de Mme Bonacieux, et l'intérêt était resté à peu près étranger à ce commencement d'amour qui en avait été la suite. Nous disons: à peu près, car l'idée qu'une jeune femme, belle, gracieuse, spirituelle, est riche en même temps, n'ôte rien à ce commencement d'amour, et tout au contraire le corrobore. 

Il y a dans l'aisance une foule de soins et de caprices aristocratiques qui vont bien à la beauté. Un bas fin et blanc, une robe de soie, une guimpe de dentelle, un joli soulier au pied, un frais ruban sur la tête, ne font point jolie une femme laide, mais font belle une femme jolie, sans compter les mains qui gagnent à tout cela; les mains, chez les femmes surtout, ont besoin de rester oisives pour rester belles. 

Puis d'Artagnan, comme le sait bien le lecteur, auquel nous n'avons pas caché l'état de sa fortune, d'Artagnan n'était pas un millionnaire; il espérait bien le devenir un jour, mais le temps qu'il se fixait lui-même pour cet heureux changement était assez éloigné. En attendant, quel désespoir que de voir une femme qu'on aime désirer ces mille riens dont les femmes composent leur bonheur, et de ne pouvoir lui donner ces mille riens! Au moins, quand la femme est riche et que l'amant ne l'est pas, ce qu'il ne peut lui offrir elle se l'offre elle-même; et quoique ce soit ordinairement avec l'argent du mari qu'elle se passe cette jouissance, il est rare que ce soit à lui qu'en revienne la reconnaissance. 

Puis d'Artagnan, disposé à être l'amant le plus tendre, était en attendant un ami très dévoué. Au milieu de ses projets amoureux sur la femme du mercier, il n'oubliait pas les siens. La jolie Mme Bonacieux était femme à promener dans la plaine Saint-Denis ou dans la foire Saint-Germain en compagnie d'Athos, de Porthos et d'Aramis, auxquels d'Artagnan serait fier de montrer une telle conquête. Puis, quand on a marché longtemps, la faim arrive; d'Artagnan depuis quelque temps avait remarqué cela. On ferait de ces petits dîners charmants où l'on touche d'un côté la main d'un ami, et de l'autre le pied d'une maîtresse. Enfin, dans les moments pressants, dans les positions extrêmes, d'Artagnan serait le sauveur de ses amis. 

Et M. Bonacieux, que d'Artagnan avait poussé dans les mains des sbires en le reniant bien haut et à qui il avait promis tout bas de le sauver? Nous devons avouer à nos lecteurs que d'Artagnan n'y songeait en aucune façon, ou que, s'il y songeait, c'était pour se dire qu'il était bien où il était, quelque part qu'il fût. L'amour est la plus égoïste de toutes les passions. 

Cependant, que nos lecteurs se rassurent: si d'Artagnan oublie son hôte ou fait semblant de l'oublier, sous prétexte qu'il ne sait pas où on l'a conduit, nous ne l'oublions pas, nous, et nous savons où il est. Mais pour le moment faisons comme le Gascon amoureux. Quant au digne mercier, nous reviendrons à lui plus tard. 

D'Artagnan, tout en réfléchissant à ses futures amours, tout en parlant à la nuit, tout en souriant aux étoiles, remontait la rue du Cherche-Midi ou Chasse-Midi, ainsi qu'on l'appelait alors. Comme il se trouvait dans le quartier d'Aramis, l'idée lui était venue d'aller faire une visite à son ami, pour lui donner quelques explications sur les motifs qui lui avaient fait envoyer Planchet avec invitation de se rendre immédiatement à la souricière. Or, si Aramis s'était trouvé chez lui lorsque Planchet y était venu, il avait sans aucun doute couru rue des Fossoyeurs, et n'y trouvant personne que ses deux autres compagnons peut-être, ils n'avaient dû savoir, ni les uns ni les autres, ce que cela voulait dire. Ce dérangement méritait donc une explication, voilà ce que disait tout haut d'Artagnan. 

Puis, tout bas, il pensait que c'était pour lui une occasion de parler de la jolie petite Mme Bonacieux, dont son esprit, sinon son cœur, était déjà tout plein. Ce n'est pas à propos d'un premier amour qu'il faut demander de la discrétion. Ce premier amour est accompagné d'une si grande joie, qu'il faut que cette joie déborde, sans cela elle vous étoufferait. 

Paris depuis deux heures était sombre et commençait à se faire désert. Onze heures sonnaient à toutes les horloges du faubourg Saint-Germain, il faisait un temps doux. D'Artagnan suivait une ruelle située sur l'emplacement où passe aujourd'hui la rue d'Assas, respirant les émanations embaumées qui venaient avec le vent de la rue de Vaugirard et qu'envoyaient les jardins rafraîchis par la rosée du soir et par la brise de la nuit. Au loin résonnaient, assourdis cependant par de bons volets, les chants des buveurs dans quelques cabarets perdus dans la plaine. Arrivé au bout de la ruelle, d'Artagnan tourna à gauche. La maison qu'habitait Aramis se trouvait située entre la rue Cassette et la rue Servandoni. 

D'Artagnan venait de dépasser la rue Cassette et reconnaissait déjà la porte de la maison de son ami, enfouie sous un massif de sycomores et de clématites qui formaient un vaste bourrelet au-dessus d'elle lorsqu'il aperçut quelque chose comme une ombre qui sortait de la rue Servandoni. Ce quelque chose était enveloppé d'un manteau, et d'Artagnan crut d'abord que c'était un homme; mais, à la petitesse de la taille, à l'incertitude de la démarche, à l'embarras du pas, il reconnut bientôt une femme. De plus, cette femme, comme si elle n'eût pas été bien sûre de la maison qu'elle cherchait, levait les yeux pour se reconnaître, s'arrêtait, retournait en arrière, puis revenait encore. D'Artagnan fut intrigué. 

«Si j'allais lui offrir mes services! pensa-t-il. À son allure, on voit qu'elle est jeune; peut-être jolie. Oh! oui. Mais une femme qui court les rues à cette heure ne sort guère que pour aller rejoindre son amant. Peste! si j'allais troubler les rendez-vous, ce serait une mauvaise porte pour entrer en relations.» 

Cependant, la jeune femme s'avançait toujours, comptant les maisons et les fenêtres. Ce n'était, au reste, chose ni longue, ni difficile. Il n'y avait que trois hôtels dans cette partie de la rue, et deux fenêtres ayant vue sur cette rue; l'une était celle d'un pavillon parallèle à celui qu'occupait Aramis, l'autre était celle d'Aramis lui-même. 

«Pardieu! se dit d'Artagnan, auquel la nièce du théologien revenait à l'esprit; pardieu! il serait drôle que cette colombe attardée cherchât la maison de notre ami. Mais sur mon âme, cela y ressemble fort. Ah! mon cher Aramis, pour cette fois, j'en veux avoir le cœur net.» 

Et d'Artagnan, se faisant le plus mince qu'il put, s'abrita dans le côté le plus obscur de la rue, près d'un banc de pierre situé au fond d'une niche. 

La jeune femme continua de s'avancer, car outre la légèreté de son allure, qui l'avait trahie, elle venait de faire entendre une petite toux qui dénonçait une voix des plus fraîches. D'Artagnan pensa que cette toux était un signal. 

Cependant, soit qu'on eût répondu à cette toux par un signe équivalent qui avait fixé les irrésolutions de la nocturne chercheuse, soit que sans secours étranger elle eût reconnu qu'elle était arrivée au bout de sa course, elle s'approcha résolument du volet d'Aramis et frappa à trois intervalles égaux avec son doigt recourbé. 

«C'est bien chez Aramis, murmura d'Artagnan. Ah! monsieur l'hypocrite! je vous y prends à faire de la théologie!» 

Les trois coups étaient à peine frappés, que la croisée intérieure s'ouvrit et qu'une lumière parut à travers les vitres du volet. 

«Ah! ah! fit l'écouteur non pas aux portes, mais aux fenêtres, ah! la visite était attendue. Allons, le volet va s'ouvrir et la dame entrera par escalade. Très bien!» 

Mais, au grand étonnement de d'Artagnan, le volet resta fermé. De plus, la lumière qui avait flamboyé un instant, disparut, et tout rentra dans l'obscurité. 

D'Artagnan pensa que cela ne pouvait durer ainsi, et continua de regarder de tous ses yeux et d'écouter de toutes ses oreilles. 

Il avait raison: au bout de quelques secondes, deux coups secs retentirent dans l'intérieur. 

La jeune femme de la rue répondit par un seul coup, et le volet s'entrouvrit. 

On juge si d'Artagnan regardait et écoutait avec avidité. 

Malheureusement, la lumière avait été transportée dans un autre appartement. Mais les yeux du jeune homme s'étaient habitués à la nuit. D'ailleurs les yeux des Gascons ont, à ce qu'on assure, comme ceux des chats, la propriété de voir pendant la nuit. 

D'Artagnan vit donc que la jeune femme tirait de sa poche un objet blanc qu'elle déploya vivement et qui prit la forme d'un mouchoir. Cet objet déployé, elle en fit remarquer le coin à son interlocuteur. 

Cela rappela à d'Artagnan ce mouchoir qu'il avait trouvé aux pieds de Mme Bonacieux, lequel lui avait rappelé celui qu'il avait trouvé aux pieds d'Aramis. 

«Que diable pouvait donc signifier ce mouchoir?» 

Placé où il était, d'Artagnan ne pouvait voir le visage d'Aramis, nous disons d'Aramis, parce que le jeune homme ne faisait aucun doute que ce fût son ami qui dialoguât de l'intérieur avec la dame de l'extérieur; la curiosité l'emporta donc sur la prudence, et, profitant de la préoccupation dans laquelle la vue du mouchoir paraissait plonger les deux personnages que nous avons mis en scène, il sortit de sa cachette, et prompt comme l'éclair, mais étouffant le bruit de ses pas, il alla se coller à un angle de la muraille, d'où son œil pouvait parfaitement plonger dans l'intérieur de l'appartement d'Aramis. 

Arrivé là, d'Artagnan pensa jeter un cri de surprise: ce n'était pas Aramis qui causait avec la nocturne visiteuse, c'était une femme. Seulement, d'Artagnan y voyait assez pour reconnaître la forme de ses vêtements, mais pas assez pour distinguer ses traits. 

Au même instant, la femme de l'appartement tira un second mouchoir de sa poche, et l'échangea avec celui qu'on venait de lui montrer. Puis, quelques mots furent prononcés entre les deux femmes. Enfin le volet se referma; la femme qui se trouvait à l'extérieur de la fenêtre se retourna, et vint passer à quatre pas de d'Artagnan en abaissant la coiffe de sa mante; mais la précaution avait été prise trop tard, d'Artagnan avait déjà reconnu Mme Bonacieux. 

Mme Bonacieux! Le soupçon que c'était elle lui avait déjà traversé l'esprit quand elle avait tiré le mouchoir de sa poche; mais quelle probabilité que Mme Bonacieux qui avait envoyé chercher M. de La Porte pour se faire reconduire par lui au Louvre, courût les rues de Paris seule à onze heures et demie du soir, au risque de se faire enlever une seconde fois? 

Il fallait donc que ce fût pour une affaire bien importante; et quelle est l'affaire importante d'une femme de vingt-cinq ans? L'amour. 

Mais était-ce pour son compte ou pour le compte d'une autre personne qu'elle s'exposait à de semblables hasards? Voilà ce que se demandait à lui-même le jeune homme, que le démon de la jalousie mordait au cœur ni plus ni moins qu'un amant en titre. 

Il y avait, au reste, un moyen bien simple de s'assurer où allait Mme Bonacieux: c'était de la suivre. Ce moyen était si simple, que d'Artagnan l'employa tout naturellement et d'instinct. 

Mais, à la vue du jeune homme qui se détachait de la muraille comme une statue de sa niche, et au bruit des pas qu'elle entendit retentir derrière elle, Mme Bonacieux jeta un petit cri et s'enfuit. 

D'Artagnan courut après elle. Ce n'était pas une chose difficile pour lui que de rejoindre une femme embarrassée dans son manteau. Il la rejoignit donc au tiers de la rue dans laquelle elle s'était engagée. La malheureuse était épuisée, non pas de fatigue, mais de terreur, et quand d'Artagnan lui posa la main sur l'épaule, elle tomba sur un genou en criant d'une voix étranglée: 

«Tuez-moi si vous voulez, mais vous ne saurez rien.» 

D'Artagnan la releva en lui passant le bras autour de la taille; mais comme il sentait à son poids qu'elle était sur le point de se trouver mal, il s'empressa de la rassurer par des protestations de dévouement. Ces protestations n'étaient rien pour Mme Bonacieux; car de pareilles protestations peuvent se faire avec les plus mauvaises intentions du monde; mais la voix était tout. La jeune femme crut reconnaître le son de cette voix: elle rouvrit les yeux, jeta un regard sur l'homme qui lui avait fait si grand-peur, et, reconnaissant d'Artagnan, elle poussa un cri de joie. 

«Oh! c'est vous, c'est vous! dit-elle; merci, mon Dieu! 

\speak  Oui, c'est moi, dit d'Artagnan, moi que Dieu a envoyé pour veiller sur vous. 

\speak  Était-ce dans cette intention que vous me suiviez?» demanda avec un sourire plein de coquetterie la jeune femme, dont le caractère un peu railleur reprenait le dessus, et chez laquelle toute crainte avait disparu du moment où elle avait reconnu un ami dans celui qu'elle avait pris pour un ennemi. 

«Non, dit d'Artagnan, non, je l'avoue; c'est le hasard qui m'a mis sur votre route; j'ai vu une femme frapper à la fenêtre d'un de mes amis\dots 

\speak  D'un de vos amis? interrompit Mme Bonacieux. 

\speak  Sans doute; Aramis est de mes meilleurs amis. 

\speak  Aramis! qu'est-ce que cela? 

\speak  Allons donc! allez-vous me dire que vous ne connaissez pas Aramis? 

\speak  C'est la première fois que j'entends prononcer ce nom. 

\speak  C'est donc la première fois que vous venez à cette maison? 

\speak  Sans doute. 

\speak  Et vous ne saviez pas qu'elle fût habitée par un jeune homme? 

\speak  Non. 

\speak  Par un mousquetaire? 

\speak  Nullement. 

\speak  Ce n'est donc pas lui que vous veniez chercher? 

\speak  Pas le moins du monde. D'ailleurs, vous l'avez bien vu, la personne à qui j'ai parlé est une femme. 

\speak  C'est vrai; mais cette femme est des amies d'Aramis. 

\speak  Je n'en sais rien. 

\speak  Puisqu'elle loge chez lui. 

\speak  Cela ne me regarde pas. 

\speak  Mais qui est-elle? 

\speak  Oh! cela n'est point mon secret. 

\speak  Chère madame Bonacieux, vous êtes charmante; mais en même temps vous êtes la femme la plus mystérieuse\dots 

\speak  Est-ce que je perds à cela? 

\speak  Non; vous êtes, au contraire, adorable. 

\speak  Alors, donnez-moi le bras. 

\speak  Bien volontiers. Et maintenant? 

\speak  Maintenant, conduisez-moi. 

\speak  Où cela? 

\speak  Où je vais. 

\speak  Mais où allez-vous? 

\speak  Vous le verrez, puisque vous me laisserez à la porte. 

\speak  Faudra-t-il vous attendre? 

\speak  Ce sera inutile. 

\speak  Vous reviendrez donc seule? 

\speak  Peut-être oui, peut-être non. 

\speak  Mais la personne qui vous accompagnera ensuite sera-t-elle un homme, sera-t-elle une femme? 

\speak  Je n'en sais rien encore. 

\speak  Je le saurai bien, moi! 

\speak  Comment cela? 

\speak  Je vous attendrai pour vous voir sortir. 

\speak  En ce cas, adieu! 

\speak  Comment cela? 

\speak  Je n'ai pas besoin de vous. 

\speak  Mais vous aviez réclamé\dots 

\speak  L'aide d'un gentilhomme, et non la surveillance d'un espion. 

\speak  Le mot est un peu dur! 

\speak  Comment appelle-t-on ceux qui suivent les gens malgré eux? 

\speak  Des indiscrets. 

\speak  Le mot est trop doux. 

\speak  Allons, madame, je vois bien qu'il faut faire tout ce que vous voulez. 

\speak  Pourquoi vous être privé du mérite de le faire tout de suite? 

\speak  N'y en a-t-il donc aucun à se repentir? 

\speak  Et vous repentez-vous réellement? 

\speak  Je n'en sais rien moi-même. Mais ce que je sais, c'est que je vous promets de faire tout ce que vous voudrez si vous me laissez vous accompagner jusqu'où vous allez. 

\speak  Et vous me quitterez après? 

\speak  Oui. 

\speak  Sans m'épier à ma sortie? 

\speak  Non. 

\speak  Parole d'honneur? 

\speak  Foi de gentilhomme! 

\speak  Prenez mon bras et marchons alors.» 

D'Artagnan offrit son bras à Mme Bonacieux, qui s'y suspendit, moitié rieuse, moitié tremblante, et tous deux gagnèrent le haut de la rue de La Harpe. Arrivée là, la jeune femme parut hésiter, comme elle avait déjà fait dans la rue de Vaugirard. Cependant, à de certains signes, elle sembla reconnaître une porte; et s'approchant de cette porte: 

«Et maintenant, monsieur, dit-elle, c'est ici que j'ai affaire; mille fois merci de votre honorable compagnie, qui m'a sauvée de tous les dangers auxquels, seule, j'eusse été exposée. Mais le moment est venu de tenir votre parole: je suis arrivée à ma destination. 

\speak  Et vous n'aurez plus rien à craindre en revenant? 

\speak  Je n'aurai à craindre que les voleurs. 

\speak  N'est-ce donc rien? 

\speak  Que pourraient-ils me prendre? je n'ai pas un denier sur moi. 

\speak  Vous oubliez ce beau mouchoir brodé, armorié. 

\speak  Lequel? 

\speak  Celui que j'ai trouvé à vos pieds et que j'ai remis dans votre poche. 

\speak  Taisez-vous, taisez-vous, malheureux! s'écria la jeune femme, voulez-vous me perdre? 

\speak  Vous voyez bien qu'il y a encore du danger pour vous, puisqu'un seul mot vous fait trembler, et que vous avouez que, si on entendait ce mot, vous seriez perdue. Ah! tenez, madame, s'écria d'Artagnan en lui saisissant la main et la couvrant d'un ardent regard, tenez! soyez plus généreuse, confiez-vous à moi; n'avez-vous donc pas lu dans mes yeux qu'il n'y a que dévouement et sympathie dans mon cœur? 

\speak  Si fait, répondit Mme Bonacieux; aussi demandez-moi mes secrets, et je vous les dirai; mais ceux des autres, c'est autre chose. 

\speak  C'est bien, dit d'Artagnan, je les découvrirai; puisque ces secrets peuvent avoir une influence sur votre vie, il faut que ces secrets deviennent les miens. 

\speak  Gardez-vous-en bien, s'écria la jeune femme avec un sérieux qui fit frissonner d'Artagnan malgré lui. Oh! ne vous mêlez en rien de ce qui me regarde, ne cherchez point à m'aider dans ce que j'accomplis; et cela, je vous le demande au nom de l'intérêt que je vous inspire, au nom du service que vous m'avez rendu! et que je n'oublierai de ma vie. Croyez bien plutôt à ce que je vous dis. Ne vous occupez plus de moi, je n'existe plus pour vous, que ce soit comme si vous ne m'aviez jamais vue. 

\speak  Aramis doit-il en faire autant que moi, madame? dit d'Artagnan piqué. 

\speak  Voilà deux ou trois fois que vous avez prononcé ce nom, monsieur, et cependant je vous ai dit que je ne le connaissais pas. 

\speak  Vous ne connaissez pas l'homme au volet duquel vous avez été frapper. Allons donc, madame! vous me croyez par trop crédule, aussi! 

\speak  Avouez que c'est pour me faire parler que vous inventez cette histoire, et que vous créez ce personnage. 

\speak  Je n'invente rien, madame, je ne crée rien, je dis l'exacte vérité. 

\speak  Et vous dites qu'un de vos amis demeure dans cette maison? 

\speak  Je le dis et je le répète pour la troisième fois, cette maison est celle qu'habite mon ami, et cet ami est Aramis. 

\speak  Tout cela s'éclaircira plus tard, murmura la jeune femme: maintenant, monsieur, taisez-vous. 

\speak  Si vous pouviez voir mon cœur tout à découvert, dit d'Artagnan, vous y liriez tant de curiosité, que vous auriez pitié de moi, et tant d'amour, que vous satisferiez à l'instant même ma curiosité. On n'a rien à craindre de ceux qui vous aiment. 

\speak  Vous parlez bien vite d'amour, monsieur! dit la jeune femme en secouant la tête. 

\speak  C'est que l'amour m'est venu vite et pour la première fois, et que je n'ai pas vingt ans.» 

La jeune femme le regarda à la dérobée. 

«Écoutez, je suis déjà sur la trace, dit d'Artagnan. Il y a trois mois, j'ai manqué avoir un duel avec Aramis pour un mouchoir pareil à celui que vous avez montré à cette femme qui était chez lui, pour un mouchoir marqué de la même manière, j'en suis sûr. 

\speak  Monsieur, dit la jeune femme, vous me fatiguez fort, je vous le jure, avec ces questions. 

\speak  Mais vous, si prudente, madame, songez-y, si vous étiez arrêtée avec ce mouchoir, et que ce mouchoir fût saisi, ne seriez-vous pas compromise? 

\speak  Pourquoi cela, les initiales ne sont-elles pas les miennes: C.B., Constance Bonacieux? 

\speak  Ou Camille de Bois-Tracy. 

\speak  Silence, monsieur, encore une fois silence! Ah! puisque les dangers que je cours pour moi-même ne vous arrêtent pas, songez à ceux que vous pouvez courir, vous! 

\speak  Moi? 

\speak  Oui, vous. Il y a danger de la prison, il y a danger de la vie à me connaître. 

\speak  Alors, je ne vous quitte plus. 

\speak  Monsieur, dit la jeune femme suppliant et joignant les mains, monsieur, au nom du Ciel, au nom de l'honneur d'un militaire, au nom de la courtoisie d'un gentilhomme, éloignez-vous; tenez, voilà minuit qui sonne, c'est l'heure où l'on m'attend. 

\speak  Madame, dit le jeune homme en s'inclinant, je ne sais rien refuser à qui me demande ainsi; soyez contente, je m'éloigne. 

\speak  Mais vous ne me suivrez pas, vous ne m'épierez pas? 

\speak  Je rentre chez moi à l'instant. 

\speak  Ah! je le savais bien, que vous étiez un brave jeune homme!» s'écria Mme Bonacieux en lui tendant une main et en posant l'autre sur le marteau d'une petite porte presque perdue dans la muraille. 

D'Artagnan saisit la main qu'on lui tendait et la baisa ardemment. 

«Ah! j'aimerais mieux ne vous avoir jamais vue, s'écria d'Artagnan avec cette brutalité naïve que les femmes préfèrent souvent aux afféteries de la politesse, parce qu'elle découvre le fond de la pensée et qu'elle prouve que le sentiment l'emporte sur la raison. 

\speak  Eh bien, reprit Mme Bonacieux d'une voix presque caressante, et en serrant la main de d'Artagnan qui n'avait pas abandonné la sienne; eh bien, je n'en dirai pas autant que vous: ce qui est perdu pour aujourd'hui n'est pas perdu pour l'avenir. Qui sait, si lorsque je serai déliée un jour, je ne satisferai pas votre curiosité? 

\speak  Et faites-vous la même promesse à mon amour? s'écria d'Artagnan au comble de la joie. 

\speak  Oh! de ce côté, je ne veux point m'engager, cela dépendra des sentiments que vous saurez m'inspirer. 

\speak  Ainsi, aujourd'hui, madame\dots 

\speak  Aujourd'hui, monsieur, je n'en suis encore qu'à la reconnaissance. 

\speak  Ah! vous êtes trop charmante, dit d'Artagnan avec tristesse, et vous abusez de mon amour. 

\speak  Non, j'use de votre générosité, voilà tout. Mais croyez-le bien, avec certaines gens tout se retrouve. 

\speak  Oh! vous me rendez le plus heureux des hommes. N'oubliez pas cette soirée, n'oubliez pas cette promesse. 

\speak  Soyez tranquille, en temps et lieu je me souviendrai de tout. Eh bien, partez donc, partez, au nom du Ciel! On m'attendait à minuit juste, et je suis en retard. 

\speak  De cinq minutes. 

\speak  Oui; mais dans certaines circonstances, cinq minutes sont cinq siècles. 

\speak  Quand on aime. 

\speak  Eh bien, qui vous dit que je n'ai pas affaire à un amoureux? 

\speak  C'est un homme qui vous attend? s'écria d'Artagnan, un homme! 

\speak  Allons, voilà la discussion qui va recommencer, fit Mme Bonacieux avec un demi-sourire qui n'était pas exempt d'une certaine teinte d'impatience. 

\speak  Non, non, je m'en vais, je pars; je crois en vous, je veux avoir tout le mérite de mon dévouement, ce dévouement dût-il être une stupidité. Adieu, madame, adieu!» 

Et comme s'il ne se fût senti la force de se détacher de la main qu'il tenait que par une secousse, il s'éloigna tout courant, tandis que Mme Bonacieux frappait, comme au volet, trois coups lents et réguliers; puis, arrivé à l'angle de la rue, il se retourna: la porte s'était ouverte et refermée, la jolie mercière avait disparu. 

D'Artagnan continua son chemin, il avait donné sa parole de ne pas épier Mme Bonacieux, et sa vie eût-elle dépendu de l'endroit où elle allait se rendre, ou de la personne qui devait l'accompagner, d'Artagnan serait rentré chez lui, puisqu'il avait dit qu'il y rentrait. Cinq minutes après, il était dans la rue des Fossoyeurs. 

«Pauvre Athos, disait-il, il ne saura pas ce que cela veut dire. Il se sera endormi en m'attendant, ou il sera retourné chez lui, et en rentrant il aura appris qu'une femme y était venue. Une femme chez Athos! Après tout, continua d'Artagnan, il y en avait bien une chez Aramis. Tout cela est fort étrange, et je serais bien curieux de savoir comment cela finira. 

\speak  Mal, monsieur, mal», répondit une voix que le jeune homme reconnut pour celle de Planchet; car tout en monologuant tout haut, à la manière des gens très préoccupés, il s'était engagé dans l'allée au fond de laquelle était l'escalier qui conduisait à sa chambre. 

«Comment, mal? que veux-tu dire, imbécile? demanda d'Artagnan, qu'est-il donc arrivé? 

\speak  Toutes sortes de malheurs. 

\speak  Lesquels? 

\speak  D'abord M. Athos est arrêté. 

\speak  Arrêté! Athos! arrêté! pourquoi? 

\speak  On l'a trouvé chez vous; on l'a pris pour vous. 

\speak  Et par qui a-t-il été arrêté? 

\speak  Par la garde qu'ont été chercher les hommes noirs que vous avez mis en fuite. 

\speak  Pourquoi ne s'est-il pas nommé? pourquoi n'a-t-il pas dit qu'il était étranger à cette affaire? 

\speak  Il s'en est bien gardé, monsieur; il s'est au contraire approché de moi et m'a dit: «C'est ton maître qui a besoin de sa liberté en ce moment, et non pas moi, puisqu'il sait tout et que je ne sais rien. On le croira arrêté, et cela lui donnera du temps; dans trois jours je dirai qui je suis, et il faudra bien qu'on me fasse sortir.» 

\speak  Bravo, Athos! noble cœur, murmura d'Artagnan, je le reconnais bien là! Et qu'ont fait les sbires? 

\speak  Quatre l'ont emmené je ne sais où, à la Bastille ou au For-l'Évêque; deux sont restés avec les hommes noirs, qui ont fouillé partout et qui ont pris tous les papiers. Enfin les deux derniers, pendant cette expédition, montaient la garde à la porte; puis, quand tout a été fini, ils sont partis, laissant la maison vide et tout ouvert. 

\speak  Et Porthos et Aramis? 

\speak  Je ne les avais pas trouvés, ils ne sont pas venus. 

\speak  Mais ils peuvent venir d'un moment à l'autre, car tu leur as fait dire que je les attendais? 

\speak  Oui, monsieur. 

\speak  Eh bien, ne bouge pas d'ici; s'ils viennent, préviens-les de ce qui m'est arrivé, qu'ils m'attendent au cabaret de la Pomme de Pin; ici il y aurait danger, la maison peut être espionnée. Je cours chez M. de Tréville pour lui annoncer tout cela, et je les y rejoins. 

\speak  C'est bien, monsieur, dit Planchet. 

\speak  Mais tu resteras, tu n'auras pas peur! dit d'Artagnan en revenant sur ses pas pour recommander le courage à son laquais. 

\speak  Soyez tranquille, monsieur, dit Planchet, vous ne me connaissez pas encore; je suis brave quand je m'y mets, allez; c'est le tout de m'y mettre; d'ailleurs je suis Picard. 

\speak  Alors, c'est convenu, dit d'Artagnan, tu te fais tuer plutôt que de quitter ton poste. 

\speak  Oui, monsieur, et il n'y a rien que je ne fasse pour prouver à monsieur que je lui suis attaché.» 

«Bon, dit en lui-même d'Artagnan, il paraît que la méthode que j'ai employée à l'égard de ce garçon est décidément la bonne: j'en userai dans l'occasion.» 

Et de toute la vitesse de ses jambes, déjà quelque peu fatiguées cependant par les courses de la journée, d'Artagnan se dirigea vers la rue du Colombier. 

M. de Tréville n'était point à son hôtel; sa compagnie était de garde au Louvre; il était au Louvre avec sa compagnie. 

Il fallait arriver jusqu'à M. de Tréville; il était important qu'il fût prévenu de ce qui se passait. D'Artagnan résolut d'essayer d'entrer au Louvre. Son costume de garde dans la compagnie de M. des Essarts lui devait être un passeport. 

Il descendit donc la rue des Petits-Augustins, et remonta le quai pour prendre le Pont-Neuf. Il avait eu un instant l'idée de passer le bac; mais en arrivant au bord de l'eau, il avait machinalement introduit sa main dans sa poche et s'était aperçu qu'il n'avait pas de quoi payer le passeur. 

Comme il arrivait à la hauteur de la rue Guénégaud, il vit déboucher de la rue Dauphine un groupe composé de deux personnes et dont l'allure le frappa. 

Les deux personnes qui composaient le groupe étaient: l'un, un homme; l'autre, une femme. 

La femme avait la tournure de Mme Bonacieux, et l'homme ressemblait à s'y méprendre à Aramis. 

En outre, la femme avait cette mante noire que d'Artagnan voyait encore se dessiner sur le volet de la rue de Vaugirard et sur la porte de la rue de La Harpe. 

De plus, l'homme portait l'uniforme des mousquetaires. 

Le capuchon de la femme était rabattu, l'homme tenait son mouchoir sur son visage; tous deux, cette double précaution l'indiquait, tous deux avaient donc intérêt à n'être point reconnus. 

Ils prirent le pont: c'était le chemin de d'Artagnan, puisque d'Artagnan se rendait au Louvre; d'Artagnan les suivit. 

D'Artagnan n'avait pas fait vingt pas, qu'il fut convaincu que cette femme, c'était Mme Bonacieux, et que cet homme, c'était Aramis. 

Il sentit à l'instant même tous les soupçons de la jalousie qui s'agitaient dans son cœur. 

Il était doublement trahi et par son ami et par celle qu'il aimait déjà comme une maîtresse. Mme Bonacieux lui avait juré ses grands dieux qu'elle ne connaissait pas Aramis, et un quart d'heure après qu'elle lui avait fait ce serment, il la retrouvait au bras d'Aramis. 

D'Artagnan ne réfléchit pas seulement qu'il connaissait la jolie mercière depuis trois heures seulement, qu'elle ne lui devait rien qu'un peu de reconnaissance pour l'avoir délivrée des hommes noirs qui voulaient l'enlever, et qu'elle ne lui avait rien promis. Il se regarda comme un amant outragé, trahi, bafoué; le sang et la colère lui montèrent au visage, il résolut de tout éclaircir. 

La jeune femme et le jeune homme s'étaient aperçus qu'ils étaient suivis, et ils avaient doublé le pas. D'Artagnan prit sa course, les dépassa, puis revint sur eux au moment où ils se trouvaient devant la Samaritaine, éclairée par un réverbère qui projetait sa lueur sur toute cette partie du pont. 

D'Artagnan s'arrêta devant eux, et ils s'arrêtèrent devant lui. 

«Que voulez-vous, monsieur? demanda le mousquetaire en reculant d'un pas et avec un accent étranger qui prouvait à d'Artagnan qu'il s'était trompé dans une partie de ses conjectures. 

\speak  Ce n'est pas Aramis! s'écria-t-il. 

\speak  Non, monsieur, ce n'est point Aramis, et à votre exclamation je vois que vous m'avez pris pour un autre, et je vous pardonne. 

\speak  Vous me pardonnez! s'écria d'Artagnan. 

\speak  Oui, répondit l'inconnu. Laissez-moi donc passer, puisque ce n'est pas à moi que vous avez affaire. 

\speak  Vous avez raison, monsieur, dit d'Artagnan, ce n'est pas à vous que j'ai affaire, c'est à madame. 

\speak  À madame! vous ne la connaissez pas, dit l'étranger. 

\speak  Vous vous trompez, monsieur, je la connais. 

\speak  Ah! fit Mme Bonacieux d'un ton de reproche, ah monsieur! j'avais votre parole de militaire et votre foi de gentilhomme; j'espérais pouvoir compter dessus. 

\speak  Et moi, madame, dit d'Artagnan embarrassé, vous m'aviez promis\dots 

\speak  Prenez mon bras, madame, dit l'étranger, et continuons notre chemin.» 

Cependant d'Artagnan, étourdi, atterré, anéanti par tout ce qui lui arrivait, restait debout et les bras croisés devant le mousquetaire et Mme Bonacieux. 

Le mousquetaire fit deux pas en avant et écarta d'Artagnan avec la main. 

D'Artagnan fit un bond en arrière et tira son épée. 

En même temps et avec la rapidité de l'éclair, l'inconnu tira la sienne. 

«Au nom du Ciel, Milord! s'écria Mme Bonacieux en se jetant entre les combattants et prenant les épées à pleines mains. 

\speak  Milord! s'écria d'Artagnan illuminé d'une idée subite, Milord! pardon, monsieur; mais est-ce que vous seriez\dots 

\speak  Milord duc de Buckingham, dit Mme Bonacieux à demi-voix; et maintenant vous pouvez nous perdre tous. 

\speak  Milord, madame, pardon, cent fois pardon; mais je l'aimais, Milord, et j'étais jaloux; vous savez ce que c'est que d'aimer, Milord; pardonnez-moi, et dites-moi comment je puis me faire tuer pour Votre Grâce. 

\speak  Vous êtes un brave jeune homme, dit Buckingham en tendant à d'Artagnan une main que celui-ci serra respectueusement; vous m'offrez vos services, je les accepte; suivez-nous à vingt pas jusqu'au Louvre; et si quelqu'un nous épie, tuez-le!» 

D'Artagnan mit son épée nue sous son bras, laissa prendre à Mme Bonacieux et au duc vingt pas d'avance et les suivit, prêt à exécuter à la lettre les instructions du noble et élégant ministre de Charles I\ier. 

Mais heureusement le jeune séide n'eut aucune occasion de donner au duc cette preuve de son dévouement, et la jeune femme et le beau mousquetaire rentrèrent au Louvre par le guichet de l'Échelle sans avoir été inquiétés\dots 

Quant à d'Artagnan, il se rendit aussitôt au cabaret de la Pomme de Pin, où il trouva Porthos et Aramis qui l'attendaient. 

Mais, sans leur donner d'autre explication sur le dérangement qu'il leur avait causé, il leur dit qu'il avait terminé seul l'affaire pour laquelle il avait cru un instant avoir besoin de leur intervention. Et maintenant, emportés que nous sommes par notre récit, laissons nos trois amis rentrer chacun chez soi, et suivons, dans les détours du Louvre, le duc de Buckingham et son guide.