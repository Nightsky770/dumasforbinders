%!TeX root=../musketeersfr.tex 

\chapter{Le Couvent Des Carmélites De Béthune}

\lettrine{L}{es} grands criminels portent avec eux une espèce de prédestination qui leur fait surmonter tous les obstacles, qui les fait échapper à tous les dangers, jusqu'au moment que la Providence, lassée, a marqué pour l'écueil de leur fortune impie. 

Il en était ainsi de Milady: elle passa au travers des croiseurs des deux nations, et arriva à Boulogne sans aucun accident. 

En débarquant à Portsmouth, Milady était une Anglaise que les persécutions de la France chassaient de La Rochelle; débarquée à Boulogne, après deux jours de traversée, elle se fit passer pour une Française que les Anglais inquiétaient à Portsmouth, dans la haine qu'ils avaient conçue contre la France. 

Milady avait d'ailleurs le plus efficace des passeports: sa beauté, sa grande mine et la générosité avec laquelle elle répandait les pistoles. Affranchie des formalités d'usage par le sourire affable et les manières galantes d'un vieux gouverneur du port, qui lui baisa la main, elle ne resta à Boulogne que le temps de mettre à la poste une lettre ainsi conçue: 

\begin{mail}{À Son Éminence Monseigneur le cardinal de Richelieu, en son camp devant La Rochelle.}{Monseigneur,} 
Que Votre Éminence se rassure, Sa Grâce le duc de Buckingham ne \textit{partira point} pour la France.	\addPS{Boulogne, 25 au soir. --- Selon les désirs de Votre Éminence, je me rends au couvent des Carmélites de Béthune où j'attendrai ses ordres.}
	\closeletter{Milady de ------} 
\end{mail}

Effectivement, le même soir, Milady se mit en route; la nuit la prit: elle s'arrêta et coucha dans une auberge; puis, le lendemain, à cinq heures du matin, elle partit, et trois heures après, elle entra à Béthune. 

Elle se fit indiquer le couvent des Carmélites et y entra aussitôt. 

La supérieure vint au-devant d'elle; Milady lui montra l'ordre du cardinal, l'abbesse lui fit donner une chambre et servir à déjeuner. 

Tout le passé s'était déjà effacé aux yeux de cette femme, et, le regard fixé vers l'avenir, elle ne voyait que la haute fortune que lui réservait le cardinal, qu'elle avait si heureusement servi, sans que son nom fût mêlé en rien à toute cette sanglante affaire. Les passions toujours nouvelles qui la consumaient donnaient à sa vie l'apparence de ces nuages qui volent dans le ciel, reflétant tantôt l'azur, tantôt le feu, tantôt le noir opaque de la tempête, et qui ne laissent d'autres traces sur la terre que la dévastation et la mort. 

Après le déjeuner, l'abbesse vint lui faire sa visite; il y a peu de distraction au cloître, et la bonne supérieure avait hâte de faire connaissance avec sa nouvelle pensionnaire. 

Milady voulait plaire à l'abbesse; or, c'était chose facile à cette femme si réellement supérieure; elle essaya d'être aimable: elle fut charmante et séduisit la bonne supérieure par sa conversation si variée et par les grâces répandues dans toute sa personne. 

L'abbesse, qui était une fille de noblesse, aimait surtout les histoires de cour, qui parviennent si rarement jusqu'aux extrémités du royaume et qui, surtout, ont tant de peine à franchir les murs des couvents, au seuil desquels viennent expirer les bruits du monde. 

Milady, au contraire, était fort au courant de toutes les intrigues aristocratiques, au milieu desquelles, depuis cinq ou six ans, elle avait constamment vécu, elle se mit donc à entretenir la bonne abbesse des pratiques mondaines de la cour de France, mêlées aux dévotions outrées du roi, elle lui fit la chronique scandaleuse des seigneurs et des dames de la cour, que l'abbesse connaissait parfaitement de nom, toucha légèrement les amours de la reine et de Buckingham, parlant beaucoup pour qu'on parlât un peu. 

Mais l'abbesse se contenta d'écouter et de sourire, le tout sans répondre. Cependant, comme Milady vit que ce genre de récit l'amusait fort, elle continua; seulement, elle fit tomber la conversation sur le cardinal. 

Mais elle était fort embarrassée; elle ignorait si l'abbesse était royaliste ou cardinaliste: elle se tint dans un milieu prudent; mais l'abbesse, de son côté, se tint dans une réserve plus prudente encore, se contentant de faire une profonde inclination de tête toutes les fois que la voyageuse prononçait le nom de Son Éminence. 

Milady commença à croire qu'elle s'ennuierait fort dans le couvent; elle résolut donc de risquer quelque chose pour savoir de suite à quoi s'en tenir. Voulant voir jusqu'où irait la discrétion de cette bonne abbesse, elle se mit à dire un mal, très dissimulé d'abord, puis très circonstancié du cardinal, racontant les amours du ministre avec Mme d'Aiguillon, avec Marion de Lorme et avec quelques autres femmes galantes. 

L'abbesse écouta plus attentivement, s'anima peu à peu et sourit. 

«Bon, dit Milady, elle prend goût à mon discours; si elle est cardinaliste, elle n'y met pas de fanatisme au moins.» 

Alors elle passa aux persécutions exercées par le cardinal sur ses ennemis. L'abbesse se contenta de se signer, sans approuver ni désapprouver. 

Cela confirma Milady dans son opinion que la religieuse était plutôt royaliste que cardinaliste. Milady continua, renchérissant de plus en plus. 

«Je suis fort ignorante de toutes ces matières-là, dit enfin l'abbesse, mais tout éloignées que nous sommes de la cour, tout en dehors des intérêts du monde où nous nous trouvons placées, nous avons des exemples fort tristes de ce que vous nous racontez là; et l'une de nos pensionnaires a bien souffert des vengeances et des persécutions de M. le cardinal. 

\speak  Une de vos pensionnaires, dit Milady; oh! mon Dieu! pauvre femme, je la plains alors. 

\speak  Et vous avez raison, car elle est bien à plaindre: prison, menaces, mauvais traitements, elle a tout souffert. Mais, après tout, reprit l'abbesse, M. le cardinal avait peut-être des motifs plausibles pour agir ainsi, et quoiqu'elle ait l'air d'un ange, il ne faut pas toujours juger les gens sur la mine.» 

«Bon! dit Milady à elle-même, qui sait! je vais peut-être découvrir quelque chose ici, je suis en veine.» 

Et elle s'appliqua à donner à son visage une expression de candeur parfaite. 

«Hélas! dit Milady, je le sais; on dit cela, qu'il ne faut pas croire aux physionomies; mais à quoi croira-t-on cependant, si ce n'est au plus bel ouvrage du Seigneur? Quant à moi, je serai trompée toute ma vie peut-être; mais je me fierai toujours à une personne dont le visage m'inspirera de la sympathie. 

\speak  Vous seriez donc tentée de croire, dit l'abbesse, que cette jeune femme est innocente? 

\speak  M. le cardinal ne punit pas que les crimes, dit-elle; il y a certaines vertus qu'il poursuit plus sévèrement que certains forfaits. 

\speak  Permettez-moi, madame, de vous exprimer ma surprise, dit l'abbesse. 

\speak  Et sur quoi? demanda Milady avec naïveté. 

\speak  Mais sur le langage que vous tenez. 

\speak  Que trouvez-vous d'étonnant à ce langage? demanda en souriant Milady. 

\speak  Vous êtes l'amie du cardinal, puisqu'il vous envoie ici, et cependant\dots 

\speak  Et cependant j'en dis du mal, reprit Milady, achevant la pensée de la supérieure. 

\speak  Au moins n'en dites-vous pas de bien. 

\speak  C'est que je ne suis pas son amie, dit-elle en soupirant, mais sa victime. 

\speak  Mais cependant cette lettre par laquelle il vous recommande à moi?\dots 

\speak  Est un ordre à moi de me tenir dans une espèce de prison dont il me fera tirer par quelques-uns de ses satellites. 

\speak  Mais pourquoi n'avez-vous pas fui? 

\speak  Où irais-je? croyez-vous qu'il y ait un endroit de la terre où ne puisse atteindre le cardinal, s'il veut se donner la peine de tendre la main? Si j'étais un homme, à la rigueur cela serait possible encore; mais une femme, que voulez-vous que fasse une femme? Cette jeune pensionnaire que vous avez ici a-t-elle essayé de fuir, elle? 

\speak  Non, c'est vrai; mais elle, c'est autre chose, je la crois retenue en France par quelque amour. 

\speak  Alors, dit Milady avec un soupir, si elle aime, elle n'est pas tout à fait malheureuse. 

\speak  Ainsi, dit l'abbesse en regardant Milady avec un intérêt croissant, c'est encore une pauvre persécutée que je vois? 

\speak  Hélas, oui, dit Milady. 

L'abbesse regarda un instant Milady avec inquiétude, comme si une nouvelle pensée surgissait dans son esprit. 

«Vous n'êtes pas ennemie de notre sainte foi? dit-elle en balbutiant. 

\speak  Moi, s'écria Milady, moi, protestante! Oh! non, j'atteste le Dieu qui nous entend que je suis au contraire fervente catholique. 

\speak  Alors, madame, dit l'abbesse en souriant, rassurez-vous; la maison où vous êtes ne sera pas une prison bien dure, et nous ferons tout ce qu'il faudra pour vous faire chérir la captivité. Il y a plus, vous trouverez ici cette jeune femme persécutée sans doute par suite de quelque intrigue de cour. Elle est aimable, gracieuse. 

\speak  Comment la nommez-vous? 

\speak  Elle m'a été recommandée par quelqu'un de très haut placé, sous le nom de Ketty. Je n'ai pas cherché à savoir son autre nom. 

\speak  Ketty! s'écria Milady; quoi! vous êtes sûre?\dots 

\speak  Qu'elle se fait appeler ainsi? Oui, madame, la connaîtriez-vous?» 

Milady sourit à elle-même et à l'idée qui lui était venue que cette jeune femme pouvait être son ancienne camérière. Il se mêlait au souvenir de cette jeune fille un souvenir de colère, et un désir de vengeance avait bouleversé les traits de Milady, qui reprirent au reste presque aussitôt l'expression calme et bienveillante que cette femme aux cent visages leur avait momentanément fait perdre. 

«Et quand pourrai-je voir cette jeune dame, pour laquelle je me sens déjà une si grande sympathie? demanda Milady. 

\speak  Mais, ce soir, dit l'abbesse, dans la journée même. Mais vous voyagez depuis quatre jours, m'avez-vous dit vous-même; ce matin vous vous êtes levée à cinq heures, vous devez avoir besoin de repos. Couchez-vous et dormez, à l'heure du dîner nous vous réveillerons.» 

Quoique Milady eût très bien pu se passer de sommeil, soutenue qu'elle était par toutes les excitations qu'une aventure nouvelle faisait éprouver à son cœur avide d'intrigues, elle n'en accepta pas moins l'offre de la supérieure: depuis douze ou quinze jours elle avait passé par tant d'émotions diverses que, si son corps de fer pouvait encore soutenir la fatigue, son âme avait besoin de repos. 

Elle prit donc congé de l'abbesse et se coucha, doucement bercée par les idées de vengeance auxquelles l'avait tout naturellement ramenée le nom de Ketty. Elle se rappelait cette promesse presque illimitée que lui avait faite le cardinal, si elle réussissait dans son entreprise. Elle avait réussi, elle pourrait donc se venger de d'Artagnan. 

Une seule chose épouvantait Milady, c'était le souvenir de son mari! le comte de La Fère, qu'elle avait cru mort ou du moins expatrié, et qu'elle retrouvait dans Athos, le meilleur ami de d'Artagnan. 

Mais aussi, s'il était l'ami de d'Artagnan, il avait dû lui prêter assistance dans toutes les menées à l'aide desquelles la reine avait déjoué les projets de Son Éminence; s'il était l'ami de d'Artagnan, il était l'ennemi du cardinal; et sans doute elle parviendrait à l'envelopper dans la vengeance aux replis de laquelle elle comptait étouffer le jeune mousquetaire. 

Toutes ces espérances étaient de douces pensées pour Milady; aussi, bercée par elles, s'endormit-elle bientôt. 

Elle fut réveillée par une voix douce qui retentit au pied de son lit. Elle ouvrit les yeux, et vit l'abbesse accompagnée d'une jeune femme aux cheveux blonds, au teint délicat, qui fixait sur elle un regard plein d'une bienveillante curiosité. 

La figure de cette jeune femme lui était complètement inconnue; toutes deux s'examinèrent avec une scrupuleuse attention, tout en échangeant les compliments d'usage: toutes deux étaient fort belles, mais de beautés tout à fait différentes. Cependant Milady sourit en reconnaissant qu'elle l'emportait de beaucoup sur la jeune femme en grand air et en façons aristocratiques. Il est vrai que l'habit de novice que portait la jeune femme n'était pas très avantageux pour soutenir une lutte de ce genre. 

L'abbesse les présenta l'une à l'autre; puis, lorsque cette formalité fut remplie, comme ses devoirs l'appelaient à l'église, elle laissa les deux jeunes femmes seules. 

La novice, voyant Milady couchée, voulait suivre la supérieure, mais Milady la retint. 

«Comment, madame, lui dit-elle, à peine vous ai-je aperçue et vous voulez déjà me priver de votre présence, sur laquelle je comptais cependant un peu, je vous l'avoue, pour le temps que j'ai à passer ici? 

\speak  Non, madame, répondit la novice, seulement je craignais d'avoir mal choisi mon temps: vous dormiez, vous êtes fatiguée. 

\speak  Eh bien, dit Milady, que peuvent demander les gens qui dorment? un bon réveil. Ce réveil, vous me l'avez donné; laissez-moi en jouir tout à mon aise.» 

Et lui prenant la main, elle l'attira sur un fauteuil qui était près de son lit. 

La novice s'assit. 

«Mon Dieu! dit-elle, que je suis malheureuse! voilà six mois que je suis ici, sans l'ombre d'une distraction, vous arrivez, votre présence allait être pour moi une compagnie charmante, et voilà que, selon toute probabilité, d'un moment à l'autre je vais quitter le couvent! 

\speak  Comment! dit Milady, vous sortez bientôt? 

\speak  Du moins je l'espère, dit la novice avec une expression de joie qu'elle ne cherchait pas le moins du monde à déguiser. 

\speak  Je crois avoir appris que vous aviez souffert de la part du cardinal, continua Milady; c'eût été un motif de plus de sympathie entre nous. 

\speak  Ce que m'a dit notre bonne mère est donc la vérité, que vous étiez aussi une victime de ce méchant cardinal? 

\speak  Chut! dit Milady, même ici ne parlons pas ainsi de lui; tous mes malheurs viennent d'avoir dit à peu près ce que vous venez de dire, devant une femme que je croyais mon amie et qui m'a trahie. Et vous êtes aussi, vous, la victime d'une trahison? 

\speak  Non, dit la novice, mais de mon dévouement à une femme que j'aimais, pour qui j'eusse donné ma vie, pour qui je la donnerais encore. 

\speak  Et qui vous a abandonnée, c'est cela! 

\speak  J'ai été assez injuste pour le croire, mais depuis deux ou trois jours j'ai acquis la preuve du contraire, et j'en remercie Dieu; il m'aurait coûté de croire qu'elle m'avait oubliée. Mais vous, madame, continua la novice, il me semble que vous êtes libre, et que si vous vouliez fuir, il ne tiendrait qu'à vous. 

\speak  Où voulez-vous que j'aille, sans amis, sans argent, dans une partie de la France que je ne connais pas, où je ne suis jamais venue?\dots 

\speak  Oh! s'écria la novice, quant à des amis, vous en aurez partout où vous vous montrerez, vous paraissez si bonne et vous êtes si belle! 

\speak  Cela n'empêche pas, reprit Milady en adoucissant son sourire de manière à lui donner une expression angélique, que je suis seule et persécutée. 

\speak  Écoutez, dit la novice, il faut avoir bon espoir dans le Ciel, voyez-vous; il vient toujours un moment où le bien que l'on a fait plaide votre cause devant Dieu, et, tenez, peut-être est-ce un bonheur pour vous, tout humble et sans pouvoir que je suis, que vous m'ayez rencontrée: car, si je sors d'ici, eh bien, j'aurai quelques amis puissants, qui, après s'être mis en campagne pour moi, pourront aussi se mettre en campagne pour vous. 

\speak  Oh! quand j'ai dit que j'étais seule, dit Milady, espérant faire parler la novice en parlant d'elle-même, ce n'est pas faute d'avoir aussi quelques connaissances haut placées; mais ces connaissances tremblent elles-mêmes devant le cardinal: la reine elle-même n'ose pas soutenir contre le terrible ministre; j'ai la preuve que Sa Majesté, malgré son excellent cœur, a plus d'une fois été obligée d'abandonner à la colère de Son Éminence les personnes qui l'avaient servie. 

\speak  Croyez-moi, madame, la reine peut avoir l'air d'avoir abandonné ces personnes-là; mais il ne faut pas en croire l'apparence: plus elles sont persécutées, plus elle pense à elles, et souvent, au moment où elles y pensent le moins, elles ont la preuve d'un bon souvenir. 

\speak  Hélas! dit Milady, je le crois: la reine est si bonne. 

\speak  Oh! vous la connaissez donc, cette belle et noble reine, que vous parlez d'elle ainsi! s'écria la novice avec enthousiasme. 

\speak  C'est-à-dire, reprit Milady, poussée dans ses retranchements, qu'elle, personnellement, je n'ai pas l'honneur de la connaître; mais je connais bon nombre de ses amis les plus intimes: je connais M. de Putange; j'ai connu en Angleterre M. Dujart; je connais M. de Tréville. 

\speak  M. de Tréville! s'écria la novice, vous connaissez M. de Tréville? 

\speak  Oui, parfaitement, beaucoup même. 

\speak  Le capitaine des mousquetaires du roi? 

\speak  Le capitaine des mousquetaires du roi. 

\speak  Oh! mais vous allez voir, s'écria la novice, que tout à l'heure nous allons être des connaissances achevées, presque des amies; si vous connaissez M. de Tréville, vous avez dû aller chez lui? 

\speak  Souvent! dit Milady, qui, entrée dans cette voie, et s'apercevant que le mensonge réussissait, voulait le pousser jusqu'au bout. 

\speak  Chez lui, vous avez dû voir quelques-uns de ses mousquetaires? 

\speak  Tous ceux qu'il reçoit habituellement! répondit Milady, pour laquelle cette conversation commençait à prendre un intérêt réel. 

\speak  Nommez-moi quelques-uns de ceux que vous connaissez, et vous verrez qu'ils seront de mes amis. 

\speak  Mais, dit Milady embarrassée, je connais M. de Louvigny, M. de Courtivron, M. de Férussac.» 

La novice la laissa dire; puis, voyant qu'elle s'arrêtait: 

«Vous ne connaissez pas, dit-elle, un gentilhomme nommé Athos?» 

Milady devint aussi pâle que les draps dans lesquels elle était couchée, et, si maîtresse qu'elle fût d'elle-même, ne put s'empêcher de pousser un cri en saisissant la main de son interlocutrice et en la dévorant du regard. 

«Quoi! qu'avez-vous? Oh! mon Dieu! demanda cette pauvre femme, ai-je donc dit quelque chose qui vous ait blessée? 

\speak  Non, mais ce nom m'a frappée, parce que, moi aussi j'ai connu ce gentilhomme, et qu'il me paraît étrange de trouver quelqu'un qui le connaisse beaucoup. 

\speak  Oh! oui! beaucoup! beaucoup! non seulement lui, mais encore ses amis: MM. Porthos et Aramis! 

\speak  En vérité! eux aussi je les connais! s'écria Milady, qui sentit le froid pénétrer jusqu'à son cœur. 

\speak  Eh bien, si vous les connaissez, vous devez savoir qu'ils sont bons et francs compagnons; que ne vous adressez-vous à eux, si vous avez besoin d'appui? 

\speak  C'est-à-dire, balbutia Milady, je ne suis liée réellement avec aucun d'eux; je les connais pour en avoir beaucoup entendu parler par un de leurs amis, M. d'Artagnan. 

\speak  Vous connaissez M. d'Artagnan!» s'écria la novice à son tour, en saisissant la main de Milady et en la dévorant des yeux. 

Puis, remarquant l'étrange expression du regard de Milady: 

«Pardon, madame, dit-elle, vous le connaissez, à quel titre? 

\speak  Mais, reprit Milady embarrassée, mais à titre d'ami. 

\speak  Vous me trompez, madame, dit la novice; vous avez été sa maîtresse. 

\speak  C'est vous qui l'avez été, madame, s'écria Milady à son tour. 

\speak  Moi! dit la novice. 

\speak  Oui, vous; je vous connais maintenant: vous êtes madame Bonacieux.» 

La jeune femme se recula, pleine de surprise et de terreur. 

«Oh! ne niez pas! répondez, reprit Milady. 

\speak  Eh bien, oui, madame! je l'aime, dit la novice; sommes-nous rivales?» 

La figure de Milady s'illumina d'un feu tellement sauvage que, dans toute autre circonstance, Mme Bonacieux se fût enfuie d'épouvante; mais elle était toute à sa jalousie. 

«Voyons, dites, madame, reprit Mme Bonacieux avec une énergie dont on l'eût crue incapable, avez-vous été ou êtes-vous sa maîtresse? 

\speak  Oh! non! s'écria Milady avec un accent qui n'admettait pas le doute sur sa vérité, jamais! jamais! 

\speak  Je vous crois, dit Mme Bonacieux; mais pourquoi donc alors vous êtes-vous écriée ainsi? 

\speak  Comment, vous ne comprenez pas! dit Milady, qui était déjà remise de son trouble, et qui avait retrouvé toute sa présence d'esprit. 

\speak  Comment voulez-vous que je comprenne? je ne sais rien. 

\speak  Vous ne comprenez pas que M. d'Artagnan étant mon ami, il m'avait prise pour confidente? 

\speak  Vraiment! 

\speak  Vous ne comprenez pas que je sais tout, votre enlèvement de la petite maison de Saint-Germain, son désespoir, celui de ses amis, leurs recherches inutiles depuis ce moment! Et comment ne voulez-vous pas que je m'en étonne, quand, sans m'en douter, je me trouve en face de vous, de vous dont nous avons parlé si souvent ensemble, de vous qu'il aime de toute la force de son âme, de vous qu'il m'avait fait aimer avant que je vous eusse vue? Ah! chère Constance, je vous trouve donc, je vous vois donc enfin!» 

Et Milady tendit ses bras à Mme Bonacieux, qui, convaincue par ce qu'elle venait de lui dire, ne vit plus dans cette femme, qu'un instant auparavant elle avait crue sa rivale, qu'une amie sincère et dévouée. 

«Oh! pardonnez-moi! pardonnez-moi! s'écria-t-elle en se laissant aller sur son épaule, je l'aime tant!» 

Ces deux femmes se tinrent un instant embrassées. Certes, si les forces de Milady eussent été à la hauteur de sa haine, Mme Bonacieux ne fût sortie que morte de cet embrassement. Mais, ne pouvant pas l'étouffer, elle lui sourit. 

«O chère belle! chère bonne petite! dit Milady, que je suis heureuse de vous voir! Laissez-moi vous regarder. Et, en disant ces mots, elle la dévorait effectivement du regard. Oui, c'est bien vous. Ah! d'après ce qu'il m'a dit, je vous reconnais à cette heure, je vous reconnais parfaitement.» 

La pauvre jeune femme ne pouvait se douter de ce qui se passait d'affreusement cruel derrière le rempart de ce front pur, derrière ces yeux si brillants où elle ne lisait que de l'intérêt et de la compassion. 

«Alors vous savez ce que j'ai souffert, dit Mme Bonacieux, puisqu'il vous a dit ce qu'il souffrait; mais souffrir pour lui, c'est du bonheur.» 

Milady reprit machinalement: 

«Oui, c'est du bonheur.» 

Elle pensait à autre chose. 

«Et puis, continua Mme Bonacieux, mon supplice touche à son terme; demain, ce soir peut-être, je le reverrai, et alors le passé n'existera plus. 

\speak  Ce soir? demain? s'écria Milady tirée de sa rêverie par ces paroles, que voulez-vous dire? attendez-vous quelque nouvelle de lui? 

\speak  Je l'attends lui-même. 

\speak  Lui-même; d'Artagnan, ici! 

\speak  Lui-même. 

\speak  Mais, c'est impossible! il est au siège de La Rochelle avec le cardinal; il ne reviendra à Paris qu'après la prise de la ville. 

\speak  Vous le croyez ainsi, mais est-ce qu'il y a quelque chose d'impossible à mon d'Artagnan, le noble et loyal gentilhomme! 

\speak  Oh! je ne puis vous croire! 

\speak  Eh bien, lisez donc!» dit, dans l'excès de son orgueil et de sa joie, la malheureuse jeune femme en présentant une lettre à Milady. 

«L'écriture de Mme de Chevreuse! se dit en elle-même Milady. Ah! j'étais bien sûre qu'ils avaient des intelligences de ce côté-là!» 

Et elle lut avidement ces quelques lignes:
\begin{quotation}
	«Ma chère enfant, tenez-vous prête; notre ami vous verra bientôt, et il ne vous verra que pour vous arracher de la prison où votre sûreté exigeait que vous fussiez cachée: préparez-vous donc au départ et ne désespérez jamais de nous.

«Notre charmant Gascon vient de se montrer brave et fidèle comme toujours, dites-lui qu'on lui est bien reconnaissant quelque part de l'avis qu'il a donné.» 
\end{quotation}

«Oui, oui, dit Milady, oui, la lettre est précise. Savez-vous quel est cet avis? 

\speak  Non. Je me doute seulement qu'il aura prévenu la reine de quelque nouvelle machination du cardinal. 

\speak  Oui, c'est cela sans doute!» dit Milady en rendant la lettre à Mme Bonacieux et en laissant retomber sa tête pensive sur sa poitrine. 

En ce moment on entendit le galop d'un cheval. 

«Oh! s'écria Mme Bonacieux en s'élançant à la fenêtre, serait-ce déjà lui?» 

Milady était restée dans son lit, pétrifiée par la surprise; tant de choses inattendues lui arrivaient tout à coup, que pour la première fois la tête lui manquait. 

«Lui! lui! murmura-t-elle, serait-ce lui?» 

Et elle demeurait dans son lit les yeux fixes. 

«Hélas, non! dit Mme Bonacieux, c'est un homme que je ne connais pas, et qui cependant a l'air de venir ici; oui, il ralentit sa course, il s'arrête à la porte, il sonne. 

Milady sauta hors de son lit. 

«Vous êtes bien sûre que ce n'est pas lui? dit-elle. 

\speak  Oh! oui, bien sûre! 

\speak  Vous avez peut-être mal vu. 

\speak  Oh! je verrais la plume de son feutre, le bout de son manteau, que je le reconnaîtrais, lui! 

Milady s'habillait toujours. 

«N'importe! cet homme vient ici, dites-vous? 

\speak  Oui, il est entré. 

\speak  C'est ou pour vous ou pour moi. 

\speak  Oh! mon Dieu, comme vous semblez agitée! 

\speak  Oui, je l'avoue, je n'ai pas votre confiance, je crains tout du cardinal. 

\speak  Chut! dit Mme Bonacieux, on vient!» 

Effectivement, la porte s'ouvrit, et la supérieure entra. 

«Est-ce vous qui arrivez de Boulogne? demanda-t-elle à Milady. 

\speak  Oui, c'est moi, répondit celle-ci, et, tâchant de ressaisir son sang-froid, qui me demande? 

\speak  Un homme qui ne veut pas dire son nom, mais qui vient de la part du cardinal. 

\speak  Et qui veut me parler? demanda Milady. 

\speak  Qui veut parler à une dame arrivant de Boulogne. 

\speak  Alors faites entrer, madame, je vous prie. 

\speak  Oh! mon Dieu! mon Dieu! dit Mme Bonacieux, serait-ce quelque mauvaise nouvelle? 

\speak  J'en ai peur. 

\speak  Je vous laisse avec cet étranger, mais aussitôt son départ, si vous le permettez, je reviendrai. 

\speak  Comment donc! je vous en prie.» 

La supérieure et Mme Bonacieux sortirent. 

Milady resta seule, les yeux fixés sur la porte; un instant après on entendit le bruit d'éperons qui retentissaient sur les escaliers, puis les pas se rapprochèrent, puis la porte s'ouvrit, et un homme parut. 

Milady jeta un cri de joie: cet homme c'était le comte de Rochefort, l'âme damnée de Son Éminence.