%!TeX root=../musketeersfr.tex 

\chapter{Le Vin D'Anjou}

\lettrine{A}{près} des nouvelles presque désespérées du roi, le bruit de sa convalescence commençait à se répandre dans le camp; et comme il avait grande hâte d'arriver en personne au siège, on disait qu'aussitôt qu'il pourrait remonter à cheval, il se remettrait en route. 

Pendant ce temps, Monsieur, qui savait que, d'un jour à l'autre, il allait être remplacé dans son commandement, soit par le duc d'Angoulême, soit par Bassompierre ou par Schomberg, qui se disputaient le commandement, faisait peu de choses, perdait ses journées en tâtonnements, et n'osait risquer quelque grande entreprise pour chasser les Anglais de l'île de Ré, où ils assiégeaient toujours la citadelle Saint-Martin et le fort de La Prée, tandis que, de leur côté, les Français assiégeaient La Rochelle. 

D'Artagnan, comme nous l'avons dit, était redevenu plus tranquille, comme il arrive toujours après un danger passé, et quand le danger semble évanoui; il ne lui restait qu'une inquiétude, c'était de n'apprendre aucune nouvelle de ses amis. 

Mais, un matin du commencement du mois de novembre, tout lui fut expliqué par cette lettre, datée de Villeroi: 
\begin{mail}{}{Monsieur d'Artagnan,} 
	MM. Athos, Porthos et Aramis, après avoir fait une bonne partie chez moi, et s'être égayés beaucoup, ont mené si grand bruit, que le prévôt du château, homme très rigide, les a consignés pour quelques jours; mais j'accomplis les ordres qu'ils m'ont donnés, de vous envoyer douze bouteilles de mon vin d'Anjou, dont ils ont fait grand cas: ils veulent que vous buviez à leur santé avec leur vin favori.
	
	Je l'ai fait, et suis, monsieur, avec un grand respect,
	
	\closeletter[Votre serviteur très humble et très obéissant,]{Godeau,\\ \textit{Hôtelier de messieurs les mousquetaires.}}
	\end{mail}

«À la bonne heure! s'écria d'Artagnan, ils pensent à moi dans leurs plaisirs comme je pensais à eux dans mon ennui; bien certainement que je boirai à leur santé et de grand cœur; mais je n'y boirai pas seul.» 

Et d'Artagnan courut chez deux gardes, avec lesquels il avait fait plus amitié qu'avec les autres, afin de les inviter à boire avec lui le délicieux petit vin d'Anjou qui venait d'arriver de Villeroi. L'un des deux gardes était invité pour le soir même, et l'autre invité pour le lendemain; la réunion fut donc fixée au surlendemain. 

D'Artagnan, en rentrant, envoya les douze bouteilles de vin à la buvette des gardes, en recommandant qu'on les lui gardât avec soin; puis, le jour de la solennité, comme le dîner était fixé pour l'heure de midi, d'Artagnan envoya, dès neuf heures, Planchet pour tout préparer. 

Planchet, tout fier d'être élevé à la dignité de maître d'hôtel, songea à tout apprêter en homme intelligent; à cet effet il s'adjoignit le valet d'un des convives de son maître, nommé Fourreau, et ce faux soldat qui avait voulu tuer d'Artagnan, et qui, n'appartenant à aucun corps, était entré à son service ou plutôt à celui de Planchet, depuis que d'Artagnan lui avait sauvé la vie. 

L'heure du festin venue, les deux convives arrivèrent, prirent place et les mets s'alignèrent sur la table. Planchet servait la serviette au bras, Fourreau débouchait les bouteilles, et Brisemont, c'était le nom du convalescent, transvasait dans des carafons de verre le vin qui paraissait avoir déposé par effet des secousses de la route. De ce vin, la première bouteille était un peu trouble vers la fin, Brisemont versa cette lie dans un verre, et d'Artagnan lui permit de la boire; car le pauvre diable n'avait pas encore beaucoup de forces. 

Les convives, après avoir mangé le potage, allaient porter le premier verre à leurs lèvres, lorsque tout à coup le canon retentit au fort Louis et au fort Neuf; aussitôt les gardes, croyant qu'il s'agissait de quelque attaque imprévue, soit des assiégés, soit des Anglais, sautèrent sur leurs épées; d'Artagnan, non moins leste, fit comme eux, et tous trois sortirent en courant, afin de se rendre à leurs postes. 

Mais à peine furent-ils hors de la buvette, qu'ils se trouvèrent fixés sur la cause de ce grand bruit; les cris de Vive le roi! Vive M. le cardinal! retentissaient de tous côtés, et les tambours battaient dans toutes les directions. 

En effet, le roi, impatient comme on l'avait dit, venait de doubler deux étapes, et arrivait à l'instant même avec toute sa maison et un renfort de dix mille hommes de troupe; ses mousquetaires le précédaient et le suivaient. D'Artagnan, placé en haie avec sa compagnie, salua d'un geste expressif ses amis, qui lui répondirent des yeux, et M. de Tréville, qui le reconnut tout d'abord. 

La cérémonie de réception achevée, les quatre amis furent bientôt dans les bras l'un de l'autre. 

«Pardieu! s'écria d'Artagnan, il n'est pas possible de mieux arriver, et les viandes n'auront pas encore eu le temps de refroidir! n'est-ce pas, messieurs? ajouta le jeune homme en se tournant vers les deux gardes, qu'il présenta à ses amis. 

\speak  Ah! ah! il paraît que nous banquetions, dit Porthos. 

\speak  J'espère, dit Aramis, qu'il n'y a pas de femmes à votre dîner! 

\speak  Est-ce qu'il y a du vin potable dans votre bicoque? demanda Athos. 

\speak  Mais, pardieu! il y a le vôtre, cher ami, répondit d'Artagnan. 

\speak  Notre vin? fit Athos étonné. 

\speak  Oui, celui que vous m'avez envoyé. 

\speak  Nous vous avons envoyé du vin? 

\speak  Mais vous savez bien, de ce petit vin des coteaux d'Anjou? 

\speak  Oui, je sais bien de quel vin vous voulez parler. 

\speak  Le vin que vous préférez. 

\speak  Sans doute, quand je n'ai ni champagne ni chambertin. 

\speak  Eh bien, à défaut de champagne et de chambertin, vous vous contenterez de celui-là. 

\speak  Nous avons donc fait venir du vin d'Anjou, gourmet que nous sommes? dit Porthos. 

\speak  Mais non, c'est le vin qu'on m'a envoyé de votre part. 

\speak  De notre part? firent les trois mousquetaires. 

\speak  Est-ce vous, Aramis, dit Athos, qui avez envoyé du vin? 

\speak  Non, et vous, Porthos? 

\speak  Non, et vous, Athos? 

\speak  Non. 

\speak  Si ce n'est pas vous, dit d'Artagnan, c'est votre hôtelier. 

\speak  Notre hôtelier? 

\speak  Eh oui! votre hôtelier, Godeau, hôtelier des mousquetaires. 

\speak  Ma foi, qu'il vienne d'où il voudra, n'importe, dit Porthos, goûtons-le, et, s'il est bon, buvons-le. 

\speak  Non pas, dit Athos, ne buvons pas le vin qui a une source inconnue. 

\speak  Vous avez raison, Athos, dit d'Artagnan. Personne de vous n'a chargé l'hôtelier Godeau de m'envoyer du vin? 

\speak  Non! et cependant il vous en a envoyé de notre part? 

\speak  Voici la lettre!» dit d'Artagnan. 

Et il présenta le billet à ses camarades. 

«Ce n'est pas son écriture! s'écria Athos, je la connais, c'est moi qui, avant de partir, ai réglé les comptes de la communauté. 

\speak  Fausse lettre, dit Porthos; nous n'avons pas été consignés. 

\speak  D'Artagnan, demanda Aramis d'un ton de reproche, comment avez-vous pu croire que nous avions fait du bruit?\dots» 

D'Artagnan pâlit, et un tremblement convulsif secoua tous ses membres. 

«Tu m'effraies, dit Athos, qui ne le tutoyait que dans les grandes occasions, qu'est-il donc arrivé? 

\speak  Courons, courons, mes amis! s'écria d'Artagnan, un horrible soupçon me traverse l'esprit! serait-ce encore une vengeance de cette femme?» 

Ce fut Athos qui pâlit à son tour. 

D'Artagnan s'élança vers la buvette, les trois mousquetaires et les deux gardes l'y suivirent. 

Le premier objet qui frappa la vue de d'Artagnan en entrant dans la salle à manger, fut Brisemont étendu par terre et se roulant dans d'atroces convulsions. 

Planchet et Fourreau, pâles comme des morts, essayaient de lui porter secours; mais il était évident que tout secours était inutile: tous les traits du moribond étaient crispés par l'agonie. 

«Ah! s'écria-t-il en apercevant d'Artagnan, ah! c'est affreux, vous avez l'air de me faire grâce et vous m'empoisonnez! 

\speak  Moi! s'écria d'Artagnan, moi, malheureux! moi! que dis-tu donc là? 

\speak  Je dis que c'est vous qui m'avez donné ce vin, je dis que c'est vous qui m'avez dit de le boire, je dis que vous avez voulu vous venger de moi, je dis que c'est affreux! 

\speak  N'en croyez rien, Brisemont, dit d'Artagnan, n'en croyez rien; je vous jure, je vous proteste\dots 

\speak  Oh! mais Dieu est là! Dieu vous punira! Mon Dieu! qu'il souffre un jour ce que je souffre! 

\speak  Sur l'évangile, s'écria d'Artagnan en se précipitant vers le moribond, je vous jure que j'ignorais que ce vin fût empoisonné et que j'allais en boire comme vous. 

\speak  Je ne vous crois pas», dit le soldat. 

Et il expira dans un redoublement de tortures. 

«Affreux! affreux! murmurait Athos, tandis que Porthos brisait les bouteilles et qu'Aramis donnait des ordres un peu tardifs pour qu'on allât chercher un confesseur. 

\speak  O mes amis! dit d'Artagnan, vous venez encore une fois de me sauver la vie, non seulement à moi, mais à ces messieurs. Messieurs, continua-t-il en s'adressant aux gardes, je vous demanderai le silence sur toute cette aventure; de grands personnages pourraient avoir trempé dans ce que vous avez vu, et le mal de tout cela retomberait sur nous. 

\speak  Ah! monsieur! balbutiait Planchet plus mort que vif; ah! monsieur! que je l'ai échappé belle! 

\speak  Comment, drôle, s'écria d'Artagnan, tu allais donc boire mon vin? 

\speak  À la santé du roi, monsieur, j'allais en boire un pauvre verre, si Fourreau ne m'avait pas dit qu'on m'appelait. 

\speak  Hélas! dit Fourreau, dont les dents claquaient de terreur, je voulais l'éloigner pour boire tout seul! 

\speak  Messieurs, dit d'Artagnan en s'adressant aux gardes, vous comprenez qu'un pareil festin ne pourrait être que fort triste après ce qui vient de se passer; ainsi recevez toutes mes excuses et remettez la partie à un autre jour, je vous prie.» 

Les deux gardes acceptèrent courtoisement les excuses de d'Artagnan, et, comprenant que les quatre amis désiraient demeurer seuls, ils se retirèrent. 

Lorsque le jeune garde et les trois mousquetaires furent sans témoins, ils se regardèrent d'un air qui voulait dire que chacun comprenait la gravité de la situation. 

«D'abord, dit Athos, sortons de cette chambre; c'est une mauvaise compagnie qu'un mort, mort de mort violente. 

\speak  Planchet, dit d'Artagnan, je vous recommande le cadavre de ce pauvre diable. Qu'il soit enterré en terre sainte. Il avait commis un crime, c'est vrai, mais il s'en était repenti.» 

Et les quatre amis sortirent de la chambre, laissant à Planchet et à Fourreau le soin de rendre les honneurs mortuaires à Brisemont. 

L'hôte leur donna une autre chambre dans laquelle il leur servit des oeufs à la coque et de l'eau, qu'Athos alla puiser lui-même à la fontaine. En quelques paroles Porthos et Aramis furent mis au courant de la situation. 

«Eh bien, dit d'Artagnan à Athos, vous le voyez, cher ami, c'est une guerre à mort.» 

Athos secoua la tête. 

«Oui, oui, dit-il, je le vois bien; mais croyez-vous que ce soit elle? 

\speak  J'en suis sûr. 

\speak  Cependant je vous avoue que je doute encore. 

\speak  Mais cette fleur de lis sur l'épaule? 

\speak  C'est une Anglaise qui aura commis quelque méfait en France, et qu'on aura flétrie à la suite de son crime. 

\speak  Athos, c'est votre femme, vous dis-je, répétait d'Artagnan, ne vous rappelez-vous donc pas comme les deux signalements se ressemblent? 

\speak  J'aurais cependant cru que l'autre était morte, je l'avais si bien pendue.» 

Ce fut d'Artagnan qui secoua la tête à son tour. 

«Mais enfin, que faire? dit le jeune homme. 

\speak  Le fait est qu'on ne peut rester ainsi avec une épée éternellement suspendue au-dessus de sa tête, dit Athos, et qu'il faut sortir de cette situation. 

\speak  Mais comment? 

\speak  Écoutez, tâchez de la rejoindre et d'avoir une explication avec elle; dites-lui: La paix ou la guerre! ma parole de gentilhomme de ne jamais rien dire de vous, de ne jamais rien faire contre vous; de votre côté serment solennel de rester neutre à mon égard: sinon, je vais trouver le chancelier, je vais trouver le roi, je vais trouver le bourreau, j'ameute la cour contre vous, je vous dénonce comme flétrie, je vous fais mettre en jugement, et si l'on vous absout, eh bien, je vous tue, foi de gentilhomme! au coin de quelque borne, comme je tuerais un chien enragé. 

\speak  J'aime assez ce moyen, dit d'Artagnan, mais comment la joindre? 

\speak  Le temps, cher ami, le temps amène l'occasion, l'occasion c'est la martingale de l'homme: plus on a engagé, plus l'on gagne quand on sait attendre. 

\speak  Oui, mais attendre entouré d'assassins et d'empoisonneurs\dots 

\speak  Bah! dit Athos, Dieu nous a gardés jusqu'à présent, Dieu nous gardera encore. 

\speak  Oui, nous; nous d'ailleurs, nous sommes des hommes, et, à tout prendre, c'est notre état de risquer notre vie: mais elle! ajouta-t-il à demi-voix. 

\speak  Qui elle? demanda Athos. 

\speak  Constance. 

\speak  Mme Bonacieux! ah! c'est juste, fit Athos; pauvre ami! j'oubliais que vous étiez amoureux. 

\speak  Eh bien, mais, dit Aramis, n'avez-vous pas vu par la lettre même que vous avez trouvée sur le misérable mort qu'elle était dans un couvent? On est très bien dans un couvent, et aussitôt le siège de La Rochelle terminé, je vous promets que pour mon compte\dots 

\speak  Bon! dit Athos, bon! oui, mon cher Aramis! nous savons que vos voeux tendent à la religion. 

\speak  Je ne suis mousquetaire que par intérim, dit humblement Aramis. 

\speak  Il paraît qu'il y a longtemps qu'il n'a reçu des nouvelles de sa maîtresse, dit tout bas Athos; mais ne faites pas attention, nous connaissons cela. 

\speak  Eh bien, dit Porthos, il me semble qu'il y aurait un moyen bien simple. 

\speak  Lequel? demanda d'Artagnan. 

\speak  Elle est dans un couvent, dites-vous? reprit Porthos. 

\speak  Oui. 

\speak  Eh bien, aussitôt le siège fini, nous l'enlevons de ce couvent. 

\speak  Mais encore faut-il savoir dans quel couvent elle est. 

\speak  C'est juste, dit Porthos. 

\speak  Mais, j'y pense, dit Athos, ne prétendez-vous pas, cher d'Artagnan, que c'est la reine qui a fait choix de ce couvent pour elle? 

\speak  Oui, je le crois du moins. 

\speak  Eh bien, mais Porthos nous aidera là-dedans. 

\speak  Et comment cela, s'il vous plaît? 

\speak  Mais par votre marquise, votre duchesse, votre princesse; elle doit avoir le bras long. 

\speak  Chut! dit Porthos en mettant un doigt sur ses lèvres, je la crois cardinaliste et elle ne doit rien savoir. 

\speak  Alors, dit Aramis, je me charge, moi, d'en avoir des nouvelles. 

\speak  Vous, Aramis, s'écrièrent les trois amis, vous, et comment cela? 

\speak  Par l'aumônier de la reine, avec lequel je suis fort lié\dots», dit Aramis en rougissant. 

Et sur cette assurance, les quatre amis, qui avaient achevé leur modeste repas, se séparèrent avec promesse de se revoir le soir même: d'Artagnan retourna aux Minimes, et les trois mousquetaires rejoignirent le quartier du roi, où ils avaient à faire préparer leur logis. 