%!TeX root=../musketeersfr.tex 

\chapter{Georges Villiers, Duc De Buckingham}

\lettrine{M}{adame} Bonacieux et le duc entrèrent au Louvre sans difficulté; Mme Bonacieux était connue pour appartenir à la reine; le duc portait l'uniforme des mousquetaires de M. de Tréville, qui, comme nous l'avons dit, était de garde ce soir-là. D'ailleurs Germain était dans les intérêts de la reine, et si quelque chose arrivait, Mme Bonacieux serait accusée d'avoir introduit son amant au Louvre, voilà tout; elle prenait sur elle le crime: sa réputation était perdue, il est vrai, mais de quelle valeur était dans le monde la réputation d'une petite mercière? 

Une fois entrés dans l'intérieur de la cour, le duc et la jeune femme suivirent le pied de la muraille pendant l'espace d'environ vingt-cinq pas; cet espace parcouru, Mme Bonacieux poussa une petite porte de service, ouverte le jour, mais ordinairement fermée la nuit; la porte céda; tous deux entrèrent et se trouvèrent dans l'obscurité, mais Mme Bonacieux connaissait tous les tours et détours de cette partie du Louvre, destinée aux gens de la suite. Elle referma les portes derrière elle, prit le duc par la main, fit quelques pas en tâtonnant, saisit une rampe, toucha du pied un degré, et commença de monter un escalier: le duc compta deux étages. Alors elle prit à droite, suivit un long corridor, redescendit un étage, fit quelques pas encore, introduisit une clef dans une serrure, ouvrit une porte et poussa le duc dans un appartement éclairé seulement par une lampe de nuit, en disant: «Restez ici, Milord duc, on va venir.» Puis elle sortit par la même porte, qu'elle ferma à la clef, de sorte que le duc se trouva littéralement prisonnier. 

Cependant, tout isolé qu'il se trouvait, il faut le dire, le duc de Buckingham n'éprouva pas un instant de crainte; un des côtés saillants de son caractère était la recherche de l'aventure et l'amour du romanesque. Brave, hardi, entreprenant, ce n'était pas la première fois qu'il risquait sa vie dans de pareilles tentatives; il avait appris que ce prétendu message d'Anne d'Autriche, sur la foi duquel il était venu à Paris, était un piège, et au lieu de regagner l'Angleterre, il avait, abusant de la position qu'on lui avait faite, déclaré à la reine qu'il ne partirait pas sans l'avoir vue. La reine avait positivement refusé d'abord, puis enfin elle avait craint que le duc, exaspéré, ne fît quelque folie. Déjà elle était décidée à le recevoir et à le supplier de partir aussitôt, lorsque, le soir même de cette décision, Mme Bonacieux, qui était chargée d'aller chercher le duc et de le conduire au Louvre, fut enlevée. Pendant deux jours on ignora complètement ce qu'elle était devenue, et tout resta en suspens. Mais une fois libre, une fois remise en rapport avec La Porte, les choses avaient repris leur cours, et elle venait d'accomplir la périlleuse entreprise que, sans son arrestation, elle eût exécutée trois jours plus tôt. 

Buckingham, resté seul, s'approcha d'une glace. Cet habit de mousquetaire lui allait à merveille. 

À trente-cinq ans qu'il avait alors, il passait à juste titre pour le plus beau gentilhomme et pour le plus élégant cavalier de France et d'Angleterre. 

Favori de deux rois, riche à millions, tout-puissant dans un royaume qu'il bouleversait à sa fantaisie et calmait à son caprice, Georges Villiers, duc de Buckingham, avait entrepris une de ces existences fabuleuses qui restent dans le cours des siècles comme un étonnement pour la postérité. 

Aussi, sûr de lui-même, convaincu de sa puissance, certain que les lois qui régissent les autres hommes ne pouvaient l'atteindre, allait-il droit au but qu'il s'était fixé, ce but fût-il si élevé et si éblouissant que c'eût été folie pour un autre que de l'envisager seulement. C'est ainsi qu'il était arrivé à s'approcher plusieurs fois de la belle et fière Anne d'Autriche et à s'en faire aimer, à force d'éblouissement. 

Georges Villiers se plaça donc devant une glace, comme nous l'avons dit, rendit à sa belle chevelure blonde les ondulations que le poids de son chapeau lui avait fait perdre, retroussa sa moustache, et le cœur tout gonflé de joie, heureux et fier de toucher au moment qu'il avait si longtemps désiré, se sourit à lui-même d'orgueil et d'espoir. 

En ce moment, une porte cachée dans la tapisserie s'ouvrit et une femme apparut. Buckingham vit cette apparition dans la glace; il jeta un cri, c'était la reine! 

Anne d'Autriche avait alors vingt-six ou vingt-sept ans, c'est-à-dire qu'elle se trouvait dans tout l'éclat de sa beauté. 

Sa démarche était celle d'une reine ou d'une déesse; ses yeux, qui jetaient des reflets d'émeraude, étaient parfaitement beaux, et tout à la fois pleins de douceur et de majesté. 

Sa bouche était petite et vermeille, et quoique sa lèvre inférieure, comme celle des princes de la maison d'Autriche, avançât légèrement sur l'autre, elle était éminemment gracieuse dans le sourire, mais aussi profondément dédaigneuse dans le mépris. 

Sa peau était citée pour sa douceur et son velouté, sa main et ses bras étaient d'une beauté surprenante, et tous les poètes du temps les chantaient comme incomparables. 

Enfin ses cheveux, qui, de blonds qu'ils étaient dans sa jeunesse, étaient devenus châtains, et qu'elle portait frisés très clair et avec beaucoup de poudre, encadraient admirablement son visage, auquel le censeur le plus rigide n'eût pu souhaiter qu'un peu moins de rouge, et le statuaire le plus exigeant qu'un peu plus de finesse dans le nez. 

Buckingham resta un instant ébloui; jamais Anne d'Autriche ne lui était apparue aussi belle, au milieu des bals, des fêtes, des carrousels, qu'elle lui apparut en ce moment, vêtue d'une simple robe de satin blanc et accompagnée de doña Estefania, la seule de ses femmes espagnoles qui n'eût pas été chassée par la jalousie du roi et par les persécutions de Richelieu. 

Anne d'Autriche fit deux pas en avant; Buckingham se précipita à ses genoux, et avant que la reine eût pu l'en empêcher, il baisa le bas de sa robe. 

«Duc, vous savez déjà que ce n'est pas moi qui vous ai fait écrire. 

\speak  Oh! oui, madame, oui, Votre Majesté, s'écria le duc; je sais que j'ai été un fou, un insensé de croire que la neige s'animerait, que le marbre s'échaufferait; mais, que voulez-vous, quand on aime, on croit facilement à l'amour; d'ailleurs je n'ai pas tout perdu à ce voyage, puisque je vous vois. 

\speak  Oui, répondit Anne, mais vous savez pourquoi et comment je vous vois, Milord. Je vous vois par pitié pour vous-même; je vous vois parce qu'insensible à toutes mes peines, vous vous êtes obstiné à rester dans une ville où, en restant, vous courez risque de la vie et me faites courir risque de mon honneur; je vous vois pour vous dire que tout nous sépare, les profondeurs de la mer, l'inimitié des royaumes, la sainteté des serments. Il est sacrilège de lutter contre tant de choses, Milord. Je vous vois enfin pour vous dire qu'il ne faut plus nous voir. 

\speak  Parlez, madame; parlez, reine, dit Buckingham; la douceur de votre voix couvre la dureté de vos paroles. Vous parlez de sacrilège! mais le sacrilège est dans la séparation des cœurs que Dieu avait formés l'un pour l'autre. 

\speak  Milord, s'écria la reine, vous oubliez que je ne vous ai jamais dit que je vous aimais. 

\speak  Mais vous ne m'avez jamais dit non plus que vous ne m'aimiez point; et vraiment, me dire de semblables paroles, ce serait de la part de Votre Majesté une trop grande ingratitude. Car, dites-moi, où trouvez-vous un amour pareil au mien, un amour que ni le temps, ni l'absence, ni le désespoir ne peuvent éteindre; un amour qui se contente d'un ruban égaré, d'un regard perdu, d'une parole échappée? 

«Il y a trois ans, madame, que je vous ai vue pour la première fois, et depuis trois ans je vous aime ainsi. 

«Voulez-vous que je vous dise comment vous étiez vêtue la première fois que je vous vis? voulez-vous que je détaille chacun des ornements de votre toilette? Tenez, je vous vois encore: vous étiez assise sur des carreaux, à la mode d'Espagne; vous aviez une robe de satin vert avec des broderies d'or et d'argent; des manches pendantes et renouées sur vos beaux bras, sur ces bras admirables, avec de gros diamants; vous aviez une fraise fermée, un petit bonnet sur votre tête, de la couleur de votre robe, et sur ce bonnet une plume de héron. 

«Oh! tenez, tenez, je ferme les yeux, et je vous vois telle que vous étiez alors; je les rouvre, et je vous vois telle que vous êtes maintenant, c'est-à-dire cent fois plus belle encore! 

\speak  Quelle folie! murmura Anne d'Autriche, qui n'avait pas le courage d'en vouloir au duc d'avoir si bien conservé son portrait dans son cœur; quelle folie de nourrir une passion inutile avec de pareils souvenirs! 

\speak  Et avec quoi voulez-vous donc que je vive? je n'ai que des souvenirs, moi. C'est mon bonheur, mon trésor, mon espérance. Chaque fois que je vous vois, c'est un diamant de plus que je renferme dans l'écrin de mon cœur. Celui-ci est le quatrième que vous laissez tomber et que je ramasse; car en trois ans, madame, je ne vous ai vue que quatre fois: cette première que je viens de vous dire, la seconde chez Mme de Chevreuse, la troisième dans les jardins d'Amiens. 

\speak  Duc, dit la reine en rougissant, ne parlez pas de cette soirée. 

\speak  Oh! parlons-en, au contraire, madame, parlons-en: c'est la soirée heureuse et rayonnante de ma vie. Vous rappelez-vous la belle nuit qu'il faisait? Comme l'air était doux et parfumé, comme le ciel était bleu et tout émaillé d'étoiles! Ah! cette fois, madame, j'avais pu être un instant seul avec vous; cette fois, vous étiez prête à tout me dire, l'isolement de votre vie, les chagrins de votre cœur. Vous étiez appuyée à mon bras, tenez, à celui-ci. Je sentais, en inclinant ma tête à votre côté, vos beaux cheveux effleurer mon visage, et chaque fois qu'ils l'effleuraient je frissonnais de la tête aux pieds. Oh! reine, reine! oh! vous ne savez pas tout ce qu'il y a de félicités du ciel, de joies du paradis enfermées dans un moment pareil. Tenez, mes biens, ma fortune, ma gloire, tout ce qu'il me reste de jours à vivre, pour un pareil instant et pour une semblable nuit! car cette nuit-là, madame, cette nuit-là vous m'aimiez, je vous le jure. 

\speak  Milord, il est possible, oui, que l'influence du lieu, que le charme de cette belle soirée, que la fascination de votre regard, que ces mille circonstances enfin qui se réunissent parfois pour perdre une femme se soient groupées autour de moi dans cette fatale soirée; mais vous l'avez vu, Milord, la reine est venue au secours de la femme qui faiblissait: au premier mot que vous avez osé dire, à la première hardiesse à laquelle j'ai eu à répondre, j'ai appelé. 

\speak  Oh! oui, oui, cela est vrai, et un autre amour que le mien aurait succombé à cette épreuve; mais mon amour, à moi, en est sorti plus ardent et plus éternel. Vous avez cru me fuir en revenant à Paris, vous avez cru que je n'oserais quitter le trésor sur lequel mon maître m'avait chargé de veiller. Ah! que m'importent à moi tous les trésors du monde et tous les rois de la terre! Huit jours après, j'étais de retour, madame. Cette fois, vous n'avez rien eu à me dire: j'avais risqué ma faveur, ma vie, pour vous voir une seconde, je n'ai pas même touché votre main, et vous m'avez pardonné en me voyant si soumis et si repentant. 

\speak  Oui, mais la calomnie s'est emparée de toutes ces folies dans lesquelles je n'étais pour rien, vous le savez bien, Milord. Le roi, excité par M. le cardinal, a fait un éclat terrible: Mme de Vernet a été chassée, Putange exilé, Mme de Chevreuse est tombée en défaveur, et lorsque vous avez voulu revenir comme ambassadeur en France, le roi lui-même, souvenez-vous-en, Milord, le roi lui-même s'y est opposé. 

\speak  Oui, et la France va payer d'une guerre le refus de son roi. Je ne puis plus vous voir, madame; eh bien, je veux chaque jour que vous entendiez parler de moi. 

«Quel but pensez-vous qu'aient eu cette expédition de Ré et cette ligue avec les protestants de La Rochelle que je projette? Le plaisir de vous voir! 

«Je n'ai pas l'espoir de pénétrer à main armée jusqu'à Paris, je le sais bien: mais cette guerre pourra amener une paix, cette paix nécessitera un négociateur, ce négociateur ce sera moi. On n'osera plus me refuser alors, et je reviendrai à Paris, et je vous reverrai, et je serai heureux un instant. Des milliers d'hommes, il est vrai, auront payé mon bonheur de leur vie; mais que m'importera, à moi, pourvu que je vous revoie! Tout cela est peut-être bien fou, peut-être bien insensé; mais, dites-moi, quelle femme a un amant plus amoureux? quelle reine a eu un serviteur plus ardent? 

\speak  Milord, Milord, vous invoquez pour votre défense des choses qui vous accusent encore; Milord, toutes ces preuves d'amour que vous voulez me donner sont presque des crimes. 

\speak  Parce que vous ne m'aimez pas, madame: si vous m'aimiez, vous verriez tout cela autrement, si vous m'aimiez, oh! mais, si vous m'aimiez, ce serait trop de bonheur et je deviendrais fou. Ah! Mme de Chevreuse dont vous parliez tout à l'heure, Mme de Chevreuse a été moins cruelle que vous; Holland l'a aimée, et elle a répondu à son amour. 

\speak  Mme de Chevreuse n'était pas reine, murmura Anne d'Autriche, vaincue malgré elle par l'expression d'un amour si profond. 

\speak  Vous m'aimeriez donc si vous ne l'étiez pas, vous, madame, dites, vous m'aimeriez donc? Je puis donc croire que c'est la dignité seule de votre rang qui vous fait cruelle pour moi; je puis donc croire que si vous eussiez été Mme de Chevreuse, le pauvre Buckingham aurait pu espérer? Merci de ces douces paroles, ô ma belle Majesté, cent fois merci. 

\speak  Ah! Milord, vous avez mal entendu, mal interprété; je n'ai pas voulu dire\dots 

\speak  Silence! Silence! dit le duc, si je suis heureux d'une erreur, n'ayez pas la cruauté de me l'enlever. Vous l'avez dit vous-même, on m'a attiré dans un piège, j'y laisserai ma vie peut-être, car, tenez, c'est étrange, depuis quelque temps j'ai des pressentiments que je vais mourir.» Et le duc sourit d'un sourire triste et charmant à la fois. 

«Oh! mon Dieu! s'écria Anne d'Autriche avec un accent d'effroi qui prouvait quel intérêt plus grand qu'elle ne le voulait dire elle prenait au duc. 

\speak  Je ne vous dis point cela pour vous effrayer, madame, non; c'est même ridicule ce que je vous dis, et croyez que je ne me préoccupe point de pareils rêves. Mais ce mot que vous venez de dire, cette espérance que vous m'avez presque donnée, aura tout payé, fût-ce même ma vie. 

\speak  Eh bien, dit Anne d'Autriche, moi aussi, duc, moi, j'ai des pressentiments, moi aussi j'ai des rêves. J'ai songé que je vous voyais couché sanglant, frappé d'une blessure. 

\speak  Au côté gauche, n'est-ce pas, avec un couteau? interrompit Buckingham. 

\speak  Oui, c'est cela, Milord, c'est cela, au côté gauche avec un couteau. Qui a pu vous dire que j'avais fait ce rêve? Je ne l'ai confié qu'à Dieu, et encore dans mes prières. 

\speak  Je n'en veux pas davantage, et vous m'aimez, madame, c'est bien. 

\speak  Je vous aime, moi? 

\speak  Oui, vous. Dieu vous enverrait-il les mêmes rêves qu'à moi, si vous ne m'aimiez pas? Aurions-nous les mêmes pressentiments, si nos deux existences ne se touchaient pas par le cœur? Vous m'aimez, ô reine, et vous me pleurerez? 

\speak  Oh! mon Dieu! mon Dieu! s'écria Anne d'Autriche, c'est plus que je n'en puis supporter. Tenez, duc, au nom du Ciel, partez, retirez-vous; je ne sais si je vous aime, ou si je ne vous aime pas; mais ce que je sais, c'est que je ne serai point parjure. Prenez donc pitié de moi, et partez. Oh! si vous êtes frappé en France, si vous mourez en France, si je pouvais supposer que votre amour pour moi fût cause de votre mort, je ne me consolerais jamais, j'en deviendrais folle. Partez donc, partez, je vous en supplie. 

\speak  Oh! que vous êtes belle ainsi! Oh! que je vous aime! dit Buckingham. 

\speak  Partez! partez! je vous en supplie, et revenez plus tard; revenez comme ambassadeur, revenez comme ministre, revenez entouré de gardes qui vous défendront, de serviteurs qui veilleront sur vous, et alors je ne craindrai plus pour vos jours, et j'aurai du bonheur à vous revoir. 

\speak  Oh! est-ce bien vrai ce que vous me dites? 

\speak  Oui\dots 

\speak  Eh bien, un gage de votre indulgence, un objet qui vienne de vous et qui me rappelle que je n'ai point fait un rêve; quelque chose que vous ayez porté et que je puisse porter à mon tour, une bague, un collier, une chaîne. 

\speak  Et partirez-vous, partirez-vous, si je vous donne ce que vous me demandez? 

\speak  Oui. 

\speak  À l'instant même? 

\speak  Oui. 

\speak  Vous quitterez la France, vous retournerez en Angleterre? 

\speak  Oui, je vous le jure! 

\speak  Attendez, alors, attendez.» 

Et Anne d'Autriche rentra dans son appartement et en sortit presque aussitôt, tenant à la main un petit coffret en bois de rose à son chiffre, tout incrusté d'or. 

«Tenez, Milord duc, tenez, dit-elle, gardez cela en mémoire de moi.» 

Buckingham prit le coffret et tomba une seconde fois à genoux. 

«Vous m'avez promis de partir, dit la reine. 

\speak  Et je tiens ma parole. Votre main, votre main, madame, et je pars.» 

Anne d'Autriche tendit sa main en fermant les yeux et en s'appuyant de l'autre sur Estefania, car elle sentait que les forces allaient lui manquer. 

Buckingham appuya avec passion ses lèvres sur cette belle main, puis se relevant: 

«Avant six mois, dit-il, si je ne suis pas mort, je vous aurai revue, madame, dussé-je bouleverser le monde pour cela.» 

Et, fidèle à la promesse qu'il avait faite, il s'élança hors de l'appartement. 

Dans le corridor, il rencontra Mme Bonacieux qui l'attendait, et qui, avec les mêmes précautions et le même bonheur, le reconduisit hors du Louvre.