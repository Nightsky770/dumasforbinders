%!TeX root=../musketeersfr.tex 

\chapter{Quatrième Journée De Captivité}

\lettrine{L}{e} lendemain, lorsque Felton entra chez Milady, il la trouva debout, montée sur un fauteuil, tenant entre ses mains une corde tissée à l'aide de quelques mouchoirs de batiste déchirés en lanières tressées les unes avec les autres et attachées bout à bout; au bruit que fit Felton en ouvrant la porte, Milady sauta légèrement à bas de son fauteuil, et essaya de cacher derrière elle cette corde improvisée, qu'elle tenait à la main. 

Le jeune homme était plus pâle encore que d'habitude, et ses yeux rougis par l'insomnie indiquaient qu'il avait passé une nuit fiévreuse. 

Cependant son front était armé d'une sérénité plus austère que jamais. 

Il s'avança lentement vers Milady, qui s'était assise, et prenant un bout de la tresse meurtrière que par mégarde ou à dessein peut-être elle avait laissée passer: 

«Qu'est-ce que cela, madame? demanda-t-il froidement. 

\speak  Cela, rien, dit Milady en souriant avec cette expression douloureuse qu'elle savait si bien donner à son sourire, l'ennui est l'ennemi mortel des prisonniers, je m'ennuyais et je me suis amusée à tresser cette corde.» 

Felton porta les yeux vers le point du mur de l'appartement devant lequel il avait trouvé Milady debout sur le fauteuil où elle était assise maintenant, et au-dessus de sa tête il aperçut un crampon doré, scellé dans le mur, et qui servait à accrocher soit des hardes, soit des armes. 

Il tressaillit, et la prisonnière vit ce tressaillement; car, quoiqu'elle eût les yeux baissés, rien ne lui échappait. 

«Et que faisiez-vous, debout sur ce fauteuil? demanda-t-il. 

\speak  Que vous importe? répondit Milady. 

\speak  Mais, reprit Felton, je désire le savoir. 

\speak  Ne m'interrogez pas, dit la prisonnière, vous savez bien qu'à nous autres, véritables chrétiens, il nous est défendu de mentir. 

\speak  Eh bien, dit Felton, je vais vous le dire, ce que vous faisiez, ou plutôt ce que vous alliez faire, vous alliez achever l'oeuvre fatale que vous nourrissez dans votre esprit: songez-y, madame, si notre Dieu défend le mensonge, il défend bien plus sévèrement encore le suicide. 

\speak  Quand Dieu voit une de ses créatures persécutée injustement, placée entre le suicide et le déshonneur, croyez-moi, monsieur, répondit Milady d'un ton de profonde conviction, Dieu lui pardonne le suicide: car, alors, le suicide c'est le martyre. 

\speak  Vous en dites trop ou trop peu; parlez, madame, au nom du Ciel, expliquez-vous. 

\speak  Que je vous raconte mes malheurs, pour que vous les traitiez de fables; que je vous dise mes projets, pour que vous alliez les dénoncer à mon persécuteur: non, monsieur; d'ailleurs, que vous importe la vie ou la mort d'une malheureuse condamnée? vous ne répondez que de mon corps, n'est-ce pas? et pourvu que vous représentiez un cadavre, qu'il soit reconnu pour le mien, on ne vous en demandera pas davantage, et peut-être, même, aurez-vous double récompense. 

\speak  Moi, madame, moi! s'écria Felton, supposer que j'accepterais jamais le prix de votre vie; oh! vous ne pensez pas ce que vous dites. 

\speak  Laissez-moi faire, Felton, laissez-moi faire, dit Milady en s'exaltant, tout soldat doit être ambitieux, n'est-ce pas? vous êtes lieutenant, eh bien, vous suivrez mon convoi avec le grade de capitaine. 

\speak  Mais que vous ai-je donc fait, dit Felton ébranlé, pour que vous me chargiez d'une pareille responsabilité devant les hommes et devant Dieu? Dans quelques jours vous allez être loin d'ici, madame, votre vie ne sera plus sous ma garde, et, ajouta-t-il avec un soupir, alors vous en ferez ce que vous voudrez. 

\speak  Ainsi, s'écria Milady comme si elle ne pouvait résister à une sainte indignation, vous, un homme pieux, vous que l'on appelle un juste, vous ne demandez qu'une chose: c'est de n'être point inculpé, inquiété pour ma mort! 

\speak  Je dois veiller sur votre vie, madame, et j'y veillerai. 

\speak  Mais comprenez-vous la mission que vous remplissez? cruelle déjà si j'étais coupable, quel nom lui donnerez-vous, quel nom le Seigneur lui donnera-t-il, si je suis innocente? 

\speak  Je suis soldat, madame, et j'accomplis les ordres que j'ai reçus. 

\speak  Croyez-vous qu'au jour du jugement dernier Dieu séparera les bourreaux aveugles des juges iniques? vous ne voulez pas que je tue mon corps, et vous vous faites l'agent de celui qui veut tuer mon âme! 

\speak  Mais, je vous le répète, reprit Felton ébranlé, aucun danger ne vous menace, et je réponds de Lord de Winter comme de moi-même. 

\speak  Insensé! s'écria Milady, pauvre insensé, qui ose répondre d'un autre homme quand les plus sages, quand les plus grands selon Dieu hésitent à répondre d'eux-mêmes, et qui se range du parti le plus fort et le plus heureux, pour accabler la plus faible et la plus malheureuse! 

\speak  Impossible, madame, impossible, murmura Felton, qui sentait au fond du cœur la justesse de cet argument: prisonnière, vous ne recouvrerez pas par moi la liberté, vivante, vous ne perdrez pas par moi la vie. 

\speak  Oui, s'écria Milady, mais je perdrai ce qui m'est bien plus cher que la vie, je perdrai l'honneur, Felton; et c'est vous, vous que je ferai responsable devant Dieu et devant les hommes de ma honte et de mon infamie.» 

Cette fois Felton, tout impassible qu'il était ou qu'il faisait semblant d'être, ne put résister à l'influence secrète qui s'était déjà emparée de lui: voir cette femme si belle, blanche comme la plus candide vision, la voir tour à tour éplorée et menaçante, subir à la fois l'ascendant de la douleur et de la beauté, c'était trop pour un visionnaire, c'était trop pour un cerveau miné par les rêves ardents de la foi extatique, c'était trop pour un cœur corrodé à la fois par l'amour du Ciel qui brûle, par la haine des hommes qui dévore. 

Milady vit le trouble, elle sentait par intuition la flamme des passions opposées qui brûlaient avec le sang dans les veines du jeune fanatique; et, pareille à un général habile qui, voyant l'ennemi prêt à reculer, marche sur lui en poussant un cri de victoire, elle se leva, belle comme une prêtresse antique, inspirée comme une vierge chrétienne et, le bras étendu, le col découvert, les cheveux épars retenant d'une main sa robe pudiquement ramenée sur sa poitrine, le regard illuminé de ce feu qui avait déjà porté le désordre dans les sens du jeune puritain, elle marcha vers lui, s'écriant sur un air véhément, de sa voix si douce, à laquelle, dans l'occasion, elle donnait un accent terrible: 
\begin{verse}
Livre à Baal sa victime.\\
Jette aux lions le martyr:\\
Dieu te fera repentir!\dots\\
Je crie à lui de l'abîme. 
\end{verse}

Felton s'arrêta sous cette étrange apostrophe, et comme pétrifié. 

«Qui êtes-vous, qui êtes-vous? s'écria-t-il en joignant les mains; êtes-vous une envoyée de Dieu, êtes-vous un ministre des enfers, êtes-vous ange ou démon, vous appelez-vous Eloa ou Astarté? 

\speak  Ne m'as-tu pas reconnue, Felton? Je ne suis ni un ange, ni un démon, je suis une fille de la terre, je suis une soeur de ta croyance, voilà tout. 

\speak  Oui! oui! dit Felton, je doutais encore, mais maintenant je crois. 

\speak  Tu crois, et cependant tu es le complice de cet enfant de Bélial qu'on appelle Lord de Winter! Tu crois, et cependant tu me laisses aux mains de mes ennemis, de l'ennemi de l'Angleterre, de l'ennemi de Dieu? Tu crois, et cependant tu me livres à celui qui remplit et souille le monde de ses hérésies et de ses débauches, à cet infâme Sardanapale que les aveugles nomment le duc de Buckingham et que les croyants appellent l'Antéchrist. 

\speak  Moi, vous livrer à Buckingham! moi! que dites-vous là? 

\speak  Ils ont des yeux, s'écria Milady, et ils ne verront pas; ils ont des oreilles, et ils n'entendront point. 

\speak  Oui, oui, dit Felton en passant ses mains sur son front couvert de sueur, comme pour en arracher son dernier doute; oui, je reconnais la voix qui me parle dans mes rêves; oui, je reconnais les traits de l'ange qui m'apparaît chaque nuit, criant à mon âme qui ne peut dormir: “Frappe, sauve l'Angleterre, sauve-toi, car tu mourras sans avoir désarmé Dieu!” Parlez, parlez! s'écria Felton, je puis vous comprendre à présent.» 

Un éclair de joie terrible, mais rapide comme la pensée, jaillit des yeux de Milady. 

Si fugitive qu'eût été cette lueur homicide, Felton la vit et tressaillit comme si cette lueur eût éclairé les abîmes du cœur de cette femme. 

Felton se rappela tout à coup les avertissements de Lord de Winter, les séductions de Milady, ses premières tentatives lors de son arrivée; il recula d'un pas et baissa la tête, mais sans cesser de la regarder: comme si, fasciné par cette étrange créature, ses yeux ne pouvaient se détacher de ses yeux. 

Milady n'était point femme à se méprendre au sens de cette hésitation. Sous ses émotions apparentes, son sang-froid glacé ne l'abandonnait point. Avant que Felton lui eût répondu et qu'elle fût forcée de reprendre cette conversation si difficile à soutenir sur le même accent d'exaltation, elle laissa retomber ses mains, et, comme si la faiblesse de la femme reprenait le dessus sur l'enthousiasme de l'inspirée: 

«Mais, non, dit-elle, ce n'est pas à moi d'être la Judith qui délivrera Béthulie de cet Holopherne. Le glaive de l'éternel est trop lourd pour mon bras. Laissez-moi donc fuir le déshonneur par la mort, laissez-moi me réfugier dans le martyre. Je ne vous demande ni la liberté, comme ferait une coupable, ni la vengeance, comme ferait une païenne. Laissez-moi mourir, voilà tout. Je vous supplie, je vous implore à genoux; laissez-moi mourir, et mon dernier soupir sera une bénédiction pour mon sauveur.» 

À cette voix douce et suppliante, à ce regard timide et abattu, Felton se rapprocha. Peu à peu l'enchanteresse avait revêtu cette parure magique qu'elle reprenait et quittait à volonté, c'est-à-dire la beauté, la douceur, les larmes et surtout l'irrésistible attrait de la volupté mystique, la plus dévorante des voluptés. 

«Hélas! dit Felton, je ne puis qu'une chose, vous plaindre si vous me prouvez que vous êtes une victime! Mais Lord de Winter a de cruels griefs contre vous. Vous êtes chrétienne, vous êtes ma soeur en religion; je me sens entraîné vers vous, moi qui n'ai aimé que mon bienfaiteur, moi qui n'ai trouvé dans la vie que des traîtres et des impies. Mais vous, madame, vous si belle en réalité, vous si pure en apparence, pour que Lord de Winter vous poursuive ainsi, vous avez donc commis des iniquités? 

\speak  Ils ont des yeux, répéta Milady avec un accent d'indicible douleur, et ils ne verront pas; ils ont des oreilles, et ils n'entendront point. 

\speak  Mais, alors, s'écria le jeune officier, parlez, parlez donc! 

\speak  Vous confier ma honte! s'écria Milady avec le rouge de la pudeur au visage, car souvent le crime de l'un est la honte de l'autre; vous confier ma honte, à vous homme, moi femme! Oh! continua-t-elle en ramenant pudiquement sa main sur ses beaux yeux, oh! jamais, jamais je ne pourrai! 

\speak  À moi, à un frère!» s'écria Felton. 

Milady le regarda longtemps avec une expression que le jeune officier prit pour du doute, et qui cependant n'était que de l'observation et surtout la volonté de fasciner. 

Felton, à son tour suppliant, joignit les mains. 

«Eh bien, dit Milady, je me fie à mon frère, j'oserai!» 

En ce moment, on entendit le pas de Lord de Winter; mais, cette fois le terrible beau-frère de Milady ne se contenta point, comme il avait fait la veille, de passer devant la porte et de s'éloigner, il s'arrêta, échangea deux mots avec la sentinelle, puis la porte s'ouvrit et il parut. 

Pendant ces deux mots échangés, Felton s'était reculé vivement, et lorsque Lord de Winter entra, il était à quelques pas de la prisonnière. 

Le baron entra lentement, et porta son regard scrutateur de la prisonnière au jeune officier: 

«Voilà bien longtemps, John, dit-il, que vous êtes ici; cette femme vous a-t-elle raconté ses crimes? alors je comprends la durée de l'entretien.» 

Felton tressaillit, et Milady sentit qu'elle était perdue si elle ne venait au secours du puritain décontenancé. 

«Ah! vous craignez que votre prisonnière ne vous échappe! dit-elle, eh bien, demandez à votre digne geôlier quelle grâce, à l'instant même, je sollicitais de lui. 

\speak  Vous demandiez une grâce? dit le baron soupçonneux. 

\speak  Oui, Milord, reprit le jeune homme confus. 

\speak  Et quelle grâce, voyons? demanda Lord de Winter. 

\speak  Un couteau qu'elle me rendra par le guichet, une minute après l'avoir reçu, répondit Felton. 

\speak  Il y a donc quelqu'un de caché ici que cette gracieuse personne veuille égorger? reprit Lord de Winter de sa voix railleuse et méprisante. 

\speak  Il y a moi, répondit Milady. 

\speak  Je vous ai donné le choix entre l'Amérique et Tyburn, reprit Lord de Winter, choisissez Tyburn, Milady: la corde est, croyez-moi, encore plus sûre que le couteau.» 

Felton pâlit et fit un pas en avant, en songeant qu'au moment où il était entré, Milady tenait une corde. 

«Vous avez raison, dit celle-ci, et j'y avais déjà pensé; puis elle ajouta d'une voix sourde: j'y penserai encore.» 

Felton sentit courir un frisson jusque dans la moelle de ses os; probablement Lord de Winter aperçut ce mouvement. 

«Méfie-toi, John, dit-il, John, mon ami, je me suis reposé sur toi, prends garde! Je t'ai prévenu! D'ailleurs, aie bon courage, mon enfant, dans trois jours nous serons délivrés de cette créature, et où je l'envoie, elle ne nuira plus à personne. 

\speak  Vous l'entendez!» s'écria Milady avec éclat, de façon que le baron crût qu'elle s'adressait au Ciel et que Felton comprît que c'était à lui. 

Felton baissa la tête et rêva. 

Le baron prit l'officier par le bras en tournant la tête sur son épaule, afin de ne pas perdre Milady de vue jusqu'à ce qu'il fût sorti. 

«Allons, allons, dit la prisonnière lorsque la porte se fut refermée, je ne suis pas encore si avancée que je le croyais. Winter a changé sa sottise ordinaire en une prudence inconnue; ce que c'est que le désir de la vengeance, et comme ce désir forme l'homme! Quant à Felton, il hésite. Ah! ce n'est pas un homme comme ce d'Artagnan maudit. Un puritain n'adore que les vierges, et il les adore en joignant les mains. Un mousquetaire aime les femmes, et il les aime en joignant les bras.» 

Cependant Milady attendit avec impatience, car elle se doutait bien que la journée ne se passerait pas sans qu'elle revit Felton. Enfin, une heure après la scène que nous venons de raconter, elle entendit que l'on parlait bas à la porte, puis bientôt la porte s'ouvrit, et elle reconnut Felton. 

Le jeune homme s'avança rapidement dans la chambre en laissant la porte ouverte derrière lui et en faisant signe à Milady de se taire; il avait le visage bouleversé. 

«Que me voulez-vous? dit-elle. 

\speak  Écoutez, répondit Felton à voix basse, je viens d'éloigner la sentinelle pour pouvoir rester ici sans qu'on sache que je suis venu, pour vous parler sans qu'on puisse entendre ce que je vous dis. Le baron vient de me raconter une histoire effroyable.» 

Milady prit son sourire de victime résignée, et secoua la tête. 

«Ou vous êtes un démon, continua Felton, ou le baron, mon bienfaiteur, mon père, est un monstre. Je vous connais depuis quatre jours, je l'aime depuis dix ans, lui; je puis donc hésiter entre vous deux: ne vous effrayez pas de ce que je vous dis, j'ai besoin d'être convaincu. Cette nuit, après minuit, je viendrai vous voir, vous me convaincrez. 

\speak  Non, Felton, non, mon frère, dit-elle, le sacrifice est trop grand, et je sens qu'il vous coûte. Non, je suis perdue, ne vous perdez pas avec moi. Ma mort sera bien plus éloquente que ma vie, et le silence du cadavre vous convaincra bien mieux que les paroles de la prisonnière. 

\speak  Taisez-vous, madame, s'écria Felton, et ne me parlez pas ainsi; je suis venu pour que vous me promettiez sur l'honneur, pour que vous me juriez sur ce que vous avez de plus sacré, que vous n'attenterez pas à votre vie. 

\speak  Je ne veux pas promettre, dit Milady, car personne plus que moi n'a le respect du serment, et, si je promettais, il me faudrait tenir. 

\speak  Eh bien, dit Felton, engagez-vous seulement jusqu'au moment où vous m'aurez revu. Si, lorsque vous m'aurez revu, vous persistez encore, eh bien, alors, vous serez libre, et moi-même je vous donnerai l'arme que vous m'avez demandée. 

\speak  Eh bien, dit Milady, pour vous j'attendrai. 

\speak  Jurez-le. 

\speak  Je le jure par notre Dieu. Êtes-vous content? 

\speak  Bien, dit Felton, à cette nuit!» 

Et il s'élança hors de l'appartement, referma la porte, et attendit en dehors, la demi-pique du soldat à la main, comme s'il eût monté la garde à sa place. 

Le soldat revenu, Felton lui rendit son arme. 

Alors, à travers le guichet dont elle s'était rapprochée, Milady vit le jeune homme se signer avec une ferveur délirante et s'en aller par le corridor avec un transport de joie. 

Quant à elle, elle revint à sa place, un sourire de sauvage mépris sur les lèvres, et elle répéta en blasphémant ce nom terrible de Dieu, par lequel elle avait juré sans jamais avoir appris à le connaître. 

«Mon Dieu! dit-elle, fanatique insensé! mon Dieu! c'est moi, moi et celui qui m'aidera à me venger.»