%!TeX root=../musketeersfr.tex 

\chapter{Le Rendez-Vous} 
	
\lettrine{D}{'Artagnan} revint chez lui tout courant, et quoiqu'il fût plus de trois heures du matin, et qu'il eût les plus méchants quartiers de Paris à traverser, il ne fit aucune mauvaise rencontre. On sait qu'il y a un dieu pour les ivrognes et les amoureux. 

Il trouva la porte de son allée entrouverte, monta son escalier, et frappa doucement et d'une façon convenue entre lui et son laquais. Planchet, qu'il avait renvoyé deux heures auparavant de l'Hôtel de Ville en lui recommandant de l'attendre, vint lui ouvrir la porte. 

«Quelqu'un a-t-il apporté une lettre pour moi? demanda vivement d'Artagnan. 

\speak  Personne n'a apporté de lettre, monsieur, répondit Planchet; mais il y en a une qui est venue toute seule. 

\speak  Que veux-tu dire, imbécile? 

\speak  Je veux dire qu'en rentrant, quoique j'eusse la clef de votre appartement dans ma poche et que cette clef ne m'eût point quitté, j'ai trouvé une lettre sur le tapis vert de la table, dans votre chambre à coucher. 

\speak  Et où est cette lettre? 

\speak  Je l'ai laissée où elle était, monsieur. Il n'est pas naturel que les lettres entrent ainsi chez les gens. Si la fenêtre était ouverte encore, ou seulement entrebâillée je ne dis pas; mais non, tout était hermétiquement fermé. Monsieur, prenez garde, car il y a très certainement quelque magie là-dessous.» 

Pendant ce temps, le jeune homme s'élançait dans la chambre et ouvrait la lettre; elle était de Mme Bonacieux, et conçue en ces termes: «On a de vifs remerciements à vous faire et à vous transmettre. Trouvez-vous ce soir vers dix heures à Saint-Cloud, en face du pavillon qui s'élève à l'angle de la maison de M. d'Estrées. --- «C. B.» 

En lisant cette lettre, d'Artagnan sentait son cœur se dilater et s'étreindre de ce doux spasme qui torture et caresse le cœur des amants. 

C'était le premier billet qu'il recevait, c'était le premier rendez-vous qui lui était accordé. Son cœur, gonflé par l'ivresse de la joie, se sentait prêt à défaillir sur le seuil de ce paradis terrestre qu'on appelait l'amour. 

«Eh bien! monsieur, dit Planchet, qui avait vu son maître rougir et pâlir successivement; eh bien! n'est-ce pas que j'avais deviné juste et que c'est quelque méchante affaire? 

\speak  Tu te trompes, Planchet, répondit d'Artagnan, et la preuve, c'est que voici un écu pour que tu boives à ma santé. 

\speak  Je remercie monsieur de l'écu qu'il me donne, et je lui promets de suivre exactement ses instructions; mais il n'en est pas moins vrai que les lettres qui entrent ainsi dans les maisons fermées\dots 

\speak  Tombent du ciel, mon ami, tombent du ciel. 

\speak  Alors, monsieur est content? demanda Planchet. 

\speak  Mon cher Planchet, je suis le plus heureux des hommes! 

\speak  Et je puis profiter du bonheur de monsieur pour aller me coucher? 

\speak  Oui, va. 

\speak  Que toutes les bénédictions du Ciel tombent sur monsieur, mais il n'en est pas moins vrai que cette lettre\dots» 

Et Planchet se retira en secouant la tête avec un air de doute que n'était point parvenu à effacer entièrement la libéralité de d'Artagnan. 

Resté seul, d'Artagnan lut et relut son billet, puis il baisa et rebaisa vingt fois ces lignes tracées par la main de sa belle maîtresse. Enfin il se coucha, s'endormit et fit des rêves d'or. 

À sept heures du matin, il se leva et appela Planchet, qui, au second appel, ouvrit la porte, le visage encore mal nettoyé des inquiétudes de la veille. 

«Planchet, lui dit d'Artagnan, je sors pour toute la journée peut-être; tu es donc libre jusqu'à sept heures du soir; mais, à sept heures du soir, tiens-toi prêt avec deux chevaux. 

\speak  Allons! dit Planchet, il paraît que nous allons encore nous faire traverser la peau en plusieurs endroits. 

\speak  Tu prendras ton mousqueton et tes pistolets. 

\speak  Eh bien, que disais-je? s'écria Planchet. Là, j'en étais sûr, maudite lettre! 

\speak  Mais rassure-toi donc, imbécile, il s'agit tout simplement d'une partie de plaisir. 

\speak  Oui! comme les voyages d'agrément de l'autre jour, où il pleuvait des balles et où il poussait des chausse-trapes. 

\speak  Au reste, si vous avez peur, monsieur Planchet, reprit d'Artagnan, j'irai sans vous; j'aime mieux voyager seul que d'avoir un compagnon qui tremble. 

\speak  Monsieur me fait injure, dit Planchet; il me semblait cependant qu'il m'avait vu à l'oeuvre. 

\speak  Oui, mais j'ai cru que tu avais usé tout ton courage d'une seule fois. 

\speak  Monsieur verra que dans l'occasion il m'en reste encore; seulement je prie monsieur de ne pas trop le prodiguer, s'il veut qu'il m'en reste longtemps. 

\speak  Crois-tu en avoir encore une certaine somme à dépenser ce soir? 

\speak  Je l'espère. 

\speak  Eh bien, je compte sur toi. 

\speak  À l'heure dite, je serai prêt; seulement je croyais que monsieur n'avait qu'un cheval à l'écurie des gardes. 

\speak  Peut-être n'y en a-t-il qu'un encore dans ce moment-ci, mais ce soir il y en aura quatre. 

\speak  Il paraît que notre voyage était un voyage de remonte? 

\speak  Justement», dit d'Artagnan. 

Et ayant fait à Planchet un dernier geste de recommandation, il sortit. 

M. Bonacieux était sur sa porte. L'intention de d'Artagnan était de passer outre, sans parler au digne mercier; mais celui-ci fit un salut si doux et si bénin, que force fut à son locataire non seulement de le lui rendre, mais encore de lier conversation avec lui. 

Comment d'ailleurs ne pas avoir un peu de condescendance pour un mari dont la femme vous a donné un rendez-vous le soir même à Saint-Cloud, en face du pavillon de M. d'Estrées! D'Artagnan s'approcha de l'air le plus aimable qu'il put prendre. 

La conversation tomba tout naturellement sur l'incarcération du pauvre homme. M. Bonacieux, qui ignorait que d'Artagnan eût entendu sa conversation avec l'inconnu de Meung, raconta à son jeune locataire les persécutions de ce monstre de M. de Laffemas, qu'il ne cessa de qualifier pendant tout son récit du titre de bourreau du cardinal et s'étendit longuement sur la Bastille, les verrous, les guichets, les soupiraux, les grilles et les instruments de torture. 

D'Artagnan l'écouta avec une complaisance exemplaire puis, lorsqu'il eut fini: 

«Et Mme Bonacieux, dit-il enfin, savez-vous qui l'avait enlevée? car je n'oublie pas que c'est à cette circonstance fâcheuse que je dois le bonheur d'avoir fait votre connaissance. 

\speak  Ah! dit M. Bonacieux, ils se sont bien gardés de me le dire, et ma femme de son côté m'a juré ses grands dieux qu'elle ne le savait pas. Mais vous-même, continua M. Bonacieux d'un ton de bonhomie parfaite, qu'êtes-vous devenu tous ces jours passés? je ne vous ai vu, ni vous ni vos amis, et ce n'est pas sur le pavé de Paris, je pense, que vous avez ramassé toute la poussière que Planchet époussetait hier sur vos bottes. 

\speak  Vous avez raison, mon cher monsieur Bonacieux, mes amis et moi nous avons fait un petit voyage. 

\speak  Loin d'ici? 

\speak  Oh! mon Dieu non, à une quarantaine de lieues seulement; nous avons été conduire M. Athos aux eaux de Forges, où mes amis sont restés. 

\speak  Et vous êtes revenu, vous, n'est-ce pas? reprit M. Bonacieux en donnant à sa physionomie son air le plus malin. Un beau garçon comme vous n'obtient pas de longs congés de sa maîtresse, et nous étions impatiemment attendu à Paris, n'est-ce pas? 

\speak  Ma foi, dit en riant le jeune homme, je vous l'avoue, d'autant mieux, mon cher monsieur Bonacieux, que je vois qu'on ne peut rien vous cacher. Oui, j'étais attendu, et bien impatiemment, je vous en réponds.» 

Un léger nuage passa sur le front de Bonacieux, mais si léger, que d'Artagnan ne s'en aperçut pas. 

«Et nous allons être récompensé de notre diligence? continua le mercier avec une légère altération dans la voix, altération que d'Artagnan ne remarqua pas plus qu'il n'avait fait du nuage momentané qui, un instant auparavant, avait assombri la figure du digne homme. 

\speak  Ah! faites donc le bon apôtre! dit en riant d'Artagnan. 

\speak  Non, ce que je vous en dis, reprit Bonacieux, c'est seulement pour savoir si nous rentrons tard. 

\speak  Pourquoi cette question, mon cher hôte? demanda d'Artagnan; est-ce que vous comptez m'attendre? 

\speak  Non, c'est que depuis mon arrestation et le vol qui a été commis chez moi, je m'effraie chaque fois que j'entends ouvrir une porte, et surtout la nuit. Dame, que voulez-vous! je ne suis point homme d'épée, moi! 

\speak  Eh bien, ne vous effrayez pas si je rentre à une heure, à deux ou trois heures du matin; si je ne rentre pas du tout, ne vous effrayez pas encore.» 

Cette fois, Bonacieux devint si pâle, que d'Artagnan ne put faire autrement que de s'en apercevoir, et lui demanda ce qu'il avait. 

«Rien, répondit Bonacieux, rien. Depuis mes malheurs seulement, je suis sujet à des faiblesses qui me prennent tout à coup, et je viens de me sentir passer un frisson. Ne faites pas attention à cela, vous qui n'avez à vous occuper que d'être heureux. 

\speak  Alors j'ai de l'occupation, car je le suis. 

\speak  Pas encore, attendez donc, vous avez dit: à ce soir. 

\speak  Eh bien, ce soir arrivera, Dieu merci! et peut-être l'attendez-vous avec autant d'impatience que moi. Peut-être, ce soir, Mme Bonacieux visitera-t-elle le domicile conjugal. 

\speak  Mme Bonacieux n'est pas libre ce soir, répondit gravement le mari; elle est retenue au Louvre par son service. 

\speak  Tant pis pour vous, mon cher hôte, tant pis; quand je suis heureux, moi, je voudrais que tout le monde le fût; mais il paraît que ce n'est pas possible.» 

Et le jeune homme s'éloigna en riant aux éclats de la plaisanterie que lui seul, pensait-il, pouvait comprendre. 

«Amusez-vous bien!» répondit Bonacieux d'un air sépulcral. 

Mais d'Artagnan était déjà trop loin pour l'entendre, et l'eut-il entendu, dans la disposition d'esprit où il était, il ne l'eût certes pas remarqué. 

Il se dirigea vers l'hôtel de M. de Tréville; sa visite de la veille avait été, on se le rappelle, très courte et très peu explicative. 

Il trouva M. de Tréville dans la joie de son âme. Le roi et la reine avaient été charmants pour lui au bal. Il est vrai que le cardinal avait été parfaitement maussade. 

À une heure du matin, il s'était retiré sous prétexte qu'il était indisposé. Quant à Leurs Majestés, elles n'étaient rentrées au Louvre qu'à six heures du matin. 

«Maintenant, dit M. de Tréville en baissant la voix et en interrogeant du regard tous les angles de l'appartement pour voir s'ils étaient bien seuls, maintenant parlons de vous, mon jeune ami, car il est évident que votre heureux retour est pour quelque chose dans la joie du roi, dans le triomphe de la reine et dans l'humiliation de Son Éminence. Il s'agit de bien vous tenir. 

\speak  Qu'ai-je à craindre, répondit d'Artagnan, tant que j'aurai le bonheur de jouir de la faveur de Leurs Majestés? 

\speak  Tout, croyez-moi. Le cardinal n'est point homme à oublier une mystification tant qu'il n'aura pas réglé ses comptes avec le mystificateur, et le mystificateur m'a bien l'air d'être certain Gascon de ma connaissance. 

\speak  Croyez-vous que le cardinal soit aussi avancé que vous et sache que c'est moi qui ai été à Londres? 

\speak  Diable! vous avez été à Londres. Est-ce de Londres que vous avez rapporté ce beau diamant qui brille à votre doigt? Prenez garde, mon cher d'Artagnan, ce n'est pas une bonne chose que le présent d'un ennemi; n'y a-t-il pas là-dessus certain vers latin\dots Attendez donc\dots 

\speak  Oui, sans doute, reprit d'Artagnan, qui n'avait jamais pu se fourrer la première règle du rudiment dans la tête, et qui, par ignorance, avait fait le désespoir de son précepteur; oui, sans doute, il doit y en avoir un. 

\speak  Il y en a un certainement, dit M. de Tréville, qui avait une teinte de lettres, et M. de Benserade me le citait l'autre jour\dots Attendez donc\dots Ah! m'y voici: \textit{Timeo Danaos et dona ferentes.} 

«Ce qui veut dire: “Défiez-vous de l'ennemi qui vous fait des présents.” 

\speak  Ce diamant ne vient pas d'un ennemi, monsieur, reprit d'Artagnan, il vient de la reine. 

\speak  De la reine! oh! oh! dit M. de Tréville. Effectivement, c'est un véritable bijou royal, qui vaut mille pistoles comme un denier. Par qui la reine vous a-t-elle fait remettre ce cadeau? 

\speak  Elle me l'a remis elle-même. 

\speak  Où cela? 

\speak  Dans le cabinet attenant à la chambre où elle a changé de toilette. 

\speak  Comment? 

\speak  En me donnant sa main à baiser. 

\speak  Vous avez baisé la main de la reine! s'écria M. de Tréville en regardant d'Artagnan. 

\speak  Sa Majesté m'a fait l'honneur de m'accorder cette grâce! 

\speak  Et cela en présence de témoins? Imprudente, trois fois imprudente! 

\speak  Non, monsieur, rassurez-vous, personne ne l'a vue», reprit d'Artagnan. Et il raconta à M. de Tréville comment les choses s'étaient passées. 

«Oh! les femmes, les femmes! s'écria le vieux soldat, je les reconnais bien à leur imagination romanesque; tout ce qui sent le mystérieux les charme; ainsi vous avez vu le bras, voilà tout; vous rencontreriez la reine, que vous ne la reconnaîtriez pas; elle vous rencontrerait, qu'elle ne saurait pas qui vous êtes. 

\speak  Non, mais grâce à ce diamant\dots, reprit le jeune homme. 

\speak  Écoutez, dit M. de Tréville, voulez-vous que je vous donne un conseil, un bon conseil, un conseil d'ami? 

\speak  Vous me ferez honneur, monsieur, dit d'Artagnan. 

\speak  Eh bien, allez chez le premier orfèvre venu et vendez-lui ce diamant pour le prix qu'il vous en donnera; si juif qu'il soit, vous en trouverez toujours bien huit cents pistoles. Les pistoles n'ont pas de nom, jeune homme, et cette bague en a un terrible, ce qui peut trahir celui qui la porte. 

\speak  Vendre cette bague! une bague qui vient de ma souveraine! jamais, dit d'Artagnan. 

\speak  Alors tournez-en le chaton en dedans, pauvre fou, car on sait qu'un cadet de Gascogne ne trouve pas de pareils bijoux dans l'écrin de sa mère. 

\speak  Vous croyez donc que j'ai quelque chose à craindre? demanda d'Artagnan. 

\speak  C'est-à-dire, jeune homme, que celui qui s'endort sur une mine dont la mèche est allumée doit se regarder comme en sûreté en comparaison de vous. 

\speak  Diable! dit d'Artagnan, que le ton d'assurance de M. de Tréville commençait à inquiéter: diable, que faut-il faire? 

\speak  Vous tenir sur vos gardes toujours et avant toute chose. Le cardinal a la mémoire tenace et la main longue; croyez-moi, il vous jouera quelque tour. 

\speak  Mais lequel? 

\speak  Eh! le sais-je, moi! est-ce qu'il n'a pas à son service toutes les ruses du démon? Le moins qui puisse vous arriver est qu'on vous arrête. 

\speak  Comment! on oserait arrêter un homme au service de Sa Majesté? 

\speak  Pardieu! on s'est bien gêné pour Athos! En tout cas, jeune homme, croyez-en un homme qui est depuis trente ans à la cour: ne vous endormez pas dans votre sécurité, ou vous êtes perdu. Bien au contraire, et c'est moi qui vous le dis, voyez des ennemis partout. Si l'on vous cherche querelle, évitez-la, fût-ce un enfant de dix ans qui vous la cherche; si l'on vous attaque de nuit ou de jour, battez en retraite et sans honte; si vous traversez un pont, tâtez les planches, de peur qu'une planche ne vous manque sous le pied; si vous passez devant une maison qu'on bâtit, regardez en l'air de peur qu'une pierre ne vous tombe sur la tête; si vous rentrez tard, faites-vous suivre par votre laquais, et que votre laquais soit armé, si toutefois vous êtes sûr de votre laquais. Défiez-vous de tout le monde, de votre ami, de votre frère, de votre maîtresse, de votre maîtresse surtout.» 

D'Artagnan rougit. 

«De ma maîtresse, répéta-t-il machinalement; et pourquoi plutôt d'elle que d'un autre? 

\speak  C'est que la maîtresse est un des moyens favoris du cardinal, il n'en a pas de plus expéditif: une femme vous vend pour dix pistoles, témoin Dalila. Vous savez les Écritures, hein?» 

D'Artagnan pensa au rendez-vous que lui avait donné Mme Bonacieux pour le soir même; mais nous devons dire, à la louange de notre héros, que la mauvaise opinion que M. de Tréville avait des femmes en général ne lui inspira pas le moindre petit soupçon contre sa jolie hôtesse. 

«Mais, à propos, reprit M. de Tréville, que sont devenus vos trois compagnons? 

\speak  J'allais vous demander si vous n'en aviez pas appris quelques nouvelles. 

\speak  Aucune, monsieur. 

\speak  Eh bien, je les ai laissés sur ma route: Porthos à Chantilly, avec un duel sur les bras; Aramis à Crèvecœur, avec une balle dans l'épaule; et Athos à Amiens, avec une accusation de faux-monnayeur sur le corps. 

\speak  Voyez-vous! dit M. de Tréville; et comment vous êtes-vous échappé, vous? 

\speak  Par miracle, monsieur, je dois le dire, avec un coup d'épée dans la poitrine, et en clouant M. le comte de Wardes sur le revers de la route de Calais, comme un papillon à une tapisserie. 

\speak  Voyez-vous encore! de Wardes, un homme au cardinal, un cousin de Rochefort. Tenez, mon cher ami, il me vient une idée. 

\speak  Dites, monsieur. 

\speak  À votre place, je ferais une chose. 

\speak  Laquelle? 

\speak  Tandis que Son Éminence me ferait chercher à Paris, je reprendrais, moi, sans tambour ni trompette, la route de Picardie, et je m'en irais savoir des nouvelles de mes trois compagnons. Que diable! ils méritent bien cette petite attention de votre part. 

\speak  Le conseil est bon, monsieur, et demain je partirai. 

\speak  Demain! et pourquoi pas ce soir? 

\speak  Ce soir, monsieur, je suis retenu à Paris par une affaire indispensable. 

\speak  Ah! jeune homme! jeune homme! quelque amourette? Prenez garde, je vous le répète: c'est la femme qui nous a perdus, tous tant que nous sommes. Croyez-moi, partez ce soir. 

\speak  Impossible! monsieur. 

\speak  Vous avez donc donné votre parole? 

\speak  Oui, monsieur. 

\speak  Alors c'est autre chose; mais promettez-moi que si vous n'êtes pas tué cette nuit, vous partirez demain. 

\speak  Je vous le promets. 

\speak  Avez-vous besoin d'argent? 

\speak  J'ai encore cinquante pistoles. C'est autant qu'il m'en faut, je le pense. 

\speak  Mais vos compagnons? 

\speak  Je pense qu'ils ne doivent pas en manquer. Nous sommes sortis de Paris chacun avec soixante-quinze pistoles dans nos poches. 

\speak  Vous reverrai-je avant votre départ? 

\speak  Non, pas que je pense, monsieur, à moins qu'il n'y ait du nouveau. 

\speak  Allons, bon voyage! 

\speak  Merci, monsieur.» 

Et d'Artagnan prit congé de M. de Tréville, touché plus que jamais de sa sollicitude toute paternelle pour ses mousquetaires. 

Il passa successivement chez Athos, chez Porthos et chez Aramis. Aucun d'eux n'était rentré. Leurs laquais aussi étaient absents, et l'on n'avait des nouvelles ni des uns, ni des autres. 

Il se serait bien informé d'eux à leurs maîtresses, mais il ne connaissait ni celle de Porthos, ni celle d'Aramis; quant à Athos, il n'en avait pas. 

En passant devant l'hôtel des Gardes, il jeta un coup d'œil dans l'écurie: trois chevaux étaient déjà rentrés sur quatre. Planchet, tout ébahi, était en train de les étriller, et avait déjà fini avec deux d'entre eux. 

«Ah! monsieur, dit Planchet en apercevant d'Artagnan, que je suis aise de vous voir! 

\speak  Et pourquoi cela, Planchet? demanda le jeune homme. 

\speak  Auriez-vous confiance en M. Bonacieux, notre hôte? 

\speak  Moi? pas le moins du monde. 

\speak  Oh! que vous faites bien, monsieur. 

\speak  Mais d'où vient cette question? 

\speak  De ce que, tandis que vous causiez avec lui, je vous observais sans vous écouter; monsieur, sa figure a changé deux ou trois fois de couleur. 

\speak  Bah! 

\speak  Monsieur n'a pas remarqué cela, préoccupé qu'il était de la lettre qu'il venait de recevoir; mais moi, au contraire, que l'étrange façon dont cette lettre était parvenue à la maison avait mis sur mes gardes, je n'ai pas perdu un mouvement de sa physionomie. 

\speak  Et tu l'as trouvée\dots? 

\speak  Traîtreuse, monsieur. 

\speak  Vraiment! 

\speak  De plus, aussitôt que monsieur l'a eu quitté et qu'il a disparu au coin de la rue, M. Bonacieux a pris son chapeau, a fermé sa porte et s'est mis à courir par la rue opposée. 

\speak  En effet, tu as raison, Planchet, tout cela me paraît fort louche, et, sois tranquille, nous ne lui paierons pas notre loyer que la chose ne nous ait été catégoriquement expliquée. 

\speak  Monsieur plaisante, mais monsieur verra. 

\speak  Que veux-tu, Planchet, ce qui doit arriver est écrit! 

\speak  Monsieur ne renonce donc pas à sa promenade de ce soir? 

\speak  Bien au contraire, Planchet, plus j'en voudrai à M. Bonacieux, et plus j'irai au rendez-vous que m'a donné cette lettre qui t'inquiète tant. 

\speak  Alors, si c'est la résolution de monsieur\dots 

\speak  Inébranlable, mon ami; ainsi donc, à neuf heures tiens-toi prêt ici, à l'hôtel; je viendrai te prendre.» 

Planchet, voyant qu'il n'y avait plus aucun espoir de faire renoncer son maître à son projet, poussa un profond soupir, et se mit à étriller le troisième cheval. 

Quant à d'Artagnan, comme c'était au fond un garçon plein de prudence, au lieu de rentrer chez lui, il s'en alla dîner chez ce prêtre gascon qui, au moment de la détresse des quatre amis, leur avait donné un déjeuner de chocolat. 