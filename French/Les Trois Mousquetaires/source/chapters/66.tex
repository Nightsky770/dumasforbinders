%!TeX root=../musketeersfr.tex 

\chapter{L'Exécution}

\lettrine{I}{l} était minuit à peu près; la lune, échancrée par sa décroissance et ensanglantée par les dernières traces de l'orage, se levait derrière la petite ville d'Armentières, qui détachait sur sa lueur blafarde la silhouette sombre de ses maisons et le squelette de son haut clocher découpé à jour. En face, la Lys roulait ses eaux pareilles à une rivière d'étain fondu; tandis que sur l'autre rive on voyait la masse noire des arbres se profiler sur un ciel orageux envahi par de gros nuages cuivrés qui faisaient une espèce de crépuscule au milieu de la nuit. À gauche s'élevait un vieux moulin abandonné, aux ailes immobiles, dans les ruines duquel une chouette faisait entendre son cri aigu, périodique et monotone. Çà et là dans la plaine, à droite et à gauche du chemin que suivait le lugubre cortège, apparaissaient quelques arbres bas et trapus, qui semblaient des nains difformes accroupis pour guetter les hommes à cette heure sinistre. 

De temps en temps un large éclair ouvrait l'horizon dans toute sa largeur, serpentait au-dessus de la masse noire des arbres et venait comme un effrayant cimeterre couper le ciel et l'eau en deux parties. Pas un souffle de vent ne passait dans l'atmosphère alourdie. Un silence de mort écrasait toute la nature; le sol était humide et glissant de la pluie qui venait de tomber, et les herbes ranimées jetaient leur parfum avec plus d'énergie. 

Deux valets traînaient Milady, qu'ils tenaient chacun par un bras; le bourreau marchait derrière, et Lord de Winter, d'Artagnan, Athos, Porthos et Aramis marchaient derrière le bourreau. 

Planchet et Bazin venaient les derniers. 

Les deux valets conduisaient Milady du côté de la rivière. Sa bouche était muette; mais ses yeux parlaient avec leur inexprimable éloquence, suppliant tour à tour chacun de ceux qu'elle regardait. 

Comme elle se trouvait de quelques pas en avant, elle dit aux valets: 

«Mille pistoles à chacun de vous si vous protégez ma fuite; mais si vous me livrez à vos maîtres, j'ai ici près des vengeurs qui vous feront payer cher ma mort.» 

Grimaud hésitait. Mousqueton tremblait de tous ses membres. 

Athos, qui avait entendu la voix de Milady, s'approcha vivement, Lord de Winter en fit autant. 

«Renvoyez ces valets, dit-il, elle leur a parlé, ils ne sont plus sûrs.» 

On appela Planchet et Bazin, qui prirent la place de Grimaud et de Mousqueton. 

Arrivés au bord de l'eau, le bourreau s'approcha de Milady et lui lia les pieds et les mains. 

Alors elle rompit le silence pour s'écrier: 

«Vous êtes des lâches, vous êtes des misérables assassins, vous vous mettez à dix pour égorger une femme; prenez garde, si je ne suis point secourue, je serai vengée. 

\speak  Vous n'êtes pas une femme, dit froidement Athos, vous n'appartenez pas à l'espèce humaine, vous êtes un démon échappé de l'enfer et que nous allons y faire rentrer. 

\speak  Ah! messieurs les hommes vertueux! dit Milady, faites attention que celui qui touchera un cheveu de ma tête est à son tour un assassin. 

\speak  Le bourreau peut tuer, sans être pour cela un assassin, madame, dit l'homme au manteau rouge en frappant sur sa large épée; c'est le dernier juge, voilà tout: \textit{Nachrichter}, comme disent nos voisins les Allemands.» 

Et, comme il la liait en disant ces paroles, Milady poussa deux ou trois cris sauvages, qui firent un effet sombre et étrange en s'envolant dans la nuit et en se perdant dans les profondeurs du bois. 

«Mais si je suis coupable, si j'ai commis les crimes dont vous m'accusez, hurlait Milady, conduisez-moi devant un tribunal, vous n'êtes pas des juges, vous, pour me condamner. 

\speak  Je vous avais proposé Tyburn, dit Lord de Winter, pourquoi n'avez-vous pas voulu? 

\speak  Parce que je ne veux pas mourir! s'écria Milady en se débattant, parce que je suis trop jeune pour mourir! 

\speak  La femme que vous avez empoisonnée à Béthune était plus jeune encore que vous, madame, et cependant elle est morte, dit d'Artagnan. 

\speak  J'entrerai dans un cloître, je me ferai religieuse, dit Milady. 

\speak  Vous étiez dans un cloître, dit le bourreau, et vous en êtes sortie pour perdre mon frère.» 

Milady poussa un cri d'effroi, et tomba sur ses genoux. 

Le bourreau la souleva sous les bras, et voulut l'emporter vers le bateau. 

«Oh! mon Dieu! s'écria-t-elle, mon Dieu! allez-vous donc me noyer!» 

Ces cris avaient quelque chose de si déchirant, que d'Artagnan, qui d'abord était le plus acharné à la poursuite de Milady, se laissa aller sur une souche, et pencha la tête, se bouchant les oreilles avec les paumes de ses mains; et cependant, malgré cela, il l'entendait encore menacer et crier. 

D'Artagnan était le plus jeune de tous ces hommes, le cœur lui manqua. 

«Oh! je ne puis voir cet affreux spectacle! je ne puis consentir à ce que cette femme meure ainsi!» 

Milady avait entendu ces quelques mots, et elle s'était reprise à une lueur d'espérance. 

«D'Artagnan! d'Artagnan! cria-t-elle, souviens-toi que je t'ai aimé!» 

Le jeune homme se leva et fit un pas vers elle. 

Mais Athos, brusquement, tira son épée, se mit sur son chemin. 

«Si vous faites un pas de plus, d'Artagnan, dit-il, nous croiserons le fer ensemble. 

D'Artagnan tomba à genoux et pria. 

«Allons, continua Athos, bourreau, fais ton devoir. 

\speak  Volontiers, Monseigneur, dit le bourreau, car aussi vrai que je suis bon catholique, je crois fermement être juste en accomplissant ma fonction sur cette femme. 

\speak  C'est bien.» 

Athos fit un pas vers Milady. 

«Je vous pardonne, dit-il, le mal que vous m'avez fait; je vous pardonne mon avenir brisé, mon honneur perdu, mon amour souillé et mon salut à jamais compromis par le désespoir où vous m'avez jeté. Mourez en paix.» 

Lord de Winter s'avança à son tour. 

«Je vous pardonne, dit-il, l'empoisonnement de mon frère, l'assassinat de Sa Grâce Lord Buckingham; je vous pardonne la mort du pauvre Felton, je vous pardonne vos tentatives sur ma personne. Mourez en paix. 

\speak  Et moi, dit d'Artagnan, pardonnez-moi, madame, d'avoir, par une fourberie indigne d'un gentilhomme, provoqué votre colère; et, en échange, je vous pardonne le meurtre de ma pauvre amie et vos vengeances cruelles pour moi, je vous pardonne et je pleure sur vous. Mourez en paix! 

\speak  \textit{I am lost!} murmura en anglais Milady. \textit{I must die.}» 

Alors elle se releva d'elle-même, jeta tout autour d'elle un de ces regards clairs qui semblaient jaillir d'un œil de flamme. 

Elle ne vit rien. 

Elle écouta et n'entendit rien. 

Elle n'avait autour d'elle que des ennemis. 

«Où vais-je mourir? dit-elle. 

\speak  Sur l'autre rive», répondit le bourreau. 

Alors il la fit entrer dans la barque, et, comme il allait y mettre le pied pour la suivre, Athos lui remit une somme d'argent. 

«Tenez, dit-il, voici le prix de l'exécution; que l'on voie bien que nous agissons en juges. 

\speak  C'est bien, dit le bourreau; et que maintenant, à son tour, cette femme sache que je n'accomplis pas mon métier, mais mon devoir.» 

Et il jeta l'argent dans la rivière. 

Le bateau s'éloigna vers la rive gauche de la Lys, emportant la coupable et l'exécuteur; tous les autres demeurèrent sur la rive droite, où ils étaient tombés à genoux. 

Le bateau glissait lentement le long de la corde du bac, sous le reflet d'un nuage pâle qui surplombait l'eau en ce moment. 

On le vit aborder sur l'autre rive; les personnages se dessinaient en noir sur l'horizon rougeâtre. 

Milady, pendant le trajet, était parvenue à détacher la corde qui liait ses pieds: en arrivant sur le rivage, elle sauta légèrement à terre et prit la fuite. 

Mais le sol était humide; en arrivant au haut du talus, elle glissa et tomba sur ses genoux. 

Une idée superstitieuse la frappa sans doute; elle comprit que le Ciel lui refusait son secours et resta dans l'attitude où elle se trouvait, la tête inclinée et les mains jointes. 

Alors on vit, de l'autre rive, le bourreau lever lentement ses deux bras, un rayon de lune se refléta sur la lame de sa large épée, les deux bras retombèrent; on entendit le sifflement du cimeterre et le cri de la victime, puis une masse tronquée s'affaissa sous le coup. 

Alors le bourreau détacha son manteau rouge, l'étendit à terre, y coucha le corps, y jeta la tête, le noua par les quatre coins, le chargea sur son épaule et remonta dans le bateau. 

Arrivé au milieu de la Lys, il arrêta la barque, et suspendant son fardeau au-dessus de la rivière: 

«Laissez passer la justice de Dieu!» cria-t-il à haute voix. 

Et il laissa tomber le cadavre au plus profond de l'eau, qui se referma sur lui. 

Trois jours après, les quatre mousquetaires rentraient à Paris; ils étaient restés dans les limites de leur congé, et le même soir ils allèrent faire leur visite accoutumée à M. de Tréville. 

«Eh bien, messieurs, leur demanda le brave capitaine, vous êtes-vous bien amusés dans votre excursion? 

\speak  Prodigieusement», répondit Athos, les dents serrées. 