%!TeX root=../musketeersfr.tex 

\chapter{La Thèse D'Aramis} 
	
\lettrine{D}{'Artagnan} n'avait rien dit à Porthos de sa blessure ni de sa procureuse. C'était un garçon fort sage que notre Béarnais, si jeune qu'il fût. En conséquence, il avait fait semblant de croire tout ce que lui avait raconté le glorieux mousquetaire, convaincu qu'il n'y a pas d'amitié qui tienne à un secret surpris, surtout quand ce secret intéresse l'orgueil; puis on a toujours une certaine supériorité morale sur ceux dont on sait la vie. 

Or d'Artagnan, dans ses projets d'intrigue à venir, et décidé qu'il était à faire de ses trois compagnons les instruments de sa fortune, d'Artagnan n'était pas fâché de réunir d'avance dans sa main les fils invisibles à l'aide desquels il comptait les mener. 

Cependant, tout le long de la route, une profonde tristesse lui serrait le cœur: il pensait à cette jeune et jolie Mme Bonacieux qui devait lui donner le prix de son dévouement; mais, hâtons-nous de le dire, cette tristesse venait moins chez le jeune homme du regret de son bonheur perdu que de la crainte qu'il éprouvait qu'il n'arrivât malheur à cette pauvre femme. Pour lui, il n'y avait pas de doute, elle était victime d'une vengeance du cardinal et comme on le sait, les vengeances de Son Éminence étaient terribles. Comment avait-il trouvé grâce devant les yeux du ministre, c'est ce qu'il ignorait lui-même et sans doute ce que lui eût révélé M. de Cavois, si le capitaine des gardes l'eût trouvé chez lui. 

Rien ne fait marcher le temps et n'abrège la route comme une pensée qui absorbe en elle-même toutes les facultés de l'organisation de celui qui pense. L'existence extérieure ressemble alors à un sommeil dont cette pensée est le rêve. Par son influence, le temps n'a plus de mesure, l'espace n'a plus de distance. On part d'un lieu, et l'on arrive à un autre, voilà tout. De l'intervalle parcouru, rien ne reste présent à votre souvenir qu'un brouillard vague dans lequel s'effacent mille images confuses d'arbres, de montagnes et de paysages. Ce fut en proie à cette hallucination que d'Artagnan franchit, à l'allure que voulut prendre son cheval, les six ou huit lieues qui séparent Chantilly de Crèvecœur, sans qu'en arrivant dans ce village il se souvînt d'aucune des choses qu'il avait rencontrées sur sa route. 

Là seulement la mémoire lui revint, il secoua la tête aperçut le cabaret où il avait laissé Aramis, et, mettant son cheval au trot, il s'arrêta à la porte. 

Cette fois ce ne fut pas un hôte, mais une hôtesse qui le reçut; d'Artagnan était physionomiste, il enveloppa d'un coup d'œil la grosse figure réjouie de la maîtresse du lieu, et comprit qu'il n'avait pas besoin de dissimuler avec elle et qu'il n'avait rien à craindre de la part d'une si joyeuse physionomie. 

«Ma bonne dame, lui demanda d'Artagnan, pourriez-vous me dire ce qu'est devenu un de mes amis, que nous avons été forcés de laisser ici il y a une douzaine de jours? 

\speak  Un beau jeune homme de vingt-trois à vingt-quatre ans, doux, aimable, bien fait? 

\speak  De plus, blessé à l'épaule. 

\speak  C'est cela! 

\speak  Justement. 

\speak  Eh bien, monsieur, il est toujours ici. 

\speak  Ah! pardieu, ma chère dame, dit d'Artagnan en mettant pied à terre et en jetant la bride de son cheval au bras de Planchet, vous me rendez la vie; où est-il, ce cher Aramis, que je l'embrasse? car, je l'avoue, j'ai hâte de le revoir. 

\speak  Pardon, monsieur, mais je doute qu'il puisse vous recevoir en ce moment. 

\speak  Pourquoi cela? est-ce qu'il est avec une femme? 

\speak  Jésus! que dites-vous là! le pauvre garçon! Non, monsieur, il n'est pas avec une femme. 

\speak  Et avec qui est-il donc? 

\speak  Avec le curé de Montdidier et le supérieur des jésuites d'Amiens. 

\speak  Mon Dieu! s'écria d'Artagnan, le pauvre garçon irait-il plus mal? 

\speak  Non, monsieur, au contraire; mais, à la suite de sa maladie, la grâce l'a touché et il s'est décidé à entrer dans les ordres. 

\speak  C'est juste, dit d'Artagnan, j'avais oublié qu'il n'était mousquetaire que par intérim. 

\speak  Monsieur insiste-t-il toujours pour le voir? 

\speak  Plus que jamais. 

\speak  Eh bien, monsieur n'a qu'à prendre l'escalier à droite dans la cour, au second, n° 5.» 

D'Artagnan s'élança dans la direction indiquée et trouva un de ces escaliers extérieurs comme nous en voyons encore aujourd'hui dans les cours des anciennes auberges. Mais on n'arrivait pas ainsi chez le futur abbé; les défilés de la chambre d'Aramis étaient gardés ni plus ni moins que les jardins d'Aramis; Bazin stationnait dans le corridor et lui barra le passage avec d'autant plus d'intrépidité qu'après bien des années d'épreuve, Bazin se voyait enfin près d'arriver au résultat qu'il avait éternellement ambitionné. 

En effet, le rêve du pauvre Bazin avait toujours été de servir un homme d'Église, et il attendait avec impatience le moment sans cesse entrevu dans l'avenir où Aramis jetterait enfin la casaque aux orties pour prendre la soutane. La promesse renouvelée chaque jour par le jeune homme que le moment ne pouvait tarder l'avait seule retenu au service d'un mousquetaire, service dans lequel, disait-il, il ne pouvait manquer de perdre son âme. 

Bazin était donc au comble de la joie. Selon toute probabilité, cette fois son maître ne se dédirait pas. La réunion de la douleur physique à la douleur morale avait produit l'effet si longtemps désiré: Aramis, souffrant à la fois du corps et de l'âme, avait enfin arrêté sur la religion ses yeux et sa pensée, et il avait regardé comme un avertissement du Ciel le double accident qui lui était arrivé, c'est-à-dire la disparition subite de sa maîtresse et sa blessure à l'épaule. 

On comprend que rien ne pouvait, dans la disposition où il se trouvait, être plus désagréable à Bazin que l'arrivée de d'Artagnan, laquelle pouvait rejeter son maître dans le tourbillon des idées mondaines qui l'avaient si longtemps entraîné. Il résolut donc de défendre bravement la porte; et comme, trahi par la maîtresse de l'auberge, il ne pouvait dire qu'Aramis était absent, il essaya de prouver au nouvel arrivant que ce serait le comble de l'indiscrétion que de déranger son maître dans la pieuse conférence qu'il avait entamée depuis le matin, et qui, au dire de Bazin, ne pouvait être terminée avant le soir. 

Mais d'Artagnan ne tint aucun compte de l'éloquent discours de maître Bazin, et comme il ne se souciait pas d'entamer une polémique avec le valet de son ami, il l'écarta tout simplement d'une main, et de l'autre il tourna le bouton de la porte n° 5. 

La porte s'ouvrit, et d'Artagnan pénétra dans la chambre. 

Aramis, en surtout noir, le chef accommodé d'une espèce de coiffure ronde et plate qui ne ressemblait pas mal à une calotte, était assis devant une table oblongue couverte de rouleaux de papier et d'énormes in-folio; à sa droite était assis le supérieur des jésuites, et à sa gauche le curé de Montdidier. Les rideaux étaient à demi clos et ne laissaient pénétrer qu'un jour mystérieux, ménagé pour une béate rêverie. Tous les objets mondains qui peuvent frapper l'œil quand on entre dans la chambre d'un jeune homme, et surtout lorsque ce jeune homme est mousquetaire, avaient disparu comme par enchantement; et, de peur sans doute que leur vue ne ramenât son maître aux idées de ce monde, Bazin avait fait main basse sur l'épée, les pistolets, le chapeau à plume, les broderies et les dentelles de tout genre et de toute espèce. 

Mais, en leur lieu et place, d'Artagnan crut apercevoir dans un coin obscur comme une forme de discipline suspendue par un clou à la muraille. 

Au bruit que fit d'Artagnan en ouvrant la porte, Aramis leva la tête et reconnut son ami. Mais, au grand étonnement du jeune homme, sa vue ne parut pas produire une grande impression sur le mousquetaire, tant son esprit était détaché des choses de la terre. 

«Bonjour, cher d'Artagnan, dit Aramis; croyez que je suis heureux de vous voir. 

\speak  Et moi aussi, dit d'Artagnan, quoique je ne sois pas encore bien sûr que ce soit à Aramis que je parle. 

\speak  À lui-même, mon ami, à lui-même; mais qui a pu vous faire douter? 

\speak  J'avais peur de me tromper de chambre, et j'ai cru d'abord entrer dans l'appartement de quelque homme Église; puis une autre erreur m'a pris en vous trouvant en compagnie de ces messieurs: c'est que vous ne fussiez gravement malade.» 

Les deux hommes noirs lancèrent sur d'Artagnan, dont ils comprirent l'intention, un regard presque menaçant; mais d'Artagnan ne s'en inquiéta pas. 

«Je vous trouble peut-être, mon cher Aramis, continua d'Artagnan; car, d'après ce que je vois, je suis porté à croire que vous vous confessez à ces messieurs.» 

Aramis rougit imperceptiblement. 

«Vous, me troubler? oh! bien au contraire, cher ami, je vous le jure; et comme preuve de ce que je dis, permettez-moi de me réjouir en vous voyant sain et sauf. 

\speak  Ah! il y vient enfin! pensa d'Artagnan, ce n'est pas malheureux. 

\speak  Car, monsieur, qui est mon ami, vient d'échapper à un rude danger, continua Aramis avec onction, en montrant de la main d'Artagnan aux deux ecclésiastiques. 

\speak  Louez Dieu, monsieur, répondirent ceux-ci en s'inclinant à l'unisson. 

\speak  Je n'y ai pas manqué, mes révérends, répondit le jeune homme en leur rendant leur salut à son tour. 

\speak  Vous arrivez à propos, cher d'Artagnan, dit Aramis, et vous allez, en prenant part à la discussion, l'éclairer de vos lumières. M. le principal d'Amiens, M. le curé de Montdidier et moi, nous argumentons sur certaines questions théologiques dont l'intérêt nous captive depuis longtemps; je serais charmé d'avoir votre avis. 

\speak  L'avis d'un homme d'épée est bien dénué de poids, répondit d'Artagnan, qui commençait à s'inquiéter de la tournure que prenaient les choses, et vous pouvez vous en tenir, croyez-moi, à la science de ces messieurs.» 

Les deux hommes noirs saluèrent à leur tour. 

«Au contraire, reprit Aramis, et votre avis nous sera précieux; voici de quoi il s'agit: M. le principal croit que ma thèse doit être surtout dogmatique et didactique. 

\speak  Votre thèse! vous faites donc une thèse? 

\speak  Sans doute, répondit le jésuite; pour l'examen qui précède l'ordination, une thèse est de rigueur. 

\speak  L'ordination! s'écria d'Artagnan, qui ne pouvait croire à ce que lui avaient dit successivement l'hôtesse et Bazin,\dots l'ordination!» 

Et il promenait ses yeux stupéfaits sur les trois personnages qu'il avait devant lui. 

«Or», continua Aramis en prenant sur son fauteuil la même pose gracieuse que s'il eût été dans une ruelle et en examinant avec complaisance sa main blanche et potelée comme une main de femme, qu'il tenait en l'air pour en faire descendre le sang: «or, comme vous l'avez entendu, d'Artagnan, M. le principal voudrait que ma thèse fût dogmatique, tandis que je voudrais, moi, qu'elle fût idéale. C'est donc pourquoi M. le principal me proposait ce sujet qui n'a point encore été traité, dans lequel je reconnais qu'il y a matière à de magnifiques développements. 

\textit{«Utraque manus in benedicendo clericis inferioribus necessaria est.»} 

D'Artagnan, dont nous connaissons l'érudition, ne sourcilla pas plus à cette citation qu'à celle que lui avait faite M. de Tréville à propos des présents qu'il prétendait que d'Artagnan avait reçus de M. de Buckingham. 

«Ce qui veut dire, reprit Aramis pour lui donner toute facilité: les deux mains sont indispensables aux prêtres des ordres inférieurs, quand ils donnent la bénédiction. 

\speak  Admirable sujet! s'écria le jésuite. 

\speak  Admirable et dogmatique!» répéta le curé qui, de la force de d'Artagnan à peu près sur le latin, surveillait soigneusement le jésuite pour emboîter le pas avec lui et répéter ses paroles comme un écho. 

Quant à d'Artagnan, il demeura parfaitement indifférent à l'enthousiasme des deux hommes noirs. 

«Oui, admirable! \textit{prorsus admirabile}! continua Aramis, mais qui exige une étude approfondie des Pères et des Écritures. Or j'ai avoué à ces savants ecclésiastiques, et cela en toute humilité, que les veilles des corps de garde et le service du roi m'avaient fait un peu négliger l'étude. Je me trouverai donc plus à mon aise, \textit{facilius natans}, dans un sujet de mon choix, qui serait à ces rudes questions théologiques ce que la morale est à la métaphysique en philosophie.» 

D'Artagnan s'ennuyait profondément, le curé aussi. 

«Voyez quel exorde! s'écria le jésuite. 

\speak  \textit{Exordium}, répéta le curé pour dire quelque chose. 

\speak  \textit{Quemadmodum minter cœlorum immensitatem.}» 

Aramis jeta un coup d'œil de côté sur d'Artagnan, et il vit que son ami bâillait à se démonter la mâchoire. 

«Parlons français, mon père, dit-il au jésuite, M. d'Artagnan goûtera plus vivement nos paroles. 

\speak  Oui, je suis fatigué de la route, dit d'Artagnan, et tout ce latin m'échappe. 

\speak  D'accord, dit le jésuite un peu dépité, tandis que le curé, transporté d'aise, tournait sur d'Artagnan un regard plein de reconnaissance; eh bien, voyez le parti qu'on tirerait de cette glose. 

\speak  Moïse, serviteur de Dieu\dots il n'est que serviteur, entendez-vous bien! Moïse bénit avec les mains; il se fait tenir les deux bras, tandis que les Hébreux battent leurs ennemis; donc il bénit avec les deux mains. D'ailleurs, que dit l'Évangile: \textit{imponite manus}, et non pas \textit{manum}. Imposez les mains, et non pas la main. 

\speak  Imposez les mains, répéta le curé en faisant un geste. 

\speak  À saint Pierre, au contraire, de qui les papes sont successeurs, continua le jésuite: \textit{Porrige digitos}. Présentez les doigts; y êtes-vous maintenant? 

\speak  Certes, répondit Aramis en se délectant, mais la chose est subtile. 

\speak  Les doigts! reprit le jésuite; saint Pierre bénit avec les doigts. Le pape bénit donc aussi avec les doigts. Et avec combien de doigts bénit-il? Avec trois doigts, un pour le Père, un pour le Fils, et un pour le Saint-Esprit.» 

Tout le monde se signa; d'Artagnan crut devoir imiter cet exemple. 

«Le pape est successeur de saint Pierre et représente les trois pouvoirs divins; le reste, \textit{ordines inferiores} de la hiérarchie ecclésiastique, bénit par le nom des saints archanges et des anges. Les plus humbles clercs, tels que nos diacres et sacristains, bénissent avec les goupillons, qui simulent un nombre indéfini de doigts bénissants. Voilà le sujet simplifié, \textit{Argumentum omni denudatum ornamento}. Je ferais avec cela, continua le jésuite, deux volumes de la taille de celui-ci.» 

Et, dans son enthousiasme, il frappait sur le saint Chrysostome in-folio qui faisait plier la table sous son poids. 

D'Artagnan frémit. 

«Certes, dit Aramis, je rends justice aux beautés de cette thèse, mais en même temps je la reconnais écrasante pour moi. J'avais choisi ce texte; dites-moi, cher d'Artagnan, s'il n'est point de votre goût: \textit{Non inutile est desiderium in oblatione}, ou mieux encore: un peu de regret ne messied pas dans une offrande au Seigneur. 

\speak  Halte-là! s'écria le jésuite, car cette thèse frise l'hérésie; il y a une proposition presque semblable dans l'\textit{Augustinus} de l'hérésiarque Jansénius, dont tôt ou tard le livre sera brûlé par les mains du bourreau. Prenez garde! mon jeune ami; vous penchez vers les fausses doctrines, mon jeune ami; vous vous perdrez! 

\speak  Vous vous perdrez, dit le curé en secouant douloureusement la tête. 

\speak  Vous touchez à ce fameux point du libre arbitre, qui est un écueil mortel. Vous abordez de front les insinuations des pélagiens et des demi-pélagiens. 

\speak  Mais, mon révérend\dots, reprit Aramis quelque peu abasourdi de la grêle d'arguments qui lui tombait sur la tête. 

\speak  Comment prouverez-vous, continua le jésuite sans lui donner le temps de parler, que l'on doit regretter le monde lorsqu'on s'offre à Dieu? écoutez ce dilemme: Dieu est Dieu, et le monde est le diable. Regretter le monde, c'est regretter le diable: voilà ma conclusion. 

\speak  C'est la mienne aussi, dit le curé. 

\speak  Mais de grâce!\dots dit Aramis. 

\speak  \textit{Desideras diabolum}, infortuné! s'écria le jésuite. 

\speak  Il regrette le diable! Ah! mon jeune ami, reprit le curé en gémissant, ne regrettez pas le diable, c'est moi qui vous en supplie.» 

D'Artagnan tournait à l'idiotisme; il lui semblait être dans une maison de fous, et qu'il allait devenir fou comme ceux qu'il voyait. Seulement il était forcé de se taire, ne comprenant point la langue qui se parlait devant lui. 

«Mais écoutez-moi donc, reprit Aramis avec une politesse sous laquelle commençait à percer un peu d'impatience, je ne dis pas que je regrette; non, je ne prononcerai jamais cette phrase qui ne serait pas orthodoxe\dots» 

Le jésuite leva les bras au ciel, et le curé en fit autant. 

«Non, mais convenez au moins qu'on a mauvaise grâce de n'offrir au Seigneur que ce dont on est parfaitement dégoûté. Ai-je raison, d'Artagnan? 

\speak  Je le crois pardieu bien!» s'écria celui-ci. 

Le curé et le jésuite firent un bond sur leur chaise. 

«Voici mon point de départ, c'est un syllogisme: le monde ne manque pas d'attraits, je quitte le monde, donc je fais un sacrifice; or l'Écriture dit positivement: Faites un sacrifice au Seigneur. 

\speak  Cela est vrai, dirent les antagonistes. 

\speak  Et puis, continua Aramis en se pinçant l'oreille pour la rendre rouge, comme il se secouait les mains pour les rendre blanches, et puis j'ai fait certain rondeau là-dessus que je communiquai à M. Voiture l'an passé, et duquel ce grand homme m'a fait mille compliments. 

\speak  Un rondeau! fit dédaigneusement le jésuite. 

\speak  Un rondeau! dit machinalement le curé. 

\speak  Dites, dites, s'écria d'Artagnan, cela nous changera quelque peu. 

\speak  Non, car il est religieux, répondit Aramis, et c'est de la théologie en vers. 

\speak  Diable! fit d'Artagnan. 

\speak  Le voici, dit Aramis d'un petit air modeste qui n'était pas exempt d'une certaine teinte d'hypocrisie: 

\begin{verse}
<Vous qui pleurez un passé plein de charmes,\\
Et qui trainez des jours infortunés,\\
Tous vos malheurs se verront terminés,\\
Quand à Dieu seul vous offrirez vos larmes,\\
Vous qui pleurez!>
\end{verse}

D'Artagnan et le curé parurent flattés. Le jésuite persista dans son opinion. 

\begin{conversation}
Gardez-vous du goût profane dans le style théologique. Que dit en effet saint Augustin? \textit{Severus sit clericorum sermo}. 

\speak Oui, que le sermon soit clair! dit le curé. 

\speak Or, se hâta d'interrompre le jésuite en voyant que son acolyte se fourvoyait, or votre thèse plaira aux dames, voilà tout; elle aura le succès d'une plaidoirie de maître Patru. 

\speak Plaise à Dieu! s'écria Aramis transporté. 

\speak Vous le voyez, s'écria le jésuite, le monde parle encore en vous à haute voix, \textit{altissima voce}. Vous suivez le monde, mon jeune ami, et je tremble que la grâce ne soit point efficace. 

\speak Rassurez-vous, mon révérend, je réponds de moi. 

\speak Présomption mondaine! 

\speak Je me connais, mon père, ma résolution est irrévocable. 

\speak Alors vous vous obstinez à poursuivre cette thèse? 

\speak Je me sens appelé à traiter celle-là, et non pas une autre; je vais donc la continuer, et demain j'espère que vous serez satisfait des corrections que j'y aurai faites d'après vos avis. 

\speak Travaillez lentement, dit le curé, nous vous laissons dans des dispositions excellentes. 

\speak Oui, le terrain est tout ensemencé, dit le jésuite, et nous n'avons pas à craindre qu'une partie du grain soit tombée sur la pierre, l'autre le long du chemin, et que les oiseaux du ciel aient mangé le reste, \textit{aves cœli comederunt illam}. 

\speak Que la peste t'étouffe avec ton latin! dit d'Artagnan, qui se sentait au bout de ses forces. 

\speak Adieu, mon fils, dit le curé, à demain. 

\speak À demain, jeune téméraire, dit le jésuite; vous promettez d'être une des lumières de l'Église; veuille le Ciel que cette lumière ne soit pas un feu dévorant.
\end{conversation}

D'Artagnan, qui pendant une heure s'était rongé les ongles d'impatience, commençait à attaquer la chair. 

Les deux hommes noirs se levèrent, saluèrent Aramis et d'Artagnan, et s'avancèrent vers la porte. Bazin, qui s'était tenu debout et qui avait écouté toute cette controverse avec une pieuse jubilation, s'élança vers eux, prit le bréviaire du curé, le missel du jésuite, et marcha respectueusement devant eux pour leur frayer le chemin. 

Aramis les conduisit jusqu'au bas de l'escalier et remonta aussitôt près de d'Artagnan qui rêvait encore. 

Restés seuls, les deux amis gardèrent d'abord un silence embarrassé; cependant il fallait que l'un des deux le rompît le premier, et comme d'Artagnan paraissait décidé à laisser cet honneur à son ami: 

«Vous le voyez, dit Aramis, vous me trouvez revenu à mes idées fondamentales. 

\speak  Oui, la grâce efficace vous a touché, comme disait ce monsieur tout à l'heure. 

\speak  Oh! ces plans de retraite sont formés depuis longtemps; et vous m'en avez déjà ouï parler, n'est-ce pas, mon ami? 

\speak  Sans doute, mais je vous avoue que j'ai cru que vous plaisantiez. 

\speak  Avec ces sortes de choses! Oh! d'Artagnan! 

\speak  Dame! on plaisante bien avec la mort. 

\speak  Et l'on a tort, d'Artagnan: car la mort, c'est la porte qui conduit à la perdition ou au salut. 

\speak  D'accord; mais, s'il vous plaît, ne théologisons pas, Aramis; vous devez en avoir assez pour le reste de la journée: quant à moi, j'ai à peu près oublié le peu de latin que je n'ai jamais su; puis, je vous l'avouerai, je n'ai rien mangé depuis ce matin dix heures, et j'ai une faim de tous les diables. 

\speak  Nous dînerons tout à l'heure, cher ami; seulement, vous vous rappellerez que c'est aujourd'hui vendredi; or, dans un pareil jour, je ne puis ni voir, ni manger de la chair. Si vous voulez vous contenter de mon dîner, il se compose de tétragones cuits et de fruits. 

\speak  Qu'entendez-vous par tétragones? demanda d'Artagnan avec inquiétude. 

\speak  J'entends des épinards, reprit Aramis, mais pour vous j'ajouterai des oeufs, et c'est une grave infraction à la règle, car les oeufs sont viande, puisqu'ils engendrent le poulet. 

\speak  Ce festin n'est pas succulent, mais n'importe; pour rester avec vous, je le subirai. 

\speak  Je vous suis reconnaissant du sacrifice, dit Aramis; mais s'il ne profite pas à votre corps, il profitera, soyez-en certain, à votre âme. 

\speak  Ainsi, décidément, Aramis, vous entrez en religion. Que vont dire nos amis, que va dire M. de Tréville? Ils vous traiteront de déserteur, je vous en préviens. 

\speak  Je n'entre pas en religion, j'y rentre. C'est Église que j'avais désertée pour le monde, car vous savez que je me suis fait violence pour prendre la casaque de mousquetaire. 

\speak  Moi, je n'en sais rien. 

\speak  Vous ignorez comment j'ai quitté le séminaire? 

\speak  Tout à fait. 

\speak  Voici mon histoire; d'ailleurs les Écritures disent: «Confessez-vous les uns aux autres», et je me confesse à vous, d'Artagnan. 

\speak  Et moi, je vous donne l'absolution d'avance, vous voyez que je suis bon homme. 

\speak  Ne plaisantez pas avec les choses saintes, mon ami. 

\speak  Alors, dites, je vous écoute. 

\speak  J'étais donc au séminaire depuis l'âge de neuf ans, j'en avais vingt dans trois jours, j'allais être abbé, et tout était dit. Un soir que je me rendais, selon mon habitude, dans une maison que je fréquentais avec plaisir --- on est jeune, que voulez-vous! on est faible, --- un officier qui me voyait d'un œil jaloux lire les vies des saints à la maîtresse de la maison, entra tout à coup et sans être annoncé. Justement, ce soir-là, j'avais traduit un épisode de Judith, et je venais de communiquer mes vers à la dame qui me faisait toutes sortes de compliments, et, penchée sur mon épaule, les relisait avec moi. La pose, qui était quelque peu abandonnée, je l'avoue, blessa cet officier; il ne dit rien, mais lorsque je sortis, il sortit derrière moi, et me rejoignant: 

«--- Monsieur l'abbé, dit-il, aimez-vous les coups de canne? 

«--- Je ne puis le dire, monsieur, répondis-je, personne n'ayant jamais osé m'en donner. 

«--- Eh bien, écoutez-moi, monsieur l'abbé, si vous retournez dans la maison où je vous ai rencontré ce soir, j'oserai, moi.» 

«Je crois que j'eus peur, je devins fort pâle, je sentis les jambes qui me manquaient, je cherchai une réponse que je ne trouvai pas, je me tus. 

«L'officier attendait cette réponse, et voyant qu'elle tardait, il se mit à rire, me tourna le dos et rentra dans la maison. Je rentrai au séminaire. 

«Je suis bon gentilhomme et j'ai le sang vif, comme vous avez pu le remarquer, mon cher d'Artagnan; l'insulte était terrible, et, tout inconnue qu'elle était restée au monde, je la sentais vivre et remuer au fond de mon cœur. Je déclarai à mes supérieurs que je ne me sentais pas suffisamment préparé pour l'ordination, et, sur ma demande, on remit la cérémonie à un an. 

«J'allai trouver le meilleur maître d'armes de Paris, je fis condition avec lui pour prendre une leçon d'escrime chaque jour, et chaque jour, pendant une année, je pris cette leçon. Puis, le jour anniversaire de celui où j'avais été insulté, j'accrochai ma soutane à un clou, je pris un costume complet de cavalier, et je me rendis à un bal que donnait une dame de mes amies, et où je savais que devait se trouver mon homme. C'était rue des Francs-Bourgeois, tout près de la Force. 

«En effet, mon officier y était; je m'approchai de lui, comme il chantait un lai d'amour en regardant tendrement une femme, et je l'interrompis au beau milieu du second couplet. 

«--- Monsieur, lui dis-je, vous déplaît-il toujours que je retourne dans certaine maison de la rue Payenne, et me donnerez-vous encore des coups de canne, s'il me prend fantaisie de vous désobéir?» 

«L'officier me regarda avec étonnement, puis il dit: 

«--- Que me voulez-vous, monsieur? Je ne vous connais pas. 

«--- Je suis, répondis-je, le petit abbé qui lit les vies des saints et qui traduit Judith en vers. 

«--- Ah! ah! je me rappelle, dit l'officier en goguenardant; que me voulez-vous? 

«--- Je voudrais que vous eussiez le loisir de venir faire un tour de promenade avec moi. 

«--- Demain matin, si vous le voulez bien, et ce sera avec le plus grand plaisir. 

«--- Non, pas demain matin, s'il vous plaît, tout de suite. 

«--- Si vous l'exigez absolument\dots 

«--- Mais oui, je l'exige. 

«--- Alors, sortons. Mesdames, dit l'officier, ne vous dérangez pas. Le temps de tuer monsieur seulement, et je reviens vous achever le dernier couplet.» 

«Nous sortîmes. 

«Je le menai rue Payenne, juste à l'endroit où un an auparavant, heure pour heure, il m'avait fait le compliment que je vous ai rapporté. Il faisait un clair de lune superbe. Nous mîmes l'épée à la main, et à la première passe, je le tuai roide. 

\speak  Diable! fit d'Artagnan. 

\speak  Or, continua Aramis, comme les dames ne virent pas revenir leur chanteur, et qu'on le trouva rue Payenne avec un grand coup d'épée au travers du corps, on pensa que c'était moi qui l'avait accommodé ainsi, et la chose fit scandale. Je fus donc pour quelque temps forcé de renoncer à la soutane. Athos, dont je fis la connaissance à cette époque, et Porthos, qui m'avait, en dehors de mes leçons d'escrime, appris quelques bottes gaillardes, me décidèrent à demander une casaque de mousquetaire. Le roi avait fort aimé mon père, tué au siège d'Arras, et l'on m'accorda cette casaque. Vous comprenez donc qu'aujourd'hui le moment est venu pour moi de rentrer dans le sein de l'église 

\speak  Et pourquoi aujourd'hui plutôt qu'hier et que demain? Que vous est-il donc arrivé aujourd'hui, qui vous donne de si méchantes idées? 

\speak  Cette blessure, mon cher d'Artagnan, m'a été un avertissement du Ciel. 

\speak  Cette blessure? bah! elle est à peu près guérie, et je suis sûr qu'aujourd'hui ce n'est pas celle-là qui vous fait le plus souffrir. 

\speak  Et laquelle? demanda Aramis en rougissant. 

\speak  Vous en avez une au cœur, Aramis, une plus vive et plus sanglante, une blessure faite par une femme.» 

L'œil d'Aramis étincela malgré lui. 

«Ah! dit-il en dissimulant son émotion sous une feinte négligence, ne parlez pas de ces choses-là; moi, penser à ces choses-là! avoir des chagrins d'amour? \textit{Vanitas vanitatum}! Me serais-je donc, à votre avis, retourné la cervelle, et pour qui? pour quelque grisette, pour quelque fille de chambre, à qui j'aurais fait la cour dans une garnison, fi! 

\speak  Pardon, mon cher Aramis, mais je croyais que vous portiez vos visées plus haut. 

\speak  Plus haut? et que suis-je pour avoir tant d'ambition? un pauvre mousquetaire fort gueux et fort obscur, qui hait les servitudes et se trouve grandement déplacé dans le monde! 

\speak  Aramis, Aramis! s'écria d'Artagnan en regardant son ami avec un air de doute. 

\speak  Poussière, je rentre dans la poussière. La vie est pleine d'humiliations et de douleurs, continua-t-il en s'assombrissant; tous les fils qui la rattachent au bonheur se rompent tour à tour dans la main de l'homme, surtout les fils d'or. O mon cher d'Artagnan! reprit Aramis en donnant à sa voix une légère teinte d'amertume, croyez-moi, cachez bien vos plaies quand vous en aurez. Le silence est la dernière joie des malheureux; gardez-vous de mettre qui que ce soit sur la trace de vos douleurs, les curieux pompent nos larmes comme les mouches font du sang d'un daim blessé. 

\speak  Hélas, mon cher Aramis, dit d'Artagnan en poussant à son tour un profond soupir, c'est mon histoire à moi-même que vous faites là. 

\speak  Comment? 

\speak  Oui, une femme que j'aimais, que j'adorais, vient de m'être enlevée de force. Je ne sais pas où elle est, où on l'a conduite; elle est peut-être prisonnière, elle est peut-être morte. 

\speak  Mais vous avez au moins la consolation de vous dire qu'elle ne vous a pas quitté volontairement; que si vous n'avez point de ses nouvelles, c'est que toute communication avec vous lui est interdite, tandis que\dots 

\speak  Tandis que\dots 

\speak  Rien, reprit Aramis, rien. 

\speak  Ainsi, vous renoncez à jamais au monde, c'est un parti pris, une résolution arrêtée? 

\speak  À tout jamais. Vous êtes mon ami aujourd'hui, demain vous ne serez plus pour moi qu'une ombre; où plutôt même, vous n'existerez plus. Quant au monde, c'est un sépulcre et pas autre chose. 

\speak  Diable! c'est fort triste ce que vous me dites là. 

\speak  Que voulez-vous! ma vocation m'attire, elle m'enlève. 

D'Artagnan sourit et ne répondit point. Aramis continua: 

«Et cependant, tandis que je tiens encore à la terre j'eusse voulu vous parler de vous, de nos amis. 

\speak  Et moi, dit d'Artagnan, j'eusse voulu vous parler de vous-même, mais je vous vois si détaché de tout; les amours, vous en faites fi; les amis sont des ombres, le monde est un sépulcre. 

\speak  Hélas! vous le verrez par vous-même, dit Aramis avec un soupir. 

\speak  N'en parlons donc plus, dit d'Artagnan, et brûlons cette lettre qui, sans doute, vous annonçait quelque nouvelle infidélité de votre grisette ou de votre fille de chambre. 

\speak  Quelle lettre? s'écria vivement Aramis. 

\speak  Une lettre qui était venue chez vous en votre absence et qu'on m'a remise pour vous. 

\speak  Mais de qui cette lettre? 

\speak  Ah! de quelque suivante éplorée, de quelque grisette au désespoir; la fille de chambre de Mme de Chevreuse peut-être, qui aura été obligée de retourner à Tours avec sa maîtresse, et qui, pour se faire pimpante, aura pris du papier parfumé et aura cacheté sa lettre avec une couronne de duchesse. 

\speak  Que dites-vous là? 

\speak  Tiens, je l'aurai perdue! dit sournoisement le jeune homme en faisant semblant de chercher. Heureusement que le monde est un sépulcre, que les hommes et par conséquent les femmes sont des ombres, que l'amour est un sentiment dont vous faites fi! 

\speak  Ah! d'Artagnan, d'Artagnan! s'écria Aramis, tu me fais mourir! 

\speak  Enfin, la voici!» dit d'Artagnan. 

Et il tira la lettre de sa poche. 

Aramis fit un bond, saisit la lettre, la lut ou plutôt la dévora, son visage rayonnait. 

«Il paraît que la suivante à un beau style, dit nonchalamment le messager. 

\speak  Merci, d'Artagnan! s'écria Aramis presque en délire. Elle a été forcée de retourner à Tours; elle ne m'est pas infidèle, elle m'aime toujours. Viens, mon ami, viens que je t'embrasse, le bonheur m'étouffe!» 

Et les deux amis se mirent à danser autour du vénérable saint Chrysostome, piétinant bravement les feuillets de la thèse qui avaient roulé sur le parquet. 

En ce moment, Bazin entrait avec les épinards et l'omelette. 

«Fuis, malheureux! s'écria Aramis en lui jetant sa calotte au visage; retourne d'où tu viens, remporte ces horribles légumes et cet affreux entremets! demande un lièvre piqué, un chapon gras, un gigot à l'ail et quatre bouteilles de vieux bourgogne.» 

Bazin, qui regardait son maître et qui ne comprenait rien à ce changement, laissa mélancoliquement glisser l'omelette dans les épinards, et les épinards sur le parquet. 

«Voilà le moment de consacrer votre existence au Roi des Rois, dit d'Artagnan, si vous tenez à lui faire une politesse: \textit{Non inutile desiderium in oblatione}. 

\speak  Allez-vous-en au diable avec votre latin! Mon cher d'Artagnan, buvons, morbleu, buvons frais, buvons beaucoup, et racontez-moi un peu ce qu'on fait là-bas.»