%!TeX root=../musketeersfr.tex 

\chapter{Le Cardinal}

\lettrine{L}{e} cardinal appuya son coude sur son manuscrit, sa joue sur sa main, et regarda un instant le jeune homme. Nul n'avait l'œil plus profondément scrutateur que le cardinal de Richelieu, et d'Artagnan sentit ce regard courir par ses veines comme une fièvre. 

Cependant il fit bonne contenance, tenant son feutre à la main, et attendant le bon plaisir de Son Éminence, sans trop d'orgueil, mais aussi sans trop d'humilité. 

«Monsieur, lui dit le cardinal, êtes-vous un d'Artagnan du Béarn? 

\speak  Oui, Monseigneur, répondit le jeune homme. 

\speak  Il y a plusieurs branches de d'Artagnan à Tarbes et dans les environs, dit le cardinal, à laquelle appartenez-vous? 

\speak  Je suis le fils de celui qui a fait les guerres de religion avec le grand roi Henri, père de Sa Gracieuse Majesté. 

\speak  C'est bien cela. C'est vous qui êtes parti, il y a sept à huit mois à peu près, de votre pays, pour venir chercher fortune dans la capitale? 

\speak  Oui, Monseigneur. 

\speak  Vous êtes venu par Meung, où il vous est arrivé quelque chose, je ne sais plus trop quoi, mais enfin quelque chose. 

Monseigneur, dit d'Artagnan, voici ce qui m'est arrivé\dots 

\speak  Inutile, inutile, reprit le cardinal avec un sourire qui indiquait qu'il connaissait l'histoire aussi bien que celui qui voulait la lui raconter; vous étiez recommandé à M. de Tréville, n'est-ce pas? 

\speak  Oui, Monseigneur; mais justement, dans cette malheureuse affaire de Meung\dots 

\speak  La lettre avait été perdue, reprit l'Éminence; oui, je sais cela; mais M. de Tréville est un habile physionomiste qui connaît les hommes à la première vue, et il vous a placé dans la compagnie de son beau-frère, M. des Essarts, en vous laissant espérer qu'un jour ou l'autre vous entreriez dans les mousquetaires. 

\speak  Monseigneur est parfaitement renseigné, dit d'Artagnan. 

Depuis ce temps-là, il vous est arrivé bien des choses: vous vous êtes promené derrière les Chartreux, un jour qu'il eût mieux valu que vous fussiez ailleurs; puis, vous avez fait avec vos amis un voyage aux eaux de Forges; eux se sont arrêtés en route; mais vous, vous avez continué votre chemin. C'est tout simple, vous aviez des affaires en Angleterre. 

\speak  Monseigneur, dit d'Artagnan tout interdit, j'allais\dots 

\speak  À la chasse, à Windsor, ou ailleurs, cela ne regarde personne. Je sais cela, moi, parce que mon état est de tout savoir. À votre retour, vous avez été reçu par une auguste personne, et je vois avec plaisir que vous avez conservé le souvenir qu'elle vous a donné.» 

\speak  D'Artagnan porta la main au diamant qu'il tenait de la reine, et en tourna vivement le chaton en dedans; mais il était trop tard. 

«Le lendemain de ce jour vous avez reçu la visite de Cavois, reprit le cardinal; il allait vous prier de passer au palais; cette visite vous ne la lui avez pas rendue, et vous avez eu tort. 

\speak  Monseigneur, je craignais d'avoir encouru la disgrâce de Votre Éminence. 

\speak  Eh! pourquoi cela, monsieur? pour avoir suivi les ordres de vos supérieurs avec plus d'intelligence et de courage que ne l'eût fait un autre, encourir ma disgrâce quand vous méritiez des éloges! Ce sont les gens qui n'obéissent pas que je punis, et non pas ceux qui, comme vous, obéissent\dots trop bien\dots Et, la preuve, rappelez-vous la date du jour où je vous avais fait dire de me venir voir, et cherchez dans votre mémoire ce qui est arrivé le soir même.» 

C'était le soir même qu'avait eu lieu l'enlèvement de Mme Bonacieux. D'Artagnan frissonna; et il se rappela qu'une demi-heure auparavant la pauvre femme était passée près de lui, sans doute encore emportée par la même puissance qui l'avait fait disparaître. 

«Enfin, continua le cardinal, comme je n'entendais pas parler de vous depuis quelque temps, j'ai voulu savoir ce que vous faisiez. D'ailleurs, vous me devez bien quelque remerciement: vous avez remarqué vous-même combien vous avez été ménagé dans toutes les circonstances. 

D'Artagnan s'inclina avec respect. 

«Cela, continua le cardinal, partait non seulement d'un sentiment d'équité naturelle, mais encore d'un plan que je m'étais tracé à votre égard. 

D'Artagnan était de plus en plus étonné. 

«Je voulais vous exposer ce plan le jour où vous reçûtes ma première invitation; mais vous n'êtes pas venu. Heureusement, rien n'est perdu pour ce retard, et aujourd'hui vous allez l'entendre. Asseyez-vous là, devant moi, monsieur d'Artagnan: vous êtes assez bon gentilhomme pour ne pas écouter debout.» 

Et le cardinal indiqua du doigt une chaise au jeune homme, qui était si étonné de ce qui se passait, que, pour obéir, il attendit un second signe de son interlocuteur. 

«Vous êtes brave, monsieur d'Artagnan, continua l'Éminence; vous êtes prudent, ce qui vaut mieux. J'aime les hommes de tête et de cœur, moi; ne vous effrayez pas, dit-il en souriant, par les hommes de cœur, j'entends les hommes de courage; mais, tout jeune que vous êtes, et à peine entrant dans le monde, vous avez des ennemis puissants: si vous n'y prenez garde, ils vous perdront! 

\speak  Hélas! Monseigneur, répondit le jeune homme, ils le feront bien facilement, sans doute; car ils sont forts et bien appuyés, tandis que moi je suis seul! 

\speak  Oui, c'est vrai; mais, tout seul que vous êtes, vous avez déjà fait beaucoup, et vous ferez encore plus, je n'en doute pas. Cependant, vous avez, je le crois, besoin d'être guidé dans l'aventureuse carrière que vous avez entreprise; car, si je ne me trompe, vous êtes venu à Paris avec l'ambitieuse idée de faire fortune. 

\speak  Je suis dans l'âge des folles espérances, Monseigneur, dit d'Artagnan. 

\speak  Il n'y a de folles espérances que pour les sots, monsieur, et vous êtes homme d'esprit. Voyons, que diriez-vous d'une enseigne dans mes gardes, et d'une compagnie après la campagne? 

\speak  Ah! Monseigneur! 

\speak  Vous acceptez, n'est-ce pas? 

\speak  Monseigneur, reprit d'Artagnan d'un air embarrassé. 

\speak  Comment, vous refusez? s'écria le cardinal avec étonnement. 

\speak  Je suis dans les gardes de Sa Majesté, Monseigneur, et je n'ai point de raisons d'être mécontent. 

\speak  Mais il me semble, dit l'Éminence, que mes gardes, à moi, sont aussi les gardes de Sa Majesté, et que, pourvu qu'on serve dans un corps français, on sert le roi. 

\speak  Monseigneur, Votre Éminence a mal compris mes paroles. 

\speak  Vous voulez un prétexte, n'est-ce pas? Je comprends. Eh bien, ce prétexte, vous l'avez. L'avancement, la campagne qui s'ouvre, l'occasion que je vous offre, voilà pour le monde; pour vous, le besoin de protections sûres; car il est bon que vous sachiez, monsieur d'Artagnan, que j'ai reçu des plaintes graves contre vous, vous ne consacrez pas exclusivement vos jours et vos nuits au service du roi.» 

D'Artagnan rougit. 

«Au reste, continua le cardinal en posant la main sur une liasse de papiers, j'ai là tout un dossier qui vous concerne; mais avant de le lire, j'ai voulu causer avec vous. Je vous sais homme de résolution et vos services bien dirigés, au lieu de vous mener à mal pourraient vous rapporter beaucoup. Allons, réfléchissez, et décidez-vous. 

\speak  Votre bonté me confond, Monseigneur, répondit d'Artagnan, et je reconnais dans Votre Éminence une grandeur d'âme qui me fait petit comme un ver de terre; mais enfin, puisque Monseigneur me permet de lui parler franchement\dots» 

D'Artagnan s'arrêta. 

«Oui, parlez. 

\speak  Eh bien, je dirai à Votre Éminence que tous mes amis sont aux mousquetaires et aux gardes du roi, et que mes ennemis, par une fatalité inconcevable, sont à Votre Éminence; je serais donc mal venu ici et mal regardé là-bas, si j'acceptais ce que m'offre Monseigneur. 

\speak  Auriez-vous déjà cette orgueilleuse idée que je ne vous offre pas ce que vous valez, monsieur? dit le cardinal avec un sourire de dédain. 

\speak  Monseigneur, Votre Éminence est cent fois trop bonne pour moi, et au contraire je pense n'avoir point encore fait assez pour être digne de ses bontés. Le siège de La Rochelle va s'ouvrir, Monseigneur; je servirai sous les yeux de Votre Éminence, et si j'ai le bonheur de me conduire à ce siège de telle façon que je mérite d'attirer ses regards, eh bien, après j'aurai au moins derrière moi quelque action d'éclat pour justifier la protection dont elle voudra bien m'honorer. Toute chose doit se faire à son temps, Monseigneur; peut-être plus tard aurai-je le droit de me donner, à cette heure j'aurais l'air de me vendre. 

\speak  C'est-à-dire que vous refusez de me servir, monsieur, dit le cardinal avec un ton de dépit dans lequel perçait cependant une sorte d'estime; demeurez donc libre et gardez vos haines et vos sympathies. 

\speak  Monseigneur\dots 

\speak  Bien, bien, dit le cardinal, je ne vous en veux pas, mais vous comprenez, on a assez de défendre ses amis et de les récompenser, on ne doit rien à ses ennemis, et cependant je vous donnerai un conseil: tenez-vous bien, monsieur d'Artagnan, car, du moment que j'aurai retiré ma main de dessus vous, je n'achèterai pas votre vie pour une obole. 

\speak  J'y tâcherai, Monseigneur, répondit le Gascon avec une noble assurance. 

\speak  Songez plus tard, et à un certain moment, s'il vous arrive malheur, dit Richelieu avec intention, que c'est moi qui ai été vous chercher, et que j'ai fait ce que j'ai pu pour que ce malheur ne vous arrivât pas. 

\speak  J'aurai, quoi qu'il arrive, dit d'Artagnan en mettant la main sur sa poitrine et en s'inclinant, une éternelle reconnaissance à Votre Éminence de ce qu'elle fait pour moi en ce moment. 

\speak  Eh bien donc! comme vous l'avez dit, monsieur d'Artagnan, nous nous reverrons après la campagne; je vous suivrai des yeux; car je serai là-bas, reprit le cardinal en montrant du doigt à d'Artagnan une magnifique armure qu'il devait endosser, et à notre retour, eh bien, nous compterons! 

\speak  Ah! Monseigneur, s'écria d'Artagnan, épargnez-moi le poids de votre disgrâce; restez neutre, Monseigneur, si vous trouvez que j'agis en galant homme. 

\speak  Jeune homme, dit Richelieu, si je puis vous dire encore une fois ce que je vous ai dit aujourd'hui, je vous promets de vous le dire.» 

Cette dernière parole de Richelieu exprimait un doute terrible; elle consterna d'Artagnan plus que n'eût fait une menace, car c'était un avertissement. Le cardinal cherchait donc à le préserver de quelque malheur qui le menaçait. Il ouvrit la bouche pour répondre, mais d'un geste hautain, le cardinal le congédia. 

D'Artagnan sortit; mais à la porte le cœur fut prêt à lui manquer, et peu s'en fallut qu'il ne rentrât. Cependant la figure grave et sévère d'Athos lui apparut: s'il faisait avec le cardinal le pacte que celui-ci lui proposait, Athos ne lui donnerait plus la main, Athos le renierait. 

Ce fut cette crainte qui le retint, tant est puissante l'influence d'un caractère vraiment grand sur tout ce qui l'entoure. 

D'Artagnan descendit par le même escalier qu'il était entré, et trouva devant la porte Athos et les quatre mousquetaires qui attendaient son retour et qui commençaient à s'inquiéter. D'un mot d'Artagnan les rassura, et Planchet courut prévenir les autres postes qu'il était inutile de monter une plus longue garde, attendu que son maître était sorti sain et sauf du Palais-Cardinal. 

Rentrés chez Athos, Aramis et Porthos s'informèrent des causes de cet étrange rendez-vous; mais d'Artagnan se contenta de leur dire que M. de Richelieu l'avait fait venir pour lui proposer d'entrer dans ses gardes avec le grade d'enseigne, et qu'il avait refusé. 

«Et vous avez eu raison», s'écrièrent d'une seule voix Porthos et Aramis. 

Athos tomba dans une profonde rêverie et ne répondit rien. Mais lorsqu'il fut seul avec d'Artagnan: 

«Vous avez fait ce que vous deviez faire, d'Artagnan, dit Athos, mais peut-être avez-vous eu tort.» 

D'Artagnan poussa un soupir; car cette voix répondait à une voix secrète de son âme, qui lui disait que de grands malheurs l'attendaient. 

La journée du lendemain se passa en préparatifs de départ; d'Artagnan alla faire ses adieux à M. de Tréville. À cette heure on croyait encore que la séparation des gardes et des mousquetaires serait momentanée, le roi tenant son parlement le jour même et devant partir le lendemain. M. de Tréville se contenta donc de demander à d'Artagnan s'il avait besoin de lui, mais d'Artagnan répondit fièrement qu'il avait tout ce qu'il lui fallait. 

La nuit réunit tous les camarades de la compagnie des gardes de M. des Essarts et de la compagnie des mousquetaires de M. de Tréville, qui avaient fait amitié ensemble. On se quittait pour se revoir quand il plairait à Dieu et s'il plaisait à Dieu. La nuit fut donc des plus bruyantes, comme on peut le penser, car, en pareil cas, on ne peut combattre l'extrême préoccupation que par l'extrême insouciance. 

Le lendemain, au premier son des trompettes, les amis se quittèrent: les mousquetaires coururent à l'hôtel de M. de Tréville, les gardes à celui de M. des Essarts. Chacun des capitaines conduisit aussitôt sa compagnie au Louvre, où le roi passait sa revue. 

Le roi était triste et paraissait malade, ce qui lui ôtait un peu de sa haute mine. En effet, la veille, la fièvre l'avait pris au milieu du parlement et tandis qu'il tenait son lit de justice. Il n'en était pas moins décidé à partir le soir même; et, malgré les observations qu'on lui avait faites, il avait voulu passer sa revue, espérant, par le premier coup de vigueur, vaincre la maladie qui commençait à s'emparer de lui. 

La revue passée, les gardes se mirent seuls en marche, les mousquetaires ne devant partir qu'avec le roi, ce qui permit à Porthos d'aller faire, dans son superbe équipage, un tour dans la rue aux Ours. 

La procureuse le vit passer dans son uniforme neuf et sur son beau cheval. Elle aimait trop Porthos pour le laisser partir ainsi; elle lui fit signe de descendre et de venir auprès d'elle. Porthos était magnifique; ses éperons résonnaient, sa cuirasse brillait, son épée lui battait fièrement les jambes. Cette fois les clercs n'eurent aucune envie de rire, tant Porthos avait l'air d'un coupeur d'oreilles. 

Le mousquetaire fut introduit près de M. Coquenard, dont le petit œil gris brilla de colère en voyant son cousin tout flambant neuf. Cependant une chose le consola intérieurement; c'est qu'on disait partout que la campagne serait rude: il espérait tout doucement, au fond du cœur, que Porthos y serait tué. 

Porthos présenta ses compliments à maître Coquenard et lui fit ses adieux; maître Coquenard lui souhaita toutes sortes de prospérités. Quant à Mme Coquenard, elle ne pouvait retenir ses larmes; mais on ne tira aucune mauvaise conséquence de sa douleur, on la savait fort attachée à ses parents, pour lesquels elle avait toujours eu de cruelles disputes avec son mari. 

Mais les véritables adieux se firent dans la chambre de Mme Coquenard: ils furent déchirants. 

Tant que la procureuse put suivre des yeux son amant, elle agita un mouchoir en se penchant hors de la fenêtre, à croire qu'elle voulait se précipiter. Porthos reçut toutes ces marques de tendresse en homme habitué à de pareilles démonstrations. Seulement, en tournant le coin de la rue, il souleva son feutre et l'agita en signe d'adieu. 

De son côté, Aramis écrivait une longue lettre. À qui? Personne n'en savait rien. Dans la chambre voisine, Ketty, qui devait partir le soir même pour Tours, attendait cette lettre mystérieuse. 

Athos buvait à petits coups la dernière bouteille de son vin d'Espagne. 

Pendant ce temps, d'Artagnan défilait avec sa compagnie. 

En arrivant au faubourg Saint-Antoine, il se retourna pour regarder gaiement la Bastille; mais, comme c'était la Bastille seulement qu'il regardait, il ne vit point Milady, qui, montée sur un cheval isabelle, le désignait du doigt à deux hommes de mauvaise mine qui s'approchèrent aussitôt des rangs pour le reconnaître. Sur une interrogation qu'ils firent du regard, Milady répondit par un signe que c'était bien lui. Puis, certaine qu'il ne pouvait plus y avoir de méprise dans l'exécution de ses ordres, elle piqua son cheval et disparut. 

Les deux hommes suivirent alors la compagnie, et, à la sortie du faubourg Saint-Antoine, montèrent sur des chevaux tout préparés qu'un domestique sans livrée tenait en les attendant.