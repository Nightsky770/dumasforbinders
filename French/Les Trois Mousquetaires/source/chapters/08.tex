%!TeX root=../musketeersfr.tex 

\chapter{Une Intrigue De Cœur}

\lettrine{C}{ependant} les quarante pistoles du roi Louis XIII, ainsi que toutes les choses de ce monde, après avoir eu un commencement avaient eu une fin, et depuis cette fin nos quatre compagnons étaient tombés dans la gêne. D'abord Athos avait soutenu pendant quelque temps l'association de ses propres deniers. Porthos lui avait succédé, et, grâce à une de ces disparitions auxquelles on était habitué, il avait pendant près de quinze jours encore subvenu aux besoins de tout le monde; enfin était arrivé le tour d'Aramis, qui s'était exécuté de bonne grâce, et qui était parvenu, disait-il, en vendant ses livres de théologie, à se procurer quelques pistoles. 

On eut alors, comme d'habitude, recours à M. de Tréville, qui fit quelques avances sur la solde; mais ces avances ne pouvaient conduire bien loin trois mousquetaires qui avaient déjà force comptes arriérés, et un garde qui n'en avait pas encore. 

Enfin, quand on vit qu'on allait manquer tout à fait, on rassembla par un dernier effort huit ou dix pistoles que Porthos joua. Malheureusement, il était dans une mauvaise veine: il perdit tout, plus vingt-cinq pistoles sur parole. 

Alors la gêne devint de la détresse, on vit les affamés suivis de leurs laquais courir les quais et les corps de garde, ramassant chez leurs amis du dehors tous les dîners qu'ils purent trouver; car, suivant l'avis d'Aramis, on devait dans la prospérité semer des repas à droite et à gauche pour en récolter quelques-uns dans la disgrâce. 

Athos fut invité quatre fois et mena chaque fois ses amis avec leurs laquais. Porthos eut six occasions et en fit également jouir ses camarades; Aramis en eut huit. C'était un homme, comme on a déjà pu s'en apercevoir, qui faisait peu de bruit et beaucoup de besogne. 

Quant à d'Artagnan, qui ne connaissait encore personne dans la capitale, il ne trouva qu'un déjeuner de chocolat chez un prêtre de son pays, et un dîner chez un cornette des gardes. Il mena son armée chez le prêtre, auquel on dévora sa provision de deux mois, et chez le cornette, qui fit des merveilles; mais, comme le disait Planchet, on ne mange toujours qu'une fois, même quand on mange beaucoup. 

D'Artagnan se trouva donc assez humilié de n'avoir eu qu'un repas et demi, car le déjeuner chez le prêtre ne pouvait compter que pour un demi-repas, à offrir à ses compagnons en échange des festins que s'étaient procurés Athos, Porthos et Aramis. Il se croyait à charge à la société, oubliant dans sa bonne foi toute juvénile qu'il avait nourri cette société pendant un mois, et son esprit préoccupé se mit à travailler activement. Il réfléchit que cette coalition de quatre hommes jeunes, braves, entreprenants et actifs devait avoir un autre but que des promenades déhanchées, des leçons d'escrime et des lazzi plus ou moins spirituels. 

En effet, quatre hommes comme eux, quatre hommes dévoués les uns aux autres depuis la bourse jusqu'à la vie, quatre hommes se soutenant toujours, ne reculant jamais, exécutant isolément ou ensemble les résolutions prises en commun; quatre bras menaçant les quatre points cardinaux ou se tournant vers un seul point, devaient inévitablement, soit souterrainement, soit au jour, soit par la mine, soit par la tranchée, soit par la ruse, soit par la force, s'ouvrir un chemin vers le but qu'ils voulaient atteindre, si bien défendu ou si éloigné qu'il fût. La seule chose qui étonnât d'Artagnan, c'est que ses compagnons n'eussent point songé à cela. 

Il y songeait, lui, et sérieusement même, se creusant la cervelle pour trouver une direction à cette force unique quatre fois multipliée avec laquelle il ne doutait pas que, comme avec le levier que cherchait Archimède, on ne parvînt à soulever le monde, --- lorsque l'on frappa doucement à la porte. D'Artagnan réveilla Planchet et lui ordonna d'aller ouvrir. 

Que de cette phrase: d'Artagnan réveilla Planchet, le lecteur n'aille pas augurer qu'il faisait nuit ou que le jour n'était point encore venu. Non! quatre heures venaient de sonner. Planchet, deux heures auparavant, était venu demander à dîner à son maître, lequel lui avait répondu par le proverbe: «Qui dort dîne.» Et Planchet dînait en dormant. 

Un homme fut introduit, de mine assez simple et qui avait l'air d'un bourgeois. 

Planchet, pour son dessert, eût bien voulu entendre la conversation; mais le bourgeois déclara à d'Artagnan que ce qu'il avait à lui dire étant important et confidentiel, il désirait demeurer en tête-à-tête avec lui. 

D'Artagnan congédia Planchet et fit asseoir son visiteur. 

Il y eut un moment de silence pendant lequel les deux hommes se regardèrent comme pour faire une connaissance préalable, après quoi d'Artagnan s'inclina en signe qu'il écoutait. 

«J'ai entendu parler de M. d'Artagnan comme d'un jeune homme fort brave, dit le bourgeois, et cette réputation dont il jouit à juste titre m'a décidé à lui confier un secret. 

\speak  Parlez, monsieur, parlez», dit d'Artagnan, qui d'instinct flaira quelque chose d'avantageux. 

Le bourgeois fit une nouvelle pause et continua: 

«J'ai ma femme qui est lingère chez la reine, monsieur, et qui ne manque ni de sagesse, ni de beauté. On me l'a fait épouser voilà bientôt trois ans, quoiqu'elle n'eût qu'un petit avoir, parce que M. de La Porte, le portemanteau de la reine, est son parrain et la protège\dots 

\speak  Eh bien, monsieur? demanda d'Artagnan. 

\speak  Eh bien, reprit le bourgeois, eh bien, monsieur, ma femme a été enlevée hier matin, comme elle sortait de sa chambre de travail. 

\speak  Et par qui votre femme a-t-elle été enlevée? 

\speak  Je n'en sais rien sûrement, monsieur, mais je soupçonne quelqu'un. 

\speak  Et quelle est cette personne que vous soupçonnez? 

\speak  Un homme qui la poursuivait depuis longtemps. 

\speak  Diable! 

\speak  Mais voulez-vous que je vous dise, monsieur, continua le bourgeois, je suis convaincu, moi, qu'il y a moins d'amour que de politique dans tout cela. 

\speak  Moins d'amour que de politique, reprit d'Artagnan d'un air fort réfléchi, et que soupçonnez-vous? 

\speak  Je ne sais pas si je devrais vous dire ce que je soupçonne\dots 

\speak  Monsieur, je vous ferai observer que je ne vous demande absolument rien, moi. C'est vous qui êtes venu. C'est vous qui m'avez dit que vous aviez un secret à me confier. Faites donc à votre guise, il est encore temps de vous retirer. 

\speak  Non, monsieur, non; vous m'avez l'air d'un honnête jeune homme, et j'aurai confiance en vous. Je crois donc que ce n'est pas à cause de ses amours que ma femme a été arrêtée, mais à cause de celles d'une plus grande dame qu'elle. 

\speak  Ah! ah! serait-ce à cause des amours de Mme de Bois-Tracy? fit d'Artagnan, qui voulut avoir l'air, vis-à-vis de son bourgeois, d'être au courant des affaires de la cour. 

\speak  Plus haut, monsieur, plus haut. 

\speak  De Mme d'Aiguillon? 

\speak  Plus haut encore. 

\speak  De Mme de Chevreuse? 

\speak  Plus haut, beaucoup plus haut! 

\speak  De la\dots d'Artagnan s'arrêta. 

\speak  Oui, monsieur, répondit si bas, qu'à peine si on put l'entendre, le bourgeois épouvanté. 

\speak  Et avec qui? 

\speak  Avec qui cela peut-il être, si ce n'est avec le duc de\dots 

\speak  Le duc de\dots 

\speak  Oui, monsieur! répondit le bourgeois, en donnant à sa voix une intonation plus sourde encore. 

\speak  Mais comment savez-vous tout cela, vous? 

\speak  Ah! comment je le sais? 

\speak  Oui, comment le savez-vous? Pas de demi-confidence, ou\dots vous comprenez. 

\speak  Je le sais par ma femme, monsieur, par ma femme elle-même. 

\speak  Qui le sait, elle, par qui? 

\speak  Par M. de La Porte. Ne vous ai-je pas dit qu'elle était la filleule de M. de La Porte, l'homme de confiance de la reine? Eh bien, M. de La Porte l'avait mise près de Sa Majesté pour que notre pauvre reine au moins eût quelqu'un à qui se fier, abandonnée comme elle l'est par le roi, espionnée comme elle l'est par le cardinal, trahie comme elle l'est par tous. 

\speak  Ah! ah! voilà qui se dessine, dit d'Artagnan. 

\speak  Or ma femme est venue il y a quatre jours, monsieur; une de ses conditions était qu'elle devait me venir voir deux fois la semaine; car, ainsi que j'ai eu l'honneur de vous le dire, ma femme m'aime beaucoup; ma femme est donc venue, et m'a confié que la reine, en ce moment-ci, avait de grandes craintes. 

\speak  Vraiment? 

\speak  Oui, M. le cardinal, à ce qu'il paraît, la poursuit et la persécute plus que jamais. Il ne peut pas lui pardonner l'histoire de la sarabande. Vous savez l'histoire de la sarabande? 

\speak  Pardieu, si je la sais! répondit d'Artagnan, qui ne savait rien du tout, mais qui voulait avoir l'air d'être au courant. 

\speak  De sorte que, maintenant, ce n'est plus de la haine, c'est de la vengeance. 

\speak  Vraiment? 

\speak  Et la reine croit\dots 

\speak  Eh bien, que croit la reine? 

\speak  Elle croit qu'on a écrit à M. le duc de Buckingham en son nom. 

\speak  Au nom de la reine? 

\speak  Oui, pour le faire venir à Paris, et une fois venu à Paris, pour l'attirer dans quelque piège. 

\speak  Diable! mais votre femme, mon cher monsieur, qu'a-t-elle à faire dans tout cela? 

\speak  On connaît son dévouement pour la reine, et l'on veut ou l'éloigner de sa maîtresse, ou l'intimider pour avoir les secrets de Sa Majesté, ou la séduire pour se servir d'elle comme d'un espion. 

\speak  C'est probable, dit d'Artagnan; mais l'homme qui l'a enlevée, le connaissez-vous? 

\speak  Je vous ai dit que je croyais le connaître. 

\speak  Son nom? 

\speak  Je ne le sais pas; ce que je sais seulement, c'est que c'est une créature du cardinal, son âme damnée. 

\speak  Mais vous l'avez vu? 

\speak  Oui, ma femme me l'a montré un jour. 

\speak  A-t-il un signalement auquel on puisse le reconnaître? 

\speak  Oh! certainement, c'est un seigneur de haute mine, poil noir, teint basané, œil perçant, dents blanches et une cicatrice à la tempe. 

\speak  Une cicatrice à la tempe! s'écria d'Artagnan, et avec cela dents blanches, œil perçant, teint basané, poil noir, et haute mine; c'est mon homme de Meung! 

\speak  C'est votre homme, dites-vous? 

\speak  Oui, oui; mais cela ne fait rien à la chose. Non, je me trompe, cela la simplifie beaucoup, au contraire: si votre homme est le mien, je ferai d'un coup deux vengeances, voilà tout; mais où rejoindre cet homme? 

\speak  Je n'en sais rien. 

\speak  Vous n'avez aucun renseignement sur sa demeure? 

\speak  Aucun; un jour que je reconduisais ma femme au Louvre, il en sortait comme elle allait y entrer, et elle me l'a fait voir. 

\speak  Diable! diable! murmura d'Artagnan, tout ceci est bien vague; par qui avez-vous su l'enlèvement de votre femme? 

\speak  Par M. de La Porte. 

\speak  Vous a-t-il donné quelque détail? 

\speak  Il n'en avait aucun. 

\speak  Et vous n'avez rien appris d'un autre côté? 

\speak  Si fait, j'ai reçu\dots 

\speak  Quoi? 

\speak  Mais je ne sais pas si je ne commets pas une grande imprudence? 

\speak  Vous revenez encore là-dessus; cependant je vous ferai observer que, cette fois, il est un peu tard pour reculer. 

\speak  Aussi je ne recule pas, mordieu! s'écria le bourgeois en jurant pour se monter la tête. D'ailleurs, foi de Bonacieux\dots 

\speak  Vous vous appelez Bonacieux? interrompit d'Artagnan. 

\speak  Oui, c'est mon nom. 

\speak  Vous disiez donc: foi de Bonacieux! pardon si je vous ai interrompu; mais il me semblait que ce nom ne m'était pas inconnu. 

\speak  C'est possible, monsieur. Je suis votre propriétaire. 

\speak  Ah! ah! fit d'Artagnan en se soulevant à demi et en saluant, vous êtes mon propriétaire? 

\speak  Oui, monsieur, oui. Et comme depuis trois mois que vous êtes chez moi, et que distrait sans doute par vos grandes occupations vous avez oublié de me payer mon loyer; comme, dis-je, je ne vous ai pas tourmenté un seul instant, j'ai pensé que vous auriez égard à ma délicatesse. 

\speak  Comment donc! mon cher monsieur Bonacieux, reprit d'Artagnan, croyez que je suis plein de reconnaissance pour un pareil procédé, et que, comme je vous l'ai dit, si je puis vous être bon à quelque chose\dots 

\speak  Je vous crois, monsieur, je vous crois, et comme j'allais vous le dire, foi de Bonacieux, j'ai confiance en vous. 

\speak  Achevez donc ce que vous avez commencé à me dire.» 

Le bourgeois tira un papier de sa poche, et le présenta à d'Artagnan. 

«Une lettre! fit le jeune homme. 

\speak  Que j'ai reçue ce matin.» 

D'Artagnan l'ouvrit, et comme le jour commençait à baisser, il s'approcha de la fenêtre. Le bourgeois le suivit. 

«Ne cherchez pas votre femme, lut d'Artagnan, elle vous sera rendue quand on n'aura plus besoin d'elle. Si vous faites une seule démarche pour la retrouver, vous êtes perdu.» 

«Voilà qui est positif, continua d'Artagnan; mais après tout, ce n'est qu'une menace. 

\speak  Oui, mais cette menace m'épouvante; moi, monsieur, je ne suis pas homme d'épée du tout, et j'ai peur de la Bastille. 

\speak  Hum! fit d'Artagnan; mais c'est que je ne me soucie pas plus de la Bastille que vous, moi. S'il ne s'agissait que d'un coup d'épée, passe encore. 

\speak  Cependant, monsieur, j'avais bien compté sur vous dans cette occasion. 

\speak  Oui? 

\speak  Vous voyant sans cesse entouré de mousquetaires à l'air fort superbe, et reconnaissant que ces mousquetaires étaient ceux de M. de Tréville, et par conséquent des ennemis du cardinal, j'avais pensé que vous et vos amis, tout en rendant justice à notre pauvre reine, seriez enchantés de jouer un mauvais tour à Son Éminence. 

\speak  Sans doute. 

\speak  Et puis j'avais pensé que, me devant trois mois de loyer dont je ne vous ai jamais parlé\dots 

\speak  Oui, oui, vous m'avez déjà donné cette raison, et je la trouve excellente. 

\speak  Comptant de plus, tant que vous me ferez l'honneur de rester chez moi, ne jamais vous parler de votre loyer à venir\dots 

\speak  Très bien. 

\speak  Et ajoutez à cela, si besoin est, comptant vous offrir une cinquantaine de pistoles si, contre toute probabilité, vous vous trouviez gêné en ce moment. 

\speak  À merveille; mais vous êtes donc riche, mon cher monsieur Bonacieux? 

\speak  Je suis à mon aise, monsieur, c'est le mot; j'ai amassé quelque chose comme deux ou trois mille écus de rente dans le commerce de la mercerie, et surtout en plaçant quelques fonds sur le dernier voyage du célèbre navigateur Jean Mocquet; de sorte que, vous comprenez, monsieur\dots Ah! mais\dots s'écria le bourgeois. 

\speak  Quoi? demanda d'Artagnan. 

\speak  Que vois-je là? 

\speak  Où? 

\speak  Dans la rue, en face de vos fenêtres, dans l'embrasure de cette porte: un homme enveloppé dans un manteau. 

\speak  C'est lui! s'écrièrent à la fois d'Artagnan et le bourgeois, chacun d'eux en même temps ayant reconnu son homme. 

\speak  Ah! cette fois-ci, s'écria d'Artagnan en sautant sur son épée, cette fois-ci, il ne m'échappera pas.» 

Et tirant son épée du fourreau, il se précipita hors de l'appartement. 

Sur l'escalier, il rencontra Athos et Porthos qui le venaient voir. Ils s'écartèrent, d'Artagnan passa entre eux comme un trait. 

«Ah çà, où cours-tu ainsi? lui crièrent à la fois les deux mousquetaires. 

\speak  L'homme de Meung!» répondit d'Artagnan, et il disparut. 

D'Artagnan avait plus d'une fois raconté à ses amis son aventure avec l'inconnu, ainsi que l'apparition de la belle voyageuse à laquelle cet homme avait paru confier une si importante missive. 

L'avis d'Athos avait été que d'Artagnan avait perdu sa lettre dans la bagarre. Un gentilhomme, selon lui --- et, au portrait que d'Artagnan avait fait de l'inconnu, ce ne pouvait être qu'un gentilhomme ---, un gentilhomme devait être incapable de cette bassesse, de voler une lettre. 

Porthos n'avait vu dans tout cela qu'un rendez-vous amoureux donné par une dame à un cavalier ou par un cavalier à une dame, et qu'était venu troubler la présence de d'Artagnan et de son cheval jaune. 

Aramis avait dit que ces sortes de choses étant mystérieuses, mieux valait ne les point approfondir. 

Ils comprirent donc, sur les quelques mots échappés à d'Artagnan, de quelle affaire il était question, et comme ils pensèrent qu'après avoir rejoint son homme ou l'avoir perdu de vue, d'Artagnan finirait toujours par remonter chez lui, ils continuèrent leur chemin. 

Lorsqu'ils entrèrent dans la chambre de d'Artagnan, la chambre était vide: le propriétaire, craignant les suites de la rencontre qui allait sans doute avoir lieu entre le jeune homme et l'inconnu, avait, par suite de l'exposition qu'il avait faite lui-même de son caractère, jugé qu'il était prudent de décamper.