%!TeX root=../musketeersfr.tex 

\chapter[Comment Athos Trouva Son Équipement]{Comment, Sans Se Déranger, Athos Trouva Son Équipement}

\lettrine{L}{e} jeune homme s'enfuit tandis qu'elle le menaçait encore d'un geste impuissant. Au moment où elle le perdit de vue, Milady tomba évanouie dans sa chambre. 

\zz
D'Artagnan était tellement bouleversé, que, sans s'inquiéter de ce que deviendrait Ketty, il traversa la moitié de Paris tout en courant, et ne s'arrêta que devant la porte d'Athos. L'égarement de son esprit, la terreur qui l'éperonnait, les cris de quelques patrouilles qui se mirent à sa poursuite, et les huées de quelques passants qui, malgré l'heure peu avancée, se rendaient à leurs affaires, ne firent que précipiter sa course. 

Il traversa la cour, monta les deux étages d'Athos et frappa à la porte à tout rompre. 

Grimaud vint ouvrir les yeux bouffis de sommeil. D'Artagnan s'élança avec tant de force dans l'antichambre qu'il faillit le culbuter en entrant. 

Malgré le mutisme habituel du pauvre garçon, cette fois la parole lui revint. 

«Hé, là, là! s'écria-t-il, que voulez-vous, coureuse? que demandez-vous, drôlesse?» 

D'Artagnan releva ses coiffes et dégagea ses mains de dessous son mantelet; à la vue de ses moustaches et de son épée nue, le pauvre diable s'aperçut qu'il avait affaire à un homme. 

Il crut alors que c'était quelque assassin. 

«Au secours! à l'aide! au secours! s'écria-t-il. 

\speak  Tais-toi, malheureux! dit le jeune homme, je suis d'Artagnan, ne me reconnais-tu pas? Où est ton maître? 

\speak  Vous, monsieur d'Artagnan! s'écria Grimaud épouvanté. Impossible. 

\speak  Grimaud, dit Athos sortant de son appartement en robe de chambre, je crois que vous vous permettez de parler. 

\speak  Ah! monsieur! c'est que\dots 

\speak  Silence.» 

Grimaud se contenta de montrer du doigt d'Artagnan à son maître. 

Athos reconnut son camarade, et, tout flegmatique qu'il était, il partit d'un éclat de rire que motivait bien la mascarade étrange qu'il avait sous les yeux: coiffes de travers, jupes tombantes sur les souliers; manches retroussées et moustaches raides d'émotion. 

«Ne riez pas, mon ami, s'écria d'Artagnan; de par le Ciel ne riez pas, car, sur mon âme, je vous le dis, il n'y a point de quoi rire.» 

Et il prononça ces mots d'un air si solennel et avec une épouvante si vraie qu'Athos lui prit aussitôt les mains en s'écriant: 

«Seriez-vous blessé, mon ami? vous êtes bien pâle! 

\speak  Non, mais il vient de m'arriver un terrible événement. Êtes-vous seul, Athos? 

\speak  Pardieu! qui voulez-vous donc qui soit chez moi à cette heure? 

\speak  Bien, bien.» 

Et d'Artagnan se précipita dans la chambre d'Athos. 

«Hé, parlez! dit celui-ci en refermant la porte et en poussant les verrous pour n'être pas dérangés. Le roi est-il mort? avez-vous tué M. le cardinal? vous êtes tout renversé; voyons, voyons, dites, car je meurs véritablement d'inquiétude. 

\speak  Athos, dit d'Artagnan se débarrassant de ses vêtements de femme et apparaissant en chemise, préparez-vous à entendre une histoire incroyable, inouïe. 

\speak  Prenez d'abord cette robe de chambre», dit le mousquetaire à son ami. 

D'Artagnan passa la robe de chambre, prenant une manche pour une autre tant il était encore ému. 

«Eh bien? dit Athos. 

\speak  Eh bien, répondit d'Artagnan en se courbant vers l'oreille d'Athos et en baissant la voix, Milady est marquée d'une fleur de lis à l'épaule. 

\speak  Ah! cria le mousquetaire comme s'il eût reçu une balle dans le cœur. 

\speak  Voyons, dit d'Artagnan, êtes-vous sûr que l'\textit{autre} soit bien morte? 

\speak  L'\textit{autre?} dit Athos d'une voix si sourde, qu'à peine si d'Artagnan l'entendit. 

\speak  Oui, celle dont vous m'avez parlé un jour à Amiens.» 

Athos poussa un gémissement et laissa tomber sa tête dans ses mains. 

«Celle-ci, continua d'Artagnan, est une femme de vingt-six à vingt-huit ans. 

\speak  Blonde, dit Athos, n'est-ce pas? 

\speak  Oui. 

\speak  Des yeux clairs, d'une clarté étrange, avec des cils et sourcils noirs? 

\speak  Oui. 

\speak  Grande, bien faite? Il lui manque une dent près de l'œillère gauche. 

\speak  Oui. 

\speak  La fleur de lis est petite, rousse de couleur et comme effacée par les couches de pâte qu'on y applique. 

\speak  Oui. 

\speak  Cependant vous dites qu'elle est anglaise! 

\speak  On l'appelle Milady, mais elle peut être française. Malgré cela, Lord de Winter n'est que son beau-frère. 

\speak  Je veux la voir, d'Artagnan. 

\speak  Prenez garde, Athos, prenez garde; vous avez voulu la tuer, elle est femme à vous rendre la pareille et à ne pas vous manquer. 

\speak  Elle n'osera rien dire, car ce serait se dénoncer elle-même. 

\speak  Elle est capable de tout! L'avez-vous jamais vue furieuse? 

\speak  Non, dit Athos. 

\speak  Une tigresse, une panthère! Ah! mon cher Athos! j'ai bien peur d'avoir attiré sur nous deux une vengeance terrible!» 

D'Artagnan raconta tout alors: la colère insensée de Milady et ses menaces de mort. 

«Vous avez raison, et, sur mon âme, je donnerais ma vie pour un cheveu, dit Athos. Heureusement, c'est après-demain que nous quittons Paris; nous allons, selon toute probabilité, à La Rochelle, et une fois partis\dots 

\speak  Elle vous suivra jusqu'au bout du monde, Athos, si elle vous reconnaît; laissez donc sa haine s'exercer sur moi seul. 

\speak  Ah! mon cher! que m'importe qu'elle me tue! dit Athos; est-ce que par hasard vous croyez que je tiens à la vie? 

\speak  Il y a quelque horrible mystère sous tout cela, Athos! cette femme est l'espion du cardinal, j'en suis sûr! 

\speak  En ce cas, prenez garde à vous. Si le cardinal ne vous a pas dans une haute admiration pour l'affaire de Londres, il vous a en grande haine; mais comme, au bout du compte, il ne peut rien vous reprocher ostensiblement, et qu'il faut que haine se satisfasse, surtout quand c'est une haine de cardinal, prenez garde à vous! Si vous sortez, ne sortez pas seul; si vous mangez, prenez vos précautions: méfiez-vous de tout enfin, même de votre ombre. 

\speak  Heureusement, dit d'Artagnan, qu'il s'agit seulement d'aller jusqu'à après-demain soir sans encombre, car une fois à l'armée nous n'aurons plus, je l'espère, que des hommes à craindre. 

\speak  En attendant, dit Athos, je renonce à mes projets de réclusion, et je vais partout avec vous: il faut que vous retourniez rue des Fossoyeurs, je vous accompagne. 

\speak  Mais si près que ce soit d'ici, reprit d'Artagnan, je ne puis y retourner comme cela. 

\speak  C'est juste», dit Athos. Et il tira la sonnette. 

Grimaud entra. 

Athos lui fit signe d'aller chez d'Artagnan, et d'en rapporter des habits. 

Grimaud répondit par un autre signe qu'il comprenait parfaitement et partit. 

«Ah çà! mais voilà qui ne nous avance pas pour l'équipement, cher ami, dit Athos; car, si je ne m'abuse, vous avez laissé toute votre défroque chez Milady, qui n'aura sans doute pas l'attention de vous la retourner. Heureusement que vous avez le saphir. 

\speak  Le saphir est à vous, mon cher Athos! ne m'avez-vous pas dit que c'était une bague de famille? 

\speak  Oui, mon père l'acheta deux mille écus, à ce qu'il me dit autrefois; il faisait partie des cadeaux de noces qu'il fit à ma mère; et il est magnifique. Ma mère me le donna, et moi, fou que j'étais, plutôt que de garder cette bague comme une relique sainte, je la donnai à mon tour à cette misérable. 

\speak  Alors, mon cher, reprenez cette bague, à laquelle je comprends que vous devez tenir. 

\speak  Moi, reprendre cette bague, après qu'elle a passé par les mains de l'infâme! jamais: cette bague est souillée, d'Artagnan. 

\speak  Vendez-la donc. 

\speak  Vendre un diamant qui vient de ma mère! je vous avoue que je regarderais cela comme une profanation. 

\speak  Alors engagez-la, on vous prêtera bien dessus un millier d'écus. Avec cette somme vous serez au-dessus de vos affaires, puis, au premier argent qui vous rentrera, vous la dégagerez, et vous la reprendrez lavée de ses anciennes taches, car elle aura passé par les mains des usuriers.» 

Athos sourit. 

«Vous êtes un charmant compagnon, dit-il, mon cher d'Artagnan; vous relevez par votre éternelle gaieté les pauvres esprits dans l'affliction. Eh bien, oui, engageons cette bague, mais à une condition! 

\speak  Laquelle? 

\speak  C'est qu'il y aura cinq cents écus pour vous et cinq cents écus pour moi. 

\speak  Y songez-vous, Athos? je n'ai pas besoin du quart de cette somme, moi qui suis dans les gardes, et en vendant ma selle je me la procurerai. Que me faut-il? Un cheval pour Planchet, voilà tout. Puis vous oubliez que j'ai une bague aussi. 

\speak  À laquelle vous tenez encore plus, ce me semble, que je ne tiens, moi, à la mienne; du moins j'ai cru m'en apercevoir. 

\speak  Oui, car dans une circonstance extrême elle peut nous tirer non seulement de quelque grand embarras mais encore de quelque grand danger; c'est non seulement un diamant précieux, mais c'est encore un talisman enchanté. 

\speak  Je ne vous comprends pas, mais je crois à ce que vous me dites. Revenons donc à ma bague, ou plutôt à la vôtre, vous toucherez la moitié de la somme qu'on nous donnera sur elle ou je la jette dans la Seine, et je doute que, comme à Polycrate, quelque poisson soit assez complaisant pour nous la rapporter. 

\speak  Eh bien, donc, j'accepte!» dit d'Artagnan. 

En ce moment Grimaud rentra accompagné de Planchet; celui-ci, inquiet de son maître et curieux de savoir ce qui lui était arrivé, avait profité de la circonstance et apportait les habits lui-même. 

D'Artagnan s'habilla, Athos en fit autant: puis quand tous deux furent prêts à sortir, ce dernier fit à Grimaud le signe d'un homme qui met en joue; celui-ci décrocha aussitôt son mousqueton et s'apprêta à accompagner son maître. 

Athos et d'Artagnan suivis de leurs valets arrivèrent sans incident à la rue des Fossoyeurs. Bonacieux était sur la porte, il regarda d'Artagnan d'un air goguenard. 

«Eh, mon cher locataire! dit-il, hâtez-vous donc, vous avez une belle jeune fille qui vous attend chez vous, et les femmes, vous le savez, n'aiment pas qu'on les fasse attendre! 

\speak  C'est Ketty!» s'écria d'Artagnan. 

Et il s'élança dans l'allée. 

Effectivement, sur le carré conduisant à sa chambre, et tapie contre sa porte, il trouva la pauvre enfant toute tremblante. Dès qu'elle l'aperçut: 

«Vous m'avez promis votre protection, vous m'avez promis de me sauver de sa colère, dit-elle; souvenez-vous que c'est vous qui m'avez perdue! 

\speak  Oui, sans doute, dit d'Artagnan, sois tranquille, Ketty. Mais qu'est-il arrivé après mon départ? 

\speak  Le sais-je? dit Ketty. Aux cris qu'elle a poussés, les laquais sont accourus; elle était folle de colère; tout ce qu'il existe d'imprécations elle les a vomies contre vous. Alors j'ai pensé qu'elle se rappellerait que c'était par ma chambre que vous aviez pénétré dans la sienne, et qu'alors elle songerait que j'étais votre complice; j'ai pris le peu d'argent que j'avais, mes hardes les plus précieuses, et je me suis sauvée. 

\speak  Pauvre enfant! Mais que vais-je faire de toi? Je pars après-demain. 

\speak  Tout ce que vous voudrez, Monsieur le chevalier, faites-moi quitter Paris, faites-moi quitter la France. 

\speak  Je ne puis cependant pas t'emmener avec moi au siège de La Rochelle, dit d'Artagnan. 

\speak  Non; mais vous pouvez me placer en province, chez quelque dame de votre connaissance: dans votre pays, par exemple. 

\speak  Ah! ma chère amie! dans mon pays les dames n'ont point de femmes de chambre. Mais, attends, j'ai ton affaire. Planchet, va me chercher Aramis: qu'il vienne tout de suite. Nous avons quelque chose de très important à lui dire. 

\speak  Je comprends, dit Athos; mais pourquoi pas Porthos? Il me semble que sa marquise\dots 

\speak  La marquise de Porthos se fait habiller par les clercs de son mari, dit d'Artagnan en riant. D'ailleurs Ketty ne voudrait pas demeurer rue aux Ours, n'est-ce pas, Ketty? 

\speak  Je demeurerai où l'on voudra, dit Ketty, pourvu que je sois bien cachée et que l'on ne sache pas où je suis. 

\speak  Maintenant, Ketty, que nous allons nous séparer, et par conséquent que tu n'es plus jalouse de moi\dots 

\speak  Monsieur le chevalier, de loin ou de près, dit Ketty, je vous aimerai toujours.» 

«Où diable la constance va-t-elle se nicher?» murmura Athos. 

«Moi aussi, dit d'Artagnan, moi aussi, je t'aimerai toujours, sois tranquille. Mais voyons, réponds-moi. Maintenant j'attache une grande importance à la question que je te fais: n'aurais-tu jamais entendu parler d'une jeune dame qu'on aurait enlevée pendant une nuit. 

\speak  Attendez donc\dots Oh! mon Dieu! monsieur le chevalier, est-ce que vous aimez encore cette femme? 

\speak  Non, c'est un de mes amis qui l'aime. Tiens, c'est Athos que voilà. 

\speak  Moi! s'écria Athos avec un accent pareil à celui d'un homme qui s'aperçoit qu'il va marcher sur une couleuvre. 

\speak  Sans doute, vous! fit d'Artagnan en serrant la main d'Athos. Vous savez bien l'intérêt que nous prenons tous à cette pauvre petite Mme Bonacieux. D'ailleurs Ketty ne dira rien: n'est-ce pas, Ketty? Tu comprends, mon enfant, continua d'Artagnan, c'est la femme de cet affreux magot que tu as vu sur le pas de la porte en entrant ici. 

\speak  Oh! mon Dieu! s'écria Ketty, vous me rappelez ma peur; pourvu qu'il ne m'ait pas reconnue! 

\speak  Comment, reconnue! tu as donc déjà vu cet homme? 

\speak  Il est venu deux fois chez Milady. 

\speak  C'est cela. Vers quelle époque? 

\speak  Mais il y a quinze ou dix-huit jours à peu près. 

\speak  Justement. 

\speak  Et hier soir il est revenu. 

\speak  Hier soir. 

\speak  Oui, un instant avant que vous vinssiez vous-même. 

\speak  Mon cher Athos, nous sommes enveloppés dans un réseau d'espions! Et tu crois qu'il t'a reconnue, Ketty? 

\speak  J'ai baissé ma coiffe en l'apercevant, mais peut-être était-il trop tard. 

\speak  Descendez, Athos, vous dont il se méfie moins que de moi, et voyez s'il est toujours sur sa porte.» 

Athos descendit et remonta bientôt. 

«Il est parti, dit-il, et la maison est fermée. 

\speak  Il est allé faire son rapport, et dire que tous les pigeons sont en ce moment au colombier. 

\speak  Eh bien, mais, envolons-nous, dit Athos, et ne laissons ici que Planchet pour nous rapporter les nouvelles. 

\speak  Un instant! Et Aramis que nous avons envoyé chercher! 

\speak  C'est juste, dit Athos, attendons Aramis. 

En ce moment Aramis entra. 

On lui exposa l'affaire, et on lui dit comment il était urgent que parmi toutes ses hautes connaissances il trouvât une place à Ketty. 

Aramis réfléchit un instant, et dit en rougissant: 

«Cela vous rendra-t-il bien réellement service, d'Artagnan? 

\speak  Je vous en serai reconnaissant toute ma vie. 

\speak  Eh bien, Mme de Bois-Tracy m'a demandé, pour une de ses amies qui habite la province, je crois, une femme de chambre sûre; et si vous pouvez, mon cher d'Artagnan, me répondre de mademoiselle\dots 

\speak  Oh! monsieur, s'écria Ketty, je serai toute dévouée, soyez-en certain, à la personne qui me donnera les moyens de quitter Paris. 

\speak  Alors, dit Aramis, cela va pour le mieux.» 

Il se mit à une table et écrivit un petit mot qu'il cacheta avec une bague, et donna le billet à Ketty. 

«Maintenant, mon enfant, dit d'Artagnan, tu sais qu'il ne fait pas meilleur ici pour nous que pour toi. Ainsi séparons-nous. Nous nous retrouverons dans des jours meilleurs. 

\speak  Et dans quelque temps que nous nous retrouvions et dans quelque lieu que ce soit, dit Ketty, vous me retrouverez vous aimant encore comme je vous aime aujourd'hui.» 

«Serment de joueur», dit Athos pendant que d'Artagnan allait reconduire Ketty sur l'escalier. 

Un instant après, les trois jeunes gens se séparèrent en prenant rendez-vous à quatre heures chez Athos et en laissant Planchet pour garder la maison. 

Aramis rentra chez lui, et Athos et d'Artagnan s'inquiétèrent du placement du saphir. 

Comme l'avait prévu notre Gascon, on trouva facilement trois cents pistoles sur la bague. De plus, le juif annonça que si on voulait la lui vendre, comme elle lui ferait un pendant magnifique pour des boucles d'oreilles, il en donnerait jusqu'à cinq cents pistoles. 

Athos et d'Artagnan, avec l'activité de deux soldats et la science de deux connaisseurs, mirent trois heures à peine à acheter tout l'équipement du mousquetaire. D'ailleurs Athos était de bonne composition et grand seigneur jusqu'au bout des ongles. Chaque fois qu'une chose lui convenait, il payait le prix demandé sans essayer même d'en rabattre. D'Artagnan voulait bien là-dessus faire ses observations, mais Athos lui posait la main sur l'épaule en souriant, et d'Artagnan comprenait que c'était bon pour lui, petit gentilhomme gascon, de marchander, mais non pour un homme qui avait les airs d'un prince. 

Le mousquetaire trouva un superbe cheval andalou, noir comme du jais, aux narines de feu, aux jambes fines et élégantes, qui prenait six ans. Il l'examina et le trouva sans défaut. On le lui fit mille livres. 

Peut-être l'eût-il eu pour moins; mais tandis que d'Artagnan discutait sur le prix avec le maquignon, Athos comptait les cent pistoles sur la table. 

Grimaud eut un cheval picard, trapu et fort, qui coûta trois cents livres. 

Mais la selle de ce dernier cheval et les armes de Grimaud achetées, il ne restait plus un sou des cent cinquante pistoles d'Athos. D'Artagnan offrit à son ami de mordre une bouchée dans la part qui lui revenait, quitte à lui rendre plus tard ce qu'il lui aurait emprunté. 

Mais Athos, pour toute réponse, se contenta de hausser les épaules. 

«Combien le juif donnait-il du saphir pour l'avoir en toute propriété? demanda Athos. 

\speak  Cinq cents pistoles. 

\speak  C'est-à-dire, deux cents pistoles de plus; cent pistoles pour vous, cent pistoles pour moi. Mais c'est une véritable fortune, cela, mon ami, retournez chez le juif. 

\speak  Comment, vous voulez\dots 

\speak  Cette bague, décidément, me rappellerait de trop tristes souvenirs; puis nous n'aurons jamais trois cents pistoles à lui rendre, de sorte que nous perdrions deux mille livres à ce marché. Allez lui dire que la bague est à lui, d'Artagnan, et revenez avec les deux cents pistoles. 

\speak  Réfléchissez, Athos. 

\speak  L'argent comptant est cher par le temps qui court, et il faut savoir faire des sacrifices. Allez, d'Artagnan, allez; Grimaud vous accompagnera avec son mousqueton.» 

Une demi-heure après, d'Artagnan revint avec les deux mille livres et sans qu'il lui fût arrivé aucun accident. 

Ce fut ainsi qu'Athos trouva dans son ménage des ressources auxquelles il ne s'attendait pas.