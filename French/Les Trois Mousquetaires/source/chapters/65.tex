%!TeX root=../musketeersfr.tex 

\chapter{Le Jugement} 
	
\lettrine{C}{'était} une nuit orageuse et sombre, de gros nuages couraient au ciel, voilant la clarté des étoiles; la lune ne devait se lever qu'à minuit. 

Parfois, à la lueur d'un éclair qui brillait à l'horizon, on apercevait la route qui se déroulait blanche et solitaire; puis, l'éclair éteint, tout rentrait dans l'obscurité. 

À chaque instant, Athos invitait d'Artagnan, toujours à la tête de la petite troupe, à reprendre son rang qu'au bout d'un instant il abandonnait de nouveau; il n'avait qu'une pensée, c'était d'aller en avant, et il allait. 

On traversa en silence le village de Festubert, où était resté le domestique blessé, puis on longea le bois de Richebourg; arrivés à Herlies, Planchet, qui dirigeait toujours la colonne, prit à gauche. 

Plusieurs fois, Lord de Winter, soit Porthos, soit Aramis, avaient essayé d'adresser la parole à l'homme au manteau rouge; mais à chaque interrogation qui lui avait été faite, il s'était incliné sans répondre. Les voyageurs avaient alors compris qu'il y avait quelque raison pour que l'inconnu gardât le silence, et ils avaient cessé de lui adresser la parole. 

D'ailleurs, l'orage grossissait, les éclairs se succédaient rapidement, le tonnerre commençait à gronder, et le vent, précurseur de l'ouragan, sifflait dans la plaine, agitant les plumes des cavaliers. 

La cavalcade prit le grand trot. 

Un peu au-delà de Fromelles, l'orage éclata; on déploya les manteaux; il restait encore trois lieues à faire: on les fit sous des torrents de pluie. 

D'Artagnan avait ôté son feutre et n'avait pas mis son manteau; il trouvait plaisir à laisser ruisseler l'eau sur son front brûlant et sur son corps agité de frissons fiévreux. 

Au moment où la petite troupe avait dépassé Goskal et allait arriver à la poste, un homme, abrité sous un arbre, se détacha du tronc avec lequel il était resté confondu dans l'obscurité, et s'avança jusqu'au milieu de la route, mettant son doigt sur ses lèvres. 

Athos reconnut Grimaud. 

«Qu'y a-t-il donc? s'écria d'Artagnan, aurait-elle quitté Armentières?» 

Grimaud fit de sa tête un signe affirmatif. D'Artagnan grinça des dents. 

«Silence, d'Artagnan! dit Athos, c'est moi qui me suis chargé de tout, c'est donc à moi d'interroger Grimaud. 

\speak  Où est-elle?» demanda Athos. 

Grimaud étendit la main dans la direction de la Lys. 

«Loin d'ici?» demanda Athos. 

Grimaud présenta à son maître son index plié. 

«Seule?» demanda Athos. 

Grimaud fit signe que oui. 

«Messieurs, dit Athos, elle est seule à une demi-lieue d'ici, dans la direction de la rivière. 

\speak  C'est bien, dit d'Artagnan, conduis-nous, Grimaud.» 

Grimaud prit à travers champs, et servit de guide à la cavalcade. 

Au bout de cinq cents pas à peu près, on trouva un ruisseau, que l'on traversa à gué. 

À la lueur d'un éclair, on aperçut le village d'Erquinghem. 

«Est-ce là?» demanda d'Artagnan. 

Grimaud secoua la tête en signe de négation. 

«Silence donc!» dit Athos. 

Et la troupe continua son chemin. 

Un autre éclair brilla; Grimaud étendit le bras, et à la lueur bleuâtre du serpent de feu on distingua une petite maison isolée, au bord de la rivière, à cent pas d'un bac. Une fenêtre était éclairée. 

«Nous y sommes», dit Athos. 

En ce moment, un homme couché dans le fossé se leva, c'était Mousqueton; il montra du doigt la fenêtre éclairée. 

«Elle est là, dit-il. 

\speak  Et Bazin? demanda Athos. 

\speak  Tandis que je gardais la fenêtre, il gardait la porte. 

\speak  Bien, dit Athos, vous êtes tous de fidèles serviteurs.» Athos sauta à bas de son cheval, dont il remit la bride aux mains de Grimaud, et s'avança vers la fenêtre après avoir fait signe au reste de la troupe de tourner du côté de la porte. 

La petite maison était entourée d'une haie vive, de deux ou trois pieds de haut. Athos franchit la haie, parvint jusqu'à la fenêtre privée de contrevents, mais dont les demi-rideaux étaient exactement tirés. 

Il monta sur le rebord de pierre, afin que son œil pût dépasser la hauteur des rideaux. 

À la lueur d'une lampe, il vit une femme enveloppée d'une mante de couleur sombre, assise sur un escabeau, près d'un feu mourant: ses coudes étaient posés sur une mauvaise table, et elle appuyait sa tête dans ses deux mains blanches comme l'ivoire. 

On ne pouvait distinguer son visage, mais un sourire sinistre passa sur les lèvres d'Athos, il n'y avait pas à s'y tromper, c'était bien celle qu'il cherchait. 

En ce moment un cheval hennit: Milady releva la tête, vit, collé à la vitre, le visage pâle d'Athos, et poussa un cri. 

Athos comprit qu'il était reconnu, poussa la fenêtre du genou et de la main, la fenêtre céda, les carreaux se rompirent. 

Et Athos, pareil au spectre de la vengeance, sauta dans la chambre. 

Milady courut à la porte et l'ouvrit; plus pâle et plus menaçant encore qu'Athos, d'Artagnan était sur le seuil. 

Milady recula en poussant un cri. D'Artagnan, croyant qu'elle avait quelque moyen de fuir et craignant qu'elle ne leur échappât, tira un pistolet de sa ceinture; mais Athos leva la main. 

«Remets cette arme à sa place, d'Artagnan, dit-il, il importe que cette femme soit jugée et non assassinée. Attends encore un instant, d'Artagnan, et tu seras satisfait. Entrez, messieurs.» 

D'Artagnan obéit, car Athos avait la voix solennelle et le geste puissant d'un juge envoyé par le Seigneur lui-même. Aussi, derrière d'Artagnan, entrèrent Porthos, Aramis, Lord de Winter et l'homme au manteau rouge. 

Les quatre valets gardaient la porte et la fenêtre. 

Milady était tombée sur sa chaise les mains étendues, comme pour conjurer cette terrible apparition; en apercevant son beau-frère, elle jeta un cri terrible. 

«Que demandez-vous? s'écria Milady. 

\speak  Nous demandons, dit Athos, Charlotte Backson, qui s'est appelée d'abord la comtesse de La Fère, puis Lady de Winter, baronne de Sheffield. 

\speak  C'est moi, c'est moi! murmura-t-elle au comble de la terreur, que me voulez-vous? 

\speak  Nous voulons vous juger selon vos crimes, dit Athos: vous serez libre de vous défendre, justifiez-vous si vous pouvez. Monsieur d'Artagnan, à vous d'accuser le premier.» 

D'Artagnan s'avança. 

«Devant Dieu et devant les hommes, dit-il, j'accuse cette femme d'avoir empoisonné Constance Bonacieux, morte hier soir.» 

Il se retourna vers Porthos et vers Aramis. 

«Nous attestons», dirent d'un seul mouvement les deux mousquetaires. 

D'Artagnan continua. 

«Devant Dieu et devant les hommes, j'accuse cette femme d'avoir voulu m'empoisonner moi-même, dans du vin qu'elle m'avait envoyé de Villeroi, avec une fausse lettre, comme si le vin venait de mes amis; Dieu m'a sauvé; mais un homme est mort à ma place, qui s'appelait Brisemont. 

\speak  Nous attestons, dirent de la même voix Porthos et Aramis. 

\speak  Devant Dieu et devant les hommes, j'accuse cette femme de m'avoir poussé au meurtre du baron de Wardes; et, comme personne n'est là pour attester la vérité de cette accusation, je l'atteste, moi. 

«J'ai dit.» 

Et d'Artagnan passa de l'autre côté de la chambre avec Porthos et Aramis. 

«À vous, Milord!» dit Athos. 

Le baron s'approcha à son tour. 

«Devant Dieu et devant les hommes, dit-il, j'accuse cette femme d'avoir fait assassiner le duc de Buckingham. 

\speak  Le duc de Buckingham assassiné? s'écrièrent d'un seul cri tous les assistants. 

\speak  Oui, dit le baron, assassiné! Sur la lettre d'avis que vous m'aviez écrite, j'avais fait arrêter cette femme, et je l'avais donnée en garde à un loyal serviteur; elle a corrompu cet homme, elle lui a mis le poignard dans la main, elle lui a fait tuer le duc, et dans ce moment peut-être Felton paie de sa tête le crime de cette furie.» 

Un frémissement courut parmi les juges à la révélation de ces crimes encore inconnus. 

«Ce n'est pas tout, reprit Lord de Winter, mon frère, qui vous avait faite son héritière, est mort en trois heures d'une étrange maladie qui laisse des taches livides sur tout le corps. Ma sœur, comment votre mari est-il mort? 

\speak  Horreur! s'écrièrent Porthos et Aramis. 

\speak  Assassin de Buckingham, assassin de Felton, assassin de mon frère, je demande justice contre vous, et je déclare que si on ne me la fait pas, je me la ferai.» 

Et Lord de Winter alla se ranger près de d'Artagnan, laissant la place libre à un autre accusateur. 

Milady laissa tomber son front dans ses deux mains et essaya de rappeler ses idées confondues par un vertige mortel. 

«À mon tour, dit Athos, tremblant lui-même comme le lion tremble à l'aspect du serpent, à mon tour. J'épousai cette femme quand elle était jeune fille, je l'épousai malgré toute ma famille; je lui donnai mon bien, je lui donnai mon nom; et un jour je m'aperçus que cette femme était flétrie: cette femme était marquée d'une fleur de lis sur l'épaule gauche. 

\speak  Oh! dit Milady en se levant, je défie de retrouver le tribunal qui a prononcé sur moi cette sentence infâme. Je défie de retrouver celui qui l'a exécutée. 

\speak  Silence, dit une voix. À ceci, c'est à moi de répondre!» 

Et l'homme au manteau rouge s'approcha à son tour. 

«Quel est cet homme, quel est cet homme?» s'écria Milady suffoquée par la terreur et dont les cheveux se dénouèrent et se dressèrent sur sa tête livide comme s'ils eussent été vivants. 

Tous les yeux se tournèrent sur cet homme, car à tous, excepté à Athos, il était inconnu. 

Encore Athos le regardait-il avec autant de stupéfaction que les autres, car il ignorait comment il pouvait se trouver mêlé en quelque chose à l'horrible drame qui se dénouait en ce moment. 

Après s'être approché de Milady, d'un pas lent et solennel, de manière que la table seule le séparât d'elle, l'inconnu ôta son masque. 

Milady regarda quelque temps avec une terreur croissante ce visage pâle encadré de cheveux et de favoris noirs, dont la seule expression était une impassibilité glacée, puis tout à coup: 

«Oh! non, non, dit-elle en se levant et en reculant jusqu'au mur; non, non, c'est une apparition infernale! ce n'est pas lui! à moi! à moi!» s'écria-t-elle d'une voix rauque en se retournant vers la muraille, comme si elle eût pu s'y ouvrir un passage avec ses mains. 

«Mais qui êtes-vous donc? s'écrièrent tous les témoins de cette scène. 

\speak  Demandez-le à cette femme, dit l'homme au manteau rouge, car vous voyez bien qu'elle m'a reconnu, elle. 

\speak  Le bourreau de Lille, le bourreau de Lille!» s'écria Milady en proie à une terreur insensée et se cramponnant des mains à la muraille pour ne pas tomber. 

Tout le monde s'écarta, et l'homme au manteau rouge resta seul debout au milieu de la salle. 

«Oh! grâce! grâce! pardon!» s'écria la misérable en tombant à genoux. 

L'inconnu laissa le silence se rétablir. 

«Je vous le disais bien qu'elle m'avait reconnu! reprit-il. Oui, je suis le bourreau de la ville de Lille, et voici mon histoire.» 

Tous les yeux étaient fixés sur cet homme dont on attendait les paroles avec une avide anxiété. 

«Cette jeune femme était autrefois une jeune fille aussi belle qu'elle est belle aujourd'hui. Elle était religieuse au couvent des bénédictines de Templemar. Un jeune prêtre au cœur simple et croyant desservait l'église de ce couvent; elle entreprit de le séduire et y réussit, elle eût séduit un saint. 

«Leurs vœux à tous deux étaient sacrés, irrévocables; leur liaison ne pouvait durer longtemps sans les perdre tous deux. Elle obtint de lui qu'ils quitteraient le pays; mais pour quitter le pays, pour fuir ensemble, pour gagner une autre partie de la France, où ils pussent vivre tranquilles parce qu'ils seraient inconnus, il fallait de l'argent; ni l'un ni l'autre n'en avait. Le prêtre vola les vases sacrés, les vendit; mais comme ils s'apprêtaient à partir ensemble, ils furent arrêtés tous deux. 

«Huit jours après, elle avait séduit le fils du geôlier et s'était sauvée. Le jeune prêtre fut condamné à dix ans de fers et à la flétrissure. J'étais le bourreau de la ville de Lille, comme dit cette femme. Je fus obligé de marquer le coupable, et le coupable, messieurs, c'était mon frère! 

«Je jurai alors que cette femme qui l'avait perdu, qui était plus que sa complice, puisqu'elle l'avait poussé au crime, partagerait au moins le châtiment. Je me doutai du lieu où elle était cachée, je la poursuivis, je l'atteignis, je la garrottai et lui imprimai la même flétrissure que j'avais imprimée à mon frère. 

«Le lendemain de mon retour à Lille, mon frère parvint à s'échapper à son tour, on m'accusa de complicité, et l'on me condamna à rester en prison à sa place tant qu'il ne se serait pas constitué prisonnier. Mon pauvre frère ignorait ce jugement; il avait rejoint cette femme, ils avaient fui ensemble dans le Berry; et là, il avait obtenu une petite cure. Cette femme passait pour sa sœur. 

«Le seigneur de la terre sur laquelle était située l'église du curé vit cette prétendue sœur et en devint amoureux, amoureux au point qu'il lui proposa de l'épouser. Alors elle quitta celui qu'elle avait perdu pour celui qu'elle devait perdre, et devint la comtesse de La Fère\dots» 

Tous les yeux se tournèrent vers Athos, dont c'était le véritable nom, et qui fit signe de la tête que tout ce qu'avait dit le bourreau était vrai. 

«Alors, reprit celui-ci, fou, désespéré, décidé à se débarrasser d'une existence à laquelle elle avait tout enlevé, honneur et bonheur, mon pauvre frère revint à Lille, et apprenant l'arrêt qui m'avait condamné à sa place, se constitua prisonnier et se pendit le même soir au soupirail de son cachot. 

«Au reste, c'est une justice à leur rendre, ceux qui m'avaient condamné me tinrent parole. À peine l'identité du cadavre fut-elle constatée qu'on me rendit ma liberté. 

«Voilà le crime dont je l'accuse, voilà la cause pour laquelle je l'ai marquée. 

\speak  Monsieur d'Artagnan, dit Athos, quelle est la peine que vous réclamez contre cette femme? 

\speak  La peine de mort, répondit d'Artagnan. 

\speak  Milord de Winter, continua Athos, quelle est la peine que vous réclamez contre cette femme? 

\speak  La peine de mort, reprit Lord de Winter. 

\speak  Messieurs Porthos et Aramis, reprit Athos, vous qui êtes ses juges, quelle est la peine que vous portez contre cette femme? 

\speak  La peine de mort», répondirent d'une voix sourde les deux mousquetaires. 

Milady poussa un hurlement affreux, et fit quelques pas vers ses juges en se traînant sur ses genoux. 

Athos étendit la main vers elle. 

«Anne de Breuil, comtesse de La Fère, Milady de Winter, dit-il, vos crimes ont lassé les hommes sur la terre et Dieu dans le ciel. Si vous savez quelque prière, dites-la, car vous êtes condamnée et vous allez mourir.» 

À ces paroles, qui ne lui laissaient aucun espoir, Milady se releva de toute sa hauteur et voulut parler, mais les forces lui manquèrent; elle sentit qu'une main puissante et implacable la saisissait par les cheveux et l'entraînait aussi irrévocablement que la fatalité entraîne l'homme: elle ne tenta donc pas même de faire résistance et sortit de la chaumière. 

Lord de Winter, d'Artagnan, Athos, Porthos et Aramis sortirent derrière elle. Les valets suivirent leurs maîtres et la chambre resta solitaire avec sa fenêtre brisée, sa porte ouverte et sa lampe fumeuse qui brûlait tristement sur la table. 
