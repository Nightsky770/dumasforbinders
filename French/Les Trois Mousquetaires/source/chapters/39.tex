%!TeX root=../musketeersfr.tex 

\chapter{Une Vision} 
	
\lettrine{\accentletter[\gravebox]{A}}{} quatre heures, les quatre amis étaient donc réunis chez Athos. Leurs préoccupations sur l'équipement avaient tout à fait disparu, et chaque visage ne conservait plus l'expression que de ses propres et secrètes inquiétudes; car derrière tout bonheur présent est cachée une crainte à venir. 

Tout à coup Planchet entra apportant deux lettres à l'adresse de d'Artagnan. 

L'une était un petit billet gentiment plié en long avec un joli cachet de cire verte sur lequel était empreinte une colombe rapportant un rameau vert. 

L'autre était une grande épître carrée et resplendissante des armes terribles de Son Éminence le cardinal-duc. 

À la vue de la petite lettre, le cœur de d'Artagnan bondit, car il avait cru reconnaître l'écriture; et quoiqu'il n'eût vu cette écriture qu'une fois, la mémoire en était restée au plus profond de son cœur. 

Il prit donc la petite épître et la décacheta vivement. 

«Promenez-vous, lui disait-on, mercredi prochain, de six heures à sept heures du soir, sur la route de Chaillot, et regardez avec soin dans les carrosses qui passeront, mais si vous tenez à votre vie et à celle des gens qui vous aiment, ne dites pas un mot, ne faites pas un mouvement qui puisse faire croire que vous avez reconnu celle qui s'expose à tout pour vous apercevoir un instant.» 

Pas de signature. 

«C'est un piège, dit Athos, n'y allez pas, d'Artagnan. 

\speak  Cependant, dit d'Artagnan, il me semble bien reconnaître l'écriture. 

\speak  Elle est peut-être contrefaite, reprit Athos; à six ou sept heures, dans ce temps-ci, la route de Chaillot est tout à fait déserte: autant que vous alliez vous promener dans la forêt de Bondy. 

\speak  Mais si nous y allions tous! dit d'Artagnan; que diable! on ne nous dévorera point tous les quatre; plus, quatre laquais; plus, les chevaux; plus, les armes. 

\speak  Puis ce sera une occasion de montrer nos équipages, dit Porthos. 

\speak  Mais si c'est une femme qui écrit, dit Aramis, et que cette femme désire ne pas être vue, songez que vous la compromettez, d'Artagnan: ce qui est mal de la part d'un gentilhomme. 

\speak  Nous resterons en arrière, dit Porthos, et lui seul s'avancera. 

\speak  Oui, mais un coup de pistolet est bientôt tiré d'un carrosse qui marche au galop. 

\speak  Bah! dit d'Artagnan, on me manquera. Nous rejoindrons alors le carrosse, et nous exterminerons ceux qui se trouvent dedans. Ce sera toujours autant d'ennemis de moins. 

\speak  Il a raison, dit Porthos; bataille; il faut bien essayer nos armes d'ailleurs. 

\speak  Bah! donnons-nous ce plaisir, dit Aramis de son air doux et nonchalant. 

\speak  Comme vous voudrez, dit Athos. 

\speak  Messieurs, dit d'Artagnan, il est quatre heures et demie, et nous avons le temps à peine d'être à six heures sur la route de Chaillot. 

\speak  Puis, si nous sortions trop tard, dit Porthos, on ne nous verrait pas, ce qui serait dommage. Allons donc nous apprêter, messieurs. 

\speak  Mais cette seconde lettre, dit Athos, vous l'oubliez; il me semble que le cachet indique cependant qu'elle mérite bien d'être ouverte: quant à moi, je vous déclare, mon cher d'Artagnan, que je m'en soucie bien plus que du petit brimborion que vous venez tout doucement de glisser sur votre cœur.» 

D'Artagnan rougit. 

«Eh bien, dit le jeune homme, voyons, messieurs, ce que me veut Son Éminence.» 

Et d'Artagnan décacheta la lettre et lut:

\begin{a4}
\vspace{-0.5cm}
\end{a4}

\begin{mail}{}{}
M. d'Artagnan, garde du roi, compagnie des Essarts, est attendu au Palais-Cardinal ce soir à huit heures. \closeletter{La Houdinière,\\ \textit{Capitaine des gardes.}}
\end{mail}

«Diable! dit Athos, voici un rendez-vous bien autrement inquiétant que l'autre. 

\speak  J'irai au second en sortant du premier, dit d'Artagnan: l'un est pour sept heures, l'autre pour huit; il y aura temps pour tout. 

\speak  Hum! je n'irais pas, dit Aramis: un galant chevalier ne peut manquer à un rendez-vous donné par une dame; mais un gentilhomme prudent peut s'excuser de ne pas se rendre chez Son Éminence, surtout lorsqu'il a quelque raison de croire que ce n'est pas pour y recevoir des compliments. 

\speak  Je suis de l'avis d'Aramis, dit Porthos. 

\speak  Messieurs, répondit d'Artagnan, j'ai déjà reçu par M. de Cavois pareille invitation de Son Éminence, je l'ai négligée, et le lendemain il m'est arrivé un grand malheur! Constance a disparu; quelque chose qui puisse advenir, j'irai. 

\speak  Si c'est un parti pris, dit Athos, faites. 

\speak  Mais la Bastille? dit Aramis. 

\speak  Bah! vous m'en tirerez, reprit d'Artagnan. 

\speak  Sans doute, reprirent Aramis et Porthos avec un aplomb admirable et comme si c'était la chose la plus simple, sans doute nous vous en tirerons; mais, en attendant, comme nous devons partir après-demain, vous feriez mieux de ne pas risquer cette Bastille. 

\speak  Faisons mieux, dit Athos, ne le quittons pas de la soirée, attendons-le chacun à une porte du palais avec trois mousquetaires derrière nous; si nous voyons sortir quelque voiture à portière fermée et à demi suspecte, nous tomberons dessus. Il y a longtemps que nous n'avons eu maille à partir avec les gardes de M. le cardinal, et M. de Tréville doit nous croire morts. 

\speak  Décidément, Athos, dit Aramis, vous étiez fait pour être général d'armée; que dites-vous du plan, messieurs? 

\speak  Admirable! répétèrent en choeur les jeunes gens. 

\speak  Eh bien, dit Porthos, je cours à l'hôtel, je préviens nos camarades de se tenir prêts pour huit heures, le rendez-vous sera sur la place du Palais-Cardinal; vous, pendant ce temps, faites seller les chevaux par les laquais. 

\speak  Mais moi, je n'ai pas de cheval, dit d'Artagnan; mais je vais en faire prendre un chez M. de Tréville. 

\speak  C'est inutile, dit Aramis, vous prendrez un des miens. 

\speak  Combien en avez-vous donc? demanda d'Artagnan. 

\speak  Trois, répondit en souriant Aramis. 

\speak  Mon cher! dit Athos, vous êtes certainement le poète le mieux monté de France et de Navarre. 

\speak  Écoutez, mon cher Aramis, vous ne saurez que faire de trois chevaux, n'est-ce pas? je ne comprends pas même que vous ayez acheté trois chevaux. 

\speak  Aussi, je n'en ai acheté que deux, dit Aramis. 

\speak  Le troisième vous est donc tombé du ciel? 

\speak  Non, le troisième m'a été amené ce matin même par un domestique sans livrée qui n'a pas voulu me dire à qui il appartenait et qui m'a affirmé avoir reçu l'ordre de son maître\dots 

\speak  Ou de sa maîtresse, interrompit d'Artagnan. 

\speak  La chose n'y fait rien, dit Aramis en rougissant\dots et qui m'a affirmé, dis-je, avoir reçu l'ordre de sa maîtresse de mettre ce cheval dans mon écurie sans me dire de quelle part il venait. 

\speak  Il n'y a qu'aux poètes que ces choses-là arrivent, reprit gravement Athos. 

\speak  Eh bien, en ce cas, faisons mieux, dit d'Artagnan; lequel des deux chevaux monterez-vous: celui que vous avez acheté, ou celui qu'on vous a donné? 

\speak  Celui que l'on m'a donné sans contredit; vous comprenez, d'Artagnan, que je ne puis faire cette injure\dots 

\speak  Au donateur inconnu, reprit d'Artagnan. 

\speak  Ou à la donatrice mystérieuse, dit Athos. 

\speak  Celui que vous avez acheté vous devient donc inutile? 

\speak  À peu près. 

\speak  Et vous l'avez choisi vous-même? 

\speak  Et avec le plus grand soin; la sûreté du cavalier, vous le savez, dépend presque toujours de son cheval! 

\speak  Eh bien, cédez-le-moi pour le prix qu'il vous a coûté! 

\speak  J'allais vous l'offrir, mon cher d'Artagnan, en vous donnant tout le temps qui vous sera nécessaire pour me rendre cette bagatelle. 

\speak  Et combien vous coûte-t-il? 

\speak  Huit cents livres. 

\speak  Voici quarante doubles pistoles, mon cher ami, dit d'Artagnan en tirant la somme de sa poche; je sais que c'est la monnaie avec laquelle on vous paie vos poèmes. 

\speak  Vous êtes donc en fonds? dit Aramis. 

\speak  Riche, richissime, mon cher!» 

Et d'Artagnan fit sonner dans sa poche le reste de ses pistoles. 

«Envoyez votre selle à l'Hôtel des Mousquetaires, et l'on vous amènera votre cheval ici avec les nôtres. 

\speak  Très bien; mais il est bientôt cinq heures, hâtons-nous.» 

Un quart d'heure après, Porthos apparut à un bout de la rue Férou sur un genet magnifique; Mousqueton le suivait sur un cheval d'Auvergne, petit, mais solide. Porthos resplendissait de joie et d'orgueil. 

En même temps Aramis apparut à l'autre bout de la rue monté sur un superbe coursier anglais; Bazin le suivait sur un cheval rouan, tenant en laisse un vigoureux mecklembourgeois: c'était la monture de d'Artagnan. 

Les deux mousquetaires se rencontrèrent à la porte: Athos et d'Artagnan les regardaient par la fenêtre. 

«Diable! dit Aramis, vous avez là un superbe cheval, mon cher Porthos. 

\speak  Oui, répondit Porthos; c'est celui qu'on devait m'envoyer tout d'abord: une mauvaise plaisanterie du mari lui a substitué l'autre; mais le mari a été puni depuis et j'ai obtenu toute satisfaction.» 

Planchet et Grimaud parurent alors à leur tour, tenant en main les montures de leurs maîtres; d'Artagnan et Athos descendirent, se mirent en selle près de leurs compagnons, et tous quatre se mirent en marche: Athos sur le cheval qu'il devait à sa femme, Aramis sur le cheval qu'il devait à sa maîtresse, Porthos sur le cheval qu'il devait à sa procureuse, et d'Artagnan sur le cheval qu'il devait à sa bonne fortune, la meilleure maîtresse qui soit. 

Les valets suivirent. 

Comme l'avait pensé Porthos, la cavalcade fit bon effet; et si Mme Coquenard s'était trouvée sur le chemin de Porthos et eût pu voir quel grand air il avait sur son beau genet d'Espagne, elle n'aurait pas regretté la saignée qu'elle avait faite au coffre-fort de son mari. 

Près du Louvre les quatre amis rencontrèrent M. de Tréville qui revenait de Saint-Germain; il les arrêta pour leur faire compliment sur leur équipage, ce qui en un instant amena autour d'eux quelques centaines de badauds. 

D'Artagnan profita de la circonstance pour parler à M. de Tréville de la lettre au grand cachet rouge et aux armes ducales; il est bien entendu que de l'autre il n'en souffla point mot. 

M. de Tréville approuva la résolution qu'il avait prise, et l'assura que, si le lendemain il n'avait pas reparu, il saurait bien le retrouver, lui, partout où il serait. 

En ce moment, l'horloge de la Samaritaine sonna six heures; les quatre amis s'excusèrent sur un rendez-vous, et prirent congé de M. de Tréville. 

Un temps de galop les conduisit sur la route de Chaillot; le jour commençait à baisser, les voitures passaient et repassaient; d'Artagnan, gardé à quelques pas par ses amis, plongeait ses regards jusqu'au fond des carrosses, et n'y apercevait aucune figure de connaissance. 

Enfin, après un quart d'heure d'attente et comme le crépuscule tombait tout à fait, une voiture apparut, arrivant au grand galop par la route de Sèvres; un pressentiment dit d'avance à d'Artagnan que cette voiture renfermait la personne qui lui avait donné rendez-vous: le jeune homme fut tout étonné lui-même de sentir son cœur battre si violemment. Presque aussitôt une tête de femme sortit par la portière, deux doigts sur la bouche, comme pour recommander le silence, ou comme pour envoyer un baiser; d'Artagnan poussa un léger cri de joie, cette femme, ou plutôt cette apparition, car la voiture était passée avec la rapidité d'une vision, était Mme Bonacieux. 

Par un mouvement involontaire, et malgré la recommandation faite, d'Artagnan lança son cheval au galop et en quelques bonds rejoignit la voiture; mais la glace de la portière était hermétiquement fermée: la vision avait disparu. 

D'Artagnan se rappela alors cette recommandation: «Si vous tenez à votre vie et à celle des personnes qui vous aiment, demeurez immobile et comme si vous n'aviez rien vu.» 

Il s'arrêta donc, tremblant non pour lui, mais pour la pauvre femme qui évidemment s'était exposée à un grand péril en lui donnant ce rendez-vous. 

La voiture continua sa route toujours marchant à fond de train, s'enfonça dans Paris et disparut. 

D'Artagnan était resté interdit à la même place et ne sachant que penser. Si c'était Mme Bonacieux et si elle revenait à Paris, pourquoi ce rendez-vous fugitif, pourquoi ce simple échange d'un coup d'œil, pourquoi ce baiser perdu? Si d'un autre côté ce n'était pas elle, ce qui était encore bien possible, car le peu de jour qui restait rendait une erreur facile, si ce n'était pas elle, ne serait-ce pas le commencement d'un coup de main monté contre lui avec l'appât de cette femme pour laquelle on connaissait son amour? 

Les trois compagnons se rapprochèrent de lui. Tous trois avaient parfaitement vu une tête de femme apparaître à la portière, mais aucun d'eux, excepté Athos, ne connaissait Mme Bonacieux. L'avis d'Athos, au reste, fut que c'était bien elle; mais moins préoccupé que d'Artagnan de ce joli visage, il avait cru voir une seconde tête, une tête d'homme au fond de la voiture. 

«S'il en est ainsi, dit d'Artagnan, ils la transportent sans doute d'une prison dans une autre. Mais que veulent-ils donc faire de cette pauvre créature, et comment la rejoindrai-je jamais? 

\speak  Ami, dit gravement Athos, rappelez-vous que les morts sont les seuls qu'on ne soit pas exposé à rencontrer sur la terre. Vous en savez quelque chose ainsi que moi, n'est-ce pas? Or, si votre maîtresse n'est pas morte, si c'est elle que nous venons de voir, vous la retrouverez un jour ou l'autre. Et peut-être, mon Dieu, ajouta-t-il avec un accent misanthropique qui lui était propre, peut être plus tôt que vous ne voudrez.» 

Sept heures et demie sonnèrent, la voiture était en retard d'une vingtaine de minutes sur le rendez-vous donné. Les amis de d'Artagnan lui rappelèrent qu'il avait une visite à faire, tout en lui faisant observer qu'il était encore temps de s'en dédire. 

Mais d'Artagnan était à la fois entêté et curieux. Il avait mis dans sa tête qu'il irait au Palais-Cardinal, et qu'il saurait ce que voulait lui dire Son Éminence. Rien ne put le faire changer de résolution. 

On arriva rue Saint-Honoré, et place du Palais-Cardinal on trouva les douze mousquetaires convoqués qui se promenaient en attendant leurs camarades. Là seulement, on leur expliqua ce dont il était question. 

D'Artagnan était fort connu dans l'honorable corps des mousquetaires du roi, où l'on savait qu'il prendrait un jour sa place; on le regardait donc d'avance comme un camarade. Il résulta de ces antécédents que chacun accepta de grand cœur la mission pour laquelle il était convié; d'ailleurs il s'agissait, selon toute probabilité, de jouer un mauvais tour à M. le cardinal et à ses gens, et pour de pareilles expéditions, ces dignes gentilshommes étaient toujours prêts. 

Athos les partagea donc en trois groupes, prit le commandement de l'un, donna le second à Aramis et le troisième à Porthos, puis chaque groupe alla s'embusquer en face d'une sortie. 

D'Artagnan, de son côté, entra bravement par la porte principale. 

Quoiqu'il se sentît vigoureusement appuyé, le jeune homme n'était pas sans inquiétude en montant pas à pas le grand escalier. Sa conduite avec Milady ressemblait tant soit peu à une trahison, et il se doutait des relations politiques qui existaient entre cette femme et le cardinal; de plus, de Wardes, qu'il avait si mal accommodé, était des fidèles de Son Éminence, et d'Artagnan savait que si Son Éminence était terrible à ses ennemis, elle était fort attachée à ses amis. 

«Si de Wardes a raconté toute notre affaire au cardinal, ce qui n'est pas douteux, et s'il m'a reconnu, ce qui est probable, je dois me regarder à peu près comme un homme condamné, disait d'Artagnan en secouant la tête. Mais pourquoi a-t-il attendu jusqu'aujourd'hui? C'est tout simple, Milady aura porté plainte contre moi avec cette hypocrite douleur qui la rend si intéressante, et ce dernier crime aura fait déborder le vase. 

«Heureusement, ajouta-t-il, mes bons amis sont en bas, et ils ne me laisseront pas emmener sans me défendre. Cependant la compagnie des mousquetaires de M. de Tréville ne peut pas faire à elle seule la guerre au cardinal, qui dispose des forces de toute la France, et devant lequel la reine est sans pouvoir et le roi sans volonté. D'Artagnan, mon ami, tu es brave, tu as d'excellentes qualités, mais les femmes te perdront!» 

Il en était à cette triste conclusion lorsqu'il entra dans l'antichambre. Il remit sa lettre à l'huissier de service qui le fit passer dans la salle d'attente et s'enfonça dans l'intérieur du palais. 

Dans cette salle d'attente étaient cinq ou six gardes de M. le cardinal, qui, reconnaissant d'Artagnan et sachant que c'était lui qui avait blessé Jussac, le regardèrent en souriant d'un singulier sourire. 

Ce sourire parut à d'Artagnan d'un mauvais augure; seulement, comme notre Gascon n'était pas facile à intimider, ou que plutôt, grâce à un grand orgueil naturel aux gens de son pays, il ne laissait pas voir facilement ce qui se passait dans son âme, quand ce qui s'y passait ressemblait à de la crainte, il se campa fièrement devant MM. les gardes et attendit la main sur la hanche, dans une attitude qui ne manquait pas de majesté. 

L'huissier rentra et fit signe à d'Artagnan de le suivre. Il sembla au jeune homme que les gardes, en le regardant s'éloigner, chuchotaient entre eux. 

Il suivit un corridor, traversa un grand salon, entra dans une bibliothèque, et se trouva en face d'un homme assis devant un bureau et qui écrivait. 

L'huissier l'introduisit et se retira sans dire une parole. D'Artagnan resta debout et examina cet homme. 

D'Artagnan crut d'abord qu'il avait affaire à quelque juge examinant son dossier, mais il s'aperçut que l'homme de bureau écrivait ou plutôt corrigeait des lignes d'inégales longueurs, en scandant des mots sur ses doigts; il vit qu'il était en face d'un poète. Au bout d'un instant, le poète ferma son manuscrit sur la couverture duquel était écrit: Mirame, \textit{tragédie en cinq actes}, et leva la tête. 

D'Artagnan reconnut le cardinal.