%!TeX root=../musketeersfr.tex 

\chapter{Le Bastion Saint-Gervais}

\lettrine{E}{n} arrivant chez ses trois amis, d'Artagnan les trouva réunis dans la même chambre: Athos réfléchissait, Porthos frisait sa moustache, Aramis disait ses prières dans un charmant petit livre d'heures relié en velours bleu. 

«Pardieu, messieurs! dit-il, j'espère que ce que vous avez à me dire en vaut la peine, sans cela je vous préviens que je ne vous pardonnerai pas de m'avoir fait venir, au lieu de me laisser reposer après une nuit passée à prendre et à démanteler un bastion. Ah! que n'étiez-vous là, messieurs! il y a fait chaud! 

\speak  Nous étions ailleurs, où il ne faisait pas froid non plus! répondit Porthos tout en faisant prendre à sa moustache un pli qui lui était particulier. 

\speak  Chut! dit Athos. 

\speak  Oh! oh! fit d'Artagnan comprenant le léger froncement de sourcils du mousquetaire, il paraît qu'il y a du nouveau ici. 

\speak  Aramis, dit Athos, vous avez été déjeuner avant-hier à l'auberge du Parpaillot, je crois? 

\speak  Oui. 

\speak  Comment est-on là? 

\speak  Mais, j'y ai fort mal mangé pour mon compte, avant-hier était un jour maigre, et ils n'avaient que du gras. 

\speak  Comment! dit Athos, dans un port de mer ils n'ont pas de poisson? 

\speak  Ils disent, reprit Aramis en se remettant à sa pieuse lecture, que la digue que fait bâtir M. le cardinal le chasse en pleine mer. 

\speak  Mais, ce n'est pas cela que je vous demandais, Aramis, reprit Athos; je vous demandais si vous aviez été bien libre, et si personne ne vous avait dérangé? 

\speak  Mais il me semble que nous n'avons pas eu trop d'importuns; oui, au fait, pour ce que vous voulez dire, Athos, nous serons assez bien au Parpaillot. 

\speak  Allons donc au Parpaillot, dit Athos, car ici les murailles sont comme des feuilles de papier.» 

D'Artagnan, qui était habitué aux manières de faire de son ami, et qui reconnaissait tout de suite à une parole, à un geste, à un signe de lui, que les circonstances étaient graves, prit le bras d'Athos et sortit avec lui sans rien dire; Porthos suivit en devisant avec Aramis. 

En route, on rencontra Grimaud, Athos lui fit signe de suivre; Grimaud, selon son habitude, obéit en silence; le pauvre garçon avait à peu près fini par désapprendre de parler. 

On arriva à la buvette du Parpaillot: il était sept heures du matin, le jour commençait à paraître; les trois amis commandèrent à déjeuner, et entrèrent dans une salle où au dire de l'hôte, ils ne devaient pas être dérangés. 

Malheureusement l'heure était mal choisie pour un conciliabule; on venait de battre la diane, chacun secouait le sommeil de la nuit, et, pour chasser l'air humide du matin, venait boire la goutte à la buvette: dragons, Suisses, gardes, mousquetaires, chevau-légers se succédaient avec une rapidité qui devait très bien faire les affaires de l'hôte, mais qui remplissait fort mal les vues des quatre amis. Aussi répondaient-ils d'une manière fort maussade aux saluts, aux toasts et aux \textit{lazzi} de leurs compagnons. 

«Allons! dit Athos, nous allons nous faire quelque bonne querelle, et nous n'avons pas besoin de cela en ce moment. D'Artagnan, racontez-nous votre nuit; nous vous raconterons la nôtre après. 

\speak  En effet, dit un chevau-léger qui se dandinait en tenant à la main un verre d'eau-de-vie qu'il dégustait lentement; en effet, vous étiez de tranchée cette nuit, messieurs les gardes, et il me semble que vous avez eu maille à partir avec les Rochelois?» 

D'Artagnan regarda Athos pour savoir s'il devait répondre à cet intrus qui se mêlait à la conversation. 

«Eh bien, dit Athos, n'entends-tu pas M. de Busigny qui te fait l'honneur de t'adresser la parole? Raconte ce qui s'est passé cette nuit, puisque ces messieurs désirent le savoir. 

\speak  N'avre-bous bas bris un pastion? demanda un Suisse qui buvait du rhum dans un verre à bière. 

\speak  Oui, monsieur, répondit d'Artagnan en s'inclinant, nous avons eu cet honneur, nous avons même, comme vous avez pu l'entendre, introduit sous un des angles un baril de poudre qui, en éclatant, a fait une fort jolie brèche; sans compter que, comme le bastion n'était pas d'hier, tout le reste de la bâtisse s'en est trouvé fort ébranlé. 

\speak  Et quel bastion est-ce? demanda un dragon qui tenait enfilée à son sabre une oie qu'il apportait pour qu'on la fît cuire. 

\speak  Le bastion Saint-Gervais, répondit d'Artagnan, derrière lequel les Rochelois inquiétaient nos travailleurs. 

\speak  Et l'affaire a été chaude? 

\speak  Mais, oui; nous y avons perdu cinq hommes, et les Rochelois huit ou dix. 

\speak  Balzampleu! fit le Suisse, qui, malgré l'admirable collection de jurons que possède la langue allemande, avait pris l'habitude de jurer en français. 

\speak  Mais il est probable, dit le chevau-léger, qu'ils vont, ce matin, envoyer des pionniers pour remettre le bastion en état. 

\speak  Oui, c'est probable, dit d'Artagnan. 

\speak  Messieurs, dit Athos, un pari! 

\speak  Ah! woui! un bari! dit le Suisse. 

\speak  Lequel? demanda le chevau-léger. 

\speak  Attendez, dit le dragon en posant son sabre comme une broche sur les deux grands chenets de fer qui soutenaient le feu de la cheminée, j'en suis. Hôtelier de malheur! une lèchefrite tout de suite, que je ne perde pas une goutte de la graisse de cette estimable volaille. 

\speak  Il avre raison, dit le Suisse, la graisse t'oie, il est très ponne avec des gonfitures. 

\speak  Là! dit le dragon. Maintenant, voyons le pari! Nous écoutons, monsieur Athos! 

\speak  Oui, le pari! dit le chevau-léger. 

\speak  Eh bien, monsieur de Busigny, je parie avec vous, dit Athos, que mes trois compagnons, MM. Porthos, Aramis, d'Artagnan et moi, nous allons déjeuner dans le bastion Saint-Gervais et que nous y tenons une heure, montre à la main, quelque chose que l'ennemi fasse pour nous déloger.» 

Porthos et Aramis se regardèrent, ils commençaient à comprendre. 

«Mais, dit d'Artagnan en se penchant à l'oreille d'Athos, tu vas nous faire tuer sans miséricorde. 

\speak  Nous sommes bien plus tués, répondit Athos, si nous n'y allons pas. 

\speak  Ah! ma foi! messieurs, dit Porthos en se renversant sur sa chaise et frisant sa moustache, voici un beau pari, j'espère. 

\speak  Aussi je l'accepte, dit M. de Busigny; maintenant il s'agit de fixer l'enjeu. 

\speak  Mais vous êtes quatre, messieurs, dit Athos, nous sommes quatre; un dîner à discrétion pour huit, cela vous va-t-il? 

\speak  À merveille, reprit M. de Busigny. 

\speak  Parfaitement, dit le dragon. 

\speak  Ça me fa», dit le Suisse. 

Le quatrième auditeur, qui, dans toute cette conversation, avait joué un rôle muet, fit un signe de la tête en signe qu'il acquiesçait à la proposition. 

«Le déjeuner de ces messieurs est prêt, dit l'hôte. 

\speak  Eh bien, apportez-le», dit Athos. 

L'hôte obéit. Athos appela Grimaud, lui montra un grand panier qui gisait dans un coin et fit le geste d'envelopper dans les serviettes les viandes apportées. 

Grimaud comprit à l'instant même qu'il s'agissait d'un déjeuner sur l'herbe, prit le panier, empaqueta les viandes, y joignit les bouteilles et prit le panier à son bras. 

«Mais où allez-vous manger mon déjeuner? dit l'hôte. 

\speak  Que vous importe, dit Athos, pourvu qu'on vous le paie?» 

Et il jeta majestueusement deux pistoles sur la table. 

«Faut-il vous rendre, mon officier? dit l'hôte. 

\speak  Non; ajoute seulement deux bouteilles de vin de Champagne et la différence sera pour les serviettes.» 

L'hôte ne faisait pas une aussi bonne affaire qu'il l'avait cru d'abord, mais il se rattrapa en glissant aux quatre convives deux bouteilles de vin d'Anjou au lieu de deux bouteilles de vin de Champagne. 

«Monsieur de Busigny, dit Athos, voulez-vous bien régler votre montre sur la mienne, ou me permettre de régler la mienne sur la vôtre? 

\speak  À merveille, monsieur! dit le chevau-léger en tirant de son gousset une fort belle montre entourée de diamants; sept heures et demie, dit-il. 

\speak  Sept heures trente-cinq minutes, dit Athos; nous saurons que j'avance de cinq minutes sur vous, monsieur.» 

Et, saluant les assistants ébahis, les quatre jeunes gens prirent le chemin du bastion Saint-Gervais, suivis de Grimaud, qui portait le panier, ignorant où il allait, mais, dans l'obéissance passive dont il avait pris l'habitude avec Athos, ne songeait pas même à le demander. 

Tant qu'ils furent dans l'enceinte du camp, les quatre amis n'échangèrent pas une parole; d'ailleurs ils étaient suivis par les curieux, qui, connaissant le pari engagé, voulaient savoir comment ils s'en tireraient. 

Mais une fois qu'ils eurent franchi la ligne de circonvallation et qu'ils se trouvèrent en plein air, d'Artagnan, qui ignorait complètement ce dont il s'agissait, crut qu'il était temps de demander une explication. 

«Et maintenant, mon cher Athos, dit-il, faites-moi l'amitié de m'apprendre où nous allons? 

\speak  Vous le voyez bien, dit Athos, nous allons au bastion. 

\speak  Mais qu'y allons-nous faire? 

\speak  Vous le savez bien, nous y allons déjeuner. 

\speak  Mais pourquoi n'avons-nous pas déjeuné au Parpaillot? 

\speak  Parce que nous avons des choses fort importantes à nous dire, et qu'il était impossible de causer cinq minutes dans cette auberge avec tous ces importuns qui vont, qui viennent, qui saluent, qui accostent; ici, du moins, continua Athos en montrant le bastion, on ne viendra pas nous déranger. 

\speak  Il me semble, dit d'Artagnan avec cette prudence qui s'alliait si bien et si naturellement chez lui à une excessive bravoure, il me semble que nous aurions pu trouver quelque endroit écarté dans les dunes, au bord de la mer. 

\speak  Où l'on nous aurait vus conférer tous les quatre ensemble, de sorte qu'au bout d'un quart d'heure le cardinal eût été prévenu par ses espions que nous tenions conseil. 

Oui, dit Aramis, Athos a raison: \textit{Animadvertuntur in desertis}. 

Un désert n'aurait pas été mal, dit Porthos, mais il s'agissait de le trouver. 

\speak  Il n'y a pas de désert où un oiseau ne puisse passer au-dessus de la tête, où un poisson ne puisse sauter au-dessus de l'eau, où un lapin ne puisse partir de son gîte, et je crois qu'oiseau, poisson, lapin, tout s'est fait espion du cardinal. Mieux vaut donc poursuivre notre entreprise, devant laquelle d'ailleurs nous ne pouvons plus reculer sans honte; nous avons fait un pari, un pari qui ne pouvait être prévu, et dont je défie qui que ce soit de deviner la véritable cause: nous allons, pour le gagner, tenir une heure dans le bastion. Ou nous serons attaqués, ou nous ne le serons pas. Si nous ne le sommes pas, nous aurons tout le temps de causer et personne ne nous entendra, car je réponds que les murs de ce bastion n'ont pas d'oreilles; si nous le sommes, nous causerons de nos affaires tout de même, et de plus, tout en nous défendant, nous nous couvrons de gloire. Vous voyez bien que tout est bénéfice. 

\speak  Oui, dit d'Artagnan, mais nous attraperons indubitablement une balle. 

\speak  Eh! mon cher, dit Athos, vous savez bien que les balles les plus à craindre ne sont pas celles de l'ennemi. 

\speak  Mais il me semble que pour une pareille expédition, nous aurions dû au moins emporter nos mousquets. 

\speak  Vous êtes un niais, ami Porthos; pourquoi nous charger d'un fardeau inutile? 

\speak  Je ne trouve pas inutile en face de l'ennemi un bon mousquet de calibre, douze cartouches et une poire à poudre. 

\speak  Oh! bien, dit Athos, n'avez-vous pas entendu ce qu'a dit d'Artagnan? 

\speak  Qu'a dit d'Artagnan? demanda Porthos. 

\speak  D'Artagnan a dit que dans l'attaque de cette nuit il y avait eu huit ou dix Français de tués et autant de Rochelois. 

\speak  Après? 

\speak  On n'a pas eu le temps de les dépouiller, n'est-ce pas? attendu qu'on avait autre chose pour le moment de plus pressé à faire. 

\speak  Eh bien? 

\speak  Eh bien, nous allons trouver leurs mousquets, leurs poires à poudre et leurs cartouches, et au lieu de quatre mousquetons et de douze balles, nous allons avoir une quinzaine de fusils et une centaine de coups à tirer. 

\speak  O Athos! dit Aramis, tu es véritablement un grand homme!» 

Porthos inclina la tête en signe d'adhésion. 

D'Artagnan seul ne paraissait pas convaincu. 

Sans doute Grimaud partageait les doutes du jeune homme; car, voyant que l'on continuait de marcher vers le bastion, chose dont il avait douté jusqu'alors, il tira son maître par le pan de son habit. 

«Où allons-nous?» demanda-t-il par geste. 

Athos lui montra le bastion. 

«Mais, dit toujours dans le même dialecte le silencieux Grimaud, nous y laisserons notre peau.» 

Athos leva les yeux et le doigt vers le ciel. 

Grimaud posa son panier à terre et s'assit en secouant la tête. 

Athos prit à sa ceinture un pistolet, regarda s'il était bien amorcé, l'arma et approcha le canon de l'oreille de Grimaud. 

Grimaud se retrouva sur ses jambes comme par un ressort. 

Athos alors lui fit signe de prendre le panier et de marcher devant. 

Grimaud obéit. 

Tout ce qu'avait gagné le pauvre garçon à cette pantomime d'un instant, c'est qu'il était passé de l'arrière-garde à l'avant-garde. 

Arrivés au bastion, les quatre amis se retournèrent. 

Plus de trois cents soldats de toutes armes étaient assemblés à la porte du camp, et dans un groupe séparé on pouvait distinguer M. de Busigny, le dragon, le Suisse et le quatrième parieur. 

Athos ôta son chapeau, le mit au bout de son épée et l'agita en l'air. 

Tous les spectateurs lui rendirent son salut, accompagnant cette politesse d'un grand hourra qui arriva jusqu'à eux. 

Après quoi, ils disparurent tous quatre dans le bastion, où les avait déjà précédés Grimaud. 