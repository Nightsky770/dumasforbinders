%!TeX root=../musketeersfr.tex 

\chapter[Où M. Séguier Chercha La Cloche]{Où M. Le Garde Des Sceaux Séguier Chercha Plus D'Une Fois La Cloche Pour La Sonner, Comme Il Le Faisait Autrefois}
	
\chaptermark{Où M. Séguier Chercha La Cloche}
	
\lettrine{I}{l} est impossible de se faire une idée de l'impression que ces quelques mots produisirent sur Louis XIII. Il rougit et pâlit successivement; et le cardinal vit tout d'abord qu'il venait de conquérir d'un seul coup tout le terrain qu'il avait perdu. 

«M. de Buckingham à Paris! s'écria-t-il, et qu'y vient-il faire? 

\speak  Sans doute conspirer avec nos ennemis les huguenots et les Espagnols. 

\speak  Non, pardieu, non! conspirer contre mon honneur avec Mme de Chevreuse, Mme de Longueville et les Condé! 

\speak  Oh! Sire, quelle idée! La reine est trop sage, et surtout aime trop Votre Majesté. 

\speak  La femme est faible, monsieur le cardinal, dit le roi; et quant à m'aimer beaucoup, j'ai mon opinion faite sur cet amour. 

\speak  Je n'en maintiens pas moins, dit le cardinal, que le duc de Buckingham est venu à Paris pour un projet tout politique. 

\speak  Et moi je suis sûr qu'il est venu pour autre chose, monsieur le cardinal; mais si la reine est coupable, qu'elle tremble! 

\speak  Au fait, dit le cardinal, quelque répugnance que j'aie à arrêter mon esprit sur une pareille trahison, Votre Majesté m'y fait penser: Mme de Lannoy, que, d'après l'ordre de Votre Majesté, j'ai interrogée plusieurs fois, m'a dit ce matin que la nuit avant celle-ci Sa Majesté avait veillé fort tard, que ce matin elle avait beaucoup pleuré et que toute la journée elle avait écrit. 

\speak  C'est cela, dit le roi; à lui sans doute, Cardinal, il me faut les papiers de la reine. 

\speak  Mais comment les prendre, Sire? Il me semble que ce n'est ni moi, ni Votre Majesté qui pouvons nous charger d'une pareille mission. 

\speak  Comment s'y est-on pris pour la maréchale d'Ancre? s'écria le roi au plus haut degré de la colère; on a fouillé ses armoires, et enfin on l'a fouillée elle-même. 

\speak  La maréchale d'Ancre n'était que la maréchale d'Ancre, une aventurière florentine, Sire, voilà tout; tandis que l'auguste épouse de Votre Majesté est Anne d'Autriche, reine de France, c'est-à-dire une des plus grandes princesses du monde. 

\speak  Elle n'en est que plus coupable, monsieur le duc! Plus elle a oublié la haute position où elle était placée, plus elle est bas descendue. Il y a longtemps d'ailleurs que je suis décidé à en finir avec toutes ces petites intrigues de politique et d'amour. Elle a aussi près d'elle un certain La Porte\dots 

\speak  Que je crois la cheville ouvrière de tout cela, je l'avoue, dit le cardinal. 

\speak  Vous pensez donc, comme moi, qu'elle me trompe? dit le roi. 

\speak  Je crois, et je le répète à Votre Majesté, que la reine conspire contre la puissance de son roi, mais je n'ai point dit contre son honneur. 

\speak  Et moi je vous dis contre tous deux; moi je vous dis que la reine ne m'aime pas; je vous dis qu'elle en aime un autre; je vous dis qu'elle aime cet infâme duc de Buckingham! Pourquoi ne l'avez-vous pas fait arrêter pendant qu'il était à Paris? 

\speak  Arrêter le duc! arrêter le premier ministre du roi Charles I\ier\! Y pensez-vous, Sire? Quel éclat! et si alors les soupçons de Votre Majesté, ce dont je continue à douter, avaient quelque consistance, quel éclat terrible! quel scandale désespérant! 

\speak  Mais puisqu'il s'exposait comme un vagabond et un larronneur, il fallait\dots» 

Louis XIII s'arrêta lui-même, effrayé de ce qu'il allait dire, tandis que Richelieu, allongeant le cou, attendait inutilement la parole qui était restée sur les lèvres du roi. 

«Il fallait? 

\speak  Rien, dit le roi, rien. Mais, pendant tout le temps qu'il a été à Paris, vous ne l'avez pas perdu de vue? 

\speak  Non, Sire. 

\speak  Où logeait-il? 

\speak  Rue de La Harpe, n° 75. 

\speak  Où est-ce, cela? 

\speak  Du côté du Luxembourg. 

\speak  Et vous êtes sûr que la reine et lui ne se sont pas vus? 

\speak  Je crois la reine trop attachée à ses devoirs, Sire. 

\speak  Mais ils ont correspondu, c'est à lui que la reine a écrit toute la journée; monsieur le duc, il me faut ces lettres! 

\speak  Sire, cependant\dots 

\speak  Monsieur le duc, à quelque prix que ce soit, je les veux. 

\speak  Je ferai pourtant observer à Votre Majesté\dots 

\speak  Me trahissez-vous donc aussi, monsieur le cardinal, pour vous opposer toujours ainsi à mes volontés? êtes-vous aussi d'accord avec l'Espagnol et avec l'Anglais, avec Mme de Chevreuse et avec la reine? 

\speak  Sire, répondit en soupirant le cardinal, je croyais être à l'abri d'un pareil soupçon. 

\speak  Monsieur le cardinal, vous m'avez entendu; je veux ces lettres. 

\speak  Il n'y aurait qu'un moyen. 

\speak  Lequel? 

\speak  Ce serait de charger de cette mission M. le garde des sceaux Séguier. La chose rentre complètement dans les devoirs de sa charge. 

\speak  Qu'on l'envoie chercher à l'instant même! 

\speak  Il doit être chez moi, Sire; je l'avais fait prier de passer, et lorsque je suis venu au Louvre, j'ai laissé l'ordre, s'il se présentait, de le faire attendre. 

\speak  Qu'on aille le chercher à l'instant même! 

\speak  Les ordres de Votre Majesté seront exécutés; mais\dots 

\speak  Mais quoi? 

\speak  Mais la reine se refusera peut-être à obéir. 

\speak  À mes ordres? 

\speak  Oui, si elle ignore que ces ordres viennent du roi. 

\speak  Eh bien, pour qu'elle n'en doute pas, je vais la prévenir moi-même. 

\speak  Votre Majesté n'oubliera pas que j'ai fait tout ce que j'ai pu pour prévenir une rupture. 

\speak  Oui, duc, je sais que vous êtes fort indulgent pour la reine, trop indulgent peut-être; et nous aurons, je vous en préviens, à parler plus tard de cela. 

\speak  Quand il plaira à Votre Majesté; mais je serai toujours heureux et fier, Sire, de me sacrifier à la bonne harmonie que je désire voir régner entre vous et la reine de France. 

\speak  Bien, cardinal, bien; mais en attendant envoyez chercher M. le garde des sceaux; moi, j'entre chez la reine. 

Et Louis XIII, ouvrant la porte de communication, s'engagea dans le corridor qui conduisait de chez lui chez Anne d'Autriche. 

La reine était au milieu de ses femmes, Mme de Guitaut, Mme de Sablé, Mme de Montbazon et Mme de Guéménée. Dans un coin était cette camériste espagnole doña Estefania, qui l'avait suivie de Madrid. Mme de Guéménée faisait la lecture, et tout le monde écoutait avec attention la lectrice, à l'exception de la reine, qui, au contraire, avait provoqué cette lecture afin de pouvoir, tout en feignant d'écouter, suivre le fil de ses propres pensées. 

Ces pensées, toutes dorées qu'elles étaient par un dernier reflet d'amour, n'en étaient pas moins tristes. Anne d'Autriche, privée de la confiance de son mari, poursuivie par la haine du cardinal, qui ne pouvait lui pardonner d'avoir repoussé un sentiment plus doux, ayant sous les yeux l'exemple de la reine mère, que cette haine avait tourmentée toute sa vie --- quoique Marie de Médicis, s'il faut en croire les mémoires du temps, eût commencé par accorder au cardinal le sentiment qu'Anne d'Autriche finit toujours par lui refuser ---, Anne d'Autriche avait vu tomber autour d'elle ses serviteurs les plus dévoués, ses confidents les plus intimes, ses favoris les plus chers. Comme ces malheureux doués d'un don funeste, elle portait malheur à tout ce qu'elle touchait, son amitié était un signe fatal qui appelait la persécution. Mme de Chevreuse et Mme de Vernel étaient exilées; enfin La Porte ne cachait pas à sa maîtresse qu'il s'attendait à être arrêté d'un instant à l'autre. 

C'est au moment où elle était plongée au plus profond et au plus sombre de ces réflexions, que la porte de la chambre s'ouvrit et que le roi entra. 

La lectrice se tut à l'instant même, toutes les dames se levèrent, et il se fit un profond silence. 

Quant au roi, il ne fit aucune démonstration de politesse; seulement, s'arrêtant devant la reine: 

«Madame, dit-il d'une voix altérée, vous allez recevoir la visite de M. le chancelier, qui vous communiquera certaines affaires dont je l'ai chargé.» 

La malheureuse reine, qu'on menaçait sans cesse de divorce, d'exil et de jugement même, pâlit sous son rouge et ne put s'empêcher de dire: 

«Mais pourquoi cette visite, Sire? Que me dira M. le chancelier que Votre Majesté ne puisse me dire elle-même?» 

Le roi tourna sur ses talons sans répondre, et presque au même instant le capitaine des gardes, M. de Guitaut, annonça la visite de M. le chancelier. 

Lorsque le chancelier parut, le roi était déjà sorti par une autre porte. 

Le chancelier entra demi-souriant, demi-rougissant. Comme nous le retrouverons probablement dans le cours de cette histoire, il n'y a pas de mal à ce que nos lecteurs fassent dès à présent connaissance avec lui. 

Ce chancelier était un plaisant homme. Ce fut Des Roches le Masle, chanoine à Notre-Dame, et qui avait été autrefois valet de chambre du cardinal, qui le proposa à Son Éminence comme un homme tout dévoué. Le cardinal s'y fia et s'en trouva bien. 

On racontait de lui certaines histoires, entre autres celle-ci: 

Après une jeunesse orageuse, il s'était retiré dans un couvent pour y expier au moins pendant quelque temps les folies de l'adolescence. 

Mais, en entrant dans ce saint lieu, le pauvre pénitent n'avait pu refermer si vite la porte, que les passions qu'il fuyait n'y entrassent avec lui. Il en était obsédé sans relâche, et le supérieur, auquel il avait confié cette disgrâce, voulant autant qu'il était en lui l'en garantir, lui avait recommandé pour conjurer le démon tentateur de recourir à la corde de la cloche et de sonner à toute volée. Au bruit dénonciateur, les moines seraient prévenus que la tentation assiégeait un frère, et toute la communauté se mettrait en prières. 

Le conseil parut bon au futur chancelier. Il conjura l'esprit malin à grand renfort de prières faites par les moines; mais le diable ne se laisse pas déposséder facilement d'une place où il a mis garnison; à mesure qu'on redoublait les exorcismes, il redoublait les tentations, de sorte que jour et nuit la cloche sonnait à toute volée, annonçant l'extrême désir de mortification qu'éprouvait le pénitent. 

Les moines n'avaient plus un instant de repos. Le jour, ils ne faisaient que monter et descendre les escaliers qui conduisaient à la chapelle; la nuit, outre complies et matines, ils étaient encore obligés de sauter vingt fois à bas de leurs lits et de se prosterner sur le carreau de leurs cellules. 

On ignore si ce fut le diable qui lâcha prise ou les moines qui se lassèrent; mais, au bout de trois mois, le pénitent reparut dans le monde avec la réputation du plus terrible possédé qui eût jamais existé. 

En sortant du couvent, il entra dans la magistrature, devint président à mortier à la place de son oncle, embrassa le parti du cardinal, ce qui ne prouvait pas peu de sagacité; devint chancelier, servit Son Éminence avec zèle dans sa haine contre la reine mère et sa vengeance contre Anne d'Autriche; stimula les juges dans l'affaire de Chalais, encouragea les essais de M. de Laffemas, grand gibecier de France; puis enfin, investi de toute la confiance du cardinal, confiance qu'il avait si bien gagnée, il en vint à recevoir la singulière commission pour l'exécution de laquelle il se présentait chez la reine. 

La reine était encore debout quand il entra, mais à peine l'eut-elle aperçu, qu'elle se rassit sur son fauteuil et fit signe à ses femmes de se rasseoir sur leurs coussins et leurs tabourets, et, d'un ton de suprême hauteur: 

«Que désirez-vous, monsieur, demanda Anne d'Autriche, et dans quel but vous présentez-vous ici? 

\speak  Pour y faire au nom du roi, madame, et sauf tout le respect que j'ai l'honneur de devoir à Votre Majesté, une perquisition exacte dans vos papiers. 

\speak  Comment, monsieur! une perquisition dans mes papiers\dots à moi! mais voilà une chose indigne! 

\speak  Veuillez me le pardonner, madame, mais, dans cette circonstance, je ne suis que l'instrument dont le roi se sert. Sa Majesté ne sort-elle pas d'ici, et ne vous a-t-elle pas invitée elle-même à vous préparer à cette visite? 

\speak  Fouillez donc, monsieur; je suis une criminelle, à ce qu'il paraît: Estefania, donnez les clefs de mes tables et de mes secrétaires.» 

Le chancelier fit pour la forme une visite dans les meubles, mais il savait bien que ce n'était pas dans un meuble que la reine avait dû serrer la lettre importante qu'elle avait écrite dans la journée. 

Quand le chancelier eut rouvert et refermé vingt fois les tiroirs du secrétaire, il fallut bien, quelque hésitation qu'il éprouvât, il fallut bien, dis-je, en venir à la conclusion de l'affaire, c'est-à-dire à fouiller la reine elle-même. Le chancelier s'avança donc vers Anne d'Autriche, et d'un ton très perplexe et d'un air fort embarrassé: 

«Et maintenant, dit-il, il me reste à faire la perquisition principale. 

\speak  Laquelle? demanda la reine, qui ne comprenait pas ou plutôt qui ne voulait pas comprendre. 

\speak  Sa Majesté est certaine qu'une lettre a été écrite par vous dans la journée; elle sait qu'elle n'a pas encore été envoyée à son adresse. Cette lettre ne se trouve ni dans votre table, ni dans votre secrétaire, et cependant cette lettre est quelque part. 

\speak  Oserez-vous porter la main sur votre reine? dit Anne d'Autriche en se dressant de toute sa hauteur et en fixant sur le chancelier ses yeux, dont l'expression était devenue presque menaçante. 

\speak  Je suis un fidèle sujet du roi, madame; et tout ce que Sa Majesté ordonnera, je le ferai. 

\speak  Eh bien, c'est vrai, dit Anne d'Autriche, et les espions de M. le cardinal l'ont bien servi. J'ai écrit aujourd'hui une lettre, cette lettre n'est point partie. La lettre est là.» 

Et la reine ramena sa belle main à son corsage. 

«Alors donnez-moi cette lettre, madame, dit le chancelier. 

\speak  Je ne la donnerai qu'au roi, monsieur, dit Anne. 

\speak  Si le roi eût voulu que cette lettre lui fût remise, madame, il vous l'eût demandée lui-même. Mais, je vous le répète, c'est moi qu'il a chargé de vous la réclamer, et si vous ne la rendiez pas\dots 

\speak  Eh bien? 

\speak  C'est encore moi qu'il a chargé de vous la prendre. 

\speak  Comment, que voulez-vous dire? 

\speak  Que mes ordres vont loin, madame, et que je suis autorisé à chercher le papier suspect sur la personne même de Votre Majesté. 

\speak  Quelle horreur! s'écria la reine. 

\speak  Veuillez donc, madame, agir plus facilement. 

\speak  Cette conduite est d'une violence infâme; savez-vous cela, monsieur? 

\speak  Le roi commande, madame, excusez-moi. 

\speak  Je ne le souffrirai pas; non, non, plutôt mourir!» s'écria la reine, chez laquelle se révoltait le sang impérieux de l'Espagnole et de l'Autrichienne. 

Le chancelier fit une profonde révérence, puis avec l'intention bien patente de ne pas reculer d'une semelle dans l'accomplissement de la commission dont il s'était chargé, et comme eût pu le faire un valet de bourreau dans la chambre de la question, il s'approcha d'Anne d'Autriche des yeux de laquelle on vit à l'instant même jaillir des pleurs de rage. 

La reine était, comme nous l'avons dit, d'une grande beauté. 

La commission pouvait donc passer pour délicate, et le roi en était arrivé, à force de jalousie contre Buckingham, à n'être plus jaloux de personne. 

Sans doute le chancelier Séguier chercha des yeux à ce moment le cordon de la fameuse cloche; mais, ne le trouvant pas, il en prit son parti et tendit la main vers l'endroit où la reine avait avoué que se trouvait le papier. 

Anne d'Autriche fit un pas en arrière, si pâle qu'on eût dit qu'elle allait mourir; et, s'appuyant de la main gauche, pour ne pas tomber, à une table qui se trouvait derrière elle, elle tira de la droite un papier de sa poitrine et le tendit au garde des sceaux. 

«Tenez, monsieur, la voilà, cette lettre, s'écria la reine d'une voix entrecoupée et frémissante, prenez-la, et me délivrez de votre odieuse présence.» 

Le chancelier, qui de son côté tremblait d'une émotion facile à concevoir, prit la lettre, salua jusqu'à terre et se retira. 

À peine la porte se fut-elle refermée sur lui, que la reine tomba à demi évanouie dans les bras de ses femmes. 

Le chancelier alla porter la lettre au roi sans en avoir lu un seul mot. Le roi la prit d'une main tremblante, chercha l'adresse, qui manquait, devint très pâle, l'ouvrit lentement, puis, voyant par les premiers mots qu'elle était adressée au roi d'Espagne, il lut très rapidement. 

C'était tout un plan d'attaque contre le cardinal. La reine invitait son frère et l'empereur d'Autriche à faire semblant, blessés qu'ils étaient par la politique de Richelieu, dont l'éternelle préoccupation fut l'abaissement de la maison d'Autriche, de déclarer la guerre à la France et d'imposer comme condition de la paix le renvoi du cardinal: mais d'amour, il n'y en avait pas un seul mot dans toute cette lettre. 

Le roi, tout joyeux, s'informa si le cardinal était encore au Louvre. On lui dit que Son Éminence attendait, dans le cabinet de travail, les ordres de Sa Majesté. 

Le roi se rendit aussitôt près de lui. 

«Tenez, duc, lui dit-il, vous aviez raison, et c'est moi qui avais tort; toute l'intrigue est politique, et il n'était aucunement question d'amour dans cette lettre, que voici. En échange, il y est fort question de vous.» 

Le cardinal prit la lettre et la lut avec la plus grande attention; puis, lorsqu'il fut arrivé au bout, il la relut une seconde fois. 

«Eh bien, Votre Majesté, dit-il, vous voyez jusqu'où vont mes ennemis: on vous menace de deux guerres, si vous ne me renvoyez pas. À votre place, en vérité, Sire, je céderais à de si puissantes instances, et ce serait de mon côté avec un véritable bonheur que je me retirerais des affaires. 

\speak  Que dites-vous là, duc? 

\speak  Je dis, Sire, que ma santé se perd dans ces luttes excessives et dans ces travaux éternels. Je dis que, selon toute probabilité, je ne pourrai pas soutenir les fatigues du siège de La Rochelle, et que mieux vaut que vous nommiez là ou M. de Condé, ou M. de Bassompierre, ou enfin quelque vaillant homme dont c'est l'état de mener la guerre, et non pas moi qui suis homme d'Église et qu'on détourne sans cesse de ma vocation pour m'appliquer à des choses auxquelles je n'ai aucune aptitude. Vous en serez plus heureux à l'intérieur, Sire, et je ne doute pas que vous n'en soyez plus grand à l'étranger. 

\speak  Monsieur le duc, dit le roi, je comprends, soyez tranquille; tous ceux qui sont nommés dans cette lettre seront punis comme ils le méritent, et la reine elle-même. 

\speak  Que dites-vous là, Sire? Dieu me garde que, pour moi, la reine éprouve la moindre contrariété! elle m'a toujours cru son ennemi, Sire, quoique Votre Majesté puisse attester que j'ai toujours pris chaudement son parti, même contre vous. Oh! si elle trahissait Votre Majesté à l'endroit de son honneur, ce serait autre chose, et je serais le premier à dire: «Pas de grâce, Sire, pas de grâce pour la coupable!» Heureusement il n'en est rien, et Votre Majesté vient d'en acquérir une nouvelle preuve. 

\speak  C'est vrai, monsieur le cardinal, dit le roi, et vous aviez raison, comme toujours; mais la reine n'en mérite pas moins toute ma colère. 

\speak  C'est vous, Sire, qui avez encouru la sienne; et véritablement, quand elle bouderait sérieusement Votre Majesté, je le comprendrais; Votre Majesté l'a traitée avec une sévérité!\dots 

\speak  C'est ainsi que je traiterai toujours mes ennemis et les vôtres, duc, si haut placés qu'ils soient et quelque péril que je coure à agir sévèrement avec eux. 

\speak  La reine est mon ennemie, mais n'est pas la vôtre, Sire; au contraire, elle est épouse dévouée, soumise et irréprochable; laissez-moi donc, Sire, intercéder pour elle près de Votre Majesté. 

\speak  Qu'elle s'humilie alors, et qu'elle revienne à moi la première! 

\speak  Au contraire, Sire, donnez l'exemple; vous avez eu le premier tort, puisque c'est vous qui avez soupçonné la reine. 

\speak  Moi, revenir le premier? dit le roi; jamais! 

\speak  Sire, je vous en supplie. 

\speak  D'ailleurs, comment reviendrais-je le premier? 

\speak  En faisant une chose que vous sauriez lui être agréable. 

\speak  Laquelle? 

\speak  Donnez un bal; vous savez combien la reine aime la danse; je vous réponds que sa rancune ne tiendra point à une pareille attention. 

\speak  Monsieur le cardinal, vous savez que je n'aime pas tous les plaisirs mondains. 

\speak  La reine ne vous en sera que plus reconnaissante, puisqu'elle sait votre antipathie pour ce plaisir; d'ailleurs ce sera une occasion pour elle de mettre ces beaux ferrets de diamants que vous lui avez donnés l'autre jour à sa fête, et dont elle n'a pas encore eu le temps de se parer. 

\speak  Nous verrons, monsieur le cardinal, nous verrons, dit le roi, qui, dans sa joie de trouver la reine coupable d'un crime dont il se souciait peu, et innocente d'une faute qu'il redoutait fort, était tout prêt à se raccommoder avec elle; nous verrons, mais, sur mon honneur, vous êtes trop indulgent. 

\speak  Sire, dit le cardinal, laissez la sévérité aux ministres, l'indulgence est la vertu royale; usez-en, et vous verrez que vous vous en trouverez bien.» 

Sur quoi le cardinal, entendant la pendule sonner onze heures, s'inclina profondément, demandant congé au roi pour se retirer, et le suppliant de se raccommoder avec la reine. 

Anne d'Autriche, qui, à la suite de la saisie de sa lettre, s'attendait à quelque reproche, fut fort étonnée de voir le lendemain le roi faire près d'elle des tentatives de rapprochement. Son premier mouvement fut répulsif, son orgueil de femme et sa dignité de reine avaient été tous deux si cruellement offensés, qu'elle ne pouvait revenir ainsi du premier coup; mais, vaincue par le conseil de ses femmes, elle eut enfin l'air de commencer à oublier. Le roi profita de ce premier moment de retour pour lui dire qu'incessamment il comptait donner une fête. 

C'était une chose si rare qu'une fête pour la pauvre Anne d'Autriche, qu'à cette annonce, ainsi que l'avait pensé le cardinal, la dernière trace de ses ressentiments disparut sinon dans son cœur, du moins sur son visage. Elle demanda quel jour cette fête devait avoir lieu, mais le roi répondit qu'il fallait qu'il s'entendît sur ce point avec le cardinal. 

En effet, chaque jour le roi demandait au cardinal à quelle époque cette fête aurait lieu, et chaque jour le cardinal, sous un prétexte quelconque, différait de la fixer. 

Dix jours s'écoulèrent ainsi. 

Le huitième jour après la scène que nous avons racontée, le cardinal reçut une lettre, au timbre de Londres, qui contenait seulement ces quelques lignes: 

«Je les ai; mais je ne puis quitter Londres, attendu que je manque d'argent; envoyez-moi cinq cents pistoles, et quatre ou cinq jours après les avoir reçues, je serai à Paris.» 

Le jour même où le cardinal avait reçu cette lettre, le roi lui adressa sa question habituelle. 

Richelieu compta sur ses doigts et se dit tout bas: 

«Elle arrivera, dit-elle, quatre ou cinq jours après avoir reçu l'argent; il faut quatre ou cinq jours à l'argent pour aller, quatre ou cinq jours à elle pour revenir, cela fait dix jours; maintenant faisons la part des vents contraires, des mauvais hasards, des faiblesses de femme, et mettons cela à douze jours. 

\speak  Eh bien, monsieur le duc, dit le roi, vous avez calculé? 

\speak  Oui, Sire: nous sommes aujourd'hui le 20 septembre; les échevins de la ville donnent une fête le 3 octobre. Cela s'arrangera à merveille, car vous n'aurez pas l'air de faire un retour vers la reine.» 

Puis le cardinal ajouta: 

«À propos, Sire, n'oubliez pas de dire à Sa Majesté, la veille de cette fête, que vous désirez voir comment lui vont ses ferrets de diamants.» 