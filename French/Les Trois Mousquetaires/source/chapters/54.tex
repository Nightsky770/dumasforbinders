%!TeX root=../musketeersfr.tex 

\chapter{Troisième Journée De Captivité}

\lettrine{F}{elton} était venu; mais il y avait encore un pas à faire: il fallait le retenir, ou plutôt il fallait qu'il restât tout seul; et Milady ne voyait encore qu'obscurément le moyen qui devait la conduire à ce résultat. 

Il fallait plus encore: il fallait le faire parler, afin de lui parler aussi: car, Milady le savait bien, sa plus grande séduction était dans sa voix, qui parcourait si habilement toute la gamme des tons, depuis la parole humaine jusqu'au langage céleste. 

Et cependant, malgré toute cette séduction, Milady pouvait échouer, car Felton était prévenu, et cela contre le moindre hasard. Dès lors, elle surveilla toutes ses actions, toutes ses paroles, jusqu'au plus simple regard de ses yeux, jusqu'à son geste, jusqu'à sa respiration, qu'on pouvait interpréter comme un soupir. Enfin, elle étudia tout comme fait un habile comédien à qui l'on vient de donner un rôle nouveau dans un emploi qu'il n'a pas l'habitude de tenir. 

Vis-à-vis de Lord de Winter sa conduite était plus facile; aussi avait-elle été arrêtée dès la veille. Rester muette et digne en sa présence, de temps en temps l'irriter par un dédain affecté, par un mot méprisant, le pousser à des menaces et à des violences qui faisaient un contraste avec sa résignation à elle, tel était son projet. Felton verrait: peut-être ne dirait-il rien; mais il verrait. 

Le matin, Felton vint comme d'habitude; mais Milady le laissa présider à tous les apprêts du déjeuner sans lui adresser la parole. Aussi, au moment où il allait se retirer, eut-elle une lueur d'espoir; car elle crut que c'était lui qui allait parler; mais ses lèvres remuèrent sans qu'aucun son sortît de sa bouche, et, faisant un effort sur lui-même, il renferma dans son cœur les paroles qui allaient s'échapper de ses lèvres, et sortit. 

Vers midi, Lord de Winter entra. 

Il faisait une assez belle journée d'hiver, et un rayon de ce pâle soleil d'Angleterre qui éclaire, mais qui n'échauffe pas, passait à travers les barreaux de la prison. 

Milady regardait par la fenêtre, et fit semblant de ne pas entendre la porte qui s'ouvrait. 

«Ah! ah! dit Lord de Winter, après avoir fait de la comédie, après avoir fait de la tragédie, voilà que nous faisons de la mélancolie.» 

La prisonnière ne répondit pas. 

«Oui, oui, continua Lord de Winter, je comprends; vous voudriez bien être en liberté sur ce rivage; vous voudriez bien, sur un bon navire, fendre les flots de cette mer verte comme de l'émeraude; vous voudriez bien, soit sur terre, soit sur l'océan, me dresser une de ces bonnes petites embuscades comme vous savez si bien les combiner. Patience! patience! Dans quatre jours, le rivage vous sera permis, la mer vous sera ouverte, plus ouverte que vous ne le voudrez, car dans quatre jours l'Angleterre sera débarrassée de vous.» 

Milady joignit les mains, et levant ses beaux yeux vers le ciel: 

«Seigneur! Seigneur! dit-elle avec une angélique suavité de geste et d'intonation, pardonnez à cet homme, comme je lui pardonne moi-même. 

\speak  Oui, prie, maudite, s'écria le baron, ta prière est d'autant plus généreuse que tu es, je te le jure, au pouvoir d'un homme qui ne pardonnera pas.» 

Et il sortit. 

Au moment où il sortait, un regard perçant glissa par la porte entrebâillée, et elle aperçut Felton qui se rangeait rapidement pour n'être pas vu d'elle. 

Alors elle se jeta à genoux et se mit à prier. 

«Mon Dieu! mon Dieu! dit-elle, vous savez pour quelle sainte cause je souffre, donnez-moi donc la force de souffrir.» 

La porte s'ouvrit doucement; la belle suppliante fit semblant de n'avoir pas entendu, et d'une voix pleine de larmes, elle continua: 

«Dieu vengeur! Dieu de bonté! laisserez-vous s'accomplir les affreux projets de cet homme!» 

Alors, seulement, elle feignit d'entendre le bruit des pas de Felton et, se relevant rapide comme la pensée, elle rougit comme si elle eût été honteuse d'avoir été surprise à genoux. 

«Je n'aime point à déranger ceux qui prient, madame, dit gravement Felton; ne vous dérangez donc pas pour moi, je vous en conjure. 

\speak  Comment savez-vous que je priais, monsieur? dit Milady d'une voix suffoquée par les sanglots; vous vous trompiez, monsieur, je ne priais pas. 

\speak  Pensez-vous donc, madame, répondit Felton de sa même voix grave, quoique avec un accent plus doux, que je me croie le droit d'empêcher une créature de se prosterner devant son Créateur? À Dieu ne plaise! D'ailleurs le repentir sied bien aux coupables; quelque crime qu'il ait commis, un coupable m'est sacré aux pieds de Dieu. 

\speak  Coupable, moi! dit Milady avec un sourire qui eût désarmé l'ange du jugement dernier. Coupable! mon Dieu, tu sais si je le suis! Dites que je suis condamnée, monsieur, à la bonne heure; mais vous le savez, Dieu qui aime les martyrs, permet que l'on condamne quelquefois les innocents. 

\speak  Fussiez-vous condamnée, fussiez-vous martyre, répondit Felton, raison de plus pour prier, et moi-même je vous aiderai de mes prières. 

\speak  Oh! vous êtes un juste, vous, s'écria Milady en se précipitant à ses pieds; tenez, je n'y puis tenir plus longtemps, car je crains de manquer de force au moment où il me faudra soutenir la lutte et confesser ma foi, écoutez donc la supplication d'une femme au désespoir. On vous abuse, monsieur, mais il n'est pas question de cela, je ne vous demande qu'une grâce, et, si vous me l'accordez, je vous bénirai dans ce monde et dans l'autre. 

\speak  Parlez au maître, madame, dit Felton; je ne suis heureusement chargé, moi, ni de pardonner ni de punir, et c'est à plus haut que moi que Dieu a remis cette responsabilité. 

\speak  À vous, non, à vous seul. Écoutez-moi, plutôt que de contribuer à ma perte, plutôt que de contribuer à mon ignominie. 

\speak  Si vous avez mérité cette honte, madame, si vous avez encouru cette ignominie, il faut la subir en l'offrant à Dieu. 

\speak  Que dites-vous? Oh! vous ne me comprenez pas! Quand je parle d'ignominie, vous croyez que je parle d'un châtiment quelconque, de la prison ou de la mort! Plût au Ciel! que m'importent, à moi, la mort ou la prison! 

\speak  C'est moi qui ne vous comprends plus, madame. 

\speak  Ou qui faites semblant de ne plus me comprendre, monsieur, répondit la prisonnière avec un sourire de doute. 

\speak  Non, madame, sur l'honneur d'un soldat, sur la foi d'un chrétien! 

\speak  Comment! vous ignorez les desseins de Lord de Winter sur moi. 

\speak  Je les ignore. 

\speak  Impossible, vous son confident! 

\speak  Je ne mens jamais, madame. 

\speak  Oh! il se cache trop peu cependant pour qu'on ne les devine pas. 

\speak  Je ne cherche à rien deviner, madame; j'attends qu'on me confie, et à part ce qu'il m'a dit devant vous, Lord de Winter ne m'a rien confié. 

\speak  Mais, s'écria Milady avec un incroyable accent de vérité, vous n'êtes donc pas son complice, vous ne savez donc pas qu'il me destine à une honte que tous les châtiments de la terre ne sauraient égaler en horreur? 

\speak  Vous vous trompez, madame, dit Felton en rougissant, Lord de Winter n'est pas capable d'un tel crime.» 

«Bon, dit Milady en elle-même, sans savoir ce que c'est, il appelle cela un crime!» 

Puis tout haut: 

«L'ami de l'infâme est capable de tout. 

\speak  Qui appelez-vous l'infâme? demanda Felton. 

\speak  Y a-t-il donc en Angleterre deux hommes à qui un semblable nom puisse convenir? 

\speak  Vous voulez parler de Georges Villiers? dit Felton, dont les regards s'enflammèrent. 

\speak  Que les païens, les gentils et les infidèles appellent duc de Buckingham, reprit Milady; je n'aurais pas cru qu'il y aurait eu un Anglais dans toute l'Angleterre qui eût eu besoin d'une si longue explication pour reconnaître celui dont je voulais parler! 

\speak  La main du Seigneur est étendue sur lui, dit Felton, il n'échappera pas au châtiment qu'il mérite.» 

Felton ne faisait qu'exprimer à l'égard du duc le sentiment d'exécration que tous les Anglais avaient voué à celui que les catholiques eux-mêmes appelaient l'exacteur, le concussionnaire, le débauché, et que les puritains appelaient tout simplement Satan. 

«Oh! mon Dieu! mon Dieu! s'écria Milady, quand je vous supplie d'envoyer à cet homme le châtiment qui lui est dû, vous savez que ce n'est pas ma propre vengeance que je poursuis, mais la délivrance de tout un peuple que j'implore. 

\speak  Le connaissez-vous donc?» demanda Felton. 

«Enfin, il m'interroge», se dit en elle-même Milady au comble de la joie d'en être arrivée si vite à un si grand résultat. 

«Oh! si je le connais! oh, oui! pour mon malheur, pour mon malheur éternel.» 

Et Milady se tordit les bras comme arrivée au paroxysme de la douleur. Felton sentit sans doute en lui-même que sa force l'abandonnait, et il fit quelques pas vers la porte; la prisonnière, qui ne le perdait pas de vue, bondit à sa poursuite et l'arrêta. 

«Monsieur! s'écria-t-elle, soyez bon, soyez clément, écoutez ma prière: ce couteau que la fatale prudence du baron m'a enlevé, parce qu'il sait l'usage que j'en veux faire; oh! écoutez-moi jusqu'au bout! ce couteau, rendez-le moi une minute seulement, par grâce, par pitié! J'embrasse vos genoux; voyez, vous fermerez la porte, ce n'est pas à vous que j'en veux: Dieu! vous en vouloir, à vous, le seul être juste, bon et compatissant que j'aie rencontré! à vous, mon sauveur peut-être! une minute, ce couteau, une minute, une seule, et je vous le rends par le guichet de la porte; rien qu'une minute, monsieur Felton, et vous m'aurez sauvé l'honneur! 

\speak  Vous tuer! s'écria Felton avec terreur, oubliant de retirer ses mains des mains de la prisonnière; vous tuer! 

\speak  J'ai dit, monsieur, murmura Milady en baissant la voix et en se laissant tomber affaissée sur le parquet, j'ai dit mon secret! il sait tout! mon Dieu, je suis perdue!» 

Felton demeurait debout, immobile et indécis. 

«Il doute encore, pensa Milady, je n'ai pas été assez vraie.» 

On entendit marcher dans le corridor; Milady reconnut le pas de Lord de Winter. Felton le reconnut aussi et s'avança vers la porte. 

Milady s'élança. 

«Oh! pas un mot, dit-elle d'une voix concentrée, pas un mot de tout ce que je vous ai dit à cet homme, ou je suis perdue, et c'est vous, vous\dots» 

Puis, comme les pas se rapprochaient, elle se tut de peur qu'on n'entendit sa voix, appuyant avec un geste de terreur infinie sa belle main sur la bouche de Felton. Felton repoussa doucement Milady, qui alla tomber sur une chaise longue. 

Lord de Winter passa devant la porte sans s'arrêter, et l'on entendit le bruit des pas qui s'éloignaient. 

Felton, pâle comme la mort, resta quelques instants l'oreille tendue et écoutant, puis quand le bruit se fut éteint tout à fait, il respira comme un homme qui sort d'un songe, et s'élança hors de l'appartement. 

«Ah! dit Milady en écoutant à son tour le bruit des pas de Felton, qui s'éloignaient dans la direction opposée à ceux de Lord de Winter, enfin tu es donc à moi!» 

Puis son front se rembrunit. 

«S'il parle au baron, dit-elle, je suis perdue, car le baron, qui sait bien que je ne me tuerai pas, me mettra devant lui un couteau entre les mains, et il verra bien que tout ce grand désespoir n'était qu'un jeu.» 

Elle alla se placer devant sa glace et se regarda; jamais elle n'avait été si belle. 

«Oh! oui! dit-elle en souriant, mais il ne lui parlera pas.» 

Le soir, Lord de Winter accompagna le souper. 

\speak  Monsieur, lui dit Milady, votre présence est-elle un accessoire obligé de ma captivité, et ne pourriez-vous pas m'épargner ce surcroît de tortures que me causent vos visites? 

\speak  Comment donc, chère soeur! dit de Winter, ne m'avez-vous pas sentimentalement annoncé, de cette jolie bouche si cruelle pour moi aujourd'hui, que vous veniez en Angleterre à cette seule fin de me voir tout à votre aise, jouissance dont, me disiez-vous, vous ressentiez si vivement la privation, que vous avez tout risqué pour cela, mal de mer, tempête, captivité! eh bien, me voilà, soyez satisfaite; d'ailleurs, cette fois ma visite a un motif.» 

Milady frissonna, elle crut que Felton avait parlé; jamais de sa vie, peut-être, cette femme, qui avait éprouvé tant d'émotions puissantes et opposées, n'avait senti battre son cœur si violemment. 

Elle était assise; Lord de Winter prit un fauteuil, le tira à son côté et s'assit auprès d'elle, puis prenant dans sa poche un papier qu'il déploya lentement: 

«Tenez, lui dit-il, je voulais vous montrer cette espèce de passeport que j'ai rédigé moi-même et qui vous servira désormais de numéro d'ordre dans la vie que je consens à vous laisser.» 

Puis ramenant ses yeux de Milady sur le papier, il lut: 

«Ordre de conduire à\dots» Le nom est en blanc, interrompit de Winter: si vous avez quelque préférence, vous me l'indiquerez; et pour peu que ce soit à un millier de lieues de Londres, il sera fait droit à votre requête. Je reprends donc: «Ordre de conduire à\dots la nommée Charlotte Backson, flétrie par la justice du royaume de France, mais libérée après châtiment; elle demeurera dans cette résidence, sans jamais s'en écarter de plus de trois lieues. En cas de tentative d'évasion, la peine de mort lui sera appliquée. Elle touchera cinq shillings par jour pour son logement et sa nourriture.» 

«Cet ordre ne me concerne pas, répondit froidement Milady, puisqu'un autre nom que le mien y est porté. 

\speak  Un nom! Est-ce que vous en avez un? 

\speak  J'ai celui de votre frère. 

\speak  Vous vous trompez, mon frère n'est que votre second mari, et le premier vit encore. Dites-moi son nom et je le mettrai en place du nom de Charlotte Backson. Non?\dots vous ne voulez pas?\dots vous gardez le silence? C'est bien! vous serez écrouée sous le nom de Charlotte Backson.» 

Milady demeura silencieuse; seulement, cette fois ce n'était plus par affectation, mais par terreur: elle crut l'ordre prêt à être exécuté: elle pensa que Lord de Winter avait avancé son départ; elle crut qu'elle était condamnée à partir le soir même. Tout dans son esprit fut donc perdu pendant un instant, quand tout à coup elle s'aperçut que l'ordre n'était revêtu d'aucune signature. 

La joie qu'elle ressentit de cette découverte fut si grande, qu'elle ne put la cacher. 

«Oui, oui, dit Lord de Winter, qui s'aperçut de ce qui se passait en elle, oui, vous cherchez la signature, et vous vous dites: tout n'est pas perdu, puisque cet acte n'est pas signé; on me le montre pour m'effrayer, voilà tout. Vous vous trompez: demain cet ordre sera envoyé à Lord Buckingham; après-demain il reviendra signé de sa main et revêtu de son sceau, et vingt-quatre heures après, c'est moi qui vous en réponds, il recevra son commencement d'exécution. Adieu, madame, voilà tout ce que j'avais à vous dire. 

\speak  Et moi je vous répondrai, monsieur, que cet abus de pouvoir, que cet exil sous un nom supposé sont une infamie. 

\speak  Aimez-vous mieux être pendue sous votre vrai nom, Milady? Vous le savez, les lois anglaises sont inexorables sur l'abus que l'on fait du mariage; expliquez-vous franchement: quoique mon nom ou plutôt le nom de mon frère se trouve mêlé dans tout cela, je risquerai le scandale d'un procès public pour être sûr que du coup je serai débarrassé de vous.» 

Milady ne répondit pas, mais devint pâle comme un cadavre. 

«Oh! je vois que vous aimez mieux la pérégrination. À merveille, madame, et il y a un vieux proverbe qui dit que les voyages forment la jeunesse. Ma foi! vous n'avez pas tort, après tout, et la vie est bonne. C'est pour cela que je ne me soucie pas que vous me l'ôtiez. Reste donc à régler l'affaire des cinq shillings; je me montre un peu parcimonieux, n'est-ce pas? cela tient à ce que je ne me soucie pas que vous corrompiez vos gardiens. D'ailleurs il vous restera toujours vos charmes pour les séduire. Usez-en si votre échec avec Felton ne vous a pas dégoûtée des tentatives de ce genre.» 

«Felton n'a point parlé, se dit Milady à elle-même, rien n'est perdu alors.» 

«Et maintenant, madame, à vous revoir. Demain je viendrai vous annoncer le départ de mon messager.» 

Lord de Winter se leva, salua ironiquement Milady et sortit. 

Milady respira: elle avait encore quatre jours devant elle; quatre jours lui suffiraient pour achever de séduire Felton. 

Une idée terrible lui vint alors, c'est que Lord de Winter enverrait peut-être Felton lui-même pour faire signer l'ordre à Buckingham; de cette façon Felton lui échappait, et pour que la prisonnière réussît il fallait la magie d'une séduction continue. 

Cependant, comme nous l'avons dit, une chose la rassurait: Felton n'avait pas parlé. 

Elle ne voulut point paraître émue par les menaces de Lord de Winter, elle se mit à table et mangea. 

Puis, comme elle avait fait la veille, elle se mit à genoux, et répéta tout haut ses prières. Comme la veille, le soldat cessa de marcher et s'arrêta pour l'écouter. 

Bientôt elle entendit des pas plus légers que ceux de la sentinelle qui venaient du fond du corridor et qui s'arrêtaient devant sa porte. 

«C'est lui», dit-elle. 

Et elle commença le même chant religieux qui la veille avait si violemment exalté Felton. 

Mais, quoique sa voix douce, pleine et sonore eût vibré plus harmonieuse et plus déchirante que jamais, la porte resta close. Il parut bien à Milady, dans un des regards furtifs qu'elle lançait sur le petit guichet, apercevoir à travers le grillage serré les yeux ardents du jeune homme mais, que ce fût une réalité ou une vision, cette fois il eut sur lui-même la puissance de ne pas entrer. 

Seulement, quelques instants après qu'elle eût fini son chant religieux, Milady crut entendre un profond soupir; puis les mêmes pas qu'elle avait entendus s'approcher s'éloignèrent lentement et comme à regret. 