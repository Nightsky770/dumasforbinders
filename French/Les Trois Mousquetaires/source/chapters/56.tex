%!TeX root=../musketeersfr.tex 

\chapter{Cinquième Journée De Captivité}

\lettrine{C}{ependant} Milady en était arrivée à un demi-triomphe, et le succès obtenu doublait ses forces. 

\zz
Il n'était pas difficile de vaincre, ainsi qu'elle l'avait fait jusque-là, des hommes prompts à se laisser séduire, et que l'éducation galante de la cour entraînait vite dans le piège; Milady était assez belle pour ne pas trouver de résistance de la part de la chair, et elle était assez adroite pour l'emporter sur tous les obstacles de l'esprit. 

Mais, cette fois, elle avait à lutter contre une nature sauvage, concentrée, insensible à force d'austérité; la religion et la pénitence avaient fait de Felton un homme inaccessible aux séductions ordinaires. Il roulait dans cette tête exaltée des plans tellement vastes, des projets tellement tumultueux, qu'il n'y restait plus de place pour aucun amour, de caprice ou de matière, ce sentiment qui se nourrit de loisir et grandit par la corruption. Milady avait donc fait brèche, avec sa fausse vertu, dans l'opinion d'un homme prévenu horriblement contre elle, et par sa beauté, dans le cœur et les sens d'un homme chaste et pur. Enfin, elle s'était donné la mesure de ses moyens, inconnus d'elle-même jusqu'alors, par cette expérience faite sur le sujet le plus rebelle que la nature et la religion pussent soumettre à son étude. 

Bien des fois néanmoins pendant la soirée elle avait désespéré du sort et d'elle-même; elle n'invoquait pas Dieu, nous le savons, mais elle avait foi dans le génie du mal, cette immense souveraineté qui règne dans tous les détails de la vie humaine, et à laquelle, comme dans la fable arabe, un grain de grenade suffit pour reconstruire un monde perdu. 

Milady, bien préparée à recevoir Felton, put dresser ses batteries pour le lendemain. Elle savait qu'il ne lui restait plus que deux jours, qu'une fois l'ordre signé par Buckingham (et Buckingham le signerait d'autant plus facilement, que cet ordre portait un faux nom, et qu'il ne pourrait reconnaître la femme dont il était question), une fois cet ordre signé, disons-nous, le baron la faisait embarquer sur-le-champ, et elle savait aussi que les femmes condamnées à la déportation usent d'armes bien moins puissantes dans leurs séductions que les prétendues femmes vertueuses dont le soleil du monde éclaire la beauté, dont la voix de la mode vante l'esprit et qu'un reflet d'aristocratie dore de ses lueurs enchantées. Être une femme condamnée à une peine misérable et infamante n'est pas un empêchement à être belle, mais c'est un obstacle à jamais redevenir puissante. Comme tous les gens d'un mérite réel, Milady connaissait le milieu qui convenait à sa nature, à ses moyens. La pauvreté lui répugnait, l'abjection la diminuait des deux tiers de sa grandeur. Milady n'était reine que parmi les reines; il fallait à sa domination le plaisir de l'orgueil satisfait. Commander aux êtres inférieurs était plutôt une humiliation qu'un plaisir pour elle. 

Certes, elle fût revenue de son exil, elle n'en doutait pas un seul instant; mais combien de temps cet exil pouvait-il durer? Pour une nature agissante et ambitieuse comme celle de Milady, les jours qu'on n'occupe point à monter sont des jours néfastes; qu'on trouve donc le mot dont on doive nommer les jours qu'on emploie à descendre! Perdre un an, deux ans, trois ans, c'est-à-dire une éternité; revenir quand d'Artagnan, heureux et triomphant, aurait, lui et ses amis, reçu de la reine la récompense qui leur était bien acquise pour les services qu'ils lui avaient rendus, c'étaient là de ces idées dévorantes qu'une femme comme Milady ne pouvait supporter. Au reste, l'orage qui grondait en elle doublait sa force, et elle eût fait éclater les murs de sa prison, si son corps eût pu prendre un seul instant les proportions de son esprit. 

Puis ce qui l'aiguillonnait encore au milieu de tout cela, c'était le souvenir du cardinal. Que devait penser, que devait dire de son silence le cardinal défiant, inquiet, soupçonneux, le cardinal, non seulement son seul appui, son seul soutien, son seul protecteur dans le présent, mais encore le principal instrument de sa fortune et de sa vengeance à venir? Elle le connaissait, elle savait qu'à son retour, après un voyage inutile, elle aurait beau arguer de la prison, elle aurait beau exalter les souffrances subies, le cardinal répondrait avec ce calme railleur du sceptique puissant à la fois par la force et par le génie: «Il ne fallait pas vous laisser prendre!» 

Alors Milady réunissait toute son énergie, murmurant au fond de sa pensée le nom de Felton, la seule lueur de jour qui pénétrât jusqu'à elle au fond de l'enfer où elle était tombée; et comme un serpent qui roule et déroule ses anneaux pour se rendre compte à lui-même de sa force, elle enveloppait d'avance Felton dans les mille replis de son inventive imagination. 

Cependant le temps s'écoulait, les heures les unes après les autres semblaient réveiller la cloche en passant, et chaque coup du battant d'airain retentissait sur le cœur de la prisonnière. À neuf heures, Lord de Winter fit sa visite accoutumée, regarda la fenêtre et les barreaux, sonda le parquet et les murs, visita la cheminée et les portes, sans que, pendant cette longue et minutieuse visite, ni lui ni Milady prononçassent une seule parole. 

Sans doute que tous deux comprenaient que la situation était devenue trop grave pour perdre le temps en mots inutiles et en colère sans effet. 

«Allons, allons, dit le baron en la quittant, vous ne vous sauverez pas encore cette nuit!» 

À dix heures, Felton vint placer une sentinelle; Milady reconnut son pas. Elle le devinait maintenant comme une maîtresse devine celui de l'amant de son cœur, et cependant Milady détestait et méprisait à la fois ce faible fanatique. 

Ce n'était point l'heure convenue, Felton n'entra point. 

Deux heures après et comme minuit sonnait, la sentinelle fut relevée. 

Cette fois c'était l'heure: aussi, à partir de ce moment, Milady attendit-elle avec impatience. 

La nouvelle sentinelle commença à se promener dans le corridor. 

Au bout de dix minutes Felton vint. 

Milady prêta l'oreille. 

«Écoutez, dit le jeune homme à la sentinelle, sous aucun prétexte ne t'éloigne de cette porte, car tu sais que la nuit dernière un soldat a été puni par Milord pour avoir quitté son poste un instant, et cependant c'est moi qui, pendant sa courte absence, avais veillé à sa place. 

\speak  Oui, je le sais, dit le soldat. 

\speak  Je te recommande donc la plus exacte surveillance. Moi, ajouta-t-il, je vais rentrer pour visiter une seconde fois la chambre de cette femme, qui a, j'en ai peur, de sinistres projets sur elle-même et que j'ai reçu l'ordre de surveiller.» 

«Bon, murmura Milady, voilà l'austère puritain qui ment!» 

Quant au soldat, il se contenta de sourire. 

«Peste! mon lieutenant, dit-il, vous n'êtes pas malheureux d'être chargé de commissions pareilles, surtout si Milord vous a autorisé à regarder jusque dans son lit.» 

Felton rougit; dans toute autre circonstance il eut réprimandé le soldat qui se permettait une pareille plaisanterie; mais sa conscience murmurait trop haut pour que sa bouche osât parler. 

«Si j'appelle, dit-il, viens; de même que si l'on vient, appelle-moi. 

\speak  Oui, mon lieutenant», dit le soldat. 

Felton entra chez Milady. Milady se leva. 

«Vous voilà? dit-elle. 

\speak  Je vous avais promis de venir, dit Felton, et je suis venu. 

\speak  Vous m'avez promis autre chose encore. 

\speak  Quoi donc? mon Dieu! dit le jeune homme, qui malgré son empire sur lui-même, sentait ses genoux trembler et la sueur poindre sur son front. 

\speak  Vous avez promis de m'apporter un couteau, et de me le laisser après notre entretien. 

\speak  Ne parlez pas de cela, madame, dit Felton, il n'y a pas de situation, si terrible qu'elle soit, qui autorise une créature de Dieu à se donner la mort. J'ai réfléchi que jamais je ne devais me rendre coupable d'un pareil péché. 

\speak  Ah! vous avez réfléchi! dit la prisonnière en s'asseyant sur son fauteuil avec un sourire de dédain; et moi aussi j'ai réfléchi. 

\speak  À quoi? 

\speak  Que je n'avais rien à dire à un homme qui ne tenait pas sa parole. 

\speak  O mon Dieu! murmura Felton. 

\speak  Vous pouvez vous retirer, dit Milady, je ne parlerai pas. 

\speak  Voilà le couteau! dit Felton tirant de sa poche l'arme que, selon sa promesse, il avait apportée, mais qu'il hésitait à remettre à sa prisonnière. 

\speak  Voyons-le, dit Milady. 

\speak  Pour quoi faire? 

\speak  Sur l'honneur, je vous le rends à l'instant même; vous le poserez sur cette table; et vous resterez entre lui et moi. 

Felton tendit l'arme à Milady, qui en examina attentivement la trempe, et qui en essaya la pointe sur le bout de son doigt. 

«Bien, dit-elle en rendant le couteau au jeune officier, celui-ci est en bel et bon acier; vous êtes un fidèle ami, Felton.» 

Felton reprit l'arme et la posa sur la table comme il venait d'être convenu avec sa prisonnière. 

Milady le suivit des yeux et fit un geste de satisfaction. 

«Maintenant, dit-elle, écoutez-moi.» 

La recommandation était inutile: le jeune officier se tenait debout devant elle, attendant ses paroles pour les dévorer. 

«Felton, dit Milady avec une solennité pleine de mélancolie, Felton, si votre soeur, la fille de votre père, vous disait: «Jeune encore, assez belle par malheur, on m'a fait tomber dans un piège, j'ai résisté; on a multiplié autour de moi les embûches, les violences, j'ai résisté; on a blasphémé la religion que je sers, le Dieu que j'adore, parce que j'appelais à mon secours ce Dieu et cette religion, j'ai résisté; alors on m'a prodigué les outrages, et comme on ne pouvait perdre mon âme, on a voulu à tout jamais flétrir mon corps; enfin\dots» 

Milady s'arrêta, et un sourire amer passa sur ses lèvres. 

«Enfin, dit Felton, enfin qu'a-t-on fait? 

\speak  Enfin, un soir, on résolut de paralyser cette résistance qu'on ne pouvait vaincre: un soir, on mêla à mon eau un narcotique puissant; à peine eus-je achevé mon repas, que je me sentis tomber peu à peu dans une torpeur inconnue. Quoique je fusse sans défiance, une crainte vague me saisit et j'essayai de lutter contre le sommeil; je me levai, je voulus courir à la fenêtre, appeler au secours, mais mes jambes refusèrent de me porter; il me semblait que le plafond s'abaissait sur ma tête et m'écrasait de son poids; je tendis les bras, j'essayai de parler, je ne pus que pousser des sons inarticulés; un engourdissement irrésistible s'emparait de moi, je me retins à un fauteuil, sentant que j'allais tomber, mais bientôt cet appui fut insuffisant pour mes bras débiles, je tombai sur un genou, puis sur les deux; je voulus crier, ma langue était glacée; Dieu ne me vit ni ne m'entendit sans doute, et je glissai sur le parquet, en proie à un sommeil qui ressemblait à la mort. 

«De tout ce qui se passa dans ce sommeil et du temps qui s'écoula pendant sa durée, je n'eus aucun souvenir; la seule chose que je me rappelle, c'est que je me réveillai couchée dans une chambre ronde, dont l'ameublement était somptueux, et dans laquelle le jour ne pénétrait que par une ouverture au plafond. Du reste, aucune porte ne semblait y donner entrée: on eût dit une magnifique prison. 

«Je fus longtemps à pouvoir me rendre compte du lieu où je me trouvais et de tous les détails que je rapporte, mon esprit semblait lutter inutilement pour secouer les pesantes ténèbres de ce sommeil auquel je ne pouvais m'arracher; j'avais des perceptions vagues d'un espace parcouru, du roulement d'une voiture, d'un rêve horrible dans lequel mes forces se seraient épuisées; mais tout cela était si sombre et si indistinct dans ma pensée, que ces événements semblaient appartenir à une autre vie que la mienne et cependant mêlée à la mienne par une fantastique dualité. 

«Quelque temps, l'état dans lequel je me trouvais me sembla si étrange, que je crus que je faisais un rêve. Je me levai chancelante, mes habits étaient près de moi, sur une chaise: je ne me rappelai ni m'être dévêtue, ni m'être couchée. Alors peu à peu la réalité se présenta à moi pleine de pudiques terreurs: je n'étais plus dans la maison que j'habitais; autant que j'en pouvais juger par la lumière du soleil, le jour était déjà aux deux tiers écoulé! c'était la veille au soir que je m'étais endormie; mon sommeil avait donc déjà duré près de vingt-quatre heures. Que s'était-il passé pendant ce long sommeil? 

«Je m'habillai aussi rapidement qu'il me fut possible. Tous mes mouvements lents et engourdis attestaient que l'influence du narcotique n'était point encore entièrement dissipée. Au reste, cette chambre était meublée pour recevoir une femme; et la coquette la plus achevée n'eût pas eu un souhait à former, qu'en promenant son regard autour de l'appartement elle n'eût vu son souhait accompli. 

«Certes, je n'étais pas la première captive qui s'était vue enfermée dans cette splendide prison; mais, vous le comprenez, Felton, plus la prison était belle, plus je m'épouvantais. 

«Oui, c'était une prison, car j'essayai vainement d'en sortir. Je sondai tous les murs afin de découvrir une porte, partout les murs rendirent un son plein et mat. 

«Je fis peut-être vingt fois le tour de cette chambre, cherchant une issue quelconque; il n'y en avait pas: je tombai écrasée de fatigue et de terreur sur un fauteuil. 

«Pendant ce temps, la nuit venait rapidement, et avec la nuit mes terreurs augmentaient: je ne savais si je devais rester où j'étais assise; il me semblait que j'étais entourée de dangers inconnus, dans lesquels j'allais tomber à chaque pas. Quoique je n'eusse rien mangé depuis la veille, mes craintes m'empêchaient de ressentir la faim. 

«Aucun bruit du dehors, qui me permît de mesurer le temps, ne venait jusqu'à moi; je présumai seulement qu'il pouvait être sept ou huit heures du soir; car nous étions au mois d'octobre, et il faisait nuit entière. 

«Tout à coup, le cri d'une porte qui tourne sur ses gonds me fit tressaillir; un globe de feu apparut au-dessus de l'ouverture vitrée du plafond, jetant une vive lumière dans ma chambre, et je m'aperçus avec terreur qu'un homme était debout à quelques pas de moi. 

«Une table à deux couverts, supportant un souper tout préparé, s'était dressée comme par magie au milieu de l'appartement. 

«Cet homme était celui qui me poursuivait depuis un an, qui avait juré mon déshonneur, et qui, aux premiers mots qui sortirent de sa bouche, me fit comprendre qu'il l'avait accompli la nuit précédente. 

\speak  L'infâme! murmura Felton. 

\speak  Oh! oui, l'infâme! s'écria Milady, voyant l'intérêt que le jeune officier, dont l'âme semblait suspendue à ses lèvres, prenait à cet étrange récit; oh! oui, l'infâme! il avait cru qu'il lui suffisait d'avoir triomphé de moi dans mon sommeil, pour que tout fût dit; il venait, espérant que j'accepterais ma honte, puisque ma honte était consommée; il venait m'offrir sa fortune en échange de mon amour. 

«Tout ce que le cœur d'une femme peut contenir de superbe mépris et de paroles dédaigneuses, je le versai sur cet homme; sans doute, il était habitué à de pareils reproches; car il m'écouta calme, souriant, et les bras croisés sur la poitrine; puis, lorsqu'il crut que j'avais tout dit, il s'avança vers moi; je bondis vers la table, je saisis un couteau, je l'appuyai sur ma poitrine. 

«Faites un pas de plus, lui dis-je, et outre mon déshonneur, vous aurez encore ma mort à vous reprocher.» 

«Sans doute, il y avait dans mon regard, dans ma voix, dans toute ma personne, cette vérité de geste, de pose et d'accent, qui porte la conviction dans les âmes les plus perverses, car il s'arrêta. 

«Votre mort! me dit-il; oh! non, vous êtes une trop charmante maîtresse pour que je consente à vous perdre ainsi, après avoir eu le bonheur de vous posséder une seule fois seulement. Adieu, ma toute belle! j'attendrai, pour revenir vous faire ma visite, que vous soyez dans de meilleures dispositions.» 

«À ces mots, il donna un coup de sifflet; le globe de flamme qui éclairait ma chambre remonta et disparut; je me retrouvai dans l'obscurité. Le même bruit d'une porte qui s'ouvre et se referme se reproduisit un instant après, le globe flamboyant descendit de nouveau, et je me retrouvai seule. 

«Ce moment fut affreux; si j'avais encore quelques doutes sur mon malheur, ces doutes s'étaient évanouis dans une désespérante réalité: j'étais au pouvoir d'un homme que non seulement je détestais, mais que je méprisais; d'un homme capable de tout, et qui m'avait déjà donné une preuve fatale de ce qu'il pouvait oser. 

\speak  Mais quel était donc cet homme? demanda Felton. 

\speak  Je passai la nuit sur une chaise, tressaillant au moindre bruit, car à minuit à peu près, la lampe s'était éteinte, et je m'étais retrouvée dans l'obscurité. Mais la nuit se passa sans nouvelle tentative de mon persécuteur; le jour vint: la table avait disparu; seulement, j'avais encore le couteau à la main. 

«Ce couteau c'était tout mon espoir. 

«J'étais écrasée de fatigue; l'insomnie brûlait mes yeux; je n'avais pas osé dormir un seul instant: le jour me rassura, j'allai me jeter sur mon lit sans quitter le couteau libérateur que je cachai sous mon oreiller. 

«Quand je me réveillai, une nouvelle table était servie. 

«Cette fois, malgré mes terreurs, en dépit de mes angoisses, une faim dévorante se faisait sentir; il y avait quarante-huit heures que je n'avais pris aucune nourriture: je mangeai du pain et quelques fruits; puis, me rappelant le narcotique mêlé à l'eau que j'avais bue, je ne touchai point à celle qui était sur la table, et j'allai remplir mon verre à une fontaine de marbre scellée dans le mur, au-dessus de ma toilette. 

«Cependant, malgré cette précaution, je ne demeurai pas moins quelque temps encore dans une affreuse angoisse; mais mes craintes, cette fois, n'étaient pas fondées: je passai la journée sans rien éprouver qui ressemblât à ce que je redoutais. 

«J'avais eu la précaution de vider à demi la carafe, pour qu'on ne s'aperçût point de ma défiance. 

«Le soir vint, et avec lui l'obscurité; cependant, si profonde qu'elle fût, mes yeux commençaient à s'y habituer; je vis, au milieu des ténèbres, la table s'enfoncer dans le plancher; un quart d'heure après, elle reparut portant mon souper; un instant après, grâce à la même lampe, ma chambre s'éclaira de nouveau. 

«J'étais résolue à ne manger que des objets auxquels il était impossible de mêler aucun somnifère: deux oeufs et quelques fruits composèrent mon repas; puis, j'allai puiser un verre d'eau à ma fontaine protectrice, et je le bus. 

«Aux premières gorgées, il me sembla qu'elle n'avait plus le même goût que le matin: un soupçon rapide me prit, je m'arrêtai; mais j'en avais déjà avalé un demi-verre. 

«Je jetai le reste avec horreur, et j'attendis, la sueur de l'épouvante au front. 

«Sans doute quelque invisible témoin m'avait vue prendre de l'eau à cette fontaine, et avait profité de ma confiance même pour mieux assurer ma perte si froidement résolue, si cruellement poursuivie. 

«Une demi-heure ne s'était pas écoulée, que les mêmes symptômes se produisirent; seulement, comme cette fois je n'avais bu qu'un demi-verre d'eau, je luttai plus longtemps, et, au lieu de m'endormir tout à fait, je tombai dans un état de somnolence qui me laissait le sentiment de ce qui se passait autour de moi, tout en m'ôtant la force ou de me défendre ou de fuir. 

«Je me traînai vers mon lit, pour y chercher la seule défense qui me restât, mon couteau sauveur; mais je ne pus arriver jusqu'au chevet: je tombai à genoux, les mains cramponnées à l'une des colonnes du pied; alors, je compris que j'étais perdue.» 

Felton pâlit affreusement, et un frisson convulsif courut par tout son corps. 

«Et ce qu'il y avait de plus affreux, continua Milady, la voix altérée comme si elle eût encore éprouvé la même angoisse qu'en ce moment terrible, c'est que, cette fois, j'avais la conscience du danger qui me menaçait; c'est que mon âme, je puis le dire, veillait dans mon corps endormi; c'est que je voyais, c'est que j'entendais: il est vrai que tout cela était comme dans un rêve; mais ce n'en était que plus effrayant. 

«Je vis la lampe qui remontait et qui peu à peu me laissait dans l'obscurité; puis j'entendis le cri si bien connu de cette porte, quoique cette porte ne se fût ouverte que deux fois. 

«Je sentis instinctivement qu'on s'approchait de moi: on dit que le malheureux perdu dans les déserts de l'Amérique sent ainsi l'approche du serpent. 

«Je voulais faire un effort, je tentai de crier; par une incroyable énergie de volonté je me relevai même, mais pour retomber aussitôt\dots et retomber dans les bras de mon persécuteur. 

\speak  Dites-moi donc quel était cet homme?» s'écria le jeune officier. 

Milady vit d'un seul regard tout ce qu'elle inspirait de souffrance à Felton, en pesant sur chaque détail de son récit; mais elle ne voulait lui faire grâce d'aucune torture. Plus profondément elle lui briserait le cœur, plus sûrement il la vengerait. Elle continua donc comme si elle n'eût point entendu son exclamation, ou comme si elle eût pensé que le moment n'était pas encore venu d'y répondre. 

«Seulement, cette fois, ce n'était plus à une espèce de cadavre inerte, sans aucun sentiment, que l'infâme avait affaire. Je vous l'ai dit: sans pouvoir parvenir à retrouver l'exercice complet de mes facultés, il me restait le sentiment de mon danger: je luttai donc de toutes mes forces et sans doute j'opposai, tout affaiblie que j'étais, une longue résistance, car je l'entendis s'écrier: 

«Ces misérables puritaines! je savais bien qu'elles lassaient leurs bourreaux, mais je les croyais moins fortes contre leurs séducteurs.« 

«Hélas! cette résistance désespérée ne pouvait durer longtemps, je sentis mes forces qui s'épuisaient, et cette fois ce ne fut pas de mon sommeil que le lâche profita, ce fut de mon évanouissement.» 

Felton écoutait sans faire entendre autre chose qu'une espèce de rugissement sourd; seulement la sueur ruisselait sur son front de marbre, et sa main cachée sous son habit déchirait sa poitrine. 

«Mon premier mouvement, en revenant à moi, fui de chercher sous mon oreiller ce couteau que je n'avais pu atteindre; s'il n'avait point servi à la défense, il pouvait au moins servir à l'expiation. 

«Mais en prenant ce couteau, Felton, une idée terrible me vint. J'ai juré de tout vous dire et je vous dirai tout; je vous ai promis la vérité, je la dirai, dût-elle me perdre. 

\speak  L'idée vous vint de vous venger de cet homme, n'est-ce pas? s'écria Felton. 

\speak  Eh bien, oui! dit Milady: cette idée n'était pas d'une chrétienne, je le sais; sans doute cet éternel ennemi de notre âme, ce lion rugissant sans cesse autour de nous la soufflait à mon esprit. Enfin, que vous dirai-je, Felton? continua Milady du ton d'une femme qui s'accuse d'un crime, cette idée me vint et ne me quitta plus sans doute. C'est de cette pensée homicide que je porte aujourd'hui la punition. 

\speak  Continuez, continuez, dit Felton, j'ai hâte de vous voir arriver à la vengeance. 

\speak  Oh! je résolus qu'elle aurait lieu le plus tôt possible, je ne doutais pas qu'il ne revînt la nuit suivante. Dans le jour je n'avais rien à craindre. 

«Aussi, quand vint l'heure du déjeuner, je n'hésitai pas à manger et à boire: j'étais résolue à faire semblant de souper, mais à ne rien prendre: je devais donc par la nourriture du matin combattre le jeûne du soir. 

«Seulement je cachai un verre d'eau soustraite à mon déjeuner, la soif ayant été ce qui m'avait le plus fait souffrir quand j'étais demeurée quarante-huit heures sans boire ni manger. 

«La journée s'écoula sans avoir d'autre influence sur moi que de m'affermir dans la résolution prise: seulement j'eus soin que mon visage ne trahît en rien la pensée de mon cœur, car je ne doutais pas que je ne fusse observée; plusieurs fois même je sentis un sourire sur mes lèvres. Felton, je n'ose pas vous dire à quelle idée je souriais, vous me prendriez en horreur\dots 

\speak  Continuez, continuez, dit Felton, vous voyez bien que j'écoute et que j'ai hâte d'arriver. 

\speak  Le soir vint, les événements ordinaires s'accomplirent; pendant l'obscurité, comme d'habitude, mon souper fut servi, puis la lampe s'alluma, et je me mis à table. 

«Je mangeai quelques fruits seulement: je fis semblant de me verser de l'eau de la carafe, mais je ne bus que celle que j'avais conservée dans mon verre, la substitution, au reste, fut faite assez adroitement pour que mes espions, si j'en avais, ne conçussent aucun soupçon. 

«Après le souper, je donnai les mêmes marques d'engourdissement que la veille; mais cette fois, comme si je succombais à la fatigue ou comme si je me familiarisais avec le danger, je me traînai vers mon lit, et je fis semblant de m'endormir. 

«Cette fois, j'avais retrouvé mon couteau sous l'oreiller, et tout en feignant de dormir, ma main serrait convulsivement la poignée. 

«Deux heures s'écoulèrent sans qu'il se passât rien de nouveau: cette fois, ô mon Dieu! qui m'eût dit cela la veille? je commençais à craindre qu'il ne vînt pas. 

«Enfin, je vis la lampe s'élever doucement et disparaître dans les profondeurs du plafond; ma chambre s'emplit de ténèbres, mais je fis un effort pour percer du regard l'obscurité. 

«Dix minutes à peu près se passèrent. Je n'entendais d'autre bruit que celui du battement de mon cœur. 

«J'implorais le Ciel pour qu'il vînt. 

«Enfin j'entendis le bruit si connu de la porte qui s'ouvrait et se refermait; j'entendis, malgré l'épaisseur du tapis, un pas qui faisait crier le parquet; je vis, malgré l'obscurité, une ombre qui approchait de mon lit. 

\speak  Hâtez-vous, hâtez-vous! dit Felton, ne voyez-vous pas que chacune de vos paroles me brûle comme du plomb fondu! 

\speak  Alors, continua Milady, alors je réunis toutes mes forces, je me rappelai que le moment de la vengeance ou plutôt de la justice avait sonné; je me regardai comme une autre Judith; je me ramassai sur moi-même, mon couteau à la main, et quand je le vis près de moi, étendant les bras pour chercher sa victime, alors, avec le dernier cri de la douleur et du désespoir, je le frappai au milieu de la poitrine. 

«Le misérable! il avait tout prévu: sa poitrine était couverte d'une cotte de mailles; le couteau s'émoussa. 

«Ah! ah! s'écria-t-il en me saisissant le bras et en m'arrachant l'arme qui m'avait si mal servie, vous en voulez à ma vie, ma belle puritaine! mais c'est plus que de la haine, cela, c'est de l'ingratitude! Allons, allons, calmez-vous, ma belle enfant! j'avais cru que vous étiez adoucie. Je ne suis pas de ces tyrans qui gardent les femmes de force: vous ne m'aimez pas, j'en doutais avec ma fatuité ordinaire; maintenant j'en suis convaincu. Demain, vous serez libre.» 

«Je n'avais qu'un désir, c'était qu'il me tuât. 

«--- Prenez garde! lui dis-je, car ma liberté c'est votre déshonneur. 

«--- Expliquez-vous, ma belle sibylle. 

«--- Oui, car, à peine sortie d'ici, je dirai tout, je dirai la violence dont vous avez usé envers moi, je dirai ma captivité. Je dénoncerai ce palais d'infamie; vous êtes bien haut placé, Milord, mais tremblez! Au-dessus de vous il y a le roi, au-dessus du roi il y a Dieu.» 

«Si maître qu'il parût de lui, mon persécuteur laissa échapper un mouvement de colère. Je ne pouvais voir l'expression de son visage, mais j'avais senti frémir son bras sur lequel était posée ma main. 

«--- Alors, vous ne sortirez pas d'ici, dit-il. 

«--- Bien, bien! m'écriai-je, alors le lieu de mon supplice sera aussi celui de mon tombeau. Bien! je mourrai ici et vous verrez si un fantôme qui accuse n'est pas plus terrible encore qu'un vivant qui menace! 

«--- On ne vous laissera aucune arme. 

«--- Il y en a une que le désespoir a mise à la portée de toute créature qui a le courage de s'en servir. Je me laisserai mourir de faim. 

«--- Voyons, dit le misérable, la paix ne vaut-elle pas mieux qu'une pareille guerre? Je vous rends la liberté à l'instant même, je vous proclame une vertu, je vous surnomme la \textit{Lucrèce de l'Angleterre}. 

«--- Et moi je dis que vous en êtes le Sextus, moi je vous dénonce aux hommes comme je vous ai déjà dénoncé à Dieu; et s'il faut que, comme Lucrèce, je signe mon accusation de mon sang, je la signerai. 

«--- Ah! ah! dit mon ennemi d'un ton railleur, alors c'est autre chose. Ma foi, au bout du compte, vous êtes bien ici, rien ne vous manquera, et si vous vous laissez mourir de faim ce sera de votre faute.» 

«À ces mots, il se retira, j'entendis s'ouvrir et se refermer la porte, et je restai abîmée, moins encore, je l'avoue, dans ma douleur, que dans la honte de ne m'être pas vengée. 

«Il me tint parole. Toute la journée, toute la nuit du lendemain s'écoulèrent sans que je le revisse. Mais moi aussi je lui tins parole, et je ne mangeai ni ne bus; j'étais, comme je le lui avais dit, résolue à me laisser mourir de faim. 

«Je passai le jour et la nuit en prière, car j'espérais que Dieu me pardonnerait mon suicide. 

«La seconde nuit la porte s'ouvrit; j'étais couchée à terre sur le parquet, les forces commençaient à m'abandonner. 

«Au bruit je me relevai sur une main. 

«Eh bien, me dit une voix qui vibrait d'une façon trop terrible à mon oreille pour que je ne la reconnusse pas, eh bien! sommes-nous un peu adoucie et paierons nous notre liberté d'une seule promesse de silence? 

«Tenez, moi, je suis bon prince, ajouta-t-il, et, quoique je n'aime pas les puritains, je leur rends justice, ainsi qu'aux puritaines, quand elles sont jolies. Allons, faites-moi un petit serment sur la croix, je ne vous en demande pas davantage. 

«--- Sur la croix! m'écriai-je en me relevant, car à cette voix abhorrée j'avais retrouvé toutes mes forces; sur la croix! je jure que nulle promesse, nulle menace, nulle torture ne me fermera la bouche; sur la croix! je jure de vous dénoncer partout comme un meurtrier, comme un larron d'honneur, comme un lâche; sur la croix! je jure, si jamais je parviens à sortir d'ici, de demander vengeance contre vous au genre humain entier. 

«--- Prenez garde! dit la voix avec un accent de menace que je n'avais pas encore entendu, j'ai un moyen suprême, que je n'emploierai qu'à la dernière extrémité, de vous fermer la bouche ou du moins d'empêcher qu'on ne croie à un seul mot de ce que vous direz.» 

«Je rassemblai toutes mes forces pour répondre par un éclat de rire. 

«Il vit que c'était entre nous désormais une guerre éternelle, une guerre à mort. 

«Écoutez, dit-il, je vous donne encore le reste de cette nuit et la journée de demain; réfléchissez: promettez de vous taire, la richesse, la considération, les honneurs mêmes vous entoureront; menacez de parler, et je vous condamne à l'infamie. 

«--- Vous! m'écriai-je, vous! 

«--- À l'infamie éternelle, ineffaçable! 

«--- Vous!» répétai-je. Oh! je vous le dis, Felton, je le croyais insensé! 

«Oui, moi! reprit-il. 

«--- Ah! laissez-moi, lui dis-je, sortez, si vous ne voulez pas qu'à vos yeux je me brise la tête contre la muraille! 

«--- C'est bien, reprit-il, vous le voulez, à demain soir! 

«--- À demain soir, répondis-je en me laissant tomber et en mordant le tapis de rage\dots» 

Felton s'appuyait sur un meuble, et Milady voyait avec une joie de démon que la force lui manquerait peut-être avant la fin du récit. 