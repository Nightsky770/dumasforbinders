%!TeX root=../musketeersfr.tex 

\chapter{Le Siège De La Rochelle}

\lettrine{L}{e} siège de La Rochelle fut un des grands événements politiques du règne de Louis XIII, et une des grandes entreprises militaires du cardinal. Il est donc intéressant, et même nécessaire, que nous en disions quelques mots; plusieurs détails de ce siège se liant d'ailleurs d'une manière trop importante à l'histoire que nous avons entrepris de raconter, pour que nous les passions sous silence. 

Les vues politiques du cardinal, lorsqu'il entreprit ce siège, étaient considérables. Exposons-les d'abord, puis nous passerons aux vues particulières qui n'eurent peut-être pas sur Son Éminence moins d'influence que les premières. 

Des villes importantes données par Henri IV aux huguenots comme places de sûreté, il ne restait plus que La Rochelle. Il s'agissait donc de détruire ce dernier boulevard du calvinisme, levain dangereux, auquel se venaient incessamment mêler des ferments de révolte civile ou de guerre étrangère. 

Espagnols, Anglais, Italiens mécontents, aventuriers de toute nation, soldats de fortune de toute secte accouraient au premier appel sous les drapeaux des protestants et s'organisaient comme une vaste association dont les branches divergeaient à loisir sur tous les points de l'Europe. 

La Rochelle, qui avait pris une nouvelle importance de la ruine des autres villes calvinistes, était donc le foyer des dissensions et des ambitions. Il y avait plus, son port était la dernière porte ouverte aux Anglais dans le royaume de France; et en la fermant à l'Angleterre, notre éternelle ennemie, le cardinal achevait l'oeuvre de Jeanne d'Arc et du duc de Guise. 

Aussi Bassompierre, qui était à la fois protestant et catholique, protestant de conviction et catholique comme commandeur du Saint-Esprit; Bassompierre, qui était allemand de naissance et français de cœur; Bassompierre, enfin, qui avait un commandement particulier au siège de La Rochelle, disait-il, en chargeant à la tête de plusieurs autres seigneurs protestants comme lui: 

«Vous verrez, messieurs, que nous serons assez bêtes pour prendre La Rochelle!» 

Et Bassompierre avait raison: la canonnade de l'île de Ré lui présageait les dragonnades des Cévennes; la prise de La Rochelle était la préface de la révocation de l'édit de Nantes. 

Mais nous l'avons dit, à côté de ces vues du ministre niveleur et simplificateur, et qui appartiennent à l'histoire, le chroniqueur est bien forcé de reconnaître les petites visées de l'homme amoureux et du rival jaloux. 

Richelieu, comme chacun sait, avait été amoureux de la reine; cet amour avait-il chez lui un simple but politique ou était-ce tout naturellement une de ces profondes passions comme en inspira Anne d'Autriche à ceux qui l'entouraient, c'est ce que nous ne saurions dire; mais en tout cas on a vu, par les développements antérieurs de cette histoire, que Buckingham l'avait emporté sur lui, et que, dans deux ou trois circonstances et particulièrement dans celles des ferrets, il l'avait, grâce au dévouement des trois mousquetaires et au courage de d'Artagnan, cruellement mystifié. 

Il s'agissait donc pour Richelieu, non seulement de débarrasser la France d'un ennemi, mais de se venger d'un rival; au reste, la vengeance devait être grande et éclatante, et digne en tout d'un homme qui tient dans sa main, pour épée de combat, les forces de tout un royaume. 

Richelieu savait qu'en combattant l'Angleterre il combattait Buckingham, qu'en triomphant de l'Angleterre il triomphait de Buckingham, enfin qu'en humiliant l'Angleterre aux yeux de l'Europe il humiliait Buckingham aux yeux de la reine. 

De son côté Buckingham, tout en mettant en avant l'honneur de l'Angleterre, était mû par des intérêts absolument semblables à ceux du cardinal; Buckingham aussi poursuivait une vengeance particulière: sous aucun prétexte, Buckingham n'avait pu rentrer en France comme ambassadeur, il voulait y rentrer comme conquérant. 

Il en résulte que le véritable enjeu de cette partie, que les deux plus puissants royaumes jouaient pour le bon plaisir de deux hommes amoureux, était un simple regard d'Anne d'Autriche. 

Le premier avantage avait été au duc de Buckingham: arrivé inopinément en vue de l'île de Ré avec quatre-vingt-dix vaisseaux et vingt mille hommes à peu près, il avait surpris le comte de Toiras, qui commandait pour le roi dans l'île; il avait, après un combat sanglant, opéré son débarquement. 

Relatons en passant que dans ce combat avait péri le baron de Chantal; le baron de Chantal laissait orpheline une petite fille de dix-huit mois. 

Cette petite fille fut depuis Mme de Sévigné. 

Le comte de Toiras se retira dans la citadelle Saint-Martin avec la garnison, et jeta une centaine d'hommes dans un petit fort qu'on appelait le fort de La Prée. 

Cet événement avait hâté les résolutions du cardinal; et en attendant que le roi et lui pussent aller prendre le commandement du siège de La Rochelle, qui était résolu, il avait fait partir Monsieur pour diriger les premières opérations, et avait fait filer vers le théâtre de la guerre toutes les troupes dont il avait pu disposer. 

C'était de ce détachement envoyé en avant-garde que faisait partie notre ami d'Artagnan. 

Le roi, comme nous l'avons dit, devait suivre, aussitôt son lit de justice tenu, mais en se levant de ce lit de justice, le 28 juin, il s'était senti pris par la fièvre; il n'en avait pas moins voulu partir, mais, son état empirant, il avait été forcé de s'arrêter à Villeroi. 

Or, où s'arrêtait le roi s'arrêtaient les mousquetaires; il en résultait que d'Artagnan, qui était purement et simplement dans les gardes, se trouvait séparé, momentanément du moins, de ses bons amis Athos, Porthos et Aramis; cette séparation, qui n'était pour lui qu'une contrariété, fût certes devenue une inquiétude sérieuse s'il eût pu deviner de quels dangers inconnus il était entouré. 

Il n'en arriva pas moins sans accident au camp établi devant La Rochelle, vers le 10 du mois de septembre de l'année 1627. 

Tout était dans le même état: le duc de Buckingham et ses Anglais, maîtres de l'île de Ré, continuaient d'assiéger mais sans succès, la citadelle de Saint-Martin et le fort de La Prée, et les hostilités avec La Rochelle étaient commencées depuis deux ou trois jours à propos d'un fort que le duc d'Angoulême venait de faire construire près de la ville. 

Les gardes, sous le commandement de M. des Essarts, avaient leur logement aux Minimes. 

Mais nous le savons, d'Artagnan, préoccupé de l'ambition de passer aux mousquetaires, avait rarement fait amitié avec ses camarades; il se trouvait donc isolé et livré à ses propres réflexions. 

Ses réflexions n'étaient pas riantes: depuis un an qu'il était arrivé à Paris, il s'était mêlé aux affaires publiques; ses affaires privées n'avaient pas fait grand chemin comme amour et comme fortune. 

Comme amour, la seule femme qu'il eût aimée était Mme Bonacieux, et Mme Bonacieux avait disparu sans qu'il pût découvrir encore ce qu'elle était devenue. 

Comme fortunes il s'était fait, lui chétif, ennemi du cardinal, c'est-à-dire d'un homme devant lequel tremblaient les plus grands du royaume, à commencer par le roi. 

Cet homme pouvait l'écraser, et cependant il ne l'avait pas fait: pour un esprit aussi perspicace que l'était d'Artagnan, cette indulgence était un jour par lequel il voyait dans un meilleur avenir. 

Puis, il s'était fait encore un autre ennemi moins à craindre, pensait-il, mais que cependant il sentait instinctivement n'être pas à mépriser: cet ennemi, c'était Milady. 

En échange de tout cela il avait acquis la protection et la bienveillance de la reine, mais la bienveillance de la reine était, par le temps qui courait, une cause de plus de persécution; et sa protection, on le sait, protégeait fort mal: témoins Chalais et Mme Bonacieux. 

Ce qu'il avait donc gagné de plus clair dans tout cela c'était le diamant de cinq ou six mille livres qu'il portait au doigt; et encore ce diamant, en supposant que d'Artagnan dans ses projets d'ambition, voulût le garder pour s'en faire un jour un signe de reconnaissance près de la reine n'avait en attendant, puisqu'il ne pouvait s'en défaire, pas plus de valeur que les cailloux qu'il foulait à ses pieds. 

Nous disons «que les cailloux qu'il foulait à ses pieds», car d'Artagnan faisait ces réflexions en se promenant solitairement sur un joli petit chemin qui conduisait du camp au village d'Angoutin; or ces réflexions l'avaient conduit plus loin qu'il ne croyait, et le jour commençait à baisser, lorsqu'au dernier rayon du soleil couchant il lui sembla voir briller derrière une haie le canon d'un mousquet. 

D'Artagnan avait l'œil vif et l'esprit prompt, il comprit que le mousquet n'était pas venu là tout seul et que celui qui le portait ne s'était pas caché derrière une haie dans des intentions amicales. Il résolut donc de gagner au large, lorsque de l'autre côté de la route, derrière un rocher, il aperçut l'extrémité d'un second mousquet. 

C'était évidemment une embuscade. 

Le jeune homme jeta un coup d'œil sur le premier mousquet et vit avec une certaine inquiétude qu'il s'abaissait dans sa direction, mais aussitôt qu'il vit l'orifice du canon immobile il se jeta ventre à terre. En même temps le coup partit, il entendit le sifflement d'une balle qui passait au-dessus de sa tête. 

Il n'y avait pas de temps à perdre, d'Artagnan se redressa d'un bond, et au même moment la balle de l'autre mousquet fit voler les cailloux à l'endroit même du chemin où il s'était jeté la face contre terre. 

D'Artagnan n'était pas un de ces hommes inutilement braves qui cherchent une mort ridicule pour qu'on dise d'eux qu'ils n'ont pas reculé d'un pas, d'ailleurs il ne s'agissait plus de courage ici, d'Artagnan était tombé dans un guet-apens. 

«S'il y a un troisième coup, se dit-il, je suis un homme perdu!» 

Et aussitôt prenant ses jambes à son cou, il s'enfuit dans la direction du camp, avec la vitesse des gens de son pays si renommés pour leur agilité; mais, quelle que fût la rapidité de sa course, le premier qui avait tiré, ayant eu le temps de recharger son arme, lui tira un second coup si bien ajusté, cette fois, que la balle traversa son feutre et le fit voler à dix pas de lui. 

Cependant, comme d'Artagnan n'avait pas d'autre chapeau, il ramassa le sien tout en courant, arriva fort essoufflé et fort pâle, dans son logis, s'assit sans rien dire à personne et se mit à réfléchir. 

Cet événement pouvait avoir trois causes: 

La première et la plus naturelle pouvait être une embuscade des Rochelois, qui n'eussent pas été fâchés de tuer un des gardes de Sa Majesté, d'abord parce que c'était un ennemi de moins, et que cet ennemi pouvait avoir une bourse bien garnie dans sa poche. 

D'Artagnan prit son chapeau, examina le trou de la balle, et secoua la tête. La balle n'était pas une balle de mousquet, c'était une balle d'arquebuse; la justesse du coup lui avait déjà donné l'idée qu'il avait été tiré par une arme particulière: ce n'était donc pas une embuscade militaire, puisque la balle n'était pas de calibre. 

Ce pouvait être un bon souvenir de M. le cardinal. On se rappelle qu'au moment même où il avait, grâce à ce bienheureux rayon de soleil, aperçu le canon du fusil, il s'étonnait de la longanimité de Son Éminence à son égard. 

Mais d'Artagnan secoua la tête. Pour les gens vers lesquels elle n'avait qu'à étendre la main, Son Éminence recourait rarement à de pareils moyens. 

Ce pouvait être une vengeance de Milady. 

Ceci, c'était plus probable. 

Il chercha inutilement à se rappeler ou les traits ou le costume des assassins; il s'était éloigné d'eux si rapidement, qu'il n'avait eu le loisir de rien remarquer. 

«Ah! mes pauvres amis, murmura d'Artagnan, où êtes-vous? et que vous me faites faute!» 

D'Artagnan passa une fort mauvaise nuit. Trois ou quatre fois il se réveilla en sursaut, se figurant qu'un homme s'approchait de son lit pour le poignarder. Cependant le jour parut sans que l'obscurité eût amené aucun incident. 

Mais d'Artagnan se douta bien que ce qui était différé n'était pas perdu. 

D'Artagnan resta toute la journée dans son logis; il se donna pour excuse, vis-à-vis de lui-même, que le temps était mauvais. 

Le surlendemain, à neuf heures, on battit aux champs. Le duc d'Orléans visitait les postes. Les gardes coururent aux armes, d'Artagnan prit son rang au milieu de ses camarades. 

Monsieur passa sur le front de bataille; puis tous les officiers supérieurs s'approchèrent de lui pour lui faire leur cour, M. des Essarts, le capitaine des gardes, comme les autres. 

Au bout d'un instant il parut à d'Artagnan que M. des Essarts lui faisait signe de s'approcher de lui: il attendit un nouveau geste de son supérieur, craignant de se tromper, mais ce geste s'étant renouvelé, il quitta les rangs et s'avança pour prendre l'ordre. 

«Monsieur va demander des hommes de bonne volonté pour une mission dangereuse, mais qui fera honneur à ceux qui l'auront accomplie, et je vous ai fait signe afin que vous vous tinssiez prêt. 

\speak  Merci, mon capitaine!» répondit d'Artagnan, qui ne demandait pas mieux que de se distinguer sous les yeux du lieutenant général. 

En effet, les Rochelois avaient fait une sortie pendant la nuit et avaient repris un bastion dont l'armée royaliste s'était emparée deux jours auparavant; il s'agissait de pousser une reconnaissance perdue pour voir comment l'armée gardait ce bastion. 

Effectivement, au bout de quelques instants, Monsieur éleva la voix et dit: 

«Il me faudrait, pour cette mission, trois ou quatre volontaires conduits par un homme sûr. 

\speak  Quant à l'homme sûr, je l'ai sous la main, Monseigneur, dit M. des Essarts en montrant d'Artagnan; et quant aux quatre ou cinq volontaires, Monseigneur n'a qu'à faire connaître ses intentions, et les hommes ne lui manqueront pas. 

\speak  Quatre hommes de bonne volonté pour venir se faire tuer avec moi!» dit d'Artagnan en levant son épée. 

Deux de ses camarades aux gardes s'élancèrent aussitôt, et deux soldats s'étant joints à eux, il se trouva que le nombre demandé était suffisant; d'Artagnan refusa donc tous les autres, ne voulant pas faire de passe-droit à ceux qui avaient la priorité. 

On ignorait si, après la prise du bastion, les Rochelois l'avaient évacué ou s'ils y avaient laissé garnison; il fallait donc examiner le lieu indiqué d'assez près pour vérifier la chose. 

D'Artagnan partit avec ses quatre compagnons et suivit la tranchée: les deux gardes marchaient au même rang que lui et les soldats venaient par-derrière. 

Ils arrivèrent ainsi, en se couvrant de revêtements, jusqu'à une centaine de pas du bastion! Là, d'Artagnan, en se retournant, s'aperçut que les deux soldats avaient disparu. 

Il crut qu'ayant eu peur ils étaient restés en arrière et continua d'avancer. 

Au détour de la contrescarpe, ils se trouvèrent à soixante pas à peu près du bastion. 

On ne voyait personne, et le bastion semblait abandonné. 

Les trois enfants perdus délibéraient s'ils iraient plus avant, lorsque tout à coup une ceinture de fumée ceignit le géant de pierre, et une douzaine de balles vinrent siffler autour de d'Artagnan et de ses deux compagnons. 

Ils savaient ce qu'ils voulaient savoir: le bastion était gardé. Une plus longue station dans cet endroit dangereux eût donc été une imprudence inutile; d'Artagnan et les deux gardes tournèrent le dos et commencèrent une retraite qui ressemblait à une fuite. 

En arrivant à l'angle de la tranchée qui allait leur servir de rempart, un des gardes tomba: une balle lui avait traversé la poitrine. L'autre, qui était sain et sauf, continua sa course vers le camp. 

D'Artagnan ne voulut pas abandonner ainsi son compagnon, et s'inclina vers lui pour le relever et l'aider à rejoindre les lignes; mais en ce moment deux coups de fusil partirent: une balle cassa la tête du garde déjà blessé, et l'autre vint s'aplatir sur le roc après avoir passé à deux pouces de d'Artagnan. 

Le jeune homme se retourna vivement, car cette attaque ne pouvait venir du bastion, qui était masqué par l'angle de la tranchée. L'idée des deux soldats qui l'avaient abandonné lui revint à l'esprit et lui rappela ses assassins de la surveille; il résolut donc cette fois de savoir à quoi s'en tenir, et tomba sur le corps de son camarade comme s'il était mort. 

Il vit aussitôt deux têtes qui s'élevaient au-dessus d'un ouvrage abandonné qui était à trente pas de là: c'étaient celles de nos deux soldats. D'Artagnan ne s'était pas trompé: ces deux hommes ne l'avaient suivi que pour l'assassiner, espérant que la mort du jeune homme serait mise sur le compte de l'ennemi. 

Seulement, comme il pouvait n'être que blessé et dénoncer leur crime, ils s'approchèrent pour l'achever; heureusement, trompés par la ruse de d'Artagnan, ils négligèrent de recharger leurs fusils. 

Lorsqu'ils furent à dix pas de lui, d'Artagnan, qui en tombant avait eu grand soin de ne pas lâcher son épée, se releva tout à coup et d'un bond se trouva près d'eux. 

Les assassins comprirent que s'ils s'enfuyaient du côté du camp sans avoir tué leur homme, ils seraient accusés par lui; aussi leur première idée fut-elle de passer à l'ennemi. L'un d'eux prit son fusil par le canon, et s'en servit comme d'une massue: il en porta un coup terrible à d'Artagnan, qui l'évita en se jetant de côté, mais par ce mouvement il livra passage au bandit, qui s'élança aussitôt vers le bastion. Comme les Rochelois qui le gardaient ignoraient dans quelle intention cet homme venait à eux, ils firent feu sur lui et il tomba frappé d'une balle qui lui brisa l'épaule. 

Pendant ce temps, d'Artagnan s'était jeté sur le second soldat, l'attaquant avec son épée; la lutte ne fut pas longue, ce misérable n'avait pour se défendre que son arquebuse déchargée; l'épée du garde glissa contre le canon de l'arme devenue inutile et alla traverser la cuisse de l'assassin, qui tomba. D'Artagnan lui mit aussitôt la pointe du fer sur la gorge. 

«Oh! ne me tuez pas! s'écria le bandit; grâce, grâce, mon officier! et je vous dirai tout. 

\speak  Ton secret vaut-il la peine que je te garde la vie au moins? demanda le jeune homme en retenant son bras. 

\speak  Oui; si vous estimez que l'existence soit quelque chose quand on a vingt-deux ans comme vous et qu'on peut arriver à tout, étant beau et brave comme vous l'êtes. 

\speak  Misérable! dit d'Artagnan, voyons, parle vite, qui t'a chargé de m'assassiner? 

\speak  Une femme que je ne connais pas, mais qu'on appelle Milady. 

\speak  Mais si tu ne connais pas cette femme, comment sais-tu son nom? 

\speak  Mon camarade la connaissait et l'appelait ainsi, c'est à lui qu'elle a eu affaire et non pas à moi; il a même dans sa poche une lettre de cette personne qui doit avoir pour vous une grande importance, à ce que je lui ai entendu dire. 

\speak  Mais comment te trouves-tu de moitié dans ce guet-apens? 

\speak  Il m'a proposé de faire le coup à nous deux et j'ai accepté. 

\speak  Et combien vous a-t-elle donné pour cette belle expédition? 

\speak  Cent louis. 

\speak  Eh bien, à la bonne heure, dit le jeune homme en riant, elle estime que je vaux quelque chose; cent louis! c'est une somme pour deux misérables comme vous: aussi je comprends que tu aies accepté, et je te fais grâce, mais à une condition! 

\speak  Laquelle? demanda le soldat inquiet en voyant que tout n'était pas fini. 

\speak  C'est que tu vas aller me chercher la lettre que ton camarade a dans sa poche. 

\speak  Mais, s'écria le bandit, c'est une autre manière de me tuer; comment voulez-vous que j'aille chercher cette lettre sous le feu du bastion? 

\speak  Il faut pourtant que tu te décides à l'aller chercher, ou je te jure que tu vas mourir de ma main. 

\speak  Grâce, monsieur, pitié! au nom de cette jeune dame que vous aimez, que vous croyez morte peut-être, et qui ne l'est pas! s'écria le bandit en se mettant à genoux et s'appuyant sur sa main, car il commençait à perdre ses forces avec son sang. 

\speak  Et d'où sais-tu qu'il y a une jeune femme que j'aime, et que j'ai cru cette femme morte? demanda d'Artagnan. 

\speak  Par cette lettre que mon camarade a dans sa poche. 

\speak  Tu vois bien alors qu'il faut que j'aie cette lettre, dit d'Artagnan; ainsi donc plus de retard, plus d'hésitation, ou quelle que soit ma répugnance à tremper une seconde fois mon épée dans le sang d'un misérable comme toi, je le jure par ma foi d'honnête homme\dots» 

Et à ces mots d'Artagnan fit un geste si menaçant, que le blessé se releva. 

«Arrêtez! arrêtez! s'écria-t-il reprenant courage à force de terreur, j'irai\dots j'irai!\dots» 

D'Artagnan prit l'arquebuse du soldat, le fit passer devant lui et le poussa vers son compagnon en lui piquant les reins de la pointe de son épée. 

C'était une chose affreuse que de voir ce malheureux, laissant sur le chemin qu'il parcourait une longue trace de sang, pâle de sa mort prochaine, essayant de se traîner sans être vu jusqu'au corps de son complice qui gisait à vingt pas de là! 

La terreur était tellement peinte sur son visage couvert d'une froide sueur, que d'Artagnan en eut pitié; et que, le regardant avec mépris: 

«Eh bien, lui dit-il, je vais te montrer la différence qu'il y a entre un homme de cœur et un lâche comme toi; reste, j'irai.» 

Et d'un pas agile, l'œil au guet, observant les mouvements de l'ennemi, s'aidant de tous les accidents de terrain, d'Artagnan parvint jusqu'au second soldat. 

Il y avait deux moyens d'arriver à son but: le fouiller sur la place, ou l'emporter en se faisant un bouclier de son corps, et le fouiller dans la tranchée. 

D'Artagnan préféra le second moyen et chargea l'assassin sur ses épaules au moment même où l'ennemi faisait feu. 

Une légère secousse, le bruit mat de trois balles qui trouaient les chairs, un dernier cri, un frémissement d'agonie prouvèrent à d'Artagnan que celui qui avait voulu l'assassiner venait de lui sauver la vie. 

D'Artagnan regagna la tranchée et jeta le cadavre auprès du blessé aussi pâle qu'un mort. 

Aussitôt il commença l'inventaire: un portefeuille de cuir, une bourse où se trouvait évidemment une partie de la somme que le bandit avait reçue, un cornet et des dés formaient l'héritage du mort. 

Il laissa le cornet et les dés où ils étaient tombés, jeta la bourse au blessé et ouvrit avidement le portefeuille. 

Au milieu de quelques papiers sans importance, il trouva la lettre suivante: c'était celle qu'il était allé chercher au risque de sa vie: «Puisque vous avez perdu la trace de cette femme et qu'elle est maintenant en sûreté dans ce couvent où vous n'auriez jamais dû la laisser arriver, tâchez au moins de ne pas manquer l'homme; sinon, vous savez que j'ai la main longue et que vous payeriez cher les cent louis que vous avez à moi.» 

Pas de signature. Néanmoins il était évident que la lettre venait de Milady. En conséquence, il la garda comme pièce à conviction, et, en sûreté derrière l'angle de la tranchée, il se mit à interroger le blessé. Celui-ci confessa qu'il s'était chargé avec son camarade, le même qui venait d'être tué, d'enlever une jeune femme qui devait sortir de Paris par la barrière de La Villette, mais que, s'étant arrêtés à boire dans un cabaret, ils avaient manqué la voiture de dix minutes. 

«Mais qu'eussiez-vous fait de cette femme? demanda d'Artagnan avec angoisse. 

\speak  Nous devions la remettre dans un hôtel de la place Royale, dit le blessé. 

\speak  Oui! oui! murmura d'Artagnan, c'est bien cela, chez Milady elle-même.» 

Alors le jeune homme comprit en frémissant quelle terrible soif de vengeance poussait cette femme à le perdre, ainsi que ceux qui l'aimaient, et combien elle en savait sur les affaires de la cour, puisqu'elle avait tout découvert. Sans doute elle devait ces renseignements au cardinal. 

Mais, au milieu de tout cela, il comprit, avec un sentiment de joie bien réel, que la reine avait fini par découvrir la prison où la pauvre Mme Bonacieux expiait son dévouement, et qu'elle l'avait tirée de cette prison. Alors la lettre qu'il avait reçue de la jeune femme et son passage sur la route de Chaillot, passage pareil à une apparition, lui furent expliqués. 

Dès lors, ainsi qu'Athos l'avait prédit, il était possible de retrouver Mme Bonacieux, et un couvent n'était pas imprenable. 

Cette idée acheva de lui remettre la clémence au cœur. Il se retourna vers le blessé qui suivait avec anxiété toutes les expressions diverses de son visage, et lui tendant le bras: 

«Allons, lui dit-il, je ne veux pas t'abandonner ainsi. Appuie-toi sur moi et retournons au camp. 

\speak  Oui, dit le blessé, qui avait peine à croire à tant de magnanimité, mais n'est-ce point pour me faire pendre? 

\speak  Tu as ma parole, dit-il, et pour la seconde fois je te donne la vie.» 

Le blessé se laissa glisser à genoux et baisa de nouveau les pieds de son sauveur; mais d'Artagnan, qui n'avait plus aucun motif de rester si près de l'ennemi, abrégea lui-même les témoignages de sa reconnaissance. 

Le garde qui était revenu à la première décharge des Rochelois avait annoncé la mort de ses quatre compagnons. On fut donc à la fois fort étonné et fort joyeux dans le régiment, quand on vit reparaître le jeune homme sain et sauf. 

D'Artagnan expliqua le coup d'épée de son compagnon par une sortie qu'il improvisa. Il raconta la mort de l'autre soldat et les périls qu'ils avaient courus. Ce récit fut pour lui l'occasion d'un véritable triomphe. Toute l'armée parla de cette expédition pendant un jour, et Monsieur lui en fit faire ses compliments. 

Au reste, comme toute belle action porte avec elle sa récompense, la belle action de d'Artagnan eut pour résultat de lui rendre la tranquillité qu'il avait perdue. En effet, d'Artagnan croyait pouvoir être tranquille, puisque, de ses deux ennemis, l'un était tué et l'autre dévoué à ses intérêts. 

Cette tranquillité prouvait une chose, c'est que d'Artagnan ne connaissait pas encore Milady.