%!TeX root=../musketeersfr.tex 

\chapter{Le Conseil Des Mousquetaires}

\lettrine{C}{omme} l'avait prévu Athos, le bastion n'était occupé que par une douzaine de morts tant Français que Rochelois. 

\zz
«Messieurs, dit Athos, qui avait pris le commandement de l'expédition, tandis que Grimaud va mettre la table, commençons par recueillir les fusils et les cartouches; nous pouvons d'ailleurs causer tout en accomplissant cette besogne. Ces messieurs, ajouta-t-il en montrant les morts, ne nous écoutent pas. 

\speak  Mais nous pourrions toujours les jeter dans le fossé, dit Porthos, après toutefois nous être assurés qu'ils n'ont rien dans leurs poches. 

\speak  Oui, dit Aramis, c'est l'affaire de Grimaud. 

\speak  Ah! bien alors, dit d'Artagnan, que Grimaud les fouille et les jette par-dessus les murailles. 

\speak  Gardons-nous-en bien, dit Athos, ils peuvent nous servir. 

\speak  Ces morts peuvent nous servir? dit Porthos. Ah çà, vous devenez fou, cher ami. 

\speak  Ne jugez pas témérairement, disent l'évangile et M. le cardinal, répondit Athos; combien de fusils, messieurs? 

\speak  Douze, répondit Aramis. 

\speak  Combien de coups à tirer? 

\speak  Une centaine. 

\speak  C'est tout autant qu'il nous en faut; chargeons les armes.» 

Les quatre mousquetaires se mirent à la besogne. Comme ils achevaient de charger le dernier fusil, Grimaud fit signe que le déjeuner était servi. 

Athos répondit, toujours par geste, que c'était bien, et indiqua à Grimaud une espèce de poivrière où celui-ci comprit qu'il se devait tenir en sentinelle. Seulement, pour adoucir l'ennui de la faction, Athos lui permit d'emporter un pain, deux côtelettes et une bouteille de vin. 

«Et maintenant, à table», dit Athos. 

Les quatre amis s'assirent à terre, les jambes croisées, comme les Turcs ou comme les tailleurs. 

«Ah! maintenant, dit d'Artagnan, que tu n'as plus la crainte d'être entendu, j'espère que tu vas nous faire part de ton secret, Athos. 

\speak  J'espère que je vous procure à la fois de l'agrément et de la gloire, messieurs, dit Athos. Je vous ai fait faire une promenade charmante; voici un déjeuner des plus succulents, et cinq cents personnes là-bas, comme vous pouvez les voir à travers les meurtrières, qui nous prennent pour des fous ou pour des héros, deux classes d'imbéciles qui se ressemblent assez. 

\speak  Mais ce secret? demanda d'Artagnan. 

\speak  Le secret, dit Athos, c'est que j'ai vu Milady hier soir.» 

D'Artagnan portait son verre à ses lèvres; mais à ce nom de Milady, la main lui trembla si fort, qu'il le posa à terre pour ne pas en répandre le contenu. 

«Tu as vu ta fem\dots 

\speak  Chut donc! interrompit Athos: vous oubliez, mon cher, que ces messieurs ne sont pas initiés comme vous dans le secret de mes affaires de ménage; j'ai vu Milady. 

\speak  Et où cela? demanda d'Artagnan. 

\speak  À deux lieues d'ici à peu près, à l'auberge du Colombier-Rouge. 

\speak  En ce cas je suis perdu, dit d'Artagnan. 

\speak  Non, pas tout à fait encore, reprit Athos; car, à cette heure, elle doit avoir quitté les côtes de France.» 

D'Artagnan respira. 

«Mais au bout du compte, demanda Porthos, qu'est-ce donc que cette Milady? 

\speak  Une femme charmante, dit Athos en dégustant un verre de vin mousseux. Canaille d'hôtelier! s'écria-t-il, qui nous donne du vin d'Anjou pour du vin de Champagne, et qui croit que nous nous y laisserons prendre! Oui, continua-t-il, une femme charmante qui a eu des bontés pour notre ami d'Artagnan, qui lui a fait je ne sais quelle noirceur dont elle a essayé de se venger, il y a un mois en voulant le faire tuer à coups de mousquet, il y a huit jours en essayant de l'empoisonner, et hier en demandant sa tête au cardinal. 

\speak  Comment! en demandant ma tête au cardinal? s'écria d'Artagnan, pâle de terreur. 

\speak  Ça, dit Porthos, c'est vrai comme l'évangile; je l'ai entendu de mes deux oreilles. 

\speak  Moi aussi, dit Aramis. 

\speak  Alors, dit d'Artagnan en laissant tomber son bras avec découragement, il est inutile de lutter plus longtemps; autant que je me brûle la cervelle et que tout soit fini! 

\speak  C'est la dernière sottise qu'il faut faire, dit Athos, attendu que c'est la seule à laquelle il n'y ait pas de remède. 

\speak  Mais je n'en réchapperai jamais, dit d'Artagnan, avec des ennemis pareils. D'abord mon inconnu de Meung; ensuite de Wardes, à qui j'ai donné trois coups d'épée; puis Milady, dont j'ai surpris le secret; enfin, le cardinal, dont j'ai fait échouer la vengeance. 

\speak  Eh bien, dit Athos, tout cela ne fait que quatre, et nous sommes quatre, un contre un. Pardieu! si nous en croyons les signes que nous fait Grimaud, nous allons avoir affaire à un bien plus grand nombre de gens. Qu'y a-t-il, Grimaud? Considérant la gravité de la circonstance, je vous permets de parler, mon ami, mais soyez laconique je vous prie. Que voyez-vous? 

\speak  Une troupe. 

\speak  De combien de personnes? 

\speak  De vingt hommes. 

\speak  Quels hommes? 

\speak  Seize pionniers, quatre soldats. 

\speak  À combien de pas sont-ils? 

\speak  À cinq cents pas. 

\speak  Bon, nous avons encore le temps d'achever cette volaille et de boire un verre de vin à ta santé, d'Artagnan! 

\speak  À ta santé! répétèrent Porthos et Aramis. 

\speak  Eh bien donc, à ma santé! quoique je ne croie pas que vos souhaits me servent à grand-chose. 

\speak  Bah! dit Athos, Dieu est grand, comme disent les sectateurs de Mahomet, et l'avenir est dans ses mains.» 

Puis, avalant le contenu de son verre, qu'il posa près de lui, Athos se leva nonchalamment, prit le premier fusil venu et s'approcha d'une meurtrière. 

Porthos, Aramis et d'Artagnan en firent autant. Quant à Grimaud, il reçut l'ordre de se placer derrière les quatre amis afin de recharger les armes. 

Au bout d'un instant on vit paraître la troupe; elle suivait une espèce de boyau de tranchée qui établissait une communication entre le bastion et la ville. 

«Pardieu! dit Athos, c'est bien la peine de nous déranger pour une vingtaine de drôles armés de pioches, de hoyaux et de pelles! Grimaud n'aurait eu qu'à leur faire signe de s'en aller, et je suis convaincu qu'ils nous eussent laissés tranquilles. 

\speak  J'en doute, observa d'Artagnan, car ils avancent fort résolument de ce côté. D'ailleurs, il y a avec les travailleurs quatre soldats et un brigadier armés de mousquets. 

\speak  C'est qu'ils ne nous ont pas vus, reprit Athos. 

\speak  Ma foi! dit Aramis, j'avoue que j'ai répugnance à tirer sur ces pauvres diables de bourgeois. 

\speak  Mauvais prêtre, répondit Porthos, qui a pitié des hérétiques! 

\speak  En vérité, dit Athos, Aramis a raison, je vais les prévenir. 

\speak  Que diable faites-vous donc? s'écria d'Artagnan, vous allez vous faire fusiller, mon cher.» 

Mais Athos ne tint aucun compte de l'avis, et, montant sur la brèche, son fusil d'une main et son chapeau de l'autre: 

«Messieurs, dit-il en s'adressant aux soldats et aux travailleurs, qui, étonnés de son apparition, s'arrêtaient à cinquante pas environ du bastion, et en les saluant courtoisement, messieurs, nous sommes, quelques amis et moi, en train de déjeuner dans ce bastion. Or, vous savez que rien n'est désagréable comme d'être dérangé quand on déjeune; nous vous prions donc, si vous avez absolument affaire ici, d'attendre que nous ayons fini notre repas, ou de repasser plus tard, à moins qu'il ne vous prenne la salutaire envie de quitter le parti de la rébellion et de venir boire avec nous à la santé du roi de France. 

\speak  Prends garde, Athos! s'écria d'Artagnan; ne vois-tu pas qu'ils te mettent en joue? 

\speak  Si fait, si fait, dit Athos, mais ce sont des bourgeois qui tirent fort mal, et qui n'ont garde de me toucher.» 

En effet, au même instant quatre coups de fusil partirent, et les balles vinrent s'aplatir autour d'Athos, mais sans qu'une seule le touchât. 

Quatre coups de fusil leur répondirent presque en même temps, mais ils étaient mieux dirigés que ceux des agresseurs, trois soldats tombèrent tués raide, et un des travailleurs fut blessé. 

«Grimaud, un autre mousquet!» dit Athos toujours sur la brèche. 

Grimaud obéit aussitôt. De leur côté, les trois amis avaient chargé leurs armes; une seconde décharge suivit la première: le brigadier et deux pionniers tombèrent morts, le reste de la troupe prit la fuite. 

«Allons, messieurs, une sortie», dit Athos. 

Et les quatre amis, s'élançant hors du fort, parvinrent jusqu'au champ de bataille, ramassèrent les quatre mousquets des soldats et la demi-pique du brigadier; et, convaincus que les fuyards ne s'arrêteraient qu'à la ville, reprirent le chemin du bastion, rapportant les trophées de leur victoire. 

«Rechargez les armes, Grimaud, dit Athos, et nous, messieurs, reprenons notre déjeuner et continuons notre conversation. Où en étions-nous? 

\speak  Je me le rappelle, dit d'Artagnan; vous disiez qu'après avoir demandé ma tête au cardinal, milady avait quitté les côtes de France. 

\speak  C'est vrai. 

\speak  Et où va-t-elle? ajouta d'Artagnan, qui se préoccupait fort de l'itinéraire que devrait suivre milady. 

\speak  Elle va en Angleterre, répondit Athos. 

\speak  Et dans quel but? 

\speak  Dans le but d'assassiner ou de faire assassiner Buckingham.» 

D'Artagnan poussa une exclamation de surprise et d'indignation. 

«Mais c'est infâme! s'écria-t-il. 

\speak  Oh! quant à cela, dit Athos, je vous prie de croire que je m'en inquiète fort peu. Maintenant que vous avez fini, Grimaud, continua Athos, prenez la demi-pique de notre brigadier, attachez-y une serviette et plantez-la au haut de notre bastion, afin que ces rebelles de Rochelois voient qu'ils ont affaire à de braves et loyaux soldats du roi.» 

Grimaud obéit sans répondre. Un instant après le drapeau blanc flottait au-dessus de la tête des quatre amis; un tonnerre d'applaudissements salua son apparition; la moitié du camp était aux barrières. 

«Comment! reprit d'Artagnan, tu t'inquiètes fort peu qu'elle tue ou qu'elle fasse tuer Buckingham? Mais le duc est notre ami. 

\speak  Le duc est Anglais, le duc combat contre nous; qu'elle fasse du duc ce qu'elle voudra, je m'en soucie comme d'une bouteille vide.» 

Et Athos envoya à quinze pas de lui une bouteille qu'il tenait, et dont il venait de transvaser jusqu'à la dernière goutte dans son verre. 

«Un instant, dit d'Artagnan, je n'abandonne pas Buckingham ainsi; il nous avait donné de fort beaux chevaux. 

\speak  Et surtout de fort belles selles, ajouta Porthos, qui, à ce moment même, portait à son manteau le galon de la sienne. 

\speak  Puis, observa Aramis, Dieu veut la conversion et non la mort du pécheur. 

\speak  \textit{Amen}, dit Athos, et nous reviendrons là-dessus plus tard, si tel est votre plaisir; mais ce qui, pour le moment, me préoccupait le plus, et je suis sûr que tu me comprendras, d'Artagnan, c'était de reprendre à cette femme une espèce de blanc-seing qu'elle avait extorqué au cardinal, et à l'aide duquel elle devait impunément se débarrasser de toi et peut-être de nous. 

\speak  Mais c'est donc un démon que cette créature? dit Porthos en tendant son assiette à Aramis, qui découpait une volaille. 

\speak  Et ce blanc-seing, dit d'Artagnan, ce blanc-seing est-il resté entre ses mains? 

\speak  Non, il est passé dans les miennes; je ne dirai pas que ce fut sans peine, par exemple, car je mentirais. 

\speak  Mon cher Athos, dit d'Artagnan, je ne compte plus les fois que je vous dois la vie. 

\speak  Alors c'était donc pour venir près d'elle que vous nous avez quittés? demanda Aramis. 

\speak  Justement. Et tu as cette lettre du cardinal? dit d'Artagnan. 

\speak  La voici», dit Athos. 

Et il tira le précieux papier de la poche de sa casaque. 

D'Artagnan le déplia d'une main dont il n'essayait pas même de dissimuler le tremblement et lut: 

\begin{mail}{5 \textit{décembre} 1627.}

C'est par mon ordre et pour le bien de l'État que le porteur du présent a fait ce qu'il a fait.
\closeletter{Richelieu}
\end{mail}

«En effet, dit Aramis, c'est une absolution dans toutes les règles. 

\speak  Il faut déchirer ce papier, s'écria d'Artagnan, qui semblait lire sa sentence de mort. 

\speak  Bien au contraire, dit Athos, il faut le conserver précieusement, et je ne donnerais pas ce papier quand on le couvrirait de pièces d'or. 

\speak  Et que va-t-elle faire maintenant? demanda le jeune homme. 

\speak  Mais, dit négligemment Athos, elle va probablement écrire au cardinal qu'un damné mousquetaire, nommé Athos, lui a arraché son sauf-conduit; elle lui donnera dans la même lettre le conseil de se débarrasser, en même temps que de lui, de ses deux amis, Porthos et Aramis; le cardinal se rappellera que ce sont les mêmes hommes qu'il rencontre toujours sur son chemin; alors, un beau matin il fera arrêter d'Artagnan, et, pour qu'il ne s'ennuie pas tout seul, il nous enverra lui tenir compagnie à la Bastille. 

\speak  Ah çà, mais, dit Porthos, il me semble que vous faites là de tristes plaisanteries, mon cher. 

\speak  Je ne plaisante pas, répondit Athos. 

\speak  Savez-vous, dit Porthos, que tordre le cou à cette damnée Milady serait un péché moins grand que de le tordre à ces pauvres diables de huguenots, qui n'ont jamais commis d'autres crimes que de chanter en français des psaumes que nous chantons en latin? 

\speak  Qu'en dit l'abbé? demanda tranquillement Athos. 

\speak  Je dis que je suis de l'avis de Porthos, répondit Aramis. 

\speak  Et moi donc! fit d'Artagnan. 

\speak  Heureusement qu'elle est loin, observa Porthos; car j'avoue qu'elle me gênerait fort ici. 

\speak  Elle me gêne en Angleterre aussi bien qu'en France, dit Athos. 

\speak  Elle me gêne partout, continua d'Artagnan. 

\speak  Mais puisque vous la teniez, dit Porthos, que ne l'avez-vous noyée, étranglée, pendue? il n'y a que les morts qui ne reviennent pas. 

\speak  Vous croyez cela, Porthos? répondit le mousquetaire avec un sombre sourire que d'Artagnan comprit seul. 

\speak  J'ai une idée, dit d'Artagnan. 

\speak  Voyons, dirent les mousquetaires. 

\speak  Aux armes!» cria Grimaud. 

Les jeunes gens se levèrent vivement et coururent aux fusils. 

Cette fois, une petite troupe s'avançait composée de vingt ou vingt-cinq hommes; mais ce n'étaient plus des travailleurs, c'étaient des soldats de la garnison. 

«Si nous retournions au camp? dit Porthos, il me semble que la partie n'est pas égale. 

\speak  Impossible pour trois raisons, répondit Athos: la première, c'est que nous n'avons pas fini de déjeuner; la seconde, c'est que nous avons encore des choses d'importance à dire; la troisième, c'est qu'il s'en manque encore de dix minutes que l'heure ne soit écoulée. 

\speak  Voyons, dit Aramis, il faut cependant arrêter un plan de bataille. 

\speak  Il est bien simple, répondit Athos: aussitôt que l'ennemi est à portée de mousquet, nous faisons feu; s'il continue d'avancer, nous faisons feu encore, nous faisons feu tant que nous avons des fusils chargés; si ce qui reste de la troupe veut encore monter à l'assaut, nous laissons les assiégeants descendre jusque dans le fossé, et alors nous leur poussons sur la tête ce pan de mur qui ne tient plus que par un miracle d'équilibre. 

\speak  Bravo! s'écria Porthos; décidément, Athos, vous étiez né pour être général, et le cardinal, qui se croit un grand homme de guerre, est bien peu de chose auprès de vous. 

\speak  Messieurs, dit Athos, pas de double emploi, je vous prie; visez bien chacun votre homme. 

\speak  Je tiens le mien, dit d'Artagnan. 

\speak  Et moi le mien dit Porthos. 

\speak  Et moi idem, dit Aramis. 

\speak  Alors feu!» dit Athos. 

Les quatre coups de fusil ne firent qu'une détonation, et quatre hommes tombèrent. 

Aussitôt le tambour battit, et la petite troupe s'avança au pas de charge. 

Alors les coups de fusil se succédèrent sans régularité, mais toujours envoyés avec la même justesse. Cependant, comme s'ils eussent connu la faiblesse numérique des amis, les Rochelois continuaient d'avancer au pas de course. 

Sur trois autres coups de fusil, deux hommes tombèrent; mais cependant la marche de ceux qui restaient debout ne se ralentissait pas. 

Arrivés au bas du bastion, les ennemis étaient encore douze ou quinze; une dernière décharge les accueillit, mais ne les arrêta point: ils sautèrent dans le fossé et s'apprêtèrent à escalader la brèche. 

«Allons, mes amis, dit Athos, finissons-en d'un coup: à la muraille! à la muraille!» 

Et les quatre amis, secondés par Grimaud, se mirent à pousser avec le canon de leurs fusils un énorme pan de mur, qui s'inclina comme si le vent le poussait, et, se détachant de sa base, tomba avec un bruit horrible dans le fossé: puis on entendit un grand cri, un nuage de poussière monta vers le ciel, et tout fut dit. 

«Les aurions-nous écrasés depuis le premier jusqu'au dernier? demanda Athos. 

\speak  Ma foi, cela m'en a l'air, dit d'Artagnan. 

\speak  Non, dit Porthos, en voilà deux ou trois qui se sauvent tout éclopés.» 

En effet, trois ou quatre de ces malheureux, couverts de boue et de sang, fuyaient dans le chemin creux et regagnaient la ville: c'était tout ce qui restait de la petite troupe. 

Athos regarda à sa montre. 

«Messieurs, dit-il, il y a une heure que nous sommes ici, et maintenant le pari est gagné, mais il faut être beaux joueurs: d'ailleurs d'Artagnan ne nous a pas dit son idée.» 

Et le mousquetaire, avec son sang-froid habituel, alla s'asseoir devant les restes du déjeuner. 

«Mon idée? dit d'Artagnan. 

\speak  Oui, vous disiez que vous aviez une idée, répliqua Athos. 

\speak  Ah! j'y suis, reprit d'Artagnan: je passe en Angleterre une seconde fois, je vais trouver M. de Buckingham et je l'avertis du complot tramé contre sa vie. 

\speak  Vous ne ferez pas cela, d'Artagnan, dit froidement Athos. 

\speak  Et pourquoi cela? ne l'ai-je pas fait déjà? 

\speak  Oui, mais à cette époque nous n'étions pas en guerre; à cette époque, M. de Buckingham était un allié et non un ennemi: ce que vous voulez faire serait taxé de trahison.» 

D'Artagnan comprit la force de ce raisonnement et se tut. 

«Mais, dit Porthos, il me semble que j'ai une idée à mon tour. 

\speak  Silence pour l'idée de M. Porthos! dit Aramis. 

\speak  Je demande un congé à M. de Tréville, sous un prétexte quelconque que vous trouverez: je ne suis pas fort sur les prétextes, moi. Milady ne me connaît pas, je m'approche d'elle sans qu'elle me redoute, et lorsque je trouve ma belle, je l'étrangle. 

\speak  Eh bien, dit Athos, je ne suis pas très éloigné d'adopter l'idée de Porthos. 

\speak  Fi donc! dit Aramis, tuer une femme! Non, tenez, moi, j'ai la véritable idée. 

\speak  Voyons votre idée, Aramis! demanda Athos, qui avait beaucoup de déférence pour le jeune mousquetaire. 

\speak  Il faut prévenir la reine. 

\speak  Ah! ma foi, oui, s'écrièrent ensemble Porthos et d'Artagnan; je crois que nous touchons au moyen. 

\speak  Prévenir la reine! dit Athos, et comment cela? Avons-nous des relations à la cour? Pouvons-nous envoyer quelqu'un à Paris sans qu'on le sache au camp? D'ici à Paris il y a cent quarante lieues; notre lettre ne sera pas à Angers que nous serons au cachot, nous. 

\speak  Quant à ce qui est de faire remettre sûrement une lettre à Sa Majesté, proposa Aramis en rougissant, moi, je m'en charge; je connais à Tours une personne adroite\dots» 

Aramis s'arrêta en voyant sourire Athos. 

«Eh bien, vous n'adoptez pas ce moyen, Athos? dit d'Artagnan. 

\speak  Je ne le repousse pas tout à fait, dit Athos, mais je voulais seulement faire observer à Aramis qu'il ne peut quitter le camp; que tout autre qu'un de nous n'est pas sûr; que, deux heures après que le messager sera parti, tous les capucins, tous les alguazils, tous les bonnets noirs du cardinal sauront votre lettre par cœur, et qu'on arrêtera vous et votre adroite personne. 

\speak  Sans compter, objecta Porthos, que la reine sauvera M. de Buckingham, mais ne nous sauvera pas du tout, nous autres. 

\speak  Messieurs, dit d'Artagnan, ce qu'objecte Porthos est plein de sens. 

\speak  Ah! ah! que se passe-t-il donc dans la ville? dit Athos. 

\speak  On bat la générale.» 

Les quatre amis écoutèrent, et le bruit du tambour parvint effectivement jusqu'à eux. 

«Vous allez voir qu'ils vont nous envoyer un régiment tout entier, dit Athos. 

\speak  Vous ne comptez pas tenir contre un régiment tout entier? dit Porthos. 

\speak  Pourquoi pas? dit le mousquetaire, je me sens en train; et je tiendrais devant une armée, si nous avions seulement eu la précaution de prendre une douzaine de bouteilles en plus. 

\speak  Sur ma parole, le tambour se rapproche, dit d'Artagnan. 

\speak  Laissez-le se rapprocher, dit Athos; il y a pour un quart d'heure de chemin d'ici à la ville, et par conséquent de la ville ici. C'est plus de temps qu'il ne nous en faut pour arrêter notre plan; si nous nous en allons d'ici, nous ne retrouverons jamais un endroit aussi convenable. Et tenez, justement, messieurs, voilà la vraie idée qui me vient. 

\speak  Dites alors. 

\speak  Permettez que je donne à Grimaud quelques ordres indispensables.» 

Athos fit signe à son valet d'approcher. 

«Grimaud, dit Athos, en montrant les morts qui gisaient dans le bastion, vous allez prendre ces messieurs, vous allez les dresser contre la muraille, vous leur mettrez leur chapeau sur la tête et leur fusil à la main. 

\speak  O grand homme! s'écria d'Artagnan, je te comprends. 

\speak  Vous comprenez? dit Porthos. 

\speak  Et toi, comprends-tu, Grimaud?» demanda Aramis. 

Grimaud fit signe que oui. 

«C'est tout ce qu'il faut, dit Athos, revenons à mon idée. 

\speak  Je voudrais pourtant bien comprendre, observa Porthos. 

\speak  C'est inutile. 

\speak  Oui, oui, l'idée d'Athos, dirent en même temps d'Artagnan et Aramis. 

\speak  Cette Milady, cette femme, cette créature, ce démon, a un beau-frère, à ce que vous m'avez dit, je crois, d'Artagnan. 

\speak  Oui, je le connais beaucoup même, et je crois aussi qu'il n'a pas une grande sympathie pour sa belle-soeur. 

\speak  Il n'y a pas de mal à cela, répondit Athos, et il la détesterait que cela n'en vaudrait que mieux. 

\speak  En ce cas nous sommes servis à souhait. 

\speak  Cependant, dit Porthos, je voudrais bien comprendre ce que fait Grimaud. 

\speak  Silence, Porthos! dit Aramis. 

\speak  Comment se nomme ce beau-frère? 

\speak  Lord de Winter. 

\speak  Où est-il maintenant? 

\speak  Il est retourné à Londres au premier bruit de guerre. 

\speak  Eh bien, voilà justement l'homme qu'il nous faut, dit Athos, c'est celui qu'il nous convient de prévenir; nous lui ferons savoir que sa belle-soeur est sur le point d'assassiner quelqu'un, et nous le prierons de ne pas la perdre de vue. Il y a bien à Londres, je l'espère, quelque établissement dans le genre des Madelonnettes ou des Filles repenties; il y fait mettre sa belle-soeur, et nous sommes tranquilles. 

\speak  Oui, dit d'Artagnan, jusqu'à ce qu'elle en sorte. 

\speak  Ah! ma foi, reprit Athos, vous en demandez trop, d'Artagnan, je vous ai donné tout ce que j'avais et je vous préviens que c'est le fond de mon sac. 

\speak  Moi, je trouve que c'est ce qu'il y a de mieux, dit Aramis; nous prévenons à la fois la reine et Lord de Winter. 

\speak  Oui, mais par qui ferons-nous porter la lettre à Tours et la lettre à Londres? 

\speak  Je réponds de Bazin, dit Aramis. 

\speak  Et moi de Planchet, continua d'Artagnan. 

\speak  En effet, dit Porthos, si nous ne pouvons nous absenter du camp, nos laquais peuvent le quitter. 

\speak  Sans doute, dit Aramis, et dès aujourd'hui nous écrivons les lettres, nous leur donnons de l'argent, et ils partent. 

\speak  Nous leur donnons de l'argent? reprit Athos, vous en avez donc, de l'argent?» 

Les quatre amis se regardèrent, et un nuage passa sur les fronts qui s'étaient un instant éclaircis. 

«Alerte! cria d'Artagnan, je vois des points noirs et des points rouges qui s'agitent là-bas; que disiez-vous donc d'un régiment, Athos? c'est une véritable armée. 

\speak  Ma foi, oui, dit Athos, les voilà. Voyez-vous les sournois qui venaient sans tambours ni trompettes. Ah! ah! tu as fini, Grimaud?» 

Grimaud fit signe que oui, et montra une douzaine de morts qu'il avait placés dans les attitudes les plus pittoresques: les uns au port d'armes, les autres ayant l'air de mettre en joue, les autres l'épée à la main. 

«Bravo! reprit Athos, voilà qui fait honneur à ton imagination. 

\speak  C'est égal, dit Porthos, je voudrais cependant bien comprendre. 

\speak  Décampons d'abord, interrompit d'Artagnan, tu comprendras après. 

\speak  Un instant, messieurs, un instant! donnons le temps à Grimaud de desservir. 

\speak  Ah! dit Aramis, voici les points noirs et les points rouges qui grandissent fort visiblement et je suis de l'avis de d'Artagnan; je crois que nous n'avons pas de temps à perdre pour regagner notre camp. 

\speak  Ma foi, dit Athos, je n'ai plus rien contre la retraite: nous avions parié pour une heure, nous sommes restés une heure et demie; il n'y a rien à dire; partons, messieurs, partons.» 

Grimaud avait déjà pris les devants avec le panier et la desserte. 

Les quatre amis sortirent derrière lui et firent une dizaine de pas. 

«Eh! s'écria Athos, que diable faisons-nous, messieurs? 

\speak  Avez-vous oublié quelque chose? demanda Aramis. 

\speak  Et le drapeau, morbleu! Il ne faut pas laisser un drapeau aux mains de l'ennemi, même quand ce drapeau ne serait qu'une serviette.» 

Et Athos s'élança dans le bastion, monta sur la plate-forme, et enleva le drapeau; seulement comme les Rochelois étaient arrivés à portée de mousquet, ils firent un feu terrible sur cet homme, qui, comme par plaisir, allait s'exposer aux coups. 

Mais on eût dit qu'Athos avait un charme attaché à sa personne, les balles passèrent en sifflant tout autour de lui, pas une ne le toucha. 

Athos agita son étendard en tournant le dos aux gens de la ville et en saluant ceux du camp. Des deux côtés de grands cris retentirent, d'un côté des cris de colère, de l'autre des cris d'enthousiasme. 

Une seconde décharge suivit la première, et trois balles, en la trouant, firent réellement de la serviette un drapeau. On entendit les clameurs de tout le camp qui criait: 

\speak  Descendez, descendez!» 

Athos descendit; ses camarades, qui l'attendaient avec anxiété, le virent paraître avec joie. 

\speak  Allons, Athos, allons, dit d'Artagnan, allongeons, allongeons; maintenant que nous avons tout trouvé, excepté l'argent, il serait stupide d'être tués.» 

Mais Athos continua de marcher majestueusement, quelque observation que pussent lui faire ses compagnons, qui, voyant toute observation inutile, réglèrent leur pas sur le sien. 

Grimaud et son panier avaient pris les devants et se trouvaient tous deux hors d'atteinte. 

Au bout d'un instant on entendit le bruit d'une fusillade enragée. 

«Qu'est-ce que cela? demanda Porthos, et sur quoi tirent-ils? je n'entends pas siffler les balles et je ne vois personne. 

\speak  Ils tirent sur nos morts, répondit Athos. 

\speak  Mais nos morts ne répondront pas. 

\speak  Justement; alors ils croiront à une embuscade, ils délibéreront; ils enverront un parlementaire, et quand ils s'apercevront de la plaisanterie, nous serons hors de la portée des balles. Voilà pourquoi il est inutile de gagner une pleurésie en nous pressant. 

\speak  Oh! je comprends, s'écria Porthos émerveillé. 

\speak  C'est bien heureux!» dit Athos en haussant les épaules. 

De leur côté, les Français, en voyant revenir les quatre amis au pas, poussaient des cris d'enthousiasme. 

Enfin une nouvelle mousquetade se fit entendre, et cette fois les balles vinrent s'aplatir sur les cailloux autour des quatre amis et siffler lugubrement à leurs oreilles. Les Rochelois venaient enfin de s'emparer du bastion. 

«Voici des gens bien maladroits, dit Athos; combien en avons-nous tué? douze? 

\speak  Ou quinze. 

\speak  Combien en avons-nous écrasé? 

\speak  Huit ou dix. 

\speak  Et en échange de tout cela pas une égratignure? Ah! si fait! Qu'avez-vous donc là à la main, d'Artagnan? du sang, ce me semble? 

\speak  Ce n'est rien, dit d'Artagnan. 

\speak  Une balle perdue? 

\speak  Pas même. 

\speak  Qu'est-ce donc alors?» 

Nous l'avons dit, Athos aimait d'Artagnan comme son enfant, et ce caractère sombre et inflexible avait parfois pour le jeune homme des sollicitudes de père. 

«Une écorchure, reprit d'Artagnan; mes doigts ont été pris entre deux pierres, celle du mur et celle de ma bague; alors la peau s'est ouverte. 

\speak  Voilà ce que c'est que d'avoir des diamants, mon maître, dit dédaigneusement Athos. 

\speak  Ah çà, mais, s'écria Porthos, il y a un diamant en effet, et pourquoi diable alors, puisqu'il y a un diamant, nous plaignons-nous de ne pas avoir d'argent? 

\speak  Tiens, au fait! dit Aramis. 

\speak  À la bonne heure, Porthos; cette fois-ci voilà une idée. 

\speak  Sans doute, dit Porthos, en se rengorgeant sur le compliment d'Athos, puisqu'il y a un diamant, vendons-le. 

\speak  Mais, dit d'Artagnan, c'est le diamant de la reine. 

\speak  Raison de plus, reprit Athos, la reine sauvant M. de Buckingham son amant, rien de plus juste; la reine nous sauvant, nous ses amis, rien de plus moral: vendons le diamant. Qu'en pense monsieur l'abbé? Je ne demande pas l'avis de Porthos, il est donné. 

\speak  Mais je pense, dit Aramis en rougissant, que sa bague ne venant pas d'une maîtresse, et par conséquent n'étant pas un gage d'amour, d'Artagnan peut la vendre. 

\speak  Mon cher, vous parlez comme la théologie en personne. Ainsi votre avis est?\dots 

\speak  De vendre le diamant, répondit Aramis. 

\speak  Eh bien, dit gaiement d'Artagnan, vendons le diamant et n'en parlons plus.» 

La fusillade continuait, mais les amis étaient hors de portée, et les Rochelois ne tiraient plus que pour l'acquit de leur conscience. 

«Ma foi, dit Athos, il était temps que cette idée vînt à Porthos; nous voici au camp. Ainsi, messieurs, pas un mot de plus sur cette affaire. On nous observe, on vient à notre rencontre, nous allons être portés en triomphe.» 

En effet, comme nous l'avons dit, tout le camp était en émoi; plus de deux mille personnes avaient assisté, comme à un spectacle, à l'heureuse forfanterie des quatre amis, forfanterie dont on était bien loin de soupçonner le véritable motif. On n'entendait que le cri de: Vivent les gardes! Vivent les mousquetaires! M. de Busigny était venu le premier serrer la main à Athos et reconnaître que le pari était perdu. Le dragon et le Suisse l'avaient suivi, tous les camarades avaient suivi le dragon et le Suisse. C'étaient des félicitations, des poignées de main, des embrassades à n'en plus finir, des rires inextinguibles à l'endroit des Rochelois; enfin, un tumulte si grand, que M. le cardinal crut qu'il y avait émeute et envoya La Houdinière, son capitaine des gardes, s'informer de ce qui se passait. 

La chose fut racontée au messager avec toute l'efflorescence de l'enthousiasme. 

«Eh bien? demanda le cardinal en voyant La Houdinière. 

\speak  Eh bien, Monseigneur, dit celui-ci, ce sont trois mousquetaires et un garde qui ont fait le pari avec M. de Busigny d'aller déjeuner au bastion Saint-Gervais, et qui, tout en déjeunant, ont tenu là deux heures contre l'ennemi, et ont tué je ne sais combien de Rochelois. 

\speak  Vous êtes-vous informé du nom de ces trois mousquetaires? 

\speak  Oui, Monseigneur. 

\speak  Comment les appelle-t-on? 

\speak  Ce sont MM. Athos, Porthos et Aramis. 

\speak  Toujours mes trois braves! murmura le cardinal. Et le garde? 

\speak  M. d'Artagnan. 

\speak  Toujours mon jeune drôle! Décidément il faut que ces quatre hommes soient à moi.» 

Le soir même, le cardinal parla à M. de Tréville de l'exploit du matin, qui faisait la conversation de tout le camp. M. de Tréville, qui tenait le récit de l'aventure de la bouche même de ceux qui en étaient les héros, la raconta dans tous ses détails à Son Éminence, sans oublier l'épisode de la serviette. 

«C'est bien, monsieur de Tréville, dit le cardinal, faites-moi tenir cette serviette, je vous prie. J'y ferai broder trois fleurs de lis d'or, et je la donnerai pour guidon à votre compagnie. 

\speak  Monseigneur, dit M. de Tréville, il y aura injustice pour les gardes: M. d'Artagnan n'est pas à moi, mais à M. des Essarts. 

\speak  Eh bien, prenez-le, dit le cardinal; il n'est pas juste que, puisque ces quatre braves militaires s'aiment tant, ils ne servent pas dans la même compagnie.» 

Le même soir, M. de Tréville annonça cette bonne nouvelle aux trois mousquetaires et à d'Artagnan, en les invitant tous les quatre à déjeuner le lendemain. 

D'Artagnan ne se possédait pas de joie. On le sait, le rêve de toute sa vie avait été d'être mousquetaire. 

Les trois amis étaient fort joyeux. 

«Ma foi! dit d'Artagnan à Athos, tu as eu une triomphante idée, et, comme tu l'as dit, nous y avons acquis de la gloire, et nous avons pu lier une conversation de la plus haute importance. 

\speak  Que nous pourrons reprendre maintenant, sans que personne nous soupçonne; car, avec l'aide de Dieu, nous allons passer désormais pour des cardinalistes.» 

Le même soir, d'Artagnan alla présenter ses hommages à M. des Essarts, et lui faire part de l'avancement qu'il avait obtenu. 

M. des Essarts, qui aimait beaucoup d'Artagnan, lui fit alors ses offres de service: ce changement de corps amenant des dépenses d'équipement. 

D'Artagnan refusa; mais, trouvant l'occasion bonne, il le pria de faire estimer le diamant qu'il lui remit, et dont il désirait faire de l'argent. 

Le lendemain à huit heures du matin, le valet de M. des Essarts entra chez d'Artagnan, et lui remit un sac d'or contenant sept mille livres. 

C'était le prix du diamant de la reine. 