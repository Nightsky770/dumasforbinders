%!TeX root=../musketeersfr.tex 

\chapter{La Femme D'Athos} 
	\lettrine[ante=«]{I}{l} reste maintenant à savoir des nouvelles d'Athos, dit d'Artagnan au fringant Aramis, quand il l'eut mis au courant de ce qui s'était passé dans la capitale depuis leur départ, et qu'un excellent dîner leur eut fait oublier à l'un sa thèse, à l'autre sa fatigue. 

\speak  Croyez-vous donc qu'il lui soit arrivé malheur? demanda Aramis. Athos est si froid, si brave et manie si habilement son épée. 

\speak  Oui, sans doute, et personne ne reconnaît mieux que moi le courage et l'adresse d'Athos, mais j'aime mieux sur mon épée le choc des lances que celui des bâtons, je crains qu'Athos n'ait été étrillé par de la valetaille, les valets sont gens qui frappent fort et ne finissent pas tôt. Voilà pourquoi, je vous l'avoue, je voudrais repartir le plus tôt possible. 

\speak  Je tâcherai de vous accompagner, dit Aramis, quoique je ne me sente guère en état de monter à cheval. Hier, j'essayai de la discipline que vous voyez sur ce mur et la douleur m'empêcha de continuer ce pieux exercice. 

\speak  C'est qu'aussi, mon cher ami, on n'a jamais vu essayer de guérir un coup d'escopette avec des coups de martinet; mais vous étiez malade, et la maladie rend la tête faible, ce qui fait que je vous excuse. 

\speak  Et quand partez-vous? 

\speak  Demain, au point du jour; reposez-vous de votre mieux cette nuit, et demain, si vous le pouvez, nous partirons ensemble. 

\speak  À demain donc, dit Aramis; car tout de fer que vous êtes, vous devez avoir besoin de repos.» 

Le lendemain, lorsque d'Artagnan entra chez Aramis, il le trouva à sa fenêtre. 

«Que regardez-vous donc là? demanda d'Artagnan. 

\speak  Ma foi! J'admire ces trois magnifiques chevaux que les garçons d'écurie tiennent en bride; c'est un plaisir de prince que de voyager sur de pareilles montures. 

\speak  Eh bien, mon cher Aramis, vous vous donnerez ce plaisir-là, car l'un de ces chevaux est à vous. 

\speak  Ah! bah, et lequel? 

\speak  Celui des trois que vous voudrez: je n'ai pas de préférence. 

\speak  Et le riche caparaçon qui le couvre est à moi aussi? 

\speak  Sans doute. 

\speak  Vous voulez rire, d'Artagnan. 

\speak  Je ne ris plus depuis que vous parlez français. 

\speak  C'est pour moi, ces fontes dorées, cette housse de velours, cette selle chevillée d'argent? 

\speak  À vous-même, comme le cheval qui piaffe est à moi, comme cet autre cheval qui caracole est à Athos. 

\speak  Peste! ce sont trois bêtes superbes. 

\speak  Je suis flatté qu'elles soient de votre goût. 

\speak  C'est donc le roi qui vous a fait ce cadeau-là? 

\speak  À coup sûr, ce n'est point le cardinal, mais ne vous inquiétez pas d'où ils viennent, et songez seulement qu'un des trois est votre propriété. 

\speak  Je prends celui que tient le valet roux. 

\speak  À merveille! 

\speak  Vive Dieu! s'écria Aramis, voilà qui me fait passer le reste de ma douleur; je monterais là-dessus avec trente balles dans le corps. Ah! sur mon âme, les beaux étriers! Holà! Bazin, venez çà, et à l'instant même.» 

Bazin apparut, morne et languissant, sur le seuil de la porte. 

«Fourbissez mon épée, redressez mon feutre, brossez mon manteau, et chargez mes pistolets! dit Aramis. 

\speak  Cette dernière recommandation est inutile, interrompit d'Artagnan: il y a des pistolets chargés dans vos fontes.» 

Bazin soupira. 

«Allons, maître Bazin, tranquillisez-vous, dit d'Artagnan; on gagne le royaume des cieux dans toutes les conditions. 

\speak  Monsieur était déjà si bon théologien! dit Bazin presque larmoyant; il fût devenu évêque et peut-être cardinal. 

\speak  Eh bien, mon pauvre Bazin, voyons, réfléchis un peu; à quoi sert d'être homme d'Église, je te prie? on n'évite pas pour cela d'aller faire la guerre; tu vois bien que le cardinal va faire la première campagne avec le pot en tête et la pertuisane au poing; et M. de Nogaret de La Valette, qu'en dis-tu? il est cardinal aussi, demande à son laquais combien de fois il lui a fait de la charpie. 

\speak  Hélas! soupira Bazin, je le sais, monsieur, tout est bouleversé dans le monde aujourd'hui.» 

Pendant ce temps, les deux jeunes gens et le pauvre laquais étaient descendus. 

«Tiens-moi l'étrier, Bazin», dit Aramis. 

Et Aramis s'élança en selle avec sa grâce et sa légèreté ordinaire; mais après quelques voltes et quelques courbettes du noble animal, son cavalier ressentit des douleurs tellement insupportables, qu'il pâlit et chancela. D'Artagnan qui, dans la prévision de cet accident, ne l'avait pas perdu des yeux, s'élança vers lui, le retint dans ses bras et le conduisit à sa chambre. 

«C'est bien, mon cher Aramis, soignez-vous, dit-il, j'irai seul à la recherche d'Athos. 

\speak  Vous êtes un homme d'airain, lui dit Aramis. 

\speak  Non, j'ai du bonheur, voilà tout, mais comment allez-vous vivre en m'attendant? plus de thèse, plus de glose sur les doigts et les bénédictions, hein?» 

Aramis sourit. 

«Je ferai des vers, dit-il. 

\speak  Oui, des vers parfumés à l'odeur du billet de la suivante de Mme de Chevreuse. Enseignez donc la prosodie à Bazin, cela le consolera. Quant au cheval, montez-le tous les jours un peu, et cela vous habituera aux manoeuvres. 

\speak  Oh! pour cela, soyez tranquille, dit Aramis, vous me retrouverez prêt à vous suivre.» 

Ils se dirent adieu et, dix minutes après, d'Artagnan, après avoir recommandé son ami à Bazin et à l'hôtesse, trottait dans la direction d'Amiens. 

Comment allait-il retrouver Athos, et même le retrouverait-il? 

La position dans laquelle il l'avait laissé était critique; il pouvait bien avoir succombé. Cette idée, en assombrissant son front, lui arracha quelques soupirs et lui fit formuler tout bas quelques serments de vengeance. De tous ses amis, Athos était le plus âgé, et partant le moins rapproché en apparence de ses goûts et de ses sympathies. 

Cependant il avait pour ce gentilhomme une préférence marquée. L'air noble et distingué d'Athos, ces éclairs de grandeur qui jaillissaient de temps en temps de l'ombre où il se tenait volontairement enfermé, cette inaltérable égalité d'humeur qui en faisait le plus facile compagnon de la terre, cette gaieté forcée et mordante, cette bravoure qu'on eût appelée aveugle si elle n'eût été le résultat du plus rare sang-froid, tant de qualités attiraient plus que l'estime, plus que l'amitié de d'Artagnan, elles attiraient son admiration. 

En effet, considéré même auprès de M. de Tréville, l'élégant et noble courtisan, Athos, dans ses jours de belle humeur, pouvait soutenir avantageusement la comparaison; il était de taille moyenne, mais cette taille était si admirablement prise et si bien proportionnée, que, plus d'une fois, dans ses luttes avec Porthos, il avait fait plier le géant dont la force physique était devenue proverbiale parmi les mousquetaires; sa tête, aux yeux perçants, au nez droit, au menton dessiné comme celui de Brutus, avait un caractère indéfinissable de grandeur et de grâce; ses mains, dont il ne prenait aucun soin, faisaient le désespoir d'Aramis, qui cultivait les siennes à grand renfort de pâte d'amandes et d'huile parfumée; le son de sa voix était pénétrant et mélodieux tout à la fois, et puis, ce qu'il y avait d'indéfinissable dans Athos, qui se faisait toujours obscur et petit, c'était cette science délicate du monde et des usages de la plus brillante société, cette habitude de bonne maison qui perçait comme à son insu dans ses moindres actions. 

S'agissait-il d'un repas, Athos l'ordonnait mieux qu'aucun homme du monde, plaçant chaque convive à la place et au rang que lui avaient faits ses ancêtres ou qu'il s'était faits lui-même. S'agissait-il de science héraldique, Athos connaissait toutes les familles nobles du royaume, leur généalogie, leurs alliances, leurs armes et l'origine de leurs armes. L'étiquette n'avait pas de minuties qui lui fussent étrangères, il savait quels étaient les droits des grands propriétaires, il connaissait à fond la vénerie et la fauconnerie, et un jour il avait, en causant de ce grand art, étonné le roi Louis XIII lui-même, qui cependant y était passé maître. 

Comme tous les grands seigneurs de cette époque, il montait à cheval et faisait des armes dans la perfection. Il y a plus: son éducation avait été si peu négligée, même sous le rapport des études scolastiques, si rares à cette époque chez les gentilshommes, qu'il souriait aux bribes de latin que détachait Aramis, et qu'avait l'air de comprendre Porthos; deux ou trois fois même, au grand étonnement de ses amis, il lui était arrivé, lorsque Aramis laissait échapper quelque erreur de rudiment, de remettre un verbe à son temps et un nom à son cas. En outre, sa probité était inattaquable, dans ce siècle où les hommes de guerre transigeaient si facilement avec leur religion et leur conscience, les amants avec la délicatesse rigoureuse de nos jours, et les pauvres avec le septième commandement de Dieu. C'était donc un homme fort extraordinaire qu'Athos. 

Et cependant, on voyait cette nature si distinguée, cette créature si belle, cette essence si fine, tourner insensiblement vers la vie matérielle, comme les vieillards tournent vers l'imbécillité physique et morale. Athos, dans ses heures de privation, et ces heures étaient fréquentes, s'éteignait dans toute sa partie lumineuse, et son côté brillant disparaissait comme dans une profonde nuit. 

Alors, le demi-dieu évanoui, il restait à peine un homme. La tête basse, l'œil terne, la parole lourde et pénible, Athos regardait pendant de longues heures soit sa bouteille et son verre, soit Grimaud, qui, habitué à lui obéir par signes, lisait dans le regard atone de son maître jusqu'à son moindre désir, qu'il satisfaisait aussitôt. La réunion des quatre amis avait-elle lieu dans un de ces moments-là, un mot, échappé avec un violent effort, était tout le contingent qu'Athos fournissait à la conversation. En échange, Athos à lui seul buvait comme quatre, et cela sans qu'il y parût autrement que par un froncement de sourcil plus indiqué et par une tristesse plus profonde. 

D'Artagnan, dont nous connaissons l'esprit investigateur et pénétrant, n'avait, quelque intérêt qu'il eût à satisfaire sa curiosité sur ce sujet, pu encore assigner aucune cause à ce marasme, ni en noter les occurrences. Jamais Athos ne recevait de lettres, jamais Athos ne faisait aucune démarche qui ne fût connue de tous ses amis. 

On ne pouvait dire que ce fût le vin qui lui donnât cette tristesse, car au contraire il ne buvait que pour combattre cette tristesse, que ce remède, comme nous l'avons dit, rendait plus sombre encore. On ne pouvait attribuer cet excès d'humeur noire au jeu, car, au contraire de Porthos, qui accompagnait de ses chants ou de ses jurons toutes les variations de la chance, Athos, lorsqu'il avait gagné, demeurait aussi impassible que lorsqu'il avait perdu. On l'avait vu, au cercle des mousquetaires, gagner un soir trois mille pistoles, les perdre jusqu'au ceinturon brodé d'or des jours de gala; regagner tout cela, plus cent louis, sans que son beau sourcil noir eût haussé ou baissé d'une demi-ligne, sans que ses mains eussent perdu leur nuance nacrée, sans que sa conversation, qui était agréable ce soir-là, eût cessé d'être calme et agréable. 

Ce n'était pas non plus, comme chez nos voisins les Anglais, une influence atmosphérique qui assombrissait son visage, car cette tristesse devenait plus intense en général vers les beaux jours de l'année; juin et juillet étaient les mois terribles d'Athos. 

Pour le présent, il n'avait pas de chagrin, il haussait les épaules quand on lui parlait de l'avenir; son secret était donc dans le passé, comme on l'avait dit vaguement à d'Artagnan. 

Cette teinte mystérieuse répandue sur toute sa personne rendait encore plus intéressant l'homme dont jamais les yeux ni la bouche, dans l'ivresse la plus complète, n'avaient rien révélé, quelle que fût l'adresse des questions dirigées contre lui. 

«Eh bien, pensait d'Artagnan, le pauvre Athos est peut-être mort à cette heure, et mort par ma faute, car c'est moi qui l'ai entraîné dans cette affaire, dont il ignorait l'origine, dont il ignorera le résultat et dont il ne devait tirer aucun profit. 

\speak  Sans compter, monsieur, répondait Planchet, que nous lui devons probablement la vie. Vous rappelez-vous comme il a crié: “Au large, d'Artagnan! je suis pris.” Et après avoir déchargé ses deux pistolets, quel bruit terrible il faisait avec son épée! On eût dit vingt hommes, ou plutôt vingt diables enragés!» 

Et ces mots redoublaient l'ardeur de d'Artagnan, qui excitait son cheval, lequel n'ayant pas besoin d'être excité emportait son cavalier au galop. 

Vers onze heures du matin, on aperçut Amiens; à onze heures et demie, on était à la porte de l'auberge maudite. 

D'Artagnan avait souvent médité contre l'hôte perfide une de ces bonnes vengeances qui consolent, rien qu'en espérance. Il entra donc dans l'hôtellerie, le feutre sur les yeux, la main gauche sur le pommeau de l'épée et faisant siffler sa cravache de la main droite. 

«Me reconnaissez-vous? dit-il à l'hôte, qui s'avançait pour le saluer. 

\speak  Je n'ai pas cet honneur, Monseigneur, répondit celui-ci les yeux encore éblouis du brillant équipage avec lequel d'Artagnan se présentait. 

\speak  Ah! vous ne me connaissez pas! 

\speak  Non, Monseigneur. 

\speak  Eh bien, deux mots vont vous rendre la mémoire. Qu'avez-vous fait de ce gentilhomme à qui vous eûtes l'audace, voici quinze jours passés à peu près, d'intenter une accusation de fausse monnaie?» 

L'hôte pâlit, car d'Artagnan avait pris l'attitude la plus menaçante, et Planchet se modelait sur son maître. 

«Ah! Monseigneur, ne m'en parlez pas, s'écria l'hôte de son ton de voix le plus larmoyant; ah! Seigneur, combien j'ai payé cette faute! Ah! malheureux que je suis! 

\speak  Ce gentilhomme, vous dis-je, qu'est-il devenu? 

\speak  Daignez m'écouter, Monseigneur, et soyez clément. Voyons, asseyez-vous, par grâce!» 

D'Artagnan, muet de colère et d'inquiétude, s'assit, menaçant comme un juge. Planchet s'adossa fièrement à son fauteuil. 

«Voici l'histoire, Monseigneur, reprit l'hôte tout tremblant, car je vous reconnais à cette heure; c'est vous qui êtes parti quand j'eus ce malheureux démêlé avec ce gentilhomme dont vous parlez. 

\speak  Oui, c'est moi; ainsi vous voyez bien que vous n'avez pas de grâce à attendre si vous ne dites pas toute la vérité. 

\speak  Aussi veuillez m'écouter, et vous la saurez tout entière. 

\speak  J'écoute. 

\speak  J'avais été prévenu par les autorités qu'un faux-monnayeur célèbre arriverait à mon auberge avec plusieurs de ses compagnons, tous déguisés sous le costume de gardes ou de mousquetaires. Vos chevaux, vos laquais, votre figure, Messeigneurs, tout m'avait été dépeint. 

\speak  Après, après? dit d'Artagnan, qui reconnut bien vite d'où venait le signalement si exactement donné. 

\speak  Je pris donc, d'après les ordres de l'autorité, qui m'envoya un renfort de six hommes, telles mesures que je crus urgentes afin de m'assurer de la personne des prétendus faux-monnayeurs. 

\speak  Encore! dit d'Artagnan, à qui ce mot de faux-monnayeur échauffait terriblement les oreilles. 

\speak  Pardonnez-moi, Monseigneur, de dire de telles choses, mais elles sont justement mon excuse. L'autorité m'avait fait peur, et vous savez qu'un aubergiste doit ménager l'autorité. 

\speak  Mais encore une fois, ce gentilhomme, où est-il? qu'est-il devenu? Est-il mort? est-il vivant? 

\speak  Patience, Monseigneur, nous y voici. Il arriva donc ce que vous savez, et dont votre départ précipité, ajouta l'hôte avec une finesse qui n'échappa point à d'Artagnan, semblait autoriser l'issue. Ce gentilhomme votre ami se défendit en désespéré. Son valet, qui, par un malheur imprévu, avait cherché querelle aux gens de l'autorité, déguisés en garçons d'écurie\dots 

\speak  Ah! misérable! s'écria d'Artagnan, vous étiez tous d'accord, et je ne sais à quoi tient que je ne vous extermine tous! 

\speak  Hélas! non, Monseigneur, nous n'étions pas tous d'accord, et vous l'allez bien voir. Monsieur votre ami (pardon de ne point l'appeler par le nom honorable qu'il porte sans doute, mais nous ignorons ce nom), monsieur votre ami, après avoir mis hors de combat deux hommes de ses deux coups de pistolet, battit en retraite en se défendant avec son épée dont il estropia encore un de mes hommes, et d'un coup du plat de laquelle il m'étourdit. 

\speak  Mais, bourreau, finiras-tu? dit d'Artagnan. Athos, que devient Athos? 

\speak  En battant en retraite, comme j'ai dit à Monseigneur, il trouva derrière lui l'escalier de la cave, et comme la porte était ouverte, il tira la clef à lui et se barricada en dedans. Comme on était sûr de le retrouver là, on le laissa libre. 

\speak  Oui, dit d'Artagnan, on ne tenait pas tout à fait à le tuer, on ne cherchait qu'à l'emprisonner. 

\speak  Juste Dieu! à l'emprisonner, Monseigneur? il s'emprisonna bien lui-même, je vous le jure. D'abord il avait fait de rude besogne, un homme était tué sur le coup et deux autres étaient blessés grièvement. Le mort et les deux blessés furent emportés par leurs camarades, et jamais je n'ai plus entendu parler ni des uns, ni des autres. Moi-même, quand je repris mes sens, j'allai trouver M. le gouverneur, auquel je racontai tout ce qui s'était passé, et auquel je demandai ce que je devais faire du prisonnier. Mais M. le gouverneur eut l'air de tomber des nues; il me dit qu'il ignorait complètement ce que je voulais dire, que les ordres qui m'étaient parvenus n'émanaient pas de lui et que si j'avais le malheur de dire à qui que ce fût qu'il était pour quelque chose dans toute cette échauffourée, il me ferait pendre. Il paraît que je m'étais trompé, monsieur, que j'avais arrêté l'un pour l'autre, et que celui qu'on devait arrêter était sauvé. 

\speak  Mais Athos? s'écria d'Artagnan, dont l'impatience se doublait de l'abandon où l'autorité laissait la chose; Athos, qu'est-il devenu? 

\speak  Comme j'avais hâte de réparer mes torts envers le prisonnier, reprit l'aubergiste, je m'acheminai vers la cave afin de lui rendre sa liberté. Ah! monsieur, ce n'était plus un homme, c'était un diable. À cette proposition de liberté, il déclara que c'était un piège qu'on lui tendait et qu'avant de sortir il entendait imposer ses conditions. Je lui dis bien humblement, car je ne me dissimulais pas la mauvaise position où je m'étais mis en portant la main sur un mousquetaire de Sa Majesté, je lui dis que j'étais prêt à me soumettre à ses conditions. 

«--- D'abord, dit-il, je veux qu'on me rende mon valet tout armé.» 

«On s'empressa d'obéir à cet ordre; car vous comprenez bien, monsieur, que nous étions disposés à faire tout ce que voudrait votre ami. M. Grimaud (il a dit ce nom, celui-là, quoiqu'il ne parle pas beaucoup), M. Grimaud fut donc descendu à la cave, tout blessé qu'il était; alors, son maître l'ayant reçu, rebarricada la porte et nous ordonna de rester dans notre boutique. 

\speak  Mais enfin, s'écria d'Artagnan, où est-il? où est Athos? 

\speak  Dans la cave, monsieur. 

\speak  Comment, malheureux, vous le retenez dans la cave depuis ce temps-là? 

\speak  Bonté divine! Non, monsieur. Nous, le retenir dans la cave! vous ne savez donc pas ce qu'il y fait, dans la cave! Ah! si vous pouviez l'en faire sortir, monsieur, je vous en serais reconnaissant toute ma vie, vous adorerais comme mon patron. 

\speak  Alors il est là, je le retrouverai là? 

\speak  Sans doute, monsieur, il s'est obstiné à y rester. Tous les jours, on lui passe par le soupirail du pain au bout d'une fourche, et de la viande quand il en demande; mais, hélas! ce n'est pas de pain et de viande qu'il fait la plus grande consommation. Une fois, j'ai essayé de descendre avec deux de mes garçons, mais il est entré dans une terrible fureur. J'ai entendu le bruit de ses pistolets qu'il armait et de son mousqueton qu'armait son domestique. Puis, comme nous leur demandions quelles étaient leurs intentions, le maître a répondu qu'ils avaient quarante coups à tirer lui et son laquais, et qu'ils les tireraient jusqu'au dernier plutôt que de permettre qu'un seul de nous mît le pied dans la cave. Alors, monsieur, j'ai été me plaindre au gouverneur, lequel m'a répondu que je n'avais que ce que je méritais, et que cela m'apprendrait à insulter les honorables seigneurs qui prenaient gîte chez moi. 

\speak  De sorte que, depuis ce temps?\dots reprit d'Artagnan ne pouvant s'empêcher de rire de la figure piteuse de son hôte. 

\speak  De sorte que, depuis ce temps, monsieur, continua celui-ci, nous menons la vie la plus triste qui se puisse voir; car, monsieur, il faut que vous sachiez que toutes nos provisions sont dans la cave; il y a notre vin en bouteilles et notre vin en pièce, la bière, l'huile et les épices, le lard et les saucissons; et comme il nous est défendu d'y descendre, nous sommes forcés de refuser le boire et le manger aux voyageurs qui nous arrivent, de sorte que tous les jours notre hôtellerie se perd. Encore une semaine avec votre ami dans ma cave, et nous sommes ruinés. 

\speak  Et ce sera justice, drôle. Ne voyait-on pas bien, à notre mine, que nous étions gens de qualité et non faussaires, dites? 

\speak  Oui, monsieur, oui, vous avez raison, dit l'hôte. Mais tenez, tenez, le voilà qui s'emporte. 

\speak  Sans doute qu'on l'aura troublé, dit d'Artagnan. 

\speak  Mais il faut bien qu'on le trouble, s'écria l'hôte; il vient de nous arriver deux gentilshommes anglais. 

\speak  Eh bien? 

\speak  Eh bien, les Anglais aiment le bon vin, comme vous savez, monsieur; ceux-ci ont demandé du meilleur. Ma femme alors aura sollicité de M. Athos la permission d'entrer pour satisfaire ces messieurs; et il aura refusé comme de coutume. Ah! bonté divine! voilà le sabbat qui redouble!» 

D'Artagnan, en effet, entendit mener un grand bruit du côté de la cave; il se leva et, précédé de l'hôte qui se tordait les mains, et suivi de Planchet qui tenait son mousqueton tout armé, il s'approcha du lieu de la scène. 

Les deux gentilshommes étaient exaspérés, ils avaient fait une longue course et mouraient de faim et de soif. 

«Mais c'est une tyrannie, s'écriaient-ils en très bon français, quoique avec un accent étranger, que ce maître fou ne veuille pas laisser à ces bonnes gens l'usage de leur vin. Ça, nous allons enfoncer la porte, et s'il est trop enragé, eh bien! nous le tuerons. 

\speak  Tout beau, messieurs! dit d'Artagnan en tirant ses pistolets de sa ceinture; vous ne tuerez personne, s'il vous plaît. 

\speak  Bon, bon, disait derrière la porte la voix calme d'Athos, qu'on les laisse un peu entrer, ces mangeurs de petits enfants, et nous allons voir.» 

Tout braves qu'ils paraissaient être, les deux gentilshommes anglais se regardèrent en hésitant; on eût dit qu'il y avait dans cette cave un de ces ogres faméliques, gigantesques héros des légendes populaires, et dont nul ne force impunément la caverne. 

Il y eut un moment de silence; mais enfin les deux Anglais eurent honte de reculer, et le plus hargneux des deux descendit les cinq ou six marches dont se composait l'escalier et donna dans la porte un coup de pied à fendre une muraille. 

«Planchet, dit d'Artagnan en armant ses pistolets, je me charge de celui qui est en haut, charge-toi de celui qui est en bas. Ah! messieurs! vous voulez de la bataille! eh bien! on va vous en donner! 

\speak  Mon Dieu, s'écria la voix creuse d'Athos, j'entends d'Artagnan, ce me semble. 

\speak  En effet, dit d'Artagnan en haussant la voix à son tour, c'est moi-même, mon ami. 

\speak  Ah! bon! alors, dit Athos, nous allons les travailler, ces enfonceurs de portes.» 

Les gentilshommes avaient mis l'épée à la main, mais ils se trouvaient pris entre deux feux; ils hésitèrent un instant encore; mais, comme la première fois, l'orgueil l'emporta, et un second coup de pied fit craquer la porte dans toute sa hauteur. 

«Range-toi, d'Artagnan, range-toi, cria Athos, range-toi, je vais tirer. 

\speak  Messieurs, dit d'Artagnan, que la réflexion n'abandonnait jamais, messieurs, songez-y! De la patience, Athos. Vous vous engagez là dans une mauvaise affaire, et vous allez être criblés. Voici mon valet et moi qui vous lâcherons trois coups de feu, autant vous arriveront de la cave; puis nous aurons encore nos épées, dont, je vous assure, mon ami et moi nous jouons passablement. Laissez-moi faire vos affaires et les miennes. Tout à l'heure vous aurez à boire, je vous en donne ma parole. 

\speak  S'il en reste», grogna la voix railleuse d'Athos. 

L'hôtelier sentit une sueur froide couler le long de son échine. 

«Comment, s'il en reste! murmura-t-il. 

\speak  Que diable! il en restera, reprit d'Artagnan; soyez donc tranquille, à eux deux ils n'auront pas bu toute la cave. Messieurs, remettez vos épées au fourreau. 

\speak  Eh bien, vous, remettez vos pistolets à votre ceinture. 

\speak  Volontiers.» 

Et d'Artagnan donna l'exemple. Puis, se retournant vers Planchet, il lui fit signe de désarmer son mousqueton. 

Les Anglais, convaincus, remirent en grommelant leurs épées au fourreau. On leur raconta l'histoire de l'emprisonnement d'Athos. Et comme ils étaient bons gentilshommes, ils donnèrent tort à l'hôtelier. 

«Maintenant, messieurs, dit d'Artagnan, remontez chez vous, et, dans dix minutes, je vous réponds qu'on vous y portera tout ce que vous pourrez désirer.» 

Les Anglais saluèrent et sortirent. 

«Maintenant que je suis seul, mon cher Athos, dit d'Artagnan, ouvrez-moi la porte, je vous en prie. 

\speak  À l'instant même», dit Athos. 

Alors on entendit un grand bruit de fagots entrechoqués et de poutres gémissantes: c'étaient les contrescarpes et les bastions d'Athos, que l'assiégé démolissait lui-même. 

Un instant après, la porte s'ébranla, et l'on vit paraître la tête pâle d'Athos qui, d'un coup d'œil rapide, explorait les environs. 

D'Artagnan se jeta à son cou et l'embrassa tendrement puis il voulut l'entraîner hors de ce séjour humide, alors il s'aperçut qu'Athos chancelait. 

«Vous êtes blessé? lui dit-il. 

\speak  Moi! pas le moins du monde; je suis ivre mort, voilà tout, et jamais homme n'a mieux fait ce qu'il fallait pour cela. Vive Dieu! mon hôte, il faut que j'en aie bu au moins pour ma part cent cinquante bouteilles. 

\speak  Miséricorde! s'écria l'hôte, si le valet en a bu la moitié du maître seulement, je suis ruiné. 

\speak  Grimaud est un laquais de bonne maison, qui ne se serait pas permis le même ordinaire que moi; il a bu à la pièce seulement; tenez, je crois qu'il a oublié de remettre le fosset. Entendez-vous? cela coule.» 

D'Artagnan partit d'un éclat de rire qui changea le frisson de l'hôte en fièvre chaude. 

En même temps, Grimaud parut à son tour derrière son maître, le mousqueton sur l'épaule, la tête tremblante, comme ces satyres ivres des tableaux de Rubens. Il était arrosé par-devant et par-derrière d'une liqueur grasse que l'hôte reconnut pour être sa meilleure huile d'olive. 

Le cortège traversa la grande salle et alla s'installer dans la meilleure chambre de l'auberge, que d'Artagnan occupa d'autorité. 

Pendant ce temps, l'hôte et sa femme se précipitèrent avec des lampes dans la cave, qui leur avait été si longtemps interdite et où un affreux spectacle les attendait. 

Au-delà des fortifications auxquelles Athos avait fait brèche pour sortir et qui se composaient de fagots, de planches et de futailles vides entassées selon toutes les règles de l'art stratégique, on voyait çà et là, nageant dans les mares d'huile et de vin, les ossements de tous les jambons mangés, tandis qu'un amas de bouteilles cassées jonchait tout l'angle gauche de la cave et qu'un tonneau, dont le robinet était resté ouvert, perdait par cette ouverture les dernières gouttes de son sang. L'image de la dévastation et de la mort, comme dit le poète de l'Antiquité, régnait là comme sur un champ de bataille. 

Sur cinquante saucissons, pendus aux solives, dix restaient à peine. 

Alors les hurlements de l'hôte et de l'hôtesse percèrent la voûte de la cave, d'Artagnan lui-même en fut ému. Athos ne tourna pas même la tête. 

Mais à la douleur succéda la rage. L'hôte s'arma d'une broche et, dans son désespoir, s'élança dans la chambre où les deux amis s'étaient retirés. 

«Du vin! dit Athos en apercevant l'hôte. 

\speak  Du vin! s'écria l'hôte stupéfait, du vin! mais vous m'en avez bu pour plus de cent pistoles; mais je suis un homme ruiné, perdu, anéanti! 

\speak  Bah! dit Athos, nous sommes constamment restés sur notre soif. 

\speak  Si vous vous étiez contentés de boire, encore; mais vous avez cassé toutes les bouteilles. 

\speak  Vous m'avez poussé sur un tas qui a dégringolé. C'est votre faute. 

\speak  Toute mon huile est perdue! 

\speak  L'huile est un baume souverain pour les blessures, et il fallait bien que ce pauvre Grimaud pansât celles que vous lui avez faites. 

\speak  Tous mes saucissons rongés! 

\speak  Il y a énormément de rats dans cette cave. 

\speak  Vous allez me payer tout cela, cria l'hôte exaspéré. 

\speak  Triple drôle!» dit Athos en se soulevant. Mais il retomba aussitôt; il venait de donner la mesure de ses forces. D'Artagnan vint à son secours en levant sa cravache. 

L'hôte recula d'un pas et se mit à fondre en larmes. 

«Cela vous apprendra, dit d'Artagnan, à traiter d'une façon plus courtoise les hôtes que Dieu vous envoie. 

\speak  Dieu\dots, dites le diable! 

\speak  Mon cher ami, dit d'Artagnan, si vous nous rompez encore les oreilles, nous allons nous renfermer tous les quatre dans votre cave, et nous verrons si véritablement le dégât est aussi grand que vous le dites. 

\speak  Eh bien, oui, messieurs, dit l'hôte, j'ai tort, je l'avoue; mais à tout péché miséricorde; vous êtes des seigneurs et je suis un pauvre aubergiste, vous aurez pitié de moi. 

\speak  Ah! si tu parles comme cela, dit Athos, tu vas me fendre le cœur, et les larmes vont couler de mes yeux comme le vin coulait de tes futailles. On n'est pas si diable qu'on en a l'air. Voyons, viens ici et causons.» 

L'hôte s'approcha avec inquiétude. 

«Viens, te dis-je, et n'aie pas peur, continua Athos. Au moment où j'allais te payer, j'avais posé ma bourse sur la table. 

\speak  Oui, Monseigneur. 

\speak  Cette bourse contenait soixante pistoles, où est-elle? 

\speak  Déposée au greffe, Monseigneur: on avait dit que c'était de la fausse monnaie. 

\speak  Eh bien, fais-toi rendre ma bourse, et garde les soixante pistoles. 

\speak  Mais Monseigneur sait bien que le greffe ne lâche pas ce qu'il tient. Si c'était de la fausse monnaie, il y aurait encore de l'espoir; mais malheureusement ce sont de bonnes pièces. 

\speak  Arrange-toi avec lui, mon brave homme, cela ne me regarde pas, d'autant plus qu'il ne me reste pas une livre. 

\speak  Voyons, dit d'Artagnan, l'ancien cheval d'Athos, où est-il? 

\speak  À l'écurie. 

\speak  Combien vaut-il? 

\speak  Cinquante pistoles tout au plus. 

\speak  Il en vaut quatre-vingts; prends-le, et que tout soit dit. 

\speak  Comment! tu vends mon cheval, dit Athos, tu vends mon Bajazet? et sur quoi ferai-je la campagne? sur Grimaud? 

\speak  Je t'en amène un autre, dit d'Artagnan. 

\speak  Un autre? 

\speak  Et magnifique! s'écria l'hôte. 

\speak  Alors, s'il y en a un autre plus beau et plus jeune, prends le vieux, et à boire! 

\speak  Duquel? demanda l'hôte tout à fait rasséréné. 

\speak  De celui qui est au fond, près des lattes; il en reste encore vingt-cinq bouteilles, toutes les autres ont été cassées dans ma chute. Montez-en six. 

\speak  Mais c'est un foudre que cet homme! dit l'hôte à part lui; s'il reste seulement quinze jours ici, et qu'il paie ce qu'il boira, je rétablirai mes affaires. 

\speak  Et n'oublie pas, continua d'Artagnan, de monter quatre bouteilles du pareil aux deux seigneurs anglais. 

\speak  Maintenant, dit Athos, en attendant qu'on nous apporte du vin, conte-moi, d'Artagnan, ce que sont devenus les autres; voyons.» 

D'Artagnan lui raconta comment il avait trouvé Porthos dans son lit avec une foulure, et Aramis à une table entre les deux théologiens. Comme il achevait, l'hôte rentra avec les bouteilles demandées et un jambon qui, heureusement pour lui, était resté hors de la cave. 

«C'est bien, dit Athos en remplissant son verre et celui de d'Artagnan, voilà pour Porthos et pour Aramis; mais vous, mon ami, qu'avez-vous et que vous est-il arrivé personnellement? Je vous trouve un air sinistre. 

\speak  Hélas! dit d'Artagnan, c'est que je suis le plus malheureux de nous tous, moi! 

\speak  Toi malheureux, d'Artagnan! dit Athos. Voyons, comment es-tu malheureux? Dis-moi cela. 

\speak  Plus tard, dit d'Artagnan. 

\speak  Plus tard! et pourquoi plus tard? parce que tu crois que je suis ivre, d'Artagnan? Retiens bien ceci: je n'ai jamais les idées plus nettes que dans le vin. Parle donc, je suis tout oreilles.» 

D'Artagnan raconta son aventure avec Mme Bonacieux. 

Athos l'écouta sans sourciller; puis, lorsqu'il eut fini: 

«Misères que tout cela, dit Athos, misères!» 

C'était le mot d'Athos. 

«Vous dites toujours misères! mon cher Athos, dit d'Artagnan; cela vous sied bien mal, à vous qui n'avez jamais aimé.» 

L'œil mort d'Athos s'enflamma soudain, mais ce ne fut qu'un éclair, il redevint terne et vague comme auparavant. 

«C'est vrai, dit-il tranquillement, je n'ai jamais aimé, moi. 

\speak  Vous voyez bien alors, cœur de pierre, dit d'Artagnan, que vous avez tort d'être dur pour nous autres cœurs tendres. 

\speak  cœurs tendres, cœurs percés, dit Athos. 

\speak  Que dites-vous? 

\speak  Je dis que l'amour est une loterie où celui qui gagne, gagne la mort! Vous êtes bien heureux d'avoir perdu, croyez-moi, mon cher d'Artagnan. Et si j'ai un conseil à vous donner, c'est de perdre toujours. 

\speak  Elle avait l'air de si bien m'aimer! 

\speak  Elle en avait l'air. 

\speak  Oh! elle m'aimait. 

\speak  Enfant! il n'y a pas un homme qui n'ait cru comme vous que sa maîtresse l'aimait, et il n'y a pas un homme qui n'ait été trompé par sa maîtresse. 

\speak  Excepté vous, Athos, qui n'en avez jamais eu. 

\speak  C'est vrai, dit Athos après un moment de silence, je n'en ai jamais eu, moi. Buvons! 

\speak  Mais alors, philosophe que vous êtes, dit d'Artagnan, instruisez-moi, soutenez-moi; j'ai besoin de savoir et d'être consolé. 

\speak  Consolé de quoi? 

\speak  De mon malheur. 

\speak  Votre malheur fait rire, dit Athos en haussant les épaules; je serais curieux de savoir ce que vous diriez si je vous racontais une histoire d'amour. 

\speak  Arrivée à vous? 

\speak  Ou à un de mes amis, qu'importe! 

\speak  Dites, Athos, dites. 

\speak  Buvons, nous ferons mieux. 

\speak  Buvez et racontez. 

\speak  Au fait, cela se peut, dit Athos en vidant et remplissant son verre, les deux choses vont à merveille ensemble. 

\speak  J'écoute», dit d'Artagnan. 

Athos se recueillit, et, à mesure qu'il se recueillait, d'Artagnan le voyait pâlir; il en était à cette période de l'ivresse où les buveurs vulgaires tombent et dorment. Lui, il rêvait tout haut sans dormir. Ce somnambulisme de l'ivresse avait quelque chose d'effrayant. 

«Vous le voulez absolument? demanda-t-il. 

\speak  Je vous en prie, dit d'Artagnan. 

\speak  Qu'il soit fait donc comme vous le désirez. Un de mes amis, un de mes amis, entendez-vous bien! pas moi, dit Athos en s'interrompant avec un sourire sombre; un des comtes de ma province, c'est-à-dire du Berry, noble comme un Dandolo ou un Montmorency, devint amoureux à vingt-cinq ans d'une jeune fille de seize, belle comme les amours. À travers la naïveté de son âge perçait un esprit ardent, un esprit non pas de femme, mais de poète; elle ne plaisait pas, elle enivrait; elle vivait dans un petit bourg, près de son frère qui était curé. Tous deux étaient arrivés dans le pays: ils venaient on ne savait d'où; mais en la voyant si belle et en voyant son frère si pieux, on ne songeait pas à leur demander d'où ils venaient. Du reste, on les disait de bonne extraction. Mon ami, qui était le seigneur du pays, aurait pu la séduire ou la prendre de force, à son gré, il était le maître; qui serait venu à l'aide de deux étrangers, de deux inconnus? Malheureusement il était honnête homme, il l'épousa. Le sot, le niais, l'imbécile! 

\speak  Mais pourquoi cela, puisqu'il l'aimait? demanda d'Artagnan. 

\speak  Attendez donc, dit Athos. Il l'emmena dans son château, et en fit la première dame de sa province; et il faut lui rendre justice, elle tenait parfaitement son rang. 

\speak  Eh bien? demanda d'Artagnan. 

\speak  Eh bien, un jour qu'elle était à la chasse avec son mari, continua Athos à voix basse et en parlant fort vite, elle tomba de cheval et s'évanouit; le comte s'élança à son secours, et comme elle étouffait dans ses habits, il les fendit avec son poignard et lui découvrit l'épaule. Devinez ce qu'elle avait sur l'épaule, d'Artagnan? dit Athos avec un grand éclat de rire. 

\speak  Puis-je le savoir? demanda d'Artagnan. 

\speak  Une fleur de lis, dit Athos. Elle était marquée!» 

Et Athos vida d'un seul trait le verre qu'il tenait à la main. 

«Horreur! s'écria d'Artagnan, que me dites-vous là? 

\speak  La vérité. Mon cher, l'ange était un démon. La pauvre fille avait volé. 

\speak  Et que fit le comte? 

\speak  Le comte était un grand seigneur, il avait sur ses terres droit de justice basse et haute: il acheva de déchirer les habits de la comtesse, il lui lia les mains derrière le dos et la pendit à un arbre. 

\speak  Ciel! Athos! un meurtre! s'écria d'Artagnan. 

\speak  Oui, un meurtre, pas davantage, dit Athos pâle comme la mort. Mais on me laisse manquer de vin, ce me semble.» 

Et Athos saisit au goulot la dernière bouteille qui restait, l'approcha de sa bouche et la vida d'un seul trait, comme il eût fait d'un verre ordinaire. 

Puis il laissa tomber sa tête sur ses deux mains; d'Artagnan demeura devant lui, saisi d'épouvante. 

«Cela m'a guéri des femmes belles, poétiques et amoureuses, dit Athos en se relevant et sans songer à continuer l'apologue du comte. Dieu vous en accorde autant! Buvons! 

\speak  Ainsi elle est morte? balbutia d'Artagnan. 

\speak  Parbleu! dit Athos. Mais tendez votre verre. Du jambon, drôle, cria Athos, nous ne pouvons plus boire! 

\speak  Et son frère? ajouta timidement d'Artagnan. 

\speak  Son frère? reprit Athos. 

\speak  Oui, le prêtre? 

\speak  Ah! je m'en informai pour le faire pendre à son tour; mais il avait pris les devants, il avait quitté sa cure depuis la veille. 

\speak  A-t-on su au moins ce que c'était que ce misérable? 

\speak  C'était sans doute le premier amant et le complice de la belle, un digne homme qui avait fait semblant d'être curé peut-être pour marier sa maîtresse et lui assurer un sort. Il aura été écartelé, je l'espère. 

\speak  Oh! mon Dieu! mon Dieu! fit d'Artagnan, tout étourdi de cette horrible aventure. 

\speak  Mangez donc de ce jambon, d'Artagnan, il est exquis, dit Athos en coupant une tranche qu'il mit sur l'assiette du jeune homme. Quel malheur qu'il n'y en ait pas eu seulement quatre comme celui-là dans la cave! j'aurais bu cinquante bouteilles de plus.» 

D'Artagnan ne pouvait plus supporter cette conversation, qui l'eût rendu fou; il laissa tomber sa tête sur ses deux mains et fit semblant de s'endormir. 

«Les jeunes gens ne savent plus boire, dit Athos en le regardant en pitié, et pourtant celui-là est des meilleurs!\dots»