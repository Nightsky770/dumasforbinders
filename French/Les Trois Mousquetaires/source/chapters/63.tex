%!TeX root=../musketeersfr.tex 

\chapter{Une Goutte D'Eau} 

\lettrine{\accentletter[\gravebox]{A}}{} peine Rochefort fut-il sorti, que Mme Bonacieux rentra. Elle trouva Milady le visage riant. 

\zz
«Eh bien, dit la jeune femme, ce que vous craigniez est donc arrivé; ce soir ou demain le cardinal vous envoie prendre? 

\speak  Qui vous a dit cela, mon enfant? demanda Milady. 

\speak  Je l'ai entendu de la bouche même du messager. 

\speak  Venez vous asseoir ici près de moi, dit Milady. 

\speak  Me voici. 

\speak  Attendez que je m'assure si personne ne nous écoute. 

\speak  Pourquoi toutes ces précautions? 

\speak  Vous allez le savoir.» 

Milady se leva et alla à la porte, l'ouvrit, regarda dans le corridor, et revint se rasseoir près de Mme Bonacieux. 

«Alors, dit-elle, il a bien joué son rôle. 

\speak  Qui cela? 

\speak  Celui qui s'est présenté à l'abbesse comme l'envoyé du cardinal. 

\speak  C'était donc un rôle qu'il jouait? 

\speak  Oui, mon enfant. 

\speak  Cet homme n'est donc pas\dots 

\speak  Cet homme, dit Milady en baissant la voix, c'est mon frère. 

\speak  Votre frère! s'écria Mme Bonacieux. 

\speak  Eh bien, il n'y a que vous qui sachiez ce secret, mon enfant; si vous le confiez à qui que ce soit au monde, je serai perdue, et vous aussi peut-être. 

\speak  Oh! mon Dieu! 

\speak  Écoutez, voici ce qui se passe: mon frère, qui venait à mon secours pour m'enlever ici de force, s'il le fallait, a rencontré l'émissaire du cardinal qui venait me chercher; il l'a suivi. Arrivé à un endroit du chemin solitaire et écarté, il a mis l'épée à la main en sommant le messager de lui remettre les papiers dont il était porteur; le messager a voulu se défendre, mon frère l'a tué. 

\speak  Oh! fit Mme Bonacieux en frissonnant. 

\speak  C'était le seul moyen, songez-y. Alors mon frère a résolu de substituer la ruse à la force: il a pris les papiers, il s'est présenté ici comme l'émissaire du cardinal lui-même, et dans une heure ou deux, une voiture doit venir me prendre de la part de Son Éminence. 

\speak  Je comprends; cette voiture, c'est votre frère qui vous l'envoie. 

\speak  Justement; mais ce n'est pas tout: cette lettre que vous avez reçue, et que vous croyez de Mme Chevreuse\dots 

\speak  Eh bien? 

\speak  Elle est fausse. 

\speak  Comment cela? 

\speak  Oui, fausse: c'est un piège pour que vous ne fassiez pas de résistance quand on viendra vous chercher. 

\speak  Mais c'est d'Artagnan qui viendra. 

\speak  Détrompez-vous, d'Artagnan et ses amis sont retenus au siège de La Rochelle. 

\speak  Comment savez-vous cela? 

\speak  Mon frère a rencontré des émissaires du cardinal en habits de mousquetaires. On vous aurait appelée à la porte, vous auriez cru avoir affaire à des amis, on vous enlevait et on vous ramenait à Paris. 

\speak  Oh! mon Dieu! ma tête se perd au milieu de ce chaos d'iniquités. Je sens que si cela durait, continua Mme Bonacieux en portant ses mains à son front, je deviendrais folle! 

\speak  Attendez\dots 

\speak  Quoi? 

\speak  J'entends le pas d'un cheval, c'est celui de mon frère qui repart; je veux lui dire un dernier adieu, venez.» 

Milady ouvrit la fenêtre et fit signe à Mme Bonacieux de l'y rejoindre. La jeune femme y alla. 

Rochefort passait au galop. 

«Adieu, frère», s'écria Milady. 

Le chevalier leva la tête, vit les deux jeunes femmes, et, tout courant, fit à Milady un signe amical de la main. 

«Ce bon Georges!» dit-elle en refermant la fenêtre avec une expression de visage pleine d'affection et de mélancolie. 

Et elle revint s'asseoir à sa place, comme si elle eût été plongée dans des réflexions toutes personnelles. 

«Chère dame! dit Mme Bonacieux, pardon de vous interrompre! mais que me conseillez-vous de faire? mon Dieu! Vous avez plus d'expérience que moi, parlez, je vous écoute. 

\speak  D'abord, dit Milady, il se peut que je me trompe et que d'Artagnan et ses amis viennent véritablement à votre secours. 

\speak  Oh! c'eût été trop beau! s'écria Mme Bonacieux, et tant de bonheur n'est pas fait pour moi! 

\speak  Alors, vous comprenez; ce serait tout simplement une question de temps, une espèce de course à qui arrivera le premier. Si ce sont vos amis qui l'emportent en rapidité, vous êtes sauvée; si ce sont les satellites du cardinal, vous êtes perdue. 

\speak  Oh! oui, oui, perdue sans miséricorde! Que faire donc? que faire? 

\speak  Il y aurait un moyen bien simple, bien naturel\dots 

\speak  Lequel, dites? 

\speak  Ce serait d'attendre, cachée dans les environs, et de s'assurer ainsi quels sont les hommes qui viendront vous demander. 

\speak  Mais où attendre? 

\speak  Oh! ceci n'est point une question: moi-même je m'arrête et je me cache à quelques lieues d'ici en attendant que mon frère vienne me rejoindre; eh bien, je vous emmène avec moi, nous nous cachons et nous attendons ensemble. 

\speak  Mais on ne me laissera pas partir, je suis ici presque prisonnière. 

\speak  Comme on croit que je pars sur un ordre du cardinal, on ne vous croira pas très pressée de me suivre. 

\speak  Eh bien? 

\speak  Eh bien, la voiture est à la porte, vous me dites adieu, vous montez sur le marchepied pour me serrer dans vos bras une dernière fois; le domestique de mon frère qui vient me prendre est prévenu, il fait un signe au postillon, et nous partons au galop. 

\speak  Mais d'Artagnan, d'Artagnan, s'il vient? 

\speak  Ne le saurons-nous pas? 

\speak  Comment? 

\speak  Rien de plus facile. Nous renvoyons à Béthune ce domestique de mon frère, à qui, je vous l'ai dit, nous pouvons nous fier; il prend un déguisement et se loge en face du couvent: si ce sont les émissaires du cardinal qui viennent, il ne bouge pas; si c'est M. d'Artagnan et ses amis, il les amène où nous sommes. 

\speak  Il les connaît donc? 

\speak  Sans doute, n'a-t-il pas vu M. d'Artagnan chez moi! 

\speak  Oh! oui, oui, vous avez raison; ainsi, tout va bien, tout est pour le mieux; mais ne nous éloignons pas d'ici. 

\speak  À sept ou huit lieues tout au plus, nous nous tenons sur la frontière par exemple, et à la première alerte, nous sortons de France. 

\speak  Et d'ici là, que faire? 

\speak  Attendre. 

\speak  Mais s'ils arrivent? 

\speak  La voiture de mon frère arrivera avant eux. 

\speak  Si je suis loin de vous quand on viendra vous prendre; à dîner ou à souper, par exemple? 

\speak  Faites une chose. 

\speak  Laquelle? 

\speak  Dites à votre bonne supérieure que, pour nous quitter le moins possible, vous lui demanderez la permission de partager mon repas. 

\speak  Le permettra-t-elle? 

\speak  Quel inconvénient y a-t-il à cela? 

\speak  Oh! très bien, de cette façon nous ne nous quitterons pas un instant! 

\speak  Eh bien, descendez chez elle pour lui faire votre demande! je me sens la tête lourde, je vais faire un tour au jardin. 

\speak  Allez, et où vous retrouverai-je? 

\speak  Ici dans une heure. 

\speak  Ici dans une heure; oh! vous êtes bonne et je vous remercie. 

\speak  Comment ne m'intéresserais-je pas à vous? Quand vous ne seriez pas belle et charmante, n'êtes-vous pas l'amie d'un de mes meilleurs amis! 

\speak  Cher d'Artagnan, oh! comme il vous remerciera! 

\speak  Je l'espère bien. Allons! tout est convenu, descendons. 

\speak  Vous allez au jardin? 

\speak  Oui. 

\speak  Suivez ce corridor, un petit escalier vous y conduit. 

\speak  À merveille! merci.» 

Et les deux femmes se quittèrent en échangeant un charmant sourire. 

Milady avait dit la vérité, elle avait la tête lourde; car ses projets mal classés s'y heurtaient comme dans un chaos. Elle avait besoin d'être seule pour mettre un peu d'ordre dans ses pensées. Elle voyait vaguement dans l'avenir; mais il lui fallait un peu de silence et de quiétude pour donner à toutes ses idées, encore confuses, une forme distincte, un plan arrêté. 

Ce qu'il y avait de plus pressé, c'était d'enlever Mme Bonacieux, de la mettre en lieu de sûreté, et là, le cas échéant, de s'en faire un otage. Milady commençait à redouter l'issue de ce duel terrible, où ses ennemis mettaient autant de persévérance qu'elle mettait, elle, d'acharnement. 

D'ailleurs elle sentait, comme on sent venir un orage, que cette issue était proche et ne pouvait manquer d'être terrible. 

Le principal pour elle, comme nous l'avons dit, était donc de tenir Mme Bonacieux entre ses mains. Mme Bonacieux, c'était la vie de d'Artagnan; c'était plus que sa vie, c'était celle de la femme qu'il aimait; c'était, en cas de mauvaise fortune, un moyen de traiter et d'obtenir sûrement de bonnes conditions. 

Or, ce point était arrêté: Mme Bonacieux, sans défiance, la suivait; une fois cachée avec elle à Armentières, il était facile de lui faire croire que d'Artagnan n'était pas venu à Béthune. Dans quinze jours au plus, Rochefort serait de retour; pendant ces quinze jours, d'ailleurs, elle aviserait à ce qu'elle aurait à faire pour se venger des quatre amis. Elle ne s'ennuierait pas, Dieu merci, car elle aurait le plus doux passe-temps que les événements pussent accorder à une femme de son caractère: une bonne vengeance à perfectionner. 

Tout en rêvant, elle jetait les yeux autour d'elle et classait dans sa tête la topographie du jardin. Milady était comme un bon général, qui prévoit tout ensemble la victoire et la défaite, et qui est tout près, selon les chances de la bataille, à marcher en avant ou à battre en retraite. 

Au bout d'une heure, elle entendit une douce voix qui l'appelait; c'était celle de Mme Bonacieux. La bonne abbesse avait naturellement consenti à tout, et, pour commencer, elles allaient souper ensemble. 

En arrivant dans la cour, elles entendirent le bruit d'une voiture qui s'arrêtait a la porte. 

«Entendez-vous? dit-elle. 

\speak  Oui, le roulement d'une voiture. 

\speak  C'est celle que mon frère nous envoie. 

\speak  Oh! mon Dieu! 

\speak  Voyons, du courage!» 

On sonna à la porte du couvent, Milady ne s'était pas trompée. 

«Montez dans votre chambre, dit-elle à Mme Bonacieux, vous avez bien quelques bijoux que vous désirez emporter. 

\speak  J'ai ses lettres, dit-elle. 

\speak  Eh bien, allez les chercher et venez me rejoindre chez moi, nous souperons à la hâte, peut-être voyagerons-nous une partie de la nuit, il faut prendre des forces. 

\speak  Grand Dieu! dit Mme Bonacieux en mettant la main sur sa poitrine, le cœur m'étouffe, je ne puis marcher. 

\speak  Du courage, allons, du courage! pensez que dans un quart d'heure vous êtes sauvée, et songez que ce que vous allez faire, c'est pour lui que vous le faites. 

\speak  Oh! oui, tout pour lui. Vous m'avez rendu mon courage par un seul mot; allez, je vous rejoins.» 

Milady monta vivement chez elle, elle y trouva le laquais de Rochefort, et lui donna ses instructions. 

Il devait attendre à la porte; si par hasard les mousquetaires paraissaient, la voiture partait au galop, faisait le tour du couvent, et allait attendre Milady à un petit village qui était situé de l'autre côté du bois. Dans ce cas, Milady traversait le jardin et gagnait le village à pied; nous l'avons dit déjà, Milady connaissait à merveille cette partie de la France. 

Si les mousquetaires ne paraissaient pas, les choses allaient comme il était convenu: Mme Bonacieux montait dans la voiture sous prétexte de lui dire adieu et Milady enlevait Mme Bonacieux. 

Mme Bonacieux entra, et pour lui ôter tout soupçon si elle en avait, Milady répéta devant elle au laquais toute la dernière partie de ses instructions. 

Milady fit quelques questions sur la voiture: c'était une chaise attelée de trois chevaux, conduite par un postillon; le laquais de Rochefort devait la précéder en courrier. 

C'était à tort que Milady craignait que Mme Bonacieux n'eût des soupçons: la pauvre jeune femme était trop pure pour soupçonner dans une autre femme une telle perfidie; d'ailleurs le nom de la comtesse de Winter, qu'elle avait entendu prononcer par l'abbesse, lui était parfaitement inconnu, et elle ignorait même qu'une femme eût eu une part si grande et si fatale aux malheurs de sa vie. 

«Vous le voyez, dit Milady, lorsque le laquais fut sorti, tout est prêt. L'abbesse ne se doute de rien et croit qu'on me vient chercher de la part du cardinal. Cet homme va donner les derniers ordres; prenez la moindre chose, buvez un doigt de vin et partons. 

\speak  Oui, dit machinalement Mme Bonacieux, oui, partons.» 

Milady lui fit signe de s'asseoir devant elle, lui versa un petit verre de vin d'Espagne et lui servit un blanc de poulet. 

«Voyez, lui dit-elle, si tout ne nous seconde pas: voici la nuit qui vient; au point du jour nous serons arrivées dans notre retraite, et nul ne pourra se douter où nous sommes. Voyons, du courage, prenez quelque chose.» 

Mme Bonacieux mangea machinalement quelques bouchées et trempa ses lèvres dans son verre. 

«Allons donc, allons donc, dit Milady portant le sien à ses lèvres, faites comme moi.» 

Mais au moment où elle l'approchait de sa bouche, sa main resta suspendue: elle venait d'entendre sur la route comme le roulement lointain d'un galop qui allait s'approchant; puis, presque en même temps, il lui sembla entendre des hennissements de chevaux. 

Ce bruit la tira de sa joie comme un bruit d'orage réveille au milieu d'un beau rêve; elle pâlit et courut à la fenêtre, tandis que Mme Bonacieux, se levant toute tremblante, s'appuyait sur sa chaise pour ne point tomber. 

On ne voyait rien encore, seulement on entendait le galop qui allait toujours se rapprochant. 

«Oh! mon Dieu, dit Mme Bonacieux, qu'est-ce que ce bruit? 

\speak  Celui de nos amis ou de nos ennemis, dit Milady avec son sang-froid terrible; restez où vous êtes, je vais vous le dire.» 

Mme Bonacieux demeura debout, muette, immobile et pâle comme une statue. 

Le bruit devenait plus fort, les chevaux ne devaient pas être à plus de cent cinquante pas; si on ne les apercevait point encore, c'est que la route faisait un coude. Toutefois, le bruit devenait si distinct qu'on eût pu compter les chevaux par le bruit saccadé de leurs fers. 

Milady regardait de toute la puissance de son attention; il faisait juste assez clair pour qu'elle pût reconnaître ceux qui venaient. 

Tout à coup, au détour du chemin, elle vit reluire des chapeaux galonnés et flotter des plumes; elle compta deux, puis cinq puis huit cavaliers; l'un d'eux précédait tous les autres de deux longueurs de cheval. 

Milady poussa un rugissement étouffé. Dans celui qui tenait la tête elle reconnut d'Artagnan. 

«Oh! mon Dieu! mon Dieu! s'écria Mme Bonacieux, qu'y a-t-il donc? 

\speak  C'est l'uniforme des gardes de M. le cardinal; pas un instant à perdre! s'écria Milady. Fuyons, fuyons! 

\speak  Oui, oui, fuyons», répéta Mme Bonacieux, mais sans pouvoir faire un pas, clouée qu'elle était à sa place par la terreur. 

On entendit les cavaliers qui passaient sous la fenêtre. 

«Venez donc! mais venez donc! s'écriait Milady en essayant de traîner la jeune femme par le bras. Grâce au jardin, nous pouvons fuir encore, j'ai la clef, mais hâtons-nous, dans cinq minutes il serait trop tard.» 

Mme Bonacieux essaya de marcher, fit deux pas et tomba sur ses genoux. 

Milady essaya de la soulever et de l'emporter, mais elle ne put en venir à bout. 

En ce moment on entendit le roulement de la voiture, qui à la vue des mousquetaires partait au galop. Puis, trois ou quatre coups de feu retentirent. 

«Une dernière fois, voulez-vous venir? s'écria Milady. 

\speak  Oh! mon Dieu! mon Dieu! vous voyez bien que les forces me manquent; vous voyez bien que je ne puis marcher: fuyez seule. 

\speak  Fuir seule! vous laisser ici! non, non, jamais», s'écria Milady. 

Tout à coup, un éclair livide jaillit de ses yeux; d'un bond, éperdue, elle courut à la table, versa dans le verre de Mme Bonacieux le contenu d'un chaton de bague qu'elle ouvrit avec une promptitude singulière. 

C'était un grain rougeâtre qui se fondit aussitôt. 

Puis, prenant le verre d'une main ferme: 

«Buvez, dit-elle, ce vin vous donnera des forces, buvez.» 

Et elle approcha le verre des lèvres de la jeune femme qui but machinalement. 

«Ah! ce n'est pas ainsi que je voulais me venger, dit Milady en reposant avec un sourire infernal le verre sur la table, mais, ma foi! on fait ce qu'on peut.» 

Et elle s'élança hors de l'appartement. 

Mme Bonacieux la regarda fuir, sans pouvoir la suivre; elle était comme ces gens qui rêvent qu'on les poursuit et qui essayent vainement de marcher. 

Quelques minutes se passèrent, un bruit affreux retentissait à la porte; à chaque instant Mme Bonacieux s'attendait à voir reparaître Milady, qui ne reparaissait pas. 

Plusieurs fois, de terreur sans doute, la sueur monta froide à son front brûlant. 

Enfin elle entendit le grincement des grilles qu'on ouvrait, le bruit des bottes et des éperons retentit par les escaliers; il se faisait un grand murmure de voix qui allaient se rapprochant, et au milieu desquelles il lui semblait entendre prononcer son nom. 

Tout à coup elle jeta un grand cri de joie et s'élança vers la porte, elle avait reconnu la voix de d'Artagnan. 

«D'Artagnan! d'Artagnan! s'écria-t-elle, est-ce vous? Par ici, par ici. 

\speak  Constance! Constance! répondit le jeune homme, où êtes-vous? mon Dieu!» 

Au même moment, la porte de la cellule céda au choc plutôt qu'elle ne s'ouvrit; plusieurs hommes se précipitèrent dans la chambre; Mme Bonacieux était tombée dans un fauteuil sans pouvoir faire un mouvement. 

D'Artagnan jeta un pistolet encore fumant qu'il tenait à la main, et tomba à genoux devant sa maîtresse, Athos repassa le sien à sa ceinture; Porthos et Aramis, qui tenaient leurs épées nues, les remirent au fourreau. 

«Oh! d'Artagnan! mon bien-aimé d'Artagnan! tu viens donc enfin, tu ne m'avais pas trompée, c'est bien toi! 

\speak  Oui, oui, Constance! réunis! 

\speak  Oh! \textit{elle} avait beau dire que tu ne viendrais pas, j'espérais sourdement; je n'ai pas voulu fuir; oh! comme j'ai bien fait, comme je suis heureuse!» 

À ce mot, \textit{elle}, Athos, qui s'était assis tranquillement, se leva tout à coup. 

«\textit{Elle!} qui, \textit{elle?} demanda d'Artagnan. 

\speak  Mais ma compagne; celle qui, par amitié pour moi, voulait me soustraire à mes persécuteurs; celle qui, vous prenant pour des gardes du cardinal, vient de s'enfuir. 

\speak  Votre compagne, s'écria d'Artagnan, devenant plus pâle que le voile blanc de sa maîtresse, de quelle compagne voulez-vous donc parler? 

\speak  De celle dont la voiture était à la porte, d'une femme qui se dit votre amie, d'Artagnan; d'une femme à qui vous avez tout raconté. 

\speak  Son nom, son nom! s'écria d'Artagnan; mon Dieu! ne savez-vous donc pas son nom? 

\speak  Si fait, on l'a prononcé devant moi, attendez\dots mais c'est étrange\dots oh! mon Dieu! ma tête se trouble, je n'y vois plus. 

\speak  À moi, mes amis, à moi! ses mains sont glacées, s'écria d'Artagnan, elle se trouve mal; grand Dieu! elle perd connaissance!» 

Tandis que Porthos appelait au secours de toute la puissance de sa voix, Aramis courut à la table pour prendre un verre d'eau; mais il s'arrêta en voyant l'horrible altération du visage d'Athos, qui, debout devant la table, les cheveux hérissés, les yeux glacés de stupeur, regardait l'un des verres et semblait en proie au doute le plus horrible. 

«Oh! disait Athos, oh! non, c'est impossible! Dieu ne permettrait pas un pareil crime. 

\speak  De l'eau, de l'eau, criait d'Artagnan, de l'eau! 

«Pauvre femme, pauvre femme!» murmurait Athos d'une voix brisée. 

Mme Bonacieux rouvrit les yeux sous les baisers de d'Artagnan. 

«Elle revient à elle! s'écria le jeune homme. Oh! mon Dieu, mon Dieu! je te remercie! 

\speak  Madame, dit Athos, madame, au nom du Ciel! à qui ce verre vide? 

\speak  À moi, monsieur\dots, répondit la jeune femme d'une voix mourante. 

\speak  Mais qui vous a versé ce vin qui était dans ce verre? 

\speak  \textit{Elle}. 

\speak  Mais, qui donc, \textit{elle?} 

\speak  Ah! je me souviens, dit Mme Bonacieux, la comtesse de Winter\dots» 

Les quatre amis poussèrent un seul et même cri, mais celui d'Athos domina tous les autres. 

En ce moment, le visage de Mme Bonacieux devint livide, une douleur sourde la terrassa, elle tomba haletante dans les bras de Porthos et d'Aramis. 

D'Artagnan saisit les mains d'Athos avec une angoisse difficile à décrire. 

«Et quoi! dit-il, tu crois\dots» 

Sa voix s'éteignit dans un sanglot. 

«Je crois tout, dit Athos en se mordant les lèvres jusqu'au sang. 

\speak  D'Artagnan, d'Artagnan! s'écria Mme Bonacieux, où es-tu? ne me quitte pas, tu vois bien que je vais mourir.» 

D'Artagnan lâcha les mains d'Athos, qu'il tenait encore entre ses mains crispées, et courut à elle. 

Son visage si beau était tout bouleversé, ses yeux vitreux n'avaient déjà plus de regard, un tremblement convulsif agitait son corps, la sueur coulait sur son front. 

«Au nom du Ciel! courez appeler Porthos, Aramis; demandez du secours! 

\speak  Inutile, dit Athos, inutile, au poison qu'elle verse il n'y a pas de contrepoison. 

\speak  Oui, oui, du secours, du secours! murmura Mme Bonacieux; du secours!» 

Puis, rassemblant toutes ses forces, elle prit la tête du jeune homme entre ses deux mains, le regarda un instant comme si toute son âme était passée dans son regard, et, avec un cri sanglotant, elle appuya ses lèvres sur les siennes. 

«Constance! Constance!» s'écria d'Artagnan. 

Un soupir s'échappa de la bouche de Mme Bonacieux, effleurant celle de d'Artagnan; ce soupir, c'était cette âme si chaste et si aimante qui remontait au ciel. 

D'Artagnan ne serrait plus qu'un cadavre entre ses bras. 

Le jeune homme poussa un cri et tomba près de sa maîtresse, aussi pâle et aussi glacé qu'elle. 

Porthos pleura, Aramis montra le poing au ciel, Athos fit le signe de la croix. 

En ce moment un homme parut sur la porte, presque aussi pâle que ceux qui étaient dans la chambre, et regarda tout autour de lui, vit Mme Bonacieux morte et d'Artagnan évanoui. 

Il apparaissait juste à cet instant de stupeur qui suit les grandes catastrophes. 

«Je ne m'étais pas trompé, dit-il, voilà M. d'Artagnan, et vous êtes ses trois amis, MM. Athos, Porthos et Aramis.» 

Ceux dont les noms venaient d'être prononcés regardaient l'étranger avec étonnement, il leur semblait à tous trois le reconnaître. 

«Messieurs, reprit le nouveau venu, vous êtes comme moi à la recherche d'une femme qui, ajouta-t-il avec un sourire terrible, a dû passer par ici, car j'y vois un cadavre!» 

Les trois amis restèrent muets; seulement la voix comme le visage leur rappelait un homme qu'ils avaient déjà vu; cependant, ils ne pouvaient se souvenir dans quelles circonstances. 

«Messieurs, continua l'étranger, puisque vous ne voulez pas reconnaître un homme qui probablement vous doit la vie deux fois, il faut bien que je me nomme; je suis Lord de Winter, le beau-frère de cette femme.» 

Les trois amis jetèrent un cri de surprise. 

Athos se leva et lui tendit la main. 

«Soyez le bienvenu, Milord, dit-il, vous êtes des nôtres. 

\speak  Je suis parti cinq heures après elle de Portsmouth, dit Lord de Winter, je suis arrivé trois heures après elle à Boulogne, je l'ai manquée de vingt minutes à Saint-Omer; enfin, à Lillers, j'ai perdu sa trace. J'allais au hasard, m'informant à tout le monde, quand je vous ai vus passer au galop; j'ai reconnu M. d'Artagnan. Je vous ai appelés, vous ne m'avez pas répondu; j'ai voulu vous suivre, mais mon cheval était trop fatigué pour aller du même train que les vôtres. Et cependant il paraît que malgré la diligence que vous avez faite, vous êtes encore arrivés trop tard! 

\speak  Vous voyez, dit Athos en montrant à Lord de Winter Mme Bonacieux morte et d'Artagnan que Porthos et Aramis essayaient de rappeler à la vie. 

\speak  Sont-ils donc morts tous deux? demanda froidement Lord de Winter. 

\speak  Non, heureusement, répondit Athos, M. d'Artagnan n'est qu'évanoui. 

\speak  Ah! tant mieux!» dit Lord de Winter. 

En effet, en ce moment d'Artagnan rouvrit les yeux. 

Il s'arracha des bras de Porthos et d'Aramis et se jeta comme un insensé sur le corps de sa maîtresse. 

Athos se leva, marcha vers son ami d'un pas lent et solennel, l'embrassa tendrement, et, comme il éclatait en sanglots, il lui dit de sa voix si noble et si persuasive: 

«Ami, sois homme: les femmes pleurent les morts, les hommes les vengent! 

\speak  Oh! oui, dit d'Artagnan, oui! si c'est pour la venger, je suis prêt à te suivre!» 

Athos profita de ce moment de force que l'espoir de la vengeance rendait à son malheureux ami pour faire signe à Porthos et à Aramis d'aller chercher la supérieure. 

Les deux amis la rencontrèrent dans le corridor, encore toute troublée et tout éperdue de tant d'événements; elle appela quelques religieuses, qui, contre toutes les habitudes monastiques, se trouvèrent en présence de cinq hommes. 

«Madame, dit Athos en passant le bras de d'Artagnan sous le sien, nous abandonnons à vos soins pieux le corps de cette malheureuse femme. Ce fut un ange sur la terre avant d'être un ange au ciel. Traitez-la comme une de vos sœurs; nous reviendrons un jour prier sur sa tombe.» 

D'Artagnan cacha sa figure dans la poitrine d'Athos et éclata en sanglots. 

«Pleure, dit Athos, pleure, cœur plein d'amour, de jeunesse et de vie! Hélas! je voudrais bien pouvoir pleurer comme toi!» 

Et il entraîna son ami, affectueux comme un père, consolant comme un prêtre, grand comme l'homme qui a beaucoup souffert. 

Tous cinq, suivis de leurs valets, tenant leurs chevaux par la bride, s'avancèrent vers la ville de Béthune, dont on apercevait le faubourg, et ils s'arrêtèrent devant la première auberge qu'ils rencontrèrent. 

«Mais, dit d'Artagnan, ne poursuivons-nous pas cette femme? 

\speak  Plus tard, dit Athos, j'ai des mesures à prendre. 

\speak  Elle nous échappera, reprit le jeune homme, elle nous échappera, Athos, et ce sera ta faute. 

\speak  Je réponds d'elle», dit Athos. 

D'Artagnan avait une telle confiance dans la parole de son ami, qu'il baissa la tête et entra dans l'auberge sans rien répondre. 

Porthos et Aramis se regardaient, ne comprenant rien à l'assurance d'Athos. 

Lord de Winter croyait qu'il parlait ainsi pour engourdir la douleur de d'Artagnan. 

«Maintenant, messieurs, dit Athos lorsqu'il se fut assuré qu'il y avait cinq chambres de libres dans l'hôtel, retirons-nous chacun chez soi; d'Artagnan a besoin d'être seul pour pleurer et vous pour dormir. Je me charge de tout, soyez tranquilles. 

\speak  Il me semble cependant, dit Lord de Winter, que s'il y a quelque mesure à prendre contre la comtesse, cela me regarde: c'est ma belle-sœur. 

\speak  Et moi, dit Athos, c'est ma femme. 

D'Artagnan tressaillit, car il comprit qu'Athos était sûr de sa vengeance, puisqu'il révélait un pareil secret; Porthos et Aramis se regardèrent en pâlissant. Lord de Winter pensa qu'Athos était fou. 

«Retirez-vous donc, dit Athos, et laissez-moi faire. Vous voyez bien qu'en ma qualité de mari cela me regarde. Seulement, d'Artagnan, si vous ne l'avez pas perdu, remettez-moi ce papier qui s'est échappé du chapeau de cet homme et sur lequel est écrit le nom de la ville\dots 

\speak  Ah! dit d'Artagnan, je comprends, ce nom écrit de sa main\dots 

\speak  Tu vois bien, dit Athos, qu'il y a un Dieu dans le ciel!» 
