%!TeX root=../musketeersfr.tex 

\chapter{Le Secret De Milady} 
	
\lettrine{D}{'Artagnan} était sorti de l'hôtel au lieu de monter tout de suite chez Ketty, malgré les instances que lui avait faites la jeune fille, et cela pour deux raisons: la première parce que de cette façon il évitait les reproches, les récriminations, les prières; la seconde, parce qu'il n'était pas fâché de lire un peu dans sa pensée, et, s'il était possible, dans celle de cette femme. 

Tout ce qu'il y avait de plus clair là-dedans, c'est que d'Artagnan aimait Milady comme un fou et qu'elle ne l'aimait pas le moins du monde. Un instant d'Artagnan comprit que ce qu'il aurait de mieux à faire serait de rentrer chez lui et d'écrire à Milady une longue lettre dans laquelle il lui avouerait que lui et de Wardes étaient jusqu'à présent absolument le même, que par conséquent il ne pouvait s'engager, sous peine de suicide, à tuer de Wardes. Mais lui aussi était éperonné d'un féroce désir de vengeance; il voulait posséder à son tour cette femme sous son propre nom; et comme cette vengeance lui paraissait avoir une certaine douceur, il ne voulait point y renoncer. 

Il fit cinq ou six fois le tour de la place Royale, se retournant de dix pas en dix pas pour regarder la lumière de l'appartement de Milady, qu'on apercevait à travers les jalousies; il était évident que cette fois la jeune femme était moins pressée que la première de rentrer dans sa chambre. 

Enfin la lumière disparut. 

Avec cette lueur s'éteignit la dernière irrésolution dans le cœur de d'Artagnan; il se rappela les détails de la première nuit, et, le cœur bondissant, la tête en feu, il rentra dans l'hôtel et se précipita dans la chambre de Ketty. 

La jeune fille, pâle comme la mort, tremblant de tous ses membres, voulut arrêter son amant; mais Milady, l'oreille au guet, avait entendu le bruit qu'avait fait d'Artagnan: elle ouvrit la porte. 

«Venez», dit-elle. 

Tout cela était d'une si incroyable imprudence, d'une si monstrueuse effronterie, qu'à peine si d'Artagnan pouvait croire à ce qu'il voyait et à ce qu'il entendait. Il croyait être entraîné dans quelqu'une de ces intrigues fantastiques comme on en accomplit en rêve. 

Il ne s'élança pas moins vers Milady, cédant à cette attraction que l'aimant exerce sur le fer. La porte se referma derrière eux. 

Ketty s'élança à son tour contre la porte. 

La jalousie, la fureur, l'orgueil offensé, toutes les passions enfin qui se disputent le cœur d'une femme amoureuse la poussaient à une révélation; mais elle était perdue si elle avouait avoir donné les mains à une pareille machination; et, par-dessus tout, d'Artagnan était perdu pour elle. Cette dernière pensée d'amour lui conseilla encore ce dernier sacrifice. 

D'Artagnan, de son côté, était arrivé au comble de tous ses voeux: ce n'était plus un rival qu'on aimait en lui, c'était lui-même qu'on avait l'air d'aimer. Une voix secrète lui disait bien au fond du cœur qu'il n'était qu'un instrument de vengeance que l'on caressait en attendant qu'il donnât la mort, mais l'orgueil, mais l'amour-propre, mais la folie faisaient taire cette voix, étouffaient ce murmure. Puis notre Gascon, avec la dose de confiance que nous lui connaissons, se comparait à de Wardes et se demandait pourquoi, au bout du compte, on ne l'aimerait pas, lui aussi, pour lui-même. 

Il s'abandonna donc tout entier aux sensations du moment. Milady ne fut plus pour lui cette femme aux intentions fatales qui l'avait un instant épouvanté, ce fut une maîtresse ardente et passionnée s'abandonnant tout entière à un amour qu'elle semblait éprouver elle-même. Deux heures à peu près s'écoulèrent ainsi. 

Cependant les transports des deux amants se calmèrent; Milady, qui n'avait point les mêmes motifs que d'Artagnan pour oublier, revint la première à la réalité et demanda au jeune homme si les mesures qui devaient amener le lendemain entre lui et de Wardes une rencontre étaient bien arrêtées d'avance dans son esprit. 

Mais d'Artagnan, dont les idées avaient pris un tout autre cours, s'oublia comme un sot et répondit galamment qu'il était bien tard pour s'occuper de duels à coups d'épée. 

Cette froideur pour les seuls intérêts qui l'occupassent effraya Milady, dont les questions devinrent plus pressantes. 

Alors d'Artagnan, qui n'avait jamais sérieusement pensé à ce duel impossible, voulut détourner la conversation, mais il n'était plus de force. 

Milady le contint dans les limites qu'elle avait tracées d'avance avec son esprit irrésistible et sa volonté de fer. 

D'Artagnan se crut fort spirituel en conseillant à Milady de renoncer, en pardonnant à de Wardes, aux projets furieux qu'elle avait formés. 

Mais aux premiers mots qu'il dit, la jeune femme tressaillit et s'éloigna. 

«Auriez-vous peur, cher d'Artagnan? dit-elle d'une voix aiguë et railleuse qui résonna étrangement dans l'obscurité. 

\speak  Vous ne le pensez pas, chère âme! répondit d'Artagnan; mais enfin, si ce pauvre comte de Wardes était moins coupable que vous ne le pensez? 

\speak  En tout cas dit gravement Milady, il m'a trompée, et du moment où il m'a trompée il a mérité la mort. 

\speak  Il mourra donc, puisque vous le condamnez!» dit d'Artagnan d'un ton si ferme, qu'il parut à Milady l'expression d'un dévouement à toute épreuve. 

Aussitôt elle se rapprocha de lui. 

Nous ne pourrions dire le temps que dura la nuit pour Milady; mais d'Artagnan croyait être près d'elle depuis deux heures à peine lorsque le jour parut aux fentes des jalousies et bientôt envahit la chambre de sa lueur blafarde. 

Alors Milady, voyant que d'Artagnan allait la quitter, lui rappela la promesse qu'il lui avait faite de la venger de de Wardes. 

«Je suis tout prêt, dit d'Artagnan, mais auparavant je voudrais être certain d'une chose. 

\speak  De laquelle? demanda Milady. 

\speak  C'est que vous m'aimez. 

\speak  Je vous en ai donné la preuve, ce me semble. 

\speak  Oui, aussi je suis à vous corps et âme. 

\speak  Merci, mon brave amant! mais de même que je vous ai prouvé mon amour, vous me prouverez le vôtre à votre tour, n'est-ce pas? 

\speak  Certainement. Mais si vous m'aimez comme vous me le dites, reprit d'Artagnan, ne craignez-vous pas un peu pour moi? 

\speak  Que puis-je craindre? 

\speak  Mais enfin, que je sois blessé dangereusement, tué même. 

\speak  Impossible, dit Milady, vous êtes un homme si vaillant et une si fine épée. 

\speak  Vous ne préféreriez donc point, reprit d'Artagnan, un moyen qui vous vengerait de même tout en rendant inutile le combat.» 

Milady regarda son amant en silence: cette lueur blafarde des premiers rayons du jour donnait à ses yeux clairs une expression étrangement funeste. 

«Vraiment, dit-elle, je crois que voilà que vous hésitez maintenant. 

\speak  Non, je n'hésite pas; mais c'est que ce pauvre comte de Wardes me fait vraiment peine depuis que vous ne l'aimez plus, et il me semble qu'un homme doit être si cruellement puni par la perte seule de votre amour, qu'il n'a pas besoin d'autre châtiment. 

\speak  Qui vous dit que je l'aie aimé? demanda Milady. 

\speak  Au moins puis-je croire maintenant sans trop de fatuité que vous en aimez un autre, dit le jeune homme d'un ton caressant, et je vous le répète, je m'intéresse au comte. 

\speak  Vous? demanda Milady. 

\speak  Oui moi. 

\speak  Et pourquoi vous? 

\speak  Parce que seul je sais\dots 

\speak  Quoi? 

\speak  Qu'il est loin d'être ou plutôt d'avoir été aussi coupable envers vous qu'il le paraît. 

\speak  En vérité! dit Milady d'un air inquiet; expliquez-vous, car je ne sais vraiment ce que vous voulez dire.» 

Et elle regardait d'Artagnan, qui la tenait embrassée avec des yeux qui semblaient s'enflammer peu à peu. 

«Oui, je suis galant homme, moi! dit d'Artagnan décidé à en finir; et depuis que votre amour est à moi, que je suis bien sûr de le posséder, car je le possède, n'est-ce pas?\dots 

\speak  Tout entier, continuez. 

\speak  Eh bien, je me sens comme transporté, un aveu me pèse. 

\speak  Un aveu? 

\speak  Si j'eusse douté de votre amour je ne l'eusse pas fait; mais vous m'aimez, ma belle maîtresse? n'est-ce pas, vous m'aimez? 

\speak  Sans doute. 

\speak  Alors si par excès d'amour je me suis rendu coupable envers vous, vous me pardonnerez? 

\speak  Peut-être!» 

D'Artagnan essaya, avec le plus doux sourire qu'il pût prendre, de rapprocher ses lèvres des lèvres de Milady, mais celle-ci l'écarta. 

«Cet aveu, dit-elle en pâlissant, quel est cet aveu? 

\speak  Vous aviez donné rendez-vous à de Wardes, jeudi dernier, dans cette même chambre, n'est-ce pas? 

\speak  Moi, non! cela n'est pas, dit Milady d'un ton de voix si ferme et d'un visage si impassible, que si d'Artagnan n'eût pas eu une certitude si parfaite, il eût douté. 

\speak  Ne mentez pas, mon bel ange, dit d'Artagnan en souriant, ce serait inutile. 

\speak  Comment cela? parlez donc! vous me faites mourir! 

\speak  Oh! rassurez-vous, vous n'êtes point coupable envers moi, et je vous ai déjà pardonné! 

\speak  Après, après? 

\speak  De Wardes ne peut se glorifier de rien. 

\speak  Pourquoi? Vous m'avez dit vous-même que cette bague\dots 

\speak  Cette bague, mon amour, c'est moi qui l'ai. Le comte de Wardes de jeudi et le d'Artagnan d'aujourd'hui sont la même personne.» 

L'imprudent s'attendait à une surprise mêlée de pudeur, à un petit orage qui se résoudrait en larmes; mais il se trompait étrangement, et son erreur ne fut pas longue. 

Pâle et terrible, Milady se redressa, et, repoussant d'Artagnan d'un violent coup dans la poitrine, elle s'élança hors du lit. 

Il faisait alors presque grand jour. 

D'Artagnan la retint par son peignoir de fine toile des Indes pour implorer son pardon; mais elle, d'un mouvement puissant et résolu, elle essaya de fuir. Alors la batiste se déchira en laissant à nu les épaules et sur l'une de ces belles épaules rondes et blanches, d'Artagnan avec un saisissement inexprimable, reconnut la fleur de lis, cette marque indélébile qu'imprime la main infamante du bourreau. 

«Grand Dieu!» s'écria d'Artagnan en lâchant le peignoir. 

Et il demeura muet, immobile et glacé sur le lit. 

Mais Milady se sentait dénoncée par l'effroi même de d'Artagnan. Sans doute il avait tout vu: le jeune homme maintenant savait son secret, secret terrible, que tout le monde ignorait, excepté lui. 

Elle se retourna, non plus comme une femme furieuse mais comme une panthère blessée. 

«Ah! misérable, dit-elle, tu m'as lâchement trahie, et de plus tu as mon secret! Tu mourras!» 

Et elle courut à un coffret de marqueterie posé sur la toilette, l'ouvrit d'une main fiévreuse et tremblante, en tira un petit poignard à manche d'or, à la lame aiguë et mince et revint d'un bond sur d'Artagnan à demi nu. 

Quoique le jeune homme fût brave, on le sait, il fut épouvanté de cette figure bouleversée, de ces pupilles dilatées horriblement, de ces joues pâles et de ces lèvres sanglantes; il recula jusqu'à la ruelle, comme il eût fait à l'approche d'un serpent qui eût rampé vers lui, et son épée se rencontrant sous sa main souillée de sueur, il la tira du fourreau. 

Mais sans s'inquiéter de l'épée, Milady essaya de remonter sur le lit pour le frapper, et elle ne s'arrêta que lorsqu'elle sentit la pointe aiguë sur sa gorge. 

Alors elle essaya de saisir cette épée avec les mains mais d'Artagnan l'écarta toujours de ses étreintes et, la lui présentant tantôt aux yeux, tantôt à la poitrine, il se laissa glisser à bas du lit, cherchant pour faire retraite la porte qui conduisait chez Ketty. 

Milady, pendant ce temps, se ruait sur lui avec d'horribles transports, rugissant d'une façon formidable. 

Cependant cela ressemblait à un duel, aussi d'Artagnan se remettait petit à petit. 

«Bien, belle dame, bien! disait-il, mais, de par Dieu, calmez-vous, ou je vous dessine une seconde fleur de lis sur l'autre épaule. 

\speak  Infâme! infâme!» hurlait Milady. 

Mais d'Artagnan, cherchant toujours la porte, se tenait sur la défensive. 

Au bruit qu'ils faisaient, elle renversant les meubles pour aller à lui, lui s'abritant derrière les meubles pour se garantir d'elle, Ketty ouvrit la porte. D'Artagnan, qui avait sans cesse manoeuvré pour se rapprocher de cette porte, n'en était plus qu'à trois pas. D'un seul élan il s'élança de la chambre de Milady dans celle de la suivante, et, rapide comme l'éclair, il referma la porte, contre laquelle il s'appuya de tout son poids tandis que Ketty poussait les verrous. 

Alors Milady essaya de renverser l'arc-boutant qui l'enfermait dans sa chambre, avec des forces bien au-dessus de celles d'une femme; puis, lorsqu'elle sentit que c'était chose impossible, elle cribla la porte de coups de poignard, dont quelques-uns traversèrent l'épaisseur du bois. 

Chaque coup était accompagné d'une imprécation terrible. 

«Vite, vite, Ketty, dit d'Artagnan à demi-voix lorsque les verrous furent mis, fais-moi sortir de l'hôtel, ou si nous lui laissons le temps de se retourner, elle me fera tuer par les laquais. 

\speak  Mais vous ne pouvez pas sortir ainsi, dit Ketty, vous êtes tout nu. 

\speak  C'est vrai, dit d'Artagnan, qui s'aperçut alors seulement du costume dans lequel il se trouvait, c'est vrai; habille-moi comme tu pourras, mais hâtons-nous; comprends-tu, il y va de la vie et de la mort!» 

Ketty ne comprenait que trop; en un tour de main elle l'affubla d'une robe à fleurs, d'une large coiffe et d'un mantelet; elle lui donna des pantoufles, dans lesquelles il passa ses pieds nus, puis elle l'entraîna par les degrés. Il était temps, Milady avait déjà sonné et réveillé tout l'hôtel. Le portier tira le cordon à la voix de Ketty au moment même où Milady, à demi nue de son côté, criait par la fenêtre: 

«N'ouvrez pas!» 