%!TeX root=../musketeersfr.tex 

\chapter[L'Épaule, Le Baudrier Et Le Mouchoir]{L'Épaule D'Athos, Le Baudrier De Porthos Et Le Mouchoir D'Aramis} 
\chaptermark{L'Épaule, Le Baudrier Et Le Mouchoir}
	
\lettrine{D}{'Artagnan}, furieux, avait traversé l'antichambre en trois bonds et s'élançait sur l'escalier, dont il comptait descendre les degrés quatre à quatre, lorsque, emporté par sa course, il alla donner tête baissée dans un mousquetaire qui sortait de chez M. de Tréville par une porte de dégagement, et, le heurtant du front à l'épaule, lui fit pousser un cri ou plutôt un hurlement. 

«Excusez-moi, dit d'Artagnan, essayant de reprendre sa course, excusez-moi, mais je suis pressé.» 

À peine avait-il descendu le premier escalier, qu'un poignet de fer le saisit par son écharpe et l'arrêta. 

«Vous êtes pressé! s'écria le mousquetaire, pâle comme un linceul; sous ce prétexte, vous me heurtez, vous dites: “Excusez-moi”, et vous croyez que cela suffit? Pas tout à fait, mon jeune homme. Croyez-vous, parce que vous avez entendu M. de Tréville nous parler un peu cavalièrement aujourd'hui, que l'on peut nous traiter comme il nous parle? Détrompez-vous, compagnon, vous n'êtes pas M. de Tréville, vous. 

\speak  Ma foi, répliqua d'Artagnan, qui reconnut Athos, lequel, après le pansement opéré par le docteur, regagnait son appartement, ma foi, je ne l'ai pas fait exprès, j'ai dit: “Excusez-moi.” Il me semble donc que c'est assez. Je vous répète cependant, et cette fois c'est trop peut-être, parole d'honneur! je suis pressé, très pressé. Lâchez-moi donc, je vous prie, et laissez-moi aller où j'ai affaire. 

\speak  Monsieur, dit Athos en le lâchant, vous n'êtes pas poli. On voit que vous venez de loin.» 

D'Artagnan avait déjà enjambé trois ou quatre degrés, mais à la remarque d'Athos il s'arrêta court. 

«Morbleu, monsieur! dit-il, de si loin que je vienne, ce n'est pas vous qui me donnerez une leçon de belles manières, je vous préviens. 

\speak  Peut-être, dit Athos. 

\speak  Ah! si je n'étais pas si pressé, s'écria d'Artagnan, et si je ne courais pas après quelqu'un\dots 

\speak  Monsieur l'homme pressé, vous me trouverez sans courir, moi, entendez-vous? 

\speak  Et où cela, s'il vous plaît? 

\speak  Près des Carmes-Deschaux. 

\speak  À quelle heure? 

\speak  Vers midi. 

\speak  Vers midi, c'est bien, j'y serai. 

\speak  Tâchez de ne pas me faire attendre, car à midi un quart je vous préviens que c'est moi qui courrai après vous et vous couperai les oreilles à la course. 

\speak  Bon! lui cria d'Artagnan; on y sera à midi moins dix minutes.» 

Et il se mit à courir comme si le diable l'emportait, espérant retrouver encore son inconnu, que son pas tranquille ne devait pas avoir conduit bien loin. 

Mais, à la porte de la rue, causait Porthos avec un soldat aux gardes. Entre les deux causeurs, il y avait juste l'espace d'un homme. D'Artagnan crut que cet espace lui suffirait, et il s'élança pour passer comme une flèche entre eux deux. Mais d'Artagnan avait compté sans le vent. Comme il allait passer, le vent s'engouffra dans le long manteau de Porthos, et d'Artagnan vint donner droit dans le manteau. Sans doute, Porthos avait des raisons de ne pas abandonner cette partie essentielle de son vêtement car, au lieu de laisser aller le pan qu'il tenait, il tira à lui, de sorte que d'Artagnan s'enroula dans le velours par un mouvement de rotation qu'explique la résistance de l'obstiné Porthos. 

D'Artagnan, entendant jurer le mousquetaire, voulut sortir de dessous le manteau qui l'aveuglait, et chercha son chemin dans le pli. Il redoutait surtout d'avoir porté atteinte à la fraîcheur du magnifique baudrier que nous connaissons; mais, en ouvrant timidement les yeux, il se trouva le nez collé entre les deux épaules de Porthos c'est-à-dire précisément sur le baudrier. 

Hélas! comme la plupart des choses de ce monde qui n'ont pour elles que l'apparence, le baudrier était d'or par-devant et de simple buffle par-derrière. Porthos, en vrai glorieux qu'il était, ne pouvant avoir un baudrier d'or tout entier, en avait au moins la moitié: on comprenait dès lors la nécessité du rhume et l'urgence du manteau. 

«Vertubleu! cria Porthos faisant tous ses efforts pour se débarrasser de d'Artagnan qui lui grouillait dans le dos, vous êtes donc enragé de vous jeter comme cela sur les gens! 

\speak  Excusez-moi, dit d'Artagnan reparaissant sous l'épaule du géant, mais je suis très pressé, je cours après quelqu'un, et\dots 

\speak  Est-ce que vous oubliez vos yeux quand vous courez, par hasard? demanda Porthos. 

\speak  Non, répondit d'Artagnan piqué, non, et grâce à mes yeux je vois même ce que ne voient pas les autres.» 

Porthos comprit ou ne comprit pas, toujours est-il que, se laissant aller à sa colère: 

«Monsieur, dit-il, vous vous ferez étriller, je vous en préviens, si vous vous frottez ainsi aux mousquetaires. 

\speak  Étriller, monsieur! dit d'Artagnan, le mot est dur. 

\speak  C'est celui qui convient à un homme habitué à regarder en face ses ennemis. 

\speak  Ah! pardieu! je sais bien que vous ne tournez pas le dos aux vôtres, vous.» 

Et le jeune homme, enchanté de son espièglerie, s'éloigna en riant à gorge déployée. 

Porthos écuma de rage et fit un mouvement pour se précipiter sur d'Artagnan. 

«Plus tard, plus tard, lui cria celui-ci, quand vous n'aurez plus votre manteau. 

\speak  À une heure donc, derrière le Luxembourg. 

\speak  Très bien, à une heure», répondit d'Artagnan en tournant l'angle de la rue. 

Mais ni dans la rue qu'il venait de parcourir, ni dans celle qu'il embrassait maintenant du regard, il ne vit personne. Si doucement qu'eût marché l'inconnu, il avait gagné du chemin; peut-être aussi était-il entré dans quelque maison. D'Artagnan s'informa de lui à tous ceux qu'il rencontra, descendit jusqu'au bac, remonta par la rue de Seine et la Croix-Rouge; mais rien, absolument rien. Cependant cette course lui fut profitable en ce sens qu'à mesure que la sueur inondait son front, son cœur se refroidissait. 

Il se mit alors à réfléchir sur les événements qui venaient de se passer; ils étaient nombreux et néfastes: il était onze heures du matin à peine, et déjà la matinée lui avait apporté la disgrâce de M. de Tréville, qui ne pouvait manquer de trouver un peu cavalière la façon dont d'Artagnan l'avait quitté. 

En outre, il avait ramassé deux bons duels avec deux hommes capables de tuer chacun trois d'Artagnan, avec deux mousquetaires enfin, c'est-à-dire avec deux de ces êtres qu'il estimait si fort qu'il les mettait, dans sa pensée et dans son cœur, au-dessus de tous les autres hommes. 

La conjecture était triste. Sûr d'être tué par Athos, on comprend que le jeune homme ne s'inquiétait pas beaucoup de Porthos. Pourtant, comme l'espérance est la dernière chose qui s'éteint dans le cœur de l'homme, il en arriva à espérer qu'il pourrait survivre, avec des blessures terribles, bien entendu, à ces deux duels, et, en cas de survivance, il se fit pour l'avenir les réprimandes suivantes: 

«Quel écervelé je fais, et quel butor je suis! Ce brave et malheureux Athos était blessé juste à l'épaule contre laquelle je m'en vais, moi, donner de la tête comme un bélier. La seule chose qui m'étonne, c'est qu'il ne m'ait pas tué roide; il en avait le droit, et la douleur que je lui ai causée a dû être atroce. Quant à Porthos! Oh! quant à Porthos, ma foi, c'est plus drôle.» 

Et malgré lui le jeune homme se mit à rire, tout en regardant néanmoins si ce rire isolé, et sans cause aux yeux de ceux qui le voyaient rire, n'allait pas blesser quelque passant. 

«Quant à Porthos, c'est plus drôle; mais je n'en suis pas moins un misérable étourdi. Se jette-t-on ainsi sur les gens sans dire gare! non! et va-t-on leur regarder sous le manteau pour y voir ce qui n'y est pas! Il m'eût pardonné bien certainement; il m'eût pardonné si je n'eusse pas été lui parler de ce maudit baudrier, à mots couverts, c'est vrai; oui, couverts joliment! Ah! maudit Gascon que je suis, je ferais de l'esprit dans la poêle à frire. Allons, d'Artagnan mon ami, continua-t-il, se parlant à lui-même avec toute l'aménité qu'il croyait se devoir, si tu en réchappes, ce qui n'est pas probable, il s'agit d'être à l'avenir d'une politesse parfaite. Désormais il faut qu'on t'admire, qu'on te cite comme modèle. Être prévenant et poli, ce n'est pas être lâche. Regardez plutôt Aramis: Aramis, c'est la douceur, c'est la grâce en personne. Eh bien, personne s'est-il jamais avisé de dire qu'Aramis était un lâche? Non, bien certainement, et désormais je veux en tout point me modeler sur lui. Ah! justement le voici.» 

D'Artagnan, tout en marchant et en monologuant, était arrivé à quelques pas de l'hôtel d'Aiguillon, et devant cet hôtel il avait aperçu Aramis causant gaiement avec trois gentilshommes des gardes du roi. De son côté, Aramis aperçut d'Artagnan; mais comme il n'oubliait point que c'était devant ce jeune homme que M. de Tréville s'était si fort emporté le matin, et qu'un témoin des reproches que les mousquetaires avaient reçus ne lui était d'aucune façon agréable, il fit semblant de ne pas le voir. D'Artagnan, tout entier au contraire à ses plans de conciliation et de courtoisie, s'approcha des quatre jeunes gens en leur faisant un grand salut accompagné du plus gracieux sourire. Aramis inclina légèrement la tête, mais ne sourit point. Tous quatre, au reste, interrompirent à l'instant même leur conversation. 

D'Artagnan n'était pas assez niais pour ne point s'apercevoir qu'il était de trop; mais il n'était pas encore assez rompu aux façons du beau monde pour se tirer galamment d'une situation fausse comme l'est, en général, celle d'un homme qui est venu se mêler à des gens qu'il connaît à peine et à une conversation qui ne le regarde pas. Il cherchait donc en lui-même un moyen de faire sa retraite le moins gauchement possible, lorsqu'il remarqua qu'Aramis avait laissé tomber son mouchoir et, par mégarde sans doute, avait mis le pied dessus; le moment lui parut arrivé de réparer son inconvenance: il se baissa, et de l'air le plus gracieux qu'il pût trouver, il tira le mouchoir de dessous le pied du mousquetaire, quelques efforts que celui-ci fît pour le retenir, et lui dit en le lui remettant: 

«Je crois, monsieur que voici un mouchoir que vous seriez fâché de perdre.» 

Le mouchoir était en effet richement brodé et portait une couronne et des armes à l'un de ses coins. Aramis rougit excessivement et arracha plutôt qu'il ne prit le mouchoir des mains du Gascon. 

«Ah! Ah! s'écria un des gardes, diras-tu encore, discret Aramis, que tu es mal avec Mme de Bois-Tracy, quand cette gracieuse dame a l'obligeance de te prêter ses mouchoirs?» 

Aramis lança à d'Artagnan un de ces regards qui font comprendre à un homme qu'il vient de s'acquérir un ennemi mortel; puis, reprenant son air doucereux: 

«Vous vous trompez, messieurs, dit-il, ce mouchoir n'est pas à moi, et je ne sais pourquoi monsieur a eu la fantaisie de me le remettre plutôt qu'à l'un de vous, et la preuve de ce que je dis, c'est que voici le mien dans ma poche.» 

À ces mots, il tira son propre mouchoir, mouchoir fort élégant aussi, et de fine batiste, quoique la batiste fût chère à cette époque, mais mouchoir sans broderie, sans armes et orné d'un seul chiffre, celui de son propriétaire. 

Cette fois, d'Artagnan ne souffla pas mot, il avait reconnu sa bévue; mais les amis d'Aramis ne se laissèrent pas convaincre par ses dénégations, et l'un d'eux, s'adressant au jeune mousquetaire avec un sérieux affecté: 

«Si cela était, dit-il, ainsi que tu le prétends, je serais forcé, mon cher Aramis, de te le redemander; car, comme tu le sais, Bois-Tracy est de mes intimes, et je ne veux pas qu'on fasse trophée des effets de sa femme. 

\speak  Tu demandes cela mal, répondit Aramis, et tout en reconnaissant la justesse de ta réclamation quant au fond, je refuserais à cause de la forme. 

\speak  Le fait est, hasarda timidement d'Artagnan, que je n'ai pas vu sortir le mouchoir de la poche de M. Aramis. Il avait le pied dessus, voilà tout, et j'ai pensé que, puisqu'il avait le pied dessus, le mouchoir était à lui. 

\speak  Et vous vous êtes trompé, mon cher monsieur», répondit froidement Aramis, peu sensible à la réparation. 

Puis, se retournant vers celui des gardes qui s'était déclaré l'ami de Bois-Tracy: 

«D'ailleurs, continua-t-il, je réfléchis, mon cher intime de Bois-Tracy, que je suis son ami non moins tendre que tu peux l'être toi-même; de sorte qu'à la rigueur ce mouchoir peut aussi bien être sorti de ta poche que de la mienne. 

\speak  Non, sur mon honneur! s'écria le garde de Sa Majesté. 

\speak  Tu vas jurer sur ton honneur et moi sur ma parole et alors il y aura évidemment un de nous deux qui mentira. Tiens, faisons mieux, Montaran, prenons-en chacun la moitié. 

\speak  Du mouchoir? 

\speak  Oui. 

\speak  Parfaitement, s'écrièrent les deux autres gardes, le jugement du roi Salomon. Décidément, Aramis, tu es plein de sagesse.» 

Les jeunes gens éclatèrent de rire, et comme on le pense bien, l'affaire n'eut pas d'autre suite. Au bout d'un instant, la conversation cessa, et les trois gardes et le mousquetaire, après s'être cordialement serré la main, tirèrent, les trois gardes de leur côté et Aramis du sien. 

«Voilà le moment de faire ma paix avec ce galant homme», se dit à part lui d'Artagnan, qui s'était tenu un peu à l'écart pendant toute la dernière partie de cette conversation. Et, sur ce bon sentiment, se rapprochant d'Aramis, qui s'éloignait sans faire autrement attention à lui: 

«Monsieur, lui dit-il, vous m'excuserez, je l'espère. 

\speak  Ah! monsieur, interrompit Aramis, permettez-moi de vous faire observer que vous n'avez point agi en cette circonstance comme un galant homme le devait faire. 

\speak  Quoi, monsieur! s'écria d'Artagnan, vous supposez\dots 

\speak  Je suppose, monsieur, que vous n'êtes pas un sot, et que vous savez bien, quoique arrivant de Gascogne, qu'on ne marche pas sans cause sur les mouchoirs de poche. Que diable! Paris n'est point pavé en batiste. 

\speak  Monsieur, vous avez tort de chercher à m'humilier, dit d'Artagnan, chez qui le naturel querelleur commençait à parler plus haut que les résolutions pacifiques. Je suis de Gascogne, c'est vrai, et puisque vous le savez, je n'aurai pas besoin de vous dire que les Gascons sont peu endurants; de sorte que, lorsqu'ils se sont excusés une fois, fût-ce d'une sottise, ils sont convaincus qu'ils ont déjà fait moitié plus qu'ils ne devaient faire. 

\speak  Monsieur, ce que je vous en dis, répondit Aramis, n'est point pour vous chercher une querelle. Dieu merci! je ne suis pas un spadassin, et n'étant mousquetaire que par intérim, je ne me bats que lorsque j'y suis forcé, et toujours avec une grande répugnance; mais cette fois l'affaire est grave, car voici une dame compromise par vous. 

\speak  Par nous, c'est-à-dire, s'écria d'Artagnan. 

\speak  Pourquoi avez-vous eu la maladresse de me rendre le mouchoir? 

\speak  Pourquoi avez-vous eu celle de le laisser tomber? 

\speak  J'ai dit et je répète, monsieur, que ce mouchoir n'est point sorti de ma poche. 

\speak  Eh bien, vous en avez menti deux fois, monsieur, car je l'en ai vu sortir, moi! 

\speak  Ah! vous le prenez sur ce ton, monsieur le Gascon! eh bien, je vous apprendrai à vivre. 

\speak  Et moi je vous renverrai à votre messe, monsieur l'abbé! Dégainez, s'il vous plaît, et à l'instant même. 

\speak  Non pas, s'il vous plaît, mon bel ami; non, pas ici, du moins. Ne voyez-vous pas que nous sommes en face de l'hôtel d'Aiguillon, lequel est plein de créatures du cardinal? Qui me dit que ce n'est pas Son Éminence qui vous a chargé de lui procurer ma tête? Or j'y tiens ridiculement, à ma tête, attendu qu'elle me semble aller assez correctement à mes épaules. Je veux donc vous tuer, soyez tranquille, mais vous tuer tout doucement, dans un endroit clos et couvert, là où vous ne puissiez vous vanter de votre mort à personne. 

\speak  Je le veux bien, mais ne vous y fiez pas, et emportez votre mouchoir, qu'il vous appartienne ou non; peut-être aurez-vous l'occasion de vous en servir. 

\speak  Monsieur est Gascon? demanda Aramis. 

\speak  Oui. Monsieur ne remet pas un rendez-vous par prudence? 

\speak  La prudence, monsieur, est une vertu assez inutile aux mousquetaires, je le sais, mais indispensable aux gens d'Église, et comme je ne suis mousquetaire que provisoirement, je tiens à rester prudent. À deux heures, j'aurai l'honneur de vous attendre à l'hôtel de M. de Tréville. Là je vous indiquerai les bons endroits.» 

Les deux jeunes gens se saluèrent, puis Aramis s'éloigna en remontant la rue qui remontait au Luxembourg, tandis que d'Artagnan, voyant que l'heure s'avançait, prenait le chemin des Carmes-Deschaux, tout en disant à part soi: 

«Décidément, je n'en puis pas revenir; mais au moins, si je suis tué, je serai tué par un mousquetaire.»
