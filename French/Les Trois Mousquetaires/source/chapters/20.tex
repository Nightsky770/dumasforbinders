%!TeX root=../musketeersfr.tex 

\chapter{Voyage} 
	
	\lettrine{\accentletter[\gravebox]{A}}{} deux heures du matin, nos quatre aventuriers sortirent de Paris par la barrière Saint-Denis; tant qu'il fit nuit, ils restèrent muets; malgré eux, ils subissaient l'influence de l'obscurité et voyaient des embûches partout. 

Aux premiers rayons du jour, leurs langues se délièrent; avec le soleil, la gaieté revint: c'était comme à la veille d'un combat, le cœur battait, les yeux riaient; on sentait que la vie qu'on allait peut-être quitter était, au bout du compte, une bonne chose. 

L'aspect de la caravane, au reste, était des plus formidables: les chevaux noirs des mousquetaires, leur tournure martiale, cette habitude de l'escadron qui fait marcher régulièrement ces nobles compagnons du soldat, eussent trahi le plus strict incognito. 

Les valets suivaient, armés jusqu'aux dents. 

Tout alla bien jusqu'à Chantilly, où l'on arriva vers les huit heures du matin. Il fallait déjeuner. On descendit devant une auberge que recommandait une enseigne représentant saint Martin donnant la moitié de son manteau à un pauvre. On enjoignit aux laquais de ne pas desseller les chevaux et de se tenir prêts à repartir immédiatement. 

On entra dans la salle commune, et l'on se mit à table. Un gentilhomme, qui venait d'arriver par la route de Dammartin, était assis à cette même table et déjeunait. Il entama la conversation sur la pluie et le beau temps; les voyageurs répondirent: il but à leur santé; les voyageurs lui rendirent sa politesse. 

Mais au moment où Mousqueton venait annoncer que les chevaux étaient prêts et où l'on se levait de table l'étranger proposa à Porthos la santé du cardinal. Porthos répondit qu'il ne demandait pas mieux, si l'étranger à son tour voulait boire à la santé du roi. L'étranger s'écria qu'il ne connaissait d'autre roi que Son Éminence. Porthos l'appela ivrogne; l'étranger tira son épée. 

«Vous avez fait une sottise, dit Athos; n'importe, il n'y a plus à reculer maintenant: tuez cet homme et venez nous rejoindre le plus vite que vous pourrez.» 

Et tous trois remontèrent à cheval et repartirent à toute bride, tandis que Porthos promettait à son adversaire de le perforer de tous les coups connus dans l'escrime. 

«Et d'un! dit Athos au bout de cinq cents pas. 

\speak  Mais pourquoi cet homme s'est-il attaqué à Porthos plutôt qu'à tout autre? demanda Aramis. 

\speak  Parce que, Porthos parlant plus haut que nous tous il l'a pris pour le chef, dit d'Artagnan. 

\speak  J'ai toujours dit que ce cadet de Gascogne était un puits de sagesse», murmura Athos. 

Et les voyageurs continuèrent leur route. 

À Beauvais, on s'arrêta deux heures, tant pour faire souffler les chevaux que pour attendre Porthos. Au bout de deux heures, comme Porthos n'arrivait pas, ni aucune nouvelle de lui, on se remit en chemin. 

À une lieue de Beauvais, à un endroit où le chemin se trouvait resserré entre deux talus, on rencontra huit ou dix hommes qui, profitant de ce que la route était dépavée en cet endroit, avaient l'air d'y travailler en y creusant des trous et en pratiquant des ornières boueuses. 

Aramis, craignant de salir ses bottes dans ce mortier artificiel, les apostropha durement. Athos voulut le retenir, il était trop tard. Les ouvriers se mirent à railler les voyageurs, et firent perdre par leur insolence la tête même au froid Athos qui poussa son cheval contre l'un d'eux. 

Alors chacun de ces hommes recula jusqu'au fossé et y prit un mousquet caché; il en résulta que nos sept voyageurs furent littéralement passés par les armes. Aramis reçut une balle qui lui traversa l'épaule, et Mousqueton une autre balle qui se logea dans les parties charnues qui prolongent le bas des reins. Cependant Mousqueton seul tomba de cheval, non pas qu'il fût grièvement blessé, mais, comme il ne pouvait voir sa blessure, sans doute il crut être plus dangereusement blessé qu'il ne l'était. 

«C'est une embuscade, dit d'Artagnan, ne brûlons pas une amorce, et en route.» 

Aramis, tout blessé qu'il était, saisit la crinière de son cheval, qui l'emporta avec les autres. Celui de Mousqueton les avait rejoints, et galopait tout seul à son rang. 

«Cela nous fera un cheval de rechange, dit Athos. 

\speak  J'aimerais mieux un chapeau, dit d'Artagnan, le mien a été emporté par une balle. C'est bien heureux, ma foi, que la lettre que je porte n'ait pas été dedans. 

\speak  Ah çà, mais ils vont tuer le pauvre Porthos quand il passera, dit Aramis. 

\speak  Si Porthos était sur ses jambes, il nous aurait rejoints maintenant, dit Athos. M'est avis que, sur le terrain, l'ivrogne se sera dégrisé.» 

Et l'on galopa encore pendant deux heures, quoique les chevaux fussent si fatigués, qu'il était à craindre qu'ils ne refusassent bientôt le service. 

Les voyageurs avaient pris la traverse, espérant de cette façon être moins inquiétés, mais, à Crève-cœur, Aramis déclara qu'il ne pouvait aller plus loin. En effet, il avait fallu tout le courage qu'il cachait sous sa forme élégante et sous ses façons polies pour arriver jusque-là. À tout moment il pâlissait, et l'on était obligé de le soutenir sur son cheval; on le descendit à la porte d'un cabaret, on lui laissa Bazin qui, au reste, dans une escarmouche, était plus embarrassant qu'utile, et l'on repartit dans l'espérance d'aller coucher à Amiens. 

«Morbleu! dit Athos, quand ils se retrouvèrent en route, réduits à deux maîtres et à Grimaud et Planchet, morbleu! je ne serai plus leur dupe, et je vous réponds qu'ils ne me feront pas ouvrir la bouche ni tirer l'épée d'ici à Calais. J'en jure\dots 

\speak  Ne jurons pas, dit d'Artagnan, galopons, si toutefois nos chevaux y consentent.» 

Et les voyageurs enfoncèrent leurs éperons dans le ventre de leurs chevaux, qui, vigoureusement stimulés, retrouvèrent des forces. On arriva à Amiens à minuit, et l'on descendit à l'auberge du Lis d'Or. 

L'hôtelier avait l'air du plus honnête homme de la terre, il reçut les voyageurs son bougeoir d'une main et son bonnet de coton de l'autre; il voulut loger les deux voyageurs chacun dans une charmante chambre, malheureusement chacune de ces chambres était à l'extrémité de l'hôtel. D'Artagnan et Athos refusèrent; l'hôte répondit qu'il n'y en avait cependant pas d'autres dignes de Leurs Excellences; mais les voyageurs déclarèrent qu'ils coucheraient dans la chambre commune, chacun sur un matelas qu'on leur jetterait à terre. L'hôte insista, les voyageurs tinrent bon; il fallut faire ce qu'ils voulurent. 

Ils venaient de disposer leur lit et de barricader leur porte en dedans, lorsqu'on frappa au volet de la cour; ils demandèrent qui était là, reconnurent la voix de leurs valets et ouvrirent. 

En effet, c'étaient Planchet et Grimaud. 

«Grimaud suffira pour garder les chevaux, dit Planchet; si ces messieurs veulent, je coucherai en travers de leur porte; de cette façon-là, ils seront sûrs qu'on n'arrivera pas jusqu'à eux. 

\speak  Et sur quoi coucheras-tu? dit d'Artagnan. 

\speak  Voici mon lit», répondit Planchet. 

Et il montra une botte de paille. 

«Viens donc, dit d'Artagnan, tu as raison: la figure de l'hôte ne me convient pas, elle est trop gracieuse. 

\speak  Ni à moi non plus», dit Athos. 

Planchet monta par la fenêtre, s'installa en travers de la porte, tandis que Grimaud allait s'enfermer dans l'écurie, répondant qu'à cinq heures du matin lui et les quatre chevaux seraient prêts. 

La nuit fut assez tranquille, on essaya bien vers les deux heures du matin d'ouvrir la porte, mais comme Planchet se réveilla en sursaut et cria: Qui va là? on répondit qu'on se trompait, et on s'éloigna. 

À quatre heures du matin, on entendit un grand bruit dans les écuries. Grimaud avait voulu réveiller les garçons d'écurie, et les garçons d'écurie le battaient. Quand on ouvrit la fenêtre, on vit le pauvre garçon sans connaissance, la tête fendue d'un coup de manche à fourche. 

Planchet descendit dans la cour et voulut seller les chevaux; les chevaux étaient fourbus. Celui de Mousqueton seul, qui avait voyagé sans maître pendant cinq ou six heures la veille, aurait pu continuer la route; mais, par une erreur inconcevable, le chirurgien vétérinaire qu'on avait envoyé chercher, à ce qu'il paraît, pour saigner le cheval de l'hôte, avait saigné celui de Mousqueton. 

Cela commençait à devenir inquiétant: tous ces accidents successifs étaient peut-être le résultat du hasard, mais ils pouvaient tout aussi bien être le fruit d'un complot. Athos et d'Artagnan sortirent, tandis que Planchet allait s'informer s'il n'y avait pas trois chevaux à vendre dans les environs. À la porte étaient deux chevaux tout équipés, frais et vigoureux. Cela faisait bien l'affaire. Il demanda où étaient les maîtres; on lui dit que les maîtres avaient passé la nuit dans l'auberge et réglaient leur compte à cette heure avec le maître. 

Athos descendit pour payer la dépense, tandis que d'Artagnan et Planchet se tenaient sur la porte de la rue; l'hôtelier était dans une chambre basse et reculée, on pria Athos d'y passer. 

Athos entra sans défiance et tira deux pistoles pour payer: l'hôte était seul et assis devant son bureau, dont un des tiroirs était entrouvert. Il prit l'argent que lui présenta Athos, le tourna et le retourna dans ses mains, et tout à coup, s'écriant que la pièce était fausse, il déclara qu'il allait le faire arrêter, lui et son compagnon, comme faux-monnayeurs. 

«Drôle! dit Athos, en marchant sur lui, je vais te couper les oreilles!» 

Au même moment, quatre hommes armés jusqu'aux dents entrèrent par les portes latérales et se jetèrent sur Athos. 

«Je suis pris, cria Athos de toutes les forces de ses poumons; au large, d'Artagnan! pique, pique!» et il lâcha deux coups de pistolet. 

D'Artagnan et Planchet ne se le firent pas répéter à deux fois, ils détachèrent les deux chevaux qui attendaient à la porte, sautèrent dessus, leur enfoncèrent leurs éperons dans le ventre et partirent au triple galop. 

«Sais-tu ce qu'est devenu Athos? demanda d'Artagnan à Planchet en courant. 

\speak  Ah! monsieur, dit Planchet, j'en ai vu tomber deux à ses deux coups, et il m'a semblé, à travers la porte vitrée, qu'il ferraillait avec les autres. 

\speak  Brave Athos! murmura d'Artagnan. Et quand on pense qu'il faut l'abandonner! Au reste, autant nous attend peut-être à deux pas d'ici. En avant, Planchet, en avant! tu es un brave homme. 

\speak  Je vous l'ai dit, monsieur, répondit Planchet, les Picards, ça se reconnaît à l'user; d'ailleurs je suis ici dans mon pays, ça m'excite.» 

Et tous deux, piquant de plus belle, arrivèrent à Saint-Omer d'une seule traite. À Saint-Omer, ils firent souffler les chevaux la bride passée à leurs bras, de peur d'accident, et mangèrent un morceau sur le pouce tout debout dans la rue; après quoi ils repartirent. 

À cent pas des portes de Calais, le cheval de d'Artagnan s'abattit, et il n'y eut pas moyen de le faire se relever: le sang lui sortait par le nez et par les yeux, restait celui de Planchet, mais celui-là s'était arrêté, et il n'y eut plus moyen de le faire repartir. 

Heureusement, comme nous l'avons dit, ils étaient à cent pas de la ville; ils laissèrent les deux montures sur le grand chemin et coururent au port. Planchet fit remarquer à son maître un gentilhomme qui arrivait avec son valet et qui ne les précédait que d'une cinquantaine de pas. 

Ils s'approchèrent vivement de ce gentilhomme, qui paraissait fort affairé. Il avait ses bottes couvertes de poussière, et s'informait s'il ne pourrait point passer à l'instant même en Angleterre. 

«Rien ne serait plus facile, répondit le patron d'un bâtiment prêt à mettre à la voile; mais, ce matin, est arrivé l'ordre de ne laisser partir personne sans une permission expresse de M. le cardinal. 

\speak  J'ai cette permission, dit le gentilhomme en tirant un papier de sa poche; la voici. 

\speak  Faites-la viser par le gouverneur du port, dit le patron, et donnez-moi la préférence. 

\speak  Où trouverai-je le gouverneur? 

\speak  À sa campagne. 

\speak  Et cette campagne est située? 

\speak  À un quart de lieue de la ville; tenez, vous la voyez d'ici, au pied de cette petite éminence, ce toit en ardoises. 

\speak  Très bien!» dit le gentilhomme. 

Et, suivi de son laquais, il prit le chemin de la maison de campagne du gouverneur. 

D'Artagnan et Planchet suivirent le gentilhomme à cinq cents pas de distance. 

Une fois hors de la ville, d'Artagnan pressa le pas et rejoignit le gentilhomme comme il entrait dans un petit bois. 

«Monsieur, lui dit d'Artagnan, vous me paraissez fort pressé? 

\speak  On ne peut plus pressé, monsieur. 

\speak  J'en suis désespéré, dit d'Artagnan, car, comme je suis très pressé aussi, je voulais vous prier de me rendre un service. 

\speak  Lequel? 

\speak  De me laisser passer le premier. 

\speak  Impossible, dit le gentilhomme, j'ai fait soixante lieues en quarante-quatre heures, et il faut que demain à midi je sois à Londres. 

\speak  J'ai fait le même chemin en quarante heures, et il faut que demain à dix heures du matin je sois à Londres. 

\speak  Désespéré, monsieur; mais je suis arrivé le premier et je ne passerai pas le second. 

\speak  Désespéré, monsieur; mais je suis arrivé le second et je passerai le premier. 

\speak  Service du roi! dit le gentilhomme. 

\speak  Service de moi! dit d'Artagnan. 

\speak  Mais c'est une mauvaise querelle que vous me cherchez là, ce me semble. 

\speak  Parbleu! que voulez-vous que ce soit? 

\speak  Que désirez-vous? 

\speak  Vous voulez le savoir? 

\speak  Certainement. 

\speak  Eh bien, je veux l'ordre dont vous êtes porteur, attendu que je n'en ai pas, moi, et qu'il m'en faut un. 

\speak  Vous plaisantez, je présume. 

\speak  Je ne plaisante jamais. 

\speak  Laissez-moi passer! 

\speak  Vous ne passerez pas. 

\speak  Mon brave jeune homme, je vais vous casser la tête. Holà, Lubin! mes pistolets. 

\speak  Planchet, dit d'Artagnan, charge-toi du valet, je me charge du maître.» 

Planchet, enhardi par le premier exploit, sauta sur Lubin, et comme il était fort et vigoureux, il le renversa les reins contre terre et lui mit le genou sur la poitrine. 

«Faites votre affaire, monsieur, dit Planchet; moi, j'ai fait la mienne.» 

Voyant cela, le gentilhomme tira son épée et fondit sur d'Artagnan; mais il avait affaire à forte partie. 

En trois secondes d'Artagnan lui fournit trois coups d'épée en disant à chaque coup: 

«Un pour Athos, un pour Porthos, un pour Aramis.» 

Au troisième coup, le gentilhomme tomba comme une masse. 

D'Artagnan le crut mort, ou tout au moins évanoui, et s'approcha pour lui prendre l'ordre; mais au moment où il étendait le bras afin de le fouiller, le blessé qui n'avait pas lâché son épée, lui porta un coup de pointe dans la poitrine en disant: 

«Un pour vous. 

\speak  Et un pour moi! au dernier les bons!» s'écria d'Artagnan furieux, en le clouant par terre d'un quatrième coup d'épée dans le ventre. 

Cette fois, le gentilhomme ferma les yeux et s'évanouit. 

D'Artagnan fouilla dans la poche où il l'avait vu remettre l'ordre de passage, et le prit. Il était au nom du comte de Wardes. 

Puis, jetant un dernier coup d'œil sur le beau jeune homme, qui avait vingt-cinq ans à peine et qu'il laissait là, gisant, privé de sentiment et peut-être mort, il poussa un soupir sur cette étrange destinée qui porte les hommes à se détruire les uns les autres pour les intérêts de gens qui leur sont étrangers et qui souvent ne savent pas même qu'ils existent. 

Mais il fut bientôt tiré de ces réflexions par Lubin, qui poussait des hurlements et criait de toutes ses forces au secours. 

Planchet lui appliqua la main sur la gorge et serra de toutes ses forces. 

«Monsieur, dit-il, tant que je le tiendrai ainsi, il ne criera pas, j'en suis bien sûr; mais aussitôt que je le lâcherai, il va se remettre à crier. Je le reconnais pour un Normand et les Normands sont entêtés.» 

En effet, tout comprimé qu'il était, Lubin essayait encore de filer des sons. 

«Attends!» dit d'Artagnan. 

Et prenant son mouchoir, il le bâillonna. 

«Maintenant, dit Planchet, lions-le à un arbre.» 

La chose fut faite en conscience, puis on tira le comte de Wardes près de son domestique; et comme la nuit commençait à tomber et que le garrotté et le blessé étaient tous deux à quelques pas dans le bois, il était évident qu'ils devaient rester jusqu'au lendemain. 

«Et maintenant, dit d'Artagnan, chez le gouverneur! 

\speak  Mais vous êtes blessé, ce me semble? dit Planchet. 

\speak  Ce n'est rien, occupons-nous du plus pressé; puis nous reviendrons à ma blessure, qui, au reste, ne me paraît pas très dangereuse.» 

Et tous deux s'acheminèrent à grands pas vers la campagne du digne fonctionnaire. 

On annonça M. le comte de Wardes. 

D'Artagnan fut introduit. 

«Vous avez un ordre signé du cardinal? dit le gouverneur. 

\speak  Oui, monsieur, répondit d'Artagnan, le voici. 

\speak  Ah! ah! il est en règle et bien recommandé, dit le gouverneur. 

\speak  C'est tout simple, répondit d'Artagnan, je suis de ses plus fidèles. 

\speak  Il paraît que Son Éminence veut empêcher quelqu'un de parvenir en Angleterre. 

\speak  Oui, un certain d'Artagnan, un gentilhomme béarnais qui est parti de Paris avec trois de ses amis dans l'intention de gagner Londres. 

\speak  Le connaissez-vous personnellement? demanda le gouverneur. 

\speak  Qui cela? 

\speak  Ce d'Artagnan? 

\speak  À merveille. 

\speak  Donnez-moi son signalement alors. 

\speak  Rien de plus facile.» 

Et d'Artagnan donna trait pour trait le signalement du comte de Wardes. 

«Est-il accompagné? demanda le gouverneur. 

\speak  Oui, d'un valet nommé Lubin. 

\speak  On veillera sur eux, et si on leur met la main dessus, Son Éminence peut être tranquille, ils seront reconduits à Paris sous bonne escorte. 

\speak  Et ce faisant, monsieur le gouverneur, dit d'Artagnan, vous aurez bien mérité du cardinal. 

\speak  Vous le reverrez à votre retour, monsieur le comte? 

\speak  Sans aucun doute. 

\speak  Dites-lui, je vous prie, que je suis bien son serviteur. 

\speak  Je n'y manquerai pas.» 

Et joyeux de cette assurance, le gouverneur visa le laissez-passer et le remit à d'Artagnan. 

D'Artagnan ne perdit pas son temps en compliments inutiles, il salua le gouverneur, le remercia et partit. 

Une fois dehors, lui et Planchet prirent leur course, et faisant un long détour, ils évitèrent le bois et rentrèrent par une autre porte. 

Le bâtiment était toujours prêt à partir, le patron attendait sur le port. 

«Eh bien? dit-il en apercevant d'Artagnan. 

\speak  Voici ma passe visée, dit celui-ci. 

\speak  Et cet autre gentilhomme? 

\speak  Il ne partira pas aujourd'hui, dit d'Artagnan, mais soyez tranquille, je paierai le passage pour nous deux. 

\speak  En ce cas, partons, dit le patron. 

\speak  Partons!» répéta d'Artagnan. 

Et il sauta avec Planchet dans le canot; cinq minutes après, ils étaient à bord. 

Il était temps: à une demi-lieue en mer, d'Artagnan vit briller une lumière et entendit une détonation. 

C'était le coup de canon qui annonçait la fermeture du port. 

Il était temps de s'occuper de sa blessure; heureusement, comme l'avait pensé d'Artagnan, elle n'était pas des plus dangereuses: la pointe de l'épée avait rencontré une côte et avait glissé le long de l'os; de plus, la chemise s'était collée aussitôt à la plaie, et à peine avait-elle répandu quelques gouttes de sang. 

D'Artagnan était brisé de fatigue: on lui étendit un matelas sur le pont, il se jeta dessus et s'endormit. 

Le lendemain, au point du jour, il se trouva à trois ou quatre lieues seulement des côtes d'Angleterre; la brise avait été faible toute la nuit, et l'on avait peu marché. 

À dix heures, le bâtiment jetait l'ancre dans le port de Douvres. 

À dix heures et demie, d'Artagnan mettait le pied sur la terre d'Angleterre, en s'écriant: 

«Enfin, m'y voilà!» 

Mais ce n'était pas tout: il fallait gagner Londres. En Angleterre, la poste était assez bien servie. D'Artagnan et Planchet prirent chacun un bidet, un postillon courut devant eux; en quatre heures ils arrivèrent aux portes de la capitale. 

D'Artagnan ne connaissait pas Londres, d'Artagnan ne savait pas un mot d'anglais; mais il écrivit le nom de Buckingham sur un papier, et chacun lui indiqua l'hôtel du duc. 

Le duc était à la chasse à Windsor, avec le roi. 

D'Artagnan demanda le valet de chambre de confiance du duc, qui, l'ayant accompagné dans tous ses voyages, parlait parfaitement français; il lui dit qu'il arrivait de Paris pour affaire de vie et de mort, et qu'il fallait qu'il parlât à son maître à l'instant même. 

La confiance avec laquelle parlait d'Artagnan convainquit Patrice; c'était le nom de ce ministre du ministre. Il fit seller deux chevaux et se chargea de conduire le jeune garde. Quant à Planchet, on l'avait descendu de sa monture, raide comme un jonc: le pauvre garçon était au bout de ses forces; d'Artagnan semblait de fer. 

On arriva au château; là on se renseigna: le roi et Buckingham chassaient à l'oiseau dans des marais situés à deux ou trois lieues de là. 

En vingt minutes on fut au lieu indiqué. Bientôt Patrice entendit la voix de son maître, qui appelait son faucon. 

«Qui faut-il que j'annonce à Milord duc? demanda Patrice. 

\speak  Le jeune homme qui, un soir, lui a cherché une querelle sur le Pont-Neuf, en face de la Samaritaine. 

\speak  Singulière recommandation! 

\speak  Vous verrez qu'elle en vaut bien une autre.» 

Patrice mit son cheval au galop, atteignit le duc et lui annonça dans les termes que nous avons dits qu'un messager l'attendait. 

Buckingham reconnut d'Artagnan à l'instant même, et se doutant que quelque chose se passait en France dont on lui faisait parvenir la nouvelle, il ne prit que le temps de demander où était celui qui la lui apportait; et ayant reconnu de loin l'uniforme des gardes, il mit son cheval au galop et vint droit à d'Artagnan. Patrice, par discrétion, se tint à l'écart. 

«Il n'est point arrivé malheur à la reine? s'écria Buckingham, répandant toute sa pensée et tout son amour dans cette interrogation. 

\speak  Je ne crois pas; cependant je crois qu'elle court quelque grand péril dont Votre Grâce seule peut la tirer. 

\speak  Moi? s'écria Buckingham. Eh quoi! je serais assez heureux pour lui être bon à quelque chose! Parlez! parlez! 

\speak  Prenez cette lettre, dit d'Artagnan. 

\speak  Cette lettre! de qui vient cette lettre? 

\speak  De Sa Majesté, à ce que je pense. 

\speak  De Sa Majesté!» dit Buckingham, pâlissant si fort que d'Artagnan crut qu'il allait se trouver mal. 

Et il brisa le cachet. 

«Quelle est cette déchirure? dit-il en montrant à d'Artagnan un endroit où elle était percée à jour. 

\speak  Ah! ah! dit d'Artagnan, je n'avais pas vu cela; c'est l'épée du comte de Wardes qui aura fait ce beau coup en me trouant la poitrine. 

\speak  Vous êtes blessé? demanda Buckingham en rompant le cachet. 

\speak  Oh! rien! dit d'Artagnan, une égratignure. 

\speak  Juste Ciel! qu'ai-je lu! s'écria le duc. Patrice, reste ici, ou plutôt rejoins le roi partout où il sera, et dis à Sa Majesté que je la supplie bien humblement de m'excuser, mais qu'une affaire de la plus haute importance me rappelle à Londres. Venez, monsieur, venez.» 

Et tous deux reprirent au galop le chemin de la capitale.