%!TeX root=../musketeersfr.tex 

\chapter{Le Pavillon} 
	
\lettrine{\accentletter[\gravebox]{A}}{} neuf heures, d'Artagnan était à l'hôtel des Gardes; il trouva Planchet sous les armes. Le quatrième cheval était arrivé. 

\zz
Planchet était armé de son mousqueton et d'un pistolet. D'Artagnan avait son épée et passa deux pistolets à sa ceinture, puis tous deux enfourchèrent chacun un cheval et s'éloignèrent sans bruit. Il faisait nuit close, et personne ne les vit sortir. Planchet se mit à la suite de son maître, et marcha par-derrière à dix pas. 

D'Artagnan traversa les quais, sortit par la porte de la Conférence et suivit alors le chemin, bien plus beau alors qu'aujourd'hui, qui mène à Saint-Cloud. 

Tant qu'on fut dans la ville, Planchet garda respectueusement la distance qu'il s'était imposée; mais dès que le chemin commença à devenir plus désert et plus obscurs il se rapprocha tout doucement: si bien que, lorsqu'on entra dans le bois de Boulogne, il se trouva tout naturellement marcher côte à côte avec son maître. En effet, nous ne devons pas dissimuler que l'oscillation des grands arbres et le reflet de la lune dans les taillis sombres lui causaient une vive inquiétude. D'Artagnan s'aperçut qu'il se passait chez son laquais quelque chose d'extraordinaire. 

«Eh bien, monsieur Planchet, lui demanda-t-il, qu'avons-nous donc? 

\speak  Ne trouvez-vous pas, monsieur, que les bois sont comme les églises? 

\speak  Pourquoi cela, Planchet? 

\speak  Parce qu'on n'ose point parler haut dans ceux-ci comme dans celles-là. 

\speak  Pourquoi n'oses-tu parler haut, Planchet? parce que tu as peur? 

\speak  Peur d'être entendu, oui, monsieur. 

\speak  Peur d'être entendu! Notre conversation est cependant morale, mon cher Planchet, et nul n'y trouverait à redire. 

\speak  Ah! monsieur! reprit Planchet en revenant à son idée mère, que ce M. Bonacieux a quelque chose de sournois dans ses sourcils et de déplaisant dans le jeu de ses lèvres! 

\speak  Qui diable te fait penser à Bonacieux? 

\speak  Monsieur, l'on pense à ce que l'on peut et non pas à ce que l'on veut. 

\speak  Parce que tu es un poltron, Planchet. 

\speak  Monsieur, ne confondons pas la prudence avec la poltronnerie; la prudence est une vertu. 

\speak  Et tu es vertueux, n'est-ce pas, Planchet? 

\speak  Monsieur, n'est-ce point le canon d'un mousquet qui brille là-bas? Si nous baissions la tête? 

\speak  En vérité, murmura d'Artagnan, à qui les recommandations de M. de Tréville revenaient en mémoire; en vérité, cet animal finirait par me faire peur.» 

Et il mit son cheval au trot. 

Planchet suivit le mouvement de son maître, exactement comme s'il eût été son ombre, et se retrouva trottant près de lui. 

«Est-ce que nous allons marcher comme cela toute la nuit, monsieur? demanda-t-il. 

\speak  Non, Planchet, car tu es arrivé, toi. 

\speak  Comment, je suis arrivé? et monsieur? 

\speak  Moi, je vais encore à quelques pas. 

\speak  Et monsieur me laisse seul ici? 

\speak  Tu as peur, Planchet? 

\speak  Non, mais je fais seulement observer à monsieur que la nuit sera très froide, que les fraîcheurs donnent des rhumatismes, et qu'un laquais qui a des rhumatismes est un triste serviteur, surtout pour un maître alerte comme monsieur. 

\speak  Eh bien, si tu as froid, Planchet, tu entreras dans un de ces cabarets que tu vois là-bas, et tu m'attendras demain matin à six heures devant la porte. 

\speak  Monsieur, j'ai bu et mangé respectueusement l'écu que vous m'avez donné ce matin; de sorte qu'il ne me reste pas un traître sou dans le cas où j'aurais froid. 

\speak  Voici une demi-pistole. À demain.» 

D'Artagnan descendit de son cheval, jeta la bride au bras de Planchet et s'éloigna rapidement en s'enveloppant dans son manteau. 

«Dieu que j'ai froid!» s'écria Planchet dès qu'il eut perdu son maître de vue; --- et pressé qu'il était de se réchauffer, il se hâta d'aller frapper à la porte d'une maison parée de tous les attributs d'un cabaret de banlieue. 

Cependant d'Artagnan, qui s'était jeté dans un petit chemin de traverse, continuait sa route et atteignait Saint-Cloud; mais, au lieu de suivre la grande rue, il tourna derrière le château, gagna une espèce de ruelle fort écartée, et se trouva bientôt en face du pavillon indiqué. Il était situé dans un lieu tout à fait désert. Un grand mur, à l'angle duquel était ce pavillon, régnait d'un côté de cette ruelle, et de l'autre une haie défendait contre les passants un petit jardin au fond duquel s'élevait une maigre cabane. 

Il était arrivé au rendez-vous, et comme on ne lui avait pas dit d'annoncer sa présence par aucun signal, il attendit. 

Nul bruit ne se faisait entendre, on eût dit qu'on était à cent lieues de la capitale. D'Artagnan s'adossa à la haie après avoir jeté un coup d'œil derrière lui. Par-delà cette haie, ce jardin et cette cabane, un brouillard sombre enveloppait de ses plis cette immensité où dort Paris, vide, béant, immensité où brillaient quelques points lumineux, étoiles funèbres de cet enfer. 

Mais pour d'Artagnan tous les aspects revêtaient une forme heureuse, toutes les idées avaient un sourire, toutes les ténèbres étaient diaphanes. L'heure du rendez-vous allait sonner. 

En effet, au bout de quelques instants, le beffroi de Saint-Cloud laissa lentement tomber dix coups de sa large gueule mugissante. 

Il y avait quelque chose de lugubre à cette voix de bronze qui se lamentait ainsi au milieu de la nuit. 

Mais chacune de ces heures qui composaient l'heure attendue vibrait harmonieusement au cœur du jeune homme. 

Ses yeux étaient fixés sur le petit pavillon situé à l'angle de la rue et dont toutes les fenêtres étaient fermées par des volets, excepté une seule du premier étage. 

À travers cette fenêtre brillait une lumière douce qui argentait le feuillage tremblant de deux ou trois tilleuls qui s'élevaient formant groupe en dehors du parc. Évidemment derrière cette petite fenêtre, si gracieusement éclairée, la jolie Mme Bonacieux l'attendait. 

Bercé par cette douce idée, d'Artagnan attendit de son côté une demi-heure sans impatience aucune, les yeux fixés sur ce charmant petit séjour dont d'Artagnan apercevait une partie de plafond aux moulures dorées, attestant l'élégance du reste de l'appartement. 

Le beffroi de Saint-Cloud sonna dix heures et demie. 

Cette fois-ci, sans que d'Artagnan comprît pourquoi, un frisson courut dans ses veines. Peut-être aussi le froid commençait-il à le gagner et prenait-il pour une impression morale une sensation tout à fait physique. 

Puis l'idée lui vint qu'il avait mal lu et que le rendez-vous était pour onze heures seulement. 

Il s'approcha de la fenêtre, se plaça dans un rayon de lumière, tira sa lettre de sa poche et la relut; il ne s'était point trompé: le rendez-vous était bien pour dix heures. 

Il alla reprendre son poste, commençant à être assez inquiet de ce silence et de cette solitude. 

Onze heures sonnèrent. 

D'Artagnan commença à craindre véritablement qu'il ne fût arrivé quelque chose à Mme Bonacieux. 

Il frappa trois coups dans ses mains, signal ordinaire des amoureux; mais personne ne lui répondit: pas même l'écho. 

Alors il pensa avec un certain dépit que peut-être la jeune femme s'était endormie en l'attendant. 

Il s'approcha du mur et essaya d'y monter; mais le mur était nouvellement crépi, et d'Artagnan se retourna inutilement les ongles. 

En ce moment il avisa les arbres, dont la lumière continuait d'argenter les feuilles, et comme l'un d'eux faisait saillie sur le chemin, il pensa que du milieu de ses branches son regard pourrait pénétrer dans le pavillon. 

L'arbre était facile. D'ailleurs d'Artagnan avait vingt ans à peine, et par conséquent se souvenait de son métier d'écolier. En un instant il fut au milieu des branches, et par les vitres transparentes ses yeux plongèrent dans l'intérieur du pavillon. 

Chose étrange et qui fit frissonner d'Artagnan de la plante des pieds à la racine des cheveux, cette douce lumière, cette calme lampe éclairait une scène de désordre épouvantable; une des vitres de la fenêtre était cassée, la porte de la chambre avait été enfoncée et, à demi brisée pendait à ses gonds; une table qui avait dû être couverte d'un élégant souper gisait à terre; les flacons en éclats, les fruits écrasés jonchaient le parquet; tout témoignait dans cette chambre d'une lutte violente et désespérée; d'Artagnan crut même reconnaître au milieu de ce pêle-mêle étrange des lambeaux de vêtements et quelques taches sanglantes maculant la nappe et les rideaux. 

Il se hâta de redescendre dans la rue avec un horrible battement de cœur, il voulait voir s'il ne trouverait pas d'autres traces de violence. 

La petite lueur suave brillait toujours dans le calme de la nuit. D'Artagnan s'aperçut alors, chose qu'il n'avait pas remarquée d'abord, car rien ne le poussait à cet examen, que le sol, battu ici, troué là, présentait des traces confuses de pas d'hommes, et de pieds de chevaux. En outre, les roues d'une voiture, qui paraissait venir de Paris, avaient creusé dans la terre molle une profonde empreinte qui ne dépassait pas la hauteur du pavillon et qui retournait vers Paris. 

Enfin d'Artagnan, en poursuivant ses recherches, trouva près du mur un gant de femme déchiré. Cependant ce gant, par tous les points où il n'avait pas touché la terre boueuse, était d'une fraîcheur irréprochable. C'était un de ces gants parfumés comme les amants aiment à les arracher d'une jolie main. 

À mesure que d'Artagnan poursuivait ses investigations, une sueur plus abondante et plus glacée perlait sur son front, son cœur était serré par une horrible angoisse, sa respiration était haletante; et cependant il se disait, pour se rassurer, que ce pavillon n'avait peut-être rien de commun avec Mme Bonacieux; que la jeune femme lui avait donné rendez-vous devant ce pavillon, et non dans ce pavillon; qu'elle avait pu être retenue à Paris par son service, par la jalousie de son mari peut-être. 

Mais tous ces raisonnements étaient battus en brèche, détruits, renversés par ce sentiment de douleur intime, qui dans certaines occasions, s'empare de tout notre être et nous crie, par tout ce qui est destiné chez nous à entendre, qu'un grand malheur plane sur nous. 

Alors d'Artagnan devint presque insensé: il courut sur la grande route, prit le même chemin qu'il avait déjà fait, s'avança jusqu'au bac, et interrogea le passeur. 

Vers les sept heures du soir, le passeur avait fait traverser la rivière à une femme enveloppée d'une mante noire, qui paraissait avoir le plus grand intérêt à ne pas être reconnue; mais, justement à cause des précautions qu'elle prenait, le passeur avait prêté une attention plus grande, et il avait reconnu que la femme était jeune et jolie. 

Il y avait alors, comme aujourd'hui, une foule de jeunes et jolies femmes qui venaient à Saint-Cloud et qui avaient intérêt à ne pas être vues, et cependant d'Artagnan ne douta point un instant que ce ne fût Mme Bonacieux qu'avait remarquée le passeur. 

D'Artagnan profita de la lampe qui brillait dans la cabane du passeur pour relire encore une fois le billet de Mme Bonacieux et s'assurer qu'il ne s'était pas trompé, que le rendez-vous était bien à Saint-Cloud et non ailleurs, devant le pavillon de M. d'Estrées et non dans une autre rue. 

Tout concourait à prouver à d'Artagnan que ses pressentiments ne le trompaient point et qu'un grand malheur était arrivé. 

Il reprit le chemin du château tout courant; il lui semblait qu'en son absence quelque chose de nouveau s'était peut-être passé au pavillon et que des renseignements l'attendaient là. 

La ruelle était toujours déserte, et la même lueur calme et douce s'épanchait de la fenêtre. 

D'Artagnan songea alors à cette masure muette et aveugle mais qui sans doute avait vu et qui peut-être pouvait parler. 

La porte de clôture était fermée, mais il sauta par-dessus la haie, et malgré les aboiements du chien à la chaîne, il s'approcha de la cabane. 

Aux premiers coups qu'il frappa, rien ne répondit. 

Un silence de mort régnait dans la cabane comme dans le pavillon; cependant, comme cette cabane était sa dernière ressource, il s'obstina. 

Bientôt il lui sembla entendre un léger bruit intérieur, bruit craintif, et qui semblait trembler lui-même d'être entendu. 

Alors d'Artagnan cessa de frapper et pria, avec un accent si plein d'inquiétude et de promesses, d'effroi et de cajolerie, que sa voix était de nature à rassurer de plus peureux. Enfin un vieux volet vermoulu s'ouvrit, ou plutôt s'entrebâilla, et se referma dès que la lueur d'une misérable lampe qui brûlait dans un coin eut éclairé le baudrier, la poignée de l'épée et le pommeau des pistolets de d'Artagnan. Cependant, si rapide qu'eût été le mouvement, d'Artagnan avait eu le temps d'entrevoir une tête de vieillard. 

«Au nom du Ciel! dit-il, écoutez-moi: j'attendais quelqu'un qui ne vient pas, je meurs d'inquiétude. Serait-il arrivé quelque malheur aux environs? Parlez.» 

La fenêtre se rouvrit lentement, et la même figure apparut de nouveau: seulement elle était plus pâle encore que la première fois. 

D'Artagnan raconta naïvement son histoire, aux noms près; il dit comment il avait rendez-vous avec une jeune femme devant ce pavillon, et comment, ne la voyant pas venir, il était monté sur le tilleul et, à la lueur de la lampe, il avait vu le désordre de la chambre. 

Le vieillard l'écouta attentivement, tout en faisant signe que c'était bien cela: puis, lorsque d'Artagnan eut fini, il hocha la tête d'un air qui n'annonçait rien de bon. 

«Que voulez-vous dire? s'écria d'Artagnan. Au nom du Ciel! voyons, expliquez-vous. 

\speak  Oh! monsieur, dit le vieillard, ne me demandez rien; car si je vous disais ce que j'ai vu, bien certainement il ne m'arriverait rien de bon. 

\speak  Vous avez donc vu quelque chose? reprit d'Artagnan. En ce cas, au nom du Ciel! continua-t-il en lui jetant une pistole, dites, dites ce que vous avez vu, et je vous donne ma foi de gentilhomme que pas une de vos paroles ne sortira de mon cœur.» 

Le vieillard lut tant de franchise et de douleur sur le visage de d'Artagnan, qu'il lui fit signe d'écouter et qu'il lui dit à voix basse: 

«Il était neuf heures à peu près, j'avais entendu quelque bruit dans la rue et je désirais savoir ce que ce pouvait être, lorsqu'en m'approchant de ma porte je m'aperçus qu'on cherchait à entrer. Comme je suis pauvre et que je n'ai pas peur qu'on me vole, j'allai ouvrir et je vis trois hommes à quelques pas de là. Dans l'ombre était un carrosse avec des chevaux attelés et des chevaux de main. Ces chevaux de main appartenaient évidemment aux trois hommes qui étaient vêtus en cavaliers. 

«--- Ah, mes bons messieurs! m'écriai-je, que demandez-vous? 

«--- Tu dois avoir une échelle? me dit celui qui paraissait le chef de l'escorte. 

«--- Oui, monsieur; celle avec laquelle je cueille mes fruits. 

«--- Donne-nous la, et rentre chez toi, voilà un écu pour le dérangement que nous te causons. Souviens-toi seulement que si tu dis un mot de ce que tu vas voir et de ce que tu vas entendre (car tu regarderas et tu écouteras, quelque menace que nous te fassions, j'en suis sûr), tu es perdu. 

«À ces mots, il me jeta un écu, que je ramassai, et il prit mon échelle. 

«Effectivement, après avoir refermé la porte de la haie derrière eux, je fis semblant de rentrer à la maison; mais j'en sortis aussitôt par la porte de derrière, et, me glissant dans l'ombre, je parvins jusqu'à cette touffe de sureau, du milieu de laquelle je pouvais tout voir sans être vu. 

«Les trois hommes avaient fait avancer la voiture sans aucun bruit, ils en tirèrent un petit homme, gros, court, grisonnant, mesquinement vêtu de couleur sombre, lequel monta avec précaution à l'échelle, regarda sournoisement dans l'intérieur de la chambre, redescendit à pas de loup et murmura à voix basse: 

«--- C'est elle! 

«Aussitôt celui qui m'avait parlé s'approcha de la porte du pavillon, l'ouvrit avec une clef qu'il portait sur lui, referma la porte et disparut, en même temps les deux autres hommes montèrent à l'échelle. Le petit vieux demeurait à la portière, le cocher maintenait les chevaux de la voiture, et un laquais les chevaux de selle. 

Tout à coup de grands cris retentirent dans le pavillon, une femme accourut à la fenêtre et l'ouvrit comme pour se précipiter. Mais aussitôt qu'elle aperçut les deux hommes, elle se rejeta en arrière; les deux hommes s'élancèrent après elle dans la chambre. 

Alors je ne vis plus rien; mais j'entendis le bruit des meubles que l'on brise. La femme criait et appelait au secours. Mais bientôt ses cris furent étouffés; les trois hommes se rapprochèrent de la fenêtre, emportant la femme dans leurs bras; deux descendirent par l'échelle et la transportèrent dans la voiture, où le petit vieux entra après elle. Celui qui était resté dans le pavillon referma la croisée, sortit un instant après par la porte et s'assura que la femme était bien dans la voiture: ses deux compagnons l'attendaient déjà à cheval, il sauta à son tour en selle, le laquais reprit sa place près du cocher; le carrosse s'éloigna au galop escorté par les trois cavaliers, et tout fut fini. À partir de ce moment-là, je n'ai plus rien vu, rien entendu.» 

D'Artagnan, écrasé par une si terrible nouvelle, resta immobile et muet, tandis que tous les démons de la colère et de la jalousie hurlaient dans son cœur. 

«Mais, mon gentilhomme, reprit le vieillard, sur lequel ce muet désespoir causait certes plus d'effet que n'en eussent produit des cris et des larmes; allons, ne vous désolez pas, ils ne vous l'ont pas tuée, voilà l'essentiel. 

\speak  Savez-vous à peu près, dit d'Artagnan, quel est l'homme qui conduisait cette infernale expédition? 

\speak  Je ne le connais pas. 

\speak  Mais puisqu'il vous a parlé, vous avez pu le voir. 

\speak  Ah! c'est son signalement que vous me demandez? 

\speak  Oui. 

\speak  Un grand sec, basané, moustaches noires, œil noir, l'air d'un gentilhomme. 

\speak  C'est cela, s'écria d'Artagnan; encore lui! toujours lui! C'est mon démon, à ce qu'il paraît! Et l'autre? 

\speak  Lequel? 

\speak  Le petit. 

\speak  Oh! celui-là n'est pas un seigneur, j'en réponds: d'ailleurs il ne portait pas l'épée, et les autres le traitaient sans aucune considération. 

\speak  Quelque laquais, murmura d'Artagnan. Ah! pauvre femme! pauvre femme! qu'en ont-ils fait? 

\speak  Vous m'avez promis le secret, dit le vieillard. 

\speak  Et je vous renouvelle ma promesse, soyez tranquille, je suis gentilhomme. Un gentilhomme n'a que sa parole, et je vous ai donné la mienne.» 

D'Artagnan reprit, l'âme navrée, le chemin du bac. Tantôt il ne pouvait croire que ce fût Mme Bonacieux, et il espérait le lendemain la retrouver au Louvre; tantôt il craignait qu'elle n'eût eu une intrigue avec quelque autre et qu'un jaloux ne l'eût surprise et fait enlever. Il flottait, il se désolait, il se désespérait. 

«Oh! si j'avais là mes amis! s'écriait-il, j'aurais au moins quelque espérance de la retrouver; mais qui sait ce qu'ils sont devenus eux-mêmes!» 

Il était minuit à peu près; il s'agissait de retrouver Planchet. D'Artagnan se fit ouvrir successivement tous les cabarets dans lesquels il aperçut un peu de lumière; dans aucun d'eux il ne retrouva Planchet. 

Au sixième, il commença de réfléchir que la recherche était un peu hasardée. D'Artagnan n'avait donné rendez-vous à son laquais qu'à six heures du matin, et quelque part qu'il fût, il était dans son droit. 

D'ailleurs, il vint au jeune homme cette idée, qu'en restant aux environs du lieu où l'événement s'était passé, il obtiendrait peut-être quelque éclaircissement sur cette mystérieuse affaire. Au sixième cabaret, comme nous l'avons dit, d'Artagnan s'arrêta donc, demanda une bouteille de vin de première qualité, s'accouda dans l'angle le plus obscur et se décida à attendre ainsi le jour; mais cette fois encore son espérance fut trompée, et quoiqu'il écoutât de toutes ses oreilles, il n'entendit, au milieu des jurons, des lazzi et des injures qu'échangeaient entre eux les ouvriers, les laquais et les rouliers qui composaient l'honorable société dont il faisait partie, rien qui pût le mettre sur la trace de la pauvre femme enlevée. Force lui fut donc, après avoir avalé sa bouteille par désoeuvrement et pour ne pas éveiller des soupçons, de chercher dans son coin la posture la plus satisfaisante possible et de s'endormir tant bien que mal. D'Artagnan avait vingt ans, on se le rappelle, et à cet âge le sommeil a des droits imprescriptibles qu'il réclame impérieusement, même sur les cœurs les plus désespérés. 

Vers six heures du matin, d'Artagnan se réveilla avec ce malaise qui accompagne ordinairement le point du jour après une mauvaise nuit. Sa toilette n'était pas longue à faire; il se tâta pour savoir si on n'avait pas profité de son sommeil pour le voler, et ayant retrouvé son diamant à son doigt, sa bourse dans sa poche et ses pistolets à sa ceinture, il se leva, paya sa bouteille et sortit pour voir s'il n'aurait pas plus de bonheur dans la recherche de son laquais le matin que la nuit. En effet, la première chose qu'il aperçut à travers le brouillard humide et grisâtre fut l'honnête Planchet qui, les deux chevaux en main, l'attendait à la porte d'un petit cabaret borgne devant lequel d'Artagnan était passé sans même soupçonner son existence.