%!TeX root=../musketeersfr.tex 

\chapter{L'Amant Et Le Mari} 

\lettrine[ante=«]{A}{h!} madame, dit d'Artagnan en entrant par la porte que lui ouvrait la jeune femme, permettez-moi de vous le dire, vous avez là un triste mari. 

\zz
\noindent --- Vous avez donc entendu notre conversation? demanda vivement Mme Bonacieux en regardant d'Artagnan avec inquiétude. 

\zz
\speak  Tout entière. 

\speak  Mais comment cela? mon Dieu! 

\speak  Par un procédé à moi connu, et par lequel j'ai entendu aussi la conversation plus animée que vous avez eue avec les sbires du cardinal. 

\speak  Et qu'avez-vous compris dans ce que nous disions? 

\speak  Mille choses: d'abord, que votre mari est un niais et un sot, heureusement; puis, que vous étiez embarrassée, ce dont j'ai été fort aise, et que cela me donne une occasion de me mettre à votre service, et Dieu sait si je suis prêt à me jeter dans le feu pour vous; enfin que la reine a besoin qu'un homme brave, intelligent et dévoué fasse pour elle un voyage à Londres. J'ai au moins deux des trois qualités qu'il vous faut, et me voilà.» 

Mme Bonacieux ne répondit pas, mais son cœur battait de joie, et une secrète espérance brilla à ses yeux. 

«Et quelle garantie me donnerez-vous, demanda-t-elle, si je consens à vous confier cette mission? 

\speak  Mon amour pour vous. Voyons, dites, ordonnez: que faut-il faire? 

\speak  Mon Dieu! mon Dieu! murmura la jeune femme, dois-je vous confier un pareil secret, monsieur? Vous êtes presque un enfant! 

\speak  Allons, je vois qu'il vous faut quelqu'un qui vous réponde de moi. 

\speak  J'avoue que cela me rassurerait fort. 

\speak  Connaissez-vous Athos? 

\speak  Non. 

\speak  Porthos? 

\speak  Non. 

\speak  Aramis? 

\speak  Non. Quels sont ces messieurs? 

\speak  Des mousquetaires du roi. Connaissez-vous M. de Tréville, leur capitaine? 

\speak  Oh! oui, celui-là, je le connais, non pas personnellement, mais pour en avoir entendu plus d'une fois parler à la reine comme d'un brave et loyal gentilhomme. 

\speak  Vous ne craignez pas que lui vous trahisse pour le cardinal, n'est-ce pas? 

\speak  Oh! non, certainement. 

\speak  Eh bien, révélez-lui votre secret, et demandez-lui, si important, si précieux, si terrible qu'il soit, si vous pouvez me le confier. 

\speak  Mais ce secret ne m'appartient pas, et je ne puis le révéler ainsi. 

\speak  Vous l'alliez bien confier à M. Bonacieux, dit d'Artagnan avec dépit. 

\speak  Comme on confie une lettre au creux d'un arbre, à l'aile d'un pigeon, au collier d'un chien. 

\speak  Et cependant, moi, vous voyez bien que je vous aime. 

\speak  Vous le dites. 

\speak  Je suis un galant homme! 

\speak  Je le crois. 

\speak  Je suis brave! 

\speak  Oh! cela, j'en suis sûre. 

\speak  Alors, mettez-moi donc à l'épreuve.» 

Mme Bonacieux regarda le jeune homme, retenue par une dernière hésitation. Mais il y avait une telle ardeur dans ses yeux, une telle persuasion dans sa voix, qu'elle se sentit entraînée à se fier à lui. D'ailleurs elle se trouvait dans une de ces circonstances où il faut risquer le tout pour le tout. La reine était aussi bien perdue par une trop grande retenue que par une trop grande confiance. Puis, avouons-le, le sentiment involontaire qu'elle éprouvait pour ce jeune protecteur la décida à parler. 

«Écoutez, lui dit-elle, je me rends à vos protestations et je cède à vos assurances. Mais je vous jure devant Dieu qui nous entend, que si vous me trahissez et que mes ennemis me pardonnent, je me tuerai en vous accusant de ma mort. 

\speak  Et moi, je vous jure devant Dieu, madame, dit d'Artagnan, que si je suis pris en accomplissant les ordres que vous me donnez, je mourrai avant de rien faire ou dire qui compromette quelqu'un.» 

Alors la jeune femme lui confia le terrible secret dont le hasard lui avait déjà révélé une partie en face de la Samaritaine. Ce fut leur mutuelle déclaration d'amour. 

D'Artagnan rayonnait de joie et d'orgueil. Ce secret qu'il possédait, cette femme qu'il aimait, la confiance et l'amour, faisaient de lui un géant. 

«Je pars, dit-il, je pars sur-le-champ. 

\speak  Comment! vous partez! s'écria Mme Bonacieux, et votre régiment, votre capitaine? 

\speak  Sur mon âme, vous m'aviez fait oublier tout cela, chère Constance! oui, vous avez raison, il me faut un congé. 

\speak  Encore un obstacle, murmura Mme Bonacieux avec douleur. 

\speak  Oh! celui-là, s'écria d'Artagnan après un moment de réflexion, je le surmonterai, soyez tranquille. 

\speak  Comment cela? 

\speak  J'irai trouver ce soir même M. de Tréville, que je chargerai de demander pour moi cette faveur à son beau-frère, M. des Essarts. 

\speak  Maintenant, autre chose. 

\speak  Quoi? demanda d'Artagnan, voyant que Mme Bonacieux hésitait à continuer. 

\speak  Vous n'avez peut-être pas d'argent? 

\speak  Peut-être est de trop, dit d'Artagnan en souriant. 

\speak  Alors, reprit Mme Bonacieux en ouvrant une armoire et en tirant de cette armoire le sac qu'une demi-heure auparavant caressait si amoureusement son mari, prenez ce sac. 

\speak  Celui du cardinal! s'écria en éclatant de rire d'Artagnan qui, comme on s'en souvient, grâce à ses carreaux enlevés, n'avait pas perdu une syllabe de la conversation du mercier et de sa femme. 

\speak  Celui du cardinal, répondit Mme Bonacieux; vous voyez qu'il se présente sous un aspect assez respectable. 

\speak  Pardieu! s'écria d'Artagnan, ce sera une chose doublement divertissante que de sauver la reine avec l'argent de Son Éminence! 

\speak  Vous êtes un aimable et charmant jeune homme, dit Mme Bonacieux. Croyez que Sa Majesté ne sera point ingrate. 

\speak  Oh! je suis déjà grandement récompensé! s'écria d'Artagnan. Je vous aime, vous me permettez de vous le dire; c'est déjà plus de bonheur que je n'en osais espérer. 

\speak  Silence! dit Mme Bonacieux en tressaillant. 

\speak  Quoi? 

\speak  On parle dans la rue. 

\speak  C'est la voix\dots 

\speak  De mon mari. Oui, je l'ai reconnue!» 

D'Artagnan courut à la porte et poussa le verrou. 

«Il n'entrera pas que je ne sois parti, dit-il, et quand je serai parti, vous lui ouvrirez. 

\speak  Mais je devrais être partie aussi, moi. Et la disparition de cet argent, comment la justifier si je suis là? 

\speak  Vous avez raison, il faut sortir. 

\speak  Sortir, comment? On nous verra si nous sortons. 

\speak  Alors il faut monter chez moi. 

\speak  Ah! s'écria Mme Bonacieux, vous me dites cela d'un ton qui me fait peur.» 

Mme Bonacieux prononça ces paroles avec une larme dans les yeux. D'Artagnan vit cette larme, et, troublé, attendri, il se jeta à ses genoux. 

«Chez moi, dit-il, vous serez en sûreté comme dans un temple, je vous en donne ma parole de gentilhomme. 

\speak  Partons, dit-elle, je me fie à vous, mon ami.» 

D'Artagnan rouvrit avec précaution le verrou, et tous deux, légers comme des ombres, se glissèrent par la porte intérieure dans l'allée, montèrent sans bruit l'escalier et rentrèrent dans la chambre de d'Artagnan. 

Une fois chez lui, pour plus de sûreté, le jeune homme barricada la porte; ils s'approchèrent tous deux de la fenêtre, et par une fente du volet ils virent M. Bonacieux qui causait avec un homme en manteau. 

À la vue de l'homme en manteau, d'Artagnan bondit, et, tirant son épée à demi, s'élança vers la porte. 

C'était l'homme de Meung. 

«Qu'allez-vous faire? s'écria Mme Bonacieux; vous nous perdez. 

\speak  Mais j'ai juré de tuer cet homme! dit d'Artagnan. 

\speak  Votre vie est vouée en ce moment et ne vous appartient pas. Au nom de la reine, je vous défends de vous jeter dans aucun péril étranger à celui du voyage. 

\speak  Et en votre nom, n'ordonnez-vous rien? 

\speak  En mon nom, dit Mme Bonacieux avec une vive émotion; en mon nom, je vous en prie. Mais écoutons, il me semble qu'ils parlent de moi.» 

D'Artagnan se rapprocha de la fenêtre et prêta l'oreille. 

M. Bonacieux avait rouvert sa porte, et voyant l'appartement vide, il était revenu à l'homme au manteau qu'un instant il avait laissé seul. 

«Elle est partie, dit-il, elle sera retournée au Louvre. 

\speak  Vous êtes sûr, répondit l'étranger, qu'elle ne s'est pas doutée dans quelles intentions vous êtes sorti? 

\speak  Non, répondit Bonacieux avec suffisance; c'est une femme trop superficielle. 

\speak  Le cadet aux gardes est-il chez lui? 

\speak  Je ne le crois pas; comme vous le voyez, son volet est fermé, et l'on ne voit aucune lumière briller à travers les fentes. 

\speak  C'est égal, il faudrait s'en assurer. 

\speak  Comment cela? 

\speak  En allant frapper à sa porte. 

\speak  Je demanderai à son valet. 

\speak  Allez.» 

Bonacieux rentra chez lui, passa par la même porte qui venait de donner passage aux deux fugitifs, monta jusqu'au palier de d'Artagnan et frappa. 

Personne ne répondit. Porthos, pour faire plus grande figure, avait emprunté ce soir-là Planchet. Quant à d'Artagnan, il n'avait garde de donner signe d'existence. 

Au moment où le doigt de Bonacieux résonna sur la porte, les deux jeunes gens sentirent bondir leurs cœurs. 

«Il n'y a personne chez lui, dit Bonacieux. 

\speak  N'importe, rentrons toujours chez vous, nous serons plus en sûreté que sur le seuil d'une porte. 

\speak  Ah! mon Dieu! murmura Mme Bonacieux, nous n'allons plus rien entendre. 

\speak  Au contraire, dit d'Artagnan, nous n'entendrons que mieux.» 

D'Artagnan enleva les trois ou quatre carreaux qui faisaient de sa chambre une autre oreille de Denys, étendit un tapis à terre, se mit à genoux, et fit signe à Mme Bonacieux de se pencher, comme il le faisait vers l'ouverture. 

«Vous êtes sûr qu'il n'y a personne? dit l'inconnu. 

\speak  J'en réponds, dit Bonacieux. 

\speak  Et vous pensez que votre femme?\dots 

\speak  Est retournée au Louvre. 

\speak  Sans parler à aucune personne qu'à vous? 

\speak  J'en suis sûr. 

\speak  C'est un point important, comprenez-vous? 

\speak  Ainsi, la nouvelle que je vous ai apportée a donc une valeur\dots? 

\speak  Très grande, mon cher Bonacieux, je ne vous le cache pas. 

\speak  Alors le cardinal sera content de moi? 

\speak  Je n'en doute pas. 

\speak  Le grand cardinal! 

\speak  Vous êtes sûr que, dans sa conversation avec vous, votre femme n'a pas prononcé de noms propres? 

\speak  Je ne crois pas. 

\speak  Elle n'a nommé ni Mme de Chevreuse, ni M. de Buckingham, ni Mme de Vernet? 

\speak  Non, elle m'a dit seulement qu'elle voulait m'envoyer à Londres pour servir les intérêts d'une personne illustre.» 

«Le traître! murmura Mme Bonacieux. 

\speak  Silence!» dit d'Artagnan en lui prenant une main qu'elle lui abandonna sans y penser. 

«N'importe, continua l'homme au manteau, vous êtes un niais de n'avoir pas feint d'accepter la commission, vous auriez la lettre à présent; État qu'on menace était sauvé, et vous\dots 

\speak  Et moi? 

\speak  Eh bien, vous! le cardinal vous donnait des lettres de noblesse\dots 

\speak  Il vous l'a dit? 

\speak  Oui, je sais qu'il voulait vous faire cette surprise. 

\speak  Soyez tranquille, reprit Bonacieux; ma femme m'adore, et il est encore temps.» 

«Le niais! murmura Mme Bonacieux. 

\speak  Silence!» dit d'Artagnan en lui serrant plus fortement la main. 

«Comment est-il encore temps? reprit l'homme au manteau. 

\speak  Je retourne au Louvre, je demande Mme Bonacieux, je dis que j'ai réfléchi, je renoue l'affaire, j'obtiens la lettre, et je cours chez le cardinal. 

\speak  Eh bien, allez vite; je reviendrai bientôt savoir le résultat de votre démarche.» 

L'inconnu sortit. 

«L'infâme! dit Mme Bonacieux en adressant encore cette épithète à son mari. 

\speak  Silence!» répéta d'Artagnan en lui serrant la main plus fortement encore. 

Un hurlement terrible interrompit alors les réflexions de d'Artagnan et de Mme Bonacieux. C'était son mari, qui s'était aperçu de la disparition de son sac et qui criait au voleur. 

«Oh! mon Dieu! s'écria Mme Bonacieux, il va ameuter tout le quartier.» 

Bonacieux cria longtemps; mais comme de pareils cris, attendu leur fréquence, n'attiraient personne dans la rue des Fossoyeurs, et que d'ailleurs la maison du mercier était depuis quelque temps assez mal famée, voyant que personne ne venait, il sortit en continuant de crier, et l'on entendit sa voix qui s'éloignait dans la direction de la rue du Bac. 

«Et maintenant qu'il est parti, à votre tour de vous éloigner, dit Mme Bonacieux; du courage, mais surtout de la prudence, et songez que vous vous devez à la reine. 

\speak  À elle et à vous! s'écria d'Artagnan. Soyez tranquille, belle Constance, je reviendrai digne de sa reconnaissance; mais reviendrai-je aussi digne de votre amour?» 

La jeune femme ne répondit que par la vive rougeur qui colora ses joues. Quelques instants après, d'Artagnan sortit à son tour, enveloppé, lui aussi, d'un grand manteau que retroussait cavalièrement le fourreau d'une longue épée. 

Mme Bonacieux le suivit des yeux avec ce long regard d'amour dont la femme accompagne l'homme qu'elle se sent aimer; mais lorsqu'il eut disparu à l'angle de la rue, elle tomba à genoux, et joignant les mains: 

«O mon Dieu! s'écria-t-elle, protégez la reine, protégez-moi!»