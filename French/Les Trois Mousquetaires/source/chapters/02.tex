%!TeX root=../musketeersfr.tex 

\chapter{L'Antichambre de M. de Tréville}

\lettrine{M}{.} de Troisvilles, comme s'appelait encore sa famille en Gascogne, ou M. de Tréville, comme il avait fini par s'appeler lui-même à Paris, avait réellement commencé comme d'Artagnan, c'est-à-dire sans un sou vaillant, mais avec ce fonds d'audace, d'esprit et d'entendement qui fait que le plus pauvre gentillâtre gascon reçoit souvent plus en ses espérances de l'héritage paternel que le plus riche gentilhomme périgourdin ou berrichon ne reçoit en réalité. Sa bravoure insolente, son bonheur plus insolent encore dans un temps où les coups pleuvaient comme grêle, l'avaient hissé au sommet de cette échelle difficile qu'on appelle la faveur de cour, et dont il avait escaladé quatre à quatre les échelons. 

Il était l'ami du roi, lequel honorait fort, comme chacun sait, la mémoire de son père Henri IV. Le père de M. de Tréville l'avait si fidèlement servi dans ses guerres contre la Ligue, qu'à défaut d'argent comptant --- chose qui toute la vie manqua au Béarnais, lequel paya constamment ses dettes avec la seule chose qu'il n'eût jamais besoin d'emprunter, c'est-à-dire avec de l'esprit ---, qu'à défaut d'argent comptant, disons-nous, il l'avait autorisé, après la reddition de Paris, à prendre pour armes un lion d'or passant sur gueules avec cette devise: \textit{fidelis et fortis}. C'était beaucoup pour l'honneur, mais c'était médiocre pour le bien-être. Aussi, quand l'illustre compagnon du grand Henri mourut, il laissa pour seul héritage à monsieur son fils son épée et sa devise. Grâce à ce double don et au nom sans tache qui l'accompagnait, M. de Tréville fut admis dans la maison du jeune prince, où il servit si bien de son épée et fut si fidèle à sa devise, que Louis XIII, une des bonnes lames du royaume, avait l'habitude de dire que, s'il avait un ami qui se battît, il lui donnerait le conseil de prendre pour second, lui d'abord, et Tréville après, et peut-être même avant lui. 

Aussi Louis XIII avait-il un attachement réel pour Tréville, attachement royal, attachement égoïste, c'est vrai, mais qui n'en était pas moins un attachement. C'est que, dans ces temps malheureux, on cherchait fort à s'entourer d'hommes de la trempe de Tréville. Beaucoup pouvaient prendre pour devise l'épithète de \textit{fort}, qui faisait la seconde partie de son exergue; mais peu de gentilshommes pouvaient réclamer l'épithète de \textit{fidèle}, qui en formait la première. Tréville était un de ces derniers; c'était une de ces rares organisations, à l'intelligence obéissante comme celle du dogue, à la valeur aveugle, à l'œil rapide, à la main prompte, à qui l'œil n'avait été donné que pour voir si le roi était mécontent de quelqu'un et la main que pour frapper ce déplaisant quelqu'un, un Besme, un Maurevers, un Poltrot de Méré, un Vitry. Enfin à Tréville, il n'avait manqué jusque-là que l'occasion; mais il la guettait, et il se promettait bien de la saisir par ses trois cheveux si jamais elle passait à la portée de sa main. Aussi Louis XIII fit-il de Tréville le capitaine de ses mousquetaires, lesquels étaient à Louis XIII, pour le dévouement ou plutôt pour le fanatisme, ce que ses ordinaires étaient à Henri III et ce que sa garde écossaise était à Louis XI. 

De son côté, et sous ce rapport, le cardinal n'était pas en reste avec le roi. Quand il avait vu la formidable élite dont Louis XIII s'entourait, ce second ou plutôt ce premier roi de France avait voulu, lui aussi, avoir sa garde. Il eut donc ses mousquetaires comme Louis XIII avait les siens et l'on voyait ces deux puissances rivales trier pour leur service, dans toutes les provinces de France et même dans tous les États étrangers, les hommes célèbres pour les grands coups d'épée. Aussi Richelieu et Louis XIII se disputaient souvent, en faisant leur partie d'échecs, le soir, au sujet du mérite de leurs serviteurs. Chacun vantait la tenue et le courage des siens, et tout en se prononçant tout haut contre les duels et contre les rixes, ils les excitaient tout bas à en venir aux mains, et concevaient un véritable chagrin ou une joie immodérée de la défaite ou de la victoire des leurs. Ainsi, du moins, le disent les mémoires d'un homme qui fut dans quelques-unes de ces défaites et dans beaucoup de ces victoires. 

Tréville avait pris le côté faible de son maître, et c'est à cette adresse qu'il devait la longue et constante faveur d'un roi qui n'a pas laissé la réputation d'avoir été très fidèle à ses amitiés. Il faisait parader ses mousquetaires devant le cardinal Armand Duplessis avec un air narquois qui hérissait de colère la moustache grise de Son Éminence. Tréville entendait admirablement bien la guerre de cette époque, où, quand on ne vivait pas aux dépens de l'ennemi, on vivait aux dépens de ses compatriotes: ses soldats formaient une légion de diables à quatre, indisciplinée pour tout autre que pour lui. 

Débraillés, avinés, écorchés, les mousquetaires du roi, ou plutôt ceux de M. de Tréville, s'épandaient dans les cabarets, dans les promenades, dans les jeux publics, criant fort et retroussant leurs moustaches, faisant sonner leurs épées, heurtant avec volupté les gardes de M. le cardinal quand ils les rencontraient; puis dégainant en pleine rue, avec mille plaisanteries; tués quelquefois, mais sûrs en ce cas d'être pleurés et vengés; tuant souvent, et sûrs alors de ne pas moisir en prison, M. de Tréville étant là pour les réclamer. Aussi M. de Tréville était-il loué sur tous les tons, chanté sur toutes les gammes par ces hommes qui l'adoraient, et qui, tout gens de sac et de corde qu'ils étaient, tremblaient devant lui comme des écoliers devant leur maître, obéissant au moindre mot, et prêts à se faire tuer pour laver le moindre reproche. 

M. de Tréville avait usé de ce levier puissant, pour le roi d'abord et les amis du roi, --- puis pour lui-même et pour ses amis. Au reste, dans aucun des mémoires de ce temps, qui a laissé tant de mémoires, on ne voit que ce digne gentilhomme ait été accusé, même par ses ennemis --- et il en avait autant parmi les gens de plume que chez les gens d'épée ---, nulle part on ne voit, disons-nous, que ce digne gentilhomme ait été accusé de se faire payer la coopération de ses séides. Avec un rare génie d'intrigue, qui le rendait l'égal des plus forts intrigants, il était resté honnête homme. Bien plus, en dépit des grandes estocades qui déhanchent et des exercices pénibles qui fatiguent, il était devenu un des plus galants coureurs de ruelles, un des plus fins damerets, un des plus alambiqués diseurs de Phébus de son époque; on parlait des bonnes fortunes de Tréville comme on avait parlé vingt ans auparavant de celles de Bassompierre --- et ce n'était pas peu dire. Le capitaine des mousquetaires était donc admiré, craint et aimé, ce qui constitue l'apogée des fortunes humaines. 

Louis XIV absorba tous les petits astres de sa cour dans son vaste rayonnement; mais son père, soleil \textit{pluribus impar}, laissa sa splendeur personnelle à chacun de ses favoris, sa valeur individuelle à chacun de ses courtisans. Outre le lever du roi et celui du cardinal, on comptait alors à Paris plus de deux cents petits levers, un peu recherchés. Parmi les deux cents petits levers celui de Tréville était un des plus courus. 

La cour de son hôtel, situé rue du Vieux-Colombier, ressemblait à un camp, et cela dès six heures du matin en été et dès huit heures en hiver. Cinquante à soixante mousquetaires, qui semblaient s'y relayer pour présenter un nombre toujours imposant, s'y promenaient sans cesse, armés en guerre et prêts à tout. Le long d'un de ses grands escaliers sur l'emplacement desquels notre civilisation bâtirait une maison tout entière, montaient et descendaient les solliciteurs de Paris qui couraient après une faveur quelconque, les gentilshommes de province avides d'être enrôlés, et les laquais chamarrés de toutes couleurs, qui venaient apporter à M. de Tréville les messages de leurs maîtres. Dans l'antichambre, sur de longues banquettes circulaires, reposaient les élus, c'est-à-dire ceux qui étaient convoqués. Un bourdonnement durait là depuis le matin jusqu'au soir, tandis que M. de Tréville, dans son cabinet contigu à cette antichambre, recevait les visites, écoutait les plaintes, donnait ses ordres et, comme le roi à son balcon du Louvre, n'avait qu'à se mettre à sa fenêtre pour passer la revue des hommes et des armes. 

Le jour où d'Artagnan se présenta, l'assemblée était imposante, surtout pour un provincial arrivant de sa province: il est vrai que ce provincial était Gascon, et que surtout à cette époque les compatriotes de d'Artagnan avaient la réputation de ne point facilement se laisser intimider. En effet, une fois qu'on avait franchi la porte massive, chevillée de longs clous à tête quadrangulaire, on tombait au milieu d'une troupe de gens d'épée qui se croisaient dans la cour, s'interpellant, se querellant et jouant entre eux. Pour se frayer un passage au milieu de toutes ces vagues tourbillonnantes, il eût fallu être officier, grand seigneur ou jolie femme. 

Ce fut donc au milieu de cette cohue et de ce désordre que notre jeune homme s'avança, le cœur palpitant, rangeant sa longue rapière le long de ses jambes maigres, et tenant une main au rebord de son feutre avec ce demi-sourire du provincial embarrassé qui veut faire bonne contenance. Avait-il dépassé un groupe, alors il respirait plus librement, mais il comprenait qu'on se retournait pour le regarder, et pour la première fois de sa vie, d'Artagnan, qui jusqu'à ce jour avait une assez bonne opinion de lui-même, se trouva ridicule. 

Arrivé à l'escalier, ce fut pis encore: il y avait sur les premières marches quatre mousquetaires qui se divertissaient à l'exercice suivant, tandis que dix ou douze de leurs camarades attendaient sur le palier que leur tour vînt de prendre place à la partie. 

Un d'eux, placé sur le degré supérieur, l'épée nue à la main, empêchait ou du moins s'efforçait d'empêcher les trois autres de monter. 

Ces trois autres s'escrimaient contre lui de leurs épées fort agiles. D'Artagnan prit d'abord ces fers pour des fleurets d'escrime, il les crut boutonnés: mais il reconnut bientôt à certaines égratignures que chaque arme, au contraire, était affilée et aiguisée à souhait, et à chacune de ces égratignures, non seulement les spectateurs, mais encore les acteurs riaient comme des fous. 

Celui qui occupait le degré en ce moment tenait merveilleusement ses adversaires en respect. On faisait cercle autour d'eux: la condition portait qu'à chaque coup le touché quitterait la partie, en perdant son tour d'audience au profit du toucheur. En cinq minutes trois furent effleurés, l'un au poignet, l'autre au menton, l'autre à l'oreille par le défenseur du degré, qui lui-même ne fut pas atteint: adresse qui lui valut, selon les conventions arrêtées, trois tours de faveur. 

Si difficile non pas qu'il fût, mais qu'il voulût être à étonner, ce passe-temps étonna notre jeune voyageur; il avait vu dans sa province, cette terre où s'échauffent cependant si promptement les têtes, un peu plus de préliminaires aux duels, et la gasconnade de ces quatre joueurs lui parut la plus forte de toutes celles qu'il avait ouïes jusqu'alors, même en Gascogne. Il se crut transporté dans ce fameux pays des géants où Gulliver alla depuis et eut si grand-peur; et cependant il n'était pas au bout: restaient le palier et l'antichambre. 

Sur le palier on ne se battait plus, on racontait des histoires de femmes, et dans l'antichambre des histoires de cour. Sur le palier, d'Artagnan rougit; dans l'antichambre, il frissonna. Son imagination éveillée et vagabonde, qui en Gascogne le rendait redoutable aux jeunes femmes de chambre et même quelquefois aux jeunes maîtresses, n'avait jamais rêvé, même dans ces moments de délire, la moitié de ces merveilles amoureuses et le quart de ces prouesses galantes, rehaussées des noms les plus connus et des détails les moins voilés. Mais si son amour pour les bonnes mœurs fut choqué sur le palier, son respect pour le cardinal fut scandalisé dans l'antichambre. Là, à son grand étonnement, d'Artagnan entendait critiquer tout haut la politique qui faisait trembler l'Europe, et la vie privée du cardinal, que tant de hauts et puissants seigneurs avaient été punis d'avoir tenté d'approfondir: ce grand homme, révéré par M. d'Artagnan père, servait de risée aux mousquetaires de M. de Tréville, qui raillaient ses jambes cagneuses et son dos voûté; quelques-uns chantaient des Noëls sur Mme d'Aiguillon, sa maîtresse, et Mme de Combalet, sa nièce, tandis que les autres liaient des parties contre les pages et les gardes du cardinal-duc, toutes choses qui paraissaient à d'Artagnan de monstrueuses impossibilités. 

Cependant, quand le nom du roi intervenait parfois tout à coup à l'improviste au milieu de tous ces quolibets cardinalesques, une espèce de bâillon calfeutrait pour un moment toutes ces bouches moqueuses; on regardait avec hésitation autour de soi, et l'on semblait craindre l'indiscrétion de la cloison du cabinet de M. de Tréville; mais bientôt une allusion ramenait la conversation sur Son Éminence, et alors les éclats reprenaient de plus belle, et la lumière n'était ménagée sur aucune de ses actions. 

«Certes, voilà des gens qui vont être embastillés et pendus, pensa d'Artagnan avec terreur, et moi sans aucun doute avec eux, car du moment où je les ai écoutés et entendus, je serai tenu pour leur complice. Que dirait monsieur mon père, qui m'a si fort recommandé le respect du cardinal, s'il me savait dans la société de pareils païens?» 

Aussi comme on s'en doute sans que je le dise, d'Artagnan n'osait se livrer à la conversation; seulement il regardait de tous ses yeux, écoutant de toutes ses oreilles, tendant avidement ses cinq sens pour ne rien perdre, et malgré sa confiance dans les recommandations paternelles, il se sentait porté par ses goûts et entraîné par ses instincts à louer plutôt qu'à blâmer les choses inouïes qui se passaient là. 

Cependant, comme il était absolument étranger à la foule des courtisans de M. de Tréville, et que c'était la première fois qu'on l'apercevait en ce lieu, on vint lui demander ce qu'il désirait. À cette demande, d'Artagnan se nomma fort humblement, s'appuya du titre de compatriote, et pria le valet de chambre qui était venu lui faire cette question de demander pour lui à M. de Tréville un moment d'audience, demande que celui-ci promit d'un ton protecteur de transmettre en temps et lieu. 

D'Artagnan, un peu revenu de sa surprise première, eut donc le loisir d'étudier un peu les costumes et les physionomies. 

Au centre du groupe le plus animé était un mousquetaire de grande taille, d'une figure hautaine et d'une bizarrerie de costume qui attirait sur lui l'attention générale. Il ne portait pas, pour le moment, la casaque d'uniforme, qui, au reste, n'était pas absolument obligatoire dans cette époque de liberté moindre mais d'indépendance plus grande, mais un justaucorps bleu de ciel, tant soit peu fané et râpé, et sur cet habit un baudrier magnifique, en broderies d'or, et qui reluisait comme les écailles dont l'eau se couvre au grand soleil. Un manteau long de velours cramoisi tombait avec grâce sur ses épaules découvrant par-devant seulement le splendide baudrier auquel pendait une gigantesque rapière. 

Ce mousquetaire venait de descendre de garde à l'instant même, se plaignait d'être enrhumé et toussait de temps en temps avec affectation. Aussi avait-il pris le manteau, à ce qu'il disait autour de lui, et tandis qu'il parlait du haut de sa tête, en frisant dédaigneusement sa moustache, on admirait avec enthousiasme le baudrier brodé, et d'Artagnan plus que tout autre. 

«Que voulez-vous, disait le mousquetaire, la mode en vient; c'est une folie, je le sais bien, mais c'est la mode. D'ailleurs, il faut bien employer à quelque chose l'argent de sa légitime. 

\speak  Ah! \textit{Porthos!} s'écria un des assistants, n'essaie pas de nous faire croire que ce baudrier te vient de la générosité paternelle: il t'aura été donné par la dame voilée avec laquelle je t'ai rencontré l'autre dimanche vers la porte Saint-Honoré. 

\speak  Non, sur mon honneur et foi de gentilhomme, je l'ai acheté moi-même, et de mes propres deniers, répondit celui qu'on venait de désigner sous le nom de Porthos. 

\speak  Oui, comme j'ai acheté, moi, dit un autre mousquetaire, cette bourse neuve, avec ce que ma maîtresse avait mis dans la vieille. 

\speak  Vrai, dit Porthos, et la preuve c'est que je l'ai payé douze pistoles.» 

L'admiration redoubla, quoique le doute continuât d'exister. 

«N'est-ce pas, \textit{Aramis?}» dit Porthos se tournant vers un autre mousquetaire. 

Cet autre mousquetaire formait un contraste parfait avec celui qui l'interrogeait et qui venait de le désigner sous le nom d'Aramis: c'était un jeune homme de vingt-deux à vingt-trois ans à peine, à la figure naïve et doucereuse, à l'œil noir et doux et aux joues roses et veloutées comme une pêche en automne; sa moustache fine dessinait sur sa lèvre supérieure une ligne d'une rectitude parfaite; ses mains semblaient craindre de s'abaisser, de peur que leurs veines ne se gonflassent, et de temps en temps il se pinçait le bout des oreilles pour les maintenir d'un incarnat tendre et transparent. D'habitude il parlait peu et lentement, saluait beaucoup, riait sans bruit en montrant ses dents, qu'il avait belles et dont, comme du reste de sa personne, il semblait prendre le plus grand soin. Il répondit par un signe de tête affirmatif à l'interpellation de son ami. 

Cette affirmation parut avoir fixé tous les doutes à l'endroit du baudrier; on continua donc de l'admirer, mais on n'en parla plus; et par un de ces revirements rapides de la pensée, la conversation passa tout à coup à un autre sujet. 

«Que pensez-vous de ce que raconte l'écuyer de Chalais?» demanda un autre mousquetaire sans interpeller directement personne, mais s'adressant au contraire à tout le monde. 

«Et que raconte-t-il? demanda Porthos d'un ton suffisant. 

\speak  Il raconte qu'il a trouvé à Bruxelles Rochefort, l'âme damnée du cardinal, déguisé en capucin; ce Rochefort maudit, grâce à ce déguisement, avait joué M. de Laigues comme un niais qu'il est. 

\speak  Comme un vrai niais, dit Porthos; mais la chose est-elle sûre? 

\speak  Je la tiens d'Aramis, répondit le mousquetaire. 

\speak  Vraiment? 

\speak  Eh! vous le savez bien, Porthos, dit Aramis; je vous l'ai racontée à vous-même hier, n'en parlons donc plus. 

\speak  N'en parlons plus, voilà votre opinion à vous, reprit Porthos. N'en parlons plus! peste! comme vous concluez vite. Comment! le cardinal fait espionner un gentilhomme, fait voler sa correspondance par un traître, un brigand, un pendard; fait, avec l'aide de cet espion et grâce à cette correspondance, couper le cou à Chalais, sous le stupide prétexte qu'il a voulu tuer le roi et marier Monsieur avec la reine! Personne ne savait un mot de cette énigme, vous nous l'apprenez hier, à la grande satisfaction de tous, et quand nous sommes encore tout ébahis de cette nouvelle, vous venez nous dire aujourd'hui: N'en parlons plus! 

\speak  Parlons-en donc, voyons, puisque vous le désirez, reprit Aramis avec patience. 

\speak  Ce Rochefort, s'écria Porthos, si j'étais l'écuyer du pauvre Chalais, passerait avec moi un vilain moment. 

\speak  Et vous, vous passeriez un triste quart d'heure avec le duc Rouge, reprit Aramis. 

\speak  Ah! le duc Rouge! bravo, bravo, le duc Rouge! répondit Porthos en battant des mains et en approuvant de la tête. Le «duc Rouge» est charmant. Je répandrai le mot, mon cher, soyez tranquille. A-t-il de l'esprit, cet Aramis! Quel malheur que vous n'ayez pas pu suivre votre vocation, mon cher! quel délicieux abbé vous eussiez fait! 

\speak  Oh! ce n'est qu'un retard momentané, reprit Aramis; un jour, je le serai. Vous savez bien, Porthos, que je continue d'étudier la théologie pour cela. 

\speak  Il le fera comme il le dit, reprit Porthos, il le fera tôt ou tard. 

\speak  Tôt, dit Aramis. 

\speak  Il n'attend qu'une chose pour le décider tout à fait et pour reprendre sa soutane, qui est pendue derrière son uniforme, reprit un mousquetaire. 

\speak  Et quelle chose attend-il? demanda un autre. 

\speak  Il attend que la reine ait donné un héritier à la couronne de France. 

\speak  Ne plaisantons pas là-dessus, messieurs, dit Porthos; grâce à Dieu, la reine est encore d'âge à le donner. 

\speak  On dit que M. de Buckingham est en France, reprit Aramis avec un rire narquois qui donnait à cette phrase, si simple en apparence, une signification passablement scandaleuse. 

\speak  Aramis, mon ami, pour cette fois vous avez tort, interrompit Porthos, et votre manie d'esprit vous entraîne toujours au-delà des bornes; si M. de Tréville vous entendait, vous seriez mal venu de parler ainsi. 

\speak  Allez-vous me faire la leçon, Porthos? s'écria Aramis, dans l'œil doux duquel on vit passer comme un éclair. 

\speak  Mon cher, soyez mousquetaire ou abbé. Soyez l'un ou l'autre, mais pas l'un et l'autre, reprit Porthos. Tenez, Athos vous l'a dit encore l'autre jour: vous mangez à tous les râteliers. Ah! ne nous fâchons pas, je vous prie, ce serait inutile, vous savez bien ce qui est convenu entre vous, Athos et moi. Vous allez chez Mme d'Aiguillon, et vous lui faites la cour; vous allez chez Mme de Bois-Tracy, la cousine de Mme de Chevreuse, et vous passez pour être fort en avant dans les bonnes grâces de la dame. Oh! mon Dieu, n'avouez pas votre bonheur, on ne vous demande pas votre secret, on connaît votre discrétion. Mais puisque vous possédez cette vertu, que diable! Faites-en usage à l'endroit de Sa Majesté. S'occupe qui voudra et comme on voudra du roi et du cardinal; mais la reine est sacrée, et si l'on en parle, que ce soit en bien. 

\speak  Porthos, vous êtes prétentieux comme Narcisse, je vous en préviens, répondit Aramis; vous savez que je hais la morale, excepté quand elle est faite par Athos. Quant à vous, mon cher, vous avez un trop magnifique baudrier pour être bien fort là-dessus. Je serai abbé s'il me convient; en attendant, je suis mousquetaire: en cette qualité, je dis ce qu'il me plaît, et en ce moment il me plaît de vous dire que vous m'impatientez. 

\speak  Aramis! 

\speak  Porthos! 

\speak  Eh! messieurs! messieurs! s'écria-t-on autour d'eux. 

\speak  M. de Tréville attend M. d'Artagnan», interrompit le laquais en ouvrant la porte du cabinet. 

À cette annonce, pendant laquelle la porte demeurait ouverte, chacun se tut, et au milieu du silence général le jeune Gascon traversa l'antichambre dans une partie de sa longueur et entra chez le capitaine des mousquetaires, se félicitant de tout son cœur d'échapper aussi à point à la fin de cette bizarre querelle.