%!TeX root=../musketeersfr.tex 

\chapter{D'Artagnan Se Dessine}

\lettrine{C}{omme} l'avaient prévu Athos et Porthos, au bout d'une demi-heure d'Artagnan rentra. Cette fois encore il avait manqué son homme, qui avait disparu comme par enchantement. D'Artagnan avait couru, l'épée à la main, toutes les rues environnantes, mais il n'avait rien trouvé qui ressemblât à celui qu'il cherchait, puis enfin il en était revenu à la chose par laquelle il aurait dû commencer peut-être, et qui était de frapper à la porte contre laquelle l'inconnu était appuyé; mais c'était inutilement qu'il avait dix ou douze fois de suite fait résonner le marteau, personne n'avait répondu, et des voisins qui, attirés par le bruit, étaient accourus sur le seuil de leur porte ou avaient mis le nez à leurs fenêtres, lui avaient assuré que cette maison, dont au reste toutes les ouvertures étaient closes, était depuis six mois complètement inhabitée. 

Pendant que d'Artagnan courait les rues et frappait aux portes, Aramis avait rejoint ses deux compagnons, de sorte qu'en revenant chez lui, d'Artagnan trouva la réunion au grand complet. 

«Eh bien? dirent ensemble les trois mousquetaires en voyant entrer d'Artagnan, la sueur sur le front et la figure bouleversée par la colère. 

\speak  Eh bien, s'écria celui-ci en jetant son épée sur le lit, il faut que cet homme soit le diable en personne; il a disparu comme un fantôme, comme une ombre, comme un spectre. 

\speak  Croyez-vous aux apparitions? demanda Athos à Porthos. 

\speak  Moi, je ne crois que ce que j'ai vu, et comme je n'ai jamais vu d'apparitions, je n'y crois pas. 

\speak  La Bible, dit Aramis, nous fait une loi d'y croire: l'ombre de Samuel apparut à Saül, et c'est un article de foi que je serais fâché de voir mettre en doute, Porthos. 

\speak  Dans tous les cas, homme ou diable, corps ou ombre, illusion ou réalité, cet homme est né pour ma damnation, car sa fuite nous fait manquer une affaire superbe, messieurs, une affaire dans laquelle il y avait cent pistoles et peut-être plus à gagner. 

\speak  Comment cela?» dirent à la fois Porthos et Aramis. 

Quant à Athos, fidèle à son système de mutisme, il se contenta d'interroger d'Artagnan du regard. 

«Planchet, dit d'Artagnan à son domestique, qui passait en ce moment la tête par la porte entrebâillée pour tâcher de surprendre quelques bribes de la conversation, descendez chez mon propriétaire, M. Bonacieux, et dites-lui de nous envoyer une demi-douzaine de bouteilles de vin de Beaugency: c'est celui que je préfère. 

\speak  Ah çà, mais vous avez donc crédit ouvert chez votre propriétaire? demanda Porthos. 

\speak  Oui, répondit d'Artagnan, à compter d'aujourd'hui, et soyez tranquilles, si son vin est mauvais, nous lui en enverrons quérir d'autre. 

\speak  Il faut user et non abuser, dit sentencieusement Aramis. 

\speak  J'ai toujours dit que d'Artagnan était la forte tête de nous quatre, fit Athos, qui, après avoir émis cette opinion à laquelle d'Artagnan répondit par un salut, retomba aussitôt dans son silence accoutumé. 

\speak  Mais enfin, voyons, qu'y a-t-il? demanda Porthos. 

\speak  Oui, dit Aramis, confiez-nous cela, mon cher ami, à moins que l'honneur de quelque dame ne se trouve intéressé à cette confidence, à ce quel cas vous feriez mieux de la garder pour vous. 

\speak  Soyez tranquilles, répondit d'Artagnan, l'honneur de personne n'aura à se plaindre de ce que j'ai à vous dire.» 

Et alors il raconta mot à mot à ses amis ce qui venait de se passer entre lui et son hôte, et comment l'homme qui avait enlevé la femme du digne propriétaire était le même avec lequel il avait eu maille à partir à l'hôtellerie du Franc Meunier. 

«Votre affaire n'est pas mauvaise, dit Athos après avoir goûté le vin en connaisseur et indiqué d'un signe de tête qu'il le trouvait bon, et l'on pourra tirer de ce brave homme cinquante à soixante pistoles. Maintenant, reste à savoir si cinquante à soixante pistoles valent la peine de risquer quatre têtes. 

\speak  Mais faites attention, s'écria d'Artagnan qu'il y a une femme dans cette affaire, une femme enlevée, une femme qu'on menace sans doute, qu'on torture peut-être, et tout cela parce qu'elle est fidèle à sa maîtresse! 

\speak  Prenez garde, d'Artagnan, prenez garde, dit Aramis, vous vous échauffez un peu trop, à mon avis, sur le sort de Mme Bonacieux. La femme a été créée pour notre perte, et c'est d'elle que nous viennent toutes nos misères.» 

Athos, à cette sentence d'Aramis, fronça le sourcil et se mordit les lèvres. 

«Ce n'est point de Mme Bonacieux que je m'inquiète, s'écria d'Artagnan, mais de la reine, que le roi abandonne, que le cardinal persécute, et qui voit tomber, les unes après les autres, les têtes de tous ses amis. 

\speak  Pourquoi aime-t-elle ce que nous détestons le plus au monde, les Espagnols et les Anglais? 

\speak  L'Espagne est sa patrie, répondit d'Artagnan, et il est tout simple qu'elle aime les Espagnols, qui sont enfants de la même terre qu'elle. Quant au second reproche que vous lui faites, j'ai entendu dire qu'elle aimait non pas les Anglais, mais un Anglais. 

\speak  Eh! ma foi, dit Athos, il faut avouer que cet Anglais était bien digne d'être aimé. Je n'ai jamais vu un plus grand air que le sien. 

\speak  Sans compter qu'il s'habille comme personne, dit Porthos. J'étais au Louvre le jour où il a semé ses perles, et pardieu! j'en ai ramassé deux que j'ai bien vendues dix pistoles pièce. Et toi, Aramis, le connais-tu? 

\speak  Aussi bien que vous, messieurs, car j'étais de ceux qui l'ont arrêté dans le jardin d'Amiens, où m'avait introduit M. de Putange, l'écuyer de la reine. J'étais au séminaire à cette époque, et l'aventure me parut cruelle pour le roi. 

\speak  Ce qui ne m'empêcherait pas, dit d'Artagnan, si je savais où est le duc de Buckingham, de le prendre par la main et de le conduire près de la reine, ne fût-ce que pour faire enrager M. le cardinal; car notre véritable, notre seul, notre éternel ennemi, messieurs, c'est le cardinal, et si nous pouvions trouver moyen de lui jouer quelque tour bien cruel, j'avoue que j'y engagerais volontiers ma tête. 

\speak  Et, reprit Athos, le mercier vous a dit, d'Artagnan, que la reine pensait qu'on avait fait venir Buckingham sur un faux avis? 

\speak  Elle en a peur. 

\speak  Attendez donc, dit Aramis. 

\speak  Quoi? demanda Porthos. 

\speak  Allez toujours, je cherche à me rappeler des circonstances. 

\speak  Et maintenant je suis convaincu, dit d'Artagnan, que l'enlèvement de cette femme de la reine se rattache aux événements dont nous parlons, et peut-être à la présence de M. de Buckingham à Paris. 

\speak  Le Gascon est plein d'idées, dit Porthos avec admiration. 

\speak  J'aime beaucoup l'entendre parler, dit Athos, son patois m'amuse. 

\speak  Messieurs, reprit Aramis, écoutez ceci. 

\speak  Écoutons Aramis, dirent les trois amis. 

\speak  Hier je me trouvais chez un savant docteur en théologie que je consulte quelquefois pour mes études\dots» 

Athos sourit. 

«Il habite un quartier désert, continua Aramis: ses goûts, sa profession l'exigent. Or, au moment où je sortais de chez lui\dots» 

Ici Aramis s'arrêta. 

«Eh bien? demandèrent ses auditeurs, au moment où vous sortiez de chez lui?» 

Aramis parut faire un effort sur lui-même, comme un homme qui, en plein courant de mensonge, se voit arrêter par quelque obstacle imprévu; mais les yeux de ses trois compagnons étaient fixés sur lui, leurs oreilles attendaient béantes, il n'y avait pas moyen de reculer. 

«Ce docteur a une nièce, continua Aramis. 

\speak  Ah! il a une nièce! interrompit Porthos. 

\speak  Dame fort respectable», dit Aramis. 

Les trois amis se mirent à rire. 

«Ah! si vous riez ou si vous doutez, reprit Aramis, vous ne saurez rien. 

\speak  Nous sommes croyants comme des mahométistes et muets comme des catafalques, dit Athos. 

\speak  Je continue donc, reprit Aramis. Cette nièce vient quelquefois voir son oncle; or elle s'y trouvait hier en même temps que moi, par hasard, et je dus m'offrir pour la conduire à son carrosse. 

\speak  Ah! elle a un carrosse, la nièce du docteur? interrompit Porthos, dont un des défauts était une grande incontinence de langue; belle connaissance, mon ami. 

\speak  Porthos, reprit Aramis, je vous ai déjà fait observer plus d'une fois que vous êtes fort indiscret, et que cela vous nuit près des femmes. 

\speak  Messieurs, messieurs, s'écria d'Artagnan, qui entrevoyait le fond de l'aventure, la chose est sérieuse; tâchons donc de ne pas plaisanter si nous pouvons. Allez, Aramis, allez. 

\speak  Tout à coup, un homme grand, brun, aux manières de gentilhomme\dots, tenez, dans le genre du vôtre, d'Artagnan. 

\speak  Le même peut-être, dit celui-ci. 

\speak  C'est possible, continua Aramis,\dots s'approcha de moi, accompagné de cinq ou six hommes qui le suivaient à dix pas en arrière, et du ton le plus poli: “Monsieur le duc, me dit-il, et vous, madame”, continua-t-il en s'adressant à la dame que j'avais sous le bras\dots 

\speak  À la nièce du docteur? 

\speak  Silence donc, Porthos! dit Athos, vous êtes insupportable. 

\speak  Veuillez monter dans ce carrosse, et cela sans essayer la moindre résistance, sans faire le moindre bruit.» 

\speak  Il vous avait pris pour Buckingham! s'écria d'Artagnan. 

\speak  Je le crois, répondit Aramis. 

\speak  Mais cette dame? demanda Porthos. 

\speak  Il l'avait prise pour la reine! dit d'Artagnan. 

\speak  Justement, répondit Aramis. 

\speak  Le Gascon est le diable! s'écria Athos, rien ne lui échappe. 

\speak  Le fait est, dit Porthos, qu'Aramis est de la taille et a quelque chose de la tournure du beau duc; mais cependant, il me semble que l'habit de mousquetaire\dots 

\speak  J'avais un manteau énorme, dit Aramis. 

\speak  Au mois de juillet, diable! fit Porthos, est-ce que le docteur craint que tu ne sois reconnu? 

\speak  Je comprends encore, dit Athos, que l'espion se soit laissé prendre par la tournure; mais le visage\dots 

\speak  J'avais un grand chapeau, dit Aramis. 

\speak  Oh! mon Dieu, s'écria Porthos, que de précautions pour étudier la théologie! 

\speak  Messieurs, messieurs, dit d'Artagnan, ne perdons pas notre temps à badiner; éparpillons-nous et cherchons la femme du mercier, c'est la clef de l'intrigue. 

\speak  Une femme de condition si inférieure! vous croyez, d'Artagnan? fit Porthos en allongeant les lèvres avec mépris. 

\speak  C'est la filleule de La Porte, le valet de confiance de la reine. Ne vous l'ai-je pas dit, messieurs? Et d'ailleurs, c'est peut-être un calcul de Sa Majesté d'avoir été, cette fois, chercher ses appuis si bas. Les hautes têtes se voient de loin, et le cardinal a bonne vue. 

\speak  Eh bien, dit Porthos, faites d'abord prix avec le mercier, et bon prix. 

\speak  C'est inutile, dit d'Artagnan, car je crois que s'il ne nous paie pas, nous serons assez payés d'un autre côté.» 

En ce moment, un bruit précipité de pas retentit dans l'escalier, la porte s'ouvrit avec fracas, et le malheureux mercier s'élança dans la chambre où se tenait le conseil. 

«Ah! messieurs, s'écria-t-il, sauvez-moi, au nom du Ciel, sauvez-moi! Il y a quatre hommes qui viennent pour m'arrêter; sauvez-moi, sauvez-moi!» 

Porthos et Aramis se levèrent. 

«Un moment, s'écria d'Artagnan en leur faisant signe de repousser au fourreau leurs épées à demi tirées; un moment, ce n'est pas du courage qu'il faut ici, c'est de la prudence. 

\speak  Cependant, s'écria Porthos, nous ne laisserons pas\dots 

\speak  Vous laisserez faire d'Artagnan, dit Athos, c'est, je le répète, la forte tête de nous tous, et moi, pour mon compte, je déclare que je lui obéis. Fais ce que tu voudras, d'Artagnan.» 

En ce moment, les quatre gardes apparurent à la porte de l'antichambre, et voyant quatre mousquetaires debout et l'épée au côté, hésitèrent à aller plus loin. 

«Entrez, messieurs, entrez, cria d'Artagnan; vous êtes ici chez moi, et nous sommes tous de fidèles serviteurs du roi et de M. le cardinal. 

\speak  Alors, messieurs, vous ne vous opposerez pas à ce que nous exécutions les ordres que nous avons reçus? demanda celui qui paraissait le chef de l'escouade. 

\speak  Au contraire, messieurs, et nous vous prêterions main-forte, si besoin était. 

\speak  Mais que dit-il donc? marmotta Porthos. 

\speak  Tu es un niais, dit Athos, silence! 

\speak  Mais vous m'avez promis\dots, dit tout bas le pauvre mercier. 

\speak  Nous ne pouvons vous sauver qu'en restant libres, répondit rapidement et tout bas d'Artagnan, et si nous faisons mine de vous défendre, on nous arrête avec vous. 

\speak  Il me semble, cependant\dots 

\speak  Venez, messieurs, venez, dit tout haut d'Artagnan; je n'ai aucun motif de défendre monsieur. Je l'ai vu aujourd'hui pour la première fois, et encore à quelle occasion, il vous le dira lui-même, pour me venir réclamer le prix de mon loyer. Est-ce vrai, monsieur Bonacieux? Répondez! 

\speak  C'est la vérité pure, s'écria le mercier, mais monsieur ne vous dit pas\dots 

\speak  Silence sur moi, silence sur mes amis, silence sur la reine surtout, ou vous perdriez tout le monde sans vous sauver. Allez, allez, messieurs, emmenez cet homme!» 

Et d'Artagnan poussa le mercier tout étourdi aux mains des gardes, en lui disant: 

«Vous êtes un maraud, mon cher; vous venez me demander de l'argent, à moi! à un mousquetaire! En prison, messieurs, encore une fois, emmenez-le en prison et gardez-le sous clef le plus longtemps possible, cela me donnera du temps pour payer.» 

Les sbires se confondirent en remerciements et emmenèrent leur proie. 

Au moment où ils descendaient, d'Artagnan frappa sur l'épaule du chef: 

«Ne boirai-je pas à votre santé et vous à la mienne? dit-il, en remplissant deux verres du vin de Beaugency qu'il tenait de la libéralité de M. Bonacieux. 

\speak  Ce sera bien de l'honneur pour moi, dit le chef des sbires, et j'accepte avec reconnaissance. 

\speak  Donc, à la vôtre, monsieur\dots comment vous nommez-vous? 

\speak  Boisrenard. 

\speak  Monsieur Boisrenard! 

\speak  À la vôtre, mon gentilhomme: comment vous nommez-vous, à votre tour, s'il vous plaît? 

\speak  D'Artagnan. 

\speak  À la vôtre, monsieur d'Artagnan! 

\speak  Et par-dessus toutes celles-là, s'écria d'Artagnan comme emporté par son enthousiasme, à celle du roi et du cardinal.» 

Le chef des sbires eût peut-être douté de la sincérité de d'Artagnan, si le vin eût été mauvais; mais le vin était bon, il fut convaincu. 

«Mais quelle diable de vilenie avez-vous donc faite là? dit Porthos lorsque l'alguazil en chef eut rejoint ses compagnons, et que les quatre amis se retrouvèrent seuls. Fi donc! quatre mousquetaires laisser arrêter au milieu d'eux un malheureux qui crie à l'aide! Un gentilhomme trinquer avec un recors! 

\speak  Porthos, dit Aramis, Athos t'a déjà prévenu que tu étais un niais, et je me range de son avis. D'Artagnan, tu es un grand homme, et quand tu seras à la place de M. de Tréville, je te demande ta protection pour me faire avoir une abbaye. 

\speak  Ah çà, je m'y perds, dit Porthos, vous approuvez ce que d'Artagnan vient de faire? 

\speak  Je le crois parbleu bien, dit Athos; non seulement j'approuve ce qu'il vient de faire, mais encore je l'en félicite. 

\speak  Et maintenant, messieurs, dit d'Artagnan sans se donner la peine d'expliquer sa conduite à Porthos, tous pour un, un pour tous, c'est notre devise, n'est-ce pas? 

\speak  Cependant\dots dit Porthos. 

\speak  Étends la main et jure!» s'écrièrent à la fois Athos et Aramis. 

Vaincu par l'exemple, maugréant tout bas, Porthos étendit la main, et les quatre amis répétèrent d'une seule voix la formule dictée par d'Artagnan: 

«Tous pour un, un pour tous.» 

«C'est bien, que chacun se retire maintenant chez soi, dit d'Artagnan comme s'il n'avait fait autre chose que de commander toute sa vie, et attention, car à partir de ce moment, nous voilà aux prises avec le cardinal.»