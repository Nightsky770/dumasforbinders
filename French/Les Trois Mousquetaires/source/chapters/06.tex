%!TeX root=../musketeersfr.tex 

\chapter{Sa Majesté Le Roi Louis Treizième} 

\lettrine{L}{'affaire} fit grand bruit. M. de Tréville gronda beaucoup tout haut contre ses mousquetaires, et les félicita tout bas; mais comme il n'y avait pas de temps à perdre pour prévenir le roi, M. de Tréville s'empressa de se rendre au Louvre. Il était déjà trop tard, le roi était enfermé avec le cardinal, et l'on dit à M. de Tréville que le roi travaillait et ne pouvait recevoir en ce moment. Le soir, M. de Tréville vint au jeu du roi. Le roi gagnait, et comme Sa Majesté était fort avare, elle était d'excellente humeur; aussi, du plus loin que le roi aperçut Tréville: 

«Venez ici, monsieur le capitaine, dit-il, venez que je vous gronde; savez-vous que Son Éminence est venue me faire des plaintes sur vos mousquetaires, et cela avec une telle émotion, que ce soir Son Éminence en est malade? Ah çà, mais ce sont des diables à quatre, des gens à pendre, que vos mousquetaires! 

\speak  Non, Sire, répondit Tréville, qui vit du premier coup d'œil comment la chose allait tourner; non, tout au contraire, ce sont de bonnes créatures, douces comme des agneaux, et qui n'ont qu'un désir, je m'en ferais garant: c'est que leur épée ne sorte du fourreau que pour le service de Votre Majesté. Mais, que voulez-vous, les gardes de M. le cardinal sont sans cesse à leur chercher querelle, et, pour l'honneur même du corps, les pauvres jeunes gens sont obligés de se défendre. 

\speak  Écoutez M. de Tréville! dit le roi, écoutez-le! ne dirait-on pas qu'il parle d'une communauté religieuse! En vérité, mon cher capitaine, j'ai envie de vous ôter votre brevet et de le donner à Mlle de Chémerault, à laquelle j'ai promis une abbaye. Mais ne pensez pas que je vous croirai ainsi sur parole. On m'appelle Louis le Juste, monsieur de Tréville, et tout à l'heure, tout à l'heure nous verrons. 

\speak  Ah! c'est parce que je me fie à cette justice, Sire, que j'attendrai patiemment et tranquillement le bon plaisir de Votre Majesté. 

\speak  Attendez donc, monsieur, attendez donc, dit le roi, je ne vous ferai pas longtemps attendre.» 

En effet, la chance tournait, et comme le roi commençait à perdre ce qu'il avait gagné, il n'était pas fâché de trouver un prétexte pour faire --- qu'on nous passe cette expression de joueur, dont, nous l'avouons, nous ne connaissons pas l'origine ---, pour faire charlemagne. Le roi se leva donc au bout d'un instant, et mettant dans sa poche l'argent qui était devant lui et dont la majeure partie venait de son gain: 

«La Vieuville, dit-il, prenez ma place, il faut que je parle à M. de Tréville pour affaire d'importance. Ah!\dots j'avais quatre-vingts louis devant moi; mettez la même somme, afin que ceux qui ont perdu n'aient point à se plaindre. La justice avant tout.» 

Puis, se retournant vers M. de Tréville et marchant avec lui vers l'embrasure d'une fenêtre: 

«Eh bien, monsieur, continua-t-il, vous dites que ce sont les gardes de l'Éminentissime qui ont été chercher querelle à vos mousquetaires? 

\speak  Oui, Sire, comme toujours. 

\speak  Et comment la chose est-elle venue, voyons? car, vous le savez, mon cher capitaine, il faut qu'un juge écoute les deux parties. 

\speak  Ah! mon Dieu! de la façon la plus simple et la plus naturelle. Trois de mes meilleurs soldats, que Votre Majesté connaît de nom et dont elle a plus d'une fois apprécié le dévouement, et qui ont, je puis l'affirmer au roi, son service fort à cœur; --- trois de mes meilleurs soldats, dis-je, MM. Athos, Porthos et Aramis, avaient fait une partie de plaisir avec un jeune cadet de Gascogne que je leur avais recommandé le matin même. La partie allait avoir lieu à Saint-Germain, je crois, et ils s'étaient donné rendez-vous aux Carmes-Deschaux, lorsqu'elle fut troublée par M. de Jussac et MM. Cahusac, Biscarat, et deux autres gardes qui ne venaient certes pas là en si nombreuse compagnie sans mauvaise intention contre les édits. 

\speak  Ah! ah! vous m'y faites penser, dit le roi: sans doute, ils venaient pour se battre eux-mêmes. 

\speak  Je ne les accuse pas, Sire, mais je laisse Votre Majesté apprécier ce que peuvent aller faire cinq hommes armés dans un lieu aussi désert que le sont les environs du couvent des Carmes. 

\speak  Oui, vous avez raison, Tréville, vous avez raison. 

\speak  Alors, quand ils ont vu mes mousquetaires, ils ont changé d'idée et ils ont oublié leur haine particulière pour la haine de corps; car Votre Majesté n'ignore pas que les mousquetaires, qui sont au roi et rien qu'au roi, sont les ennemis naturels des gardes, qui sont à M. le cardinal. 

\speak  Oui, Tréville, oui, dit le roi mélancoliquement, et c'est bien triste, croyez-moi, de voir ainsi deux partis en France, deux têtes à la royauté; mais tout cela finira, Tréville, tout cela finira. Vous dites donc que les gardes ont cherché querelle aux mousquetaires? 

\speak  Je dis qu'il est probable que les choses se sont passées ainsi, mais je n'en jure pas, Sire. Vous savez combien la vérité est difficile à connaître, et à moins d'être doué de cet instinct admirable qui a fait nommer Louis XIII le Juste\dots 

\speak  Et vous avez raison, Tréville; mais ils n'étaient pas seuls, vos mousquetaires, il y avait avec eux un enfant? 

\speak  Oui, Sire, et un homme blessé, de sorte que trois mousquetaires du roi, dont un blessé, et un enfant, non seulement ont tenu tête à cinq des plus terribles gardes de M. le cardinal, mais encore en ont porté quatre à terre. 

\speak  Mais c'est une victoire, cela! s'écria le roi tout rayonnant; une victoire complète! 

\speak  Oui, Sire, aussi complète que celle du pont de Cé. 

\speak  Quatre hommes, dont un blessé, et un enfant, dites-vous? 

\speak  Un jeune homme à peine; lequel s'est même si parfaitement conduit en cette occasion, que je prendrai la liberté de le recommander à Votre Majesté. 

\speak  Comment s'appelle-t-il? 

\speak  D'Artagnan, Sire. C'est le fils d'un de mes plus anciens amis; le fils d'un homme qui a fait avec le roi votre père, de glorieuse mémoire, la guerre de partisan. 

\speak  Et vous dites qu'il s'est bien conduit, ce jeune homme? Racontez-moi cela, Tréville; vous savez que j'aime les récits de guerre et de combat.» 

Et le roi Louis XIII releva fièrement sa moustache en se posant sur la hanche. 

«Sire, reprit Tréville, comme je vous l'ai dit M. d'Artagnan est presque un enfant, et comme il n'a pas l'honneur d'être mousquetaire, il était en habit bourgeois; les gardes de M. le cardinal, reconnaissant sa grande jeunesse et, de plus, qu'il était étranger au corps, l'invitèrent donc à se retirer avant qu'ils attaquassent. 

\speak  Alors, vous voyez bien, Tréville, interrompit le roi, que ce sont eux qui ont attaqué. 

\speak  C'est juste, Sire: ainsi, plus de doute; ils le sommèrent donc de se retirer; mais il répondit qu'il était mousquetaire de cœur et tout à Sa Majesté, qu'ainsi donc il resterait avec messieurs les mousquetaires. 

\speak  Brave jeune homme! murmura le roi. 

\speak  En effet, il demeura avec eux; et Votre Majesté a là un si ferme champion, que ce fut lui qui donna à Jussac ce terrible coup d'épée qui met si fort en colère M. le cardinal. 

\speak  C'est lui qui a blessé Jussac? s'écria le roi; lui, un enfant! Ceci, Tréville, c'est impossible. 

\speak  C'est comme j'ai l'honneur de le dire à Votre Majesté. 

\speak  Jussac, une des premières lames du royaume! 

\speak  Eh bien, Sire! il a trouvé son maître. 

\speak  Je veux voir ce jeune homme, Tréville, je veux le voir, et si l'on peut faire quelque chose, eh bien, nous nous en occuperons. 

\speak  Quand Votre Majesté daignera-t-elle le recevoir? 

\speak  Demain à midi, Tréville. 

\speak  L'amènerai-je seul? 

\speak  Non, amenez-les-moi tous les quatre ensemble. Je veux les remercier tous à la fois; les hommes dévoués sont rares, Tréville, et il faut récompenser le dévouement. 

\speak  À midi, Sire, nous serons au Louvre. 

\speak  Ah! par le petit escalier, Tréville, par le petit escalier. Il est inutile que le cardinal sache\dots 

\speak  Oui, Sire. 

\speak  Vous comprenez, Tréville, un édit est toujours un édit; il est défendu de se battre, au bout du compte. 

\speak  Mais cette rencontre, Sire, sort tout à fait des conditions ordinaires d'un duel: c'est une rixe, et la preuve, c'est qu'ils étaient cinq gardes du cardinal contre mes trois mousquetaires et M. d'Artagnan. 

\speak  C'est juste, dit le roi; mais n'importe, Tréville, venez toujours par le petit escalier.» 

Tréville sourit. Mais comme c'était déjà beaucoup pour lui d'avoir obtenu de cet enfant qu'il se révoltât contre son maître, il salua respectueusement le roi, et avec son agrément prit congé de lui. 

Dès le soir même, les trois mousquetaires furent prévenus de l'honneur qui leur était accordé. Comme ils connaissaient depuis longtemps le roi, ils n'en furent pas trop échauffés: mais d'Artagnan, avec son imagination gasconne, y vit sa fortune à venir, et passa la nuit à faire des rêves d'or. Aussi, dès huit heures du matin, était-il chez Athos. 

D'Artagnan trouva le mousquetaire tout habillé et prêt à sortir. Comme on n'avait rendez-vous chez le roi qu'à midi, il avait formé le projet, avec Porthos et Aramis, d'aller faire une partie de paume dans un tripot situé tout près des écuries du Luxembourg. Athos invita d'Artagnan à les suivre, et malgré son ignorance de ce jeu, auquel il n'avait jamais joué, celui-ci accepta, ne sachant que faire de son temps, depuis neuf heures du matin qu'il était à peine jusqu'à midi. 

Les deux mousquetaires étaient déjà arrivés et pelotaient ensemble. Athos, qui était très fort à tous les exercices du corps, passa avec d'Artagnan du côté opposé, et leur fit défi. Mais au premier mouvement qu'il essaya, quoiqu'il jouât de la main gauche, il comprit que sa blessure était encore trop récente pour lui permettre un pareil exercice. D'Artagnan resta donc seul, et comme il déclara qu'il était trop maladroit pour soutenir une partie en règle, on continua seulement à s'envoyer des balles sans compter le jeu. Mais une de ces balles, lancée par le poignet herculéen de Porthos, passa si près du visage de d'Artagnan, qu'il pensa que si, au lieu de passer à côté, elle eût donné dedans, son audience était probablement perdue, attendu qu'il lui eût été de toute impossibilité de se présenter chez le roi. Or, comme de cette audience, dans son imagination gasconne, dépendait tout son avenir, il salua poliment Porthos et Aramis, déclarant qu'il ne reprendrait la partie que lorsqu'il serait en état de leur tenir tête, et il s'en revint prendre place près de la corde et dans la galerie. 

Malheureusement pour d'Artagnan, parmi les spectateurs se trouvait un garde de Son Éminence, lequel, tout échauffé encore de la défaite de ses compagnons, arrivée la veille seulement, s'était promis de saisir la première occasion de la venger. Il crut donc que cette occasion était venue, et s'adressant à son voisin: 

«Il n'est pas étonnant, dit-il, que ce jeune homme ait eu peur d'une balle, c'est sans doute un apprenti mousquetaire.» 

D'Artagnan se retourna comme si un serpent l'eût mordu, et regarda fixement le garde qui venait de tenir cet insolent propos. 

«Pardieu! reprit celui-ci en frisant insolemment, sa moustache, regardez-moi tant que vous voudrez, mon petit monsieur, j'ai dit ce que j'ai dit. 

\speak  Et comme ce que vous avez dit est trop clair pour que vos paroles aient besoin d'explication, répondit d'Artagnan à voix basse, je vous prierai de me suivre. 

\speak  Et quand cela? demanda le garde avec le même air railleur. 

\speak  Tout de suite, s'il vous plaît. 

\speak  Et vous savez qui je suis, sans doute? 

\speak Moi, je l'ignore complètement, et je ne m'en inquiète guère. 

\speak  Et vous avez tort, car, si vous saviez mon nom, peut-être seriez-vous moins pressé. 

\speak  Comment vous appelez-vous? 

\speak  Bernajoux, pour vous servir. 

\speak  Eh bien, monsieur Bernajoux, dit tranquillement d'Artagnan, je vais vous attendre sur la porte. 

\speak  Allez, monsieur, je vous suis. 

\speak  Ne vous pressez pas trop, monsieur, qu'on ne s'aperçoive pas que nous sortons ensemble; vous comprenez que pour ce que nous allons faire, trop de monde nous gênerait. 

\speak  C'est bien», répondit le garde, étonné que son nom n'eût pas produit plus d'effet sur le jeune homme. 

En effet, le nom de Bernajoux était connu de tout le monde, de d'Artagnan seul excepté, peut-être; car c'était un de ceux qui figuraient le plus souvent dans les rixes journalières que tous les édits du roi et du cardinal n'avaient pu réprimer. 

Porthos et Aramis étaient si occupés de leur partie, et Athos les regardait avec tant d'attention, qu'ils ne virent pas même sortir leur jeune compagnon, lequel, ainsi qu'il l'avait dit au garde de Son Éminence, s'arrêta sur la porte; un instant après, celui-ci descendit à son tour. Comme d'Artagnan n'avait pas de temps à perdre, vu l'audience du roi qui était fixée à midi, il jeta les yeux autour de lui, et voyant que la rue était déserte: 

«Ma foi, dit-il à son adversaire, il est bien heureux pour vous, quoique vous vous appeliez Bernajoux, de n'avoir affaire qu'à un apprenti mousquetaire; cependant, soyez tranquille, je ferai de mon mieux. En garde! 

\speak  Mais, dit celui que d'Artagnan provoquait ainsi, il me semble que le lieu est assez mal choisi, et que nous serions mieux derrière l'abbaye de Saint-Germain ou dans le Pré-aux-Clercs. 

\speak  Ce que vous dites est plein de sens, répondit d'Artagnan; malheureusement j'ai peu de temps à moi, ayant un rendez-vous à midi juste. En garde donc, monsieur, en garde!» 

Bernajoux n'était pas homme à se faire répéter deux fois un pareil compliment. Au même instant son épée brilla à sa main, et il fondit sur son adversaire que, grâce à sa grande jeunesse, il espérait intimider. 

Mais d'Artagnan avait fait la veille son apprentissage, et tout frais émoulu de sa victoire, tout gonflé de sa future faveur, il était résolu à ne pas reculer d'un pas: aussi les deux fers se trouvèrent-ils engagés jusqu'à la garde, et comme d'Artagnan tenait ferme à sa place, ce fut son adversaire qui fit un pas de retraite. Mais d'Artagnan saisit le moment où, dans ce mouvement, le fer de Bernajoux déviait de la ligne, il dégagea, se fendit et toucha son adversaire à l'épaule. Aussitôt d'Artagnan, à son tour, fit un pas de retraite et releva son épée; mais Bernajoux lui cria que ce n'était rien, et se fendant aveuglément sur lui, il s'enferra de lui-même. Cependant, comme il ne tombait pas, comme il ne se déclarait pas vaincu, mais que seulement il rompait du côté de l'hôtel de M. de La Trémouille au service duquel il avait un parent, d'Artagnan, ignorant lui-même la gravité de la dernière blessure que son adversaire avait reçue, le pressait vivement, et sans doute allait l'achever d'un troisième coup, lorsque la rumeur qui s'élevait de la rue s'étant étendue jusqu'au jeu de paume, deux des amis du garde, qui l'avaient entendu échanger quelques paroles avec d'Artagnan et qui l'avaient vu sortir à la suite de ces paroles, se précipitèrent l'épée à la main hors du tripot et tombèrent sur le vainqueur. Mais aussitôt Athos, Porthos et Aramis parurent à leur tour et au moment où les deux gardes attaquaient leur jeune camarade, les forcèrent à se retourner. En ce moment Bernajoux tomba; et comme les gardes étaient seulement deux contre quatre, ils se mirent à crier: «À nous, l'hôtel de La Trémouille!» À ces cris, tout ce qui était dans l'hôtel sortit, se ruant sur les quatre compagnons, qui de leur côté se mirent à crier: «À nous, mousquetaires!» 

Ce cri était ordinairement entendu; car on savait les mousquetaires ennemis de Son Éminence, et on les aimait pour la haine qu'ils portaient au cardinal. Aussi les gardes des autres compagnies que celles appartenant au duc Rouge, comme l'avait appelé Aramis, prenaient-ils en général parti dans ces sortes de querelles pour les mousquetaires du roi. De trois gardes de la compagnie de M. des Essarts qui passaient, deux vinrent donc en aide aux quatre compagnons, tandis que l'autre courait à l'hôtel de M. de Tréville, criant: «À nous, mousquetaires, à nous!» Comme d'habitude, l'hôtel de M. de Tréville était plein de soldats de cette arme, qui accoururent au secours de leurs camarades; la mêlée devint générale, mais la force était aux mousquetaires: les gardes du cardinal et les gens de M. de La Trémouille se retirèrent dans l'hôtel, dont ils fermèrent les portes assez à temps pour empêcher que leurs ennemis n'y fissent irruption en même temps qu'eux. Quant au blessé, il y avait été tout d'abord transporté et, comme nous l'avons dit, en fort mauvais état. 

L'agitation était à son comble parmi les mousquetaires et leurs alliés, et l'on délibérait déjà si, pour punir l'insolence qu'avaient eue les domestiques de M. de La Trémouille de faire une sortie sur les mousquetaires du roi, on ne mettrait pas le feu à son hôtel. La proposition en avait été faite et accueillie avec enthousiasme, lorsque heureusement onze heures sonnèrent; d'Artagnan et ses compagnons se souvinrent de leur audience, et comme ils eussent regretté que l'on fît un si beau coup sans eux, ils parvinrent à calmer les têtes. On se contenta donc de jeter quelques pavés dans les portes, mais les portes résistèrent: alors on se lassa; d'ailleurs ceux qui devaient être regardés comme les chefs de l'entreprise avaient depuis un instant quitté le groupe et s'acheminaient vers l'hôtel de M. de Tréville, qui les attendait, déjà au courant de cette algarade. 

«Vite, au Louvre, dit-il, au Louvre sans perdre un instant, et tâchons de voir le roi avant qu'il soit prévenu par le cardinal; nous lui raconterons la chose comme une suite de l'affaire d'hier, et les deux passeront ensemble.» 

M. de Tréville, accompagné des quatre jeunes gens, s'achemina donc vers le Louvre; mais, au grand étonnement du capitaine des mousquetaires, on lui annonça que le roi était allé courre le cerf dans la forêt de Saint-Germain. M. de Tréville se fit répéter deux fois cette nouvelle, et à chaque fois ses compagnons virent son visage se rembrunir. 

«Est-ce que Sa Majesté, demanda-t-il, avait dès hier le projet de faire cette chasse? 

\speak  Non, Votre Excellence, répondit le valet de chambre, c'est le grand veneur qui est venu lui annoncer ce matin qu'on avait détourné cette nuit un cerf à son intention. Il a d'abord répondu qu'il n'irait pas, puis il n'a pas su résister au plaisir que lui promettait cette chasse, et après le dîner il est parti. 

\speak  Et le roi a-t-il vu le cardinal? demanda M. de Tréville. 

\speak  Selon toute probabilité, répondit le valet de chambre, car j'ai vu ce matin les chevaux au carrosse de Son Éminence, j'ai demandé où elle allait, et l'on m'a répondu: “À Saint-Germain.” 

\speak  Nous sommes prévenus, dit M. de Tréville, messieurs, je verrai le roi ce soir; mais quant à vous, je ne vous conseille pas de vous y hasarder.» 

L'avis était trop raisonnable et surtout venait d'un homme qui connaissait trop bien le roi, pour que les quatre jeunes gens essayassent de le combattre. M. de Tréville les invita donc à rentrer chacun chez eux et à attendre de ses nouvelles. 

En entrant à son hôtel, M. de Tréville songea qu'il fallait prendre date en portant plainte le premier. Il envoya un de ses domestiques chez M. de La Trémouille avec une lettre dans laquelle il le priait de mettre hors de chez lui le garde de M. le cardinal, et de réprimander ses gens de l'audace qu'ils avaient eue de faire leur sortie contre les mousquetaires. Mais M. de La Trémouille, déjà prévenu par son écuyer dont, comme on le sait, Bernajoux était le parent, lui fit répondre que ce n'était ni à M. de Tréville, ni à ses mousquetaires de se plaindre, mais bien au contraire à lui dont les mousquetaires avaient chargé les gens et voulu brûler l'hôtel. Or, comme le débat entre ces deux seigneurs eût pu durer longtemps, chacun devant naturellement s'entêter dans son opinion, M. de Tréville avisa un expédient qui avait pour but de tout terminer: c'était d'aller trouver lui-même M. de La Trémouille. 

Il se rendit donc aussitôt à son hôtel et se fit annoncer. 

Les deux seigneurs se saluèrent poliment, car, s'il n'y avait pas amitié entre eux, il y avait du moins estime. Tous deux étaient gens de cœur et d'honneur; et comme M. de La Trémouille, protestant, et voyant rarement le roi, n'était d'aucun parti, il n'apportait en général dans ses relations sociales aucune prévention. Cette fois, néanmoins, son accueil quoique poli fut plus froid que d'habitude. 

«Monsieur, dit M. de Tréville, nous croyons avoir à nous plaindre chacun l'un de l'autre, et je suis venu moi-même pour que nous tirions de compagnie cette affaire au clair. 

\speak  Volontiers, répondit M. de La Trémouille; mais je vous préviens que je suis bien renseigné, et tout le tort est à vos mousquetaires. 

\speak  Vous êtes un homme trop juste et trop raisonnable, monsieur, dit M. de Tréville, pour ne pas accepter la proposition que je vais faire. 

\speak  Faites, monsieur, j'écoute. 

\speak  Comment se trouve M. Bernajoux, le parent de votre écuyer? 

\speak  Mais, monsieur, fort mal. Outre le coup d'épée qu'il a reçu dans le bras, et qui n'est pas autrement dangereux, il en a encore ramassé un autre qui lui a traversé le poumon, de sorte que le médecin en dit de pauvres choses. 

\speak  Mais le blessé a-t-il conservé sa connaissance? 

\speak  Parfaitement. 

\speak  Parle-t-il? 

\speak  Avec difficulté, mais il parle. 

\speak  Eh bien, monsieur! rendons-nous près de lui; adjurons-le, au nom du Dieu devant lequel il va être appelé peut-être, de dire la vérité. Je le prends pour juge dans sa propre cause, monsieur, et ce qu'il dira je le croirai.» 

M. de La Trémouille réfléchit un instant, puis, comme il était difficile de faire une proposition plus raisonnable, il accepta. 

Tous deux descendirent dans la chambre où était le blessé. Celui-ci, en voyant entrer ces deux nobles seigneurs qui venaient lui faire visite, essaya de se relever sur son lit, mais il était trop faible, et, épuisé par l'effort qu'il avait fait, il retomba presque sans connaissance. 

M. de La Trémouille s'approcha de lui et lui fit respirer des sels qui le rappelèrent à la vie. Alors M. de Tréville, ne voulant pas qu'on pût l'accuser d'avoir influencé le malade, invita M. de La Trémouille à l'interroger lui-même. 

Ce qu'avait prévu M. de Tréville arriva. Placé entre la vie et la mort comme l'était Bernajoux, il n'eut pas même l'idée de taire un instant la vérité, et il raconta aux deux seigneurs les choses exactement, telles qu'elles s'étaient passées. 

C'était tout ce que voulait M. de Tréville; il souhaita à Bernajoux une prompte convalescence, prit congé de M. de La Trémouille, rentra à son hôtel et fit aussitôt prévenir les quatre amis qu'il les attendait à dîner. 

M. de Tréville recevait fort bonne compagnie, toute anticardinaliste d'ailleurs. On comprend donc que la conversation roula pendant tout le dîner sur les deux échecs que venaient d'éprouver les gardes de Son Éminence. Or, comme d'Artagnan avait été le héros de ces deux journées, ce fut sur lui que tombèrent toutes les félicitations, qu'Athos, Porthos et Aramis lui abandonnèrent non seulement en bons camarades, mais en hommes qui avaient eu assez souvent leur tour pour qu'ils lui laissassent le sien. 

Vers six heures, M. de Tréville annonça qu'il était tenu d'aller au Louvre; mais comme l'heure de l'audience accordée par Sa Majesté était passée, au lieu de réclamer l'entrée par le petit escalier, il se plaça avec les quatre jeunes gens dans l'antichambre. Le roi n'était pas encore revenu de la chasse. Nos jeunes gens attendaient depuis une demi-heure à peine, mêlés à la foule des courtisans, lorsque toutes les portes s'ouvrirent et qu'on annonça Sa Majesté. 

À cette annonce, d'Artagnan se sentit frémir jusqu'à la moelle des os. L'instant qui allait suivre devait, selon toute probabilité, décider du reste de sa vie. Aussi ses yeux se fixèrent-ils avec angoisse sur la porte par laquelle devait entrer le roi. 

Louis XIII parut, marchant le premier; il était en costume de chasse, encore tout poudreux, ayant de grandes bottes et tenant un fouet à la main. Au premier coup d'œil, d'Artagnan jugea que l'esprit du roi était à l'orage. 

Cette disposition, toute visible qu'elle était chez Sa Majesté, n'empêcha pas les courtisans de se ranger sur son passage: dans les antichambres royales, mieux vaut encore être vu d'un œil irrité que de n'être pas vu du tout. Les trois mousquetaires n'hésitèrent donc pas, et firent un pas en avant, tandis que d'Artagnan au contraire restait caché derrière eux; mais quoique le roi connût personnellement Athos, Porthos et Aramis, il passa devant eux sans les regarder, sans leur parler et comme s'il ne les avait jamais vus. Quant à M. de Tréville, lorsque les yeux du roi s'arrêtèrent un instant sur lui, il soutint ce regard avec tant de fermeté, que ce fut le roi qui détourna la vue; après quoi, tout en grommelant, Sa Majesté rentra dans son appartement. 

«Les affaires vont mal, dit Athos en souriant, et nous ne serons pas encore fait chevaliers de l'ordre cette fois-ci. 

\speak  Attendez ici dix minutes, dit M. de Tréville; et si au bout de dix minutes vous ne me voyez pas sortir, retournez à mon hôtel: car il sera inutile que vous m'attendiez plus longtemps.» 

Les quatre jeunes gens attendirent dix minutes, un quart d'heure, vingt minutes; et voyant que M. de Tréville ne reparaissait point, ils sortirent fort inquiets de ce qui allait arriver. 

M. de Tréville était entré hardiment dans le cabinet du roi, et avait trouvé Sa Majesté de très méchante humeur, assise sur un fauteuil et battant ses bottes du manche de son fouet, ce qui ne l'avait pas empêché de lui demander avec le plus grand flegme des nouvelles de sa santé. 

«Mauvaise, monsieur, mauvaise, répondit le roi, je m'ennuie.» 

C'était en effet la pire maladie de Louis XIII, qui souvent prenait un de ses courtisans, l'attirait à une fenêtre et lui disait: «Monsieur un tel, ennuyons-nous ensemble.» 

«Comment! Votre Majesté s'ennuie! dit M. de Tréville. N'a-t-elle donc pas pris aujourd'hui le plaisir de la chasse? 

\speak  Beau plaisir, monsieur! Tout dégénère, sur mon âme, et je ne sais si c'est le gibier qui n'a plus de voie ou les chiens qui n'ont plus de nez. Nous lançons un cerf dix cors, nous le courons six heures, et quand il est prêt à tenir, quand Saint-Simon met déjà le cor à sa bouche pour sonner l'hallali, crac! toute la meute prend le change et s'emporte sur un daguet. Vous verrez que je serai obligé de renoncer à la chasse à courre comme j'ai renoncé à la chasse au vol. Ah! je suis un roi bien malheureux, monsieur de Tréville! je n'avais plus qu'un gerfaut, et il est mort avant-hier. 

\speak  En effet, Sire, je comprends votre désespoir, et le malheur est grand; mais il vous reste encore, ce me semble, bon nombre de faucons, d'éperviers et de tiercelets. 

\speak  Et pas un homme pour les instruire, les fauconniers s'en vont, il n'y a plus que moi qui connaisse l'art de la vénerie. Après moi tout sera dit, et l'on chassera avec des traquenards, des pièges, des trappes. Si j'avais le temps encore de former des élèves! mais oui, M. le cardinal est là qui ne me laisse pas un instant de repos, qui me parle de l'Espagne, qui me parle de l'Autriche, qui me parle de l'Angleterre! Ah! à propos de M. le cardinal, monsieur de Tréville, je suis mécontent de vous.» 

M. de Tréville attendait le roi à cette chute. Il connaissait le roi de longue main; il avait compris que toutes ses plaintes n'étaient qu'une préface, une espèce d'excitation pour s'encourager lui-même, et que c'était où il était arrivé enfin qu'il en voulait venir. 

«Et en quoi ai-je été assez malheureux pour déplaire à Votre Majesté? demanda M. de Tréville en feignant le plus profond étonnement. 

\speak  Est-ce ainsi que vous faites votre charge, monsieur? continua le roi sans répondre directement à la question de M. de Tréville; est-ce pour cela que je vous ai nommé capitaine de mes mousquetaires, que ceux-ci assassinent un homme, émeuvent tout un quartier et veulent brûler Paris sans que vous en disiez un mot? Mais, au reste, continua le roi, sans doute que je me hâte de vous accuser, sans doute que les perturbateurs sont en prison et que vous venez m'annoncer que justice est faite. 

\speak  Sire, répondit tranquillement M. de Tréville, je viens vous la demander au contraire. 

\speak  Et contre qui? s'écria le roi. 

\speak  Contre les calomniateurs, dit M. de Tréville. 

\speak  Ah! voilà qui est nouveau, reprit le roi. N'allez-vous pas dire que vos trois mousquetaires damnés, Athos, Porthos et Aramis et votre cadet de Béarn, ne se sont pas jetés comme des furieux sur le pauvre Bernajoux, et ne l'ont pas maltraité de telle façon qu'il est probable qu'il est en train de trépasser à cette heure! N'allez-vous pas dire qu'ensuite ils n'ont pas fait le siège de l'hôtel du duc de La Trémouille, et qu'ils n'ont point voulu le brûler! ce qui n'aurait peut-être pas été un très grand malheur en temps de guerre, vu que c'est un nid de huguenots, mais ce qui, en temps de paix, est un fâcheux exemple. Dites, n'allez-vous pas nier tout cela? 

\speak  Et qui vous a fait ce beau récit, Sire? demanda tranquillement M. de Tréville. 

\speak  Qui m'a fait ce beau récit, monsieur! et qui voulez-vous que ce soit, si ce n'est celui qui veille quand je dors, qui travaille quand je m'amuse, qui mène tout au-dedans et au-dehors du royaume, en France comme en Europe? 

\speak  Sa Majesté veut parler de Dieu, sans doute, dit M. de Tréville, car je ne connais que Dieu qui soit si fort au-dessus de Sa Majesté. 

\speak  Non monsieur; je veux parler du soutien de l'État, de mon seul serviteur, de mon seul ami, de M. le cardinal. 

\speak  Son Éminence n'est pas Sa Sainteté, Sire. 

\speak  Qu'entendez-vous par là, monsieur? 

\speak  Qu'il n'y a que le pape qui soit infaillible, et que cette infaillibilité ne s'étend pas aux cardinaux. 

\speak  Vous voulez dire qu'il me trompe, vous voulez dire qu'il me trahit. Vous l'accusez alors. Voyons, dites, avouez franchement que vous l'accusez. 

\speak  Non, Sire; mais je dis qu'il se trompe lui-même, je dis qu'il a été mal renseigné; je dis qu'il a eu hâte d'accuser les mousquetaires de Votre Majesté, pour lesquels il est injuste, et qu'il n'a pas été puiser ses renseignements aux bonnes sources. 

\speak  L'accusation vient de M. de La Trémouille, du duc lui-même. Que répondrez-vous à cela? 

\speak  Je pourrais répondre, Sire, qu'il est trop intéressé dans la question pour être un témoin bien impartial; mais loin de là, Sire, je connais le duc pour un loyal gentilhomme, et je m'en rapporterai à lui, mais à une condition, Sire. 

\speak  Laquelle? 

\speak  C'est que Votre Majesté le fera venir, l'interrogera, mais elle-même, en tête-à-tête, sans témoins, et que je reverrai Votre Majesté aussitôt qu'elle aura reçu le duc. 

\speak  Oui-da! fit le roi, et vous vous en rapporterez à ce que dira M. de La Trémouille? 

\speak  Oui, Sire. 

\speak  Vous accepterez son jugement? 

\speak  Sans doute. 

\speak  Et vous vous soumettrez aux réparations qu'il exigera? 

\speak  Parfaitement. 

\speak  La Chesnaye! fit le roi. La Chesnaye!» 

Le valet de chambre de confiance de Louis XIII, qui se tenait toujours à la porte, entra. 

«La Chesnaye, dit le roi, qu'on aille à l'instant même me quérir M. de La Trémouille; je veux lui parler ce soir. 

\speak  Votre Majesté me donne sa parole qu'elle ne verra personne entre M. de La Trémouille et moi? 

\speak  Personne, foi de gentilhomme. 

\speak  À demain, Sire, alors. 

\speak  À demain, monsieur. 

\speak  À quelle heure, s'il plaît à Votre Majesté? 

\speak  À l'heure que vous voudrez. 

\speak  Mais, en venant par trop matin, je crains de réveiller votre Majesté. 

\speak  Me réveiller? Est-ce que je dors? Je ne dors plus, monsieur; je rêve quelquefois, voilà tout. Venez donc d'aussi bon matin que vous voudrez, à sept heures; mais gare à vous, si vos mousquetaires sont coupables! 

\speak  Si mes mousquetaires sont coupables, Sire, les coupables seront remis aux mains de Votre Majesté, qui ordonnera d'eux selon son bon plaisir. Votre Majesté exige-t-elle quelque chose de plus? qu'elle parle, je suis prêt à lui obéir. 

\speak  Non, monsieur, non, et ce n'est pas sans raison qu'on m'a appelé Louis le Juste. À demain donc, monsieur, à demain. 

\speak  Dieu garde jusque-là Votre Majesté!» 

Si peu que dormit le roi, M. de Tréville dormit plus mal encore; il avait fait prévenir dès le soir même ses trois mousquetaires et leur compagnon de se trouver chez lui à six heures et demie du matin. Il les emmena avec lui sans rien leur affirmer, sans leur rien promettre, et ne leur cachant pas que leur faveur et même la sienne tenaient à un coup de dés. 

Arrivé au bas du petit escalier, il les fit attendre. Si le roi était toujours irrité contre eux, ils s'éloigneraient sans être vus; si le roi consentait à les recevoir, on n'aurait qu'à les faire appeler. 

En arrivant dans l'antichambre particulière du roi, M. de Tréville trouva La Chesnaye, qui lui apprit qu'on n'avait pas rencontré le duc de La Trémouille la veille au soir à son hôtel, qu'il était rentré trop tard pour se présenter au Louvre, qu'il venait seulement d'arriver, et qu'il était à cette heure chez le roi. 

Cette circonstance plut beaucoup à M. de Tréville, qui, de cette façon, fut certain qu'aucune suggestion étrangère ne se glisserait entre la déposition de M. de La Trémouille et lui. 

En effet, dix minutes s'étaient à peine écoulées, que la porte du cabinet s'ouvrit et que M. de Tréville en vit sortir le duc de La Trémouille, lequel vint à lui et lui dit: 

«Monsieur de Tréville, Sa Majesté vient de m'envoyer quérir pour savoir comment les choses s'étaient passées hier matin à mon hôtel. Je lui ai dit la vérité, c'est-à-dire que la faute était à mes gens, et que j'étais prêt à vous en faire mes excuses. Puisque je vous rencontre, veuillez les recevoir, et me tenir toujours pour un de vos amis. 

\speak  Monsieur le duc, dit M. de Tréville, j'étais si plein de confiance dans votre loyauté, que je n'avais pas voulu près de Sa Majesté d'autre défenseur que vous-même. Je vois que je ne m'étais pas abusé, et je vous remercie de ce qu'il y a encore en France un homme de qui on puisse dire sans se tromper ce que j'ai dit de vous. 

\speak  C'est bien, c'est bien! dit le roi qui avait écouté tous ces compliments entre les deux portes; seulement, dites-lui, Tréville, puisqu'il se prétend un de vos amis, que moi aussi je voudrais être des siens, mais qu'il me néglige; qu'il y a tantôt trois ans que je ne l'ai vu, et que je ne le vois que quand je l'envoie chercher. Dites-lui tout cela de ma part, car ce sont de ces choses qu'un roi ne peut dire lui-même. 

\speak  Merci, Sire, merci, dit le duc; mais que Votre Majesté croie bien que ce ne sont pas ceux, je ne dis point cela pour M. de Tréville, que ce ne sont point ceux qu'elle voit à toute heure du jour qui lui sont le plus dévoués. 

\speak  Ah! vous avez entendu ce que j'ai dit; tant mieux, duc, tant mieux, dit le roi en s'avançant jusque sur la porte. Ah! c'est vous, Tréville! où sont vos mousquetaires? Je vous avais dit avant-hier de me les amener, pourquoi ne l'avez-vous pas fait? 

\speak  Ils sont en bas, Sire, et avec votre congé La Chesnaye va leur dire de monter. 

\speak  Oui, oui, qu'ils viennent tout de suite; il va être huit heures, et à neuf heures j'attends une visite. Allez, monsieur le duc, et revenez surtout. Entrez, Tréville.» 

Le duc salua et sortit. Au moment où il ouvrait la porte, les trois mousquetaires et d'Artagnan, conduits par La Chesnaye, apparaissaient au haut de l'escalier. 

«Venez, mes braves, dit le roi, venez; j'ai à vous gronder.» 

Les mousquetaires s'approchèrent en s'inclinant; d'Artagnan les suivait par-derrière. 

«Comment diable! continua le roi; à vous quatre, sept gardes de Son Éminence mis hors de combat en deux jours! C'est trop, messieurs, c'est trop. À ce compte-là, Son Éminence serait forcée de renouveler sa compagnie dans trois semaines, et moi de faire appliquer les édits dans toute leur rigueur. Un par hasard, je ne dis pas; mais sept en deux jours, je le répète, c'est trop, c'est beaucoup trop. 

\speak  Aussi, Sire, Votre Majesté voit qu'ils viennent tout contrits et tout repentants lui faire leurs excuses. 

\speak  Tout contrits et tout repentants! Hum! fit le roi, je ne me fie point à leurs faces hypocrites; il y a surtout là-bas une figure de Gascon. Venez ici, monsieur.» 

D'Artagnan, qui comprit que c'était à lui que le compliment s'adressait, s'approcha en prenant son air le plus désespéré. 

«Eh bien, que me disiez-vous donc que c'était un jeune homme? c'est un enfant, monsieur de Tréville, un véritable enfant! Et c'est celui-là qui a donné ce rude coup d'épée à Jussac? 

\speak  Et ces deux beaux coups d'épée à Bernajoux. 

\speak  Véritablement! 

\speak  Sans compter, dit Athos, que s'il ne m'avait pas tiré des mains de Biscarat, je n'aurais très certainement pas l'honneur de faire en ce moment-ci ma très humble révérence à Votre Majesté. 

\speak  Mais c'est donc un véritable démon que ce Béarnais, ventre-saint-gris! monsieur de Tréville comme eût dit le roi mon père. À ce métier-là, on doit trouer force pourpoints et briser force épées. Or les Gascons sont toujours pauvres, n'est-ce pas? 

\speak  Sire, je dois dire qu'on n'a pas encore trouvé des mines d'or dans leurs montagnes, quoique le Seigneur dût bien ce miracle en récompense de la manière dont ils ont soutenu les prétentions du roi votre père. 

\speak  Ce qui veut dire que ce sont les Gascons qui m'ont fait roi moi-même, n'est-ce pas, Tréville, puisque je suis le fils de mon père? Eh bien, à la bonne heure, je ne dis pas non. La Chesnaye, allez voir si, en fouillant dans toutes mes poches, vous trouverez quarante pistoles; et si vous les trouvez, apportez-les-moi. Et maintenant, voyons, jeune homme, la main sur la conscience, comment cela s'est-il passé?» 

D'Artagnan raconta l'aventure de la veille dans tous ses détails: comment, n'ayant pas pu dormir de la joie qu'il éprouvait à voir Sa Majesté, il était arrivé chez ses amis trois heures avant l'heure de l'audience; comment ils étaient allés ensemble au tripot, et comment, sur la crainte qu'il avait manifestée de recevoir une balle au visage, il avait été raillé par Bernajoux, lequel avait failli payer cette raillerie de la perte de la vie, et M. de La Trémouille, qui n'y était pour rien, de la perte de son hôtel. 

«C'est bien cela, murmurait le roi; oui, c'est ainsi que le duc m'a raconté la chose. Pauvre cardinal! sept hommes en deux jours, et de ses plus chers; mais c'est assez comme cela, messieurs, entendez-vous! c'est assez: vous avez pris votre revanche de la rue Férou, et au-delà; vous devez être satisfaits. 

\speak  Si Votre Majesté l'est, dit Tréville, nous le sommes. 

\speak  Oui, je le suis, ajouta le roi en prenant une poignée d'or de la main de La Chesnaye, et la mettant dans celle de d'Artagnan. Voici, dit-il, une preuve de ma satisfaction.» 

À cette époque, les idées de fierté qui sont de mise de nos jours n'étaient point encore de mode. Un gentilhomme recevait de la main à la main de l'argent du roi, et n'en était pas le moins du monde humilié. D'Artagnan mit donc les quarante pistoles dans sa poche sans faire aucune façon, et en remerciant tout au contraire grandement Sa Majesté. 

«Là, dit le roi en regardant sa pendule, là, et maintenant qu'il est huit heures et demie, retirez-vous; car, je vous l'ai dit, j'attends quelqu'un à neuf heures. Merci de votre dévouement, messieurs. J'y puis compter, n'est-ce pas? 

\speak  Oh! Sire, s'écrièrent d'une même voix les quatre compagnons, nous nous ferions couper en morceaux pour Votre Majesté. 

\speak  Bien, bien; mais restez entiers: cela vaut mieux, et vous me serez plus utiles. Tréville, ajouta le roi à demi-voix pendant que les autres se retiraient, comme vous n'avez pas de place dans les mousquetaires et que d'ailleurs pour entrer dans ce corps nous avons décidé qu'il fallait faire un noviciat, placez ce jeune homme dans la compagnie des gardes de M. des Essarts, votre beau-frère. Ah! pardieu! Tréville, je me réjouis de la grimace que va faire le cardinal: il sera furieux, mais cela m'est égal; je suis dans mon droit.» 

Et le roi salua de la main Tréville, qui sortit et s'en vint rejoindre ses mousquetaires, qu'il trouva partageant avec d'Artagnan les quarante pistoles. 

Et le cardinal, comme l'avait dit Sa Majesté, fut effectivement furieux, si furieux que pendant huit jours il abandonna le jeu du roi, ce qui n'empêchait pas le roi de lui faire la plus charmante mine du monde, et toutes les fois qu'il le rencontrait de lui demander de sa voix la plus caressante: 

«Eh bien, monsieur le cardinal, comment vont ce pauvre Bernajoux et ce pauvre Jussac, qui sont à vous?» 