%!TeX root=../musketeersfr.tex 

\chapter{Officier}

\lettrine{C}{ependant} le cardinal attendait des nouvelles d'Angleterre, mais aucune nouvelle n'arrivait, si ce n'est fâcheuse et menaçante. 

\zz
Si bien que La Rochelle fût investie, si certain que pût paraître le succès, grâce aux précautions prises et surtout à la digue qui ne laissait plus pénétrer aucune barque dans la ville assiégée, cependant le blocus pouvait durer longtemps encore; et c'était un grand affront pour les armes du roi et une grande gêne pour M. le cardinal, qui n'avait plus, il est vrai, à brouiller Louis XIII avec Anne d'Autriche, la chose était faite, mais à raccommoder M. de Bassompierre, qui était brouillé avec le duc d'Angoulême. 

Quant à Monsieur, qui avait commencé le siège, il laissait au cardinal le soin de l'achever. 

La ville, malgré l'incroyable persévérance de son maire, avait tenté une espèce de mutinerie pour se rendre; le maire avait fait pendre les émeutiers. Cette exécution calma les plus mauvaises têtes, qui se décidèrent alors à se laisser mourir de faim. Cette mort leur paraissait toujours plus lente et moins sûre que le trépas par strangulation. 

De leur côté, de temps en temps, les assiégeants prenaient des messagers que les Rochelois envoyaient à Buckingham ou des espions que Buckingham envoyait aux Rochelois. Dans l'un et l'autre cas le procès était vite fait. M. le cardinal disait ce seul mot: Pendu! On invitait le roi à venir voir la pendaison. Le roi venait languissamment, se mettait en bonne place pour voir l'opération dans tous ses détails: cela le distrayait toujours un peu et lui faisait prendre le siège en patience, mais cela ne l'empêchait pas de s'ennuyer fort, de parler à tout moment de retourner à Paris; de sorte que si les messagers et les espions eussent fait défaut, Son Éminence, malgré toute son imagination, se fût trouvée fort embarrassée. 

Néanmoins le temps passait, les Rochelois ne se rendaient pas: le dernier espion que l'on avait pris était porteur d'une lettre. Cette lettre disait bien à Buckingham que la ville était à toute extrémité; mais, au lieu d'ajouter: «Si votre secours n'arrive pas avant quinze jours, nous nous rendrons», elle ajoutait tout simplement: «Si votre secours n'arrive pas avant quinze jours, nous serons tous morts de faim quand il arrivera.» 

Les Rochelois n'avaient donc espoir qu'en Buckingham. Buckingham était leur Messie. Il était évident que si un jour ils apprenaient d'une manière certaine qu'il ne fallait plus compter sur Buckingham, avec l'espoir leur courage tomberait. 

Le cardinal attendait donc avec grande impatience des nouvelles d'Angleterre qui devaient annoncer que Buckingham ne viendrait pas. 

La question d'emporter la ville de vive force, débattue souvent dans le conseil du roi, avait toujours été écartée; d'abord La Rochelle semblait imprenable, puis le cardinal, quoi qu'il eût dit, savait bien que l'horreur du sang répandu en cette rencontre, où Français devaient combattre contre Français, était un mouvement rétrograde de soixante ans imprimé à la politique, et le cardinal était, à cette époque, ce qu'on appelle aujourd'hui un homme de progrès. En effet, le sac de La Rochelle, l'assassinat de trois ou quatre mille huguenots qui se fussent fait tuer ressemblaient trop, en 1628, au massacre de la Saint-Barthélémy, en 1572; et puis, par-dessus tout cela, ce moyen extrême, auquel le roi, bon catholique, ne répugnait aucunement, venait toujours échouer contre cet argument des généraux assiégeants: La Rochelle est imprenable autrement que par la famine. 

Le cardinal ne pouvait écarter de son esprit la crainte où le jetait sa terrible émissaire, car il avait compris, lui aussi, les proportions étranges de cette femme, tantôt serpent, tantôt lion. L'avait-elle trahi? était-elle morte? Il la connaissait assez, en tout cas, pour savoir qu'en agissant pour lui ou contre lui, amie ou ennemie, elle ne demeurait pas immobile sans de grands empêchements. C'était ce qu'il ne pouvait savoir. 

Au reste, il comptait, et avec raison, sur Milady: il avait deviné dans le passé de cette femme de ces choses terribles que son manteau rouge pouvait seul couvrir; et il sentait que, pour une cause ou pour une autre, cette femme lui était acquise, ne pouvant trouver qu'en lui un appui supérieur au danger qui la menaçait. 

Il résolut donc de faire la guerre tout seul et de n'attendre tout succès étranger que comme on attend une chance heureuse. Il continua de faire élever la fameuse digue qui devait affamer La Rochelle; en attendant, il jeta les yeux sur cette malheureuse ville, qui renfermait tant de misère profonde et tant d'héroïques vertus, et, se rappelant le mot de Louis XI, son prédécesseur politique, comme lui-même était le prédécesseur de Robespierre, il murmura cette maxime du compère de Tristan: «Diviser pour régner.» 

Henri IV, assiégeant Paris, faisait jeter par-dessus les murailles du pain et des vivres; le cardinal fit jeter des petits billets par lesquels il représentait aux Rochelois combien la conduite de leurs chefs était injuste, égoïste et barbare; ces chefs avaient du blé en abondance, et ne le partageaient pas; ils adoptaient cette maxime, car eux aussi avaient des maximes, que peu importait que les femmes, les enfants et les vieillards mourussent, pourvu que les hommes qui devaient défendre leurs murailles restassent forts et bien portants. Jusque-là, soit dévouement, soit impuissance de réagir contre elle, cette maxime, sans être généralement adoptée, était cependant passée de la théorie à la pratique; mais les billets vinrent y porter atteinte. Les billets rappelaient aux hommes que ces enfants, ces femmes, ces vieillards qu'on laissait mourir étaient leurs fils, leurs épouses et leurs pères; qu'il serait plus juste que chacun fût réduit à la misère commune, afin qu'une même position fit prendre des résolutions unanimes. 

Ces billets firent tout l'effet qu'en pouvait attendre celui qui les avait écrits, en ce qu'ils déterminèrent un grand nombre d'habitants à ouvrir des négociations particulières avec l'armée royale. 

Mais au moment où le cardinal voyait déjà fructifier son moyen et s'applaudissait de l'avoir mis en usage, un habitant de La Rochelle, qui avait pu passer à travers les lignes royales, Dieu sait comment, tant était grande la surveillance de Bassompierre, de Schomberg et du duc d'Angoulême, surveillés eux-mêmes par le cardinal, un habitant de La Rochelle, disons-nous, entra dans la ville, venant de Portsmouth et disant qu'il avait vu une flotte magnifique prête à mettre à la voile avant huit jours. De plus, Buckingham annonçait au maire qu'enfin la grande ligue contre la France allait se déclarer, et que le royaume allait être envahi à la fois par les armées anglaises, impériales et espagnoles. Cette lettre fut lue publiquement sur toutes les places, on en afficha des copies aux angles des rues, et ceux-là mêmes qui avaient commencé d'ouvrir des négociations les interrompirent, résolus d'attendre ce secours si pompeusement annoncé. 

Cette circonstance inattendue rendit à Richelieu ses inquiétudes premières, et le força malgré lui à tourner de nouveau les yeux de l'autre côté de la mer. 

Pendant ce temps, exempte des inquiétudes de son seul et véritable chef, l'armée royale menait joyeuse vie; les vivres ne manquaient pas au camp, ni l'argent non plus; tous les corps rivalisaient d'audace et de gaieté. Prendre des espions et les pendre, faire des expéditions hasardeuses sur la digue ou sur la mer, imaginer des folies, les exécuter froidement, tel était le passe-temps qui faisait trouver courts à l'armée ces jours si longs, non seulement pour les Rochelois, rongés par la famine et l'anxiété, mais encore pour le cardinal qui les bloquait si vivement. 

Quelquefois, quand le cardinal, toujours chevauchant comme le dernier gendarme de l'armée, promenait son regard pensif sur ces ouvrages, si lents au gré de son désir, qu'élevaient sous son ordre les ingénieurs qu'il faisait venir de tous les coins du royaume de France, s'il rencontrait un mousquetaire de la compagnie de Tréville, il s'approchait de lui, le regardait d'une façon singulière, et ne le reconnaissant pas pour un de nos quatre compagnons, il laissait aller ailleurs son regard profond et sa vaste pensée. 

Un jour où, rongé d'un mortel ennui, sans espérance dans les négociations avec la ville, sans nouvelles d'Angleterre, le cardinal était sorti sans autre but que de sortir, accompagné seulement de Cahusac et de La Houdinière, longeant les grèves et mêlant l'immensité de ses rêves à l'immensité de l'océan, il arriva au petit pas de son cheval sur une colline du haut de laquelle il aperçut derrière une haie, couchés sur le sable et prenant au passage un de ces rayons de soleil si rares à cette époque de l'année, sept hommes entourés de bouteilles vides. Quatre de ces hommes étaient nos mousquetaires s'apprêtant à écouter la lecture d'une lettre que l'un d'eux venait de recevoir. Cette lettre était si importante, qu'elle avait fait abandonner sur un tambour des cartes et des dés. 

Les trois autres s'occupaient à décoiffer une énorme dame-jeanne de vin de Collioure; c'étaient les laquais de ces messieurs. 

Le cardinal, comme nous l'avons dit, était de sombre humeur, et rien, quand il était dans cette situation d'esprit, ne redoublait sa maussaderie comme la gaieté des autres. D'ailleurs, il avait une préoccupation étrange, c'était de croire toujours que les causes mêmes de sa tristesse excitaient la gaieté des étrangers. Faisant signe à La Houdinière et à Cahusac de s'arrêter, il descendit de cheval et s'approcha de ces rieurs suspects, espérant qu'à l'aide du sable qui assourdissait ses pas, et de la haie qui voilait sa marche, il pourrait entendre quelques mots de cette conversation qui lui paraissait si intéressante; à dix pas de la haie seulement il reconnut le babil gascon de d'Artagnan, et comme il savait déjà que ces hommes étaient des mousquetaires, il ne douta pas que les trois autres ne fussent ceux qu'on appelait les inséparables, c'est-à-dire Athos, Porthos et Aramis. 

On juge si son désir d'entendre la conversation s'augmenta de cette découverte; ses yeux prirent une expression étrange, et d'un pas de chat-tigre il s'avança vers la haie; mais il n'avait pu saisir encore que des syllabes vagues et sans aucun sens positif, lorsqu'un cri sonore et bref le fit tressaillir et attira l'attention des mousquetaires. 

«Officier! cria Grimaud. 

\speak  Vous parlez, je crois, drôle», dit Athos se soulevant sur un coude et fascinant Grimaud de son regard flamboyant. 

Aussi Grimaud n'ajouta-t-il point une parole, se contentant de tendre le doigt indicateur dans la direction de la haie et dénonçant par ce geste le cardinal et son escorte. 

D'un seul bond les quatre mousquetaires furent sur pied et saluèrent avec respect. 

Le cardinal semblait furieux. 

«Il paraît qu'on se fait garder chez messieurs les mousquetaires! dit-il. Est-ce que l'Anglais vient par terre, ou serait-ce que les mousquetaires se regardent comme des officiers supérieurs? 

\speak  Monseigneur, répondit Athos, car au milieu de l'effroi général lui seul avait conservé ce calme et ce sang-froid de grand seigneur qui ne le quittaient jamais, Monseigneur, les mousquetaires, lorsqu'ils ne sont pas de service, ou que leur service est fini, boivent et jouent aux dés, et ils sont des officiers très supérieurs pour leurs laquais. 

\speak  Des laquais! grommela le cardinal, des laquais qui ont la consigne d'avertir leurs maîtres quand passe quelqu'un, ce ne sont point des laquais, ce sont des sentinelles. 

\speak  Son Éminence voit bien cependant que si nous n'avions point pris cette précaution, nous étions exposés à la laisser passer sans lui présenter nos respects et lui offrir nos remerciements pour la grâce qu'elle nous a faite de nous réunir. D'Artagnan, continua Athos, vous qui tout à l'heure demandiez cette occasion d'exprimer votre reconnaissance à Monseigneur, la voici venue, profitez-en. 

Ces mots furent prononcés avec ce flegme imperturbable qui distinguait Athos dans les heures du danger, et cette excessive politesse qui faisait de lui dans certains moments un roi plus majestueux que les rois de naissance. 

D'Artagnan s'approcha et balbutia quelques paroles de remerciements, qui bientôt expirèrent sous le regard assombri du cardinal. 

«N'importe, messieurs, continua le cardinal sans paraître le moins du monde détourné de son intention première par l'incident qu'Athos avait soulevé; n'importe, messieurs, je n'aime pas que de simples soldats, parce qu'ils ont l'avantage de servir dans un corps privilégié, fassent ainsi les grands seigneurs, et la discipline est la même pour eux que pour tout le monde.» 

Athos laissa le cardinal achever parfaitement sa phrase et, s'inclinant en signe d'assentiment, il reprit à son tour: 

«La discipline, Monseigneur, n'a en aucune façon, je l'espère, été oubliée par nous. Nous ne sommes pas de service, et nous avons cru que, n'étant pas de service, nous pouvions disposer de notre temps comme bon nous semblait. Si nous sommes assez heureux pour que Son Éminence ait quelque ordre particulier à nous donner, nous sommes prêts à lui obéir. Monseigneur voit, continua Athos en fronçant le sourcil, car cette espèce d'interrogatoire commençait à l'impatienter, que, pour être prêts à la moindre alerte, nous sommes sortis avec nos armes.» 

Et il montra du doigt au cardinal les quatre mousquets en faisceau près du tambour sur lequel étaient les cartes et les dés. 

«Que Votre Éminence veuille croire, ajouta d'Artagnan, que nous nous serions portés au-devant d'elle si nous eussions pu supposer que c'était elle qui venait vers nous en si petite compagnie.» 

Le cardinal se mordait les moustaches et un peu les lèvres. 

«Savez-vous de quoi vous avez l'air, toujours ensemble, comme vous voilà, armés comme vous êtes, et gardés par vos laquais? dit le cardinal, vous avez l'air de quatre conspirateurs. 

\speak  Oh! quant à ceci, Monseigneur, c'est vrai, dit Athos, et nous conspirons, comme Votre Éminence a pu le voir l'autre matin, seulement c'est contre les Rochelois. 

\speak  Eh! messieurs les politiques, reprit le cardinal en fronçant le sourcil à son tour, on trouverait peut-être dans vos cervelles le secret de bien des choses qui sont ignorées, si on pouvait y lire comme vous lisiez dans cette lettre que vous avez cachée quand vous m'avez vu venir.» 

Le rouge monta à la figure d'Athos, il fit un pas vers Son Éminence. 

«On dirait que vous nous soupçonnez réellement, Monseigneur, et que nous subissons un véritable interrogatoire; s'il en est ainsi, que Votre Éminence daigne s'expliquer, et nous saurons du moins à quoi nous en tenir. 

\speak  Et quand cela serait un interrogatoire, reprit le cardinal, d'autres que vous en ont subi, monsieur Athos, et y ont répondu. 

\speak  Aussi, Monseigneur, ai-je dit à Votre Éminence qu'elle n'avait qu'à questionner, et que nous étions prêts à répondre. 

\speak  Quelle était cette lettre que vous alliez lire, monsieur Aramis, et que vous avez cachée? 

\speak  Une lettre de femme, Monseigneur. 

\speak  Oh! je conçois, dit le cardinal, il faut être discret pour ces sortes de lettres; mais cependant on peut les montrer à un confesseur, et, vous le savez, j'ai reçu les ordres. 

\speak  Monseigneur, dit Athos avec un calme d'autant plus terrible qu'il jouait sa tête en faisant cette réponse, la lettre est d'une femme, mais elle n'est signée ni Marion de Lorme, ni Mme d'Aiguillon.» 

Le cardinal devint pâle comme la mort, un éclair fauve sortit de ses yeux; il se retourna comme pour donner un ordre à Cahusac et à La Houdinière. Athos vit le mouvement; il fit un pas vers les mousquetons, sur lesquels les trois amis avaient les yeux fixés en hommes mal disposés à se laisser arrêter. Le cardinal était, lui, troisième; les mousquetaires, y compris les laquais, étaient sept: il jugea que la partie serait d'autant moins égale, qu'Athos et ses compagnons conspiraient réellement; et, par un de ces retours rapides qu'il tenait toujours à sa disposition, toute sa colère se fondit dans un sourire. 

«Allons, allons! dit-il, vous êtes de braves jeunes gens, fiers au soleil, fidèles dans l'obscurité; il n'y a pas de mal à veiller sur soi quand on veille si bien sur les autres; messieurs, je n'ai point oublié la nuit où vous m'avez servi d'escorte pour aller au Colombier-Rouge; s'il y avait quelque danger à craindre sur la route que je vais suivre, je vous prierais de m'accompagner; mais, comme il n'y en a pas, restez où vous êtes, achevez vos bouteilles, votre partie et votre lettre. Adieu, messieurs.» 

Et, remontant sur son cheval, que Cahusac lui avait amené, il les salua de la main et s'éloigna. 

Les quatre jeunes gens, debout et immobiles, le suivirent des yeux sans dire un seul mot jusqu'à ce qu'il eût disparu. 

Puis ils se regardèrent. 

Tous avaient la figure consternée, car malgré l'adieu amical de Son Éminence, ils comprenaient que le cardinal s'en allait la rage dans le cœur. 

Athos seul souriait d'un sourire puissant et dédaigneux. Quand le cardinal fut hors de la portée de la voix et de la vue: 

«Ce Grimaud a crié bien tard!» dit Porthos, qui avait grande envie de faire tomber sa mauvaise humeur sur quelqu'un. 

Grimaud allait répondre pour s'excuser. Athos leva le doigt et Grimaud se tut. 

«Auriez-vous rendu la lettre, Aramis? dit d'Artagnan. 

\speak  Moi, dit Aramis de sa voix la plus flûtée, j'étais décidé: s'il avait exigé que la lettre lui fût remise, je lui présentais la lettre d'une main, et de l'autre je lui passais mon épée au travers du corps. 

\speak  Je m'y attendais bien, dit Athos; voilà pourquoi je me suis jeté entre vous et lui. En vérité, cet homme est bien imprudent de parler ainsi à d'autres hommes; on dirait qu'il n'a jamais eu affaire qu'à des femmes et à des enfants. 

\speak  Mon cher Athos, dit d'Artagnan, je vous admire, mais cependant nous étions dans notre tort, après tout. 

\speak  Comment, dans notre tort! reprit Athos. À qui donc cet air que nous respirons? à qui cet océan sur lequel s'étendent nos regards? à qui ce sable sur lequel nous étions couchés? à qui cette lettre de votre maîtresse? Est-ce au cardinal? Sur mon honneur, cet homme se figure que le monde lui appartient: vous étiez là, balbutiant, stupéfait, anéanti; on eût dit que la Bastille se dressait devant vous et que la gigantesque Méduse vous changeait en pierre. Est-ce que c'est conspirer, voyons, que d'être amoureux? Vous êtes amoureux d'une femme que le cardinal a fait enfermer, vous voulez la tirer des mains du cardinal; c'est une partie que vous jouez avec Son Éminence: cette lettre c'est votre jeu; pourquoi montreriez-vous votre jeu à votre adversaire? cela ne se fait pas. Qu'il le devine, à la bonne heure! nous devinons bien le sien, nous! 

\speak  Au fait, dit d'Artagnan, c'est plein de sens, ce que vous dites là, Athos. 

\speak  En ce cas, qu'il ne soit plus question de ce qui vient de se passer, et qu'Aramis reprenne la lettre de sa cousine où M. le cardinal l'a interrompue.» 

Aramis tira la lettre de sa poche, les trois amis se rapprochèrent de lui, et les trois laquais se groupèrent de nouveau auprès de la dame-jeanne. 

«Vous n'aviez lu qu'une ligne ou deux, dit d'Artagnan, reprenez donc la lettre à partir du commencement. 

«Volontiers», dit Aramis. 
\begin{mail}{}{Mon cher cousin,}
Je crois bien que je me déciderai à partir pour Stenay, où ma soeur a fait entrer notre petite servante dans le couvent des Carmélites; cette pauvre enfant s'est résignée, elle sait qu'elle ne peut vivre autre part sans que le salut de son âme soit en danger. Cependant, si les affaires de notre famille s'arrangent comme nous le désirons, je crois qu'elle courra le risque de se damner, et qu'elle reviendra près de ceux qu'elle regrette, d'autant plus qu'elle sait qu'on pense toujours à elle. En attendant, elle n'est pas trop malheureuse: tout ce qu'elle désire c'est une lettre de son prétendu. Je sais bien que ces sortes de denrées passent difficilement par les grilles; mais, après tout, comme je vous en ai donné des preuves, mon cher cousin, je ne suis pas trop maladroite et je me chargerai de cette commission. Ma soeur vous remercie de votre bon et éternel souvenir. Elle a eu un instant de grande inquiétude; mais enfin elle est quelque peu rassurée maintenant, ayant envoyé son commis là-bas afin qu'il ne s'y passe rien d'imprévu.

Adieu, mon cher cousin, donnez-nous de vos nouvelles le plus souvent que vous pourrez, c'est-à-dire toutes les fois que vous croirez pouvoir le faire sûrement.
\closeletter[Je vous embrasse.]{Marie Michon}
\end{mail}


«Oh! que ne vous dois-je pas, Aramis? s'écria d'Artagnan. Chère Constance! j'ai donc enfin de ses nouvelles; elle vit, elle est en sûreté dans un couvent, elle est à Stenay! Où prenez-vous Stenay, Athos? 

\speak  Mais à quelques lieues des frontières; une fois le siège levé, nous pourrons aller faire un tour de ce côté. 

\speak  Et ce ne sera pas long, il faut l'espérer, dit Porthos, car on a, ce matin, pendu un espion, lequel a déclaré que les Rochelois en étaient aux cuirs de leurs souliers. En supposant qu'après avoir mangé le cuir ils mangent la semelle, je ne vois pas trop ce qui leur restera après, à moins de se manger les uns les autres. 

\speak  Pauvres sots! dit Athos en vidant un verre d'excellent vin de Bordeaux, qui, sans avoir à cette époque la réputation qu'il a aujourd'hui, ne la méritait pas moins; pauvres sots! comme si la religion catholique n'était pas la plus avantageuse et la plus agréable des religions! C'est égal, reprit-il après avoir fait claquer sa langue contre son palais, ce sont de braves gens. Mais que diable faites-vous donc, Aramis? continua Athos; vous serrez cette lettre dans votre poche? 

\speak  Oui, dit d'Artagnan, Athos a raison, il faut la brûler; encore, qui sait si M. le cardinal n'a pas un secret pour interroger les cendres? 

\speak  Il doit en avoir un, dit Athos. 

\speak  Mais que voulez-vous faire de cette lettre? demanda Porthos. 

\speak  Venez ici, Grimaud», dit Athos. 

Grimaud se leva et obéit. 

«Pour vous punir d'avoir parlé sans permission, mon ami, vous allez manger ce morceau de papier, puis, pour vous récompenser du service que vous nous aurez rendu, vous boirez ensuite ce verre de vin; voici la lettre d'abord, mâchez avec énergie.» 

Grimaud sourit, et, les yeux fixés sur le verre qu'Athos venait de remplir bord à bord, il broya le papier et l'avala. 

«Bravo, maître Grimaud! dit Athos, et maintenant prenez ceci; bien, je vous dispense de dire merci.» 

Grimaud avala silencieusement le verre de vin de Bordeaux, mais ses yeux levés au ciel parlaient, pendant tout le temps que dura cette douce occupation, un langage qui, pour être muet, n'en était pas moins expressif. 

«Et maintenant, dit Athos, à moins que M. le cardinal n'ait l'ingénieuse idée de faire ouvrir le ventre à Grimaud, je crois que nous pouvons être à peu près tranquilles.» 

Pendant ce temps, Son Éminence continuait sa promenade mélancolique en murmurant entre ses moustaches: 

«Décidément, il faut que ces quatre hommes soient à moi.» 