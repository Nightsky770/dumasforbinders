%!TeX root=../musketeersfr.tex 

\chapter{Un Moyen De Tragédie Classique}

\lettrine{A}{près} un moment de silence employé par Milady à observer le jeune homme qui l'écoutait, elle continua son récit: 

\zz
«Il y avait près de trois jours que je n'avais ni bu ni mangé, je souffrais des tortures atroces: parfois il me passait comme des nuages qui me serraient le front, qui me voilaient les yeux: c'était le délire. 

«Le soir vint; j'étais si faible, qu'à chaque instant je m'évanouissais et à chaque fois que je m'évanouissais je remerciais Dieu, car je croyais que j'allais mourir. 

«Au milieu de l'un de ces évanouissements, j'entendis la porte s'ouvrir; la terreur me rappela à moi. 

«Mon persécuteur entra suivi d'un homme masqué, il était masqué lui-même; mais je reconnus son pas, je reconnus cet air imposant que l'enfer a donné à sa personne pour le malheur de l'humanité. 

«Eh bien, me dit-il, êtes-vous décidée à me faire le serment que je vous ai demandé? 

«Vous l'avez dit, les puritains n'ont qu'une parole: la mienne, vous l'avez entendue, c'est de vous poursuivre sur la terre au tribunal des hommes, dans le ciel au tribunal de Dieu! 

«Ainsi, vous persistez? 

«Je le jure devant ce Dieu qui m'entend: je prendrai le monde entier à témoin de votre crime, et cela jusqu'à ce que j'aie trouvé un vengeur. 

«Vous êtes une prostituée, dit-il d'une voix tonnante, et vous subirez le supplice des prostituées! Flétrie aux yeux du monde que vous invoquerez, tâchez de prouver à ce monde que vous n'êtes ni coupable ni folle!» 

«Puis s'adressant à l'homme qui l'accompagnait: 

«Bourreau, dit-il, fais ton devoir.» 

\speak  Oh! son nom, son nom! s'écria Felton; son nom, dites-le-moi! 

\speak  Alors, malgré mes cris, malgré ma résistance, car je commençais à comprendre qu'il s'agissait pour moi de quelque chose de pire que la mort, le bourreau me saisit, me renversa sur le parquet, me meurtrit de ses étreintes, et suffoquée par les sanglots, presque sans connaissance invoquant Dieu, qui ne m'écoutait pas, je poussai tout à coup un effroyable cri de douleur et de honte; un fer brûlant, un fer rouge, le fer du bourreau, s'était imprimé sur mon épaule.» 

Felton poussa un rugissement. 

«Tenez, dit Milady, en se levant alors avec une majesté de reine, --- tenez, Felton, voyez comment on a inventé un nouveau martyre pour la jeune fille pure et cependant victime de la brutalité d'un scélérat. Apprenez à connaître le cœur des hommes, et désormais faites-vous moins facilement l'instrument de leurs injustes vengeances.» 

Milady d'un geste rapide ouvrit sa robe, déchira la batiste qui couvrait son sein, et, rouge d'une feinte colère et d'une honte jouée, montra au jeune homme l'empreinte ineffaçable qui déshonorait cette épaule si belle. 

«Mais, s'écria Felton, c'est une fleur de lis que je vois là! 

\speak  Et voilà justement où est l'infamie, répondit Milady. La flétrissure d'Angleterre!\dots il fallait prouver quel tribunal me l'avait imposée, et j'aurais fait un appel public à tous les tribunaux du royaume; mais la flétrissure de France\dots oh! par elle, j'étais bien réellement flétrie.» 

C'en était trop pour Felton. 

Pâle, immobile, écrasé par cette révélation effroyable, ébloui par la beauté surhumaine de cette femme qui se dévoilait à lui avec une impudeur qu'il trouva sublime, il finit par tomber à genoux devant elle comme faisaient les premiers chrétiens devant ces pures et saintes martyres que la persécution des empereurs livrait dans le cirque à la sanguinaire lubricité des populaces. La flétrissure disparut, la beauté seule resta. 

«Pardon, pardon! s'écria Felton, oh! pardon!» 

Milady lut dans ses yeux: Amour, amour. 

«Pardon de quoi? demanda-t-elle. 

\speak  Pardon de m'être joint à vos persécuteurs.» 

Milady lui tendit la main. 

«Si belle, si jeune!» s'écria Felton en couvrant cette main de baisers. 

Milady laissa tomber sur lui un de ces regards qui d'un esclave font un roi. 

Felton était puritain: il quitta la main de cette femme pour baiser ses pieds. 

Il ne l'aimait déjà plus, il l'adorait. 

Quand cette crise fut passée, quand Milady parut avoir recouvré son sang-froid, qu'elle n'avait jamais perdu; lorsque Felton eut vu se refermer sous le voile de la chasteté ces trésors d'amour qu'on ne lui cachait si bien que pour les lui faire désirer plus ardemment: 

«Ah! maintenant, dit-il, je n'ai plus qu'une chose à vous demander, c'est le nom de votre véritable bourreau; car pour moi il n'y en a qu'un; l'autre était l'instrument, voilà tout. 

\speak  Eh quoi, frère! s'écria Milady, il faut encore que je te le nomme, et tu ne l'as pas deviné? 

\speak  Quoi! reprit Felton, lui!\dots encore lui!\dots toujours lui!\dots Quoi! le vrai coupable\dots 

\speak  Le vrai coupable, dit Milady, c'est le ravageur de l'Angleterre, le persécuteur des vrais croyants, le lâche ravisseur de l'honneur de tant de femmes, celui qui pour un caprice de son cœur corrompu va faire verser tant de sang à deux royaumes, qui protège les protestants aujourd'hui et qui les trahira demain\dots 

\speak  Buckingham! c'est donc Buckingham!» s'écria Felton exaspéré. 

Milady cacha son visage dans ses mains, comme si elle n'eût pu supporter la honte que lui rappelait ce nom. 

«Buckingham, le bourreau de cette angélique créature! s'écria Felton. Et tu ne l'as pas foudroyé, mon Dieu! et tu l'as laissé noble, honoré, puissant pour notre perte à tous! 

\speak  Dieu abandonne qui s'abandonne lui-même, dit Milady. 

\speak  Mais il veut donc attirer sur sa tête le châtiment réservé aux maudits! continua Felton avec une exaltation croissante, il veut donc que la vengeance humaine prévienne la justice céleste! 

\speak  Les hommes le craignent et l'épargnent. 

\speak  Oh! moi, dit Felton, je ne le crains pas et je ne l'épargnerai pas!\dots» 

Milady sentit son âme baignée d'une joie infernale. 

«Mais comment Lord de Winter, mon protecteur, mon père, demanda Felton, se trouve-t-il mêlé à tout cela? 

\speak  Écoutez, Felton, reprit Milady, car à côté des hommes lâches et méprisables, il est encore des natures grandes et généreuses. J'avais un fiancé, un homme que j'aimais et qui m'aimait; un cœur comme le vôtre, Felton, un homme comme vous. Je vins à lui et je lui racontai tout, il me connaissait, celui-là, et ne douta point un instant. C'était un grand seigneur, c'était un homme en tout point l'égal de Buckingham. Il ne dit rien, il ceignit seulement son épée, s'enveloppa de son manteau et se rendit à Buckingham Palace. 

\speak  Oui, oui, dit Felton, je comprends; quoique avec de pareils hommes ce ne soit pas l'épée qu'il faille employer, mais le poignard. 

\speak  Buckingham était parti depuis la veille, envoyé comme ambassadeur en Espagne, où il allait demander la main de l'infante pour le roi Charles I\ier\, qui n'était alors que prince de Galles. Mon fiancé revint. 

«Écoutez, me dit-il, cet homme est parti, et pour le moment, par conséquent, il échappe à ma vengeance; mais en attendant soyons unis, comme nous devions l'être, puis rapportez-vous-en à Lord de Winter pour soutenir son honneur et celui de sa femme.» 

\speak  Lord de Winter! s'écria Felton. 

\speak  Oui, dit Milady, Lord de Winter, et maintenant vous devez tout comprendre, n'est-ce pas? Buckingham resta plus d'un an absent. Huit jours avant son arrivée, Lord de Winter mourut subitement, me laissant sa seule héritière. D'où venait le coup? Dieu, qui sait tout, le sait sans doute, moi je n'accuse personne\dots 

\speak  Oh! quel abîme, quel abîme! s'écria Felton. 

\speak  Lord de Winter était mort sans rien dire à son frère. Le secret terrible devait être caché à tous, jusqu'à ce qu'il éclatât comme la foudre sur la tête du coupable. Votre protecteur avait vu avec peine ce mariage de son frère aîné avec une jeune fille sans fortune. Je sentis que je ne pouvais attendre d'un homme trompé dans ses espérances d'héritage aucun appui. Je passai en France résolue à y demeurer pendant tout le reste de ma vie. Mais toute ma fortune est en Angleterre; les communications fermées par la guerre, tout me manqua: force fut alors d'y revenir; il y a six jours j'abordais à Portsmouth. 

\speak  Eh bien? dit Felton. 

\speak  Eh bien, Buckingham apprit sans doute mon retour, il en parla à Lord de Winter, déjà prévenu contre moi, et lui dit que sa belle-soeur était une prostituée, une femme flétrie. La voix pure et noble de mon mari n'était plus là pour me défendre. Lord de Winter crut tout ce qu'on lui dit, avec d'autant plus de facilité qu'il avait intérêt à le croire. Il me fit arrêter, me conduisit ici, me remit sous votre garde. Vous savez le reste: après-demain il me bannit, il me déporte; après-demain il me relègue parmi les infâmes. Oh! la trame est bien ourdie, allez! le complot est habile et mon honneur n'y survivra pas. Vous voyez bien qu'il faut que je meure, Felton; Felton, donnez-moi ce couteau!» 

Et à ces mots, comme si toutes ses forces étaient épuisées, Milady se laissa aller débile et languissante entre les bras du jeune officier, qui, ivre d'amour, de colère et de voluptés inconnues, la reçut avec transport, la serra contre son cœur, tout frissonnant à l'haleine de cette bouche si belle, tout éperdu au contact de ce sein si palpitant. 

«Non, non, dit-il; non, tu vivras honorée et pure, tu vivras pour triompher de tes ennemis.» 

Milady le repoussa lentement de la main en l'attirant du regard; mais Felton, à son tour, s'empara d'elle, l'implorant comme une Divinité. 

«Oh! la mort, la mort! dit-elle en voilant sa voix et ses paupières, oh! la mort plutôt que la honte; Felton, mon frère, mon ami, je t'en conjure! 

\speak  Non, s'écria Felton, non, tu vivras, et tu seras vengée! 

\speak  Felton, je porte malheur à tout ce qui m'entoure! Felton, abandonne-moi! Felton, laisse-moi mourir! 

\speak  Eh bien, nous mourrons donc ensemble!» s'écria-t-il en appuyant ses lèvres sur celles de la prisonnière. 

Plusieurs coups retentirent à la porte; cette fois, Milady le repoussa réellement. 

«Écoutez, dit-elle, on nous a entendus, on vient! c'en est fait, nous sommes perdus! 

\speak  Non, dit Felton, c'est la sentinelle qui me prévient seulement qu'une ronde arrive. 

\speak  Alors, courez à la porte et ouvrez vous-même.» 

Felton obéit; cette femme était déjà toute sa pensée, toute son âme. 

Il se trouva en face d'un sergent commandant une patrouille de surveillance. 

«Eh bien, qu'y a-t-il? demanda le jeune lieutenant. 

\speak  Vous m'aviez dit d'ouvrir la porte si j'entendais crier au secours, dit le soldat, mais vous aviez oublié de me laisser la clef; je vous ai entendu crier sans comprendre ce que vous disiez, j'ai voulu ouvrir la porte, elle était fermée en dedans, alors j'ai appelé le sergent. 

\speak  Et me voilà», dit le sergent. 

Felton, égaré, presque fou, demeurait sans voix. 

Milady comprit que c'était à elle de s'emparer de la situation, elle courut à la table et prit le couteau qu'y avait déposé Felton: 

«Et de quel droit voulez-vous m'empêcher de mourir? dit-elle. 

\speak  Grand Dieu!» s'écria Felton en voyant le couteau luire à sa main. 

En ce moment, un éclat de rire ironique retentit dans le corridor. 

Le baron, attiré par le bruit, en robe de chambre, son épée sous le bras, se tenait debout sur le seuil de la porte. 

«Ah! ah! dit-il, nous voici au dernier acte de la tragédie; vous le voyez, Felton, le drame a suivi toutes les phases que j'avais indiquées; mais soyez tranquille, le sang ne coulera pas.» 

Milady comprit qu'elle était perdue si elle ne donnait pas à Felton une preuve immédiate et terrible de son courage. 

«Vous vous trompez, Milord, le sang coulera, et puisse ce sang retomber sur ceux qui le font couler!» 

Felton jeta un cri et se précipita vers elle; il était trop tard: Milady s'était frappée. Mais le couteau avait rencontré, heureusement, nous devrions dire adroitement, le busc de fer qui, à cette époque, défendait comme une cuirasse la poitrine des femmes; il avait glissé en déchirant la robe, et avait pénétré de biais entre la chair et les côtes. 

La robe de Milady n'en fut pas moins tachée de sang en une seconde. 

Milady était tombée à la renverse et semblait évanouie. 

Felton arracha le couteau. 

«Voyez, Milord, dit-il d'un air sombre, voici une femme qui était sous ma garde et qui s'est tuée! 

\speak  Soyez tranquille, Felton, dit Lord de Winter, elle n'est pas morte, les démons ne meurent pas si facilement, soyez tranquille et allez m'attendre chez moi. 

\speak  Mais, Milord\dots 

\speak  Allez, je vous l'ordonne.» 

À cette injonction de son supérieur, Felton obéit; mais, en sortant, il mit le couteau dans sa poitrine. 

Quant à Lord de Winter, il se contenta d'appeler la femme qui servait Milady et, lorsqu'elle fut venue, lui recommandant la prisonnière toujours évanouie, il la laissa seule avec elle. 

Cependant, comme à tout prendre, malgré ses soupçons, la blessure pouvait être grave, il envoya, à l'instant même, un homme à cheval chercher un médecin.