%!TeX root=../musketeersfr.tex 

\chapter{Premiere Journée De Captivité}

\lettrine{R}{evenons} à Milady, qu'un regard jeté sur les côtes de France nous a fait perdre de vue un instant. 
	
\zz
Nous la retrouverons dans la position désespérée où nous l'avons laissée, se creusant un abîme de sombres réflexions, sombre enfer à la porte duquel elle a presque laissé l'espérance: car pour la première fois elle doute, pour la première fois elle craint. 

Dans deux occasions sa fortune lui a manqué, dans deux occasions elle s'est vue découverte et trahie, et dans ces deux occasions, c'est contre le génie fatal envoyé sans doute par le Seigneur pour la combattre qu'elle a échoué: d'Artagnan l'a vaincue, elle, cette invincible puissance du mal. 

Il l'a abusée dans son amour, humiliée dans son orgueil, trompée dans son ambition, et maintenant voilà qu'il la perd dans sa fortune, qu'il l'atteint dans sa liberté, qu'il la menace même dans sa vie. Bien plus, il a levé un coin de son masque, cette égide dont elle se couvre et qui la rend si forte. 

D'Artagnan a détourné de Buckingham, qu'elle hait, comme elle hait tout ce qu'elle a aimé, la tempête dont le menaçait Richelieu dans la personne de la reine. D'Artagnan s'est fait passer pour de Wardes, pour lequel elle avait une de ces fantaisies de tigresse, indomptables comme en ont les femmes de ce caractère. D'Artagnan connaît ce terrible secret qu'elle a juré que nul ne connaîtrait sans mourir. Enfin, au moment où elle vient d'obtenir un blanc-seing à l'aide duquel elle va se venger de son ennemi, le blanc-seing lui est arraché des mains, et c'est d'Artagnan qui la tient prisonnière et qui va l'envoyer dans quelque immonde Botany-Bay, dans quelque Tyburn infâme de l'océan Indien. 

Car tout cela lui vient de d'Artagnan sans doute; de qui viendraient tant de hontes amassées sur sa tête, sinon de lui? Lui seul a pu transmettre à Lord de Winter tous ces affreux secrets, qu'il a découverts les uns après les autres par une sorte de fatalité. Il connaît son beau-frère, il lui aura écrit. 

Que de haine elle distille! Là, immobile, et les yeux ardents et fixes dans son appartement désert, comme les éclats de ses rugissements sourds, qui parfois s'échappent avec sa respiration du fond de sa poitrine, accompagnent bien le bruit de la houle qui monte, gronde, mugit et vient se briser, comme un désespoir éternel et impuissant, contre les rochers sur lesquels est bâti ce château sombre et orgueilleux! Comme, à la lueur des éclairs que sa colère orageuse fait briller dans son esprit, elle conçoit contre Mme Bonacieux, contre Buckingham, et surtout contre d'Artagnan, de magnifiques projets de vengeance, perdus dans les lointains de l'avenir! 

Oui, mais pour se venger il faut être libre, et pour être libre, quand on est prisonnier, il faut percer un mur, desceller des barreaux, trouer un plancher; toutes entreprises que peut mener à bout un homme patient et fort mais devant lesquelles doivent échouer les irritations fébriles d'une femme. D'ailleurs, pour faire tout cela il faut avoir le temps, des mois, des années, et elle\dots elle a dix ou douze jours, à ce que lui a dit Lord de Winter, son fraternel et terrible geôlier. 

Et cependant, si elle était un homme, elle tenterait tout cela, et peut-être réussirait-elle: pourquoi donc le Ciel s'est-il ainsi trompé, en mettant cette âme virile dans ce corps frêle et délicat! 

Aussi les premiers moments de la captivité ont été terribles: quelques convulsions de rage qu'elle n'a pu vaincre ont payé sa dette de faiblesse féminine à la nature. Mais peu à peu elle a surmonté les éclats de sa folle colère, les frémissements nerveux qui ont agité son corps ont disparu, et maintenant elle s'est repliée sur elle-même comme un serpent fatigué qui se repose. 

«Allons, allons; j'étais folle de m'emporter ainsi, dit-elle en plongeant dans la glace, qui reflète dans ses yeux son regard brûlant, par lequel elle semble s'interroger elle-même. Pas de violence, la violence est une preuve de faiblesse. D'abord je n'ai jamais réussi par ce moyen: peut-être, si j'usais de ma force contre des femmes, aurais-je chance de les trouver plus faibles encore que moi, et par conséquent de les vaincre; mais c'est contre des hommes que je lutte, et je ne suis qu'une femme pour eux. Luttons en femme, ma force est dans ma faiblesse.» 

Alors, comme pour se rendre compte à elle-même des changements qu'elle pouvait imposer à sa physionomie si expressive et si mobile, elle lui fit prendre à la fois toutes les expressions, depuis celle de la colère qui crispait ses traits, jusqu'à celle du plus doux, du plus affectueux et du plus séduisant sourire. Puis ses cheveux prirent successivement sous ses mains savantes les ondulations qu'elle crut pouvoir aider aux charmes de son visage. Enfin elle murmura, satisfaite d'elle-même: 

«Allons, rien n'est perdu. Je suis toujours belle.» 

Il était huit heures du soir à peu près. Milady aperçut un lit; elle pensa qu'un repos de quelques heures rafraîchirait non seulement sa tête et ses idées, mais encore son teint. Cependant, avant de se coucher, une idée meilleure lui vint. Elle avait entendu parler de souper. Déjà elle était depuis une heure dans cette chambre, on ne pouvait tarder à lui apporter son repas. La prisonnière ne voulut pas perdre de temps, et elle résolut de faire, dès cette même soirée, quelque tentative pour sonder le terrain, en étudiant le caractère des gens auxquels sa garde était confiée. 

Une lumière apparut sous la porte; cette lumière annonçait le retour de ses geôliers. Milady, qui s'était levée, se rejeta vivement sur son fauteuil, la tête renversée en arrière, ses beaux cheveux dénoués et épars, sa gorge demi-nue sous ses dentelles froissées, une main sur son cœur et l'autre pendante. 

On ouvrit les verrous, la porte grinça sur ses gonds, des pas retentirent dans la chambre et s'approchèrent. 

«Posez là cette table», dit une voix que la prisonnière reconnut pour celle de Felton. 

L'ordre fut exécuté. 

«Vous apporterez des flambeaux et ferez relever la sentinelle», continua Felton. 

Ce double ordre que donna aux mêmes individus le jeune lieutenant prouva à Milady que ses serviteurs étaient les mêmes hommes que ses gardiens, c'est-à-dire des soldats. 

Les ordres de Felton étaient, au reste, exécutés avec une silencieuse rapidité qui donnait une bonne idée de l'état florissant dans lequel il maintenait la discipline. 

Enfin, Felton, qui n'avait pas encore regardé Milady, se retourna vers elle. 

«Ah! ah! dit-il, elle dort, c'est bien: à son réveil elle soupera.» 

Et il fit quelques pas pour sortir. 

«Mais, mon lieutenant, dit un soldat moins stoïque que son chef, et qui s'était approché de Milady, cette femme ne dort pas. 

\speak  Comment, elle ne dort pas? dit Felton, que fait-elle donc, alors? 

\speak  Elle est évanouie; son visage est très pâle, et j'ai beau écouter, je n'entends pas sa respiration. 

\speak  Vous avez raison, dit Felton après avoir regardé Milady de la place où il se trouvait, sans faire un pas vers elle, allez prévenir Lord de Winter que sa prisonnière est évanouie, car je ne sais que faire, le cas n'ayant pas été prévu.» 

Le soldat sortit pour obéir aux ordres de son officier; Felton s'assit sur un fauteuil qui se trouvait par hasard près de la porte et attendit sans dire une parole, sans faire un geste. Milady possédait ce grand art, tant étudié par les femmes, de voir à travers ses longs cils sans avoir l'air d'ouvrir les paupières: elle aperçut Felton qui lui tournait le dos, elle continua de le regarder pendant dix minutes à peu près, et pendant ces dix minutes, l'impassible gardien ne se retourna pas une seule fois. 

Elle songea alors que Lord de Winter allait venir et rendre, par sa présence, une nouvelle force à son geôlier: sa première épreuve était perdue, elle en prit son parti en femme qui compte sur ses ressources; en conséquence elle leva la tête, ouvrit les yeux et soupira faiblement. 

À ce soupir, Felton se retourna enfin. 

«Ah! vous voici réveillée, madame! dit-il, je n'ai donc plus affaire ici! Si vous avez besoin de quelque chose, vous appellerez. 

\speak  Oh! mon Dieu, mon Dieu! que j'ai souffert!» murmura Milady avec cette voix harmonieuse qui, pareille à celle des enchanteresses antiques, charmait tous ceux qu'elle voulait perdre. 

Et elle prit en se redressant sur son fauteuil une position plus gracieuse et plus abandonnée encore que celle qu'elle avait lorsqu'elle était couchée. 

Felton se leva. 

«Vous serez servie ainsi trois fois par jour, madame, dit-il: le matin à neuf heures, dans la journée à une heure, et le soir à huit heures. Si cela ne vous convient pas, vous pouvez indiquer vos heures au lieu de celles que je vous propose, et, sur ce point, on se conformera à vos désirs. 

\speak  Mais vais-je donc rester toujours seule dans cette grande et triste chambre? demanda Milady. 

\speak  Une femme des environs a été prévenue, elle sera demain au château, et viendra toutes les fois que vous désirerez sa présence. 

\speak  Je vous rends grâce, monsieur», répondit humblement la prisonnière. 

Felton fit un léger salut et se dirigea vers la porte. Au moment où il allait en franchir le seuil, Lord de Winter parut dans le corridor, suivi du soldat qui était allé lui porter la nouvelle de l'évanouissement de Milady. Il tenait à la main un flacon de sels. «Eh bien! qu'est-ce? et que se passe-t-il donc ici? dit-il d'une voix railleuse en voyant sa prisonnière debout et Felton prêt à sortir. Cette morte est-elle donc déjà ressuscitée? Pardieu, Felton, mon enfant, tu n'as donc pas vu qu'on te prenait pour un novice et qu'on te jouait le premier acte d'une comédie dont nous aurons sans doute le plaisir de suivre tous les développements? 

\speak  Je l'ai bien pensé, Milord, dit Felton; mais, enfin, comme la prisonnière est femme, après tout, j'ai voulu avoir les égards que tout homme bien né doit à une femme, sinon pour elle, du moins pour lui-même.» 

Milady frissonna par tout son corps. Ces paroles de Felton passaient comme une glace par toutes ses veines. 

«Ainsi, reprit de Winter en riant, ces beaux cheveux savamment étalés, cette peau blanche et ce langoureux regard ne t'ont pas encore séduit, cœur de pierre? 

\speak  Non, Milord, répondit l'impassible jeune homme, et croyez-moi bien, il faut plus que des manèges et des coquetteries de femme pour me corrompre. 

\speak  En ce cas, mon brave lieutenant, laissons Milady chercher autre chose et allons souper; ah! sois tranquille, elle a l'imagination féconde et le second acte de la comédie ne tardera pas à suivre le premier.» 

Et à ces mots Lord de Winter passa son bras sous celui de Felton et l'emmena en riant. 

«Oh! je trouverai bien ce qu'il te faut, murmura Milady entre ses dents; sois tranquille, pauvre moine manqué, pauvre soldat converti qui t'es taillé ton uniforme dans un froc.» 

«À propos, reprit de Winter en s'arrêtant sur le seuil de la porte, il ne faut pas, Milady, que cet échec vous ôte l'appétit. Tâtez de ce poulet et de ces poissons que je n'ai pas fait empoisonner, sur l'honneur. Je m'accommode assez de mon cuisinier, et comme il ne doit pas hériter de moi, j'ai en lui pleine et entière confiance. Faites comme moi. Adieu, chère soeur! à votre prochain évanouissement.» 

C'était tout ce que pouvait supporter Milady: ses mains se crispèrent sur son fauteuil, ses dents grincèrent sourdement, ses yeux suivirent le mouvement de la porte qui se fermait derrière Lord de Winter et Felton; et, lorsqu'elle se vit seule, une nouvelle crise de désespoir la prit; elle jeta les yeux sur la table, vit briller un couteau, s'élança et le saisit; mais son désappointement fut cruel: la lame en était ronde et d'argent flexible. 

Un éclat de rire retentit derrière la porte mal fermée, et la porte se rouvrit. 

«Ah! ah! s'écria Lord de Winter; ah! ah! vois-tu bien, mon brave Felton, vois-tu ce que je t'avais dit: ce couteau, c'était pour toi; mon enfant, elle t'aurait tué; vois-tu, c'est un de ses travers, de se débarrasser ainsi, d'une façon ou de l'autre, des gens qui la gênent. Si je t'eusse écouté, le couteau eût été pointu et d'acier: alors plus de Felton, elle t'aurait égorgé et, après toi, tout le monde. Vois donc, John, comme elle sait bien tenir son couteau.» 

En effet, Milady tenait encore l'arme offensive dans sa main crispée, mais ces derniers mots, cette suprême insulte, détendirent ses mains, ses forces et jusqu'à sa volonté. 

Le couteau tomba par terre. 

«Vous avez raison, Milord, dit Felton avec un accent de profond dégoût qui retentit jusqu'au fond du cœur de Milady, vous avez raison et c'est moi qui avais tort.» 

Et tous deux sortirent de nouveau. 

Mais cette fois, Milady prêta une oreille plus attentive que la première fois, et elle entendit leurs pas s'éloigner et s'éteindre dans le fond du corridor. 

«Je suis perdue, murmura-t-elle, me voilà au pouvoir de gens sur lesquels je n'aurai pas plus de prise que sur des statues de bronze ou de granit; ils me savent par cœur et sont cuirassés contre toutes mes armes. 

«Il est cependant impossible que cela finisse comme ils l'ont décidé.» 

En effet, comme l'indiquait cette dernière réflexion, ce retour instinctif à l'espérance, dans cette âme profonde la crainte et les sentiments faibles ne surnageaient pas longtemps. Milady se mit à table, mangea de plusieurs mets, but un peu de vin d'Espagne, et sentit revenir toute sa résolution. 

Avant de se coucher elle avait déjà commenté, analysé, retourné sur toutes leurs faces, examiné sous tous les points, les paroles, les pas, les gestes, les signes et jusqu'au silence de ses geôliers, et de cette étude profonde, habile et savante, il était résulté que Felton était, à tout prendre, le plus vulnérable de ses deux persécuteurs. 

Un mot surtout revenait à l'esprit de la prisonnière: 

«Si je t'eusse écouté», avait dit Lord de Winter à Felton. 

Donc Felton avait parlé en sa faveur, puisque Lord de Winter n'avait pas voulu écouter Felton. 

«Faible ou forte, répétait Milady, cet homme a donc une lueur de pitié dans son âme; de cette lueur je ferai un incendie qui le dévorera. 

«Quant à l'autre, il me connaît, il me craint et sait ce qu'il a à attendre de moi si jamais je m'échappe de ses mains, il est donc inutile de rien tenter sur lui. Mais Felton, c'est autre chose; c'est un jeune homme naïf, pur et qui semble vertueux; celui-là, il y a moyen de le perdre.» 

Et Milady se coucha et s'endormit le sourire sur les lèvres; quelqu'un qui l'eût vue dormant eût dit une jeune fille rêvant à la couronne de fleurs qu'elle devait mettre sur son front à la prochaine fête. 