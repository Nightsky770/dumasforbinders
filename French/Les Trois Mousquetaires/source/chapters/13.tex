%!TeX root=../musketeersfr.tex 

\chapter{Monsieur Bonacieux}
	
	\lettrine{I}{l} y avait dans tout cela, comme on a pu le remarquer, un personnage dont, malgré sa position précaire, on n'avait paru s'inquiéter que fort médiocrement; ce personnage était M. Bonacieux, respectable martyr des intrigues politiques et amoureuses qui s'enchevêtraient si bien les unes aux autres, dans cette époque à la fois si chevaleresque et si galante. 

Heureusement --- le lecteur se le rappelle ou ne se le rappelle pas --- heureusement que nous avons promis de ne pas le perdre de vue. 

Les estafiers qui l'avaient arrêté le conduisirent droit à la Bastille, où on le fit passer tout tremblant devant un peloton de soldats qui chargeaient leurs mousquets. 

De là, introduit dans une galerie demi-souterraine, il fut, de la part de ceux qui l'avaient amené, l'objet des plus grossières injures et des plus farouches traitements. Les sbires voyaient qu'ils n'avaient pas affaire à un gentilhomme, et ils le traitaient en véritable croquant. 

Au bout d'une demi-heure à peu près, un greffier vint mettre fin à ses tortures, mais non pas à ses inquiétudes, en donnant l'ordre de conduire M. Bonacieux dans la chambre des interrogatoires. Ordinairement on interrogeait les prisonniers chez eux, mais avec M. Bonacieux on n'y faisait pas tant de façons. 

Deux gardes s'emparèrent du mercier, lui firent traverser une cour, le firent entrer dans un corridor où il y avait trois sentinelles, ouvrirent une porte et le poussèrent dans une chambre basse, où il n'y avait pour tous meubles qu'une table, une chaise et un commissaire. Le commissaire était assis sur la chaise et occupé à écrire sur la table. 

Les deux gardes conduisirent le prisonnier devant la table et, sur un signe du commissaire, s'éloignèrent hors de la portée de la voix. 

Le commissaire, qui jusque-là avait tenu sa tête baissée sur ses papiers, la releva pour voir à qui il avait affaire. Ce commissaire était un homme à la mine rébarbative, au nez pointu, aux pommettes jaunes et saillantes, aux yeux petits mais investigateurs et vifs, à la physionomie tenant à la fois de la fouine et du renard. Sa tête, supportée par un cou long et mobile, sortait de sa large robe noire en se balançant avec un mouvement à peu près pareil à celui de la tortue tirant sa tête hors de sa carapace. 

Il commença par demander à M. Bonacieux ses nom et prénoms, son âge, son état et son domicile. 

L'accusé répondit qu'il s'appelait Jacques-Michel Bonacieux, qu'il était âgé de cinquante et un ans, mercier retiré et qu'il demeurait rue des Fossoyeurs, n° 11. 

Le commissaire alors, au lieu de continuer à l'interroger, lui fit un grand discours sur le danger qu'il y a pour un bourgeois obscur à se mêler des choses publiques. 

Il compliqua cet exorde d'une exposition dans laquelle il raconta la puissance et les actes de M. le cardinal, ce ministre incomparable, ce vainqueur des ministres passés, cet exemple des ministres à venir: actes et puissance que nul ne contrecarrait impunément. 

Après cette deuxième partie de son discours, fixant son regard d'épervier sur le pauvre Bonacieux, il l'invita à réfléchir à la gravité de sa situation. 

Les réflexions du mercier étaient toutes faites: il donnait au diable l'instant où M. de La Porte avait eu l'idée de le marier avec sa filleule, et l'instant surtout où cette filleule avait été reçue dame de la lingerie chez la reine. 

Le fond du caractère de maître Bonacieux était un profond égoïsme mêlé à une avarice sordide, le tout assaisonné d'une poltronnerie extrême. L'amour que lui avait inspiré sa jeune femme, étant un sentiment tout secondaire, ne pouvait lutter avec les sentiments primitifs que nous venons d'énumérer. 

Bonacieux réfléchit, en effet, sur ce qu'on venait de lui dire. 

«Mais, monsieur le commissaire, dit-il timidement, croyez bien que je connais et que j'apprécie plus que personne le mérite de l'incomparable Éminence par laquelle nous avons l'honneur d'être gouvernés. 

\speak  Vraiment? demanda le commissaire d'un air de doute; mais s'il en était véritablement ainsi, comment seriez-vous à la Bastille? 

\speak  Comment j'y suis, ou plutôt pourquoi j'y suis, répliqua M. Bonacieux, voilà ce qu'il m'est parfaitement impossible de vous dire, vu que je l'ignore moi-même; mais, à coup sûr, ce n'est pas pour avoir désobligé, sciemment du moins, M. le cardinal. 

\speak  Il faut cependant que vous ayez commis un crime, puisque vous êtes ici accusé de haute trahison. 

\speak  De haute trahison! s'écria Bonacieux épouvanté, de haute trahison! et comment voulez-vous qu'un pauvre mercier qui déteste les huguenots et qui abhorre les Espagnols soit accusé de haute trahison? Réfléchissez, monsieur, la chose est matériellement impossible. 

\speak  Monsieur Bonacieux, dit le commissaire en regardant l'accusé comme si ses petits yeux avaient la faculté de lire jusqu'au plus profond des cœurs, monsieur Bonacieux, vous avez une femme? 

\speak  Oui, monsieur, répondit le mercier tout tremblant, sentant que c'était là où les affaires allaient s'embrouiller; c'est-à-dire, j'en avais une. 

\speak  Comment? vous en aviez une! qu'en avez-vous fait, si vous ne l'avez plus? 

\speak  On me l'a enlevée, monsieur. 

\speak  On vous l'a enlevée? dit le commissaire. Ah!» 

Bonacieux sentit à ce «ah!» que l'affaire s'embrouillait de plus en plus. 

«On vous l'a enlevée! reprit le commissaire, et savez-vous quel est l'homme qui a commis ce rapt? 

\speak  Je crois le connaître. 

\speak  Quel est-il? 

\speak  Songez que je n'affirme rien, monsieur le commissaire, et que je soupçonne seulement. 

\speak  Qui soupçonnez-vous? Voyons, répondez franchement.» 

M. Bonacieux était dans la plus grande perplexité: devait-il tout nier ou tout dire? En niant tout, on pouvait croire qu'il en savait trop long pour avouer; en disant tout, il faisait preuve de bonne volonté. Il se décida donc à tout dire. 

«Je soupçonne, dit-il, un grand brun, de haute mine, lequel a tout à fait l'air d'un grand seigneur; il nous a suivis plusieurs fois, à ce qu'il m'a semblé, quand j'attendais ma femme devant le guichet du Louvre pour la ramener chez moi.» 

Le commissaire parut éprouver quelque inquiétude. 

«Et son nom? dit-il. 

\speak  Oh! quant à son nom, je n'en sais rien, mais si je le rencontre jamais, je le reconnaîtrai à l'instant même, je vous en réponds, fût-il entre mille personnes.» 

Le front du commissaire se rembrunit. 

«Vous le reconnaîtriez entre mille, dites-vous? continua-t-il\dots 

\speak  C'est-à-dire, reprit Bonacieux, qui vit qu'il avait fait fausse route, c'est-à-dire\dots 

\speak  Vous avez répondu que vous le reconnaîtriez, dit le commissaire; c'est bien, en voici assez pour aujourd'hui; il faut, avant que nous allions plus loin, que quelqu'un soit prévenu que vous connaissez le ravisseur de votre femme. 

\speak  Mais je ne vous ai pas dit que je le connaissais! s'écria Bonacieux au désespoir. Je vous ai dit au contraire\dots 

\speak  Emmenez le prisonnier, dit le commissaire aux deux gardes. 

\speak  Et où faut-il le conduire? demanda le greffier. 

\speak  Dans un cachot. 

\speak  Dans lequel? 

\speak  Oh! mon Dieu, dans le premier venu, pourvu qu'il ferme bien», répondit le commissaire avec une indifférence qui pénétra d'horreur le pauvre Bonacieux. 

«Hélas! hélas! se dit-il, le malheur est sur ma tête; ma femme aura commis quelque crime effroyable; on me croit son complice, et l'on me punira avec elle: elle en aura parlé, elle aura avoué qu'elle m'avait tout dit; une femme, c'est si faible! Un cachot, le premier venu! c'est cela! une nuit est bientôt passée; et demain, à la roue, à la potence! Oh! mon Dieu! mon Dieu! ayez pitié de moi!» 

Sans écouter le moins du monde les lamentations de maître Bonacieux, lamentations auxquelles d'ailleurs ils devaient être habitués, les deux gardes prirent le prisonnier par un bras, et l'emmenèrent, tandis que le commissaire écrivait en hâte une lettre que son greffier attendait. 

Bonacieux ne ferma pas l'œil, non pas que son cachot fût par trop désagréable, mais parce que ses inquiétudes étaient trop grandes. Il resta toute la nuit sur son escabeau, tressaillant au moindre bruit; et quand les premiers rayons du jour se glissèrent dans sa chambre, l'aurore lui parut avoir pris des teintes funèbres. 

Tout à coup, il entendit tirer les verrous, et il fit un soubresaut terrible. Il croyait qu'on venait le chercher pour le conduire à l'échafaud; aussi, lorsqu'il vit purement et simplement paraître, au lieu de l'exécuteur qu'il attendait, son commissaire et son greffier de la veille, il fut tout près de leur sauter au cou. 

«Votre affaire s'est fort compliquée depuis hier au soir, mon brave homme, lui dit le commissaire, et je vous conseille de dire toute la vérité; car votre repentir peut seul conjurer la colère du cardinal. 

\speak  Mais je suis prêt à tout dire, s'écria Bonacieux, du moins tout ce que je sais. Interrogez, je vous prie. 

\speak  Où est votre femme, d'abord? 

\speak  Mais puisque je vous ai dit qu'on me l'avait enlevée. 

\speak  Oui, mais depuis hier cinq heures de l'après-midi, grâce à vous, elle s'est échappée. 

\speak  Ma femme s'est échappée! s'écria Bonacieux. Oh! la malheureuse! monsieur, si elle s'est échappée, ce n'est pas ma faute, je vous le jure. 

\speak  Qu'alliez-vous donc alors faire chez M. d'Artagnan votre voisin, avec lequel vous avez eu une longue conférence dans la journée? 

\speak  Ah! oui, monsieur le commissaire, oui, cela est vrai, et j'avoue que j'ai eu tort. J'ai été chez M. d'Artagnan. 

\speak  Quel était le but de cette visite? 

\speak  De le prier de m'aider à retrouver ma femme. Je croyais que j'avais droit de la réclamer; je me trompais, à ce qu'il paraît, et je vous en demande bien pardon. 

\speak  Et qu'a répondu M. d'Artagnan? 

\speak  M. d'Artagnan m'a promis son aide; mais je me suis bientôt aperçu qu'il me trahissait. 

\speak  Vous en imposez à la justice! M. d'Artagnan a fait un pacte avec vous, et en vertu de ce pacte il a mis en fuite les hommes de police qui avaient arrêté votre femme, et l'a soustraite à toutes les recherches. 

\speak  M. d'Artagnan a enlevé ma femme! Ah çà, mais que me dites-vous là? 

\speak  Heureusement M. d'Artagnan est entre nos mains, et vous allez lui être confronté. 

\speak  Ah! ma foi, je ne demande pas mieux, s'écria Bonacieux; je ne serais pas fâché de voir une figure de connaissance. 

\speak  Faites entrer M. d'Artagnan», dit le commissaire aux deux gardes. 

Les deux gardes firent entrer Athos. 

«Monsieur d'Artagnan, dit le commissaire en s'adressant à Athos, déclarez ce qui s'est passé entre vous et monsieur. 

\speak  Mais! s'écria Bonacieux, ce n'est pas M. d'Artagnan que vous me montrez là! 

\speak  Comment! ce n'est pas M. d'Artagnan? s'écria le commissaire. 

\speak  Pas le moins du monde, répondit Bonacieux. 

\speak  Comment se nomme monsieur? demanda le commissaire. 

\speak  Je ne puis vous le dire, je ne le connais pas. 

\speak  Comment! vous ne le connaissez pas? 

\speak  Non. 

\speak  Vous ne l'avez jamais vu? 

\speak  Si fait; mais je ne sais comment il s'appelle. 

\speak  Votre nom? demanda le commissaire. 

\speak  Athos, répondit le mousquetaire. 

\speak  Mais ce n'est pas un nom d'homme, ça, c'est un nom de montagne! s'écria le pauvre interrogateur qui commençait à perdre la tête. 

\speak  C'est mon nom, dit tranquillement Athos. 

\speak  Mais vous avez dit que vous vous nommiez d'Artagnan. 

\speak  Moi? 

\speak  Oui, vous. 

\speak  C'est-à-dire que c'est à moi qu'on a dit: «Vous êtes M. d'Artagnan?» J'ai répondu: «Vous croyez?» Mes gardes se sont écriés qu'ils en étaient sûrs. Je n'ai pas voulu les contrarier. D'ailleurs je pouvais me tromper. 

\speak  Monsieur, vous insultez à la majesté de la justice. 

\speak  Aucunement, fit tranquillement Athos. 

\speak  Vous êtes M. d'Artagnan. 

\speak  Vous voyez bien que vous me le dites encore. 

\speak  Mais, s'écria à son tour M. Bonacieux, je vous dis, monsieur le commissaire, qu'il n'y a pas un instant de doute à avoir. M. d'Artagnan est mon hôte, et par conséquent, quoiqu'il ne me paie pas mes loyers, et justement même à cause de cela, je dois le connaître. M. d'Artagnan est un jeune homme de dix-neuf à vingt ans à peine, et monsieur en a trente au moins. M. d'Artagnan est dans les gardes de M. des Essarts, et monsieur est dans la compagnie des mousquetaires de M. de Tréville: regardez l'uniforme, monsieur le commissaire, regardez l'uniforme. 

\speak  C'est vrai, murmura le commissaire; c'est pardieu vrai.» 

En ce moment la porte s'ouvrit vivement, et un messager, introduit par un des guichetiers de la Bastille, remit une lettre au commissaire. 

«Oh! la malheureuse! s'écria le commissaire. 

\speak  Comment? que dites-vous? de qui parlez-vous? Ce n'est pas de ma femme, j'espère! 

\speak  Au contraire, c'est d'elle. Votre affaire est bonne, allez. 

\speak  Ah çà, s'écria le mercier exaspéré, faites-moi le plaisir de me dire, monsieur, comment mon affaire à moi peut s'empirer de ce que fait ma femme pendant que je suis en prison! 

\speak  Parce que ce qu'elle fait est la suite d'un plan arrêté entre vous, plan infernal! 

\speak  Je vous jure, monsieur le commissaire, que vous êtes dans la plus profonde erreur, que je ne sais rien au monde de ce que devait faire ma femme, que je suis entièrement étranger à ce qu'elle a fait, et que, si elle a fait des sottises, je la renie, je la démens, je la maudis. 

\speak  Ah çà, dit Athos au commissaire, si vous n'avez plus besoin de moi ici, renvoyez-moi quelque part, il est très ennuyeux, votre monsieur Bonacieux. 

\speak  Reconduisez les prisonniers dans leurs cachots, dit le commissaire en désignant d'un même geste Athos et Bonacieux, et qu'ils soient gardés plus sévèrement que jamais. 

\speak  Cependant, dit Athos avec son calme habituel, si c'est à M. d'Artagnan que vous avez affaire, je ne vois pas trop en quoi je puis le remplacer. 

\speak  Faites ce que j'ai dit! s'écria le commissaire, et le secret le plus absolu! Vous entendez!» 

Athos suivit ses gardes en levant les épaules, et M. Bonacieux en poussant des lamentations à fendre le cœur d'un tigre. 

On ramena le mercier dans le même cachot où il avait passé la nuit, et l'on l'y laissa toute la journée. Toute la journée Bonacieux pleura comme un véritable mercier, n'étant pas du tout homme d'épée, il nous l'a dit lui-même. 

Le soir, vers les neuf heures, au moment où il allait se décider à se mettre au lit, il entendit des pas dans son corridor. Ces pas se rapprochèrent de son cachot, sa porte s'ouvrit, des gardes parurent. 

«Suivez-moi, dit un exempt qui venait à la suite des gardes. 

\speak  Vous suivre! s'écria Bonacieux; vous suivre à cette heure-ci! et où cela, mon Dieu? 

\speak  Où nous avons l'ordre de vous conduire. 

\speak  Mais ce n'est pas une réponse, cela. 

\speak  C'est cependant la seule que nous puissions vous faire. 

\speak  Ah! mon Dieu, mon Dieu, murmura le pauvre mercier, pour cette fois je suis perdu!» 

Et il suivit machinalement et sans résistance les gardes qui venaient le quérir. 

Il prit le même corridor qu'il avait déjà pris, traversa une première cour, puis un second corps de logis; enfin, à la porte de la cour d'entrée, il trouva une voiture entourée de quatre gardes à cheval. On le fit monter dans cette voiture, l'exempt se plaça près de lui, on ferma la portière à clef, et tous deux se trouvèrent dans une prison roulante. 

La voiture se mit en mouvement, lente comme un char funèbre. À travers la grille cadenassée, le prisonnier apercevait les maisons et le pavé, voilà tout; mais, en véritable Parisien qu'il était, Bonacieux reconnaissait chaque rue aux bornes, aux enseignes, aux réverbères. Au moment d'arriver à Saint-Paul, lieu où l'on exécutait les condamnés de la Bastille, il faillit s'évanouir et se signa deux fois. Il avait cru que la voiture devait s'arrêter là. La voiture passa cependant. 

Plus loin, une grande terreur le prit encore, ce fut en côtoyant le cimetière Saint-Jean où on enterrait les criminels d'État. Une seule chose le rassura un peu, c'est qu'avant de les enterrer on leur coupait généralement la tête, et que sa tête à lui était encore sur ses épaules. Mais lorsqu'il vit que la voiture prenait la route de la Grève, qu'il aperçut les toits aigus de l'hôtel de ville, que la voiture s'engagea sous l'arcade, il crut que tout était fini pour lui, voulut se confesser à l'exempt, et, sur son refus, poussa des cris si pitoyables que l'exempt annonça que, s'il continuait à l'assourdir ainsi, il lui mettrait un bâillon. 

Cette menace rassura quelque peu Bonacieux: si l'on eût dû l'exécuter en Grève, ce n'était pas la peine de le bâillonner, puisqu'on était presque arrivé au lieu de l'exécution. En effet, la voiture traversa la place fatale sans s'arrêter. Il ne restait plus à craindre que la Croix-du-Trahoir: la voiture en prit justement le chemin. 

Cette fois, il n'y avait plus de doute, c'était à la Croix-du-Trahoir qu'on exécutait les criminels subalternes. Bonacieux s'était flatté en se croyant digne de Saint-Paul ou de la place de Grève: c'était à la Croix-du-Trahoir qu'allaient finir son voyage et sa destinée! Il ne pouvait voir encore cette malheureuse croix, mais il la sentait en quelque sorte venir au-devant de lui. Lorsqu'il n'en fut plus qu'à une vingtaine de pas, il entendit une rumeur, et la voiture s'arrêta. C'était plus que n'en pouvait supporter le pauvre Bonacieux, déjà écrasé par les émotions successives qu'il avait éprouvées; il poussa un faible gémissement, qu'on eût pu prendre pour le dernier soupir d'un moribond, et il s'évanouit.