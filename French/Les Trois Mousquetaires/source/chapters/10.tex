%!TeX root=../musketeersfr.tex 

\chapter{Une Souricière Au XVII\ieme\ Siècle} 
	
	\lettrine{L}{'invention} de la souricière ne date pas de nos jours; dès que les sociétés, en se formant, eurent inventé une police quelconque, cette police, à son tour, inventa les souricières. 

\zz
Comme peut-être nos lecteurs ne sont pas familiarisés encore avec l'argot de la rue de Jérusalem, et que c'est, depuis que nous écrivons --- et il y a quelque quinze ans de cela ---, la première fois que nous employons ce mot appliqué à cette chose, expliquons-leur ce que c'est qu'une souricière. 

Quand, dans une maison quelle qu'elle soit, on a arrêté un individu soupçonné d'un crime quelconque, on tient secrète l'arrestation; on place quatre ou cinq hommes en embuscade dans la première pièce, on ouvre la porte à tous ceux qui frappent, on la referme sur eux et on les arrête; de cette façon, au bout de deux ou trois jours, on tient à peu près tous les familiers de l'établissement. 

Voilà ce que c'est qu'une souricière. 

On fit donc une souricière de l'appartement de maître Bonacieux, et quiconque y apparut fut pris et interrogé par les gens de M. le cardinal. Il va sans dire que, comme une allée particulière conduisait au premier étage qu'habitait d'Artagnan, ceux qui venaient chez lui étaient exemptés de toutes visites. 

D'ailleurs les trois mousquetaires y venaient seuls; ils s'étaient mis en quête chacun de son côté, et n'avaient rien trouvé, rien découvert. Athos avait été même jusqu'à questionner M. de Tréville, chose qui, vu le mutisme habituel du digne mousquetaire, avait fort étonné son capitaine. Mais M. de Tréville ne savait rien, sinon que, la dernière fois qu'il avait vu le cardinal, le roi et la reine, le cardinal avait l'air fort soucieux, que le roi était inquiet, et que les yeux rouges de la reine indiquaient qu'elle avait veillé ou pleuré. Mais cette dernière circonstance l'avait peu frappé, la reine, depuis son mariage, veillant et pleurant beaucoup. 

M. de Tréville recommanda en tout cas à Athos le service du roi et surtout celui de la reine, le priant de faire la même recommandation à ses camarades. 

Quant à d'Artagnan, il ne bougeait pas de chez lui. Il avait converti sa chambre en observatoire. Des fenêtres il voyait arriver ceux qui venaient se faire prendre; puis, comme il avait ôté les carreaux du plancher, qu'il avait creusé le parquet et qu'un simple plafond le séparait de la chambre au-dessous, où se faisaient les interrogatoires, il entendait tout ce qui se passait entre les inquisiteurs et les accusés. 

Les interrogatoires, précédés d'une perquisition minutieuse opérée sur la personne arrêtée, étaient presque toujours ainsi conçus: 

«Mme Bonacieux vous a-t-elle remis quelque chose pour son mari ou pour quelque autre personne? 

\speak  M. Bonacieux vous a-t-il remis quelque chose pour sa femme ou pour quelque autre personne? 

\speak  L'un et l'autre vous ont-ils fait quelque confidence de vive voix?» 

«S'ils savaient quelque chose, ils ne questionneraient pas ainsi, se dit à lui-même d'Artagnan. Maintenant, que cherchent-ils à savoir? Si le duc de Buckingham ne se trouve point à Paris et s'il n'a pas eu ou s'il ne doit point avoir quelque entrevue avec la reine.» 

D'Artagnan s'arrêta à cette idée, qui, d'après tout ce qu'il avait entendu, ne manquait pas de probabilité. 

En attendant, la souricière était en permanence, et la vigilance de d'Artagnan aussi. 

Le soir du lendemain de l'arrestation du pauvre Bonacieux, comme Athos venait de quitter d'Artagnan pour se rendre chez M. de Tréville, comme neuf heures venaient de sonner, et comme Planchet, qui n'avait pas encore fait le lit, commençait sa besogne, on entendit frapper à la porte de la rue; aussitôt cette porte s'ouvrit et se referma: quelqu'un venait de se prendre à la souricière. 

D'Artagnan s'élança vers l'endroit décarrelé, se coucha ventre à terre et écouta. 

Des cris retentirent bientôt, puis des gémissements qu'on cherchait à étouffer. D'interrogatoire, il n'en était pas question. 

«Diable! se dit d'Artagnan, il me semble que c'est une femme: on la fouille, elle résiste, --- on la violente, --- les misérables!» 

Et d'Artagnan, malgré sa prudence, se tenait à quatre pour ne pas se mêler à la scène qui se passait au-dessous de lui. 

«Mais je vous dis que je suis la maîtresse de la maison, messieurs; je vous dis que je suis Mme Bonacieux, je vous dis que j'appartiens à la reine!» s'écriait la malheureuse femme. 

«Mme Bonacieux! murmura d'Artagnan; serais-je assez heureux pour avoir trouvé ce que tout le monde cherche?» 

«C'est justement vous que nous attendions», reprirent les interrogateurs. 

La voix devint de plus en plus étouffée: un mouvement tumultueux fit retentir les boiseries. La victime résistait autant qu'une femme peut résister à quatre hommes. 

«Pardon, messieurs, par\dots», murmura la voix, qui ne fit plus entendre que des sons inarticulés. 

«Ils la bâillonnent, ils vont l'entraîner, s'écria d'Artagnan en se redressant comme par un ressort. Mon épée; bon, elle est à mon côté. Planchet! 

\speak  Monsieur? 

\speak  Cours chercher Athos, Porthos et Aramis. L'un des trois sera sûrement chez lui, peut-être tous les trois seront-ils rentrés. Qu'ils prennent des armes, qu'ils viennent, qu'ils accourent. Ah! je me souviens, Athos est chez M. de Tréville. 

\speak  Mais où allez-vous, monsieur, où allez-vous? 

\speak  Je descends par la fenêtre, s'écria d'Artagnan, afin d'être plus tôt arrivé; toi, remets les carreaux, balaie le plancher, sors par la porte et cours où je te dis. 

\speak  Oh! monsieur, monsieur, vous allez vous tuer, s'écria Planchet. 

\speak  Tais-toi, imbécile», dit d'Artagnan. Et s'accrochant de la main au rebord de sa fenêtre, il se laissa tomber du premier étage, qui heureusement n'était pas élevé, sans se faire une écorchure. 

Puis il alla aussitôt frapper à la porte en murmurant: 

«Je vais me faire prendre à mon tour dans la souricière, et malheur aux chats qui se frotteront à pareille souris.» 

À peine le marteau eut-il résonné sous la main du jeune homme, que le tumulte cessa, que des pas s'approchèrent, que la porte s'ouvrit, et que d'Artagnan, l'épée nue, s'élança dans l'appartement de maître Bonacieux, dont la porte, sans doute mue par un ressort, se referma d'elle-même sur lui. 

Alors ceux qui habitaient encore la malheureuse maison de Bonacieux et les voisins les plus proches entendirent de grands cris, des trépignements, un cliquetis d'épées et un bruit prolongé de meubles. Puis, un moment après, ceux qui, surpris par ce bruit, s'étaient mis aux fenêtres pour en connaître la cause, purent voir la porte se rouvrir et quatre hommes vêtus de noir non pas en sortir, mais s'envoler comme des corbeaux effarouchés, laissant par terre et aux angles des tables des plumes de leurs ailes, c'est-à-dire des loques de leurs habits et des bribes de leurs manteaux. 

D'Artagnan était vainqueur sans beaucoup de peine, il faut le dire, car un seul des alguazils était armé, encore se défendit-il pour la forme. Il est vrai que les trois autres avaient essayé d'assommer le jeune homme avec les chaises, les tabourets et les poteries; mais deux ou trois égratignures faites par la flamberge du Gascon les avaient épouvantés. Dix minutes avaient suffi à leur défaite et d'Artagnan était resté maître du champ de bataille. 

Les voisins, qui avaient ouvert leurs fenêtres avec le sang-froid particulier aux habitants de Paris dans ces temps d'émeutes et de rixes perpétuelles, les refermèrent dès qu'ils eurent vu s'enfuir les quatre hommes noirs: leur instinct leur disait que, pour le moment, tout était fini. 

D'ailleurs il se faisait tard, et alors comme aujourd'hui on se couchait de bonne heure dans le quartier du Luxembourg. 

D'Artagnan, resté seul avec Mme Bonacieux, se retourna vers elle: la pauvre femme était renversée sur un fauteuil et à demi évanouie. D'Artagnan l'examina d'un coup d'œil rapide. 

C'était une charmante femme de vingt-cinq à vingt-six ans, brune avec des yeux bleus, ayant un nez légèrement retroussé, des dents admirables, un teint marbré de rose et d'opale. Là cependant s'arrêtaient les signes qui pouvaient la faire confondre avec une grande dame. Les mains étaient blanches, mais sans finesse: les pieds n'annonçaient pas la femme de qualité. Heureusement d'Artagnan n'en était pas encore à se préoccuper de ces détails. 

Tandis que d'Artagnan examinait Mme Bonacieux, et en était aux pieds, comme nous l'avons dit, il vit à terre un fin mouchoir de batiste, qu'il ramassa selon son habitude, et au coin duquel il reconnut le même chiffre qu'il avait vu au mouchoir qui avait failli lui faire couper la gorge avec Aramis. 

Depuis ce temps, d'Artagnan se méfiait des mouchoirs armoriés; il remit donc sans rien dire celui qu'il avait ramassé dans la poche de Mme Bonacieux. En ce moment, Mme Bonacieux reprenait ses sens. Elle ouvrit les yeux, regarda avec terreur autour d'elle, vit que l'appartement était vide, et qu'elle était seule avec son libérateur. Elle lui tendit aussitôt les mains en souriant. Mme Bonacieux avait le plus charmant sourire du monde. 

«Ah! monsieur! dit-elle, c'est vous qui m'avez sauvée; permettez-moi que je vous remercie. 

\speak  Madame, dit d'Artagnan, je n'ai fait que ce que tout gentilhomme eût fait à ma place, vous ne me devez donc aucun remerciement. 

\speak  Si fait, monsieur, si fait, et j'espère vous prouver que vous n'avez pas rendu service à une ingrate. Mais que me voulaient donc ces hommes, que j'ai pris d'abord pour des voleurs, et pourquoi M. Bonacieux n'est-il point ici? 

\speak  Madame, ces hommes étaient bien autrement dangereux que ne pourraient être des voleurs, car ce sont des agents de M. le cardinal, et quant à votre mari, M. Bonacieux, il n'est point ici parce qu'hier on est venu le prendre pour le conduire à la Bastille. 

\speak  Mon mari à la Bastille! s'écria Mme Bonacieux, oh! mon Dieu! qu'a-t-il donc fait? pauvre cher homme! lui, l'innocence même!» 

Et quelque chose comme un sourire perçait sur la figure encore tout effrayée de la jeune femme. 

«Ce qu'il a fait, madame? dit d'Artagnan. Je crois que son seul crime est d'avoir à la fois le bonheur et le malheur d'être votre mari. 

\speak  Mais, monsieur, vous savez donc\dots 

\speak  Je sais que vous avez été enlevée, madame. 

\speak  Et par qui? Le savez-vous? Oh! si vous le savez, dites-le-moi. 

\speak  Par un homme de quarante à quarante-cinq ans, aux cheveux noirs, au teint basané, avec une cicatrice à la tempe gauche. 

\speak  C'est cela, c'est cela; mais son nom? 

\speak  Ah! son nom? c'est ce que j'ignore. 

\speak  Et mon mari savait-il que j'avais été enlevée? 

\speak  Il en avait été prévenu par une lettre que lui avait écrite le ravisseur lui-même. 

\speak  Et soupçonne-t-il, demanda Mme Bonacieux avec embarras, la cause de cet événement? 

\speak  Il l'attribuait, je crois, à une cause politique. 

\speak  J'en ai douté d'abord, et maintenant je le pense comme lui. Ainsi donc, ce cher M. Bonacieux ne m'a pas soupçonnée un seul instant\dots? 

\speak  Ah! loin de là, madame, il était trop fier de votre sagesse et surtout de votre amour.» 

Un second sourire presque imperceptible effleura les lèvres rosées de la belle jeune femme. 

«Mais, continua d'Artagnan, comment vous êtes-vous enfuie? 

\speak  J'ai profité d'un moment où l'on m'a laissée seule, et comme je savais depuis ce matin à quoi m'en tenir sur mon enlèvement, à l'aide de mes draps je suis descendue par la fenêtre; alors, comme je croyais mon mari ici, je suis accourue. 

\speak  Pour vous mettre sous sa protection? 

\speak  Oh! non, pauvre cher homme, je savais bien qu'il était incapable de me défendre; mais comme il pouvait nous servir à autre chose, je voulais le prévenir. 

\speak  De quoi? 

\speak  Oh! ceci n'est pas mon secret, je ne puis donc pas vous le dire. 

\speak  D'ailleurs, dit d'Artagnan (pardon, madame, si, tout garde que je suis, je vous rappelle à la prudence), d'ailleurs je crois que nous ne sommes pas ici en lieu opportun pour faire des confidences. Les hommes que j'ai mis en fuite vont revenir avec main-forte; s'ils nous retrouvent ici nous sommes perdus. J'ai bien fait prévenir trois de mes amis, mais qui sait si on les aura trouvés chez eux! 

\speak  Oui, oui, vous avez raison, s'écria Mme Bonacieux effrayée; fuyons, sauvons-nous.» 

À ces mots, elle passa son bras sous celui de d'Artagnan et l'entraîna vivement. 

«Mais où fuir? dit d'Artagnan, où nous sauver? 

\speak  Éloignons-nous d'abord de cette maison, puis après nous verrons.» 

Et la jeune femme et le jeune homme, sans se donner la peine de refermer la porte, descendirent rapidement la rue des Fossoyeurs, s'engagèrent dans la rue des Fossés-Monsieur-le-Prince et ne s'arrêtèrent qu'à la place Saint-Sulpice. 

«Et maintenant, qu'allons-nous faire, demanda d'Artagnan, et où voulez-vous que je vous conduise? 

\speak  Je suis fort embarrassée de vous répondre, je vous l'avoue, dit Mme Bonacieux; mon intention était de faire prévenir M. de La Porte par mon mari, afin que M. de La Porte pût nous dire précisément ce qui s'était passé au Louvre depuis trois jours, et s'il n'y avait pas danger pour moi de m'y présenter. 

\speak  Mais moi, dit d'Artagnan, je puis aller prévenir M. de La Porte. 

\speak  Sans doute; seulement il n'y a qu'un malheur: c'est qu'on connaît M. Bonacieux au Louvre et qu'on le laisserait passer, lui, tandis qu'on ne vous connaît pas, vous, et que l'on vous fermera la porte. 

\speak  Ah! bah, dit d'Artagnan, vous avez bien à quelque guichet du Louvre un concierge qui vous est dévoué, et qui grâce à un mot d'ordre\dots» 

Mme Bonacieux regarda fixement le jeune homme. 

«Et si je vous donnais ce mot d'ordre, dit-elle, l'oublieriez-vous aussitôt que vous vous en seriez servi? 

\speak  Parole d'honneur, foi de gentilhomme! dit d'Artagnan avec un accent à la vérité duquel il n'y avait pas à se tromper. 

\speak  Tenez, je vous crois; vous avez l'air d'un brave jeune homme, d'ailleurs votre fortune est peut-être au bout de votre dévouement. 

\speak  Je ferai sans promesse et de conscience tout ce que je pourrai pour servir le roi et être agréable à la reine, dit d'Artagnan; disposez donc de moi comme d'un ami. 

\speak  Mais moi, où me mettrez-vous pendant ce temps-là? 

\speak  N'avez-vous pas une personne chez laquelle M. de La Porte puisse revenir vous prendre? 

\speak  Non, je ne veux me fier à personne. 

\speak  Attendez, dit d'Artagnan; nous sommes à la porte d'Athos. Oui, c'est cela. 

\speak  Qu'est-ce qu'Athos? 

\speak  Un de mes amis. 

\speak  Mais s'il est chez lui et qu'il me voie? 

\speak  Il n'y est pas, et j'emporterai la clef après vous avoir fait entrer dans son appartement. 

\speak  Mais s'il revient? 

\speak  Il ne reviendra pas; d'ailleurs on lui dirait que j'ai amené une femme, et que cette femme est chez lui. 

\speak  Mais cela me compromettra très fort, savez-vous! 

\speak  Que vous importe! on ne vous connaît pas; d'ailleurs nous sommes dans une situation à passer par-dessus quelques convenances! 

\speak  Allons donc chez votre ami. Où demeure-t-il? 

\speak  Rue Férou, à deux pas d'ici. 

\speak  Allons.» 

Et tous deux reprirent leur course. Comme l'avait prévu d'Artagnan, Athos n'était pas chez lui: il prit la clef, qu'on avait l'habitude de lui donner comme à un ami de la maison, monta l'escalier et introduisit Mme Bonacieux dans le petit appartement dont nous avons déjà fait la description. 

«Vous êtes chez vous, dit-il; attendez, fermez la porte en dedans et n'ouvrez à personne, à moins que vous n'entendiez frapper trois coups ainsi: tenez; et il frappa trois fois: deux coups rapprochés l'un de l'autre et assez forts, un coup plus distant et plus léger. 

\speak  C'est bien, dit Mme Bonacieux; maintenant, à mon tour de vous donner mes instructions. 

\speak  J'écoute. 

\speak  Présentez-vous au guichet du Louvre, du côté de la rue de l'Échelle, et demandez Germain. 

\speak  C'est bien. Après? 

\speak  Il vous demandera ce que vous voulez, et alors vous lui répondrez par ces deux mots: Tours et Bruxelles. Aussitôt il se mettra à vos ordres. 

\speak  Et que lui ordonnerai-je? 

\speak  D'aller chercher M. de La Porte, le valet de chambre de la reine. 

\speak  Et quand il l'aura été chercher et que M. de La Porte sera venu? 

\speak  Vous me l'enverrez. 

\speak  C'est bien, mais où et comment vous reverrai-je? 

\speak  Y tenez-vous beaucoup à me revoir? 

\speak  Certainement. 

\speak  Eh bien, reposez-vous sur moi de ce soin, et soyez tranquille. 

\speak  Je compte sur votre parole. 

\speak  Comptez-y.» 

D'Artagnan salua Mme Bonacieux en lui lançant le coup d'œil le plus amoureux qu'il lui fût possible de concentrer sur sa charmante petite personne, et tandis qu'il descendait l'escalier, il entendit la porte se fermer derrière lui à double tour. En deux bonds il fut au Louvre: comme il entrait au guichet de Échelle, dix heures sonnaient. Tous les événements que nous venons de raconter s'étaient succédé en une demi-heure. 

Tout s'exécuta comme l'avait annoncé Mme Bonacieux. Au mot d'ordre convenu, Germain s'inclina; dix minutes après, La Porte était dans la loge; en deux mots, d'Artagnan le mit au fait et lui indiqua où était Mme Bonacieux. La Porte s'assura par deux fois de l'exactitude de l'adresse, et partit en courant. Cependant, à peine eut-il fait dix pas, qu'il revint. 

«Jeune homme, dit-il à d'Artagnan, un conseil. 

\speak  Lequel? 

\speak  Vous pourriez être inquiété pour ce qui vient de se passer. 

\speak  Vous croyez? 

\speak  Oui. Avez-vous quelque ami dont la pendule retarde? 

\speak  Eh bien? 

\speak  Allez le voir pour qu'il puisse témoigner que vous étiez chez lui à neuf heures et demie. En justice, cela s'appelle un alibi.» 

D'Artagnan trouva le conseil prudent; il prit ses jambes à son cou, il arriva chez M. de Tréville, mais, au lieu de passer au salon avec tout le monde, il demanda à entrer dans son cabinet. Comme d'Artagnan était un des habitués de l'hôtel, on ne fit aucune difficulté d'accéder à sa demande; et l'on alla prévenir M. de Tréville que son jeune compatriote, ayant quelque chose d'important à lui dire, sollicitait une audience particulière. Cinq minutes après, M. de Tréville demandait à d'Artagnan ce qu'il pouvait faire pour son service et ce qui lui valait sa visite à une heure si avancée. 

«Pardon, monsieur! dit d'Artagnan, qui avait profité du moment où il était resté seul pour retarder l'horloge de trois quarts d'heure; j'ai pensé que, comme il n'était que neuf heures vingt-cinq minutes, il était encore temps de me présenter chez vous. 

\speak  Neuf heures vingt-cinq minutes! s'écria M. de Tréville en regardant sa pendule; mais c'est impossible! 

\speak  Voyez plutôt, monsieur, dit d'Artagnan, voilà qui fait foi. 

\speak  C'est juste, dit M. de Tréville, j'aurais cru qu'il était plus tard. Mais voyons, que me voulez-vous?» 

Alors d'Artagnan fit à M. de Tréville une longue histoire sur la reine. Il lui exposa les craintes qu'il avait conçues à l'égard de Sa Majesté; il lui raconta ce qu'il avait entendu dire des projets du cardinal à l'endroit de Buckingham, et tout cela avec une tranquillité et un aplomb dont M. de Tréville fut d'autant mieux la dupe, que lui-même, comme nous l'avons dit, avait remarqué quelque chose de nouveau entre le cardinal, le roi et la reine. 

À dix heures sonnant, d'Artagnan quitta M. de Tréville, qui le remercia de ses renseignements, lui recommanda d'avoir toujours à cœur le service du roi et de la reine, et qui rentra dans le salon. Mais, au bas de l'escalier, d'Artagnan se souvint qu'il avait oublié sa canne: en conséquence, il remonta précipitamment, rentra dans le cabinet, d'un tour de doigt remit la pendule à son heure, pour qu'on ne pût pas s'apercevoir, le lendemain, qu'elle avait été dérangée, et sûr désormais qu'il y avait un témoin pour prouver son alibi, il descendit l'escalier et se trouva bientôt dans la rue.