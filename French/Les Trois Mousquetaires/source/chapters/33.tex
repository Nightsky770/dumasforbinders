%!TeX root=../musketeersfr.tex 

\chapter{Soubrette Et Maîtresse}

\lettrine{C}{ependant,} comme nous l'avons dit, malgré les cris de sa conscience et les sages conseils d'Athos, d'Artagnan devenait d'heure en heure plus amoureux de Milady; aussi ne manquait-il pas tous les jours d'aller lui faire une cour à laquelle l'aventureux Gascon était convaincu qu'elle ne pouvait, tôt ou tard, manquer de répondre. 

Un soir qu'il arrivait le nez au vent, léger comme un homme qui attend une pluie d'or, il rencontra la soubrette sous la porte cochère; mais cette fois la jolie Ketty ne se contenta point de lui sourire en passant, elle lui prit doucement la main. 

«Bon! fit d'Artagnan, elle est chargée de quelque message pour moi de la part de sa maîtresse; elle va m'assigner quelque rendez-vous qu'on n'aura pas osé me donner de vive voix.» 

Et il regarda la belle enfant de l'air le plus vainqueur qu'il put prendre. 

«Je voudrais bien vous dire deux mots, monsieur le chevalier\dots, balbutia la soubrette. 

\speak  Parle, mon enfant, parle, dit d'Artagnan, j'écoute. 

\speak  Ici, impossible: ce que j'ai à vous dire est trop long et surtout trop secret. 

\speak  Eh bien, mais comment faire alors? 

\speak  Si monsieur le chevalier voulait me suivre, dit timidement Ketty. 

\speak  Où tu voudras, ma belle enfant. 

\speak  Alors, venez.» 

Et Ketty, qui n'avait point lâché la main de d'Artagnan, l'entraîna par un petit escalier sombre et tournant, et, après lui avoir fait monter une quinzaine de marches, ouvrit une porte. 

«Entrez, monsieur le chevalier, dit-elle, ici nous serons seuls et nous pourrons causer. 

\speak  Et quelle est donc cette chambre, ma belle enfant? demanda d'Artagnan. 

\speak  C'est la mienne, monsieur le chevalier; elle communique avec celle de ma maîtresse par cette porte. Mais soyez tranquille, elle ne pourra entendre ce que nous dirons, jamais elle ne se couche qu'à minuit.» 

D'Artagnan jeta un coup d'œil autour de lui. La petite chambre était charmante de goût et de propreté; mais, malgré lui, ses yeux se fixèrent sur cette porte que Ketty lui avait dit conduire à la chambre de Milady. 

Ketty devina ce qui se passait dans l'âme du jeune homme et poussa un soupir. 

«Vous aimez donc bien ma maîtresse, monsieur le chevalier, dit-elle. 

\speak  Oh! plus que je ne puis dire! j'en suis fou!» 

Ketty poussa un second soupir. 

«Hélas! monsieur, dit-elle, c'est bien dommage! 

\speak  Et que diable vois-tu donc là de si fâcheux? demanda d'Artagnan. 

\speak  C'est que, monsieur, reprit Ketty, ma maîtresse ne vous aime pas du tout. 

\speak  Hein! fit d'Artagnan, t'aurait-elle chargée de me le dire? 

\speak  Oh! non pas, monsieur! mais c'est moi qui, par intérêt pour vous, ai pris la résolution de vous en prévenir. 

\speak  Merci, ma bonne Ketty, mais de l'intention seulement, car la confidence, tu en conviendras, n'est point agréable. 

\speak  C'est-à-dire que vous ne croyez point à ce que je vous ai dit, n'est-ce pas? 

\speak  On a toujours peine à croire de pareilles choses, ma belle enfant, ne fût-ce que par amour-propre. 

\speak  Donc vous ne me croyez pas? 

\speak  J'avoue que jusqu'à ce que tu daignes me donner quelques preuves de ce que tu avances\dots 

\speak  Que dites-vous de celle-ci?» 

Et Ketty tira de sa poitrine un petit billet. 

«Pour moi? dit d'Artagnan en s'emparant vivement de la lettre. 

\speak  Non, pour un autre. 

\speak  Pour un autre? 

\speak  Oui. 

\speak  Son nom, son nom! s'écria d'Artagnan. 

\speak  Voyez l'adresse. 

\speak  M. le comte de Wardes.» 

Le souvenir de la scène de Saint-Germain se présenta aussitôt à l'esprit du présomptueux Gascon; par un mouvement rapide comme la pensée, il déchira l'enveloppe malgré le cri que poussa Ketty en voyant ce qu'il allait faire, ou plutôt ce qu'il faisait. 

«Oh! mon Dieu! monsieur le chevalier, dit-elle, que faites-vous? 

\speak  Moi, rien!» dit d'Artagnan, et il lut: 

«Vous n'avez pas répondu à mon premier billet; êtes-vous donc souffrant, ou bien auriez-vous oublié quels yeux vous me fîtes au bal de Mme de Guise? Voici l'occasion, comte! ne la laissez pas échapper.» 

D'Artagnan pâlit; il était blessé dans son amour-propre, il se crut blessé dans son amour. 

«Pauvre cher monsieur d'Artagnan! dit Ketty d'une voix pleine de compassion et en serrant de nouveau la main du jeune homme. 

\speak  Tu me plains, bonne petite! dit d'Artagnan. 

\speak  Oh! oui, de tout mon cœur! car je sais ce que c'est que l'amour, moi! 

\speak  Tu sais ce que c'est que l'amour? dit d'Artagnan la regardant pour la première fois avec une certaine attention. 

\speak  Hélas! oui. 

\speak  Eh bien, au lieu de me plaindre, alors, tu ferais bien mieux de m'aider à me venger de ta maîtresse. 

\speak  Et quelle sorte de vengeance voudriez-vous en tirer? 

\speak  Je voudrais triompher d'elle, supplanter mon rival. 

\speak  Je ne vous aiderai jamais à cela, monsieur le chevalier! dit vivement Ketty. 

\speak  Et pourquoi cela? demanda d'Artagnan. 

\speak  Pour deux raisons. 

\speak  Lesquelles? 

\speak  La première, c'est que jamais ma maîtresse ne vous a aimé. 

\speak  Qu'en sais-tu? 

\speak  Vous l'avez blessée au cœur. 

\speak  Moi! en quoi puis-je l'avoir blessée, moi qui, depuis que je la connais, vis à ses pieds comme un esclave! parle, je t'en prie. 

\speak  Je n'avouerais jamais cela qu'à l'homme\dots qui lirait jusqu'au fond de mon âme!» 

D'Artagnan regarda Ketty pour la seconde fois. La jeune fille était d'une fraîcheur et d'une beauté que bien des duchesses eussent achetées de leur couronne. 

«Ketty, dit-il, je lirai jusqu'au fond de ton âme quand tu voudras; qu'à cela ne tienne, ma chère enfant.» 

Et il lui donna un baiser sous lequel la pauvre enfant devint rouge comme une cerise. 

«Oh! non, s'écria Ketty, vous ne m'aimez pas! C'est ma maîtresse que vous aimez, vous me l'avez dit tout à l'heure. 

\speak  Et cela t'empêche-t-il de me faire connaître la seconde raison? 

\speak  La seconde raison, monsieur le chevalier, reprit Ketty enhardie par le baiser d'abord et ensuite par l'expression des yeux du jeune homme, c'est qu'en amour chacun pour soi.» 

Alors seulement d'Artagnan se rappela les coups d'œil languissants de Ketty, ses rencontres dans l'antichambre, sur l'escalier, dans le corridor, ses frôlements de main chaque fois qu'elle le rencontrait, et ses soupirs étouffés; mais, absorbé par le désir de plaire à la grande dame, il avait dédaigné la soubrette: qui chasse l'aigle ne s'inquiète pas du passereau. 

Mais cette fois notre Gascon vit d'un seul coup d'œil tout le parti qu'on pouvait tirer de cet amour que Ketty venait d'avouer d'une façon si naïve ou si effrontée: interception des lettres adressées au comte de Wardes, intelligences dans la place, entrée à toute heure dans la chambre de Ketty, contiguë à celle de sa maîtresse. Le perfide, comme on le voit, sacrifiait déjà en idée la pauvre fille pour obtenir Milady de gré ou de force. 

«Eh bien, dit-il à la jeune fille, veux-tu, ma chère Ketty, que je te donne une preuve de cet amour dont tu doutes? 

\speak  De quel amour? demanda la jeune fille. 

\speak  De celui que je suis tout prêt à ressentir pour toi. 

\speak  Et quelle est cette preuve? 

\speak  Veux-tu que ce soir je passe avec toi le temps que je passe ordinairement avec ta maîtresse? 

\speak  Oh! oui, dit Ketty en battant des mains, bien volontiers. 

\speak  Eh bien, ma chère enfant, dit d'Artagnan en s'établissant dans un fauteuil, viens çà que je te dise que tu es la plus jolie soubrette que j'aie jamais vue!» 

Et il le lui dit tant et si bien, que la pauvre enfant, qui ne demandait pas mieux que de le croire, le crut\dots Cependant, au grand étonnement de d'Artagnan, la jolie Ketty se défendait avec une certaine résolution. 

Le temps passe vite, lorsqu'il se passe en attaques et en défenses. 

Minuit sonna, et l'on entendit presque en même temps retentir la sonnette dans la chambre de Milady. 

«Grand Dieu! s'écria Ketty, voici ma maîtresse qui m'appelle! Partez, partez vite!» 

D'Artagnan se leva, prit son chapeau comme s'il avait l'intention d'obéir; puis, ouvrant vivement la porte d'une grande armoire au lieu d'ouvrir celle de l'escalier, il se blottit dedans au milieu des robes et des peignoirs de Milady. 

«Que faites-vous donc?» s'écria Ketty. 

D'Artagnan, qui d'avance avait pris la clef, s'enferma dans son armoire sans répondre. 

«Eh bien, cria Milady d'une voix aigre, dormez-vous donc que vous ne venez pas quand je sonne?» 

Et d'Artagnan entendit qu'on ouvrit violemment la porte de communication. 

«Me voici, Milady, me voici», s'écria Ketty en s'élançant à la rencontre de sa maîtresse. 

Toutes deux rentrèrent dans la chambre à coucher et comme la porte de communication resta ouverte, d'Artagnan put entendre quelque temps encore Milady gronder sa suivante, puis enfin elle s'apaisa, et la conversation tomba sur lui tandis que Ketty accommodait sa maîtresse. 

«Eh bien, dit Milady, je n'ai pas vu notre Gascon ce soir? 

\speak  Comment, madame, dit Ketty, il n'est pas venu! Serait-il volage avant d'être heureux? 

\speak  Oh non! il faut qu'il ait été empêché par M. de Tréville ou par M. des Essarts. Je m'y connais, Ketty, et je le tiens, celui-là. 

\speak  Qu'en fera madame? 

\speak  Ce que j'en ferai!\dots Sois tranquille, Ketty, il y a entre cet homme et moi une chose qu'il ignore\dots il a manqué me faire perdre mon crédit près de Son Éminence\dots Oh! je me vengerai! 

\speak  Je croyais que madame l'aimait? 

\speak  Moi, l'aimer! je le déteste! Un niais, qui tient la vie de Lord de Winter entre ses mains et qui ne le tue pas, et qui me fait perdre trois cent mille livres de rente! 

\speak  C'est vrai, dit Ketty, votre fils était le seul héritier de son oncle, et jusqu'à sa majorité vous auriez eu la jouissance de sa fortune.» 

D'Artagnan frissonna jusqu'à la moelle des os en entendant cette suave créature lui reprocher, avec cette voix stridente qu'elle avait tant de peine à cacher dans la conversation, de n'avoir pas tué un homme qu'il l'avait vue combler d'amitié. 

«Aussi, continua Milady, je me serais déjà vengée sur lui-même, si, je ne sais pourquoi, le cardinal ne m'avait recommandé de le ménager. 

\speak  Oh! oui, mais madame n'a point ménagé cette petite femme qu'il aimait. 

\speak  Oh! la mercière de la rue des Fossoyeurs: est-ce qu'il n'a pas déjà oublié qu'elle existait? La belle vengeance, ma foi!» 

Une sueur froide coulait sur le front de d'Artagnan: c'était donc un monstre que cette femme. 

Il se remit à écouter, mais malheureusement la toilette était finie. 

«C'est bien, dit Milady, rentrez chez vous et demain tâchez enfin d'avoir une réponse à cette lettre que je vous ai donnée. 

\speak  Pour M. de Wardes? dit Ketty. 

\speak  Sans doute, pour M. de Wardes. 

\speak  En voilà un, dit Ketty, qui m'a bien l'air d'être tout le contraire de ce pauvre M. d'Artagnan. 

\speak  Sortez, mademoiselle, dit Milady, je n'aime pas les commentaires.» 

D'Artagnan entendit la porte qui se refermait, puis le bruit de deux verrous que mettait Milady afin de s'enfermer chez elle; de son côté, mais le plus doucement qu'elle put, Ketty donna à la serrure un tour de clef; d'Artagnan alors poussa la porte de l'armoire. 

«O mon Dieu! dit tout bas Ketty, qu'avez-vous? et comme vous êtes pâle! 

\speak  L'abominable créature! murmura d'Artagnan. 

\speak  Silence! silence! sortez, dit Ketty; il n'y a qu'une cloison entre ma chambre et celle de Milady, on entend de l'une tout ce qui se dit dans l'autre! 

\speak  C'est justement pour cela que je ne sortirai pas, dit d'Artagnan. 

\speak  Comment? fit Ketty en rougissant. 

\speak  Ou du moins que je sortirai\dots plus tard.» 

Et il attira Ketty à lui; il n'y avait plus moyen de résister, la résistance fait tant de bruit! aussi Ketty céda. 

C'était un mouvement de vengeance contre Milady. D'Artagnan trouva qu'on avait raison de dire que la vengeance est le plaisir des dieux. Aussi, avec un peu de cœur, se serait-il contenté de cette nouvelle conquête; mais d'Artagnan n'avait que de l'ambition et de l'orgueil. 

Cependant, il faut le dire à sa louange, le premier emploi qu'il avait fait de son influence sur Ketty avait été d'essayer de savoir d'elle ce qu'était devenue Mme Bonacieux, mais la pauvre fille jura sur le crucifix à d'Artagnan qu'elle l'ignorait complètement, sa maîtresse ne laissant jamais pénétrer que la moitié de ses secrets; seulement, elle croyait pouvoir répondre qu'elle n'était pas morte. 

Quant à la cause qui avait manqué faire perdre à Milady son crédit près du cardinal, Ketty n'en savait pas davantage; mais cette fois, d'Artagnan était plus avancé qu'elle: comme il avait aperçu Milady sur un bâtiment consigné au moment où lui-même quittait l'Angleterre, il se douta qu'il était question cette fois des ferrets de diamants. 

Mais ce qu'il y avait de plus clair dans tout cela, c'est que la haine véritable, la haine profonde, la haine invétérée de Milady lui venait de ce qu'il n'avait pas tué son beau-frère. 

D'Artagnan retourna le lendemain chez Milady. Elle était de fort méchante humeur, d'Artagnan se douta que c'était le défaut de réponse de M. de Wardes qui l'agaçait ainsi. Ketty entra; mais Milady la reçut fort durement. Un coup d'œil qu'elle lança à d'Artagnan voulait dire: Vous voyez ce que je souffre pour vous. 

Cependant vers la fin de la soirée, la belle lionne s'adoucit, elle écouta en souriant les doux propos de d'Artagnan, elle lui donna même sa main à baiser. 

D'Artagnan sortit ne sachant plus que penser: mais comme c'était un garçon à qui on ne faisait pas facilement perdre la tête, tout en faisant sa cour à Milady il avait bâti dans son esprit un petit plan. 

Il trouva Ketty à la porte, et comme la veille il monta chez elle pour avoir des nouvelles. Ketty avait été fort grondée, on l'avait accusée de négligence. Milady ne comprenait rien au silence du comte de Wardes, et elle lui avait ordonné d'entrer chez elle à neuf heures du matin pour y prendre une troisième lettre. 

D'Artagnan fit promettre à Ketty de lui apporter chez lui cette lettre le lendemain matin; la pauvre fille promit tout ce que voulut son amant: elle était folle. 

Les choses se passèrent comme la veille: d'Artagnan s'enferma dans son armoire, Milady appela, fit sa toilette, renvoya Ketty et referma sa porte. Comme la veille d'Artagnan ne rentra chez lui qu'à cinq heures du matin. 

À onze heures, il vit arriver Ketty; elle tenait à la main un nouveau billet de Milady. Cette fois, la pauvre enfant n'essaya pas même de le disputer à d'Artagnan; elle le laissa faire; elle appartenait corps et âme à son beau soldat. 

D'Artagnan ouvrit le billet et lut ce qui suit: 

«Voilà la troisième fois que je vous écris pour vous dire que je vous aime. Prenez garde que je ne vous écrive une quatrième pour vous dire que je vous déteste. 

«Si vous vous repentez de la façon dont vous avez agi avec moi, la jeune fille qui vous remettra ce billet vous dira de quelle manière un galant homme peut obtenir son pardon.» 

D'Artagnan rougit et pâlit plusieurs fois en lisant ce billet. 

«Oh! vous l'aimez toujours! dit Ketty, qui n'avait pas détourné un instant les yeux du visage du jeune homme. 

\speak  Non, Ketty, tu te trompes, je ne l'aime plus; mais je veux me venger de ses mépris. 

\speak  Oui, je connais votre vengeance; vous me l'avez dite. 

\speak  Que t'importe, Ketty! tu sais bien que c'est toi seule que j'aime. 

\speak  Comment peut-on savoir cela? 

\speak  Par le mépris que je ferai d'elle.» 

Ketty soupira. 

D'Artagnan prit une plume et écrivit: 

\begin{mail}{}{Madame,}
	
Jusqu'ici j'avais douté que ce fût bien à moi que vos deux premiers billets eussent été adressés, tant je me croyais indigne d'un pareil honneur; d'ailleurs j'étais si souffrant, que j'eusse en tout cas hésité à y répondre. 

Mais aujourd'hui il faut bien que je croie à l'excès de vos bontés, puisque non seulement votre lettre, mais encore votre suivante, m'affirme que j'ai le bonheur d'être aimé de vous. 

Elle n'a pas besoin de me dire de quelle manière un galant homme peut obtenir son pardon. J'irai donc vous demander le mien ce soir à onze heures. Tarder d'un jour serait à mes yeux, maintenant, vous faire une nouvelle offense. 

\closeletter[Celui que vous avez rendu le plus heureux des hommes.]{Comte de Wardes.}
\end{mail}

Ce billet était d'abord un faux, c'était ensuite une indélicatesse; c'était même, au point de vue de nos mœurs actuelles, quelque chose comme une infamie; mais on se ménageait moins à cette époque qu'on ne le fait aujourd'hui. D'ailleurs d'Artagnan, par ses propres aveux, savait Milady coupable de trahison à des chefs plus importants, et il n'avait pour elle qu'une estime fort mince. Et cependant malgré ce peu d'estime, il sentait qu'une passion insensée le brûlait pour cette femme. Passion ivre de mépris, mais passion ou soif, comme on voudra. 

L'intention de d'Artagnan était bien simple: par la chambre de Ketty il arrivait à celle de sa maîtresse; il profitait du premier moment de surprise, de honte, de terreur pour triompher d'elle; peut-être aussi échouerait-il, mais il fallait bien donner quelque chose au hasard. Dans huit jours la campagne s'ouvrait, et il fallait partir; d'Artagnan n'avait pas le temps de filer le parfait amour. 

«Tiens, dit le jeune homme en remettant à Ketty le billet tout cacheté, donne cette lettre à Milady; c'est la réponse de M. de Wardes.» 

La pauvre Ketty devint pâle comme la mort, elle se doutait de ce que contenait le billet. 

«Écoute, ma chère enfant, lui dit d'Artagnan, tu comprends qu'il faut que tout cela finisse d'une façon ou de l'autre; Milady peut découvrir que tu as remis le premier billet à mon valet, au lieu de le remettre au valet du comte; que c'est moi qui ai décacheté les autres qui devaient être décachetés par M. de Wardes; alors Milady te chasse, et, tu la connais, ce n'est pas une femme à borner là sa vengeance. 

\speak  Hélas! dit Ketty, pour qui me suis-je exposée à tout cela? 

\speak  Pour moi, je le sais bien, ma toute belle, dit le jeune homme, aussi je t'en suis bien reconnaissant, je te le jure. 

\speak  Mais enfin, que contient votre billet? 

\speak  Milady te le dira. 

\speak  Ah! vous ne m'aimez pas! s'écria Ketty, et je suis bien malheureuse!» 

À ce reproche il y a une réponse à laquelle les femmes se trompent toujours; d'Artagnan répondit de manière que Ketty demeurât dans la plus grande erreur. 

Cependant elle pleura beaucoup avant de se décider à remettre cette lettre à Milady, mais enfin elle se décida, c'est tout ce que voulait d'Artagnan. 

D'ailleurs il lui promit que le soir il sortirait de bonne heure de chez sa maîtresse, et qu'en sortant de chez sa maîtresse il monterait chez elle. 

Cette promesse acheva de consoler la pauvre Ketty. 