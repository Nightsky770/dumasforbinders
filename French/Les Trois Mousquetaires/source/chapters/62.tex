%!TeX root=../musketeersfr.tex 

\chapter{Deux Variétés De Démons} 
	
\lettrine[ante=«]{A}{h!} s'écrièrent ensemble Rochefort et Milady, c'est vous! 

\zz
\noindent— Oui, c'est moi. 

\zz
\noindent— Et vous arrivez\dots? demanda Milady. 

\zz
\noindent— De La Rochelle, et vous? 

\speak  D'Angleterre. 

\speak  Buckingham? 

\speak  Mort ou blessé dangereusement; comme je partais sans avoir rien pu obtenir de lui, un fanatique venait de l'assassiner. 

\speak  Ah! fit Rochefort avec un sourire, voilà un hasard bien heureux! et qui satisfera Son Éminence! L'avez-vous prévenue? 

\speak  Je lui ai écrit de Boulogne. Mais comment êtes-vous ici? 

\speak  Son Éminence, inquiète, m'a envoyé à votre recherche. 

\speak  Je suis arrivée d'hier seulement. 

\speak  Et qu'avez-vous fait depuis hier? 

\speak  Je n'ai pas perdu mon temps. 

\speak  Oh! je m'en doute bien! 

\speak  Savez-vous qui j'ai rencontré ici? 

\speak  Non. 

\speak  Devinez. 

\speak  Comment voulez-vous?\dots 

\speak  Cette jeune femme que la reine a tirée de prison. 

\speak  La maîtresse du petit d'Artagnan? 

\speak  Oui, Mme Bonacieux, dont le cardinal ignorait la retraite. 

\speak  Eh bien, dit Rochefort, voilà encore un hasard qui peut aller de pair avec l'autre, M. le cardinal est en vérité un homme privilégié. 

\speak  Comprenez-vous mon étonnement, continua Milady, quand je me suis trouvée face à face avec cette femme? 

\speak  Vous connaît-elle? 

\speak  Non. 

\speak  Alors elle vous regarde comme une étrangère?» 

Milady sourit. 

«Je suis sa meilleure amie! 

\speak  Sur mon honneur, dit Rochefort, il n'y a que vous, ma chère comtesse, pour faire de ces miracles-là. 

\speak  Et bien m'en a pris, chevalier, dit Milady, car savez-vous ce qui se passe? 

\speak  Non. 

\speak  On va la venir chercher demain ou après-demain avec un ordre de la reine. 

\speak  Vraiment? et qui cela? 

\speak  D'Artagnan et ses amis. 

\speak  En vérité ils en feront tant, que nous serons obligés de les envoyer à la Bastille. 

\speak  Pourquoi n'est-ce point déjà fait? 

\speak  Que voulez-vous! parce que M. le cardinal a pour ces hommes une faiblesse que je ne comprends pas. 

\speak  Vraiment? 

\speak  Oui. 

\speak  Eh bien, dites-lui ceci, Rochefort: dites-lui que notre conversation à l'auberge du Colombier-Rouge a été entendue par ces quatre hommes; dites-lui qu'après son départ l'un d'eux est monté et m'a arraché par violence le sauf-conduit qu'il m'avait donné; dites-lui qu'ils avaient fait prévenir Lord de Winter de mon passage en Angleterre; que, cette fois encore, ils ont failli faire échouer ma mission, comme ils ont fait échouer celle des ferrets; dites-lui que parmi ces quatre hommes, deux seulement sont à craindre, d'Artagnan et Athos; dites-lui que le troisième, Aramis, est l'amant de Mme de Chevreuse: il faut laisser vivre celui-là, on sait son secret, il peut être utile; quant au quatrième, Porthos, c'est un sot, un fat et un niais, qu'il ne s'en occupe même pas. 

\speak  Mais ces quatre hommes doivent être à cette heure au siège de La Rochelle. 

\speak  Je le croyais comme vous; mais une lettre que Mme Bonacieux a reçue de Mme de Chevreuse, et qu'elle a eu l'imprudence de me communiquer, me porte à croire que ces quatre hommes au contraire sont en campagne pour la venir enlever. 

\speak  Diable! comment faire? 

\speak  Que vous a dit le cardinal à mon égard? 

\speak  De prendre vos dépêches écrites ou verbales, de revenir en poste, et, quand il saura ce que vous avez fait, il avisera à ce que vous devez faire. 

\speak  Je dois donc rester ici? demanda Milady. 

\speak  Ici ou dans les environs. 

\speak  Vous ne pouvez m'emmener avec vous? 

\speak  Non, l'ordre est formel: aux environs du camp, vous pourriez être reconnue, et votre présence, vous le comprenez, compromettrait Son Éminence, surtout après ce qui vient de se passer là-bas. Seulement, dites-moi d'avance où vous attendrez des nouvelles du cardinal, que je sache toujours où vous retrouver. 

\speak  Écoutez, il est probable que je ne pourrai rester ici. 

\speak  Pourquoi? 

\speak  Vous oubliez que mes ennemis peuvent arriver d'un moment à l'autre. 

\speak  C'est vrai; mais alors cette petite femme va échapper à Son Éminence? 

\speak  Bah! dit Milady avec un sourire qui n'appartenait qu'à elle, vous oubliez que je suis sa meilleure amie. 

\speak  Ah! c'est vrai! je puis donc dire au cardinal, à l'endroit de cette femme\dots 

\speak  Qu'il soit tranquille. 

\speak  Voilà tout? 

\speak  Il saura ce que cela veut dire. 

\speak  Il le devinera. Maintenant, voyons, que dois-je faire? 

\speak  Repartir à l'instant même; il me semble que les nouvelles que vous reportez valent bien la peine que l'on fasse diligence. 

\speak  Ma chaise s'est cassée en entrant à Lillers. 

\speak  À merveille! 

\speak  Comment, à merveille? 

\speak  Oui, j'ai besoin de votre chaise, moi, dit la comtesse. 

\speak  Et comment partirai-je, alors? 

\speak  À franc étrier. 

\speak  Vous en parlez bien à votre aise, cent quatre-vingts lieues. 

\speak  Qu'est-ce que cela? 

\speak  On les fera. Après? 

\speak  Après: en passant à Lillers, vous me renvoyez la chaise avec ordre à votre domestique de se mettre à ma disposition. 

\speak  Bien. 

\speak  Vous avez sans doute sur vous quelque ordre du cardinal? 

\speak  J'ai mon plein pouvoir. 

\speak  Vous le montrez à l'abbesse, et vous dites qu'on viendra me chercher, soit aujourd'hui, soit demain, et que j'aurai à suivre la personne qui se présentera en votre nom. 

\speak  Très bien! 

\speak  N'oubliez pas de me traiter durement en parlant de moi à l'abbesse. 

\speak  À quoi bon? 

\speak  Je suis une victime du cardinal. Il faut bien que j'inspire de la confiance à cette pauvre petite Mme Bonacieux. 

\speak  C'est juste. Maintenant voulez-vous me faire un rapport de tout ce qui est arrivé? 

\speak  Mais je vous ai raconté les événements, vous avez bonne mémoire, répétez les choses comme je vous les ai dites, un papier se perd. 

\speak  Vous avez raison; seulement que je sache où vous retrouver, que je n'aille pas courir inutilement dans les environs. 

\speak  C'est juste, attendez. 

\speak  Voulez-vous une carte? 

\speak  Oh! je connais ce pays à merveille. 

\speak  Vous? quand donc y êtes-vous venue? 

\speak  J'y ai été élevée. 

\speak  Vraiment? 

\speak  C'est bon à quelque chose, vous le voyez, que d'avoir été élevée quelque part. 

\speak  Vous m'attendrez donc\dots? 

\speak  Laissez-moi réfléchir un instant; eh! tenez, à Armentières. 

\speak  Qu'est-ce que cela, Armentières? 

\speak  Une petite ville sur la Lys! je n'aurai qu'à traverser la rivière et je suis en pays étranger. 

\speak  À merveille! mais il est bien entendu que vous ne traverserez la rivière qu'en cas de danger. 

\speak  C'est bien entendu. 

\speak  Et, dans ce cas, comment saurai-je où vous êtes? 

\speak  Vous n'avez pas besoin de votre laquais? 

\speak  Non. 

\speak  C'est un homme sûr? 

\speak  À l'épreuve. 

\speak  Donnez-le-moi; personne ne le connaît, je le laisse à l'endroit que je quitte, et il vous conduit où je suis. 

\speak  Et vous dites que vous m'attendez à Argentières? 

\speak  À Armentières, répondit Milady. 

\speak  Écrivez-moi ce nom-là sur un morceau de papier, de peur que je l'oublie; ce n'est pas compromettant, un nom de ville, n'est-ce pas? 

\speak  Eh, qui sait? N'importe, dit Milady en écrivant le nom sur une demi-feuille de papier, je me compromets. 

\speak  Bien! dit Rochefort en prenant des mains de Milady le papier, qu'il plia et qu'il enfonça dans la coiffe de son feutre; d'ailleurs, soyez tranquille, je vais faire comme les enfants, et, dans le cas où je perdrais ce papier, répéter le nom tout le long de la route. Maintenant est-ce tout? 

\speak  Je le crois. 

\speak  Cherchons bien: Buckingham mort ou grièvement blessé; votre entretien avec le cardinal entendu des quatre mousquetaires; Lord de Winter prévenu de votre arrivée à Portsmouth; d'Artagnan et Athos à la Bastille; Aramis l'amant de Mme de Chevreuse; Porthos un fat; Mme Bonacieux retrouvée; vous envoyer la chaise le plus tôt possible; mettre mon laquais à votre disposition; faire de vous une victime du cardinal, pour que l'abbesse ne prenne aucun soupçon; Armentières sur les bords de la Lys. Est-ce cela? 

\speak  En vérité, mon cher chevalier, vous êtes un miracle de mémoire. À propos, ajoutez une chose\dots 

\speak  Laquelle? 

\speak  J'ai vu de très jolis bois qui doivent toucher au jardin du couvent, dites qu'il m'est permis de me promener dans ces bois; qui sait? j'aurai peut-être besoin de sortir par une porte de derrière. 

\speak  Vous pensez à tout. 

\speak  Et vous, vous oubliez une chose\dots 

\speak  Laquelle? 

\speak  C'est de me demander si j'ai besoin d'argent. 

\speak  C'est juste, combien voulez-vous? 

\speak  Tout ce que vous aurez d'or. 

\speak  J'ai cinq cents pistoles à peu près. 

\speak  J'en ai autant: avec mille pistoles on fait face à tout; videz vos poches. 

\speak  Voilà, comtesse. 

\speak  Bien, mon cher comte! et vous partez\dots? 

\speak  Dans une heure; le temps de manger un morceau, pendant lequel j'enverrai chercher un cheval de poste. 

\speak  À merveille! Adieu, chevalier! 

\speak  Adieu, comtesse! 

\speak  Recommandez-moi au cardinal, dit Milady. 

\speak  Recommandez-moi à Satan», répliqua Rochefort. 

Milady et Rochefort échangèrent un sourire et se séparèrent. 

Une heure après, Rochefort partit au grand galop de son cheval; cinq heures après il passait à Arras. 

Nos lecteurs savent déjà comment il avait été reconnu par d'Artagnan, et comment cette reconnaissance, en inspirant des craintes aux quatre mousquetaires, avait donné une nouvelle activité à leur voyage. 
