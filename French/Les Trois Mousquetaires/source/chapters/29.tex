%!TeX root=../musketeersfr.tex 

\chapter{La Chasse À L'Équipement}

\lettrine{L}{e} plus préoccupé des quatre amis était bien certainement d'Artagnan, quoique d'Artagnan, en sa qualité de garde, fût bien plus facile à équiper que messieurs les mousquetaires, qui étaient des seigneurs; mais notre cadet de Gascogne était, comme on a pu le voir, d'un caractère prévoyant et presque avare, et avec cela (expliquez les contraires) glorieux presque à rendre des points à Porthos. À cette préoccupation de sa vanité, d'Artagnan joignait en ce moment une inquiétude moins égoïste. Quelques informations qu'il eût pu prendre sur Mme Bonacieux, il ne lui en était venu aucune nouvelle. M. de Tréville en avait parlé à la reine; la reine ignorait où était la jeune mercière et avait promis de la faire chercher. 

Mais cette promesse était bien vague et ne rassurait guère d'Artagnan. 

Athos ne sortait pas de sa chambre; il était résolu à ne pas risquer une enjambée pour s'équiper. 

«Il nous reste quinze jours, disait-il à ses amis; eh bien, si au bout de ces quinze jours je n'ai rien trouvé, ou plutôt si rien n'est venu me trouver, comme je suis trop bon catholique pour me casser la tête d'un coup de pistolet, je chercherai une bonne querelle à quatre gardes de Son Éminence ou à huit Anglais, et je me battrai jusqu'à ce qu'il y en ait un qui me tue, ce qui, sur la quantité, ne peut manquer de m'arriver. On dira alors que je suis mort pour le roi, de sorte que j'aurai fait mon service sans avoir eu besoin de m'équiper.» 

Porthos continuait à se promener, les mains derrière le dos, en hochant la tête de haut en bas et disant: 

«Je poursuivrai mon idée.» 

Aramis, soucieux et mal frisé, ne disait rien. 

On peut voir par ces détails désastreux que la désolation régnait dans la communauté. 

Les laquais, de leur côté, comme les coursiers d'Hippolyte, partageaient la triste peine de leurs maîtres. Mousqueton faisait des provisions de croûtes; Bazin, qui avait toujours donné dans la dévotion, ne quittait plus les églises; Planchet regardait voler les mouches; et Grimaud, que la détresse générale ne pouvait déterminer à rompre le silence imposé par son maître, poussait des soupirs à attendrir des pierres. 

Les trois amis --- car, ainsi que nous l'avons dit, Athos avait juré de ne pas faire un pas pour s'équiper --- les trois amis sortaient donc de grand matin et rentraient fort tard. Ils erraient par les rues, regardant sur chaque pavé pour savoir si les personnes qui y étaient passées avant eux n'y avaient pas laissé quelque bourse. On eût dit qu'ils suivaient des pistes, tant ils étaient attentifs partout où ils allaient. Quand ils se rencontraient, ils avaient des regards désolés qui voulaient dire: As-tu trouvé quelque chose? 

Cependant, comme Porthos avait trouvé le premier son idée, et comme il l'avait poursuivie avec persistance, il fut le premier à agir. C'était un homme d'exécution que ce digne Porthos. D'Artagnan l'aperçut un jour qu'il s'acheminait vers l'église Saint-Leu, et le suivit instinctivement: il entra au lieu saint après avoir relevé sa moustache et allongé sa royale, ce qui annonçait toujours de sa part les intentions les plus conquérantes. Comme d'Artagnan prenait quelques précautions pour se dissimuler, Porthos crut n'avoir pas été vu. D'Artagnan entra derrière lui. Porthos alla s'adosser au côté d'un pilier; d'Artagnan, toujours inaperçu, s'appuya de l'autre. 

Justement il y avait un sermon, ce qui faisait que l'église était fort peuplée. Porthos profita de la circonstance pour lorgner les femmes: grâce aux bons soins de Mousqueton l'extérieur était loin d'annoncer la détresse de l'intérieur; son feutre était bien un peu râpé, sa plume était bien un peu déteinte, ses broderies étaient bien un peu ternies, ses dentelles étaient bien éraillées; mais dans la demi-teinte toutes ces bagatelles disparaissaient, et Porthos était toujours le beau Porthos. 

D'Artagnan remarqua, sur le banc le plus rapproché du pilier où Porthos et lui étaient adossés, une espèce de beauté mûre, un peu jaune, un peu sèche, mais raide et hautaine sous ses coiffes noires. Les yeux de Porthos s'abaissaient furtivement sur cette dame, puis papillonnaient au loin dans la nef. 

De son côté, la dame, qui de temps en temps rougissait, lançait avec la rapidité de l'éclair un coup d'œil sur le volage Porthos, et aussitôt les yeux de Porthos de papillonner avec fureur. Il était clair que c'était un manège qui piquait au vif la dame aux coiffes noires, car elle se mordait les lèvres jusqu'au sang, se grattait le bout du nez, et se démenait désespérément sur son siège. 

Ce que voyant, Porthos retroussa de nouveau sa moustache, allongea une seconde fois sa royale, et se mit à faire des signaux à une belle dame qui était près du choeur, et qui non seulement était une belle dame, mais encore une grande dame sans doute, car elle avait derrière elle un négrillon qui avait apporté le coussin sur lequel elle était agenouillée, et une suivante qui tenait le sac armorié dans lequel on renfermait le livre où elle lisait sa messe. 

La dame aux coiffes noires suivit à travers tous ses détours le regard de Porthos, et reconnut qu'il s'arrêtait sur la dame au coussin de velours, au négrillon et à la suivante. 

Pendant ce temps, Porthos jouait serré: c'était des clignements d'yeux, des doigts posés sur les lèvres, de petits sourires assassins qui réellement assassinaient la belle dédaignée. 

Aussi poussa-t-elle, en forme de \textit{mea culpa} et en se frappant la poitrine, un hum! tellement vigoureux que tout le monde, même la dame au coussin rouge, se retourna de son côté; Porthos tint bon: pourtant il avait bien compris, mais il fit le sourd. 

La dame au coussin rouge fit un grand effet, car elle était fort belle, sur la dame aux coiffes noires, qui vit en elle une rivale véritablement à craindre; un grand effet sur Porthos, qui la trouva plus jolie que la dame aux coiffes noires; un grand effet sur d'Artagnan, qui reconnut la dame de Meung, de Calais et de Douvres, que son persécuteur, l'homme à la cicatrice, avait saluée du nom de Milady. 

D'Artagnan, sans perdre de vue la dame au coussin rouge, continua de suivre le manège de Porthos, qui l'amusait fort; il crut deviner que la dame aux coiffes noires était la procureuse de la rue aux Ours, d'autant mieux que l'église Saint-Leu n'était pas très éloignée de ladite rue. 

Il devina alors par induction que Porthos cherchait à prendre sa revanche de sa défaite de Chantilly, alors que la procureuse s'était montrée si récalcitrante à l'endroit de la bourse. 

Mais, au milieu de tout cela, d'Artagnan remarqua aussi que pas une figure ne correspondait aux galanteries de Porthos. Ce n'étaient que chimères et illusions; mais pour un amour réel, pour une jalousie véritable, y a-t-il d'autre réalité que les illusions et les chimères? 

Le sermon finit: la procureuse s'avança vers le bénitier; Porthos l'y devança, et, au lieu d'un doigt, y mit toute la main. La procureuse sourit, croyant que c'était pour elle que Porthos se mettait en frais: mais elle fut promptement et cruellement détrompée: lorsqu'elle ne fut plus qu'à trois pas de lui, il détourna la tête, fixant invariablement les yeux sur la dame au coussin rouge, qui s'était levée et qui s'approchait suivie de son négrillon et de sa fille de chambre. 

Lorsque la dame au coussin rouge fut près de Porthos, Porthos tira sa main toute ruisselante du bénitier; la belle dévote toucha de sa main effilée la grosse main de Porthos, fit en souriant le signe de la croix et sortit de l'église. 

C'en fut trop pour la procureuse: elle ne douta plus que cette dame et Porthos fussent en galanterie. Si elle eût été une grande dame, elle se serait évanouie, mais comme elle n'était qu'une procureuse, elle se contenta de dire au mousquetaire avec une fureur concentrée: 

«Eh! monsieur Porthos, vous ne m'en offrez pas à moi, d'eau bénite?» 

Porthos fit, au son de cette voix, un soubresaut comme ferait un homme qui se réveillerait après un somme de cent ans. 

«Ma\dots madame! s'écria-t-il, est-ce bien vous? Comment se porte votre mari, ce cher monsieur Coquenard? Est-il toujours aussi ladre qu'il était? Où avais-je donc les yeux, que je ne vous ai pas même aperçue pendant les deux heures qu'a duré ce sermon? 

\speak  J'étais à deux pas de vous, monsieur, répondit la procureuse; mais vous ne m'avez pas aperçue parce que vous n'aviez d'yeux que pour la belle dame à qui vous venez de donner de l'eau bénite.» 

Porthos feignit d'être embarrassé. 

«Ah! dit-il, vous avez remarqué\dots 

\speak  Il eût fallu être aveugle pour ne pas le voir. 

\speak  Oui, dit négligemment Porthos, c'est une duchesse de mes amies avec laquelle j'ai grand-peine à me rencontrer à cause de la jalousie de son mari, et qui m'avait fait prévenir qu'elle viendrait aujourd'hui, rien que pour me voir, dans cette chétive église, au fond de ce quartier perdu. 

\speak  Monsieur Porthos, dit la procureuse, auriez-vous la bonté de m'offrir le bras pendant cinq minutes, je causerais volontiers avec vous? 

\speak  Comment donc, madame», dit Porthos en se clignant de l'œil à lui-même comme un joueur qui rit de la dupe qu'il va faire. 

Dans ce moment, d'Artagnan passait poursuivant Milady; il jeta un regard de côté sur Porthos, et vit ce coup d'œil triomphant. 

«Eh! eh! se dit-il à lui même en raisonnant dans le sens de la morale étrangement facile de cette époque galante, en voici un qui pourrait bien être équipé pour le terme voulu.» 

Porthos, cédant à la pression du bras de sa procureuse comme une barque cède au gouvernail, arriva au cloître Saint-Magloire, passage peu fréquenté, enfermé d'un tourniquet à ses deux bouts. On n'y voyait, le jour, que mendiants qui mangeaient ou enfants qui jouaient. 

«Ah! monsieur Porthos! s'écria la procureuse, quand elle se fut assurée qu'aucune personne étrangère à la population habituelle de la localité ne pouvait les voir ni les entendre; ah! monsieur Porthos! vous êtes un grand vainqueur, à ce qu'il paraît! 

\speak  Moi, madame! dit Porthos en se rengorgeant, et pourquoi cela? 

\speak  Et les signes de tantôt, et l'eau bénite? Mais c'est une princesse pour le moins, que cette dame avec son négrillon et sa fille de chambre! 

\speak  Vous vous trompez; mon Dieu, non, répondit Porthos, c'est tout bonnement une duchesse. 

\speak  Et ce coureur qui attendait à la porte, et ce carrosse avec un cocher à grande livrée qui attendait sur son siège?» 

Porthos n'avait vu ni le coureur, ni le carrosse; mais, de son regard de femme jalouse, Mme Coquenard avait tout vu. 

Porthos regretta de n'avoir pas, du premier coup, fait la dame au coussin rouge princesse. 

«Ah! vous êtes l'enfant chéri des belles, monsieur Porthos! reprit en soupirant la procureuse. 

\speak  Mais, répondit Porthos, vous comprenez qu'avec un physique comme celui dont la nature m'a doué, je ne manque pas de bonnes fortunes. 

\speak  Mon Dieu! comme les hommes oublient vite! s'écria la procureuse en levant les yeux au ciel. 

\speak  Moins vite encore que les femmes, ce me semble, répondit Porthos; car enfin, moi, madame, je puis dire que j'ai été votre victime, lorsque blessé, mourant, je me suis vu abandonné des chirurgiens; moi, le rejeton d'une famille illustre, qui m'étais fié à votre amitié, j'ai manqué mourir de mes blessures d'abord, et de faim ensuite dans une mauvaise auberge de Chantilly, et cela sans que vous ayez daigné répondre une seule fois aux lettres brûlantes que je vous ai écrites. 

\speak  Mais, monsieur Porthos\dots, murmura la procureuse, qui sentait qu'à en juger par la conduite des plus grandes dames de ce temps-là, elle était dans son tort. 

\speak  Moi qui avais sacrifié pour vous la comtesse de Penaflor\dots 

\speak  Je le sais bien. 

\speak  La baronne de\dots 

\speak  Monsieur Porthos, ne m'accablez pas. 

\speak  La duchesse de\dots 

\speak  Monsieur Porthos, soyez généreux! 

\speak  Vous avez raison, madame, et je n'achèverai pas. 

\speak  Mais c'est mon mari qui ne veut pas entendre parler de prêter. 

\speak  Madame Coquenard, dit Porthos, rappelez-vous la première lettre que vous m'avez écrite et que je conserve gravée dans ma mémoire.» 

La procureuse poussa un gémissement. 

«Mais c'est qu'aussi, dit-elle, la somme que vous demandiez à emprunter était un peu bien forte. 

\speak  Madame Coquenard, je vous donnais la préférence. Je n'ai eu qu'à écrire à la duchesse de\dots Je ne veux pas dire son nom, car je ne sais pas ce que c'est que de compromettre une femme; mais ce que je sais, c'est que je n'ai eu qu'à lui écrire pour qu'elle m'en envoyât quinze cents.» 

La procureuse versa une larme. 

«Monsieur Porthos, dit-elle, je vous jure que vous m'avez grandement punie, et que si dans l'avenir vous vous retrouviez en pareille passe, vous n'auriez qu'à vous adresser à moi. 

\speak  Fi donc, madame! dit Porthos comme révolté, ne parlons pas argent, s'il vous plaît, c'est humiliant. 

\speak  Ainsi, vous ne m'aimez plus!» dit lentement et tristement la procureuse. 

Porthos garda un majestueux silence. 

«C'est ainsi que vous me répondez? Hélas! je comprends. 

\speak  Songez à l'offense que vous m'avez faite, madame: elle est restée là, dit Porthos, en posant la main à son cœur et en l'y appuyant avec force. 

\speak  Je la réparerai; voyons, mon cher Porthos! 

\speak  D'ailleurs, que vous demandais-je, moi? reprit Porthos avec un mouvement d'épaules plein de bonhomie; un prêt, pas autre chose. Après tout, je ne suis pas un homme déraisonnable. Je sais que vous n'êtes pas riche, madame Coquenard, et que votre mari est obligé de sangsurer les pauvres plaideurs pour en tirer quelques pauvres écus. Oh! si vous étiez comtesse, marquise ou duchesse, ce serait autre chose, et vous seriez impardonnable.» 

La procureuse fut piquée. 

«Apprenez, monsieur Porthos, dit-elle, que mon coffre-fort, tout coffre-fort de procureuse qu'il est, est peut-être mieux garni que celui de toutes vos mijaurées ruinées. 

\speak  Double offense que vous m'avez faite alors, dit Porthos en dégageant le bras de la procureuse de dessous le sien; car si vous êtes riche, madame Coquenard, alors votre refus n'a plus d'excuse. 

\speak  Quand je dis riche, reprit la procureuse, qui vit qu'elle s'était laissé entraîner trop loin, il ne faut pas prendre le mot au pied de la lettre. Je ne suis pas précisément riche, je suis à mon aise. 

\speak  Tenez, madame, dit Porthos, ne parlons plus de tout cela, je vous en prie. Vous m'avez méconnu; toute sympathie est éteinte entre nous. 

\speak  Ingrat que vous êtes! 

\speak  Ah! je vous conseille de vous plaindre! dit Porthos. 

\speak  Allez donc avec votre belle duchesse! je ne vous retiens plus. 

\speak  Eh! elle n'est déjà point si décharnée, que je crois! 

\speak  Voyons, monsieur Porthos, encore une fois, c'est la dernière: m'aimez-vous encore? 

\speak  Hélas! madame, dit Porthos du ton le plus mélancolique qu'il put prendre, quand nous allons entrer en campagne, dans une campagne où mes pressentiments me disent que je serai tué\dots 

\speak  Oh! ne dites pas de pareilles choses! s'écria la procureuse en éclatant en sanglots. 

\speak  Quelque chose me le dit, continua Porthos en mélancolisant de plus en plus. 

\speak  Dites plutôt que vous avez un nouvel amour. 

\speak  Non pas, je vous parle franc. Nul objet nouveau ne me touche, et même je sens là, au fond de mon cœur, quelque chose qui parle pour vous. Mais, dans quinze jours, comme vous le savez ou comme vous ne le savez pas, cette fatale campagne s'ouvre; je vais être affreusement préoccupé de mon équipement. Puis je vais faire un voyage dans ma famille, au fond de la Bretagne, pour réaliser la somme nécessaire à mon départ.» 

Porthos remarqua un dernier combat entre l'amour et l'avarice. 

«Et comme, continua-t-il, la duchesse que vous venez de voir à l'église a ses terres près des miennes, nous ferons le voyage ensemble. Les voyages, vous le savez, paraissent beaucoup moins longs quand on les fait à deux. 

\speak  Vous n'avez donc point d'amis à Paris, monsieur Porthos? dit la procureuse. 

\speak  J'ai cru en avoir, dit Porthos en prenant son air mélancolique, mais j'ai bien vu que je me trompais. 

\speak  Vous en avez, monsieur Porthos, vous en avez, reprit la procureuse dans un transport qui la surprit elle-même; revenez demain à la maison. Vous êtes le fils de ma tante, mon cousin par conséquent; vous venez de Noyon en Picardie, vous avez plusieurs procès à Paris, et pas de procureur. Retiendrez-vous bien tout cela? 

\speak  Parfaitement, madame. 

\speak  Venez à l'heure du dîner. 

\speak  Fort bien. 

\speak  Et tenez ferme devant mon mari, qui est retors, malgré ses soixante-seize ans. 

\speak  Soixante-seize ans! peste! le bel âge! reprit Porthos. 

\speak  Le grand âge, vous voulez dire, monsieur Porthos. Aussi le pauvre cher homme peut me laisser veuve d'un moment à l'autre, continua la procureuse en jetant un regard significatif à Porthos. Heureusement que, par contrat de mariage, nous nous sommes tout passé au dernier vivant. 

\speak  Tout? dit Porthos. 

\speak  Tout. 

\speak  Vous êtes femme de précaution, je le vois, ma chère madame Coquenard, dit Porthos en serrant tendrement la main de la procureuse. 

\speak  Nous sommes donc réconciliés, cher monsieur Porthos? dit-elle en minaudant. 

\speak  Pour la vie, répliqua Porthos sur le même air. 

\speak  Au revoir donc, mon traître. 

\speak  Au revoir, mon oublieuse. 

\speak  À demain, mon ange! 

\speak  À demain, flamme de ma vie!» 