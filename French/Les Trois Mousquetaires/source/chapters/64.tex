%!TeX root=../musketeersfr.tex 

\chapter{L'Homme Au Manteau Rouge}

\lettrine{L}{e} désespoir d'Athos avait fait place à une douleur concentrée, qui rendait plus lucides encore les brillantes facultés d'esprit de cet homme. 

\zz
Tout entier à une seule pensée, celle de la promesse qu'il avait faite et de la responsabilité qu'il avait prise, il se retira le dernier dans sa chambre, pria l'hôte de lui procurer une carte de la province, se courba dessus, interrogea les lignes tracées, reconnut que quatre chemins différents se rendaient de Béthune à Armentières, et fit appeler les valets. 

Planchet, Grimaud, Mousqueton et Bazin se présentèrent et reçurent les ordres clairs, ponctuels et graves d'Athos. 

Ils devaient partir au point du jour, le lendemain, et se rendre à Armentières, chacun par une route différente. Planchet, le plus intelligent des quatre, devait suivre celle par laquelle avait disparu la voiture sur laquelle les quatre amis avaient tiré, et qui était accompagnée, on se le rappelle, du domestique de Rochefort. 

Athos mit les valets en campagne d'abord, parce que, depuis que ces hommes étaient à son service et à celui de ses amis, il avait reconnu en chacun d'eux des qualités différentes et essentielles. 

Puis, des valets qui interrogent inspirent aux passants moins de défiance que leurs maîtres, et trouvent plus de sympathie chez ceux auxquels ils s'adressent. 

Enfin, Milady connaissait les maîtres, tandis qu'elle ne connaissait pas les valets; au contraire, les valets connaissaient parfaitement Milady. 

Tous quatre devaient se trouver réunis le lendemain à onze heures à l'endroit indiqué; s'ils avaient découvert la retraite de Milady, trois resteraient à la garder, le quatrième reviendrait à Béthune pour prévenir Athos et servir de guide aux quatre amis. 

Ces dispositions prises, les valets se retirèrent à leur tour. 

Athos alors se leva de sa chaise, ceignit son épée, s'enveloppa dans son manteau et sortit de l'hôtel; il était dix heures à peu près. À dix heures du soir, on le sait, en province les rues sont peu fréquentées. Athos cependant cherchait visiblement quelqu'un à qui il pût adresser une question. Enfin il rencontra un passant attardé, s'approcha de lui, lui dit quelques paroles; l'homme auquel il s'adressait recula avec terreur, cependant il répondit aux paroles du mousquetaire par une indication. Athos offrit à cet homme une demi-pistole pour l'accompagner, mais l'homme refusa. 

Athos s'enfonça dans la rue que l'indicateur avait désignée du doigt; mais, arrivé à un carrefour, il s'arrêta de nouveau, visiblement embarrassé. Cependant, comme, plus qu'aucun autre lieu, le carrefour lui offrait la chance de rencontrer quelqu'un, il s'y arrêta. En effet, au bout d'un instant, un veilleur de nuit passa. Athos lui répéta la même question qu'il avait déjà faite à la première personne qu'il avait rencontrée, le veilleur de nuit laissa apercevoir la même terreur, refusa à son tour d'accompagner Athos, et lui montra de la main le chemin qu'il devait suivre. 

Athos marcha dans la direction indiquée et atteignit le faubourg situé à l'extrémité de la ville opposée à celle par laquelle lui et ses compagnons étaient entrés. Là il parut de nouveau inquiet et embarrassé, et s'arrêta pour la troisième fois. 

Heureusement un mendiant passa, qui s'approcha d'Athos pour lui demander l'aumône. Athos lui proposa un écu pour l'accompagner où il allait. Le mendiant hésita un instant, mais à la vue de la pièce d'argent qui brillait dans l'obscurité, il se décida et marcha devant Athos. 

Arrivé à l'angle d'une rue, il lui montra de loin une petite maison isolée, solitaire, triste; Athos s'en approcha, tandis que le mendiant, qui avait reçu son salaire, s'en éloignait à toutes jambes. 

Athos en fit le tour, avant de distinguer la porte au milieu de la couleur rougeâtre dont cette maison était peinte; aucune lumière ne paraissait à travers les gerçures des contrevents, aucun bruit ne pouvait faire supposer qu'elle fût habitée, elle était sombre et muette comme un tombeau. 

Trois fois Athos frappa sans qu'on lui répondît. Au troisième coup cependant des pas intérieurs se rapprochèrent; enfin la porte s'entrebâilla, et un homme de haute taille, au teint pâle, aux cheveux et à la barbe noire, parut. 

Athos et lui échangèrent quelques mots à voix basse, puis l'homme à la haute taille fit signe au mousquetaire qu'il pouvait entrer. Athos profita à l'instant même de la permission, et la porte se referma derrière lui. 

L'homme qu'Athos était venu chercher si loin et qu'il avait trouvé avec tant de peine, le fit entrer dans son laboratoire, où il était occupé à retenir avec des fils de fer les os cliquetants d'un squelette. Tout le corps était déjà rajusté: la tête seule était posée sur une table. 

Tout le reste de l'ameublement indiquait que celui chez lequel on se trouvait s'occupait de sciences naturelles: il y avait des bocaux pleins de serpents, étiquetés selon les espèces; des lézards desséchés reluisaient comme des émeraudes taillées dans de grands cadres de bois noir; enfin, des bottes d'herbes sauvages, odoriférantes et sans doute douées de vertus inconnues au vulgaire des hommes, étaient attachées au plafond et descendaient dans les angles de l'appartement. 

Du reste, pas de famille, pas de serviteurs; l'homme à la haute taille habitait seul cette maison. 

Athos jeta un coup d'œil froid et indifférent sur tous les objets que nous venons de décrire, et, sur l'invitation de celui qu'il venait chercher, il s'assit près de lui. 

Alors il lui expliqua la cause de sa visite et le service qu'il réclamait de lui; mais à peine eut-il exposé sa demande, que l'inconnu, qui était resté debout devant le mousquetaire, recula de terreur et refusa. Alors Athos tira de sa poche un petit papier sur lequel étaient écrites deux lignes accompagnées d'une signature et d'un sceau, et le présenta à celui qui donnait trop prématurément ces signes de répugnance. L'homme à la grande taille eut à peine lu ces deux lignes, vu la signature et reconnu le sceau, qu'il s'inclina en signe qu'il n'avait plus aucune objection à faire, et qu'il était prêt à obéir. 

Athos n'en demanda pas davantage; il se leva, salua, sortit, reprit en s'en allant le chemin qu'il avait suivi pour venir, rentra dans l'hôtel et s'enferma chez lui. 

Au point du jour, d'Artagnan entra dans sa chambre et demanda ce qu'il fallait faire. 

«Attendre», répondit Athos. 

Quelques instants après, la supérieure du couvent fit prévenir les mousquetaires que l'enterrement de la victime de Milady aurait lieu à midi. Quant à l'empoisonneuse, on n'en avait pas eu de nouvelles; seulement elle avait dû fuir par le jardin, sur le sable duquel on avait reconnu la trace de ses pas et dont on avait retrouvé la porte fermée; quant à la clé, elle avait disparu. 

À l'heure indiquée, Lord de Winter et les quatre amis se rendirent au couvent: les cloches sonnaient à toute volée, la chapelle était ouverte, la grille du chœur était fermée. Au milieu du chœur, le corps de la victime, revêtue de ses habits de novice, était exposé. De chaque côté du chœur et derrière des grilles s'ouvrant sur le couvent était toute la communauté des Carmélites, qui écoutait de là le service divin et mêlait son chant au chant des prêtres, sans voir les profanes et sans être vue d'eux. 

À la porte de la chapelle, d'Artagnan sentit son courage qui fuyait de nouveau; il se retourna pour chercher Athos, mais Athos avait disparu. 

Fidèle à sa mission de vengeance, Athos s'était fait conduire au jardin; et là, sur le sable, suivant les pas légers de cette femme qui avait laissé une trace sanglante partout où elle avait passé, il s'avança jusqu'à la porte qui donnait sur le bois, se la fit ouvrir, et s'enfonça dans la forêt. 

Alors tous ses doutes se confirmèrent: le chemin par lequel la voiture avait disparu contournait la forêt. Athos suivit le chemin quelque temps les yeux fixés sur le sol; de légères taches de sang, qui provenaient d'une blessure faite ou à l'homme qui accompagnait la voiture en courrier, ou à l'un des chevaux, piquetaient le chemin. Au bout de trois quarts de lieue à peu près, à cinquante pas de Festubert, une tache de sang plus large apparaissait; le sol était piétiné par les chevaux. Entre la forêt et cet endroit dénonciateur, un peu en arrière de la terre écorchée, on retrouvait la même trace de petits pas que dans le jardin; la voiture s'était arrêtée. 

En cet endroit, Milady était sortie du bois et était montée dans la voiture. 

Satisfait de cette découverte qui confirmait tous ses soupçons, Athos revint à l'hôtel et trouva Planchet qui l'attendait avec impatience. 

Tout était comme l'avait prévu Athos. 

Planchet avait suivi la route, avait comme Athos remarqué les taches de sang, comme Athos il avait reconnu l'endroit où les chevaux s'étaient arrêtés; mais il avait poussé plus loin qu'Athos, de sorte qu'au village de Festubert, en buvant dans une auberge, il avait, sans avoir eu besoin de questionner, appris que la veille, à huit heures et demie du soir, un homme blessé, qui accompagnait une dame qui voyageait dans une chaise de poste, avait été obligé de s'arrêter, ne pouvant aller plus loin. L'accident avait été mis sur le compte de voleurs qui auraient arrêté la chaise dans le bois. L'homme était resté dans le village, la femme avait relayé et continué son chemin. 

Planchet se mit en quête du postillon qui avait conduit la chaise, et le retrouva. Il avait conduit la dame jusqu'à Fromelles, et de Fromelles elle était partie pour Armentières. Planchet prit la traverse, et à sept heures du matin il était à Armentières. 

Il n'y avait qu'un seul hôtel, celui de la Poste. Planchet alla s'y présenter comme un laquais sans place qui cherchait une condition. Il n'avait pas causé dix minutes avec les gens de l'auberge, qu'il savait qu'une femme seule était arrivée à onze heures du soir, avait pris une chambre, avait fait venir le maître d'hôtel et lui avait dit qu'elle désirerait demeurer quelque temps dans les environs. 

Planchet n'avait pas besoin d'en savoir davantage. Il courut au rendez-vous, trouva les trois laquais exacts à leur poste, les plaça en sentinelles à toutes les issues de l'hôtel, et vint trouver Athos, qui achevait de recevoir les renseignements de Planchet, lorsque ses amis rentrèrent. 

Tous les visages étaient sombres et crispés, même le doux visage d'Aramis. 

«Que faut-il faire? demanda d'Artagnan. 

\speak  Attendre», répondit Athos. 

Chacun se retira chez soi. 

À huit heures du soir, Athos donna l'ordre de seller les chevaux, et fit prévenir Lord de Winter et ses amis qu'ils eussent à se préparer pour l'expédition. 

En un instant tous cinq furent prêts. Chacun visita ses armes et les mit en état. Athos descendit le premier et trouva d'Artagnan déjà à cheval et s'impatientant. 

«Patience, dit Athos, il nous manque encore quelqu'un.» 

Les quatre cavaliers regardèrent autour d'eux avec étonnement, car ils cherchaient inutilement dans leur esprit quel était ce quelqu'un qui pouvait leur manquer. 

En ce moment Planchet amena le cheval d'Athos, le mousquetaire sauta légèrement en selle. 

«Attendez-moi, dit-il, je reviens.» 

Et il partit au galop. 

Un quart d'heure après, il revint effectivement accompagné d'un homme masqué et enveloppé d'un grand manteau rouge. 

Lord de Winter et les trois mousquetaires s'interrogèrent du regard. Nul d'entre eux ne put renseigner les autres, car tous ignoraient ce qu'était cet homme. Cependant ils pensèrent que cela devait être ainsi, puisque la chose se faisait par l'ordre d'Athos. 

À neuf heures, guidée par Planchet, la petite cavalcade se mit en route, prenant le chemin qu'avait suivi la voiture. 

C'était un triste aspect que celui de ces six hommes courant en silence, plongés chacun dans sa pensée, mornes comme le désespoir, sombres comme le châtiment.