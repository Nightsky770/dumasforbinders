%!TeX root=../musketeersfr.tex 

\chapter{Plan De Campagne} 
	
\lettrine{D}{'Artagnan} se rendit droit chez M. de Tréville. Il avait réfléchi que, dans quelques minutes, le cardinal serait averti par ce damné inconnu, qui paraissait être son agent, et il pensait avec raison qu'il n'y avait pas un instant à perdre. 

Le cœur du jeune homme débordait de joie. Une occasion où il y avait à la fois gloire à acquérir et argent à gagner se présentait à lui, et, comme premier encouragement, venait de le rapprocher d'une femme qu'il adorait. Ce hasard faisait donc presque du premier coup, pour lui, plus qu'il n'eût osé demander à la Providence. 

M. de Tréville était dans son salon avec sa cour habituelle de gentilshommes. D'Artagnan, que l'on connaissait comme un familier de la maison, alla droit à son cabinet et le fit prévenir qu'il l'attendait pour chose d'importance. 

D'Artagnan était là depuis cinq minutes à peine, lorsque M. de Tréville entra. Au premier coup d'œil et à la joie qui se peignait sur son visage, le digne capitaine comprit qu'il se passait effectivement quelque chose de nouveau. 

Tout le long de la route, d'Artagnan s'était demandé s'il se confierait à M. de Tréville, ou si seulement il lui demanderait de lui accorder carte blanche pour une affaire secrète. Mais M. de Tréville avait toujours été si parfait pour lui, il était si fort dévoué au roi et à la reine, il haïssait si cordialement le cardinal, que le jeune homme résolut de tout lui dire. 

«Vous m'avez fait demander, mon jeune ami? dit M. de Tréville. 

\speak  Oui, monsieur, dit d'Artagnan, et vous me pardonnerez, je l'espère, de vous avoir dérangé, quand vous saurez de quelle chose importante il est question. 

\speak  Dites alors, je vous écoute. 

\speak  Il ne s'agit de rien de moins, dit d'Artagnan, en baissant la voix, que de l'honneur et peut-être de la vie de la reine. 

\speak  Que dites-vous là? demanda M. de Tréville en regardant tout autour de lui s'ils étaient bien seuls, et en ramenant son regard interrogateur sur d'Artagnan. 

\speak  Je dis, monsieur, que le hasard m'a rendu maître d'un secret\dots 

\speak  Que vous garderez, j'espère, jeune homme, sur votre vie. 

\speak  Mais que je dois vous confier, à vous, Monsieur, car vous seul pouvez m'aider dans la mission que je viens de recevoir de Sa Majesté. 

\speak  Ce secret est-il à vous? 

\speak  Non, monsieur, c'est celui de la reine. 

\speak  Êtes-vous autorisé par Sa Majesté à me le confier? 

\speak  Non, monsieur, car au contraire le plus profond mystère m'est recommandé. 

\speak  Et pourquoi donc allez-vous le trahir vis-à-vis de moi? 

\speak  Parce que, je vous le dis, sans vous je ne puis rien, et que j'ai peur que vous ne me refusiez la grâce que je viens vous demander, si vous ne savez pas dans quel but je vous la demande. 

\speak  Gardez votre secret, jeune homme, et dites-moi ce que vous désirez. 

\speak  Je désire que vous obteniez pour moi, de M. des Essarts, un congé de quinze jours. 

\speak  Quand cela? 

\speak  Cette nuit même. 

\speak  Vous quittez Paris? 

\speak  Je vais en mission. 

\speak  Pouvez-vous me dire où? 

\speak  À Londres. 

\speak  Quelqu'un a-t-il intérêt à ce que vous n'arriviez pas à votre but? 

\speak  Le cardinal, je le crois, donnerait tout au monde pour m'empêcher de réussir. 

\speak  Et vous partez seul? 

\speak  Je pars seul. 

\speak  En ce cas, vous ne passerez pas Bondy; c'est moi qui vous le dis, foi de Tréville. 

\speak  Comment cela? 

\speak  On vous fera assassiner. 

\speak  Je serai mort en faisant mon devoir. 

\speak  Mais votre mission ne sera pas remplie. 

\speak  C'est vrai, dit d'Artagnan. 

\speak  Croyez-moi, continua Tréville, dans les entreprises de ce genre, il faut être quatre pour arriver un. 

\speak  Ah! vous avez raison, Monsieur, dit d'Artagnan; mais vous connaissez Athos, Porthos et Aramis, et vous savez si je puis disposer d'eux. 

\speak  Sans leur confier le secret que je n'ai pas voulu savoir? 

\speak  Nous nous sommes juré, une fois pour toutes, confiance aveugle et dévouement à toute épreuve; d'ailleurs vous pouvez leur dire que vous avez toute confiance en moi, et ils ne seront pas plus incrédules que vous. 

\speak  Je puis leur envoyer à chacun un congé de quinze jours, voilà tout: à Athos, que sa blessure fait toujours souffrir, pour aller aux eaux de Forges! à Porthos et à Aramis, pour suivre leur ami, qu'ils ne veulent pas abandonner dans une si douloureuse position. L'envoi de leur congé sera la preuve que j'autorise leur voyage. 

\speak  Merci, monsieur, et vous êtes cent fois bon. 

\speak  Allez donc les trouver à l'instant même, et que tout s'exécute cette nuit. Ah! et d'abord écrivez-moi votre requête à M. des Essarts. Peut-être aviez-vous un espion à vos trousses, et votre visite, qui dans ce cas est déjà connue du cardinal, sera légitimée ainsi.» 

D'Artagnan formula cette demande, et M. de Tréville, en la recevant de ses mains, assura qu'avant deux heures du matin les quatre congés seraient au domicile respectif des voyageurs. 

«Ayez la bonté d'envoyer le mien chez Athos, dit d'Artagnan. Je craindrais, en rentrant chez moi, d'y faire quelque mauvaise rencontre. 

\speak  Soyez tranquille. Adieu et bon voyage! À propos!» dit M. de Tréville en le rappelant. 

D'Artagnan revint sur ses pas. 

«Avez-vous de l'argent?» 

D'Artagnan fit sonner le sac qu'il avait dans sa poche. 

«Assez? demanda M. de Tréville. 

\speak  Trois cents pistoles. 

\speak  C'est bien, on va au bout du monde avec cela; allez donc.» 

D'Artagnan salua M. de Tréville, qui lui tendit la main; d'Artagnan la lui serra avec un respect mêlé de reconnaissance. Depuis qu'il était arrivé à Paris, il n'avait eu qu'à se louer de cet excellent homme, qu'il avait toujours trouvé digne, loyal et grand. 

Sa première visite fut pour Aramis; il n'était pas revenu chez son ami depuis la fameuse soirée où il avait suivi Mme Bonacieux. Il y a plus: à peine avait-il vu le jeune mousquetaire, et à chaque fois qu'il l'avait revu, il avait cru remarquer une profonde tristesse empreinte sur son visage. 

Ce soir encore, Aramis veillait sombre et rêveur; d'Artagnan lui fit quelques questions sur cette mélancolie profonde; Aramis s'excusa sur un commentaire du dix-huitième chapitre de saint Augustin qu'il était forcé d'écrire en latin pour la semaine suivante, et qui le préoccupait beaucoup. 

Comme les deux amis causaient depuis quelques instants, un serviteur de M. de Tréville entra porteur d'un paquet cacheté. 

«Qu'est-ce là? demanda Aramis. 

\speak  Le congé que monsieur a demandé, répondit le laquais. 

\speak  Moi, je n'ai pas demandé de congé. 

\speak  Taisez-vous et prenez, dit d'Artagnan. Et vous, mon ami, voici une demi-pistole pour votre peine; vous direz à M. de Tréville que M. Aramis le remercie bien sincèrement. Allez.» 

Le laquais salua jusqu'à terre et sortit. 

«Que signifie cela? demanda Aramis. 

\speak  Prenez ce qu'il vous faut pour un voyage de quinze jours, et suivez-moi. 

\speak  Mais je ne puis quitter Paris en ce moment, sans savoir\dots» 

Aramis s'arrêta. 

«Ce qu'elle est devenue, n'est-ce pas? continua d'Artagnan. 

\speak  Qui? reprit Aramis. 

\speak  La femme qui était ici, la femme au mouchoir brodé. 

\speak  Qui vous a dit qu'il y avait une femme ici? répliqua Aramis en devenant pâle comme la mort. 

\speak  Je l'ai vue. 

\speak  Et vous savez qui elle est? 

\speak  Je crois m'en douter, du moins. 

\speak  Écoutez, dit Aramis, puisque vous savez tant de choses, savez-vous ce qu'est devenue cette femme? 

\speak  Je présume qu'elle est retournée à Tours. 

\speak  À Tours? oui, c'est bien cela, vous la connaissez. Mais comment est-elle retournée à Tours sans me rien dire? 

\speak  Parce qu'elle a craint d'être arrêtée. 

\speak  Comment ne m'a-t-elle pas écrit? 

\speak  Parce qu'elle craint de vous compromettre. 

\speak  D'Artagnan, vous me rendez la vie! s'écria Aramis. Je me croyais méprisé, trahi. J'étais si heureux de la revoir! Je ne pouvais croire qu'elle risquât sa liberté pour moi, et cependant pour quelle cause serait-elle revenue à Paris? 

\speak  Pour la cause qui aujourd'hui nous fait aller en Angleterre. 

\speak  Et quelle est cette cause? demanda Aramis. 

\speak  Vous le saurez un jour, Aramis; mais, pour le moment, j'imiterai la retenue de la \textit{nièce du docteur}.» 

Aramis sourit, car il se rappelait le conte qu'il avait fait certain soir à ses amis. 

«Eh bien, donc, puisqu'elle a quitté Paris et que vous en êtes sûr, d'Artagnan, rien ne m'y arrête plus, et je suis prêt à vous suivre. Vous dites que nous allons?\dots 

\speak  Chez Athos, pour le moment, et si vous voulez venir, je vous invite même à vous hâter, car nous avons déjà perdu beaucoup de temps. À propos, prévenez Bazin. 

\speak  Bazin vient avec nous? demanda Aramis. 

\speak  Peut-être. En tout cas, il est bon qu'il nous suive pour le moment chez Athos.» 

Aramis appela Bazin, et après lui avoir ordonné de le venir joindre chez Athos: 

«Partons donc», dit-il en prenant son manteau, son épée et ses trois pistolets, et en ouvrant inutilement trois ou quatre tiroirs pour voir s'il n'y trouverait pas quelque pistole égarée. Puis, quand il se fut bien assuré que cette recherche était superflue, il suivit d'Artagnan en se demandant comment il se faisait que le jeune cadet aux gardes sût aussi bien que lui quelle était la femme à laquelle il avait donné l'hospitalité, et sût mieux que lui ce qu'elle était devenue. 

Seulement, en sortant, Aramis posa sa main sur le bras de d'Artagnan, et le regardant fixement: 

«Vous n'avez parlé de cette femme à personne? dit-il. 

\speak  À personne au monde. 

\speak  Pas même à Athos et à Porthos? 

\speak  Je ne leur en ai pas soufflé le moindre mot. 

\speak  À la bonne heure.» 

Et, tranquille sur ce point important, Aramis continua son chemin avec d'Artagnan, et tous deux arrivèrent bien tôt chez Athos. 

Ils le trouvèrent tenant son congé d'une main et la lettre de M. de Tréville de l'autre. 

«Pouvez-vous m'expliquer ce que signifient ce congé et cette lettre que je viens de recevoir?» dit Athos étonné. 

\begin{mail}{}{Mon cher Athos,}
Je veux bien, puisque votre santé l'exige absolument, que vous vous reposiez quinze jours. Allez donc prendre les eaux de Forges ou telles autres qui vous conviendront, et rétablissez-vous promptement 

\closeletter[Votre affectionné]{Tréville}
\end{mail}


«Eh bien, ce congé et cette lettre signifient qu'il faut me suivre, Athos. 

\speak  Aux eaux de Forges? 

\speak  Là ou ailleurs. 

\speak  Pour le service du roi? 

\speak  Du roi ou de la reine: ne sommes-nous pas serviteurs de Leurs Majestés?» 

En ce moment, Porthos entra. 

«Pardieu, dit-il, voici une chose étrange: depuis quand, dans les mousquetaires, accorde-t-on aux gens des congés sans qu'ils les demandent? 

\speak  Depuis, dit d'Artagnan, qu'ils ont des amis qui les demandent pour eux. 

\speak  Ah! ah! dit Porthos, il paraît qu'il y a du nouveau ici? 

\speak  Oui, nous partons, dit Aramis. 

\speak  Pour quel pays? demanda Porthos. 

\speak  Ma foi, je n'en sais trop rien, dit Athos; demande cela à d'Artagnan. 

\speak  Pour Londres, messieurs, dit d'Artagnan. 

\speak  Pour Londres! s'écria Porthos; et qu'allons-nous faire à Londres? 

\speak  Voilà ce que je ne puis vous dire, messieurs, et il faut vous fier à moi. 

\speak  Mais pour aller à Londres, ajouta Porthos, il faut de l'argent, et je n'en ai pas. 

\speak  Ni moi, dit Aramis. 

\speak  Ni moi, dit Athos. 

\speak  J'en ai, moi, reprit d'Artagnan en tirant son trésor de sa poche et en le posant sur la table. Il y a dans ce sac trois cents pistoles; prenons-en chacun soixante-quinze; c'est autant qu'il en faut pour aller à Londres et pour en revenir. D'ailleurs, soyez tranquilles, nous n'y arriverons pas tous, à Londres. 

\speak  Et pourquoi cela? 

\speak  Parce que, selon toute probabilité, il y en aura quelques-uns d'entre nous qui resteront en route. 

\speak  Mais est-ce donc une campagne que nous entreprenons? 

\speak  Et des plus dangereuses, je vous en avertis. 

\speak  Ah çà, mais, puisque nous risquons de nous faire tuer, dit Porthos, je voudrais bien savoir pourquoi, au moins? 

\speak  Tu en seras bien plus avancé! dit Athos. 

\speak  Cependant, dit Aramis, je suis de l'avis de Porthos. 

\speak  Le roi a-t-il l'habitude de vous rendre des comptes? Non; il vous dit tout bonnement: “Messieurs, on se bat en Gascogne ou dans les Flandres; allez vous battre”, et vous y allez. Pourquoi? vous ne vous en inquiétez même pas. 

\speak  D'Artagnan a raison, dit Athos, voilà nos trois congés qui viennent de M. de Tréville, et voilà trois cents pistoles qui viennent je ne sais d'où. Allons nous faire tuer où l'on nous dit d'aller. La vie vaut-elle la peine de faire autant de questions? D'Artagnan, je suis prêt à te suivre. 

\speak  Et moi aussi, dit Porthos. 

\speak  Et moi aussi, dit Aramis. Aussi bien, je ne suis pas fâché de quitter Paris. J'ai besoin de distractions. 

\speak  Eh bien, vous en aurez, des distractions, messieurs, soyez tranquilles, dit d'Artagnan. 

\speak  Et maintenant, quand partons-nous? dit Athos. 

\speak  Tout de suite, répondit d'Artagnan, il n'y a pas une minute à perdre. 

\speak  Holà! Grimaud, Planchet, Mousqueton, Bazin! crièrent les quatre jeunes gens appelant leurs laquais, graissez nos bottes et ramenez les chevaux de l'hôtel.» 

En effet, chaque mousquetaire laissait à l'hôtel général comme à une caserne son cheval et celui de son laquais. 

Planchet, Grimaud, Mousqueton et Bazin partirent en toute hâte. 

«Maintenant, dressons le plan de campagne, dit Porthos. Où allons-nous d'abord? 

\speak  À Calais, dit d'Artagnan; c'est la ligne la plus directe pour arriver à Londres. 

\speak  Eh bien, dit Porthos, voici mon avis. 

\speak  Parle. 

\speak  Quatre hommes voyageant ensemble seraient suspects: d'Artagnan nous donnera à chacun ses instructions, je partirai en avant par la route de Boulogne pour éclairer le chemin; Athos partira deux heures après par celle d'Amiens; Aramis nous suivra par celle de Noyon; quant à d'Artagnan, il partira par celle qu'il voudra, avec les habits de Planchet, tandis que Planchet nous suivra en d'Artagnan et avec l'uniforme des gardes. 

\speak  Messieurs, dit Athos, mon avis est qu'il ne convient pas de mettre en rien des laquais dans une pareille affaire: un secret peut par hasard être trahi par des gentilshommes, mais il est presque toujours vendu par des laquais. 

\speak  Le plan de Porthos me semble impraticable, dit d'Artagnan, en ce que j'ignore moi-même quelles instructions je puis vous donner. Je suis porteur d'une lettre, voilà tout. Je n'ai pas et ne puis faire trois copies de cette lettre, puisqu'elle est scellée; il faut donc, à mon avis, voyager de compagnie. Cette lettre est là, dans cette poche. Et il montra la poche où était la lettre. Si je suis tué, l'un de vous la prendra et vous continuerez la route; s'il est tué, ce sera le tour d'un autre, et ainsi de suite; pourvu qu'un seul arrive, c'est tout ce qu'il faut. 

\speak  Bravo, d'Artagnan! ton avis est le mien, dit Athos. Il faut être conséquent, d'ailleurs: je vais prendre les eaux, vous m'accompagnerez; au lieu des eaux de Forges, je vais prendre les eaux de mer; je suis libre. On veut nous arrêter, je montre la lettre de M. de Tréville, et vous montrez vos congés; on nous attaque, nous nous défendons; on nous juge, nous soutenons mordicus que nous n'avions d'autre intention que de nous tremper un certain nombre de fois dans la mer; on aurait trop bon marché de quatre hommes isolés, tandis que quatre hommes réunis font une troupe. Nous armerons les quatre laquais de pistolets et de mousquetons; si l'on envoie une armée contre nous, nous livrerons bataille, et le survivant, comme l'a dit d'Artagnan, portera la lettre. 

\speak  Bien dit, s'écria Aramis; tu ne parles pas souvent, Athos, mais quand tu parles, c'est comme saint Jean Bouche d'or. J'adopte le plan d'Athos. Et toi, Porthos? 

\speak  Moi aussi, dit Porthos, s'il convient à d'Artagnan. D'Artagnan, porteur de la lettre, est naturellement le chef de l'entreprise; qu'il décide, et nous exécuterons. 

\speak  Eh bien, dit d'Artagnan, je décide que nous adoptions le plan d'Athos et que nous partions dans une demi-heure. 

\speak  Adopté!» reprirent en chœur les trois mousquetaires. 

Et chacun, allongeant la main vers le sac, prit soixante-quinze pistoles et fit ses préparatifs pour partir à l'heure convenue. 