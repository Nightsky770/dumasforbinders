%!TeX root=../musketeersfr.tex 

\chapter{Un Dîner De Procureur}

\lettrine{C}{ependant} le duel dans lequel Porthos avait joué un rôle si brillant ne lui avait pas fait oublier le dîner auquel l'avait invité la femme du procureur. Le lendemain, vers une heure, il se fit donner le dernier coup de brosse par Mousqueton, et s'achemina vers la rue aux Ours, du pas d'un homme qui est en double bonne fortune. 

Son cœur battait, mais ce n'était pas, comme celui de d'Artagnan, d'un jeune et impatient amour. Non, un intérêt plus matériel lui fouettait le sang, il allait enfin franchir ce seuil mystérieux, gravir cet escalier inconnu qu'avaient monté, un à un, les vieux écus de maître Coquenard. 

Il allait voir en réalité certain bahut dont vingt fois il avait vu l'image dans ses rêves; bahut de forme longue et profonde, cadenassé, verrouillé, scellé au sol; bahut dont il avait si souvent entendu parler, et que les mains un peu sèches, il est vrai, mais non pas sans élégance de la procureuse, allaient ouvrir à ses regards admirateurs. 

Et puis lui, l'homme errant sur la terre, l'homme sans fortune, l'homme sans famille, le soldat habitué aux auberges, aux cabarets, aux tavernes, aux posadas, le gourmet forcé pour la plupart du temps de s'en tenir aux lippées de rencontre, il allait tâter des repas de ménage, savourer un intérieur confortable, et se laisser faire à ces petits soins, qui, plus on est dur, plus ils plaisent, comme disent les vieux soudards. 

Venir en qualité de cousin s'asseoir tous les jours à une bonne table, dérider le front jaune et plissé du vieux procureur, plumer quelque peu les jeunes clercs en leur apprenant la bassette, le passe-dix et le lansquenet dans leurs plus fines pratiques, et en leur gagnant par manière d'honoraires, pour la leçon qu'il leur donnerait en une heure, leurs économies d'un mois, tout cela souriait énormément à Porthos. 

Le mousquetaire se retraçait bien, de-ci, de-là, les mauvais propos qui couraient dès ce temps-là sur les procureurs et qui leur ont survécu: la lésine, la rognure, les jours de jeûne, mais comme, après tout, sauf quelques accès d'économie que Porthos avait toujours trouvés fort intempestifs, il avait vu la procureuse assez libérale, pour une procureuse, bien entendu, il espéra rencontrer une maison montée sur un pied flatteur. 

Cependant, à la porte, le mousquetaire eut quelques doutes, l'abord n'était point fait pour engager les gens: allée puante et noire, escalier mal éclairé par des barreaux au travers desquels filtrait le jour gris d'une cour voisine; au premier une porte basse et ferrée d'énorme clous comme la porte principale du Grand-Châtelet. 

Porthos heurta du doigt; un grand clerc pâle et enfoui sous une forêt de cheveux vierges vint ouvrir et salua de l'air d'un homme forcé de respecter à la fois dans un autre la haute taille qui indique la force, l'habit militaire qui indique l'état, et la mine vermeille qui indique l'habitude de bien vivre. 

Autre clerc plus petit derrière le premier, autre clerc plus grand derrière le second, saute-ruisseau de douze ans derrière le troisième. 

En tout, trois clercs et demi; ce qui, pour le temps, annonçait une étude des plus achalandées. 

Quoique le mousquetaire ne dût arriver qu'à une heure, depuis midi la procureuse avait l'œil au guet et comptait sur le cœur et peut-être aussi sur l'estomac de son adorateur pour lui faire devancer l'heure. 

Mme Coquenard arriva donc par la porte de l'appartement, presque en même temps que son convive arrivait par la porte de l'escalier, et l'apparition de la digne dame le tira d'un grand embarras. Les clercs avaient l'œil curieux, et lui, ne sachant trop que dire à cette gamme ascendante et descendante, demeurait la langue muette. 

«C'est mon cousin, s'écria la procureuse; entrez donc, entrez donc, monsieur Porthos.» 

Le nom de Porthos fit son effet sur les clercs, qui se mirent à rire; mais Porthos se retourna, et tous les visages rentrèrent dans leur gravité. 

On arriva dans le cabinet du procureur après avoir traversé l'antichambre où étaient les clercs, et l'étude où ils auraient dû être: cette dernière chambre était une sorte de salle noire et meublée de paperasses. En sortant de l'étude on laissa la cuisine à droite, et l'on entra dans la salle de réception. 

Toutes ces pièces qui se commandaient n'inspirèrent point à Porthos de bonnes idées. Les paroles devaient s'entendre de loin par toutes ces portes ouvertes; puis, en passant, il avait jeté un regard rapide et investigateur sur la cuisine, et il s'avouait à lui-même, à la honte de la procureuse et à son grand regret, à lui, qu'il n'y avait pas vu ce feu, cette animation, ce mouvement qui, au moment d'un bon repas, règnent ordinairement dans ce sanctuaire de la gourmandise. 

Le procureur avait sans doute été prévenu de cette visite, car il ne témoigna aucune surprise à la vue de Porthos, qui s'avança jusqu'à lui d'un air assez dégagé et le salua courtoisement. 

«Nous sommes cousins, à ce qu'il paraît, monsieur Porthos?» dit le procureur en se soulevant à la force des bras sur son fauteuil de canne. 

Le vieillard, enveloppé dans un grand pourpoint noir où se perdait son corps fluet, était vert et sec; ses petits yeux gris brillaient comme des escarboucles, et semblaient, avec sa bouche grimaçante, la seule partie de son visage où la vie fût demeurée. Malheureusement les jambes commençaient à refuser le service à toute cette machine osseuse; depuis cinq ou six mois que cet affaiblissement s'était fait sentir, le digne procureur était à peu près devenu l'esclave de sa femme. 

Le cousin fut accepté avec résignation, voilà tout. Maître Coquenard ingambe eût décliné toute parenté avec M. Porthos. 

«Oui, monsieur, nous sommes cousins, dit sans se déconcerter Porthos, qui, d'ailleurs, n'avait jamais compté être reçu par le mari avec enthousiasme. 

\speak  Par les femmes, je crois?» dit malicieusement le procureur. 

Porthos ne sentit point cette raillerie et la prit pour une naïveté dont il rit dans sa grosse moustache. Mme Coquenard, qui savait que le procureur naïf était une variété fort rare dans l'espèce, sourit un peu et rougit beaucoup. 

Maître Coquenard avait, dès l'arrivée de Porthos, jeté les yeux avec inquiétude sur une grande armoire placée en face de son bureau de chêne. Porthos comprit que cette armoire, quoiqu'elle ne répondît point par la forme à celle qu'il avait vue dans ses songes, devait être le bienheureux bahut, et il s'applaudit de ce que la réalité avait six pieds de plus en hauteur que le rêve. 

Maître Coquenard ne poussa pas plus loin ses investigations généalogiques, mais en ramenant son regard inquiet de l'armoire sur Porthos, il se contenta de dire: 

«Monsieur notre cousin, avant son départ pour la campagne, nous fera bien la grâce de dîner une fois avec nous, n'est-ce pas, madame Coquenard!» 

Cette fois, Porthos reçut le coup en plein estomac et le sentit; il paraît que de son côté Mme Coquenard non plus n'y fut pas insensible, car elle ajouta: 

«Mon cousin ne reviendra pas s'il trouve que nous le traitons mal; mais, dans le cas contraire, il a trop peu de temps à passer à Paris, et par conséquent à nous voir, pour que nous ne lui demandions pas presque tous les instants dont il peut disposer jusqu'à son départ. 

\speak  Oh! mes jambes, mes pauvres jambes! où êtes-vous?» murmura Coquenard. Et il essaya de sourire. 

Ce secours qui était arrivé à Porthos au moment où il était attaqué dans ses espérances gastronomiques inspira au mousquetaire beaucoup de reconnaissance pour sa procureuse. 

Bientôt l'heure du dîner arriva. On passa dans la salle à manger, grande pièce noire qui était située en face de la cuisine. 

Les clercs, qui, à ce qu'il paraît, avaient senti dans la maison des parfums inaccoutumés, étaient d'une exactitude militaire, et tenaient en main leurs tabourets, tout prêts qu'ils étaient à s'asseoir. On les voyait d'avance remuer les mâchoires avec des dispositions effrayantes. 

«Tudieu! pensa Porthos en jetant un regard sur les trois affamés, car le saute-ruisseau n'était pas, comme on le pense bien, admis aux honneurs de la table magistrale; tudieu! à la place de mon cousin, je ne garderais pas de pareils gourmands. On dirait des naufragés qui n'ont pas mangé depuis six semaines.» 

Maître Coquenard entra, poussé sur son fauteuil à roulettes par Mme Coquenard, à qui Porthos, à son tour, vint en aide pour rouler son mari jusqu'à la table. 

À peine entré, il remua le nez et les mâchoires à l'exemple de ses clercs. 

«Oh! oh! dit-il, voici un potage qui est engageant!» 

«Que diable sentent-ils donc d'extraordinaire dans ce potage?» dit Porthos à l'aspect d'un bouillon pâle, abondant, mais parfaitement aveugle, et sur lequel quelques croûtes nageaient rares comme les îles d'un archipel. 

Mme Coquenard sourit, et, sur un signe d'elle, tout le monde s'assit avec empressement. 

Maître Coquenard fut le premier servi, puis Porthos; ensuite Mme Coquenard emplit son assiette, et distribua les croûtes sans bouillon aux clercs impatients. 

En ce moment la porte de la salle à manger s'ouvrit d'elle-même en criant, et Porthos, à travers les battants entrebâillés, aperçut le petit clerc, qui, ne pouvant prendre part au festin, mangeait son pain à la double odeur de la cuisine et de la salle à manger. 

Après le potage la servante apporta une poule bouillie; magnificence qui fit dilater les paupières des convives, de telle façon qu'elles semblaient prêtes à se fendre. 

«On voit que vous aimez votre famille, madame Coquenard, dit le procureur avec un sourire presque tragique; voilà certes une galanterie que vous faites à votre cousin.» 

La pauvre poule était maigre et revêtue d'une de ces grosses peaux hérissées que les os ne percent jamais malgré leurs efforts; il fallait qu'on l'eût cherchée bien longtemps avant de la trouver sur le perchoir où elle s'était retirée pour mourir de vieillesse. 

«Diable! pensa Porthos, voilà qui est fort triste; je respecte la vieillesse, mais j'en fais peu de cas bouillie ou rôtie.» 

Et il regarda à la ronde pour voir si son opinion était partagée; mais tout au contraire de lui, il ne vit que des yeux flamboyants, qui dévoraient d'avance cette sublime poule, objet de ses mépris. 

Mme Coquenard tira le plat à elle, détacha adroitement les deux grandes pattes noires, qu'elle plaça sur l'assiette de son mari; trancha le cou, qu'elle mit avec la tête à part pour elle-même; leva l'aile pour Porthos, et remit à la servante, qui venait de l'apporter, l'animal qui s'en retourna presque intact, et qui avait disparu avant que le mousquetaire eût eu le temps d'examiner les variations que le désappointement amène sur les visages, selon les caractères et les tempéraments de ceux qui l'éprouvent. 

Au lieu de poulet, un plat de fèves fit son entrée, plat énorme, dans lequel quelques os de mouton, qu'on eût pu, au premier abord, croire accompagnés de viande, faisaient semblant de se montrer. 

Mais les clercs ne furent pas dupes de cette supercherie, et les mines lugubres devinrent des visages résignés. 

Mme Coquenard distribua ce mets aux jeunes gens avec la modération d'une bonne ménagère. 

Le tour du vin était venu. Maître Coquenard versa d'une bouteille de grès fort exiguë le tiers d'un verre à chacun des jeunes gens, s'en versa à lui-même dans des proportions à peu près égales, et la bouteille passa aussitôt du côté de Porthos et de Mme Coquenard. 

Les jeunes gens remplissaient d'eau ce tiers de vin, puis, lorsqu'ils avaient bu la moitié du verre, ils le remplissaient encore, et ils faisaient toujours ainsi; ce qui les amenait à la fin du repas à avaler une boisson qui de la couleur du rubis était passée à celle de la topaze brûlée. 

Porthos mangea timidement son aile de poule, et frémit lorsqu'il sentit sous la table le genou de la procureuse qui venait trouver le sien. Il but aussi un demi-verre de ce vin fort ménagé, et qu'il reconnut pour cet horrible cru de Montreuil, la terreur des palais exercés. 

Maître Coquenard le regarda engloutir ce vin pur et soupira. 

«Mangerez-vous bien de ces fèves, mon cousin Porthos?» dit Mme Coquenard de ce ton qui veut dire: croyez-moi, n'en mangez pas. 

«Du diable si j'en goûte!» murmura tout bas Porthos\dots 

Puis tout haut: 

«Merci, ma cousine, dit-il, je n'ai plus faim.» 

Il se fit un silence: Porthos ne savait quelle contenance tenir. Le procureur répéta plusieurs fois: 

«Ah! madame Coquenard! je vous en fais mon compliment, votre dîner était un véritable festin; Dieu! ai-je mangé!» 

Maître Coquenard avait mangé son potage, les pattes noires de la poule et le seul os de mouton où il y eût un peu de viande. 

Porthos crut qu'on le mystifiait, et commença à relever sa moustache et à froncer le sourcil; mais le genou de Mme Coquenard vint tout doucement lui conseiller la patience. 

Ce silence et cette interruption de service, qui étaient restés inintelligibles pour Porthos, avaient au contraire une signification terrible pour les clercs: sur un regard du procureur, accompagné d'un sourire de Mme Coquenard, ils se levèrent lentement de table, plièrent leurs serviettes plus lentement encore, puis ils saluèrent et partirent. 

«Allez, jeunes gens, allez faire la digestion en travaillant», dit gravement le procureur. 

Les clercs partis, Mme Coquenard se leva et tira d'un buffet un morceau de fromage, des confitures de coings et un gâteau qu'elle avait fait elle-même avec des amandes et du miel. 

Maître Coquenard fronça le sourcil, parce qu'il voyait trop de mets; Porthos se pinça les lèvres, parce qu'il voyait qu'il n'y avait pas de quoi dîner. 

Il regarda si le plat de fèves était encore là, le plat de fèves avait disparu. 

«Festin décidément, s'écria maître Coquenard en s'agitant sur sa chaise, véritable festin, \textit{epulæ epularum;} Lucullus dîne chez Lucullus.» 

Porthos regarda la bouteille qui était près de lui, et il espéra qu'avec du vin, du pain et du fromage il dînerait; mais le vin manquait, la bouteille était vide; M. et Mme Coquenard n'eurent point l'air de s'en apercevoir. 

«C'est bien, se dit Porthos à lui-même, me voilà prévenu.» 

Il passa la langue sur une petite cuillerée de confitures, et s'englua les dents dans la pâte collante de Mme Coquenard. 

«Maintenant, se dit-il, le sacrifice est consommé. Ah! si je n'avais pas l'espoir de regarder avec Mme Coquenard dans l'armoire de son mari!» 

Maître Coquenard, après les délices d'un pareil repas, qu'il appelait un excès, éprouva le besoin de faire sa sieste. Porthos espérait que la chose aurait lieu séance tenante et dans la localité même; mais le procureur maudit ne voulut entendre à rien: il fallut le conduire dans sa chambre et il cria tant qu'il ne fut pas devant son armoire, sur le rebord de laquelle, pour plus de précaution encore, il posa ses pieds. 

La procureuse emmena Porthos dans une chambre voisine et l'on commença de poser les bases de la réconciliation. 

«Vous pourrez venir dîner trois fois la semaine, dit Mme Coquenard. 

\speak  Merci, dit Porthos, je n'aime pas à abuser; d'ailleurs, il faut que je songe à mon équipement. 

\speak  C'est vrai, dit la procureuse en gémissant\dots c'est ce malheureux équipement. 

\speak  Hélas! oui, dit Porthos, c'est lui. 

\speak  Mais de quoi donc se compose l'équipement de votre corps, monsieur Porthos? 

\speak  Oh! de bien des choses, dit Porthos; les mousquetaires, comme vous savez, sont soldats d'élite, et il leur faut beaucoup d'objets inutiles aux gardes ou aux Suisses. 

\speak  Mais encore, détaillez-le-moi. 

\speak  Mais cela peut aller à\dots», dit Porthos, qui aimait mieux discuter le total que le menu. 

La procureuse attendait frémissante. 

«À combien? dit-elle, j'espère bien que cela ne passe point\dots» 

Elle s'arrêta, la parole lui manquait. 

«Oh! non, dit Porthos, cela ne passe point deux mille cinq cents livres; je crois même qu'en y mettant de l'économie, avec deux mille livres je m'en tirerai. 

\speak  Bon Dieu, deux mille livres! s'écria-t-elle, mais c'est une fortune.» 

Porthos fit une grimace des plus significatives, Mme Coquenard la comprit. 

«Je demandais le détail, dit-elle, parce qu'ayant beaucoup de parents et de pratiques dans le commerce, j'étais presque sûre d'obtenir les choses à cent pour cent au-dessous du prix où vous les payeriez vous-même. 

\speak  Ah! ah! fit Porthos, si c'est cela que vous avez voulu dire! 

\speak  Oui, cher monsieur Porthos! ainsi ne vous faut-il pas d'abord un cheval? 

\speak  Oui, un cheval. 

\speak  Eh bien, justement j'ai votre affaire. 

\speak  Ah! dit Porthos rayonnant, voilà donc qui va bien quant à mon cheval; ensuite il me faut le harnachement complet, qui se compose d'objets qu'un mousquetaire seul peut acheter, et qui ne montera pas, d'ailleurs, à plus de trois cents livres. 

\speak  Trois cents livres: alors mettons trois cents livres» dit la procureuse avec un soupir. 

Porthos sourit: on se souvient qu'il avait la selle qui lui venait de Buckingham, c'était donc trois cents livres qu'il comptait mettre sournoisement dans sa poche. 

«Puis, continua-t-il, il y a le cheval de mon laquais et ma valise; quant aux armes, il est inutile que vous vous en préoccupiez, je les ai. 

\speak  Un cheval pour votre laquais? reprit en hésitant la procureuse; mais c'est bien grand seigneur, mon ami. 

\speak  Eh! madame! dit fièrement Porthos, est-ce que je suis un croquant, par hasard? 

\speak  Non; je vous disais seulement qu'un joli mulet avait quelquefois aussi bon air qu'un cheval, et qu'il me semble qu'en vous procurant un joli mulet pour Mousqueton\dots 

\speak  Va pour un joli mulet, dit Porthos; vous avez raison, j'ai vu de très grands seigneurs espagnols dont toute la suite était à mulets. Mais alors, vous comprenez, madame Coquenard, un mulet avec des panaches et des grelots? 

\speak  Soyez tranquille, dit la procureuse. 

\speak  Reste la valise, reprit Porthos. 

\speak  Oh! que cela ne vous inquiète point, s'écria Mme Coquenard: mon mari a cinq ou six valises, vous choisirez la meilleure; il y en a une surtout qu'il affectionnait dans ses voyages, et qui est grande à tenir un monde. 

\speak  Elle est donc vide, votre valise? demanda naïvement Porthos. 

\speak  Assurément qu'elle est vide, répondit naïvement de son côté la procureuse. 

\speak  Ah! mais la valise dont j'ai besoin est une valise bien garnie, ma chère.» 

Mme Coquenard poussa de nouveaux soupirs. Molière n'avait pas encore écrit sa scène de l'Avare. Mme Coquenard a donc le pas sur Harpagon. 

Enfin le reste de l'équipement fut successivement débattu de la même manière; et le résultat de la scène fut que la procureuse demanderait à son mari un prêt de huit cents livres en argent, et fournirait le cheval et le mulet qui auraient l'honneur de porter à la gloire Porthos et Mousqueton. 

Ces conditions arrêtées, et les intérêts stipulés ainsi que l'époque du remboursement, Porthos prit congé de Mme Coquenard. Celle-ci voulait bien le retenir en lui faisant les yeux doux; mais Porthos prétexta les exigences du service, et il fallut que la procureuse cédât le pas au roi. 

Le mousquetaire rentra chez lui avec une faim de fort mauvaise humeur. 