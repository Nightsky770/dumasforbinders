%!TeX root=../musketeersfr.tex 

\chapter{Évasion}

\lettrine{C}{omme} l'avait pensé Lord de Winter, la blessure de Milady n'était pas dangereuse; aussi dès qu'elle se trouva seule avec la femme que le baron avait fait appeler et qui se hâtait de la déshabiller, rouvrit-elle les yeux. 

Cependant, il fallait jouer la faiblesse et la douleur; ce n'étaient pas choses difficiles pour une comédienne comme Milady; aussi la pauvre femme fut-elle si complètement dupe de sa prisonnière, que, malgré ses instances, elle s'obstina à la veiller toute la nuit. 

Mais la présence de cette femme n'empêchait pas Milady de songer. 

Il n'y avait plus de doute, Felton était convaincu, Felton était à elle: un ange apparût-il au jeune homme pour accuser Milady, il le prendrait certainement, dans la disposition d'esprit où il se trouvait, pour un envoyé du démon. 

Milady souriait à cette pensée, car Felton, c'était désormais sa seule espérance, son seul moyen de salut. 

Mais Lord de Winter pouvait l'avoir soupçonné, mais Felton maintenant pouvait être surveillé lui-même. 

Vers les quatre heures du matin, le médecin arriva; mais depuis le temps où Milady s'était frappée, la blessure s'était déjà refermée: le médecin ne put donc en mesurer ni la direction, ni la profondeur; il reconnut seulement au pouls de la malade que le cas n'était point grave. 

Le matin, Milady, sous prétexte qu'elle n'avait pas dormi de la nuit et qu'elle avait besoin de repos, renvoya la femme qui veillait près d'elle. 

Elle avait une espérance, c'est que Felton arriverait à l'heure du déjeuner, mais Felton ne vint pas. 

Ses craintes s'étaient-elles réalisées? Felton, soupçonné par le baron, allait-il lui manquer au moment décisif? Elle n'avait plus qu'un jour: Lord de Winter lui avait annoncé son embarquement pour le 23 et l'on était arrivé au matin du 22. 

Néanmoins, elle attendit encore assez patiemment jusqu'à l'heure du dîner. 

Quoiqu'elle n'eût pas mangé le matin, le dîner fut apporté à l'heure habituelle; Milady s'aperçut alors avec effroi que l'uniforme des soldats qui la gardaient était changé. 

Alors elle se hasarda à demander ce qu'était devenu Felton. On lui répondit que Felton était monté à cheval il y avait une heure, et était parti. 

Elle s'informa si le baron était toujours au château; le soldat répondit que oui, et qu'il avait ordre de le prévenir si la prisonnière désirait lui parler. 

Milady répondit qu'elle était trop faible pour le moment, et que son seul désir était de demeurer seule. 

Le soldat sortit, laissant le dîner servi. 

Felton était écarté, les soldats de marine étaient changés, on se défiait donc de Felton. 

C'était le dernier coup porté à la prisonnière. 

Restée seule, elle se leva; ce lit où elle se tenait par prudence et pour qu'on la crût gravement blessée, la brûlait comme un brasier ardent. Elle jeta un coup d'œil sur la porte: le baron avait fait clouer une planche sur le guichet; il craignait sans doute que, par cette ouverture, elle ne parvint encore, par quelque moyen diabolique, à séduire les gardes. 

Milady sourit de joie; elle pouvait donc se livrer à ses transports sans être observée: elle parcourait la chambre avec l'exaltation d'une folle furieuse ou d'une tigresse enfermée dans une cage de fer. Certes, si le couteau lui fût resté, elle eût songé, non plus à se tuer elle-même, mais, cette fois, à tuer le baron. 

À six heures, Lord de Winter entra; il était armé jusqu'aux dents. Cet homme, dans lequel, jusque-là, Milady n'avait vu qu'un gentleman assez niais, était devenu un admirable geôlier: il semblait tout prévoir, tout deviner, tout prévenir. 

Un seul regard jeté sur Milady lui apprit ce qui se passait dans son âme. 

«Soit, dit-il, mais vous ne me tuerez point encore aujourd'hui; vous n'avez plus d'armes, et d'ailleurs je suis sur mes gardes. Vous aviez commencé à pervertir mon pauvre Felton: il subissait déjà votre infernale influence, mais je veux le sauver, il ne vous verra plus, tout est fini. Rassemblez vos hardes, demain vous partirez. J'avais fixé l'embarquement au 24, mais j'ai pensé que plus la chose serait rapprochée, plus elle serait sûre. Demain à midi j'aurai l'ordre de votre exil, signé Buckingham. Si vous dites un seul mot à qui que ce soit avant d'être sur le navire, mon sergent vous fera sauter la cervelle, et il en a l'ordre; si, sur le navire, vous dites un mot à qui que ce soit avant que le capitaine vous le permette, le capitaine vous fait jeter à la mer, c'est convenu. Au revoir, voilà ce que pour aujourd'hui j'avais à vous dire. Demain je vous reverrai pour vous faire mes adieux!» 

Et sur ces paroles le baron sortit. 

Milady avait écouté toute cette menaçante tirade le sourire du dédain sur les lèvres, mais la rage dans le cœur. 

On servit le souper; Milady sentit qu'elle avait besoin de forces, elle ne savait pas ce qui pouvait se passer pendant cette nuit qui s'approchait menaçante, car de gros nuages roulaient au ciel, et des éclairs lointains annonçaient un orage. 

L'orage éclata vers les dix heures du soir: Milady sentait une consolation à voir la nature partager le désordre de son cœur; la foudre grondait dans l'air comme la colère dans sa pensée, il lui semblait que la rafale, en passant, échevelait son front comme les arbres dont elle courbait les branches et enlevait les feuilles; elle hurlait comme l'ouragan, et sa voix se perdait dans la grande voix de la nature, qui, elle aussi, semblait gémir et se désespérer. 

Tout à coup elle entendit frapper à une vitre, et, à la lueur d'un éclair, elle vit le visage d'un homme apparaître derrière les barreaux. 

Elle courut à la fenêtre et l'ouvrit. 

«Felton! s'écria-t-elle, je suis sauvée! 

\speak  Oui, dit Felton! mais silence, silence! il me faut le temps de scier vos barreaux. Prenez garde seulement qu'ils ne vous voient par le guichet. 

\speak  Oh! c'est une preuve que le Seigneur est pour nous, Felton, reprit Milady, ils ont fermé le guichet avec une planche. 

\speak  C'est bien, Dieu les a rendus insensés! dit Felton. 

\speak  Mais que faut-il que je fasse? demanda Milady. 

\speak  Rien, rien; refermez la fenêtre seulement. Couchez-vous, ou, du moins, mettez-vous dans votre lit tout habillée; quand j'aurai fini, je frapperai aux carreaux. Mais pourrez-vous me suivre? 

\speak  Oh! oui. 

\speak  Votre blessure? 

\speak  Me fait souffrir, mais ne m'empêche pas de marcher. 

\speak  Tenez-vous donc prête au premier signal.» 

Milady referma la fenêtre, éteignit la lampe, et alla, comme le lui avait recommandé Felton, se blottir dans son lit. Au milieu des plaintes de l'orage, elle entendait le grincement de la lime contre les barreaux, et, à la lueur de chaque éclair, elle apercevait l'ombre de Felton derrière les vitres. 

Elle passa une heure sans respirer, haletante, la sueur sur le front, et le cœur serré par une épouvantable angoisse à chaque mouvement qu'elle entendait dans le corridor. 

Il y a des heures qui durent une année. 

Au bout d'une heure, Felton frappa de nouveau. 

Milady bondit hors de son lit et alla ouvrir. Deux barreaux de moins formaient une ouverture à passer un homme. 

«Êtes-vous prête? demanda Felton. 

\speak  Oui. Faut-il que j'emporte quelque chose? 

\speak  De l'or, si vous en avez. 

\speak  Oui, heureusement on m'a laissé ce que j'en avais. 

\speak  Tant mieux, car j'ai usé tout le mien pour fréter une barque. 

\speak  Prenez», dit Milady en mettant aux mains de Felton un sac plein d'or. 

Felton prit le sac et le jeta au pied du mur. 

«Maintenant, dit-il, voulez-vous venir? 

\speak  Me voici.» 

Milady monta sur un fauteuil et passa tout le haut de son corps par la fenêtre: elle vit le jeune officier suspendu au-dessus de l'abîme par une échelle de corde. 

Pour la première fois, un mouvement de terreur lui rappela qu'elle était femme. 

Le vide l'épouvantait. 

«Je m'en étais douté, dit Felton. 

\speak  Ce n'est rien, ce n'est rien, dit Milady, je descendrai les yeux fermés. 

\speak  Avez-vous confiance en moi? dit Felton. 

\speak  Vous le demandez? 

\speak  Rapprochez vos deux mains; croisez-les, c'est bien.» 

Felton lui lia les deux poignets avec son mouchoir, puis par-dessus le mouchoir, avec une corde. 

«Que faites-vous? demanda Milady avec surprise. 

\speak  Passez vos bras autour de mon cou et ne craignez rien. 

\speak  Mais je vous ferai perdre l'équilibre, et nous nous briserons tous les deux. 

\speak  Soyez tranquille, je suis marin.» 

Il n'y avait pas une seconde à perdre; Milady passa ses deux bras autour du cou de Felton et se laissa glisser hors de la fenêtre. 

Felton se mit à descendre les échelons lentement et un à un. Malgré la pesanteur des deux corps, le souffle de l'ouragan les balançait dans l'air. 

Tout à coup Felton s'arrêta. 

«Qu'y a-t-il? demanda Milady. 

\speak  Silence, dit Felton, j'entends des pas. 

\speak  Nous sommes découverts!» 

Il se fit un silence de quelques instants. 

«Non, dit Felton, ce n'est rien. 

\speak  Mais enfin quel est ce bruit? 

\speak  Celui de la patrouille qui va passer sur le chemin de ronde. 

\speak  Où est le chemin de ronde? 

\speak  Juste au-dessous de nous. 

\speak  Elle va nous découvrir. 

\speak  Non, s'il ne fait pas d'éclairs. 

\speak  Elle heurtera le bas de l'échelle. 

\speak  Heureusement elle est trop courte de six pieds. 

\speak  Les voilà, mon Dieu! 

\speak  Silence!» 

Tous deux restèrent suspendus, immobiles et sans souffle, à vingt pieds du sol; pendant ce temps les soldats passaient au-dessous riant et causant. 

Il y eut pour les fugitifs un moment terrible. 

La patrouille passa; on entendit le bruit des pas qui s'éloignait, et le murmure des voix qui allait s'affaiblissant. 

«Maintenant, dit Felton, nous sommes sauvés.» 

Milady poussa un soupir et s'évanouit. 

Felton continua de descendre. Parvenu au bas de l'échelle, et lorsqu'il ne sentit plus d'appui pour ses pieds, il se cramponna avec ses mains; enfin, arrivé au dernier échelon il se laissa pendre à la force des poignets et toucha la terre. Il se baissa, ramassa le sac d'or et le prit entre ses dents. 

Puis il souleva Milady dans ses bras, et s'éloigna vivement du côté opposé à celui qu'avait pris la patrouille. Bientôt il quitta le chemin de ronde, descendit à travers les rochers, et, arrivé au bord de la mer, fit entendre un coup de sifflet. 

Un signal pareil lui répondit, et, cinq minutes après, il vit apparaître une barque montée par quatre hommes. 

La barque s'approcha aussi près qu'elle put du rivage, mais il n'y avait pas assez de fond pour qu'elle pût toucher le bord; Felton se mit à l'eau jusqu'à la ceinture, ne voulant confier à personne son précieux fardeau. 

Heureusement la tempête commençait à se calmer, et cependant la mer était encore violente; la petite barque bondissait sur les vagues comme une coquille de noix. 

«Au sloop, dit Felton, et nagez vivement.» 

Les quatre hommes se mirent à la rame; mais la mer était trop grosse pour que les avirons eussent grande prise dessus. 

Toutefois on s'éloignait du château; c'était le principal. La nuit était profondément ténébreuse, et il était déjà presque impossible de distinguer le rivage de la barque, à plus forte raison n'eût-on pas pu distinguer la barque du rivage. 

Un point noir se balançait sur la mer. 

C'était le sloop. 

Pendant que la barque s'avançait de son côté de toute la force de ses quatre rameurs, Felton déliait la corde, puis le mouchoir qui liait les mains de Milady. 

Puis, lorsque ses mains furent déliées, il prit de l'eau de la mer et la lui jeta au visage. 

Milady poussa un soupir et ouvrit les yeux. 

«Où suis-je? dit-elle. 

\speak  Sauvée, répondit le jeune officier. 

\speak  Oh! sauvée! sauvée! s'écria-t-elle. Oui, voici le ciel, voici la mer! Cet air que je respire, c'est celui de la liberté. Ah!\dots merci, Felton, merci!» 

Le jeune homme la pressa contre son cœur. 

«Mais qu'ai-je donc aux mains? demanda Milady; il me semble qu'on m'a brisé les poignets dans un étau.» 

En effet, Milady souleva ses bras: elle avait les poignets meurtris. 

«Hélas! dit Felton en regardant ces belles mains et en secouant doucement la tête. 

\speak  Oh! ce n'est rien, ce n'est rien! s'écria Milady: maintenant je me rappelle!» 

Milady chercha des yeux autour d'elle. 

«Il est là», dit Felton en poussant du pied le sac d'or. 

On s'approchait du sloop. Le marin de quart héla la barque, la barque répondit. 

«Quel est ce bâtiment? demanda Milady. 

\speak  Celui que j'ai frété pour vous. 

\speak  Où va-t-il me conduire? 

\speak  Où vous voudrez, pourvu que, moi, vous me jetiez à Portsmouth. 

\speak  Qu'allez-vous faire à Portsmouth? demanda Milady. 

\speak  Accomplir les ordres de Lord de Winter, dit Felton avec un sombre sourire. 

\speak  Quels ordres? demanda Milady. 

\speak  Vous ne comprenez donc pas? dit Felton. 

\speak  Non; expliquez-vous, je vous en prie. 

\speak  Comme il se défiait de moi, il a voulu vous garder lui-même, et m'a envoyé à sa place faire signer à Buckingham l'ordre de votre déportation. 

\speak  Mais s'il se défiait de vous, comment vous a-t-il confié cet ordre? 

\speak  Étais-je censé savoir ce que je portais? 

\speak  C'est juste. Et vous allez à Portsmouth? 

\speak  Je n'ai pas de temps à perdre: c'est demain le 23, et Buckingham part demain avec la flotte. 

\speak  Il part demain, pour où part-il? 

\speak  Pour La Rochelle. 

\speak  Il ne faut pas qu'il parte! s'écria Milady, oubliant sa présence d'esprit accoutumée. 

\speak  Soyez tranquille, répondit Felton, il ne partira pas.» 

Milady tressaillit de joie; elle venait de lire au plus profond du cœur du jeune homme: la mort de Buckingham y était écrite en toutes lettres. 

«Felton\dots, dit-elle, vous êtes grand comme Judas Macchabée! Si vous mourez, je meurs avec vous: voilà tout ce que je puis vous dire. 

\speak  Silence! dit Felton, nous sommes arrivés.» 

En effet, on touchait au sloop. 

Felton monta le premier à l'échelle et donna la main à Milady, tandis que les matelots la soutenaient, car la mer était encore fort agitée. 

Un instant après ils étaient sur le pont. 

«Capitaine, dit Felton, voici la personne dont je vous ai parlé, et qu'il faut conduire saine et sauve en France. 

\speak  Moyennant mille pistoles, dit le capitaine. 

\speak  Je vous en ai donné cinq cents. 

\speak  C'est juste, dit le capitaine. 

\speak  Et voilà les cinq cents autres, reprit Milady, en portant la main au sac d'or. 

\speak  Non, dit le capitaine, je n'ai qu'une parole, et je l'ai donnée à ce jeune homme; les cinq cents autres pistoles ne me sont dues qu'en arrivant à Boulogne. 

\speak  Et nous y arriverons? 

\speak  Sains et saufs, dit le capitaine, aussi vrai que je m'appelle Jack Buttler. 

\speak  Eh bien, dit Milady, si vous tenez votre parole, ce n'est pas cinq cents, mais mille pistoles que je vous donnerai. 

\speak  Hurrah pour vous alors, ma belle dame, cria le capitaine, et puisse Dieu m'envoyer souvent des pratiques comme Votre Seigneurie! 

\speak  En attendant, dit Felton, conduisez-nous dans la petite baie de Chichester, en avant de Portsmouth; vous savez qu'il est convenu que vous nous conduirez là.» 

Le capitaine répondit en commandant la manoeuvre nécessaire, et vers les sept heures du matin le petit bâtiment jetait l'ancre dans la baie désignée. 

Pendant cette traversée, Felton avait tout raconté à Milady: comment, au lieu d'aller à Londres, il avait frété le petit bâtiment, comment il était revenu, comment il avait escaladé la muraille en plaçant dans les interstices des pierres, à mesure qu'il montait, des crampons, pour assurer ses pieds, et comment enfin, arrivé aux barreaux, il avait attaché l'échelle, Milady savait le reste. 

De son côté, Milady essaya d'encourager Felton dans son projet, mais aux premiers mots qui sortirent de sa bouche, elle vit bien que le jeune fanatique avait plutôt besoin d'être modéré que d'être affermi. 

Il fut convenu que Milady attendrait Felton jusqu'à dix heures; si à dix heures il n'était pas de retour, elle partirait. 

Alors, en supposant qu'il fût libre, il la rejoindrait en France, au couvent des Carmélites de Béthune. 