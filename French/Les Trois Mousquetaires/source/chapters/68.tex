%!TeX root=../musketeersfr.tex 

\chapter{Épilogue}

\lettrine{L}{a} Rochelle, privée du secours de la flotte anglaise et de la division promise par Buckingham, se rendit après un siège d'un an. Le 28 octobre 1628, on signa la capitulation. 

\zz
Le roi fit son entrée à Paris le 23 décembre de la même année. On lui fit un triomphe comme s'il revenait de vaincre l'ennemi et non des Français. Il entra par le faubourg Saint-Jacques sous des arcs de verdure. 

D'Artagnan prit possession de son grade. Porthos quitta le service et épousa, dans le courant de l'année suivante, Mme Coquenard, le coffre tant convoité contenait huit cent mille livres. 

Mousqueton eut une livrée magnifique, et de plus la satisfaction, qu'il avait ambitionnée toute sa vie, de monter derrière un carrosse doré. 

Aramis, après un voyage en Lorraine, disparut tout à coup et cessa d'écrire à ses amis. On apprit plus tard, par Mme de Chevreuse, qui le dit à deux ou trois de ses amants, qu'il avait pris l'habit dans un couvent de Nancy. 

Bazin devint frère lai. 

Athos resta mousquetaire sous les ordres de d'Artagnan jusqu'en 1633, époque à laquelle, à la suite d'un voyage qu'il fit en Touraine, il quitta aussi le service sous prétexte qu'il venait de recueillir un petit héritage en Roussillon. 

Grimaud suivit Athos. 

D'Artagnan se battit trois fois avec Rochefort et le blessa trois fois. 

«Je vous tuerai probablement à la quatrième, lui dit-il en lui tendant la main pour le relever. 

\speak  Il vaut donc mieux, pour vous et pour moi, que nous en restions là, répondit le blessé. Corbleu! je suis plus votre ami que vous ne pensez, car dès la première rencontre j'aurais pu, en disant un mot au cardinal, vous faire couper le cou.» 

Ils s'embrassèrent cette fois, mais de bon cœur et sans arrière-pensée. 

Planchet obtint de Rochefort le grade de sergent dans les gardes. 

M. Bonacieux vivait fort tranquille, ignorant parfaitement ce qu'était devenue sa femme et ne s'en inquiétant guère. Un jour, il eut l'imprudence de se rappeler au souvenir du cardinal; le cardinal lui fit répondre qu'il allait pourvoir à ce qu'il ne manquât jamais de rien désormais. 

En effet, le lendemain, M. Bonacieux, étant sorti à sept heures du soir de chez lui pour se rendre au Louvre, ne reparut plus rue des Fossoyeurs; l'avis de ceux qui parurent les mieux informés fut qu'il était nourri et logé dans quelque château royal aux frais de sa généreuse Éminence.