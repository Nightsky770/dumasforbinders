%!TeX root=../musketeersfr.tex 

\chapter{La Comtesse De Winter}

\lettrine{T}{out} le long de la route, le duc se fit mettre au courant par d'Artagnan non pas de tout ce qui s'était passé, mais de ce que d'Artagnan savait. En rapprochant ce qu'il avait entendu sortir de la bouche du jeune homme de ses souvenirs à lui, il put donc se faire une idée assez exacte d'une position de la gravité de laquelle, au reste, la lettre de la reine, si courte et si peu explicite qu'elle fût, lui donnait la mesure. Mais ce qui l'étonnait surtout, c'est que le cardinal, intéressé comme il l'était à ce que le jeune homme ne mît pas le pied en Angleterre, ne fût point parvenu à l'arrêter en route. Ce fut alors, et sur la manifestation de cet étonnement, que d'Artagnan lui raconta les précautions prises, et comment, grâce au dévouement de ses trois amis qu'il avait éparpillés tout sanglants sur la route, il était arrivé à en être quitte pour le coup d'épée qui avait traversé le billet de la reine, et qu'il avait rendu à M. de Wardes en si terrible monnaie. Tout en écoutant ce récit, fait avec la plus grande simplicité, le duc regardait de temps en temps le jeune homme d'un air étonné, comme s'il n'eût pas pu comprendre que tant de prudence, de courage et de dévouement s'alliât avec un visage qui n'indiquait pas encore vingt ans. 

Les chevaux allaient comme le vent, et en quelques minutes ils furent aux portes de Londres. D'Artagnan avait cru qu'en arrivant dans la ville le duc allait ralentir l'allure du sien, mais il n'en fut pas ainsi: il continua sa route à fond de train, s'inquiétant peu de renverser ceux qui étaient sur son chemin. En effet, en traversant la Cité deux ou trois accidents de ce genre arrivèrent; mais Buckingham ne détourna pas même la tête pour regarder ce qu'étaient devenus ceux qu'il avait culbutés. D'Artagnan le suivait au milieu de cris qui ressemblaient fort à des malédictions. 

En entrant dans la cour de l'hôtel, Buckingham sauta à bas de son cheval, et, sans s'inquiéter de ce qu'il deviendrait, il lui jeta la bride sur le cou et s'élança vers le perron. D'Artagnan en fit autant, avec un peu plus d'inquiétude, cependant, pour ces nobles animaux dont il avait pu apprécier le mérite; mais il eut la consolation de voir que trois ou quatre valets s'étaient déjà élancés des cuisines et des écuries, et s'emparaient aussitôt de leurs montures. 

Le duc marchait si rapidement, que d'Artagnan avait peine à le suivre. Il traversa successivement plusieurs salons d'une élégance dont les plus grands seigneurs de France n'avaient pas même l'idée, et il parvint enfin dans une chambre à coucher qui était à la fois un miracle de goût et de richesse. Dans l'alcôve de cette chambre était une porte, prise dans la tapisserie, que le duc ouvrit avec une petite clef d'or qu'il portait suspendue à son cou par une chaîne du même métal. Par discrétion, d'Artagnan était resté en arrière; mais au moment où Buckingham franchissait le seuil de cette porte, il se retourna, et voyant l'hésitation du jeune homme: 

«Venez, lui dit-il, et si vous avez le bonheur d'être admis en la présence de Sa Majesté, dites-lui ce que vous avez vu.» 

Encouragé par cette invitation, d'Artagnan suivit le duc, qui referma la porte derrière lui. 

Tous deux se trouvèrent alors dans une petite chapelle toute tapissée de soie de Perse et brochée d'or, ardemment éclairée par un grand nombre de bougies. Au-dessus d'une espèce d'autel, et au-dessous d'un dais de velours bleu surmonté de plumes blanches et rouges, était un portrait de grandeur naturelle représentant Anne d'Autriche, si parfaitement ressemblant, que d'Artagnan poussa un cri de surprise: on eût cru que la reine allait parler. 

Sur l'autel, et au-dessous du portrait, était le coffret qui renfermait les ferrets de diamants. 

Le duc s'approcha de l'autel, s'agenouilla comme eût pu faire un prêtre devant le Christ; puis il ouvrit le coffret. 

«Tenez, lui dit-il en tirant du coffre un gros noeud de ruban bleu tout étincelant de diamants; tenez, voici ces précieux ferrets avec lesquels j'avais fait le serment d'être enterré. La reine me les avait donnés, la reine me les reprend: sa volonté, comme celle de Dieu, soit faite en toutes choses.» 

Puis il se mit à baiser les uns après les autres ces ferrets dont il fallait se séparer. Tout à coup, il poussa un cri terrible. 

«Qu'y a-t-il? demanda d'Artagnan avec inquiétude, et que vous arrive-t-il, Milord? 

\speak  Il y a que tout est perdu, s'écria Buckingham en devenant pâle comme un trépassé; deux de ces ferrets manquent, il n'y en a plus que dix. 

\speak  Milord les a-t-il perdus, ou croit-il qu'on les lui ait volés? 

\speak  On me les a volés, reprit le duc, et c'est le cardinal qui a fait le coup. Tenez, voyez, les rubans qui les soutenaient ont été coupés avec des ciseaux. 

\speak  Si Milord pouvait se douter qui a commis le vol\dots Peut-être la personne les a-t-elle encore entre les mains. 

\speak  Attendez, attendez! s'écria le duc. La seule fois que j'ai mis ces ferrets, c'était au bal du roi, il y a huit jours, à Windsor. La comtesse de Winter, avec laquelle j'étais brouillé, s'est rapprochée de moi à ce bal. Ce raccommodement, c'était une vengeance de femme jalouse. Depuis ce jour, je ne l'ai pas revue. Cette femme est un agent du cardinal. 

\speak  Mais il en a donc dans le monde entier! s'écria d'Artagnan. 

\speak  Oh! oui, oui, dit Buckingham en serrant les dents de colère; oui, c'est un terrible lutteur. Mais cependant, quand doit avoir lieu ce bal? 

\speak  Lundi prochain. 

\speak  Lundi prochain! cinq jours encore, c'est plus de temps qu'il ne nous en faut. Patrice! s'écria le duc en ouvrant la porte de la chapelle, Patrice!» 

Son valet de chambre de confiance parut. 

«Mon joaillier et mon secrétaire!» 

Le valet de chambre sortit avec une promptitude et un mutisme qui prouvaient l'habitude qu'il avait contractée d'obéir aveuglément et sans réplique. 

Mais, quoique ce fût le joaillier qui eût été appelé le premier, ce fut le secrétaire qui parut d'abord. C'était tout simple, il habitait l'hôtel. Il trouva Buckingham assis devant une table dans sa chambre à coucher, et écrivant quelques ordres de sa propre main. 

«Monsieur Jackson, lui dit-il, vous allez vous rendre de ce pas chez le lord-chancelier, et lui dire que je le charge de l'exécution de ces ordres. Je désire qu'ils soient promulgués à l'instant même. 

\speak  Mais, Monseigneur, si le lord-chancelier m'interroge sur les motifs qui ont pu porter Votre Grâce à une mesure si extraordinaire, que répondrai-je? 

\speak  Que tel a été mon bon plaisir, et que je n'ai de compte à rendre à personne de ma volonté. 

\speak  Sera-ce la réponse qu'il devra transmettre à Sa Majesté, reprit en souriant le secrétaire, si par hasard Sa Majesté avait la curiosité de savoir pourquoi aucun vaisseau ne peut sortir des ports de la Grande-Bretagne? 

\speak  Vous avez raison, monsieur, répondit Buckingham; il dirait en ce cas au roi que j'ai décidé la guerre, et que cette mesure est mon premier acte d'hostilité contre la France.» 

Le secrétaire s'inclina et sortit. 

«Nous voilà tranquilles de ce côté, dit Buckingham en se retournant vers d'Artagnan. Si les ferrets ne sont point déjà partis pour la France, ils n'y arriveront qu'après vous. 

\speak  Comment cela? 

\speak  Je viens de mettre un embargo sur tous les bâtiments qui se trouvent à cette heure dans les ports de Sa Majesté, et, à moins de permission particulière, pas un seul n'osera lever l'ancre.» 

D'Artagnan regarda avec stupéfaction cet homme qui mettait le pouvoir illimité dont il était revêtu par la confiance d'un roi au service de ses amours. Buckingham vit, à l'expression du visage du jeune homme, ce qui se passait dans sa pensée, et il sourit. 

«Oui, dit-il, oui, c'est qu'Anne d'Autriche est ma véritable reine; sur un mot d'elle, je trahirais mon pays, je trahirais mon roi, je trahirais mon Dieu. Elle m'a demandé de ne point envoyer aux protestants de La Rochelle le secours que je leur avais promis, et je l'ai fait. Je manquais à ma parole, mais qu'importe! j'obéissais à son désir; n'ai-je point été grandement payé de mon obéissance, dites? car c'est à cette obéissance que je dois son portrait.» 

D'Artagnan admira à quels fils fragiles et inconnus sont parfois suspendues les destinées d'un peuple et la vie des hommes. 

Il en était au plus profond de ses réflexions, lorsque l'orfèvre entra: c'était un Irlandais des plus habiles dans son art, et qui avouait lui-même qu'il gagnait cent mille livres par an avec le duc de Buckingham. 

«Monsieur O'Reilly, lui dit le duc en le conduisant dans la chapelle, voyez ces ferrets de diamants, et dites-moi ce qu'ils valent la pièce.» 

L'orfèvre jeta un seul coup d'œil sur la façon élégante dont ils étaient montés, calcula l'un dans l'autre la valeur des diamants, et sans hésitation aucune: 

«Quinze cents pistoles la pièce, Milord, répondit-il. 

\speak  Combien faudrait-il de jours pour faire deux ferrets comme ceux-là? Vous voyez qu'il en manque deux. 

\speak  Huit jours, Milord. 

\speak  Je les paierai trois mille pistoles la pièce, il me les faut après-demain. 

\speak  Milord les aura. 

\speak  Vous êtes un homme précieux, monsieur O'Reilly, mais ce n'est pas le tout: ces ferrets ne peuvent être confiés à personne, il faut qu'ils soient faits dans ce palais. 

\speak  Impossible, Milord, il n'y a que moi qui puisse les exécuter pour qu'on ne voie pas la différence entre les nouveaux et les anciens. 

\speak  Aussi, mon cher monsieur O'Reilly, vous êtes mon prisonnier, et vous voudriez sortir à cette heure de mon palais que vous ne le pourriez pas; prenez-en donc votre parti. Nommez-moi ceux de vos garçons dont vous aurez besoin, et désignez-moi les ustensiles qu'ils doivent apporter.» 

L'orfèvre connaissait le duc, il savait que toute observation était inutile, il en prit donc à l'instant même son parti. 

«Il me sera permis de prévenir ma femme? demanda-t-il. 

\speak  Oh! il vous sera même permis de la voir, mon cher monsieur O'Reilly: votre captivité sera douce, soyez tranquille; et comme tout dérangement vaut un dédommagement, voici, en dehors du prix des deux ferrets, un bon de mille pistoles pour vous faire oublier l'ennui que je vous cause.» 

D'Artagnan ne revenait pas de la surprise que lui causait ce ministre, qui remuait à pleines mains les hommes et les millions. 

Quant à l'orfèvre, il écrivit à sa femme en lui envoyant le bon de mille pistoles, et en la chargeant de lui retourner en échange son plus habile apprenti, un assortiment de diamants dont il lui donnait le poids et le titre, et une liste des outils qui lui étaient nécessaires. 

Buckingham conduisit l'orfèvre dans la chambre qui lui était destinée, et qui, au bout d'une demi-heure, fut transformée en atelier. Puis il mit une sentinelle à chaque porte, avec défense de laisser entrer qui que ce fût, à l'exception de son valet de chambre Patrice. Il est inutile d'ajouter qu'il était absolument défendu à l'orfèvre O'Reilly et à son aide de sortir sous quelque prétexte que ce fût. Ce point réglé, le duc revint à d'Artagnan. 

«Maintenant, mon jeune ami, dit-il, l'Angleterre est à nous deux; que voulez-vous, que désirez-vous? 

\speak  Un lit, répondit d'Artagnan; c'est, pour le moment, je l'avoue, la chose dont j'ai le plus besoin.» 

Buckingham donna à d'Artagnan une chambre qui touchait à la sienne. Il voulait garder le jeune homme sous sa main, non pas qu'il se défiât de lui, mais pour avoir quelqu'un à qui parler constamment de la reine. 

Une heure après fut promulguée dans Londres l'ordonnance de ne laisser sortir des ports aucun bâtiment chargé pour la France, pas même le paquebot des lettres. Aux yeux de tous, c'était une déclaration de guerre entre les deux royaumes. 

Le surlendemain, à onze heures, les deux ferrets en diamants étaient achevés, mais si exactement imités, mais si parfaitement pareils, que Buckingham ne put reconnaître les nouveaux des anciens, et que les plus exercés en pareille matière y auraient été trompés comme lui. 

Aussitôt il fit appeler d'Artagnan. 

«Tenez, lui dit-il, voici les ferrets de diamants que vous êtes venu chercher, et soyez mon témoin que tout ce que la puissance humaine pouvait faire, je l'ai fait. 

\speak  Soyez tranquille, Milord: je dirai ce que j'ai vu; mais Votre Grâce me remet les ferrets sans la boîte? 

\speak  La boîte vous embarrasserait. D'ailleurs la boîte m'est d'autant plus précieuse, qu'elle me reste seule. Vous direz que je la garde. 

\speak  Je ferai votre commission mot à mot, Milord. 

\speak  Et maintenant, reprit Buckingham en regardant fixement le jeune homme, comment m'acquitterai-je jamais envers vous?» 

D'Artagnan rougit jusqu'au blanc des yeux. Il vit que le duc cherchait un moyen de lui faire accepter quelque chose, et cette idée que le sang de ses compagnons et le sien lui allait être payé par de l'or anglais lui répugnait étrangement. 

«Entendons-nous, Milord, répondit d'Artagnan, et pesons bien les faits d'avance, afin qu'il n'y ait point de méprise. Je suis au service du roi et de la reine de France, et fais partie de la compagnie des gardes de M. des Essarts, lequel, ainsi que son beau-frère M. de Tréville, est tout particulièrement attaché à Leurs Majestés. J'ai donc tout fait pour la reine et rien pour Votre Grâce. Il y a plus, c'est que peut-être n'eussé-je rien fait de tout cela, s'il ne se fût agi d'être agréable à quelqu'un qui est ma dame à moi, comme la reine est la vôtre. 

\speak  Oui, dit le duc en souriant, et je crois même connaître cette autre personne, c'est\dots 

\speak  Milord, je ne l'ai point nommée, interrompit vivement le jeune homme. 

\speak  C'est juste, dit le duc; c'est donc à cette personne que je dois être reconnaissant de votre dévouement. 

\speak  Vous l'avez dit, Milord, car justement à cette heure qu'il est question de guerre, je vous avoue que je ne vois dans votre Grâce qu'un Anglais, et par conséquent qu'un ennemi que je serais encore plus enchanté de rencontrer sur le champ de bataille que dans le parc de Windsor ou dans les corridors du Louvre; ce qui, au reste, ne m'empêchera pas d'exécuter de point en point ma mission et de me faire tuer, si besoin est, pour l'accomplir; mais, je le répète à Votre Grâce, sans qu'elle ait personnellement pour cela plus à me remercier de ce que je fais pour moi dans cette seconde entrevue, que de ce que j'ai déjà fait pour elle dans la première. 

\speak  Nous disons, nous: “Fier comme un Écossais”, murmura Buckingham. 

\speak  Et nous disons, nous: “Fier comme un Gascon”, répondit d'Artagnan. Les Gascons sont les Écossais de la France.» 

D'Artagnan salua le duc et s'apprêta à partir. 

«Eh bien, vous vous en allez comme cela? Par où? Comment? 

\speak  C'est vrai. 

\speak  Dieu me damne! les Français ne doutent de rien! 

\speak  J'avais oublié que l'Angleterre était une île, et que vous en étiez le roi. 

\speak  Allez au port, demandez le brick \textit{le Sund}, remettez cette lettre au capitaine; il vous conduira à un petit port où certes on ne vous attend pas, et où n'abordent ordinairement que des bâtiments pêcheurs. 

\speak  Ce port s'appelle? 

\speak  Saint-Valery; mais, attendez donc: arrivé là, vous entrerez dans une mauvaise auberge sans nom et sans enseigne, un véritable bouge à matelots; il n'y a pas à vous tromper, il n'y en a qu'une. 

\speak  Après? 

\speak  Vous demanderez l'hôte, et vous lui direz: \textit{Forward}. 

\speak  Ce qui veut dire? 

\speak  En avant: c'est le mot d'ordre. Il vous donnera un cheval tout sellé et vous indiquera le chemin que vous devez suivre; vous trouverez ainsi quatre relais sur votre route. Si vous voulez, à chacun d'eux, donner votre adresse à Paris, les quatre chevaux vous y suivront; vous en connaissez déjà deux, et vous m'avez paru les apprécier en amateur: ce sont ceux que nous montions; rapportez-vous en à moi, les autres ne leur sont point inférieurs. Ces quatre chevaux sont équipés pour la campagne. Si fier que vous soyez, vous ne refuserez pas d'en accepter un et de faire accepter les trois autres à vos compagnons: c'est pour nous faire la guerre, d'ailleurs. La fin excuse les moyens, comme vous dites, vous autres Français, n'est-ce pas? 

\speak  Oui, Milord, j'accepte, dit d'Artagnan; et s'il plaît à Dieu, nous ferons bon usage de vos présents. 

\speak  Maintenant, votre main, jeune homme; peut-être nous rencontrerons-nous bientôt sur le champ de bataille; mais, en attendant, nous nous quitterons bons amis, je l'espère. 

\speak  Oui, Milord, mais avec l'espérance de devenir ennemis bientôt. 

\speak  Soyez tranquille, je vous le promets. 

\speak  Je compte sur votre parole, Milord.» 

D'Artagnan salua le duc et s'avança vivement vers le port. 

En face la Tour de Londres, il trouva le bâtiment désigné, remit sa lettre au capitaine, qui la fit viser par le gouverneur du port, et appareilla aussitôt. 

Cinquante bâtiments étaient en partance et attendaient. 

En passant bord à bord de l'un d'eux, d'Artagnan crut reconnaître la femme de Meung, la même que le gentilhomme inconnu avait appelée «Milady», et que lui, d'Artagnan, avait trouvée si belle; mais grâce au courant du fleuve et au bon vent qui soufflait, son navire allait si vite qu'au bout d'un instant on fut hors de vue. 

Le lendemain, vers neuf heures du matin, on aborda à Saint-Valery. 

D'Artagnan se dirigea à l'instant même vers l'auberge indiquée, et la reconnut aux cris qui s'en échappaient: on parlait de guerre entre l'Angleterre et la France comme de chose prochaine et indubitable, et les matelots joyeux faisaient bombance. 

D'Artagnan fendit la foule, s'avança vers l'hôte, et prononça le mot \textit{Forward}. À l'instant même, l'hôte lui fit signe de le suivre, sortit avec lui par une porte qui donnait dans la cour, le conduisit à l'écurie où l'attendait un cheval tout sellé, et lui demanda s'il avait besoin de quelque autre chose. 

«J'ai besoin de connaître la route que je dois suivre, dit d'Artagnan. 

\speak  Allez d'ici à Blangy, et de Blangy à Neufchâtel. À Neufchâtel, entrez à l'auberge de la \textit{Herse d'Or}, donnez le mot d'ordre à l'hôtelier, et vous trouverez comme ici un cheval tout sellé. 

\speak  Dois-je quelque chose? demanda d'Artagnan. 

\speak  Tout est payé, dit l'hôte, et largement. Allez donc, et que Dieu vous conduise! 

\speak  Amen!» répondit le jeune homme en partant au galop. 

Quatre heures après, il était à Neufchâtel. 

Il suivit strictement les instructions reçues; à Neufchâtel, comme à Saint-Valery, il trouva une monture toute sellée et qui l'attendait; il voulut transporter les pistolets de la selle qu'il venait de quitter à la selle qu'il allait prendre: les fontes étaient garnies de pistolets pareils. 

«Votre adresse à Paris? 

\speak  Hôtel des Gardes, compagnie des Essarts. 

\speak  Bien, répondit celui-ci. 

\speak  Quelle route faut-il prendre? demanda à son tour d'Artagnan. 

\speak  Celle de Rouen; mais vous laisserez la ville à votre droite. Au petit village d'Écouis, vous vous arrêterez, il n'y a qu'une auberge, l'\textit{Écu de France}. Ne la jugez pas d'après son apparence; elle aura dans ses écuries un cheval qui vaudra celui-ci. 

\speak  Même mot d'ordre? 

\speak  Exactement. 

\speak  Adieu, maître! 

\speak  Bon voyage, gentilhomme! avez-vous besoin de quelque chose?» 

D'Artagnan fit signe de la tête que non, et repartit à fond de train. À Écouis, la même scène se répéta: il trouva un hôte aussi prévenant, un cheval frais et reposé; il laissa son adresse comme il l'avait fait, et repartit du même train pour Pontoise. À Pontoise, il changea une dernière fois de monture, et à neuf heures il entrait au grand galop dans la cour de l'hôtel de M. de Tréville. 

Il avait fait près de soixante lieues en douze heures. 

M. de Tréville le reçut comme s'il l'avait vu le matin même; seulement, en lui serrant la main un peu plus vivement que de coutume, il lui annonça que la compagnie de M. des Essarts était de garde au Louvre et qu'il pouvait se rendre à son poste.