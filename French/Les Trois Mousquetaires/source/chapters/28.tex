%!TeX root=../musketeersfr.tex 

\chapter{Retour} 
	
\lettrine{D}{'Artagnan} était resté étourdi de la terrible confidence d'Athos; cependant bien des choses lui paraissaient encore obscures dans cette demi-révélation; d'abord elle avait été faite par un homme tout à fait ivre à un homme qui l'était à moitié, et cependant, malgré ce vague que fait monter au cerveau la fumée de deux ou trois bouteilles de bourgogne, d'Artagnan, en se réveillant le lendemain matin, avait chaque parole d'Athos aussi présente à son esprit que si, à mesure qu'elles étaient tombées de sa bouche, elles s'étaient imprimées dans son esprit. Tout ce doute ne lui donna qu'un plus vif désir d'arriver à une certitude, et il passa chez son ami avec l'intention bien arrêtée de renouer sa conversation de la veille mais il trouva Athos de sens tout à fait rassis, c'est-à-dire le plus fin et le plus impénétrable des hommes. 

Au reste, le mousquetaire, après avoir échangé avec lui une poignée de main, alla le premier au-devant de sa pensée. 

«J'étais bien ivre hier, mon cher d'Artagnan, dit-il, j'ai senti cela ce matin à ma langue, qui était encore fort épaisse, et à mon pouls qui était encore fort agité; je parie que j'ai dit mille extravagances.» 

Et, en disant ces mots, il regarda son ami avec une fixité qui l'embarrassa. 

«Mais non pas, répliqua d'Artagnan, et, si je me le rappelle bien, vous n'avez rien dit que de fort ordinaire. 

\speak  Ah! vous m'étonnez! Je croyais vous avoir raconté une histoire des plus lamentables.» 

Et il regardait le jeune homme comme s'il eût voulu lire au plus profond de son cœur. 

«Ma foi! dit d'Artagnan, il paraît que j'étais encore plus ivre que vous, puisque je ne me souviens de rien.» 

Athos ne se paya point de cette parole, et il reprit: 

«Vous n'êtes pas sans avoir remarqué, mon cher ami, que chacun a son genre d'ivresse, triste ou gaie, moi j'ai l'ivresse triste, et, quand une fois je suis gris, ma manière est de raconter toutes les histoires lugubres que ma sotte nourrice m'a inculquées dans le cerveau. C'est mon défaut; défaut capital, j'en conviens; mais, à cela près, je suis bon buveur.» 

Athos disait cela d'une façon si naturelle, que d'Artagnan fut ébranlé dans sa conviction. 

«Oh! c'est donc cela, en effet, reprit le jeune homme en essayant de ressaisir la vérité, c'est donc cela que je me souviens, comme, au reste, on se souvient d'un rêve, que nous avons parlé de pendus. 

\speak  Ah! vous voyez bien, dit Athos en pâlissant et cependant en essayant de rire, j'en étais sûr, les pendus sont mon cauchemar, à moi. 

\speak  Oui, oui, reprit d'Artagnan, et voilà la mémoire qui me revient; oui, il s'agissait\dots attendez donc\dots il s'agissait d'une femme. 

\speak  Voyez, répondit Athos en devenant presque livide, c'est ma grande histoire de la femme blonde, et quand je raconte celle-là, c'est que je suis ivre mort. 

\speak  Oui, c'est cela, dit d'Artagnan, l'histoire de la femme blonde, grande et belle, aux yeux bleus. 

\speak  Oui, et pendue. 

\speak  Par son mari, qui était un seigneur de votre connaissance, continua d'Artagnan en regardant fixement Athos. 

\speak  Eh bien, voyez cependant comme on compromettrait un homme quand on ne sait plus ce que l'on dit, reprit Athos en haussant les épaules, comme s'il se fût pris lui-même en pitié. Décidément, je ne veux plus me griser, d'Artagnan, c'est une trop mauvaise habitude.» 

D'Artagnan garda le silence. 

Puis Athos, changeant tout à coup de conversation: 

«À propos, dit-il, je vous remercie du cheval que vous m'avez amené. 

\speak  Est-il de votre goût? demanda d'Artagnan. 

\speak  Oui, mais ce n'était pas un cheval de fatigue. 

\speak  Vous vous trompez; j'ai fait avec lui dix lieues en moins d'une heure et demie, et il n'y paraissait pas plus que s'il eût fait le tour de la place Saint-Sulpice. 

\speak  Ah çà, vous allez me donner des regrets. 

\speak  Des regrets? 

\speak  Oui, je m'en suis défait. 

\speak  Comment cela? 

\speak  Voici le fait: ce matin, je me suis réveillé à six heures, vous dormiez comme un sourd, et je ne savais que faire; j'étais encore tout hébété de notre débauche d'hier; je descendis dans la grande salle, et j'avisai un de nos Anglais qui marchandait un cheval à un maquignon, le sien étant mort hier d'un coup de sang. Je m'approchai de lui, et comme je vis qu'il offrait cent pistoles d'un alezan brûlé: «Par Dieu, lui dis-je, mon gentilhomme, moi aussi j'ai un cheval à vendre. 

«--- Et très beau même, dit-il, je l'ai vu hier, le valet de votre ami le tenait en main. 

«--- Trouvez-vous qu'il vaille cent pistoles? 

«--- Oui, et voulez-vous me le donner pour ce prix-là? 

«--- Non, mais je vous le joue. 

«--- Vous me le jouez? 

«--- Oui. 

«--- À quoi? 

«--- Aux dés.» 

«Ce qui fut dit fut fait; et j'ai perdu le cheval. Ah! mais par exemple, continua Athos, j'ai regagné le caparaçon.» 

D'Artagnan fit une mine assez maussade. 

«Cela vous contrarie? dit Athos. 

\speak  Mais oui, je vous l'avoue, reprit d'Artagnan; ce cheval devait servir à nous faire reconnaître un jour de bataille; c'était un gage, un souvenir. Athos, vous avez eu tort. 

\speak  Eh! mon cher ami, mettez-vous à ma place, reprit le mousquetaire; je m'ennuyais à périr, moi, et puis, d'honneur, je n'aime pas les chevaux anglais. Voyons, s'il ne s'agit que d'être reconnu par quelqu'un, eh bien, la selle suffira; elle est assez remarquable. Quant au cheval, nous trouverons quelque excuse pour motiver sa disparition. Que diable! un cheval est mortel; mettons que le mien a eu la morve ou le farcin.» 

D'Artagnan ne se déridait pas. 

«Cela me contrarie, continua Athos, que vous paraissiez tant tenir à ces animaux, car je ne suis pas au bout de mon histoire. 

\speak  Qu'avez-vous donc fait encore? 

\speak  Après avoir perdu mon cheval, neuf contre dix, voyez le coup, l'idée me vint de jouer le vôtre. 

\speak  Oui, mais vous vous en tîntes, j'espère, à l'idée? 

\speak  Non pas, je la mis à exécution à l'instant même. 

\speak  Ah! par exemple! s'écria d'Artagnan inquiet. 

\speak  Je jouai, et je perdis. 

\speak  Mon cheval? 

\speak  Votre cheval; sept contre huit; faute d'un point\dots, vous connaissez le proverbe. 

\speak  Athos, vous n'êtes pas dans votre bon sens, je vous jure! 

\speak  Mon cher, c'était hier, quand je vous contais mes sottes histoires, qu'il fallait me dire cela, et non pas ce matin. Je le perdis donc avec tous les équipages et harnais possibles. 

\speak  Mais c'est affreux! 

\speak  Attendez donc, vous n'y êtes point, je ferais un joueur excellent, si je ne m'entêtais pas; mais je m'entête, c'est comme quand je bois; je m'entêtai donc\dots 

\speak  Mais que pûtes-vous jouer, il ne vous restait plus rien? 

\speak  Si fait, si fait, mon ami; il nous restait ce diamant qui brille à votre doigt, et que j'avais remarqué hier. 

\speak  Ce diamant! s'écria d'Artagnan, en portant vivement la main à sa bague. 

\speak  Et comme je suis connaisseur, en ayant eu quelques-uns pour mon propre compte, je l'avais estimé mille pistoles. 

\speak  J'espère, dit sérieusement d'Artagnan à demi mort de frayeur, que vous n'avez aucunement fait mention de mon diamant? 

\speak  Au contraire, cher ami; vous comprenez, ce diamant devenait notre seule ressource; avec lui, je pouvais regagner nos harnais et nos chevaux, et, de plus, l'argent pour faire la route. 

\speak  Athos, vous me faites frémir! s'écria d'Artagnan. 

\speak  Je parlai donc de votre diamant à mon partenaire, lequel l'avait aussi remarqué. Que diable aussi, mon cher, vous portez à votre doigt une étoile du ciel, et vous ne voulez pas qu'on y fasse attention! Impossible! 

\speak  Achevez, mon cher; achevez! dit d'Artagnan, car, d'honneur! avec votre sang-froid, vous me faites mourir! 

\speak  Nous divisâmes donc ce diamant en dix parties de cent pistoles chacune. 

\speak  Ah! vous voulez rire et m'éprouver? dit d'Artagnan que la colère commençait à prendre aux cheveux comme Minerve prend Achille, dans \textit{l'Iliade}. 

\speak  Non, je ne plaisante pas, mordieu! j'aurais bien voulu vous y voir, vous! il y avait quinze jours que je n'avais envisagé face humaine et que j'étais là à m'abrutir en m'abouchant avec des bouteilles. 

\speak  Ce n'est point une raison pour jouer mon diamant, cela? répondit d'Artagnan en serrant sa main avec une crispation nerveuse. 

\speak  Écoutez donc la fin; dix parties de cent pistoles chacune en dix coups sans revanche. En treize coups je perdis tout. En treize coups! Le nombre 13 m'a toujours été fatal, c'était le 13 du mois de juillet que\dots 

\speak  Ventrebleu! s'écria d'Artagnan en se levant de table, l'histoire du jour lui faisant oublier celle de la veille. 

\speak  Patience, dit Athos, j'avais un plan. L'Anglais était un original, je l'avais vu le matin causer avec Grimaud, et Grimaud m'avait averti qu'il lui avait fait des propositions pour entrer à son service. Je lui joue Grimaud, le silencieux Grimaud, divisé en dix portions. 

\speak  Ah! pour le coup! dit d'Artagnan éclatant de rire malgré lui. 

\speak  Grimaud lui-même, entendez-vous cela! et avec les dix parts de Grimaud, qui ne vaut pas en tout un ducaton, je regagne le diamant. Dites maintenant que la persistance n'est pas une vertu. 

\speak  Ma foi, c'est très drôle! s'écria d'Artagnan consolé et se tenant les côtes de rire. 

\speak  Vous comprenez que, me sentant en veine, je me remis aussitôt à jouer sur le diamant. 

\speak  Ah! diable, dit d'Artagnan assombri de nouveau. 

\speak  J'ai regagné vos harnais, puis votre cheval, puis mes harnais, puis mon cheval, puis reperdu. Bref, j'ai rattrapé votre harnais, puis le mien. Voilà où nous en sommes. C'est un coup superbe; aussi je m'en suis tenu là.» 

D'Artagnan respira comme si on lui eût enlevé l'hôtellerie de dessus la poitrine. 

«Enfin, le diamant me reste? dit-il timidement. 

\speak  Intact! cher ami; plus les harnais de votre Bucéphale et du mien. 

\speak  Mais que ferons-nous de nos harnais sans chevaux? 

\speak  J'ai une idée sur eux. 

\speak  Athos, vous me faites frémir. 

\speak  Écoutez, vous n'avez pas joué depuis longtemps, vous, d'Artagnan? 

\speak  Et je n'ai point l'envie de jouer. 

\speak  Ne jurons de rien. Vous n'avez pas joué depuis longtemps, disais-je, vous devez donc avoir la main bonne. 

\speak  Eh bien, après? 

\speak  Eh bien, l'Anglais et son compagnon sont encore là. J'ai remarqué qu'ils regrettaient beaucoup les harnais. Vous, vous paraissez tenir à votre cheval. A votre place, je jouerais vos harnais contre votre cheval. 

\speak  Mais il ne voudra pas un seul harnais. 

\speak  Jouez les deux, pardieu! je ne suis point un égoïste comme vous, moi. 

\speak  Vous feriez cela? dit d'Artagnan indécis, tant la confiance d'Athos commençait à le gagner à son insu. 

\speak  Parole d'honneur, en un seul coup. 

\speak  Mais c'est qu'ayant perdu les chevaux, je tenais énormément à conserver les harnais. 

\speak  Jouez votre diamant, alors. 

\speak  Oh! ceci, c'est autre chose; jamais, jamais. 

\speak  Diable! dit Athos, je vous proposerais bien de jouer Planchet; mais comme cela a déjà été fait, l'Anglais ne voudrait peut-être plus. 

\speak  Décidément, mon cher Athos, dit d'Artagnan, j'aime mieux ne rien risquer. 

\speak  C'est dommage, dit froidement Athos, l'Anglais est cousu de pistoles. Eh! mon Dieu, essayez un coup, un coup est bientôt joué. 

\speak  Et si je perds? 

\speak  Vous gagnerez. 

\speak  Mais si je perds? 

\speak  Eh bien, vous donnerez les harnais. 

\speak  Va pour un coup», dit d'Artagnan. 

Athos se mit en quête de l'Anglais et le trouva dans l'écurie, où il examinait les harnais d'un œil de convoitise. L'occasion était bonne. Il fit ses conditions: les deux harnais contre un cheval ou cent pistoles, à choisir. L'Anglais calcula vite: les deux harnais valaient trois cents pistoles à eux deux; il topa. 

D'Artagnan jeta les dés en tremblant et amena le nombre trois; sa pâleur effraya Athos, qui se contenta de dire: 

«Voilà un triste coup, compagnon; vous aurez les chevaux tout harnachés, monsieur.» 

L'Anglais, triomphant, ne se donna même la peine de rouler les dés, il les jeta sur la table sans regarder, tant il était sûr de la victoire; d'Artagnan s'était détourné pour cacher sa mauvaise humeur. 

«Tiens, tiens, tiens, dit Athos avec sa voix tranquille, ce coup de dés est extraordinaire, et je ne l'ai vu que quatre fois dans ma vie: deux as!» 

L'Anglais regarda et fut saisi d'étonnement, d'Artagnan regarda et fut saisi de plaisir. 

«Oui, continua Athos, quatre fois seulement: une fois chez M. de Créquy; une autre fois chez moi, à la campagne, dans mon château de\dots quand j'avais un château; une troisième fois chez M. de Tréville, où il nous surprit tous; enfin une quatrième fois au cabaret, où il échut à moi et où je perdis sur lui cent louis et un souper. 

\speak  Alors, monsieur reprend son cheval, dit l'Anglais. 

\speak  Certes, dit d'Artagnan. 

\speak  Alors il n'y a pas de revanche? 

\speak  Nos conditions disaient: pas de revanche, vous vous le rappelez? 

\speak  C'est vrai; le cheval va être rendu à votre valet, monsieur. 

\speak  Un moment, dit Athos; avec votre permission, monsieur, je demande à dire un mot à mon ami. 

\speak  Dites.» 

Athos tira d'Artagnan à part. 

«Eh bien, lui dit d'Artagnan, que me veux-tu encore, tentateur, tu veux que je joue, n'est-ce pas? 

\speak  Non, je veux que vous réfléchissiez. 

\speak  À quoi? 

\speak  Vous allez reprendre le cheval, n'est-ce pas? 

\speak  Sans doute. 

\speak  Vous avez tort, je prendrais les cent pistoles; vous savez que vous avez joué les harnais contre le cheval ou cent pistoles, à votre choix. 

\speak  Oui. 

\speak  Je prendrais les cent pistoles. 

\speak  Eh bien, moi, je prends le cheval. 

\speak  Et vous avez tort, je vous le répète; que ferons-nous d'un cheval pour nous deux, je ne puis pas monter en croupe; nous aurions l'air des deux fils Aymon qui ont perdu leurs frères; vous ne pouvez pas m'humilier en chevauchant près de moi, en chevauchant sur ce magnifique destrier. Moi, sans balancer un seul instant, je prendrais les cent pistoles, nous avons besoin d'argent pour revenir à Paris. 

\speak  Je tiens à ce cheval, Athos. 

\speak  Et vous avez tort, mon ami; un cheval prend un écart, un cheval bute et se couronne, un cheval mange dans un râtelier où a mangé un cheval morveux: voilà un cheval ou plutôt cent pistoles perdues; il faut que le maître nourrisse son cheval, tandis qu'au contraire cent pistoles nourrissent leur maître. 

\speak  Mais comment reviendrons-nous? 

\speak  Sur les chevaux de nos laquais, pardieu! on verra toujours bien à l'air de nos figures que nous sommes gens de condition. 

\speak  La belle mine que nous aurons sur des bidets, tandis qu'Aramis et Porthos caracoleront sur leurs chevaux! 

\speak  Aramis! Porthos! s'écria Athos, et il se mit à rire. 

\speak  Quoi? demanda d'Artagnan, qui ne comprenait rien à l'hilarité de son ami. 

\speak  Bien, bien, continuons, dit Athos. 

\speak  Ainsi, votre avis\dots? 

\speak  Est de prendre les cent pistoles, d'Artagnan; avec les cent pistoles nous allons festiner jusqu'à la fin du mois; nous avons essuyé des fatigues, voyez-vous, et il sera bon de nous reposer un peu. 

\speak  Me reposer! oh! non, Athos, aussitôt à Paris je me mets à la recherche de cette pauvre femme. 

\speak  Eh bien, croyez-vous que votre cheval vous sera aussi utile pour cela que de bons louis d'or? Prenez les cent pistoles, mon ami, prenez les cent pistoles.» 

D'Artagnan n'avait besoin que d'une raison pour se rendre. Celle-là lui parut excellente. D'ailleurs, en résistant plus longtemps, il craignait de paraître égoïste aux yeux d'Athos; il acquiesça donc et choisit les cent pistoles, que l'Anglais lui compta sur-le-champ. 

Puis l'on ne songea plus qu'à partir. La paix signée avec l'aubergiste, outre le vieux cheval d'Athos, coûta six pistoles; d'Artagnan et Athos prirent les chevaux de Planchet et de Grimaud, les deux valets se mirent en route à pied, portant les selles sur leurs têtes. 

Si mal montés que fussent les deux amis, ils prirent bientôt les devants sur leurs valets et arrivèrent à Crèvecœur. De loin ils aperçurent Aramis mélancoliquement appuyé sur sa fenêtre et regardant, comme \textit{ma soeur Anne}, poudroyer l'horizon. 

«Holà, eh! Aramis! que diable faites-vous donc là? crièrent les deux amis. 

\speak  Ah! c'est vous, d'Artagnan, c'est vous Athos, dit le jeune homme; je songeais avec quelle rapidité s'en vont les biens de ce monde, et mon cheval anglais, qui s'éloignait et qui vient de disparaître au milieu d'un tourbillon de poussière, m'était une vivante image de la fragilité des choses de la terre. La vie elle-même peut se résoudre en trois mots: \textit{Erat, est, fuit}. 

\speak  Cela veut dire au fond? demanda d'Artagnan, qui commençait à se douter de la vérité. 

\speak  Cela veut dire que je viens de faire un marché de dupe: soixante louis, un cheval qui, à la manière dont il file, peut faire au trot cinq lieues à l'heure.» 

D'Artagnan et Athos éclatèrent de rire. 

«Mon cher d'Artagnan, dit Aramis, ne m'en veuillez pas trop, je vous prie: nécessité n'a pas de loi; d'ailleurs je suis le premier puni, puisque cet infâme maquignon m'a volé cinquante louis au moins. Ah! vous êtes bons ménagers, vous autres! vous venez sur les chevaux de vos laquais et vous faites mener vos chevaux de luxe en main, doucement et à petites journées.» 

Au même instant un fourgon, qui depuis quelques instants pointait sur la route d'Amiens, s'arrêta, et l'on vit sortir Grimaud et Planchet leurs selles sur la tête. Le fourgon retournait à vide vers Paris, et les deux laquais s'étaient engagés, moyennant leur transport, à désaltérer le voiturier tout le long de la route. 

«Qu'est-ce que cela? dit Aramis en voyant ce qui se passait; rien que les selles? 

\speak  Comprenez-vous maintenant? dit Athos. 

\speak  Mes amis, c'est exactement comme moi. J'ai conservé le harnais, par instinct. Holà, Bazin! portez mon harnais neuf auprès de celui de ces messieurs. 

\speak  Et qu'avez-vous fait de vos curés? demanda d'Artagnan. 

\speak  Mon cher, je les ai invités à dîner le lendemain, dit Aramis: il y a ici du vin exquis, cela soit dit en passant; je les ai grisés de mon mieux; alors le curé m'a défendu de quitter la casaque, et le jésuite m'a prié de le faire recevoir mousquetaire. 

\speak  Sans thèse! cria d'Artagnan, sans thèse! je demande la suppression de la thèse, moi! 

\speak  Depuis lors, continua Aramis, je vis agréablement. J'ai commencé un poème en vers d'une syllabe; c'est assez difficile, mais le mérite en toutes choses est dans la difficulté. La matière est galante, je vous lirai le premier chant, il a quatre cents vers et dure une minute. 

\speak  Ma foi, mon cher Aramis, dit d'Artagnan, qui détestait presque autant les vers que le latin, ajoutez au mérite de la difficulté celui de la brièveté, et vous êtes sûr au moins que votre poème aura deux mérites. 

\speak  Puis, continua Aramis, il respire des passions honnêtes, vous verrez. Ah çà, mes amis, nous retournons donc à Paris? Bravo, je suis prêt; nous allons donc revoir ce bon Porthos, tant mieux. Vous ne croyez pas qu'il me manquait, ce grand niais-là? Ce n'est pas lui qui aurait vendu son cheval, fût-ce contre un royaume. Je voudrais déjà le voir sur sa bête et sur sa selle. Il aura, j'en suis sûr, l'air du grand mogol.» 

On fit une halte d'une heure pour faire souffler les chevaux; Aramis solda son compte, plaça Bazin dans le fourgon avec ses camarades, et l'on se mit en route pour aller retrouver Porthos. 

On le trouva debout, moins pâle que ne l'avait vu d'Artagnan à sa première visite, et assis à une table où, quoiqu'il fût seul, figurait un dîner de quatre personnes; ce dîner se composait de viandes galamment troussées, de vins choisis et de fruits superbes. 

«Ah! pardieu! dit-il en se levant, vous arrivez à merveille, messieurs, j'en étais justement au potage, et vous allez dîner avec moi. 

\speak  Oh! oh! fit d'Artagnan, ce n'est pas Mousqueton qui a pris au lasso de pareilles bouteilles, puis voilà un fricandeau piqué et un filet de boeuf\dots 

\speak  Je me refais, dit Porthos, je me refais, rien n'affaiblit comme ces diables de foulures; avez-vous eu des foulures, Athos? 

\speak  Jamais; seulement je me rappelle que dans notre échauffourée de la rue Férou je reçus un coup d'épée qui, au bout de quinze ou dix-huit jours, m'avait produit exactement le même effet. 

\speak  Mais ce dîner n'était pas pour vous seul, mon cher Porthos? dit Aramis. 

\speak  Non, dit Porthos; j'attendais quelques gentilshommes du voisinage qui viennent de me faire dire qu'ils ne viendraient pas; vous les remplacerez et je ne perdrai pas au change. Holà, Mousqueton! des sièges, et que l'on double les bouteilles! 

\speak  Savez-vous ce que nous mangeons ici? dit Athos au bout de dix minutes. 

\speak  Pardieu! répondit d'Artagnan, moi je mange du veau piqué aux cardons et à la moelle. 

\speak  Et moi des filets d'agneau, dit Porthos. 

\speak  Et moi un blanc de volaille, dit Aramis. 

\speak  Vous vous trompez tous, messieurs, répondit Athos, vous mangez du cheval. 

\speak  Allons donc! dit d'Artagnan. 

\speak  Du cheval!» fit Aramis avec une grimace de dégoût. 

Porthos seul ne répondit pas. 

«Oui, du cheval; n'est-ce pas, Porthos, que nous mangeons du cheval? Peut-être même les caparaçons avec! 

\speak  Non, messieurs, j'ai gardé le harnais, dit Porthos. 

\speak  Ma foi, nous nous valons tous, dit Aramis: on dirait que nous nous sommes donné le mot. 

\speak  Que voulez-vous, dit Porthos, ce cheval faisait honte à mes visiteurs, et je n'ai pas voulu les humilier! 

\speak  Puis, votre duchesse est toujours aux eaux, n'est-ce pas? reprit d'Artagnan. 

\speak  Toujours, répondit Porthos. Or, ma foi, le gouverneur de la province, un des gentilshommes que j'attendais aujourd'hui à dîner, m'a paru le désirer si fort que je le lui ai donné. 

\speak  Donné! s'écria d'Artagnan. 

\speak  Oh! mon Dieu! oui, donné! c'est le mot, dit Porthos; car il valait certainement cent cinquante louis, et le ladre n'a voulu me le payer que quatre-vingts. 

\speak  Sans la selle? dit Aramis. 

\speak  Oui, sans la selle. 

\speak  Vous remarquerez, messieurs, dit Athos, que c'est encore Porthos qui a fait le meilleur marché de nous tous.» 

Ce fut alors un hourra de rires dont le pauvre Porthos fut tout saisi; mais on lui expliqua bientôt la raison de cette hilarité, qu'il partagea bruyamment selon sa coutume. 

«De sorte que nous sommes tous en fonds? dit d'Artagnan. 

\speak  Mais pas pour mon compte, dit Athos; j'ai trouvé le vin d'Espagne d'Aramis si bon, que j'en ai fait charger une soixantaine de bouteilles dans le fourgon des laquais: ce qui m'a fort désargenté. 

\speak  Et moi, dit Aramis, imaginez donc que j'avais donné jusqu'à mon dernier sou à l'église de Montdidier et aux jésuites d'Amiens; que j'avais pris en outre des engagements qu'il m'a fallu tenir, des messes commandées pour moi et pour vous, messieurs, que l'on dira, messieurs, et dont je ne doute pas que nous ne nous trouvions à merveille. 

\speak  Et moi, dit Porthos, ma foulure, croyez-vous qu'elle ne m'a rien coûté? sans compter la blessure de Mousqueton, pour laquelle j'ai été obligé de faire venir le chirurgien deux fois par jour, lequel m'a fait payer ses visites double sous prétexte que cet imbécile de Mousqueton avait été se faire donner une balle dans un endroit qu'on ne montre ordinairement qu'aux apothicaires; aussi je lui ai bien recommandé de ne plus se faire blesser là. 

\speak  Allons, allons, dit Athos, en échangeant un sourire avec d'Artagnan et Aramis, je vois que vous vous êtes conduit grandement à l'égard du pauvre garçon: c'est d'un bon maître. 

\speak  Bref, continua Porthos, ma dépense payée, il me restera bien une trentaine d'écus. 

\speak  Et à moi une dizaine de pistoles, dit Aramis. 

\speak  Allons, allons, dit Athos, il paraît que nous sommes les Crésus de la société. Combien vous reste-t-il sur vos cent pistoles, d'Artagnan? 

\speak  Sur mes cent pistoles? D'abord, je vous en ai donné cinquante. 

\speak  Vous croyez? 

\speak  Pardieu! --- Ah! c'est vrai, je me rappelle. 

\speak  Puis, j'en ai payé six à l'hôte. 

\speak  Quel animal que cet hôte! pourquoi lui avez-vous donné six pistoles? 

\speak  C'est vous qui m'avez dit de les lui donner. 

\speak  C'est vrai que je suis trop bon. Bref, en reliquat? 

\speak  Vingt-cinq pistoles, dit d'Artagnan. 

\speak  Et moi, dit Athos en tirant quelque menue monnaie de sa poche, moi\dots 

\speak  Vous, rien. 

\speak  Ma foi, ou si peu de chose, que ce n'est pas la peine de rapporter à la masse. 

\speak  Maintenant, calculons combien nous possédons en tout: Porthos? 

\speak  Trente écus. 

\speak  Aramis? 

\speak  Dix pistoles. 

\speak  Et vous, d'Artagnan? 

\speak  Vingt-cinq. 

\speak  Cela fait en tout? dit Athos. 

\speak  Quatre cent soixante-quinze livres! dit d'Artagnan, qui comptait comme Archimède. 

\speak  Arrivés à Paris, nous en aurons bien encore quatre cents, dit Porthos, plus les harnais. 

\speak  Mais nos chevaux d'escadron? dit Aramis. 

\speak  Eh bien, des quatre chevaux des laquais nous en ferons deux de maître que nous tirerons au sort; avec les quatre cents livres, on en fera un demi pour un des démontés, puis nous donnerons les grattures de nos poches à d'Artagnan, qui a la main bonne, et qui ira les jouer dans le premier tripot venu, voilà. 

\speak  Dînons donc, dit Porthos, cela refroidit.» 

Les quatre amis, plus tranquilles désormais sur leur avenir, firent honneur au repas, dont les restes furent abandonnés à MM. Mousqueton, Bazin, Planchet et Grimaud. 

En arrivant à Paris, d'Artagnan trouva une lettre de M. de Tréville qui le prévenait que, sur sa demande, le roi venait de lui accorder la faveur d'entrer dans les mousquetaires. 

Comme c'était tout ce que d'Artagnan ambitionnait au monde, à part bien entendu le désir de retrouver Mme Bonacieux, il courut tout joyeux chez ses camarades, qu'il venait de quitter il y avait une demi-heure, et qu'il trouva fort tristes et fort préoccupés. Ils étaient réunis en conseil chez Athos: ce qui indiquait toujours des circonstances d'une certaine gravité. 

M. de Tréville venait de les faire prévenir que l'intention bien arrêtée de Sa Majesté étant d'ouvrir la campagne le 1er mai, ils eussent à préparer incontinent leurs équipages. 

Les quatre philosophes se regardèrent tout ébahis: M. de Tréville ne plaisantait pas sous le rapport de la discipline. 

«Et à combien estimez-vous ces équipages? dit d'Artagnan. 

\speak  Oh! il n'y a pas à dire, reprit Aramis, nous venons de faire nos comptes avec une lésinerie de Spartiates, et il nous faut à chacun quinze cents livres. 

\speak  Quatre fois quinze font soixante, soit six mille livres, dit Athos. 

\speak  Moi, dit d'Artagnan, il me semble qu'avec mille livres chacun, il est vrai que je ne parle pas en Spartiate, mais en procureur\dots» 

Ce mot de procureur réveilla Porthos. 

«Tiens, j'ai une idée! dit-il. 

\speak  C'est déjà quelque chose: moi, je n'en ai pas même l'ombre, fit froidement Athos, mais quant à d'Artagnan, messieurs, le bonheur d'être désormais des nôtres l'a rendu fou; mille livres! je déclare que pour moi seul il m'en faut deux mille. 

\speak  Quatre fois deux font huit, dit alors Aramis: c'est donc huit mille livres qu'il nous faut pour nos équipages, sur lesquels équipages, il est vrai, nous avons déjà les selles. 

\speak  Plus, dit Athos, en attendant que d'Artagnan qui allait remercier M. de Tréville eût fermé la porte, plus ce beau diamant qui brille au doigt de notre ami. Que diable! d'Artagnan est trop bon camarade pour laisser des frères dans l'embarras, quand il porte à son médius la rançon d'un roi.»