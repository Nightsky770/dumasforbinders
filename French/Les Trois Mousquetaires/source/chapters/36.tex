%!TeX root=../musketeersfr.tex 

\chapter{Rêve De Vengeance}

\lettrine{L}{e} soir Milady donna l'ordre d'introduire M. d'Artagnan aussitôt qu'il viendrait, selon son habitude. Mais il ne vint pas. 

\zz
Le lendemain Ketty vint voir de nouveau le jeune homme et lui raconta tout ce qui s'était passé la veille: d'Artagnan sourit; cette jalouse colère de Milady, c'était sa vengeance. 

Le soir Milady fut plus impatiente encore que la veille, elle renouvela l'ordre relatif au Gascon; mais comme la veille elle l'attendit inutilement. 

Le lendemain Ketty se présenta chez d'Artagnan, non plus joyeuse et alerte comme les deux jours précédents, mais au contraire triste à mourir. 

D'Artagnan demanda à la pauvre fille ce qu'elle avait; mais celle-ci, pour toute réponse, tira une lettre de sa poche et la lui remit. 

Cette lettre était de l'écriture de Milady: seulement cette fois elle était bien à l'adresse de d'Artagnan et non à celle de M. de Wardes. 

Il l'ouvrit et lut ce qui suit: 

\begin{mail}{}{Cher monsieur d'Artagnan,}
C'est mal de négliger ainsi ses amis, surtout au moment où l'on va les quitter pour si longtemps. Mon beau-frère et moi nous avons attendu hier et avant-hier inutilement. En sera-t-il de même ce soir?

\closeletter[Votre bien reconnaissante,]{Lady Clarick}
\end{mail}

«C'est tout simple, dit d'Artagnan, et je m'attendais à cette lettre. Mon crédit hausse de la baisse du comte de Wardes. 

\speak  Est-ce que vous irez? demanda Ketty. 

\speak  Écoute, ma chère enfant, dit le Gascon, qui cherchait à s'excuser à ses propres yeux de manquer à la promesse qu'il avait faite à Athos, tu comprends qu'il serait impolitique de ne pas se rendre à une invitation si positive. Milady, en ne me voyant pas revenir, ne comprendrait rien à l'interruption de mes visites, elle pourrait se douter de quelque chose, et qui peut dire jusqu'où irait la vengeance d'une femme de cette trempe? 

\speak  Oh! mon Dieu! dit Ketty, vous savez présenter les choses de façon que vous avez toujours raison. Mais vous allez encore lui faire la cour; et si cette fois vous alliez lui plaire sous votre véritable nom et votre vrai visage, ce serait bien pis que la première fois!» 

L'instinct faisait deviner à la pauvre fille une partie de ce qui allait arriver. 

D'Artagnan la rassura du mieux qu'il put et lui promit de rester insensible aux séductions de Milady. 

Il lui fit répondre qu'il était on ne peut plus reconnaissant de ses bontés et qu'il se rendrait à ses ordres; mais il n'osa lui écrire de peur de ne pouvoir, à des yeux aussi exercés que ceux de Milady, déguiser suffisamment son écriture. 

À neuf heures sonnant, d'Artagnan était place Royale. Il était évident que les domestiques qui attendaient dans l'antichambre étaient prévenus, car aussitôt que d'Artagnan parut, avant même qu'il eût demandé si Milady était visible, un d'eux courut l'annoncer. 

«Faites entrer», dit Milady d'une voix brève, mais si perçante que d'Artagnan l'entendit de l'antichambre. 

On l'introduisit. 

«Je n'y suis pour personne, dit Milady; entendez-vous, pour personne.» 

Le laquais sortit. 

D'Artagnan jeta un regard curieux sur Milady: elle était pâle et avait les yeux fatigués, soit par les larmes, soit par l'insomnie. On avait avec intention diminué le nombre habituel des lumières, et cependant la jeune femme ne pouvait arriver à cacher les traces de la fièvre qui l'avait dévorée depuis deux jours. 

D'Artagnan s'approcha d'elle avec sa galanterie ordinaire; elle fit alors un effort suprême pour le recevoir, mais jamais physionomie plus bouleversée ne démentit sourire plus aimable. 

Aux questions que d'Artagnan lui fit sur sa santé: 

«Mauvaise, répondit-elle, très mauvaise. 

\speak  Mais alors, dit d'Artagnan, je suis indiscret, vous avez besoin de repos sans doute et je vais me retirer. 

\speak  Non pas, dit Milady; au contraire, restez, monsieur d'Artagnan, votre aimable compagnie me distraira.» 

«Oh! oh! pensa d'Artagnan, elle n'a jamais été si charmante, défions-nous.» 

Milady prit l'air le plus affectueux qu'elle put prendre, et donna tout l'éclat possible à sa conversation. En même temps cette fièvre qui l'avait abandonnée un instant revenait rendre l'éclat à ses yeux, le coloris à ses joues, le carmin à ses lèvres. D'Artagnan retrouva la Circé qui l'avait déjà enveloppé de ses enchantements. Son amour, qu'il croyait éteint et qui n'était qu'assoupi, se réveilla dans son cœur. Milady souriait et d'Artagnan sentait qu'il se damnerait pour ce sourire. 

Il y eut un moment où il sentit quelque chose comme un remords de ce qu'il avait fait contre elle. 

Peu à peu Milady devint plus communicative. Elle demanda à d'Artagnan s'il avait une maîtresse. 

«Hélas! dit d'Artagnan de l'air le plus sentimental qu'il put prendre, pouvez-vous être assez cruelle pour me faire une pareille question, à moi qui, depuis que je vous ai vue, ne respire et ne soupire que par vous et pour vous!» 

Milady sourit d'un étrange sourire. 

«Ainsi vous m'aimez? dit-elle. 

\speak  Ai-je besoin de vous le dire, et ne vous en êtes-vous point aperçue? 

\speak  Si fait; mais, vous le savez, plus les cœurs sont fiers, plus ils sont difficiles à prendre. 

\speak  Oh! les difficultés ne m'effraient pas, dit d'Artagnan; il n'y a que les impossibilités qui m'épouvantent. 

\speak  Rien n'est impossible, dit Milady, à un véritable amour. 

\speak  Rien, madame? 

\speak  Rien», reprit Milady. 

«Diable! reprit d'Artagnan à part lui, la note est changée. Deviendrait-elle amoureuse de moi, par hasard, la capricieuse, et serait-elle disposée à me donner à moi-même quelque autre saphir pareil à celui qu'elle m'a donné me prenant pour de Wardes?» 

D'Artagnan rapprocha vivement son siège de celui de Milady. 

«Voyons, dit-elle, que feriez-vous bien pour prouver cet amour dont vous parlez? 

\speak  Tout ce qu'on exigerait de moi. Qu'on ordonne, et je suis prêt. 

\speak  À tout? 

\speak  À tout! s'écria d'Artagnan qui savait d'avance qu'il n'avait pas grand-chose à risquer en s'engageant ainsi. 

\speak  Eh bien, causons un peu, dit à son tour Milady en rapprochant son fauteuil de la chaise de d'Artagnan. 

\speak  Je vous écoute, madame», dit celui-ci. 

Milady resta un instant soucieuse et comme indécise puis paraissant prendre une résolution: 

«J'ai un ennemi, dit-elle. 

\speak  Vous, madame! s'écria d'Artagnan jouant la surprise, est-ce possible, mon Dieu? belle et bonne comme vous l'êtes! 

\speak  Un ennemi mortel. 

\speak  En vérité? 

\speak  Un ennemi qui m'a insultée si cruellement que c'est entre lui et moi une guerre à mort. Puis-je compter sur vous comme auxiliaire?» 

D'Artagnan comprit sur-le-champ où la vindicative créature en voulait venir. 

«Vous le pouvez, madame, dit-il avec emphase, mon bras et ma vie vous appartiennent comme mon amour. 

\speak  Alors, dit Milady, puisque vous êtes aussi généreux qu'amoureux\dots» 

Elle s'arrêta. 

«Eh bien? demanda d'Artagnan. 

\speak  Eh bien, reprit Milady après un moment de silence, cessez dès aujourd'hui de parler d'impossibilités. 

\speak  Ne m'accablez pas de mon bonheur», s'écria d'Artagnan en se précipitant à genoux et en couvrant de baisers les mains qu'on lui abandonnait. 

\speak  Venge-moi de cet infâme de Wardes, murmura Milady entre ses dents, et je saurai bien me débarrasser de toi ensuite, double sot, lame d'épée vivante! 

\speak  Tombe volontairement entre mes bras après m'avoir raillé si effrontément, hypocrite et dangereuse femme, pensait d'Artagnan de son côté, et ensuite je rirai de toi avec celui que tu veux tuer par ma main.» 

D'Artagnan releva la tête. 

«Je suis prêt, dit-il. 

\speak  Vous m'avez donc comprise, cher monsieur d'Artagnan! dit Milady. 

\speak  Je devinerais un de vos regards. 

\speak  Ainsi vous emploieriez pour moi votre bras, qui s'est déjà acquis tant de renommée? 

\speak  À l'instant même. 

Mais moi, dit Milady, comment paierai-je un pareil service; je connais les amoureux, ce sont des gens qui ne font rien pour rien? 

\speak  Vous savez la seule réponse que je désire, dit d'Artagnan, la seule qui soit digne de vous et de moi!» 

Et il l'attira doucement vers lui. 

Elle résista à peine. 

«Intéressé! dit-elle en souriant. 

\speak  Ah! s'écria d'Artagnan véritablement emporté par la passion que cette femme avait le don d'allumer dans son cœur, ah! c'est que mon bonheur me paraît invraisemblable, et qu'ayant toujours peur de le voir s'envoler comme un rêve, j'ai hâte d'en faire une réalité. 

\speak  Eh bien, méritez donc ce prétendu bonheur. 

\speak  Je suis à vos ordres, dit d'Artagnan. 

\speak  Bien sûr? fit Milady avec un dernier doute. 

\speak  Nommez-moi l'infâme qui a pu faire pleurer vos beaux yeux. 

\speak  Qui vous dit que j'ai pleuré? dit-elle. 

\speak  Il me semblait\dots 

\speak  Les femmes comme moi ne pleurent pas, dit Milady. 

\speak  Tant mieux! Voyons, dites-moi comment il s'appelle. 

\speak  Songez que son nom c'est tout mon secret. 

\speak  Il faut cependant que je sache son nom. 

\speak  Oui, il le faut; voyez si j'ai confiance en vous! 

\speak  Vous me comblez de joie. Comment s'appelle-t-il? 

\speak  Vous le connaissez. 

\speak  Vraiment? 

\speak  Oui. 

\speak  Ce n'est pas un de mes amis? reprit d'Artagnan en jouant l'hésitation pour faire croire à son ignorance. 

\speak  Si c'était un de vos amis, vous hésiteriez donc?» s'écria Milady. Et un éclair de menace passa dans ses yeux. 

«Non, fût-ce mon frère!» s'écria d'Artagnan comme emporté par l'enthousiasme. 

Notre Gascon s'avançait sans risque; car il savait où il allait. 

«J'aime votre dévouement, dit Milady. 

\speak  Hélas! n'aimez-vous que cela en moi? demanda d'Artagnan. 

\speak  Je vous aime aussi, vous», dit-elle en lui prenant la main. 

Et l'ardente pression fit frissonner d'Artagnan, comme si, par le toucher, cette fièvre qui brûlait Milady le gagnait lui-même. 

«Vous m'aimez, vous! s'écria-t-il. Oh! si cela était, ce serait à en perdre la raison.» 

Et il l'enveloppa de ses deux bras. Elle n'essaya point d'écarter ses lèvres de son baiser, seulement elle ne le lui rendit pas. 

Ses lèvres étaient froides: il sembla à d'Artagnan qu'il venait d'embrasser une statue. 

Il n'en était pas moins ivre de joie, électrisé d'amour, il croyait presque à la tendresse de Milady; il croyait presque au crime de de Wardes. Si de Wardes eût été en ce moment sous sa main, il l'eût tué. 

Milady saisit l'occasion. 

«Il s'appelle\dots, dit-elle à son tour. 

\speak  De Wardes, je le sais, s'écria d'Artagnan. 

\speak  Et comment le savez-vous?» demanda Milady en lui saisissant les deux mains et en essayant de lire par ses yeux jusqu'au fond de son âme. 

D'Artagnan sentit qu'il s'était laissé emporter, et qu'il avait fait une faute. 

«Dites, dites, mais dites donc! répétait Milady, comment le savez-vous? 

\speak  Comment je le sais? dit d'Artagnan. 

\speak  Oui. 

\speak  Je le sais, parce que, hier, de Wardes, dans un salon où j'étais, a montré une bague qu'il a dit tenir de vous. 

\speak  Le misérable!» s'écria Milady. 

L'épithète, comme on le comprend bien, retentit jusqu'au fond du cœur de d'Artagnan. 

«Eh bien? continua-t-elle. 

\speak  Eh bien, je vous vengerai de ce misérable, reprit d'Artagnan en se donnant des airs de don Japhet d'Arménie. 

\speak  Merci, mon brave ami! s'écria Milady; et quand serai-je vengée? 

\speak  Demain, tout de suite, quand vous voudrez.» 

Milady allait s'écrier: «Tout de suite»; mais elle réfléchit qu'une pareille précipitation serait peu gracieuse pour d'Artagnan. 

D'ailleurs, elle avait mille précautions à prendre, mille conseils à donner à son défenseur, pour qu'il évitât les explications devant témoins avec le comte. Tout cela se trouva prévu par un mot de d'Artagnan. 

«Demain, dit-il, vous serez vengée ou je serai mort. 

\speak  Non! dit-elle, vous me vengerez; mais vous ne mourrez pas. C'est un lâche. 

\speak  Avec les femmes peut-être, mais pas avec les hommes. J'en sais quelque chose, moi. 

\speak  Mais il me semble que dans votre lutte avec lui, vous n'avez pas eu à vous plaindre de la fortune. 

\speak  La fortune est une courtisane: favorable hier, elle peut me trahir demain. 

\speak  Ce qui veut dire que vous hésitez maintenant. 

\speak  Non, je n'hésite pas, Dieu m'en garde; mais serait-il juste de me laisser aller à une mort possible sans m'avoir donné au moins un peu plus que de l'espoir?» 

Milady répondit par un coup d'œil qui voulait dire: 

«N'est-ce que cela? parlez donc.» 

Puis, accompagnant le coup d'œil de paroles explicatives. 

«C'est trop juste, dit-elle tendrement. 

\speak  Oh! vous êtes un ange, dit le jeune homme. 

\speak  Ainsi, tout est convenu? dit-elle. 

\speak  Sauf ce que je vous demande, chère âme! 

\speak  Mais, lorsque je vous dis que vous pouvez vous fier à ma tendresse? 

\speak  Je n'ai pas de lendemain pour attendre. 

\speak  Silence; j'entends mon frère: il est inutile qu'il vous trouve ici.» 

Elle sonna; Ketty parut. 

«Sortez par cette porte, dit-elle en poussant une petit porte dérobée, et revenez à onze heures; nous achèverons cet entretien: Ketty vous introduira chez moi.» 

La pauvre enfant pensa tomber à la renverse en entendant ces paroles. 

«Eh bien, que faites-vous, mademoiselle, à demeurer immobile comme une statue? Allons, reconduisez le chevalier; et ce soir, à onze heures, vous avez entendu!» 

«Il paraît que ses rendez-vous sont à onze heures, pensa d'Artagnan: c'est une habitude prise.» 

Milady lui tendit une main qu'il baisa tendrement. 

«Voyons, dit-il en se retirant et en répondant à peine aux reproches de Ketty, voyons, ne soyons pas un sot; décidément cette femme est une grande scélérate: prenons garde.» 