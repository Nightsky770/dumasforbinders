%!TeX root=../musketeersfr.tex 

\chapter{Conclusion}

\lettrine{L}{e} 6 du mois suivant, le roi, tenant la promesse qu'il avait faite au cardinal de quitter Paris pour revenir à La Rochelle, sortit de sa capitale tout étourdi encore de la nouvelle qui venait de s'y répandre que Buckingham venait d'être assassiné. 

Quoique prévenue que l'homme qu'elle avait tant aimé courait un danger, la reine, lorsqu'on lui annonça cette mort, ne voulut pas la croire; il lui arriva même de s'écrier imprudemment: 

«C'est faux! il vient de m'écrire.» 

Mais le lendemain il lui fallut bien croire à cette fatale nouvelle; La Porte, retenu comme tout le monde en Angleterre par les ordres du roi Charles I\ier, arriva porteur du dernier et funèbre présent que Buckingham envoyait à la reine. 

La joie du roi avait été très vive; il ne se donna pas la peine de la dissimuler et la fit même éclater avec affectation devant la reine. Louis XIII, comme tous les cœurs faibles, manquait de générosité. 

Mais bientôt le roi redevint sombre et mal portant: son front n'était pas de ceux qui s'éclaircissent pour longtemps; il sentait qu'en retournant au camp il allait reprendre son esclavage, et cependant il y retournait. 

Le cardinal était pour lui le serpent fascinateur et il était, lui, l'oiseau qui voltige de branche en branche sans pouvoir lui échapper. 

Aussi le retour vers La Rochelle était-il profondément triste. Nos quatre amis surtout faisaient l'étonnement de leurs camarades; ils voyageaient ensemble, côte à côte, l'œil sombre et la tête baissée. Athos relevait seul de temps en temps son large front; un éclair brillait dans ses yeux, un sourire amer passait sur ses lèvres, puis, pareil à ses camarades, il se laissait de nouveau aller à ses rêveries. 

Aussitôt l'arrivée de l'escorte dans une ville, dès qu'ils avaient conduit le roi à son logis, les quatre amis se retiraient ou chez eux ou dans quelque cabaret écarté, où ils ne jouaient ni ne buvaient; seulement ils parlaient à voix basse en regardant avec attention si nul ne les écoutait. 

Un jour que le roi avait fait halte sur la route pour voler la pie, et que les quatre amis, selon leur habitude, au lieu de suivre la chasse, s'étaient arrêtés dans un cabaret sur la grande route, un homme, qui venait de La Rochelle à franc étrier, s'arrêta à la porte pour boire un verre de vin, et plongea son regard dans l'intérieur de la chambre où étaient attablés les quatre mousquetaires. 

«Holà! monsieur d'Artagnan! dit-il, n'est-ce point vous que je vois là-bas?» 

D'Artagnan leva la tête et poussa un cri de joie. Cet homme qu'il appelait son fantôme, c'était son inconnu de Meung, de la rue des Fossoyeurs et d'Arras. 

D'Artagnan tira son épée et s'élança vers la porte. 

Mais cette fois, au lieu de fuir, l'inconnu s'élança à bas de son cheval, et s'avança à la rencontre de d'Artagnan. 

«Ah! monsieur, dit le jeune homme, je vous rejoins donc enfin; cette fois vous ne m'échapperez pas. 

\speak  Ce n'est pas mon intention non plus, monsieur, car cette fois je vous cherchais; au nom du roi, je vous arrête et dis que vous ayez à me rendre votre épée, monsieur, et cela sans résistance; il y va de la tête, je vous en avertis. 

\speak  Qui êtes-vous donc? demanda d'Artagnan en baissant son épée, mais sans la rendre encore. 

\speak  Je suis le chevalier de Rochefort, répondit l'inconnu, l'écuyer de M. le cardinal de Richelieu, et j'ai ordre de vous ramener à Son Éminence. 

\speak  Nous retournons auprès de Son Éminence, monsieur le chevalier, dit Athos en s'avançant, et vous accepterez bien la parole de M. d'Artagnan, qu'il va se rendre en droite ligne à La Rochelle. 

\speak  Je dois le remettre entre les mains des gardes qui le ramèneront au camp. 

\speak  Nous lui en servirons, monsieur, sur notre parole de gentilshommes; mais sur notre parole de gentilshommes aussi, ajouta Athos en fronçant le sourcil, M. d'Artagnan ne nous quittera pas.» 

Le chevalier de Rochefort jeta un coup d'œil en arrière et vit que Porthos et Aramis s'étaient placés entre lui et la porte; il comprit qu'il était complètement à la merci de ces quatre hommes. 

«Messieurs, dit-il, si M. d'Artagnan veut me rendre son épée, et joindre sa parole à la vôtre, je me contenterai de votre promesse de conduire M. d'Artagnan au quartier de Mgr le cardinal. 

\speak  Vous avez ma parole, monsieur, dit d'Artagnan, et voici mon épée. 

\speak  Cela me va d'autant mieux, ajouta Rochefort, qu'il faut que je continue mon voyage. 

\speak  Si c'est pour rejoindre Milady, dit froidement Athos, c'est inutile, vous ne la retrouverez pas. 

\speak  Qu'est-elle donc devenue? demanda vivement Rochefort. 

\speak  Revenez au camp et vous le saurez.» 

Rochefort demeura un instant pensif, puis, comme on n'était plus qu'à une journée de Surgères, jusqu'où le cardinal devait venir au-devant du roi, il résolut de suivre le conseil d'Athos et de revenir avec eux. 

D'ailleurs ce retour lui offrait un avantage, c'était de surveiller lui-même son prisonnier. 

On se remit en route. 

Le lendemain, à trois heures de l'après-midi, on arriva à Surgères. Le cardinal y attendait Louis XIII. Le ministre et le roi y échangèrent force caresses, se félicitèrent de l'heureux hasard qui débarrassait la France de l'ennemi acharné qui ameutait l'Europe contre elle. Après quoi, le cardinal, qui avait été prévenu par Rochefort que d'Artagnan était arrêté, et qui avait hâte de le voir, prit congé du roi en l'invitant à venir voir le lendemain les travaux de la digue qui étaient achevés. 

En revenant le soir à son quartier du pont de La Pierre, le cardinal trouva debout, devant la porte de la maison qu'il habitait, d'Artagnan sans épée et les trois mousquetaires armés. 

Cette fois, comme il était en force, il les regarda sévèrement, et fit signe de l'œil et de la main à d'Artagnan de le suivre. 

D'Artagnan obéit. 

«Nous t'attendrons, d'Artagnan», dit Athos assez haut pour que le cardinal l'entendit. 

Son Éminence fronça le sourcil, s'arrêta un instant, puis continua son chemin sans prononcer une seule parole. 

D'Artagnan entra derrière le cardinal, et Rochefort derrière d'Artagnan; la porte fut gardée. 

Son Éminence se rendit dans la chambre qui lui servait de cabinet, et fit signe à Rochefort d'introduire le jeune mousquetaire. 

Rochefort obéit et se retira. 

D'Artagnan resta seul en face du cardinal; c'était sa seconde entrevue avec Richelieu, et il avoua depuis qu'il avait été bien convaincu que ce serait la dernière. 

Richelieu resta debout, appuyé contre la cheminée, une table était dressée entre lui et d'Artagnan. 

«Monsieur, dit le cardinal, vous avez été arrêté par mes ordres. 

\speak  On me l'a dit, Monseigneur. 

\speak  Savez-vous pourquoi? 

\speak  Non, Monseigneur; car la seule chose pour laquelle je pourrais être arrêté est encore inconnue de Son Éminence.» 

Richelieu regarda fixement le jeune homme. 

«Oh! Oh! dit-il, que veut dire cela? 

\speak  Si Monseigneur veut m'apprendre d'abord les crimes qu'on m'impute, je lui dirai ensuite les faits que j'ai accomplis. 

\speak  On vous impute des crimes qui ont fait choir des têtes plus hautes que la vôtre, monsieur! dit le cardinal. 

\speak  Lesquels, Monseigneur? demanda d'Artagnan avec un calme qui étonna le cardinal lui-même. 

\speak  On vous impute d'avoir correspondu avec les ennemis du royaume, on vous impute d'avoir surpris les secrets de l'État, on vous impute d'avoir essayé de faire avorter les plans de votre général. 

\speak  Et qui m'impute cela, Monseigneur? dit d'Artagnan, qui se doutait que l'accusation venait de Milady: une femme flétrie par la justice du pays, une femme qui a épousé un homme en France et un autre en Angleterre, une femme qui a empoisonné son second mari et qui a tenté de m'empoisonner moi-même! 

\speak  Que dites-vous donc là? Monsieur, s'écria le cardinal étonné, et de quelle femme parlez-vous ainsi? 

\speak  De Milady de Winter, répondit d'Artagnan; oui, de Milady de Winter, dont, sans doute, Votre Éminence ignorait tous les crimes lorsqu'elle l'a honorée de sa confiance. 

\speak  Monsieur, dit le cardinal, si Milady de Winter a commis les crimes que vous dites, elle sera punie. 

\speak  Elle l'est, Monseigneur. 

\speak  Et qui l'a punie? 

\speak  Nous. 

\speak  Elle est en prison? 

\speak  Elle est morte. 

\speak  Morte! répéta le cardinal, qui ne pouvait croire à ce qu'il entendait: morte! n'avez-vous pas dit qu'elle était morte? 

\speak  Trois fois elle avait essayé de me tuer, et je lui avais pardonné, mais elle a tué la femme que j'aimais. Alors, mes amis et moi, nous l'avons prise, jugée et condamnée.» 

D'Artagnan alors raconta l'empoisonnement de Mme Bonacieux dans le couvent des Carmélites de Béthune, le jugement de la maison isolée, l'exécution sur les bords de la Lys. 

Un frisson courut par tout le corps du cardinal, qui cependant ne frissonnait pas facilement. 

Mais tout à coup, comme subissant l'influence d'une pensée muette, la physionomie du cardinal, sombre jusqu'alors, s'éclaircit peu à peu et arriva à la plus parfaite sérénité. 

«Ainsi, dit-il avec une voix dont la douceur contrastait avec la sévérité de ses paroles, vous vous êtes constitués juges, sans penser que ceux qui n'ont pas mission de punir et qui punissent sont des assassins! 

\speak  Monseigneur, je vous jure que je n'ai pas eu un instant l'intention de défendre ma tête contre vous. Je subirai le châtiment que Votre Éminence voudra bien m'infliger. Je ne tiens pas assez à la vie pour craindre la mort. 

\speak  Oui, je le sais, vous êtes un homme de cœur, monsieur, dit le cardinal avec une voix presque affectueuse; je puis donc vous dire d'avance que vous serez jugé, condamné même. 

\speak  Un autre pourrait répondre à Votre Éminence qu'il a sa grâce dans sa poche; moi je me contenterai de vous dire: «Ordonnez, Monseigneur, je suis prêt.» 

\speak  Votre grâce? dit Richelieu surpris. 

\speak  Oui, Monseigneur, dit d'Artagnan. 

\speak  Et signée de qui? du roi?» 

Et le cardinal prononça ces mots avec une singulière expression de mépris. 

«Non, de Votre Éminence. 

\speak  De moi? vous êtes fou, monsieur? 

\speak  Monseigneur reconnaîtra sans doute son écriture.» 

Et d'Artagnan présenta au cardinal le précieux papier qu'Athos avait arraché à Milady, et qu'il avait donné à d'Artagnan pour lui servir de sauvegarde. 

Son Éminence prit le papier et lut d'une voix lente et en appuyant sur chaque syllabe: 

\begin{mail}{Au camp de la Rochelle, ce 5 août 1628.}
	
C'est par mon ordre et pour le bien de l'État que le porteur du présent a fait ce qu'il a fait.
\closeletter{Richelieu}
\end{mail}

Le cardinal, après avoir lu ces deux lignes, tomba dans une rêverie profonde, mais il ne rendit pas le papier à d'Artagnan. 

«Il médite de quel genre de supplice il me fera mourir, se dit tout bas d'Artagnan; eh bien, ma foi! il verra comment meurt un gentilhomme.» 

Le jeune mousquetaire était en excellente disposition pour trépasser héroïquement. 

Richelieu pensait toujours, roulait et déroulait le papier dans ses mains. Enfin il leva la tête, fixa son regard d'aigle sur cette physionomie loyale, ouverte, intelligente, lut sur ce visage sillonné de larmes toutes les souffrances qu'il avait endurées depuis un mois, et songea pour la troisième ou quatrième fois combien cet enfant de vingt et un ans avait d'avenir, et quelles ressources son activité, son courage et son esprit pouvaient offrir à un bon maître. 

D'un autre côté, les crimes, la puissance, le génie infernal de Milady l'avaient plus d'une fois épouvanté. Il sentait comme une joie secrète d'être à jamais débarrassé de ce complice dangereux. 

Il déchira lentement le papier que d'Artagnan lui avait si généreusement remis. 

«Je suis perdu», dit en lui-même d'Artagnan. 

Et il s'inclina profondément devant le cardinal en homme qui dit: «Seigneur, que votre volonté soit faite!» 

Le cardinal s'approcha de la table, et, sans s'asseoir, écrivit quelques lignes sur un parchemin dont les deux tiers étaient déjà remplis et y apposa son sceau. 

«Ceci est ma condamnation, dit d'Artagnan; il m'épargne l'ennui de la Bastille et les lenteurs d'un jugement. C'est encore fort aimable à lui.» 

«Tenez, monsieur, dit le cardinal au jeune homme, je vous ai pris un blanc-seing et je vous en rends un autre. Le nom manque sur ce brevet: vous l'écrirez vous-même.» 

D'Artagnan prit le papier en hésitant et jeta les yeux dessus. 

C'était une lieutenance dans les mousquetaires. 

D'Artagnan tomba aux pieds du cardinal. 

«Monseigneur, dit-il, ma vie est à vous; disposez-en désormais; mais cette faveur que vous m'accordez, je ne la mérite pas: j'ai trois amis qui sont plus méritants et plus dignes\dots 

\speak  Vous êtes un brave garçon, d'Artagnan, interrompit le cardinal en lui frappant familièrement sur l'épaule, charmé qu'il était d'avoir vaincu cette nature rebelle. Faites de ce brevet ce qu'il vous plaira. Seulement rappelez-vous que, quoique le nom soit en blanc, c'est à vous que je le donne. 

\speak  Je ne l'oublierai jamais, répondit d'Artagnan. Votre Éminence peut en être certaine.» 

Le cardinal se retourna et dit à haute voix: 

«Rochefort!» 

Le chevalier, qui sans doute était derrière la porte entra aussitôt. 

«Rochefort, dit le cardinal, vous voyez M. d'Artagnan; je le reçois au nombre de mes amis; ainsi donc que l'on s'embrasse et que l'on soit sage si l'on tient à conserver sa tête. 

Rochefort et d'Artagnan s'embrassèrent du bout des lèvres; mais le cardinal était là, qui les observait de son œil vigilant. 

Ils sortirent de la chambre en même temps. 

«Nous nous retrouverons, n'est-ce pas, monsieur? 

\speak  Quand il vous plaira, fit d'Artagnan. 

\speak  L'occasion viendra, répondit Rochefort. 

\speak  Hein?» fit Richelieu en ouvrant la porte. 

Les deux hommes se sourirent, se serrèrent la main et saluèrent Son Éminence. 

«Nous commencions à nous impatienter, dit Athos. 

\speak  Me voilà, mes amis! répondit d'Artagnan, non seulement libre, mais en faveur. 

\speak  Vous nous conterez cela? 

\speak  Dès ce soir.» 

En effet, dès le soir même d'Artagnan se rendit au logis d'Athos, qu'il trouva en train de vider sa bouteille de vin d'Espagne, occupation qu'il accomplissait religieusement tous les soirs. 

Il lui raconta ce qui s'était passé entre le cardinal et lui, et tirant le brevet de sa poche: 

«Tenez, mon cher Athos, voilà, dit-il, qui vous revient tout naturellement.» 

Athos sourit de son doux et charmant sourire. 

«Amis, dit-il, pour Athos c'est trop; pour le comte de La Fère, c'est trop peu. Gardez ce brevet, il est à vous; hélas, mon Dieu! vous l'avez acheté assez cher.» 

D'Artagnan sortit de la chambre d'Athos, et entra dans celle de Porthos. 

Il le trouva vêtu d'un magnifique habit, couvert de broderies splendides, et se mirant dans une glace. 

«Ah! ah! dit Porthos, c'est vous, cher ami! comment trouvez-vous que ce vêtement me va? 

\speak  À merveille, dit d'Artagnan, mais je viens vous proposer un habit qui vous ira mieux encore. 

\speak  Lequel? demanda Porthos. 

\speak  Celui de lieutenant aux mousquetaires. 

D'Artagnan raconta à Porthos son entrevue avec le cardinal, et tirant le brevet de sa poche: 

«Tenez, mon cher, dit-il, écrivez votre nom là-dessus, et soyez bon chef pour moi. 

Porthos jeta les yeux sur le brevet, et le rendit à d'Artagnan, au grand étonnement du jeune homme. 

«Oui, dit-il, cela me flatterait beaucoup, mais je n'aurais pas assez longtemps à jouir de cette faveur. Pendant notre expédition de Béthune, le mari de ma duchesse est mort; de sorte que, mon cher, le coffre du défunt me tendant les bras, j'épouse la veuve. Tenez, j'essayais mon habit de noce; gardez la lieutenance, mon cher, gardez.» 

Et il rendit le brevet à d'Artagnan. 

Le jeune homme entra chez Aramis. 

Il le trouva agenouillé devant un prie-Dieu, le front appuyé contre son livre d'heures ouvert. 

Il lui raconta son entrevue avec le cardinal, et tirant pour la troisième fois son brevet de sa poche: 

«Vous, notre ami, notre lumière, notre protecteur invisible, dit-il, acceptez ce brevet; vous l'avez mérité plus que personne, par votre sagesse et vos conseils toujours suivis de si heureux résultats. 

\speak  Hélas, cher ami! dit Aramis, nos dernières aventures m'ont dégoûté tout à fait de la vie d'homme d'épée. Cette fois, mon parti est pris irrévocablement, après le siège j'entre chez les lazaristes. Gardez ce brevet, d'Artagnan, le métier des armes vous convient, vous serez un brave et aventureux capitaine.» 

D'Artagnan, l'œil humide de reconnaissance et brillant de joie, revint à Athos, qu'il trouva toujours attablé et mirant son dernier verre de malaga à la lueur de la lampe. 

«Eh bien, dit-il, eux aussi m'ont refusé. 

\speak  C'est que personne, cher ami, n'en était plus digne que vous.» 

Il prit une plume, écrivit sur le brevet le nom de d'Artagnan, et le lui remit. 

«Je n'aurai donc plus d'amis, dit le jeune homme, hélas! plus rien, que d'amers souvenirs\dots» 

Et il laissa tomber sa tête entre ses deux mains, tandis que deux larmes roulaient le long de ses joues. 

«Vous êtes jeune, vous, répondit Athos, et vos souvenirs amers ont le temps de se changer en doux souvenirs!»
