%!TeX root=../musketeersfr.tex 

\chapter{L'Audience}

\lettrine{M}{.} de Tréville était pour le moment de fort méchante humeur; néanmoins il salua poliment le jeune homme, qui s'inclina jusqu'à terre, et il sourit en recevant son compliment, dont l'accent béarnais lui rappela à la fois sa jeunesse et son pays, double souvenir qui fait sourire l'homme à tous les âges. Mais, se rapprochant presque aussitôt de l'antichambre et faisant à d'Artagnan un signe de la main, comme pour lui demander la permission d'en finir avec les autres avant de commencer avec lui, il appela trois fois, en grossissant la voix à chaque fois, de sorte qu'il parcourut tous les tons intervallaires entre l'accent impératif et l'accent irrité: 

«Athos! Porthos! Aramis!» 

Les deux mousquetaires avec lesquels nous avons déjà fait connaissance, et qui répondaient aux deux derniers de ces trois noms, quittèrent aussitôt les groupes dont ils faisaient partie et s'avancèrent vers le cabinet, dont la porte se referma derrière eux dès qu'ils en eurent franchi le seuil. Leur contenance, bien qu'elle ne fût pas tout à fait tranquille, excita cependant par son laisser-aller à la fois plein de dignité et de soumission, l'admiration de d'Artagnan, qui voyait dans ces hommes des demi-dieux, et dans leur chef un Jupiter olympien armé de tous ses foudres. 

Quand les deux mousquetaires furent entrés, quand la porte fut refermée derrière eux, quand le murmure bourdonnant de l'antichambre, auquel l'appel qui venait d'être fait avait sans doute donné un nouvel aliment eut recommencé; quand enfin M. de Tréville eut trois ou quatre fois arpenté, silencieux et le sourcil froncé, toute la longueur de son cabinet, passant chaque fois devant Porthos et Aramis, roides et muets comme à la parade, il s'arrêta tout à coup en face d'eux, et les couvrant des pieds à la tête d'un regard irrité: 

«Savez-vous ce que m'a dit le roi, s'écria-t-il, et cela pas plus tard qu'hier au soir? le savez-vous, messieurs? 

\speak  Non, répondirent après un instant de silence les deux mousquetaires; non, monsieur, nous l'ignorons. 

\speak  Mais j'espère que vous nous ferez l'honneur de nous le dire, ajouta Aramis de son ton le plus poli et avec la plus gracieuse révérence. 

\speak  Il m'a dit qu'il recruterait désormais ses mousquetaires parmi les gardes de M. le cardinal! 

\speak  Parmi les gardes de M. le cardinal! et pourquoi cela? demanda vivement Porthos. 

\speak  Parce qu'il voyait bien que sa piquette avait besoin d'être ragaillardie par un mélange de bon vin.» 

Les deux mousquetaires rougirent jusqu'au blanc des yeux. D'Artagnan ne savait où il en était et eût voulu être à cent pieds sous terre. 

«Oui, oui, continua M. de Tréville en s'animant, oui, et Sa Majesté avait raison, car, sur mon honneur, il est vrai que les mousquetaires font triste figure à la cour. M. le cardinal racontait hier au jeu du roi, avec un air de condoléance qui me déplut fort, qu'avant-hier ces damnés mousquetaires, ces diables à quatre --- il appuyait sur ces mots avec un accent ironique qui me déplut encore davantage ---, ces pourfendeurs, ajoutait-il en me regardant de son œil de chat-tigre, s'étaient attardés rue Férou, dans un cabaret, et qu'une ronde de ses gardes --- j'ai cru qu'il allait me rire au nez --- avait été forcée d'arrêter les perturbateurs. Morbleu! vous devez en savoir quelque chose! Arrêter des mousquetaires! Vous en étiez, vous autres, ne vous en défendez pas, on vous a reconnus, et le cardinal vous a nommés. Voilà bien ma faute, oui, ma faute, puisque c'est moi qui choisis mes hommes. Voyons, vous, Aramis, pourquoi diable m'avez-vous demandé la casaque quand vous alliez être si bien sous la soutane? Voyons, vous, Porthos, n'avez-vous un si beau baudrier d'or que pour y suspendre une épée de paille? Et Athos! je ne vois pas Athos. Où est-il? 

\speak  Monsieur, répondit tristement Aramis, il est malade, fort malade. 

\speak  Malade, fort malade, dites-vous? et de quelle maladie? 

\speak  On craint que ce ne soit de la petite vérole, monsieur, répondit Porthos voulant mêler à son tour un mot à la conversation, et ce qui serait fâcheux en ce que très certainement cela gâterait son visage. 

\speak  De la petite vérole! Voilà encore une glorieuse histoire que vous me contez là, Porthos!\dots Malade de la petite vérole, à son âge?\dots Non pas!\dots mais blessé sans doute, tué peut-être\dots Ah! si je le savais!\dots Sangdieu! messieurs les mousquetaires, je n'entends pas que l'on hante ainsi les mauvais lieux, qu'on se prenne de querelle dans la rue et qu'on joue de l'épée dans les carrefours. Je ne veux pas enfin qu'on prête à rire aux gardes de M. le cardinal, qui sont de braves gens, tranquilles, adroits, qui ne se mettent jamais dans le cas d'être arrêtés, et qui d'ailleurs ne se laisseraient pas arrêter, eux!\dots j'en suis sûr\dots Ils aimeraient mieux mourir sur la place que de faire un pas en arrière\dots Se sauver, détaler, fuir, c'est bon pour les mousquetaires du roi, cela!» 

Porthos et Aramis frémissaient de rage. Ils auraient volontiers étranglé M. de Tréville, si au fond de tout cela ils n'avaient pas senti que c'était le grand amour qu'il leur portait qui le faisait leur parler ainsi. Ils frappaient le tapis du pied, se mordaient les lèvres jusqu'au sang et serraient de toute leur force la garde de leur épée. Au-dehors on avait entendu appeler, comme nous l'avons dit, Athos, Porthos et Aramis, et l'on avait deviné, à l'accent de la voix de M. de Tréville, qu'il était parfaitement en colère. Dix têtes curieuses étaient appuyées à la tapisserie et pâlissaient de fureur, car leurs oreilles collées à la porte ne perdaient pas une syllabe de ce qui se disait, tandis que leurs bouches répétaient au fur et à mesure les paroles insultantes du capitaine à toute la population de l'antichambre. En un instant depuis la porte du cabinet jusqu'à la porte de la rue, tout l'hôtel fut en ébullition. 

«Ah! les mousquetaires du roi se font arrêter par les gardes de M. le cardinal», continua M. de Tréville aussi furieux à l'intérieur que ses soldats, mais saccadant ses paroles et les plongeant une à une pour ainsi dire et comme autant de coups de stylet dans la poitrine de ses auditeurs. «Ah! six gardes de Son Éminence arrêtent six mousquetaires de Sa Majesté! Morbleu! j'ai pris mon parti. Je vais de ce pas au Louvre; je donne ma démission de capitaine des mousquetaires du roi pour demander une lieutenance dans les gardes du cardinal, et s'il me refuse, morbleu! je me fais abbé.» 

À ces paroles, le murmure de l'extérieur devint une explosion: partout on n'entendait que jurons et blasphèmes. Les morbleu! les sangdieu! les morts de tous les diables! se croisaient dans l'air. D'Artagnan cherchait une tapisserie derrière laquelle se cacher, et se sentait une envie démesurée de se fourrer sous la table. 

«Eh bien, mon capitaine, dit Porthos hors de lui, la vérité est que nous étions six contre six, mais nous avons été pris en traître, et avant que nous eussions eu le temps de tirer nos épées, deux d'entre nous étaient tombés morts, et Athos, blessé grièvement, ne valait guère mieux. Car vous le connaissez, Athos; eh bien, capitaine, il a essayé de se relever deux fois, et il est retombé deux fois. Cependant nous ne nous sommes pas rendus, non! l'on nous a entraînés de force. En chemin, nous nous sommes sauvés. Quant à Athos, on l'avait cru mort, et on l'a laissé bien tranquillement sur le champ de bataille, ne pensant pas qu'il valût la peine d'être emporté. Voilà l'histoire. Que diable, capitaine! on ne gagne pas toutes les batailles. Le grand Pompée a perdu celle de Pharsale, et le roi François I\ier\, qui, à ce que j'ai entendu dire, en valait bien un autre, a perdu cependant celle de Pavie. 

\speak  Et j'ai l'honneur de vous assurer que j'en ai tué un avec sa propre épée, dit Aramis, car la mienne s'est brisée à la première parade\dots Tué ou poignardé, monsieur, comme il vous sera agréable. 

\speak  Je ne savais pas cela, reprit M. de Tréville d'un ton un peu radouci. M. le cardinal avait exagéré, à ce que je vois. 

\speak  Mais de grâce, monsieur, continua Aramis, qui, voyant son capitaine s'apaiser, osait hasarder une prière, de grâce, monsieur, ne dites pas qu'Athos lui-même est blessé: il serait au désespoir que cela parvint aux oreilles du roi, et comme la blessure est des plus graves, attendu qu'après avoir traversé l'épaule elle pénètre dans la poitrine, il serait à craindre\dots» 

Au même instant la portière se souleva, et une tête noble et belle, mais affreusement pâle, parut sous la frange. 

«Athos! s'écrièrent les deux mousquetaires. 

\speak  Athos! répéta M. de Tréville lui-même. 

\speak  Vous m'avez mandé, monsieur, dit Athos à M. de Tréville d'une voix affaiblie mais parfaitement calme, vous m'avez demandé, à ce que m'ont dit nos camarades, et je m'empresse de me rendre à vos ordres; voilà, monsieur, que me voulez-vous?» 

Et à ces mots le mousquetaire, en tenue irréprochable, sanglé comme de coutume, entra d'un pas ferme dans le cabinet. M. de Tréville, ému jusqu'au fond du cœur de cette preuve de courage, se précipita vers lui. 

«J'étais en train de dire à ces messieurs, ajouta-t-il, que je défends à mes mousquetaires d'exposer leurs jours sans nécessité, car les braves gens sont bien chers au roi, et le roi sait que ses mousquetaires sont les plus braves gens de la terre. Votre main, Athos.» 

Et sans attendre que le nouveau venu répondît de lui-même à cette preuve d'affection, M. de Tréville saisissait sa main droite et la lui serrait de toutes ses forces, sans s'apercevoir qu'Athos, quel que fût son empire sur lui-même, laissait échapper un mouvement de douleur et pâlissait encore, ce que l'on aurait pu croire impossible. 

La porte était restée entrouverte, tant l'arrivée d'Athos, dont, malgré le secret gardé, la blessure était connue de tous, avait produit de sensation. Un brouhaha de satisfaction accueillit les derniers mots du capitaine et deux ou trois têtes, entraînées par l'enthousiasme, apparurent par les ouvertures de la tapisserie. Sans doute, M. de Tréville allait réprimer par de vives paroles cette infraction aux lois de l'étiquette, lorsqu'il sentit tout à coup la main d'Athos se crisper dans la sienne, et qu'en portant les yeux sur lui il s'aperçut qu'il allait s'évanouir. Au même instant Athos, qui avait rassemblé toutes ses forces pour lutter contre la douleur, vaincu enfin par elle, tomba sur le parquet comme s'il fût mort. 

«Un chirurgien! cria M. de Tréville. Le mien, celui du roi, le meilleur! Un chirurgien! ou, sangdieu! mon brave Athos va trépasser.» 

Aux cris de M. de Tréville, tout le monde se précipita dans son cabinet sans qu'il songeât à en fermer la porte à personne, chacun s'empressant autour du blessé. Mais tout cet empressement eût été inutile, si le docteur demandé ne se fût trouvé dans l'hôtel même; il fendit la foule, s'approcha d'Athos toujours évanoui, et, comme tout ce bruit et tout ce mouvement le gênait fort, il demanda comme première chose et comme la plus urgente que le mousquetaire fût emporté dans une chambre voisine. Aussitôt M. de Tréville ouvrit une porte et montra le chemin à Porthos et à Aramis, qui emportèrent leur camarade dans leurs bras. Derrière ce groupe marchait le chirurgien, et derrière le chirurgien, la porte se referma. 

Alors le cabinet de M. de Tréville, ce lieu ordinairement si respecté, devint momentanément une succursale de l'antichambre. Chacun discourait, pérorait, parlait haut, jurant, sacrant, donnant le cardinal et ses gardes à tous les diables. 

Un instant après, Porthos et Aramis rentrèrent; le chirurgien et M. de Tréville seuls étaient restés près du blessé. 

Enfin M. de Tréville rentra à son tour. Le blessé avait repris connaissance; le chirurgien déclarait que l'état du mousquetaire n'avait rien qui pût inquiéter ses amis, sa faiblesse ayant été purement et simplement occasionnée par la perte de son sang. 

Puis M. de Tréville fit un signe de la main, et chacun se retira, excepté d'Artagnan, qui n'oubliait point qu'il avait audience et qui, avec sa ténacité de Gascon, était demeuré à la même place. 

Lorsque tout le monde fut sorti et que la porte fut refermée, M. de Tréville se retourna et se trouva seul avec le jeune homme. L'événement qui venait d'arriver lui avait quelque peu fait perdre le fil de ses idées. Il s'informa de ce que lui voulait l'obstiné solliciteur. D'Artagnan alors se nomma, et M. de Tréville, se rappelant d'un seul coup tous ses souvenirs du présent et du passé, se trouva au courant de sa situation. 

«Pardon lui dit-il en souriant, pardon, mon cher compatriote, mais je vous avais parfaitement oublié. Que voulez-vous! un capitaine n'est rien qu'un père de famille chargé d'une plus grande responsabilité qu'un père de famille ordinaire. Les soldats sont de grands enfants; mais comme je tiens à ce que les ordres du roi, et surtout ceux de M. le cardinal, soient exécutés\dots» 

D'Artagnan ne put dissimuler un sourire. À ce sourire, M. de Tréville jugea qu'il n'avait point affaire à un sot, et venant droit au fait, tout en changeant de conversation: 

«J'ai beaucoup aimé monsieur votre père, dit-il. Que puis-je faire pour son fils? hâtez-vous, mon temps n'est pas à moi. 

\speak  Monsieur, dit d'Artagnan, en quittant Tarbes et en venant ici, je me proposais de vous demander, en souvenir de cette amitié dont vous n'avez pas perdu mémoire, une casaque de mousquetaire; mais, après tout ce que je vois depuis deux heures, je comprends qu'une telle faveur serait énorme, et je tremble de ne point la mériter. 

\speak  C'est une faveur en effet, jeune homme, répondit M. de Tréville; mais elle peut ne pas être si fort au-dessus de vous que vous le croyez ou que vous avez l'air de le croire. Toutefois une décision de Sa Majesté a prévu ce cas, et je vous annonce avec regret qu'on ne reçoit personne mousquetaire avant l'épreuve préalable de quelques campagnes, de certaines actions d'éclat, ou d'un service de deux ans dans quelque autre régiment moins favorisé que le nôtre.» 

D'Artagnan s'inclina sans rien répondre. Il se sentait encore plus avide d'endosser l'uniforme de mousquetaire depuis qu'il y avait de si grandes difficultés à l'obtenir. 

«Mais, continua Tréville en fixant sur son compatriote un regard si perçant qu'on eût dit qu'il voulait lire jusqu'au fond de son cœur, mais, en faveur de votre père, mon ancien compagnon, comme je vous l'ai dit, je veux faire quelque chose pour vous, jeune homme. Nos cadets de Béarn ne sont ordinairement pas riches, et je doute que les choses aient fort changé de face depuis mon départ de la province. Vous ne devez donc pas avoir de trop, pour vivre, de l'argent que vous avez apporté avec vous.» 

D'Artagnan se redressa d'un air fier qui voulait dire qu'il ne demandait l'aumône à personne. 

«C'est bien, jeune homme, c'est bien, continua Tréville, je connais ces airs-là, je suis venu à Paris avec quatre écus dans ma poche, et je me serais battu avec quiconque m'aurait dit que je n'étais pas en état d'acheter le Louvre.» 

D'Artagnan se redressa de plus en plus; grâce à la vente de son cheval, il commençait sa carrière avec quatre écus de plus que M. de Tréville n'avait commencé la sienne. 

«Vous devez donc, disais-je, avoir besoin de conserver ce que vous avez, si forte que soit cette somme; mais vous devez avoir besoin aussi de vous perfectionner dans les exercices qui conviennent à un gentilhomme. J'écrirai dès aujourd'hui une lettre au directeur de l'académie royale, et dès demain il vous recevra sans rétribution aucune. Ne refusez pas cette petite douceur. Nos gentilshommes les mieux nés et les plus riches la sollicitent quelquefois, sans pouvoir l'obtenir. Vous apprendrez le manège du cheval, l'escrime et la danse; vous y ferez de bonnes connaissances, et de temps en temps vous reviendrez me voir pour me dire où vous en êtes et si je puis faire quelque chose pour vous.» 

D'Artagnan, tout étranger qu'il fût encore aux façons de cour, s'aperçut de la froideur de cet accueil. 

«Hélas, monsieur, dit-il, je vois combien la lettre de recommandation que mon père m'avait remise pour vous me fait défaut aujourd'hui! 

\speak  En effet, répondit M. de Tréville, je m'étonne que vous ayez entrepris un aussi long voyage sans ce viatique obligé, notre seule ressource à nous autres Béarnais. 

\speak  Je l'avais, monsieur, et, Dieu merci, en bonne forme, s'écria d'Artagnan; mais on me l'a perfidement dérobé.» 

Et il raconta toute la scène de Meung, dépeignit le gentilhomme inconnu dans ses moindres détails, le tout avec une chaleur, une vérité qui charmèrent M. de Tréville. 

«Voilà qui est étrange, dit ce dernier en méditant; vous aviez donc parlé de moi tout haut? 

\speak  Oui, monsieur, sans doute j'avais commis cette imprudence; que voulez-vous, un nom comme le vôtre devait me servir de bouclier en route: jugez si je me suis mis souvent à couvert!» 

La flatterie était fort de mise alors, et M. de Tréville aimait l'encens comme un roi ou comme un cardinal. Il ne put donc s'empêcher de sourire avec une visible satisfaction, mais ce sourire s'effaça bientôt, et revenant de lui-même à l'aventure de Meung: 

«Dites-moi, continua-t-il, ce gentilhomme n'avait-il pas une légère cicatrice à la tempe? 

\speak  Oui, comme le ferait l'éraflure d'une balle. 

\speak  N'était-ce pas un homme de belle mine? 

\speak  Oui. 

\speak  De haute taille? 

\speak  Oui. 

\speak  Pâle de teint et brun de poil? 

\speak  Oui, oui, c'est cela. Comment se fait-il, monsieur, que vous connaissiez cet homme? Ah! si jamais je le retrouve, et je le retrouverai, je vous le jure, fût-ce en enfer\dots 

\speak  Il attendait une femme? continua Tréville. 

\speak  Il est du moins parti après avoir causé un instant avec celle qu'il attendait. 

\speak  Vous ne savez pas quel était le sujet de leur conversation? 

\speak  Il lui remettait une boîte, lui disait que cette boîte contenait ses instructions, et lui recommandait de ne l'ouvrir qu'à Londres. 

\speak  Cette femme était anglaise? 

\speak  Il l'appelait Milady. 

\speak  C'est lui! murmura Tréville, c'est lui! je le croyais encore à Bruxelles! 

\speak  Oh! monsieur, si vous savez quel est cet homme, s'écria d'Artagnan, indiquez-moi qui il est et d'où il est, puis je vous tiens quitte de tout, même de votre promesse de me faire entrer dans les mousquetaires; car avant toute chose je veux me venger. 

\speak  Gardez-vous-en bien, jeune homme, s'écria Tréville; si vous le voyez venir, au contraire, d'un côté de la rue, passez de l'autre! Ne vous heurtez pas à un pareil rocher: il vous briserait comme un verre. 

\speak  Cela n'empêche pas, dit d'Artagnan, que si jamais je le retrouve\dots 

\speak  En attendant, reprit Tréville, ne le cherchez pas, si j'ai un conseil à vous donner.» 

Tout à coup Tréville s'arrêta, frappé d'un soupçon subit. Cette grande haine que manifestait si hautement le jeune voyageur pour cet homme, qui, chose assez peu vraisemblable, lui avait dérobé la lettre de son père, cette haine ne cachait-elle pas quelque perfidie? ce jeune homme n'était-il pas envoyé par Son Éminence? ne venait-il pas pour lui tendre quelque piège? ce prétendu d'Artagnan n'était-il pas un émissaire du cardinal qu'on cherchait à introduire dans sa maison, et qu'on avait placé près de lui pour surprendre sa confiance et pour le perdre plus tard, comme cela s'était mille fois pratiqué? Il regarda d'Artagnan plus fixement encore cette seconde fois que la première. Il fut médiocrement rassuré par l'aspect de cette physionomie pétillante d'esprit astucieux et d'humilité affectée. 

«Je sais bien qu'il est Gascon, pensa-t-il; mais il peut l'être aussi bien pour le cardinal que pour moi. Voyons, éprouvons-le.» 

«Mon ami, lui dit-il lentement, je veux, comme au fils de mon ancien ami, car je tiens pour vraie l'histoire de cette lettre perdue, je veux, dis-je, pour réparer la froideur que vous avez d'abord remarquée dans mon accueil, vous découvrir les secrets de notre politique. Le roi et le cardinal sont les meilleurs amis; leurs apparents démêlés ne sont que pour tromper les sots. Je ne prétends pas qu'un compatriote, un joli cavalier, un brave garçon, fait pour avancer, soit la dupe de toutes ces feintises et donne comme un niais dans le panneau, à la suite de tant d'autres qui s'y sont perdus. Songez bien que je suis dévoué à ces deux maîtres tout-puissants, et que jamais mes démarches sérieuses n'auront d'autre but que le service du roi et celui de M. le cardinal, un des plus illustres génies que la France ait produits. Maintenant, jeune homme, réglez-vous là-dessus, et si vous avez, soit de famille, soit par relations, soit d'instinct même, quelqu'une de ces inimitiés contre le cardinal telles que nous les voyons éclater chez les gentilshommes, dites-moi adieu, et quittons-nous. Je vous aiderai en mille circonstances, mais sans vous attacher à ma personne. J'espère que ma franchise, en tout cas, vous fera mon ami; car vous êtes jusqu'à présent le seul jeune homme à qui j'aie parlé comme je le fais.» 

Tréville se disait à part lui: 

«Si le cardinal m'a dépêché ce jeune renard, il n'aura certes pas manqué, lui qui sait à quel point je l'exècre, de dire à son espion que le meilleur moyen de me faire la cour est de me dire pis que pendre de lui; aussi, malgré mes protestations, le rusé compère va-t-il me répondre bien certainement qu'il a l'Éminence en horreur.» 

Il en fut tout autrement que s'y attendait Tréville; d'Artagnan répondit avec la plus grande simplicité: 

«Monsieur, j'arrive à Paris avec des intentions toutes semblables. Mon père m'a recommandé de ne souffrir rien du roi, de M. le cardinal et de vous, qu'il tient pour les trois premiers de France.» 

D'Artagnan ajoutait M. de Tréville aux deux autres, comme on peut s'en apercevoir, mais il pensait que cette adjonction ne devait rien gâter. 

«J'ai donc la plus grande vénération pour M. le cardinal, continua-t-il, et le plus profond respect pour ses actes. Tant mieux pour moi, monsieur, si vous me parlez, comme vous le dites, avec franchise; car alors vous me ferez l'honneur d'estimer cette ressemblance de goût; mais si vous avez eu quelque défiance, bien naturelle d'ailleurs, je sens que je me perds en disant la vérité; mais, tant pis, vous ne laisserez pas que de m'estimer, et c'est à quoi je tiens plus qu'à toute chose au monde.» 

M. de Tréville fut surpris au dernier point. Tant de pénétration, tant de franchise enfin, lui causait de l'admiration, mais ne levait pas entièrement ses doutes: plus ce jeune homme était supérieur aux autres jeunes gens, plus il était à redouter s'il se trompait. Néanmoins il serra la main à d'Artagnan, et lui dit: 

«Vous êtes un honnête garçon, mais dans ce moment je ne puis faire que ce que je vous ai offert tout à l'heure. Mon hôtel vous sera toujours ouvert. Plus tard, pouvant me demander à toute heure et par conséquent saisir toutes les occasions, vous obtiendrez probablement ce que vous désirez obtenir. 

\speak  C'est-à-dire, monsieur, reprit d'Artagnan, que vous attendez que je m'en sois rendu digne. Eh bien, soyez tranquille, ajouta-t-il avec la familiarité du Gascon, vous n'attendrez pas longtemps.» 

Et il salua pour se retirer, comme si désormais le reste le regardait. 

«Mais attendez donc, dit M. de Tréville en l'arrêtant, je vous ai promis une lettre pour le directeur de l'académie. Êtes-vous trop fier pour l'accepter, mon jeune gentilhomme? 

\speak  Non, monsieur, dit d'Artagnan; je vous réponds qu'il n'en sera pas de celle-ci comme de l'autre. Je la garderai si bien qu'elle arrivera, je vous le jure, à son adresse, et malheur à celui qui tenterait de me l'enlever!» 

M. de Tréville sourit à cette fanfaronnade, et, laissant son jeune compatriote dans l'embrasure de la fenêtre où ils se trouvaient et où ils avaient causé ensemble, il alla s'asseoir à une table et se mit à écrire la lettre de recommandation promise. Pendant ce temps, d'Artagnan, qui n'avait rien de mieux à faire, se mit à battre une marche contre les carreaux, regardant les mousquetaires qui s'en allaient les uns après les autres, et les suivant du regard jusqu'à ce qu'ils eussent disparu au tournant de la rue. 

M. de Tréville, après avoir écrit la lettre, la cacheta et, se levant, s'approcha du jeune homme pour la lui donner; mais au moment même où d'Artagnan étendait la main pour la recevoir, M. de Tréville fut bien étonné de voir son protégé faire un soubresaut, rougir de colère et s'élancer hors du cabinet en criant: 

«Ah! sangdieu! il ne m'échappera pas, cette fois. 

\speak  Et qui cela? demanda M. de Tréville. 

\speak  Lui, mon voleur! répondit d'Artagnan. Ah! traître!» 

Et il disparut. 

«Diable de fou! murmura M. de Tréville. À moins toutefois, ajouta-t-il, que ce ne soit une manière adroite de s'esquiver, en voyant qu'il a manqué son coup.»