%!TeX root=../musketeersfr.tex 

\chapter{Le Ménage Bonacieux} 
	
	\lettrine{C}{'était} la seconde fois que le cardinal revenait sur ce point des ferrets de diamants avec le roi. Louis XIII fut donc frappé de cette insistance, et pensa que cette recommandation cachait un mystère. 

Plus d'une fois le roi avait été humilié que le cardinal, dont la police, sans avoir atteint encore la perfection de la police moderne, était excellente, fût mieux instruit que lui-même de ce qui se passait dans son propre ménage. Il espéra donc, dans une conversation avec Anne d'Autriche, tirer quelque lumière de cette conversation et revenir ensuite près de Son Éminence avec quelque secret que le cardinal sût ou ne sût pas, ce qui, dans l'un ou l'autre cas, le rehaussait infiniment aux yeux de son ministre. 

Il alla donc trouver la reine, et, selon son habitude, l'aborda avec de nouvelles menaces contre ceux qui l'entouraient. Anne d'Autriche baissa la tête, laissa s'écouler le torrent sans répondre et espérant qu'il finirait par s'arrêter; mais ce n'était pas cela que voulait Louis XIII; Louis XIII voulait une discussion de laquelle jaillît une lumière quelconque, convaincu qu'il était que le cardinal avait quelque arrière-pensée et lui machinait une surprise terrible comme en savait faire Son Éminence. Il arriva à ce but par sa persistance à accuser. 

«Mais, s'écria Anne d'Autriche, lassée de ces vagues attaques; mais, Sire, vous ne me dites pas tout ce que vous avez dans le cœur. Qu'ai-je donc fait? Voyons, quel crime ai-je donc commis? Il est impossible que Votre Majesté fasse tout ce bruit pour une lettre écrite à mon frère.» 

Le roi, attaqué à son tour d'une manière si directe, ne sut que répondre; il pensa que c'était là le moment de placer la recommandation qu'il ne devait faire que la veille de la fête. 

«Madame, dit-il avec majesté, il y aura incessamment bal à l'hôtel de ville; j'entends que, pour faire honneur à nos braves échevins, vous y paraissiez en habit de cérémonie, et surtout parée des ferrets de diamants que je vous ai donnés pour votre fête. Voici ma réponse.» 

La réponse était terrible. Anne d'Autriche crut que Louis XIII savait tout, et que le cardinal avait obtenu de lui cette longue dissimulation de sept ou huit jours, qui était au reste dans son caractère. Elle devint excessivement pâle, appuya sur une console sa main d'une admirable beauté, et qui semblait alors une main de cire, et regardant le roi avec des yeux épouvantés, elle ne répondit pas une seule syllabe. 

«Vous entendez, madame, dit le roi, qui jouissait de cet embarras dans toute son étendue, mais sans en deviner la cause, vous entendez? 

\speak  Oui, Sire, j'entends, balbutia la reine. 

\speak  Vous paraîtrez à ce bal? 

\speak  Oui. 

\speak  Avec vos ferrets? 

\speak  Oui.» 

La pâleur de la reine augmenta encore, s'il était possible; le roi s'en aperçut, et en jouit avec cette froide cruauté qui était un des mauvais côtés de son caractère. 

«Alors, c'est convenu, dit le roi, et voilà tout ce que j'avais à vous dire. 

\speak  Mais quel jour ce bal aura-t-il lieu?» demanda Anne d'Autriche. 

Louis XIII sentit instinctivement qu'il ne devait pas répondre à cette question, la reine l'ayant faite d'une voix presque mourante. 

«Mais très incessamment, madame, dit-il; mais je ne me rappelle plus précisément la date du jour, je la demanderai au cardinal. 

\speak  C'est donc le cardinal qui vous a annoncé cette fête? s'écria la reine. 

\speak  Oui, madame, répondit le roi étonné; mais pourquoi cela? 

\speak  C'est lui, qui vous a dit de m'inviter à y paraître avec ces ferrets? 

\speak  C'est-à-dire, madame\dots 

\speak  C'est lui, Sire, c'est lui! 

\speak  Eh bien qu'importe que ce soit lui ou moi? y a-t-il un crime à cette invitation? 

\speak  Non, Sire. 

\speak  Alors vous paraîtrez? 

\speak  Oui, Sire. 

\speak  C'est bien, dit le roi en se retirant, c'est bien, j'y compte.» 

La reine fit une révérence, moins par étiquette que parce que ses genoux se dérobaient sous elle. 

Le roi partit enchanté. 

«Je suis perdue, murmura la reine, perdue, car le cardinal sait tout, et c'est lui qui pousse le roi, qui ne sait rien encore, mais qui saura tout bientôt. Je suis perdue! Mon Dieu! mon Dieu! mon Dieu!» 

Elle s'agenouilla sur un coussin et pria, la tête enfoncée entre ses bras palpitants. 

En effet, la position était terrible. Buckingham était retourné à Londres, Mme de Chevreuse était à Tours. Plus surveillée que jamais, la reine sentait sourdement qu'une de ses femmes la trahissait, sans savoir dire laquelle. La Porte ne pouvait pas quitter le Louvre. Elle n'avait pas une âme au monde à qui se fier. 

Aussi, en présence du malheur qui la menaçait et de l'abandon qui était le sien, éclata-t-elle en sanglots. 

«Ne puis-je donc être bonne à rien à Votre Majesté?» dit tout à coup une voix pleine de douceur et de pitié. 

La reine se retourna vivement, car il n'y avait pas à se tromper à l'expression de cette voix: c'était une amie qui parlait ainsi. 

En effet, à l'une des portes qui donnaient dans l'appartement de la reine apparut la jolie Mme Bonacieux; elle était occupée à ranger les robes et le linge dans un cabinet, lorsque le roi était entré; elle n'avait pas pu sortir, et avait tout entendu. 

La reine poussa un cri perçant en se voyant surprise, car dans son trouble elle ne reconnut pas d'abord la jeune femme qui lui avait été donnée par La Porte. 

«Oh! ne craignez rien, madame, dit la jeune femme en joignant les mains et en pleurant elle-même des angoisses de la reine; je suis à Votre Majesté corps et âme, et si loin que je sois d'elle, si inférieure que soit ma position, je crois que j'ai trouvé un moyen de tirer Votre Majesté de peine. 

\speak  Vous! ô Ciel! vous! s'écria la reine; mais voyons regardez-moi en face. Je suis trahie de tous côtés, puis-je me fier à vous? 

\speak  Oh! madame! s'écria la jeune femme en tombant à genoux: sur mon âme, je suis prête à mourir pour Votre Majesté!» 

Ce cri était sorti du plus profond du cœur, et, comme le premier, il n'y avait pas à se tromper. 

«Oui, continua Mme Bonacieux, oui, il y a des traîtres ici; mais, par le saint nom de la Vierge, je vous jure que personne n'est plus dévoué que moi à Votre Majesté. Ces ferrets que le roi redemande, vous les avez donnés au duc de Buckingham, n'est-ce pas? Ces ferrets étaient enfermés dans une petite boîte en bois de rose qu'il tenait sous son bras? Est-ce que je me trompe? Est-ce que ce n'est pas cela? 

\speak  Oh! mon Dieu! mon Dieu! murmura la reine dont les dents claquaient d'effroi. 

\speak  Eh bien, ces ferrets, continua Mme Bonacieux, il faut les ravoir. 

\speak  Oui, sans doute, il le faut, s'écria la reine; mais comment faire, comment y arriver? 

\speak  Il faut envoyer quelqu'un au duc. 

\speak  Mais qui?\dots qui?\dots à qui me fier? 

\speak  Ayez confiance en moi, madame; faites-moi cet honneur, ma reine, et je trouverai le messager, moi! 

\speak  Mais il faudra écrire! 

\speak  Oh! oui. C'est indispensable. Deux mots de la main de Votre Majesté et votre cachet particulier. 

\speak  Mais ces deux mots, c'est ma condamnation. C'est le divorce, l'exil! 

\speak  Oui, s'ils tombent entre des mains infâmes! Mais je réponds que ces deux mots seront remis à leur adresse. 

\speak  Oh! mon Dieu! il faut donc que je remette ma vie, mon honneur, ma réputation entre vos mains! 

\speak  Oui! oui, madame, il le faut, et je sauverai tout cela, moi! 

\speak  Mais comment? dites-le-moi au moins. 

\speak  Mon mari a été remis en liberté il y a deux ou trois jours; je n'ai pas encore eu le temps de le revoir. C'est un brave et honnête homme qui n'a ni haine, ni amour pour personne. Il fera ce que je voudrai: il partira sur un ordre de moi, sans savoir ce qu'il porte, et il remettra la lettre de Votre Majesté, sans même savoir qu'elle est de Votre Majesté, à l'adresse qu'elle indiquera.» 

La reine prit les deux mains de la jeune femme avec un élan passionné, la regarda comme pour lire au fond de son cœur, et ne voyant que sincérité dans ses beaux yeux, elle l'embrassa tendrement. 

«Fais cela, s'écria-t-elle, et tu m'auras sauvé la vie, tu m'auras sauvé l'honneur! 

\speak  Oh! n'exagérez pas le service que j'ai le bonheur de vous rendre; je n'ai rien à sauver à Votre Majesté, qui est seulement victime de perfides complots. 

\speak  C'est vrai, c'est vrai, mon enfant, dit la reine, et tu as raison. 

\speak  Donnez-moi donc cette lettre, madame, le temps presse.» 

La reine courut à une petite table sur laquelle se trouvaient encre, papier et plumes: elle écrivit deux lignes, cacheta la lettre de son cachet et la remit à Mme Bonacieux. 

«Et maintenant, dit la reine, nous oublions une chose nécessaire. 

\speak  Laquelle? 

\speak  L'argent.» 

Mme Bonacieux rougit. 

«Oui, c'est vrai, dit-elle, et j'avouerai à Votre Majesté que mon mari\dots 

\speak  Ton mari n'en a pas, c'est cela que tu veux dire. 

\speak  Si fait, il en a, mais il est fort avare, c'est là son défaut. Cependant, que Votre Majesté ne s'inquiète pas, nous trouverons moyen\dots 

\speak  C'est que je n'en ai pas non plus, dit la reine (ceux qui liront les Mémoires de Mme de Motteville ne s'étonneront pas de cette réponse); mais, attends.» 

Anne d'Autriche courut à son écrin. 

«Tiens, dit-elle, voici une bague d'un grand prix à ce qu'on assure; elle vient de mon frère le roi d'Espagne, elle est à moi et j'en puis disposer. Prends cette bague et fais-en de l'argent, et que ton mari parte. 

\speak  Dans une heure vous serez obéie. 

\speak  Tu vois l'adresse, ajouta la reine, parlant si bas qu'à peine pouvait-on entendre ce qu'elle disait: à Milord duc de Buckingham, à Londres. 

\speak  La lettre sera remise à lui-même. 

\speak  Généreuse enfant!» s'écria Anne d'Autriche. 

Mme Bonacieux baisa les mains de la reine, cacha le papier dans son corsage et disparut avec la légèreté d'un oiseau. 

Dix minutes après, elle était chez elle; comme elle l'avait dit à la reine, elle n'avait pas revu son mari depuis sa mise en liberté; elle ignorait donc le changement qui s'était fait en lui à l'endroit du cardinal, changement qu'avaient opéré la flatterie et l'argent de Son Éminence et qu'avaient corroboré, depuis, deux ou trois visites du comte de Rochefort, devenu le meilleur ami de Bonacieux, auquel il avait fait croire sans beaucoup de peine qu'aucun sentiment coupable n'avait amené l'enlèvement de sa femme, mais que c'était seulement une précaution politique. 

Elle trouva M. Bonacieux seul: le pauvre homme remettait à grand-peine de l'ordre dans la maison, dont il avait trouvé les meubles à peu près brisés et les armoires à peu près vides, la justice n'étant pas une des trois choses que le roi Salomon indique comme ne laissant point de traces de leur passage. Quant à la servante, elle s'était enfuie lors de l'arrestation de son maître. La terreur avait gagné la pauvre fille au point qu'elle n'avait cessé de marcher de Paris jusqu'en Bourgogne, son pays natal. 

Le digne mercier avait, aussitôt sa rentrée dans sa maison, fait part à sa femme de son heureux retour, et sa femme lui avait répondu pour le féliciter et pour lui dire que le premier moment qu'elle pourrait dérober à ses devoirs serait consacré tout entier à lui rendre visite. 

Ce premier moment s'était fait attendre cinq jours, ce qui, dans toute autre circonstance, eût paru un peu bien long à maître Bonacieux; mais il avait, dans la visite qu'il avait faite au cardinal et dans les visites que lui faisait Rochefort, ample sujet à réflexion, et, comme on sait, rien ne fait passer le temps comme de réfléchir. 

D'autant plus que les réflexions de Bonacieux étaient toutes couleur de rose. Rochefort l'appelait son ami, son cher Bonacieux, et ne cessait de lui dire que le cardinal faisait le plus grand cas de lui. Le mercier se voyait déjà sur le chemin des honneurs et de la fortune. 

De son côté, Mme Bonacieux avait réfléchi, mais, il faut le dire, à tout autre chose que l'ambition; malgré elle, ses pensées avaient eu pour mobile constant ce beau jeune homme si brave et qui paraissait si amoureux. Mariée à dix-huit ans à M. Bonacieux, ayant toujours vécu au milieu des amis de son mari, peu susceptibles d'inspirer un sentiment quelconque à une jeune femme dont le cœur était plus élevé que sa position, Mme Bonacieux était restée insensible aux séductions vulgaires; mais, à cette époque surtout, le titre de gentilhomme avait une grande influence sur la bourgeoisie, et d'Artagnan était gentilhomme; de plus, il portait l'uniforme des gardes, qui, après l'uniforme des mousquetaires, était le plus apprécié des dames. Il était, nous le répétons, beau, jeune, aventureux; il parlait d'amour en homme qui aime et qui a soif d'être aimé; il y en avait là plus qu'il n'en fallait pour tourner une tête de vingt-trois ans, et Mme Bonacieux en était arrivée juste à cet âge heureux de la vie. 

Les deux époux, quoiqu'ils ne se fussent pas vus depuis plus de huit jours, et que pendant cette semaine de graves événements eussent passé entre eux, s'abordèrent donc avec une certaine préoccupation; néanmoins, M. Bonacieux manifesta une joie réelle et s'avança vers sa femme à bras ouverts. 

Mme Bonacieux lui présenta le front. 

«Causons un peu, dit-elle. 

\speak  Comment? dit Bonacieux étonné. 

\speak  Oui, sans doute, j'ai une chose de la plus haute importance à vous dire. 

\speak  Au fait, et moi aussi, j'ai quelques questions assez sérieuses à vous adresser. Expliquez-moi un peu votre enlèvement, je vous prie. 

\speak  Il ne s'agit point de cela pour le moment, dit Mme Bonacieux. 

\speak  Et de quoi s'agit-il donc? de ma captivité? 

\speak  Je l'ai apprise le jour même; mais comme vous n'étiez coupable d'aucun crime, comme vous n'étiez complice d'aucune intrigue, comme vous ne saviez rien enfin qui pût vous compromettre, ni vous, ni personne, je n'ai attaché à cet événement que l'importance qu'il méritait. 

\speak  Vous en parlez bien à votre aise, madame! reprit Bonacieux blessé du peu d'intérêt que lui témoignait sa femme; savez-vous que j'ai été plongé un jour et une nuit dans un cachot de la Bastille? 

\speak  Un jour et une nuit sont bientôt passés; laissons donc votre captivité, et revenons à ce qui m'amène près de vous. 

\speak  Comment? ce qui vous amène près de moi! N'est-ce donc pas le désir de revoir un mari dont vous êtes séparée depuis huit jours? demanda le mercier piqué au vif. 

\speak  C'est cela d'abord, et autre chose ensuite. 

\speak  Parlez! 

\speak  Une chose du plus haut intérêt et de laquelle dépend notre fortune à venir peut-être. 

\speak  Notre fortune a fort changé de face depuis que je vous ai vue, madame Bonacieux, et je ne serais pas étonné que d'ici à quelques mois elle ne fît envie à beaucoup de gens. 

\speak  Oui, surtout si vous voulez suivre les instructions que je vais vous donner. 

\speak  À moi? 

\speak  Oui, à vous. Il y a une bonne et sainte action à faire, monsieur, et beaucoup d'argent à gagner en même temps.» 

Mme Bonacieux savait qu'en parlant d'argent à son mari, elle le prenait par son faible. 

Mais un homme, fût-ce un mercier, lorsqu'il a causé dix minutes avec le cardinal de Richelieu, n'est plus le même homme. 

«Beaucoup d'argent à gagner! dit Bonacieux en allongeant les lèvres. 

\speak  Oui, beaucoup. 

\speak  Combien, à peu près? 

\speak  Mille pistoles peut-être. 

\speak  Ce que vous avez à me demander est donc bien grave? 

\speak  Oui. 

\speak  Que faut-il faire? 

\speak  Vous partirez sur-le-champ, je vous remettrai un papier dont vous ne vous dessaisirez sous aucun prétexte, et que vous remettrez en main propre. 

\speak  Et pour où partirai-je? 

\speak  Pour Londres. 

\speak  Moi, pour Londres! Allons donc, vous raillez, je n'ai pas affaire à Londres. 

\speak  Mais d'autres ont besoin que vous y alliez. 

\speak  Quels sont ces autres? Je vous avertis, je ne fais plus rien en aveugle, et je veux savoir non seulement à quoi je m'expose, mais encore pour qui je m'expose. 

\speak  Une personne illustre vous envoie, une personne illustre vous attend: la récompense dépassera vos désirs, voilà tout ce que je puis vous promettre. 

\speak  Des intrigues encore, toujours des intrigues! merci, je m'en défie maintenant, et M. le cardinal m'a éclairé là-dessus. 

\speak  Le cardinal! s'écria Mme Bonacieux, vous avez vu le cardinal? 

\speak  Il m'a fait appeler, répondit fièrement le mercier. 

\speak  Et vous vous êtes rendu à son invitation, imprudent que vous êtes. 

\speak  Je dois dire que je n'avais pas le choix de m'y rendre ou de ne pas m'y rendre, car j'étais entre deux gardes. Il est vrai encore de dire que, comme alors je ne connaissais pas Son Éminence, si j'avais pu me dispenser de cette visite, j'en eusse été fort enchanté. 

\speak  Il vous a donc maltraité? il vous a donc fait des menaces? 

\speak  Il m'a tendu la main et m'a appelé son ami, --- son ami! entendez-vous, madame? --- je suis l'ami du grand cardinal! 

\speak  Du grand cardinal! 

\speak  Lui contesteriez-vous ce titre, par hasard, madame? 

\speak  Je ne lui conteste rien, mais je vous dis que la faveur d'un ministre est éphémère, et qu'il faut être fou pour s'attacher à un ministre; il est des pouvoirs au-dessus du sien, qui ne reposent pas sur le caprice d'un homme ou l'issue d'un événement; c'est à ces pouvoirs qu'il faut se rallier. 

\speak  J'en suis fâché, madame, mais je ne connais pas d'autre pouvoir que celui du grand homme que j'ai l'honneur de servir. 

\speak  Vous servez le cardinal? 

\speak  Oui, madame, et comme son serviteur je ne permettrai pas que vous vous livriez à des complots contre la sûreté de l'État, et que vous serviez, vous, les intrigues d'une femme qui n'est pas française et qui a le cœur espagnol. Heureusement, le grand cardinal est là, son regard vigilant surveille et pénètre jusqu'au fond du cœur.» 

Bonacieux répétait mot pour mot une phrase qu'il avait entendu dire au comte de Rochefort; mais la pauvre femme, qui avait compté sur son mari et qui, dans cet espoir, avait répondu de lui à la reine, n'en frémit pas moins, et du danger dans lequel elle avait failli se jeter, et de l'impuissance dans laquelle elle se trouvait. Cependant connaissant la faiblesse et surtout la cupidité de son mari elle ne désespérait pas de l'amener à ses fins. 

«Ah! vous êtes cardinaliste, monsieur, s'écria-t-elle ah! vous servez le parti de ceux qui maltraitent votre femme et qui insultent votre reine! 

\speak  Les intérêts particuliers ne sont rien devant les intérêts de tous. Je suis pour ceux qui sauvent l'État», dit avec emphase Bonacieux. 

C'était une autre phrase du comte de Rochefort, qu'il avait retenue et qu'il trouvait l'occasion de placer. 

«Et savez-vous ce que c'est que l'État dont vous parlez? dit Mme Bonacieux en haussant les épaules. Contentez-vous d'être un bourgeois sans finesse aucune, et tournez-vous du côté qui vous offre le plus d'avantages. 

\speak  Eh! eh! dit Bonacieux en frappant sur un sac à la panse arrondie et qui rendit un son argentin; que dites-vous de ceci, madame la prêcheuse? 

\speak  D'où vient cet argent? 

\speak  Vous ne devinez pas? 

\speak  Du cardinal? 

\speak  De lui et de mon ami le comte de Rochefort. 

\speak  Le comte de Rochefort! mais c'est lui qui m'a enlevée! 

\speak  Cela se peut, madame. 

\speak  Et vous recevez de l'argent de cet homme? 

\speak  Ne m'avez-vous pas dit que cet enlèvement était tout politique? 

\speak  Oui; mais cet enlèvement avait pour but de me faire trahir ma maîtresse, de m'arracher par des tortures des aveux qui pussent compromettre l'honneur et peut-être la vie de mon auguste maîtresse. 

\speak  Madame, reprit Bonacieux, votre auguste maîtresse est une perfide Espagnole, et ce que le cardinal fait est bien fait. 

\speak  Monsieur, dit la jeune femme, je vous savais lâche, avare et imbécile, mais je ne vous savais pas infâme! 

\speak  Madame, dit Bonacieux, qui n'avait jamais vu sa femme en colère, et qui reculait devant le courroux conjugal; madame, que dites-vous donc? 

\speak  Je dis que vous êtes un misérable! continua Mme Bonacieux, qui vit qu'elle reprenait quelque influence sur son mari. Ah! vous faites de la politique, vous! et de la politique cardinaliste encore! Ah! vous vous vendez, corps et âme, au démon pour de l'argent. 

\speak  Non, mais au cardinal. 

\speak  C'est la même chose! s'écria la jeune femme. Qui dit Richelieu, dit Satan. 

\speak  Taisez-vous, madame, taisez-vous, on pourrait vous entendre! 

\speak  Oui, vous avez raison, et je serais honteuse pour vous de votre lâcheté. 

\speak  Mais qu'exigez-vous donc de moi? voyons! 

\speak  Je vous l'ai dit: que vous partiez à l'instant même, monsieur, que vous accomplissiez loyalement la commission dont je daigne vous charger, et à cette condition j'oublie tout, je pardonne, et il y a plus --- elle lui tendit la main --- je vous rends mon amitié.» 

Bonacieux était poltron et avare; mais il aimait sa femme: il fut attendri. Un homme de cinquante ans ne tient pas longtemps rancune à une femme de vingt-trois. Mme Bonacieux vit qu'il hésitait: 

«Allons, êtes-vous décidé? dit-elle. 

\speak  Mais, ma chère amie, réfléchissez donc un peu à ce que vous exigez de moi; Londres est loin de Paris, fort loin, et peut-être la commission dont vous me chargez n'est-elle pas sans dangers. 

\speak  Qu'importe, si vous les évitez! 

\speak  Tenez, madame Bonacieux, dit le mercier, tenez, décidément, je refuse: les intrigues me font peur. J'ai vu la Bastille, moi. Brrrrou! c'est affreux, la Bastille! Rien que d'y penser, j'en ai la chair de poule. On m'a menacé de la torture. Savez-vous ce que c'est que la torture? Des coins de bois qu'on vous enfonce entre les jambes jusqu'à ce que les os éclatent! Non, décidément, je n'irai pas. Et morbleu! que n'y allez-vous vous-même? car, en vérité, je crois que je me suis trompé sur votre compte jusqu'à présent: je crois que vous êtes un homme, et des plus enragés encore! 

\speak  Et vous, vous êtes une femme, une misérable femme, stupide et abrutie. Ah! vous avez peur! Eh bien, si vous ne partez pas à l'instant même, je vous fais arrêter par l'ordre de la reine, et je vous fais mettre à cette Bastille que vous craignez tant.» 

Bonacieux tomba dans une réflexion profonde, il pesa mûrement les deux colères dans son cerveau, celle du cardinal et celle de la reine: celle du cardinal l'emporta énormément. 

«Faites-moi arrêter de la part de la reine, dit-il, et moi je me réclamerai de Son Éminence.» 

Pour le coup, Mme Bonacieux vit qu'elle avait été trop loin, et elle fut épouvantée de s'être si fort avancée. Elle contempla un instant avec effroi cette figure stupide, d'une résolution invincible, comme celle des sots qui ont peur. 

«Eh bien, soit! dit-elle. Peut-être, au bout du compte, avez-vous raison: un homme en sait plus long que les femmes en politique, et vous surtout, monsieur Bonacieux, qui avez causé avec le cardinal. Et cependant, il est bien dur, ajouta-t-elle, que mon mari, un homme sur l'affection duquel je croyais pouvoir compter, me traite aussi disgracieusement et ne satisfasse point à ma fantaisie. 

\speak  C'est que vos fantaisies peuvent mener trop loin, reprit Bonacieux triomphant, et je m'en défie. 

\speak  J'y renoncerai donc, dit la jeune femme en soupirant; c'est bien, n'en parlons plus. 

\speak  Si, au moins, vous me disiez quelle chose je vais faire à Londres, reprit Bonacieux, qui se rappelait un peu tard que Rochefort lui avait recommandé d'essayer de surprendre les secrets de sa femme. 

\speak  Il est inutile que vous le sachiez, dit la jeune femme, qu'une défiance instinctive repoussait maintenant en arrière: il s'agissait d'une bagatelle comme en désirent les femmes, d'une emplette sur laquelle il y avait beaucoup à gagner.» 

Mais plus la jeune femme se défendait, plus au contraire Bonacieux pensa que le secret qu'elle refusait de lui confier était important. Il résolut donc de courir à l'instant même chez le comte de Rochefort, et de lui dire que la reine cherchait un messager pour l'envoyer à Londres. 

«Pardon, si je vous quitte, ma chère madame Bonacieux, dit-il; mais, ne sachant pas que vous me viendriez voir, j'avais pris rendez-vous avec un de mes amis, je reviens à l'instant même, et si vous voulez m'attendre seulement une demi-minute, aussitôt que j'en aurai fini avec cet ami, je reviens vous prendre, et, comme il commence à se faire tard, je vous reconduis au Louvre. 

\speak  Merci, monsieur, répondit Mme Bonacieux: vous n'êtes point assez brave pour m'être d'une utilité quelconque, et je m'en retournerai bien au Louvre toute seule. 

\speak  Comme il vous plaira, madame Bonacieux, reprit l'ex-mercier. Vous reverrai-je bientôt? 

\speak  Sans doute; la semaine prochaine, je l'espère, mon service me laissera quelque liberté, et j'en profiterai pour revenir mettre de l'ordre dans nos affaires, qui doivent être quelque peu dérangées. 

\speak  C'est bien; je vous attendrai. Vous ne m'en voulez pas? 

\speak  Moi! pas le moins du monde. 

\speak  À bientôt, alors? 

\speak  À bientôt.» 

Bonacieux baisa la main de sa femme, et s'éloigna rapidement. 

«Allons, dit Mme Bonacieux, lorsque son mari eut refermé la porte de la rue, et qu'elle se trouva seule, il ne manquait plus à cet imbécile que d'être cardinaliste! Et moi qui avais répondu à la reine, moi qui avais promis à ma pauvre maîtresse\dots Ah! mon Dieu, mon Dieu! elle va me prendre pour quelqu'une de ces misérables dont fourmille le palais, et qu'on a placées près d'elle pour l'espionner! Ah! monsieur Bonacieux! je ne vous ai jamais beaucoup aimé; maintenant, c'est bien pis: je vous hais! et, sur ma parole, vous me le paierez!» 

Au moment où elle disait ces mots, un coup frappé au plafond lui fit lever la tête, et une voix, qui parvint à elle à travers le plancher, lui cria: 

«Chère madame Bonacieux, ouvrez-moi la petite porte de l'allée, et je vais descendre près de vous.» 