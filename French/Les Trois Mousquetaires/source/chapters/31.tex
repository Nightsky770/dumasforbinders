%!TeX root=../musketeersfr.tex 

\chapter{Anglais Et Français} 

\lettrine{L}{'heure} venue, on se rendit avec les quatre laquais, derrière le Luxembourg, dans un enclos abandonné aux chèvres. Athos donna une pièce de monnaie au chevrier pour qu'il s'écartât. Les laquais furent chargés de faire sentinelle. 

Bientôt une troupe silencieuse s'approcha du même enclos, y pénétra et joignit les mousquetaires; puis, selon les habitudes d'outre-mer, les présentations eurent lieu. 

Les Anglais étaient tous gens de la plus haute qualité, les noms bizarres de leurs adversaires furent donc pour eux un sujet non seulement de surprise, mais encore d'inquiétude. 

«Mais, avec tout cela, dit Lord de Winter quand les trois amis eurent été nommés, nous ne savons pas qui vous êtes, et nous ne nous battrons pas avec des noms pareils; ce sont des noms de bergers, cela. 

\speak  Aussi, comme vous le supposez bien, Milord, ce sont de faux noms, dit Athos. 

\speak  Ce qui ne nous donne qu'un plus grand désir de connaître les noms véritables, répondit l'Anglais. 

\speak  Vous avez bien joué contre nous sans les connaître, dit Athos, à telles enseignes que vous nous avez gagné nos deux chevaux? 

\speak  C'est vrai, mais nous ne risquions que nos pistoles; cette fois nous risquons notre sang: on joue avec tout le monde, on ne se bat qu'avec ses égaux. 

\speak  C'est juste», dit Athos. Et il prit à l'écart celui des quatre Anglais avec lequel il devait se battre, et lui dit son nom tout bas. 

Porthos et Aramis en firent autant de leur côté. 

«Cela vous suffit-il, dit Athos à son adversaire, et me trouvez-vous assez grand seigneur pour me faire la grâce de croiser l'épée avec moi? 

\speak  Oui, monsieur, dit l'Anglais en s'inclinant. 

\speak  Eh bien, maintenant, voulez-vous que je vous dise une chose? reprit froidement Athos. 

\speak  Laquelle? demanda l'Anglais. 

\speak  C'est que vous auriez aussi bien fait de ne pas exiger que je me fisse connaître. 

\speak  Pourquoi cela? 

\speak  Parce qu'on me croit mort, que j'ai des raisons pour désirer qu'on ne sache pas que je vis, et que je vais être obligé de vous tuer, pour que mon secret ne coure pas les champs.» 

L'Anglais regarda Athos, croyant que celui-ci plaisantait; mais Athos ne plaisantait pas le moins du monde. 

«Messieurs, dit-il en s'adressant à la fois à ses compagnons et à leurs adversaires, y sommes-nous? 

\speak  Oui, répondirent tout d'une voix Anglais et Français. 

\speak  Alors, en garde», dit Athos. 

Et aussitôt huit épées brillèrent aux rayons du soleil couchant, et le combat commença avec un acharnement bien naturel entre gens deux fois ennemis. 

Athos s'escrimait avec autant de calme et de méthode que s'il eût été dans une salle d'armes. 

Porthos, corrigé sans doute de sa trop grande confiance par son aventure de Chantilly, jouait un jeu plein de finesse et de prudence. 

Aramis, qui avait le troisième chant de son poème à finir, se dépêchait en homme très pressé. 

Athos, le premier, tua son adversaire: il ne lui avait porté qu'un coup, mais, comme il l'en avait prévenu, le coup avait été mortel. L'épée lui traversa le cœur. 

Porthos, le second, étendit le sien sur l'herbe: il lui avait percé la cuisse. Alors, comme l'Anglais, sans faire plus longue résistance, lui avait rendu son épée, Porthos le prit dans ses bras et le porta dans son carrosse. 

Aramis poussa le sien si vigoureusement, qu'après avoir rompu une cinquantaine de pas, il finit par prendre la fuite à toutes jambes et disparut aux huées des laquais. 

Quant à d'Artagnan, il avait joué purement et simplement un jeu défensif; puis, lorsqu'il avait vu son adversaire bien fatigué, il lui avait, d'une vigoureuse flanconade, fait sauter son épée. Le baron, se voyant désarmé, fit deux ou trois pas en arrière; mais, dans ce mouvement, son pied glissa, et il tomba à la renverse. 

D'Artagnan fut sur lui d'un seul bond, et lui portant l'épée à la gorge: 

«Je pourrais vous tuer, monsieur, dit-il à l'Anglais, et vous êtes bien entre mes mains, mais je vous donne la vie pour l'amour de votre soeur.» 

D'Artagnan était au comble de la joie; il venait de réaliser le plan qu'il avait arrêté d'avance, et dont le développement avait fait éclore sur son visage les sourires dont nous avons parlé. 

L'Anglais, enchanté d'avoir affaire à un gentilhomme d'aussi bonne composition, serra d'Artagnan entre ses bras, fit mille caresses aux trois mousquetaires, et, comme l'adversaire de Porthos était déjà installé dans la voiture et que celui d'Aramis avait pris la poudre d'escampette, on ne songea plus qu'au défunt. 

Comme Porthos et Aramis le déshabillaient dans l'espérance que sa blessure n'était pas mortelle, une grosse bourse s'échappa de sa ceinture. D'Artagnan la ramassa et la tendit à Lord de Winter. 

«Et que diable voulez-vous que je fasse de cela? dit l'Anglais. 

\speak  Vous la rendrez à sa famille, dit d'Artagnan. 

\speak  Sa famille se soucie bien de cette misère: elle hérite de quinze mille louis de rente: gardez cette bourse pour vos laquais.» 

D'Artagnan mit la bourse dans sa poche. 

«Et maintenant, mon jeune ami, car vous me permettrez, je l'espère, de vous donner ce nom, dit Lord de Winter, dès ce soir, si vous le voulez bien, je vous présenterai à ma soeur, Lady Clarick; car je veux qu'elle vous prenne à son tour dans ses bonnes grâces, et, comme elle n'est point tout à fait mal en cour, peut-être dans l'avenir un mot dit par elle ne vous serait-il point inutile.» 

D'Artagnan rougit de plaisir, et s'inclina en signe d'assentiment. 

Pendant ce temps, Athos s'était approché de d'Artagnan. 

«Que voulez-vous faire de cette bourse? lui dit-il tout bas à l'oreille. 

\speak  Mais je comptais vous la remettre, mon cher Athos. 

\speak  À moi? et pourquoi cela? 

\speak  Dame, vous l'avez tué: ce sont les dépouilles opimes. 

\speak  Moi, héritier d'un ennemi! dit Athos, pour qui donc me prenez-vous? 

\speak  C'est l'habitude à la guerre, dit d'Artagnan; pourquoi ne serait-ce pas l'habitude dans un duel? 

\speak  Même sur le champ de bataille, dit Athos, je n'ai jamais fait cela.» 

Porthos leva les épaules. Aramis, d'un mouvement de lèvres, approuva Athos. 

«Alors, dit d'Artagnan, donnons cet argent aux laquais, comme Lord de Winter nous a dit de le faire. 

\speak  Oui, dit Athos, donnons cette bourse, non à nos laquais, mais aux laquais anglais.» 

Athos prit la bourse, et la jeta dans la main du cocher: 

«Pour vous et vos camarades.» 

Cette grandeur de manières dans un homme entièrement dénué frappa Porthos lui-même, et cette générosité française, redite par Lord de Winter et son ami, eut partout un grand succès, excepté auprès de MM. Grimaud, Mousqueton, Planchet et Bazin. 

Lord de Winter, en quittant d'Artagnan, lui donna l'adresse de sa soeur; elle demeurait place Royale, qui était alors le quartier à la mode, au n° 6. D'ailleurs, il s'engageait à le venir prendre pour le présenter. D'Artagnan lui donna rendez-vous à huit heures, chez Athos. 

Cette présentation à Milady occupait fort la tête de notre Gascon. Il se rappelait de quelle façon étrange cette femme avait été mêlée jusque-là dans sa destinée. Selon sa conviction, c'était quelque créature du cardinal, et cependant il se sentait invinciblement entraîné vers elle, par un de ces sentiments dont on ne se rend pas compte. Sa seule crainte était que Milady ne reconnût en lui l'homme de Meung et de Douvres. Alors, elle saurait qu'il était des amis de M. de Tréville, et par conséquent qu'il appartenait corps et âme au roi, ce qui, dès lors, lui ferait perdre une partie de ses avantages, puisque, connu de Milady comme il la connaissait, il jouerait avec elle à jeu égal. Quant à ce commencement d'intrigue entre elle et le comte de Wardes, notre présomptueux ne s'en préoccupait que médiocrement, bien que le marquis fût jeune, beau, riche et fort avant dans la faveur du cardinal. Ce n'est pas pour rien que l'on a vingt ans, et surtout que l'on est né à Tarbes. 

D'Artagnan commença par aller faire chez lui une toilette flamboyante; puis, il s'en revint chez Athos, et, selon son habitude, lui raconta tout. Athos écouta ses projets; puis il secoua la tête, et lui recommanda la prudence avec une sorte d'amertume. 

«Quoi! lui dit-il, vous venez de perdre une femme que vous disiez bonne, charmante, parfaite, et voilà que vous courez déjà après une autre!» 

D'Artagnan sentit la vérité de ce reproche. 

«J'aimais Mme Bonacieux avec le cœur, tandis que j'aime Milady avec la tête, dit-il; en me faisant conduire chez elle, je cherche surtout à m'éclairer sur le rôle qu'elle joue à la cour. 

\speak  Le rôle qu'elle joue, pardieu! il n'est pas difficile à deviner d'après tout ce que vous m'avez dit. C'est quelque émissaire du cardinal: une femme qui vous attirera dans un piège, où vous laisserez votre tête tout bonnement. 

\speak  Diable! mon cher Athos, vous voyez les choses bien en noir, ce me semble. 

\speak  Mon cher, je me défie des femmes; que voulez-vous! je suis payé pour cela, et surtout des femmes blondes. Milady est blonde, m'avez-vous dit? 

\speak  Elle a les cheveux du plus beau blond qui se puisse voir. 

\speak  Ah! mon pauvre d'Artagnan, fit Athos. 

\speak  Écoutez, je veux m'éclairer; puis, quand je saurai ce que je désire savoir, je m'éloignerai. 

\speak  Éclairez-vous», dit flegmatiquement Athos. 

Lord de Winter arriva à l'heure dite, mais Athos, prévenu à temps, passa dans la seconde pièce. Il trouva donc d'Artagnan seul, et, comme il était près de huit heures, il emmena le jeune homme. 

Un élégant carrosse attendait en bas, et comme il était attelé de deux excellents chevaux, en un instant on fut place Royale. 

Milady Clarick reçut gracieusement d'Artagnan. Son hôtel était d'une somptuosité remarquable; et, bien que la plupart des Anglais, chassés par la guerre, quittassent la France, ou fussent sur le point de la quitter, Milady venait de faire faire chez elle de nouvelles dépenses: ce qui prouvait que la mesure générale qui renvoyait les Anglais ne la regardait pas. 

«Vous voyez, dit Lord de Winter en présentant d'Artagnan à sa soeur, un jeune gentilhomme qui a tenu ma vie entre ses mains, et qui n'a point voulu abuser de ses avantages, quoique nous fussions deux fois ennemis, puisque c'est moi qui l'ai insulté, et que je suis anglais. Remerciez-le donc, madame, si vous avez quelque amitié pour moi.» 

Milady fronça légèrement le sourcil; un nuage à peine visible passa sur son front, et un sourire tellement étrange apparut sur ses lèvres, que le jeune homme, qui vit cette triple nuance, en eut comme un frisson. 

Le frère ne vit rien; il s'était retourné pour jouer avec le singe favori de Milady, qui l'avait tiré par son pourpoint. 

«Soyez le bienvenu, monsieur, dit Milady d'une voix dont la douceur singulière contrastait avec les symptômes de mauvaise humeur que venait de remarquer d'Artagnan, vous avez acquis aujourd'hui des droits éternels à ma reconnaissance.» 

L'Anglais alors se retourna et raconta le combat sans omettre un détail. Milady l'écouta avec la plus grande attention; cependant on voyait facilement, quelque effort qu'elle fît pour cacher ses impressions, que ce récit ne lui était point agréable. Le sang lui montait à la tête, et son petit pied s'agitait impatiemment sous sa robe. 

Lord de Winter ne s'aperçut de rien. Puis, lorsqu'il eut fini, il s'approcha d'une table où étaient servis sur un plateau une bouteille de vin d'Espagne et des verres. Il emplit deux verres et d'un signe invita d'Artagnan à boire. 

D'Artagnan savait que c'était fort désobliger un Anglais que de refuser de toaster avec lui. Il s'approcha donc de la table, et prit le second verre. Cependant il n'avait point perdu de vue Milady, et dans la glace il s'aperçut du changement qui venait de s'opérer sur son visage. Maintenant qu'elle croyait n'être plus regardée, un sentiment qui ressemblait à de la férocité animait sa physionomie. Elle mordait son mouchoir à belles dents. 

Cette jolie petite soubrette, que d'Artagnan avait déjà remarquée, entra alors; elle dit en anglais quelques mots à Lord de Winter, qui demanda aussitôt à d'Artagnan la permission de se retirer, s'excusant sur l'urgence de l'affaire qui l'appelait, et chargeant sa soeur d'obtenir son pardon. 

D'Artagnan échangea une poignée de main avec Lord de Winter et revint près de Milady. Le visage de cette femme, avec une mobilité surprenante, avait repris son expression gracieuse, seulement quelques petites taches rouges disséminées sur son mouchoir indiquaient qu'elle s'était mordu les lèvres jusqu'au sang. 

Ses lèvres étaient magnifiques, on eût dit du corail. 

La conversation prit une tournure enjouée. Milady paraissait s'être entièrement remise. Elle raconta que Lord de Winter n'était que son beau-frère et non son frère: elle avait épousé un cadet de famille qui l'avait laissée veuve avec un enfant. Cet enfant était le seul héritier de Lord de Winter, si Lord de Winter ne se mariait point. Tout cela laissait voir à d'Artagnan un voile qui enveloppait quelque chose, mais il ne distinguait pas encore sous ce voile. 

Au reste, au bout d'une demi-heure de conversation, d'Artagnan était convaincu que Milady était sa compatriote: elle parlait le français avec une pureté et une élégance qui ne laissaient aucun doute à cet égard. 

D'Artagnan se répandit en propos galants et en protestations de dévouement. À toutes les fadaises qui échappèrent à notre Gascon, Milady sourit avec bienveillance. L'heure de se retirer arriva. D'Artagnan prit congé de Milady et sortit du salon le plus heureux des hommes. 

Sur l'escalier il rencontra la jolie soubrette, laquelle le frôla doucement en passant, et, tout en rougissant jusqu'aux yeux, lui demanda pardon de l'avoir touché, d'une voix si douce, que le pardon lui fut accordé à l'instant même. 

D'Artagnan revint le lendemain et fut reçu encore mieux que la veille. Lord de Winter n'y était point, et ce fut Milady qui lui fit cette fois tous les honneurs de la soirée. Elle parut prendre un grand intérêt à lui, lui demanda d'où il était, quels étaient ses amis, et s'il n'avait pas pensé quelquefois à s'attacher au service de M. le cardinal. 

D'Artagnan, qui, comme on le sait, était fort prudent pour un garçon de vingt ans, se souvint alors de ses soupçons sur Milady; il lui fit un grand éloge de Son Éminence, lui dit qu'il n'eût point manqué d'entrer dans les gardes du cardinal au lieu d'entrer dans les gardes du roi, s'il eût connu par exemple M. de Cavois au lieu de connaître M. de Tréville. 

Milady changea de conversation sans affectation aucune, et demanda à d'Artagnan de la façon la plus négligée du monde s'il n'avait jamais été en Angleterre. 

D'Artagnan répondit qu'il y avait été envoyé par M. de Tréville pour traiter d'une remonte de chevaux et qu'il en avait même ramené quatre comme échantillon. 

Milady, dans le cours de la conversation, se pinça deux ou trois fois les lèvres: elle avait affaire à un Gascon qui jouait serré. 

À la même heure que la veille d'Artagnan se retira. Dans le corridor il rencontra encore la jolie Ketty; c'était le nom de la soubrette. Celle-ci le regarda avec une expression de mystérieuse bienveillance à laquelle il n'y avait point à se tromper. Mais d'Artagnan était si préoccupé de la maîtresse, qu'il ne remarquait absolument que ce qui venait d'elle. 

D'Artagnan revint chez Milady le lendemain et le surlendemain, et chaque fois Milady lui fit un accueil plus gracieux. 

Chaque fois aussi, soit dans l'antichambre, soit dans le corridor, soit sur l'escalier, il rencontrait la jolie soubrette. 

Mais, comme nous l'avons dit, d'Artagnan ne faisait aucune attention à cette persistance de la pauvre Ketty.