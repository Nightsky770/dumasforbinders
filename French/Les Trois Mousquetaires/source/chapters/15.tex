%!TeX root=../musketeersfr.tex 

\chapter{Gens De Robe Et Gens D'Épée}

\lettrine{L}{e} lendemain du jour où ces événements étaient arrivés, Athos n'ayant point reparu, M. de Tréville avait été prévenu par d'Artagnan et par Porthos de sa disparition. 

\zz
Quant à Aramis, il avait demandé un congé de cinq jours, et il était à Rouen, disait-on, pour affaires de famille. 

M. de Tréville était le père de ses soldats. Le moindre et le plus inconnu d'entre eux, dès qu'il portait l'uniforme de la compagnie, était aussi certain de son aide et de son appui qu'aurait pu l'être son frère lui-même. 

Il se rendit donc à l'instant chez le lieutenant criminel. On fit venir l'officier qui commandait le poste de la Croix-Rouge, et les renseignements successifs apprirent qu'Athos était momentanément logé au For-l'Évêque. 

Athos avait passé par toutes les épreuves que nous avons vu Bonacieux subir. 

Nous avons assisté à la scène de confrontation entre les deux captifs. Athos, qui n'avait rien dit jusque-là de peur que d'Artagnan, inquiété à son tour, n'eût point le temps qu'il lui fallait, Athos déclara, à partir de ce moment, qu'il se nommait Athos et non d'Artagnan. 

Il ajouta qu'il ne connaissait ni monsieur, ni madame Bonacieux, qu'il n'avait jamais parlé ni à l'un, ni à l'autre; qu'il était venu vers les dix heures du soir pour faire visite à M. d'Artagnan, son ami, mais que jusqu'à cette heure il était resté chez M. de Tréville, où il avait dîné; vingt témoins, ajouta-t-il, pouvaient attester le fait, et il nomma plusieurs gentilshommes distingués, entre autres M. le duc de La Trémouille. 

Le second commissaire fut aussi étourdi que le premier de la déclaration simple et ferme de ce mousquetaire, sur lequel il aurait bien voulu prendre la revanche que les gens de robe aiment tant à gagner sur les gens d'épée; mais le nom de M. de Tréville et celui de M. le duc de La Trémouille méritaient réflexion. 

Athos fut aussi envoyé au cardinal, mais malheureusement le cardinal était au Louvre chez le roi. 

C'était précisément le moment où M. de Tréville, sortant de chez le lieutenant criminel et de chez le gouverneur du For-l'Évêque, sans avoir pu trouver Athos, arriva chez Sa Majesté. 

Comme capitaine des mousquetaires, M. de Tréville avait à toute heure ses entrées chez le roi. 

On sait quelles étaient les préventions du roi contre la reine, préventions habilement entretenues par le cardinal, qui, en fait d'intrigues, se défiait infiniment plus des femmes que des hommes. Une des grandes causes surtout de cette prévention était l'amitié d'Anne d'Autriche pour Mme de Chevreuse. Ces deux femmes l'inquiétaient plus que les guerres avec l'Espagne, les démêlés avec l'Angleterre et l'embarras des finances. À ses yeux et dans sa conviction, Mme de Chevreuse servait la reine non seulement dans ses intrigues politiques, mais, ce qui le tourmentait bien plus encore, dans ses intrigues amoureuses. 

Au premier mot de ce qu'avait dit M. le cardinal, que Mme de Chevreuse, exilée à Tours et qu'on croyait dans cette ville, était venue à Paris et, pendant cinq jours qu'elle y était restée, avait dépisté la police, le roi était entré dans une furieuse colère. Capricieux et infidèle, le roi voulait être appelé Louis le \textit{Juste et Louis le Chaste}. La postérité comprendra difficilement ce caractère, que l'histoire n'explique que par des faits et jamais par des raisonnements. 

Mais lorsque le cardinal ajouta que non seulement Mme de Chevreuse était venue à Paris, mais encore que la reine avait renoué avec elle à l'aide d'une de ces correspondances mystérieuses qu'à cette époque on nommait une cabale; lorsqu'il affirma que lui, le cardinal, allait démêler les fils les plus obscurs de cette intrigue, quand, au moment d'arrêter sur le fait, en flagrant délit, nanti de toutes les preuves, l'émissaire de la reine près de l'exilée, un mousquetaire avait osé interrompre violemment le cours de la justice en tombant, l'épée à la main, sur d'honnêtes gens de loi chargés d'examiner avec impartialité toute l'affaire pour la mettre sous les yeux du roi, --- Louis XIII ne se contint plus, il fit un pas vers l'appartement de la reine avec cette pâle et muette indignation qui, lorsqu'elle éclatait, conduisait ce prince jusqu'à la plus froide cruauté. 

Et cependant, dans tout cela, le cardinal n'avait pas encore dit un mot du duc de Buckingham. 

Ce fut alors que M. de Tréville entra, froid, poli et dans une tenue irréprochable. 

Averti de ce qui venait de se passer par la présence du cardinal et par l'altération de la figure du roi, M. de Tréville se sentit fort comme Samson devant les Philistins. 

Louis XIII mettait déjà la main sur le bouton de la porte; au bruit que fit M. de Tréville en entrant, il se retourna. 

«Vous arrivez bien, monsieur, dit le roi, qui, lorsque ses passions étaient montées à un certain point, ne savait pas dissimuler, et j'en apprends de belles sur le compte de vos mousquetaires. 

\speak  Et moi, dit froidement M. de Tréville, j'en ai de belles à apprendre à Votre Majesté sur ses gens de robe. 

\speak  Plaît-il? dit le roi avec hauteur. 

\speak  J'ai l'honneur d'apprendre à Votre Majesté, continua M. de Tréville du même ton, qu'un parti de procureurs, de commissaires et de gens de police, gens fort estimables mais fort acharnés, à ce qu'il paraît, contre l'uniforme, s'est permis d'arrêter dans une maison, d'emmener en pleine rue et de jeter au For-l'Évêque, tout cela sur un ordre que l'on a refusé de me représenter, un de mes mousquetaires, ou plutôt des vôtres, Sire, d'une conduite irréprochable, d'une réputation presque illustre, et que Votre Majesté connaît favorablement, M. Athos. 

\speak  Athos, dit le roi machinalement; oui, au fait, je connais ce nom. 

\speak  Que Votre Majesté se le rappelle, dit M. de Tréville; M. Athos est ce mousquetaire qui, dans le fâcheux duel que vous savez, a eu le malheur de blesser grièvement M. de Cahusac. --- à propos, Monseigneur, continua Tréville en s'adressant au cardinal, M. de Cahusac est tout à fait rétabli, n'est-ce pas? 

\speak  Merci! dit le cardinal en se pinçant les lèvres de colère. 

\speak  M. Athos était donc allé rendre visite à l'un de ses amis alors absent, continua M. de Tréville, à un jeune Béarnais, cadet aux gardes de Sa Majesté, compagnie des Essarts; mais à peine venait-il de s'installer chez son ami et de prendre un livre en l'attendant, qu'une nuée de recors et de soldats mêlés ensemble vint faire le siège de la maison, enfonça plusieurs portes\dots» 

Le cardinal fit au roi un signe qui signifiait: «C'est pour l'affaire dont je vous ai parlé.» 

«Nous savons tout cela, répliqua le roi, car tout cela s'est fait pour notre service. 

\speak  Alors, dit Tréville, c'est aussi pour le service de Votre Majesté qu'on a saisi un de mes mousquetaires innocent, qu'on l'a placé entre deux gardes comme un malfaiteur, et qu'on a promené au milieu d'une populace insolente ce galant homme, qui a versé dix fois son sang pour le service de Votre Majesté et qui est prêt à le répandre encore. 

\speak  Bah! dit le roi ébranlé, les choses se sont passées ainsi? 

\speak  M. de Tréville ne dit pas, reprit le cardinal avec le plus grand flegme, que ce mousquetaire innocent, que ce galant homme venait, une heure auparavant, de frapper à coups d'épée quatre commissaires instructeurs délégués par moi afin d'instruire une affaire de la plus haute importance. 

\speak  Je défie Votre Éminence de le prouver, s'écria M. de Tréville avec sa franchise toute gasconne et sa rudesse toute militaire, car, une heure auparavant M. Athos, qui, je le confierai à Votre Majesté, est un homme de la plus haute qualité, me faisait l'honneur, après avoir dîné chez moi, de causer dans le salon de mon hôtel avec M. le duc de La Trémouille et M. le comte de Châlus, qui s'y trouvaient.» 

Le roi regarda le cardinal. 

«Un procès-verbal fait foi, dit le cardinal répondant tout haut à l'interrogation muette de Sa Majesté, et les gens maltraités ont dressé le suivant, que j'ai l'honneur de présenter à Votre Majesté. 

\speak  Procès-verbal de gens de robe vaut-il la parole d'honneur, répondit fièrement Tréville, d'homme d'épée? 

\speak  Allons, allons, Tréville, taisez-vous, dit le roi. 

\speak  Si Son Éminence a quelque soupçon contre un de mes mousquetaires, dit Tréville, la justice de M. le cardinal est assez connue pour que je demande moi-même une enquête. 

\speak  Dans la maison où cette descente de justice a été faite, continua le cardinal impassible, loge, je crois, un Béarnais ami du mousquetaire. 

\speak  Votre Éminence veut parler de M. d'Artagnan? 

\speak  Je veux parler d'un jeune homme que vous protégez, Monsieur de Tréville. 

\speak  Oui, Votre Éminence, c'est cela même. 

\speak  Ne soupçonnez-vous pas ce jeune homme d'avoir donné de mauvais conseils\dots 

\speak  À M. Athos, à un homme qui a le double de son âge? interrompit M. de Tréville; non, Monseigneur. D'ailleurs, M. d'Artagnan a passé la soirée chez moi. 

\speak  Ah çà, dit le cardinal, tout le monde a donc passé la soirée chez vous? 

\speak  Son Éminence douterait-elle de ma parole? dit Tréville, le rouge de la colère au front. 

\speak  Non, Dieu m'en garde! dit le cardinal; mais, seulement, à quelle heure était-il chez vous? 

\speak  Oh! cela je puis le dire sciemment à Votre Éminence, car, comme il entrait, je remarquai qu'il était neuf heures et demie à la pendule, quoique j'eusse cru qu'il était plus tard. 

\speak  Et à quelle heure est-il sorti de votre hôtel? 

\speak  À dix heures et demie: une heure après l'événement. 

\speak  Mais, enfin, répondit le cardinal, qui ne soupçonnait pas un instant la loyauté de Tréville, et qui sentait que la victoire lui échappait, mais, enfin, Athos a été pris dans cette maison de la rue des Fossoyeurs. 

\speak  Est-il défendu à un ami de visiter un ami? à un mousquetaire de ma compagnie de fraterniser avec un garde de la compagnie de M. des Essarts? 

\speak  Oui, quand la maison où il fraternise avec cet ami est suspecte. 

\speak  C'est que cette maison est suspecte, Tréville, dit le roi; peut-être ne le saviez-vous pas? 

\speak  En effet, Sire, je l'ignorais. En tout cas, elle peut être suspecte partout; mais je nie qu'elle le soit dans la partie qu'habite M. d'Artagnan; car je puis vous affirmer, Sire, que, si j'en crois ce qu'il a dit, il n'existe pas un plus dévoué serviteur de Sa Majesté, un admirateur plus profond de M. le cardinal. 

\speak  N'est-ce pas ce d'Artagnan qui a blessé un jour Jussac dans cette malheureuse rencontre qui a eu lieu près du couvent des Carmes-Déchaussés? demanda le roi en regardant le cardinal, qui rougit de dépit. 

\speak  Et le lendemain, Bernajoux. Oui Sire, oui, c'est bien cela, et Votre Majesté a bonne mémoire. 

\speak  Allons, que résolvons-nous? dit le roi. 

\speak  Cela regarde Votre Majesté plus que moi, dit le cardinal. J'affirmerais la culpabilité. 

\speak  Et moi je la nie, dit Tréville. Mais Sa Majesté a des juges, et ses juges décideront. 

\speak  C'est cela, dit le roi, renvoyons la cause devant les juges: c'est leur affaire de juger, et ils jugeront. 

\speak  Seulement, reprit Tréville, il est bien triste qu'en ce temps malheureux où nous sommes, la vie la plus pure, la vertu la plus incontestable n'exemptent pas un homme de l'infamie et de la persécution. Aussi l'armée sera-t-elle peu contente, je puis en répondre, d'être en butte à des traitements rigoureux à propos d'affaires de police.» 

Le mot était imprudent; mais M. de Tréville l'avait lancé avec connaissance de cause. Il voulait une explosion, parce qu'en cela la mine fait du feu, et que le feu éclaire. 

«Affaires de police! s'écria le roi, relevant les paroles de M. de Tréville: affaires de police! et qu'en savez-vous, monsieur? Mêlez-vous de vos mousquetaires, et ne me rompez pas la tête. Il semble, à vous entendre, que, si par malheur on arrête un mousquetaire, la France est en danger. Eh! que de bruit pour un mousquetaire! j'en ferai arrêter dix, ventrebleu! cent, même; toute la compagnie! et je ne veux pas que l'on souffle mot. 

\speak  Du moment où ils sont suspects à Votre Majesté, dit Tréville, les mousquetaires sont coupables; aussi, me voyez-vous, Sire, prêt à vous rendre mon épée; car après avoir accusé mes soldats, M. le cardinal, je n'en doute pas, finira par m'accuser moi-même; ainsi mieux vaut que je me constitue prisonnier avec M. Athos, qui est arrêté déjà, et M. d'Artagnan, qu'on va arrêter sans doute. 

\speak  Tête gasconne, en finirez-vous? dit le roi. 

\speak  Sire, répondit Tréville sans baisser le moindrement la voix, ordonnez qu'on me rende mon mousquetaire, ou qu'il soit jugé. 

\speak  On le jugera, dit le cardinal. 

\speak  Eh bien, tant mieux; car, dans ce cas, je demanderai à Sa Majesté la permission de plaider pour lui.» 

Le roi craignit un éclat. 

«Si Son Éminence, dit-il, n'avait pas personnellement des motifs\dots» 

Le cardinal vit venir le roi, et alla au-devant de lui: 

«Pardon, dit-il, mais du moment où Votre Majesté voit en moi un juge prévenu, je me retire. 

\speak  Voyons, dit le roi, me jurez-vous, par mon père, que M. Athos était chez vous pendant l'événement, et qu'il n'y a point pris part? 

\speak  Par votre glorieux père et par vous-même, qui êtes ce que j'aime et ce que je vénère le plus au monde, je le jure! 

\speak  Veuillez réfléchir, Sire, dit le cardinal. Si nous relâchons ainsi le prisonnier, on ne pourra plus connaître la vérité. 

\speak  M. Athos sera toujours là, reprit M. de Tréville, prêt à répondre quand il plaira aux gens de robe de l'interroger. Il ne désertera pas, monsieur le cardinal; soyez tranquille, je réponds de lui, moi. 

\speak  Au fait, il ne désertera pas, dit le roi; on le retrouvera toujours, comme dit M. de Tréville. D'ailleurs, ajouta-t-il en baissant la voix et en regardant d'un air suppliant Son Éminence, donnons-leur de la sécurité: cela est politique.» 

Cette politique de Louis XIII fit sourire Richelieu. 

«Ordonnez, Sire, dit-il, vous avez le droit de grâce. 

\speak  Le droit de grâce ne s'applique qu'aux coupables, dit Tréville, qui voulait avoir le dernier mot, et mon mousquetaire est innocent. Ce n'est donc pas grâce que vous allez faire, Sire, c'est justice. 

\speak  Et il est au For-l'Évêque? dit le roi. 

\speak  Oui, Sire, et au secret, dans un cachot, comme le dernier des criminels. 

\speak  Diable! diable! murmura le roi, que faut-il faire? 

\speak  Signer l'ordre de mise en liberté, et tout sera dit, reprit le cardinal; je crois, comme Votre Majesté, que la garantie de M. de Tréville est plus que suffisante.» 

Tréville s'inclina respectueusement avec une joie qui n'était pas sans mélange de crainte; il eût préféré une résistance opiniâtre du cardinal à cette soudaine facilité. 

Le roi signa l'ordre d'élargissement, et Tréville l'emporta sans retard. 

Au moment où il allait sortir, le cardinal lui fit un sourire amical, et dit au roi: 

«Une bonne harmonie règne entre les chefs et les soldats, dans vos mousquetaires, Sire; voilà qui est bien profitable au service et bien honorable pour tous.» 

«Il me jouera quelque mauvais tour incessamment, se disait Tréville; on n'a jamais le dernier mot avec un pareil homme. Mais hâtons-nous, car le roi peut changer d'avis tout à l'heure; et au bout du compte, il est plus difficile de remettre à la Bastille ou au For-l'Évêque un homme qui en est sorti, que d'y garder un prisonnier qu'on y tient.» 

M. de Tréville fit triomphalement son entrée au For-l'Évêque, où il délivra le mousquetaire, que sa paisible indifférence n'avait pas abandonné. 

Puis, la première fois qu'il revit d'Artagnan: 

«Vous l'échappez belle, lui dit-il; voilà votre coup d'épée à Jussac payé. Reste bien encore celui de Bernajoux, mais il ne faudrait pas trop vous y fier.» 

Au reste, M. de Tréville avait raison de se défier du cardinal et de penser que tout n'était pas fini, car à peine le capitaine des mousquetaires eut-il fermé la porte derrière lui, que Son Éminence dit au roi: 

«Maintenant que nous ne sommes plus que nous deux, nous allons causer sérieusement, s'il plaît à Votre Majesté. Sire, M. de Buckingham était à Paris depuis cinq jours et n'en est parti que ce matin.» 