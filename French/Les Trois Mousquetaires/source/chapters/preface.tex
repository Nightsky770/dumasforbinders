\addchap{Préface} 

\begin{center}\scshape
Dans laquelle il est établi que, malgré leurs nom en \textit{os} et en \textit{is}, les héros de l'histoire que nous allons avoir l'honneur de raconter a nos lecteurs n'ont rien de mythologique. 
\end{center}
	
\lettrine{I}{l} y a un an à peu près, qu'en faisant à la Bibliothèque royale des recherches pour mon histoire de Louis XIV, je tombai par hasard sur les \textit{Mémoires de M. d'Artagnan}, imprimés --- comme la plus grande partie des ouvrages de cette époque, où les auteurs tenaient à dire la vérité sans aller faire un tour plus ou moins long à la Bastille --- à Amsterdam, chez Pierre Rouge. Le titre me séduisit: je les emportai chez moi, avec la permission de M. le conservateur; bien entendu, je les dévorai. 

Mon intention n'est pas de faire ici une analyse de ce curieux ouvrage, et je me contenterai d'y renvoyer ceux de mes lecteurs qui apprécient les tableaux d'époques. Ils y trouveront des portraits crayonnés de main de maître; et, quoique les esquisses soient, pour la plupart du temps, tracées sur des portes de caserne et sur des murs de cabaret, ils n'y reconnaîtront pas moins, aussi ressemblantes que dans l'histoire de M. Anquetil, les images de Louis XIII, d'Anne d'Autriche, de Richelieu, de Mazarin et de la plupart des courtisans de l'époque. 

Mais, comme on le sait, ce qui frappe l'esprit capricieux du poète n'est pas toujours ce qui impressionne la masse des lecteurs. Or, tout en admirant, comme les autres admireront sans doute, les détails que nous avons signalés, la chose qui nous préoccupa le plus est une chose à laquelle bien certainement personne avant nous n'avait fait la moindre attention. 

D'Artagnan raconte qu'à sa première visite à M. de Tréville, le capitaine des mousquetaires du roi, il rencontra dans son antichambre trois jeunes gens servant dans l'illustre corps où il sollicitait l'honneur d'être reçu, et ayant nom Athos, Porthos et Aramis. 

Nous l'avouons, ces trois noms étrangers nous frappèrent, et il nous vint aussitôt à l'esprit qu'ils n'étaient que des pseudonymes à l'aide desquels d'Artagnan avait déguisé des noms peut-être illustres, si toutefois les porteurs de ces noms d'emprunt ne les avaient pas choisis eux-mêmes le jour où, par caprice, par mécontentement ou par défaut de fortune, ils avaient endossé la simple casaque de mousquetaire. 

Dès lors nous n'eûmes plus de repos que nous n'eussions retrouvé, dans les ouvrages contemporains, une trace quelconque de ces noms extraordinaires qui avaient fort éveillé notre curiosité. 

Le seul catalogue des livres que nous lûmes pour arriver à ce but remplirait un feuilleton tout entier, ce qui serait peut-être fort instructif, mais à coups sûr peu amusant pour nos lecteurs. Nous nous contenterons donc de leur dire qu'au moment où, découragé de tant d'investigations infructueuses, nous allions abandonner notre recherche, nous trouvâmes enfin, guidé par les conseils de notre illustre et savant ami Paulin Paris, un manuscrit in-folio, coté le n° 4772 ou 4773, nous ne nous le rappelons plus bien, ayant pour titre: 

«Mémoires de M. le comte de La Fère, concernant quelques-uns des événements qui se passèrent en France vers la fin du règne du roi Louis XIII et le commencement du règne du roi Louis XIV.» 

On devine si notre joie fut grande, lorsqu'en feuilletant ce manuscrit, notre dernier espoir, nous trouvâmes à la vingtième page le nom d'Athos, à la vingt-septième le nom de Porthos, et à la trente et unième le nom d'Aramis. 

La découverte d'un manuscrit complètement inconnu, dans une époque où la science historique est poussée à un si haut degré, nous parut presque miraculeuse. Aussi nous hâtâmes-nous de solliciter la permission de le faire imprimer, dans le but de nous présenter un jour avec le bagage des autres à l'Académie des inscriptions et belles-lettres, si nous n'arrivions, chose fort probable, à entrer à l'Académie française avec notre propre bagage. Cette permission, nous devons le dire, nous fut gracieusement accordée; ce que nous consignons ici pour donner un démenti public aux malveillants qui prétendent que nous vivons sous un gouvernement assez médiocrement disposé à l'endroit des gens de lettres. 

Or, c'est la première partie de ce précieux manuscrit que nous offrons aujourd'hui à nos lecteurs, en lui restituant le titre qui lui convient, prenant l'engagement, si, comme nous n'en doutons pas, cette première partie obtient le succès qu'elle mérite, de publier incessamment la seconde. 

En attendant, comme le parrain est un second père, nous invitons le lecteur à s'en prendre à nous, et non au comte de La Fère, de son plaisir ou de son ennui. 

Cela posé, passons à notre histoire.