%!TeX root=../musketeersfr.tex 

\chapter{L'Auberge Du Colombier-Rouge} 
	
\lettrine{\accentletter[\gravebox]{A}}{} peine arrivé au camp, le roi, qui avait si grande hâte de se trouver en face de l'ennemi, et qui, à meilleur droit que le cardinal, partageait sa haine contre Buckingham, voulut faire toutes les dispositions, d'abord pour chasser les Anglais de l'île de Ré, ensuite pour presser le siège de La Rochelle; mais, malgré lui, il fut retardé par les dissensions qui éclatèrent entre MM. de Bassompierre et Schomberg, contre le duc d'Angoulême. 

MM. de Bassompierre et Schomberg étaient maréchaux de France, et réclamaient leur droit de commander l'armée sous les ordres du roi; mais le cardinal, qui craignait que Bassompierre, huguenot au fond du cœur, ne pressât faiblement les Anglais et les Rochelois, ses frères en religion, poussait au contraire le duc d'Angoulême, que le roi, à son instigation, avait nommé lieutenant général. Il en résulta que, sous peine de voir MM. de Bassompierre et Schomberg déserter l'armée, on fut obligé de faire à chacun un commandement particulier: Bassompierre prit ses quartiers au nord de la ville, depuis La Leu jusqu'à Dompierre; le duc d'Angoulême à l'est, depuis Dompierre jusqu'à Périgny; et M. de Schomberg au midi, depuis Périgny jusqu'à Angoutin. 

Le logis de Monsieur était à Dompierre. 

Le logis du roi était tantôt à Étré, tantôt à La Jarrie. 

Enfin le logis du cardinal était sur les dunes, au pont de La Pierre, dans une simple maison sans aucun retranchement. 

De cette façon, Monsieur surveillait Bassompierre; le roi, le duc d'Angoulême, et le cardinal, M. de Schomberg. 

Aussitôt cette organisation établie, on s'était occupé de chasser les Anglais de l'île. 

La conjoncture était favorable: les Anglais, qui ont, avant toute chose, besoin de bons vivres pour être de bons soldats, ne mangeant que des viandes salées et de mauvais biscuits, avaient force malades dans leur camp; de plus, la mer, fort mauvaise à cette époque de l'année sur toutes les côtes de l'océan, mettait tous les jours quelque petit bâtiment à mal; et la plage, depuis la pointe de l'Aiguillon jusqu'à la tranchée, était littéralement, à chaque marée, couverte des débris de pinasses, de roberges et de felouques; il en résultait que, même les gens du roi se tinssent-ils dans leur camp, il était évident qu'un jour ou l'autre Buckingham, qui ne demeurait dans l'île de Ré que par entêtement, serait obligé de lever le siège. 

Mais, comme M. de Toiras fit dire que tout se préparait dans le camp ennemi pour un nouvel assaut, le roi jugea qu'il fallait en finir et donna les ordres nécessaires pour une affaire décisive. 

Notre intention n'étant pas de faire un journal de siège, mais au contraire de n'en rapporter que les événements qui ont trait à l'histoire que nous racontons, nous nous contenterons de dire en deux mots que l'entreprise réussit au grand étonnement du roi et à la grande gloire de M. le cardinal. Les Anglais, repoussés pied à pied, battus dans toutes les rencontres, écrasés au passage de l'île de Loix, furent obligés de se rembarquer, laissant sur le champ de bataille deux mille hommes parmi lesquels cinq colonels, trois lieutenant-colonels, deux cent cinquante capitaines et vingt gentilshommes de qualité, quatre pièces de canon et soixante drapeaux qui furent apportés à Paris par Claude de Saint-Simon, et suspendus en grande pompe aux voûtes de Notre-Dame. 

Des Te Deum furent chantés au camp, et de là se répandirent par toute la France. 

Le cardinal resta donc maître de poursuivre le siège sans avoir, du moins momentanément, rien à craindre de la part des Anglais. 

Mais, comme nous venons de le dire, le repos n'était que momentané. 

Un envoyé du duc de Buckingham, nommé Montaigu, avait été pris, et l'on avait acquis la preuve d'une ligue entre l'Empire, l'Espagne, l'Angleterre et la Lorraine. 

Cette ligue était dirigée contre la France. 

De plus, dans le logis de Buckingham, qu'il avait été forcé d'abandonner plus précipitamment qu'il ne l'avait cru, on avait trouvé des papiers qui confirmaient cette ligue, et qui, à ce qu'assure M. le cardinal dans ses mémoires, compromettaient fort Mme de Chevreuse, et par conséquent la reine. 

C'était sur le cardinal que pesait toute la responsabilité, car on n'est pas ministre absolu sans être responsable; aussi toutes les ressources de son vaste génie étaient-elles tendues nuit et jour, et occupées à écouter le moindre bruit qui s'élevait dans un des grands royaumes de l'Europe. 

Le cardinal connaissait l'activité et surtout la haine de Buckingham; si la ligue qui menaçait la France triomphait, toute son influence était perdue: la politique espagnole et la politique autrichienne avaient leurs représentants dans le cabinet du Louvre, où elles n'avaient encore que des partisans; lui Richelieu, le ministre français, le ministre national par excellence, était perdu. Le roi, qui, tout en lui obéissant comme un enfant, le haïssait comme un enfant hait son maître, l'abandonnait aux vengeances réunies de Monsieur et de la reine; il était donc perdu, et peut-être la France avec lui. Il fallait parer à tout cela. 

Aussi vit-on les courriers, devenus à chaque instant plus nombreux, se succéder nuit et jour dans cette petite maison du pont de La Pierre, où le cardinal avait établi sa résidence. 

C'étaient des moines qui portaient si mal le froc, qu'il était facile de reconnaître qu'ils appartenaient surtout à l'église militante; des femmes un peu gênées dans leurs costumes de pages, et dont les larges trousses ne pouvaient entièrement dissimuler les formes arrondies; enfin des paysans aux mains noircies, mais à la jambe fine, et qui sentaient l'homme de qualité à une lieue à la ronde. 

Puis encore d'autres visites moins agréables, car deux ou trois fois le bruit se répandit que le cardinal avait failli être assassiné. 

Il est vrai que les ennemis de Son Éminence disaient que c'était elle-même qui mettait en campagne les assassins maladroits, afin d'avoir le cas échéant le droit d'user de représailles; mais il ne faut croire ni à ce que disent les ministres, ni à ce que disent leurs ennemis. 

Ce qui n'empêchait pas, au reste, le cardinal, à qui ses plus acharnés détracteurs n'ont jamais contesté la bravoure personnelle, de faire force courses nocturnes tantôt pour communiquer au duc d'Angoulême des ordres importants, tantôt pour aller se concerter avec le roi, tantôt pour aller conférer avec quelque messager qu'il ne voulait pas qu'on laissât entrer chez lui. 

De leur côté les mousquetaires qui n'avaient pas grand-chose à faire au siège n'étaient pas tenus sévèrement et menaient joyeuse vie. Cela leur était d'autant plus facile, à nos trois compagnons surtout, qu'étant des amis de M. de Tréville, ils obtenaient facilement de lui de s'attarder et de rester après la fermeture du camp avec des permissions particulières. 

Or, un soir que d'Artagnan, qui était de tranchée, n'avait pu les accompagner, Athos, Porthos et Aramis, montés sur leurs chevaux de bataille, enveloppés de manteaux de guerre, une main sur la crosse de leurs pistolets, revenaient tous trois d'une buvette qu'Athos avait découverte deux jours auparavant sur la route de La Jarrie, et qu'on appelait le Colombier-Rouge, suivant le chemin qui conduisait au camp, tout en se tenant sur leurs gardes, comme nous l'avons dit, de peur d'embuscade, lorsqu'à un quart de lieue à peu près du village de Boisnar ils crurent entendre le pas d'une cavalcade qui venait à eux; aussitôt tous trois s'arrêtèrent, serrés l'un contre l'autre, et attendirent, tenant le milieu de la route: au bout d'un instant, et comme la lune sortait justement d'un nuage, ils virent apparaître au détour d'un chemin deux cavaliers qui, en les apercevant, s'arrêtèrent à leur tour, paraissant délibérer s'ils devaient continuer leur route ou retourner en arrière. Cette hésitation donna quelques soupçons aux trois amis, et Athos, faisant quelques pas en avant, cria de sa voix ferme: 

«Qui vive? 

\speak  Qui vive vous-même? répondit un de ces deux cavaliers. 

\speak  Ce n'est pas répondre, cela! dit Athos. Qui vive? Répondez, ou nous chargeons. 

\speak  Prenez garde à ce que vous allez faire, messieurs! dit alors une voix vibrante qui paraissait avoir l'habitude du commandement. 

\speak  C'est quelque officier supérieur qui fait sa ronde de nuit, dit Athos, que voulez-vous faire, messieurs? 

\speak  Qui êtes-vous? dit la même voix du même ton de commandement; répondez à votre tour, ou vous pourriez vous mal trouver de votre désobéissance. 

\speak  Mousquetaires du roi, dit Athos, de plus en plus convaincu que celui qui les interrogeait en avait le droit. 

\speak  Quelle compagnie? 

\speak  Compagnie de Tréville. 

\speak  Avancez à l'ordre, et venez me rendre compte de ce que vous faites ici, à cette heure.» 

Les trois compagnons s'avancèrent, l'oreille un peu basse, car tous trois maintenant étaient convaincus qu'ils avaient affaire à plus fort qu'eux; on laissa, au reste, à Athos le soin de porter la parole. 

Un des deux cavaliers, celui qui avait pris la parole en second lieu, était à dix pas en avant de son compagnon; Athos fit signe à Porthos et à Aramis de rester de leur côté en arrière, et s'avança seul. 

«Pardon, mon officier! dit Athos; mais nous ignorions à qui nous avions affaire, et vous pouvez voir que nous faisions bonne garde. 

\speak  Votre nom? dit l'officier, qui se couvrait une partie du visage avec son manteau. 

\speak  Mais vous-même, monsieur, dit Athos qui commençait à se révolter contre cette inquisition; donnez-moi, je vous prie, la preuve que vous avez le droit de m'interroger. 

\speak  Votre nom? reprit une seconde fois le cavalier en laissant tomber son manteau de manière à avoir le visage découvert. 

\speak  Monsieur le cardinal! s'écria le mousquetaire stupéfait. 

\speak  Votre nom? reprit pour la troisième fois Son Éminence. 

\speak  Athos», dit le mousquetaire. 

Le cardinal fit un signe à l'écuyer, qui se rapprocha. 

«Ces trois mousquetaires nous suivront, dit-il à voix basse, je ne veux pas qu'on sache que je suis sorti du camp, et, en nous suivant, nous serons sûrs qu'ils ne le diront à personne. 

\speak  Nous sommes gentilshommes, Monseigneur, dit Athos; demandez-nous donc notre parole et ne vous inquiétez de rien. Dieu merci, nous savons garder un secret.» 

Le cardinal fixa ses yeux perçants sur ce hardi interlocuteur. 

«Vous avez l'oreille fine, monsieur Athos, dit le cardinal; mais maintenant, écoutez ceci: ce n'est point par défiance que je vous prie de me suivre, c'est pour ma sûreté: sans doute vos deux compagnons sont MM. Porthos et Aramis? 

\speak  Oui, Votre Éminence, dit Athos, tandis que les deux mousquetaires restés en arrière s'approchaient, le chapeau à la main. 

\speak  Je vous connais, messieurs, dit le cardinal, je vous connais: je sais que vous n'êtes pas tout à fait de mes amis, et j'en suis fâché, mais je sais que vous êtes de braves et loyaux gentilshommes, et qu'on peut se fier à vous. Monsieur Athos, faites-moi donc l'honneur de m'accompagner, vous et vos deux amis, et alors j'aurai une escorte à faire envie à Sa Majesté, si nous la rencontrons.» 

Les trois mousquetaires s'inclinèrent jusque sur le cou de leurs chevaux. 

«Eh bien, sur mon honneur, dit Athos, Votre Éminence a raison de nous emmener avec elle: nous avons rencontré sur la route des visages affreux, et nous avons même eu avec quatre de ces visages une querelle au Colombier-Rouge. 

\speak  Une querelle, et pourquoi, messieurs? dit le cardinal, je n'aime pas les querelleurs, vous le savez! 

\speak  C'est justement pour cela que j'ai l'honneur de prévenir Votre Éminence de ce qui vient d'arriver; car elle pourrait l'apprendre par d'autres que par nous, et, sur un faux rapport, croire que nous sommes en faute. 

\speak  Et quels ont été les résultats de cette querelle? demanda le cardinal en fronçant le sourcil. 

\speak  Mais mon ami Aramis, que voici, a reçu un petit coup d'épée dans le bras, ce qui ne l'empêchera pas, comme Votre Éminence peut le voir, de monter à l'assaut demain, si Votre Éminence ordonne l'escalade. 

\speak  Mais vous n'êtes pas hommes à vous laisser donner des coups d'épée ainsi, dit le cardinal: voyons, soyez francs, messieurs, vous en avez bien rendu quelques-uns; confessez-vous, vous savez que j'ai le droit de donner l'absolution. 

\speak  Moi, Monseigneur, dit Athos, je n'ai pas même mis l'épée à la main, mais j'ai pris celui à qui j'avais affaire à bras-le-corps et je l'ai jeté par la fenêtre; il paraît qu'en tombant, continua Athos avec quelque hésitation, il s'est cassé la cuisse. 

\speak  Ah! ah! fit le cardinal; et vous, monsieur Porthos? 

\speak  Moi, Monseigneur, sachant que le duel est défendu, j'ai saisi un banc, et j'en ai donné à l'un de ces brigands un coup qui, je crois, lui a brisé l'épaule. 

\speak  Bien, dit le cardinal; et vous, monsieur Aramis? 

\speak  Moi, Monseigneur, comme je suis d'un naturel très doux et que, d'ailleurs, ce que Monseigneur ne sait peut-être pas, je suis sur le point de rentrer dans les ordres, je voulais séparer mes camarades, quand un de ces misérables m'a donné traîtreusement un coup d'épée à travers le bras gauche: alors la patience m'a manqué, j'ai tiré mon épée à mon tour, et comme il revenait à la charge, je crois avoir senti qu'en se jetant sur moi il se l'était passée au travers du corps: je sais bien qu'il est tombé seulement, et il m'a semblé qu'on l'emportait avec ses deux compagnons. 

\speak  Diable, messieurs! dit le cardinal, trois hommes hors de combat pour une dispute de cabaret, vous n'y allez pas de main morte; et à propos de quoi était venue la querelle? 

\speak  Ces misérables étaient ivres, dit Athos, et sachant qu'il y avait une femme qui était arrivée le soir dans le cabaret, ils voulaient forcer la porte. 

\speak  Forcer la porte! dit le cardinal, et pour quoi faire? 

\speak  Pour lui faire violence sans doute, dit Athos; j'ai eu l'honneur de dire à Votre Éminence que ces misérables étaient ivres. 

\speak  Et cette femme était jeune et jolie? demanda le cardinal avec une certaine inquiétude. 

\speak  Nous ne l'avons pas vue, Monseigneur, dit Athos. 

\speak  Vous ne l'avez pas vue; ah! très bien, reprit vivement le cardinal; vous avez bien fait de défendre l'honneur d'une femme, et, comme c'est à l'auberge du Colombier-Rouge que je vais moi-même, je saurai si vous m'avez dit la vérité. 

\speak  Monseigneur, dit fièrement Athos, nous sommes gentilshommes, et pour sauver notre tête, nous ne ferions pas un mensonge. 

\speak  Aussi je ne doute pas de ce que vous me dites, monsieur Athos, je n'en doute pas un seul instant; mais, ajouta-t-il pour changer la conversation, cette dame était donc seule? 

\speak  Cette dame avait un cavalier enfermé avec elle, dit Athos; mais, comme malgré le bruit ce cavalier ne s'est pas montré, il est à présumer que c'est un lâche. 

\speak  Ne jugez pas témérairement, dit l'évangile», répliqua le cardinal. 

Athos s'inclina. 

«Et maintenant, messieurs, c'est bien, continua Son Éminence, je sais ce que je voulais savoir; suivez-moi.» 

Les trois mousquetaires passèrent derrière le cardinal, qui s'enveloppa de nouveau le visage de son manteau et remit son cheval en marche, se tenant à huit ou dix pas en avant de ses quatre compagnons. 

On arriva bientôt à l'auberge silencieuse et solitaire; sans doute l'hôte savait quel illustre visiteur il attendait, et en conséquence il avait renvoyé les importuns. 

Dix pas avant d'arriver à la porte, le cardinal fit signe à son écuyer et aux trois mousquetaires de faire halte, un cheval tout sellé était attaché au contrevent, le cardinal frappa trois coups et de certaine façon. 

Un homme enveloppé d'un manteau sortit aussitôt et échangea quelques rapides paroles avec le cardinal; après quoi il remonta à cheval et repartit dans la direction de Surgères, qui était aussi celle de Paris. 

«Avancez, messieurs, dit le cardinal. 

\speak  Vous m'avez dit la vérité, mes gentilshommes, dit-il en s'adressant aux trois mousquetaires, il ne tiendra pas à moi que notre rencontre de ce soir ne vous soit avantageuse; en attendant, suivez-moi.» 

Le cardinal mit pied à terre, les trois mousquetaires en firent autant; le cardinal jeta la bride de son cheval aux mains de son écuyer, les trois mousquetaires attachèrent les brides des leurs aux contrevents. 

L'hôte se tenait sur le seuil de la porte; pour lui, le cardinal n'était qu'un officier venant visiter une dame. 

«Avez-vous quelque chambre au rez-de-chaussée où ces messieurs puissent m'attendre près d'un bon feu?» dit le cardinal. 

L'hôte ouvrit la porte d'une grande salle, dans laquelle justement on venait de remplacer un mauvais poêle par une grande et excellente cheminée. 

«J'ai celle-ci, répondit-il. 

\speak  C'est bien, dit le cardinal; entrez là, messieurs, et veuillez m'attendre; je ne serai pas plus d'une demi-heure.» 

Et tandis que les trois mousquetaires entraient dans la chambre du rez-de-chaussée, le cardinal, sans demander plus amples renseignements, monta l'escalier en homme qui n'a pas besoin qu'on lui indique son chemin.