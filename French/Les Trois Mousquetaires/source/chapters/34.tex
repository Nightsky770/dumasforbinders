%!TeX root=../musketeersfr.tex 

\chapter[L'équipement D'Aramis Et De Porthos]{Où Il Est Traité De L'équipement D'Aramis Et De Porthos}

\lettrine{D}{epuis} que les quatre amis étaient chacun à la chasse de son équipement, il n'y avait plus entre eux de réunion arrêtée. On dînait les uns sans les autres, où l'on se trouvait, ou plutôt où l'on pouvait. Le service, de son côté, prenait aussi sa part de ce temps précieux, qui s'écoulait si vite. Seulement on était convenu de se trouver une fois la semaine, vers une heure, au logis d'Athos, attendu que ce dernier, selon le serment qu'il avait fait, ne passait plus le seuil de sa porte. 

C'était le jour même où Ketty était venue trouver d'Artagnan chez lui, jour de réunion. 

À peine Ketty fut-elle sortie, que d'Artagnan se dirigea vers la rue Férou. 

Il trouva Athos et Aramis qui philosophaient. Aramis avait quelques velléités de revenir à la soutane. Athos, selon ses habitudes, ne le dissuadait ni ne l'encourageait. Athos était pour qu'on laissât à chacun son libre arbitre. Il ne donnait jamais de conseils qu'on ne les lui demandât. Encore fallait-il les lui demander deux fois. 

«En général, on ne demande de conseils, disait-il, que pour ne les pas suivre; ou, si on les a suivis, que pour avoir quelqu'un à qui l'on puisse faire le reproche de les avoir donnés.» 

Porthos arriva un instant après d'Artagnan. Les quatre amis se trouvaient donc réunis. 

Les quatre visages exprimaient quatre sentiments différents: celui de Porthos la tranquillité, celui de d'Artagnan l'espoir, celui d'Aramis l'inquiétude, celui d'Athos l'insouciance. 

Au bout d'un instant de conversation dans laquelle Porthos laissa entrevoir qu'une personne haut placée avait bien voulu se charger de le tirer d'embarras, Mousqueton entra. 

Il venait prier Porthos de passer à son logis, où, disait-il d'un air fort piteux, sa présence était urgente. 

«Sont-ce mes équipages? demanda Porthos. 

\speak  Oui et non, répondit Mousqueton. 

\speak  Mais enfin que veux-tu dire?\dots 

\speak  Venez, monsieur.» 

Porthos se leva, salua ses amis et suivit Mousqueton. 

Un instant après, Bazin apparut au seuil de la porte. 

«Que me voulez-vous, mon ami? dit Aramis avec cette douceur de langage que l'on remarquait en lui chaque fois que ses idées le ramenaient vers l'église\dots 

\speak  Un homme attend monsieur à la maison, répondit Bazin. 

\speak  Un homme! quel homme? 

\speak  Un mendiant. 

\speak  Faites-lui l'aumône, Bazin, et dites-lui de prier pour un pauvre pécheur. 

\speak  Ce mendiant veut à toute force vous parler, et prétend que vous serez bien aise de le voir. 

\speak  N'a-t-il rien dit de particulier pour moi? 

\speak  Si fait. “Si M. Aramis, a-t-il dit, hésite à me venir trouver, vous lui annoncerez que j'arrive de Tours.” 

\speak  De Tours? s'écria Aramis; messieurs, mille pardons, mais sans doute cet homme m'apporte des nouvelles que j'attendais.» 

Et, se levant aussitôt, il s'éloigna rapidement. 

Restèrent Athos et d'Artagnan. 

«Je crois que ces gaillards-là ont trouvé leur affaire. Qu'en pensez-vous, d'Artagnan? dit Athos. 

\speak  Je sais que Porthos était en bon train, dit d'Artagnan; et quant à Aramis, à vrai dire, je n'en ai jamais été sérieusement inquiet: mais vous, mon cher Athos, vous qui avez si généreusement distribué les pistoles de l'Anglais qui étaient votre bien légitime, qu'allez-vous faire? 

\speak  Je suis fort content d'avoir tué ce drôle, mon enfant, vu que c'est pain bénit que de tuer un Anglais: mais si j'avais empoché ses pistoles, elles me pèseraient comme un remords. 

\speak  Allons donc, mon cher Athos! vous avez vraiment des idées inconcevables. 

\speak  Passons, passons! Que me disait donc M. de Tréville, qui me fit l'honneur de me venir voir hier, que vous hantez ces Anglais suspects que protège le cardinal? 

\speak  C'est-à-dire que je rends visite à une Anglaise, celle dont je vous ai parlé. 

\speak  Ah! oui, la femme blonde au sujet de laquelle je vous ai donné des conseils que naturellement vous vous êtes bien gardé de suivre. 

\speak  Je vous ai donné mes raisons. 

\speak  Oui; vous voyez là votre équipement, je crois, à ce que vous m'avez dit. 

\speak  Point du tout! j'ai acquis la certitude que cette femme était pour quelque chose dans l'enlèvement de Mme Bonacieux. 

\speak  Oui, et je comprends; pour retrouver une femme, vous faites la cour à une autre: c'est le chemin le plus long, mais le plus amusant. 

D'Artagnan fut sur le point de tout raconter à Athos; mais un point l'arrêta: Athos était un gentilhomme sévère sur le point d'honneur, et il y avait, dans tout ce petit plan que notre amoureux avait arrêté à l'endroit de Milady, certaines choses qui, d'avance, il en était sûr, n'obtiendraient pas l'assentiment du puritain; il préféra donc garder le silence, et comme Athos était l'homme le moins curieux de la terre, les confidences de d'Artagnan en étaient restées là. 

Nous quitterons donc les deux amis, qui n'avaient rien de bien important à se dire, pour suivre Aramis. 

À cette nouvelle, que l'homme qui voulait lui parler arrivait de Tours, nous avons vu avec quelle rapidité le jeune homme avait suivi ou plutôt devancé Bazin; il ne fit donc qu'un saut de la rue Férou à la rue de Vaugirard. 

En entrant chez lui, il trouva effectivement un homme de petite taille, aux yeux intelligents, mais couvert de haillons. 

«C'est vous qui me demandez? dit le mousquetaire. 

\speak  C'est-à-dire que je demande M. Aramis: est-ce vous qui vous appelez ainsi? 

\speak  Moi-même: vous avez quelque chose à me remettre? 

\speak  Oui, si vous me montrez certain mouchoir brodé. 

\speak  Le voici, dit Aramis en tirant une clef de sa poitrine, et en ouvrant un petit coffret de bois d'ébène incrusté de nacre, le voici, tenez. 

\speak  C'est bien, dit le mendiant, renvoyez votre laquais.» 

En effet, Bazin, curieux de savoir ce que le mendiant voulait à son maître, avait réglé son pas sur le sien, et était arrivé presque en même temps que lui; mais cette célérité ne lui servit pas à grand-chose; sur l'invitation du mendiant, son maître lui fit signe de se retirer, et force lui fut d'obéir. 

Bazin parti, le mendiant jeta un regard rapide autour de lui, afin d'être sûr que personne ne pouvait ni le voir ni l'entendre, et ouvrant sa veste en haillons mal serrée par une ceinture de cuir, il se mit à découdre le haut de son pourpoint, d'où il tira une lettre. 

Aramis jeta un cri de joie à la vue du cachet, baisa l'écriture, et avec un respect presque religieux, il ouvrit l'épître qui contenait ce qui suit: 

«Ami, le sort veut que nous soyons séparés quelque temps encore; mais les beaux jours de la jeunesse ne sont pas perdus sans retour. Faites votre devoir au camp; je fais le mien autre part. Prenez ce que le porteur vous remettra; faites la campagne en beau et bon gentilhomme, et pensez à moi, qui baise tendrement vos yeux noirs. 

«Adieu, ou plutôt au revoir!» 

Le mendiant décousait toujours; il tira une à une de ses sales habits cent cinquante doubles pistoles d'Espagne, qu'il aligna sur la table; puis, il ouvrit la porte, salua et partit avant que le jeune homme, stupéfait, eût osé lui adresser une parole. 

Aramis alors relut la lettre, et s'aperçut que cette lettre avait un \textit{post-scriptum}. 

«\textit{P.-S}. --- Vous pouvez faire accueil au porteur, qui est comte et grand d'Espagne.» 

«Rêves dorés! s'écria Aramis. Oh! la belle vie! oui, nous sommes jeunes! oui, nous aurons encore des jours heureux! Oh! à toi, mon amour, mon sang, ma vie! tout, tout, tout, ma belle maîtresse!» 

Et il baisait la lettre avec passion, sans même regarder l'or qui étincelait sur la table. 

Bazin gratta à la porte; Aramis n'avait plus de raison pour le tenir à distance; il lui permit d'entrer. 

Bazin resta stupéfait à la vue de cet or, et oublia qu'il venait annoncer d'Artagnan, qui, curieux de savoir ce que c'était que le mendiant, venait chez Aramis en sortant de chez Athos. 

Or, comme d'Artagnan ne se gênait pas avec Aramis, voyant que Bazin oubliait de l'annoncer, il s'annonça lui-même. 

«Ah! diable, mon cher Aramis, dit d'Artagnan, si ce sont là les pruneaux qu'on nous envoie de Tours, vous en ferez mon compliment au jardinier qui les récolte. 

\speak  Vous vous trompez, mon cher, dit Aramis toujours discret: c'est mon libraire qui vient de m'envoyer le prix de ce poème en vers d'une syllabe que j'avais commencé là-bas. 

\speak  Ah! vraiment! dit d'Artagnan; eh bien, votre libraire est généreux, mon cher Aramis, voilà tout ce que je puis vous dire. 

\speak  Comment, monsieur! s'écria Bazin, un poème se vend si cher! c'est incroyable! Oh! monsieur! vous faites tout ce que vous voulez, vous pouvez devenir l'égal de M. de Voiture et de M. de Benserade. J'aime encore cela, moi. Un poète, c'est presque un abbé. Ah! monsieur Aramis, mettez-vous donc poète, je vous en prie. 

\speak  Bazin, mon ami, dit Aramis, je crois que vous vous mêlez à la conversation.» 

Bazin comprit qu'il était dans son tort; il baissa la tête, et sortit. 

«Ah! dit d'Artagnan avec un sourire, vous vendez vos productions au poids de l'or: vous êtes bien heureux, mon ami; mais prenez garde, vous allez perdre cette lettre qui sort de votre casaque, et qui est sans doute aussi de votre libraire.» 

Aramis rougit jusqu'au blanc des yeux, renfonça sa lettre, et reboutonna son pourpoint. 

«Mon cher d'Artagnan, dit-il, nous allons, si vous le voulez bien, aller trouver nos amis; et puisque je suis riche, nous recommencerons aujourd'hui à dîner ensemble en attendant que vous soyez riches à votre tour. 

\speak  Ma foi! dit d'Artagnan, avec grand plaisir. Il y a longtemps que nous n'avons fait un dîner convenable; et comme j'ai pour mon compte une expédition quelque peu hasardeuse à faire ce soir, je ne serais pas fâché, je l'avoue, de me monter un peu la tête avec quelques bouteilles de vieux bourgogne. 

\speak  Va pour le vieux bourgogne; je ne le déteste pas non plus», dit Aramis, auquel la vue de l'or avait enlevé comme avec la main ses idées de retraite. 

Et ayant mis trois ou quatre doubles pistoles dans sa poche pour répondre aux besoins du moment, il enferma les autres dans le coffre d'ébène incrusté de nacre, où était déjà le fameux mouchoir qui lui avait servi de talisman. 

Les deux amis se rendirent d'abord chez Athos, qui, fidèle au serment qu'il avait fait de ne pas sortir, se chargea de faire apporter à dîner chez lui: comme il entendait à merveille les détails gastronomiques, d'Artagnan et Aramis ne firent aucune difficulté de lui abandonner ce soin important. 

Ils se rendaient chez Porthos, lorsque, au coin de la rue du Bac, ils rencontrèrent Mousqueton, qui, d'un air piteux, chassait devant lui un mulet et un cheval. 

D'Artagnan poussa un cri de surprise, qui n'était pas exempt d'un mélange de joie. 

«Ah! mon cheval jaune! s'écria-t-il. Aramis, regardez ce cheval! 

\speak  Oh! l'affreux roussin! dit Aramis. 

\speak  Eh bien, mon cher, reprit d'Artagnan, c'est le cheval sur lequel je suis venu à Paris. 

\speak  Comment, monsieur connaît ce cheval? dit Mousqueton. 

\speak  Il est d'une couleur originale, fit Aramis; c'est le seul que j'aie jamais vu de ce poil-là. 

\speak  Je le crois bien, reprit d'Artagnan, aussi je l'ai vendu trois écus, et il faut bien que ce soit pour le poil, car la carcasse ne vaut certes pas dix-huit livres. Mais comment ce cheval se trouve-t-il entre tes mains, Mousqueton? 

\speak  Ah! dit le valet, ne m'en parlez pas, monsieur, c'est un affreux tour du mari de notre duchesse! 

\speak  Comment cela, Mousqueton? 

\speak  Oui nous sommes vus d'un très bon œil par une femme de qualité, la duchesse de\dots; mais pardon! mon maître m'a recommandé d'être discret: elle nous avait forcés d'accepter un petit souvenir, un magnifique genet d'Espagne et un mulet andalou, que c'était merveilleux à voir; le mari a appris la chose, il a confisqué au passage les deux magnifiques bêtes qu'on nous envoyait, et il leur a substitué ces horribles animaux! 

\speak  Que tu lui ramènes? dit d'Artagnan. 

\speak  Justement! reprit Mousqueton; vous comprenez que nous ne pouvons point accepter de pareilles montures en échange de celles que l'on nous avait promises. 

\speak  Non, pardieu, quoique j'eusse voulu voir Porthos sur mon Bouton-d'Or; cela m'aurait donné une idée de ce que j'étais moi-même, quand je suis arrivé à Paris. Mais que nous ne t'arrêtions pas, Mousqueton; va faire la commission de ton maître, va. Est-il chez lui? 

\speak  Oui, monsieur, dit Mousqueton, mais bien maussade, allez!» 

Et il continua son chemin vers le quai des Grands-Augustins, tandis que les deux amis allaient sonner à la porte de l'infortuné Porthos. Celui-ci les avait vus traversant la cour, et il n'avait garde d'ouvrir. Ils sonnèrent donc inutilement. 

Cependant, Mousqueton continuait sa route, et, traversant le Pont-Neuf, toujours chassant devant lui ses deux haridelles, il atteignit la rue aux Ours. Arrivé là, il attacha, selon les ordres de son maître, cheval et mulet au marteau de la porte du procureur; puis, sans s'inquiéter de leur sort futur, il s'en revint trouver Porthos et lui annonça que sa commission était faite. 

Au bout d'un certain temps, les deux malheureuses bêtes, qui n'avaient pas mangé depuis le matin, firent un tel bruit en soulevant et en laissant retomber le marteau de la porte, que le procureur ordonna à son saute-ruisseau d'aller s'informer dans le voisinage à qui appartenaient ce cheval et ce mulet. 

Mme Coquenard reconnut son présent, et ne comprit rien d'abord à cette restitution; mais bientôt la visite de Porthos l'éclaira. Le courroux qui brillait dans les yeux du mousquetaire, malgré la contrainte qu'il s'imposait, épouvanta la sensible amante. En effet, Mousqueton n'avait point caché à son maître qu'il avait rencontré d'Artagnan et Aramis, et que d'Artagnan, dans le cheval jaune, avait reconnu le bidet béarnais sur lequel il était venu à Paris, et qu'il avait vendu trois écus. 

Porthos sortit après avoir donné rendez-vous à la procureuse dans le cloître Saint-Magloire. Le procureur, voyant que Porthos partait, l'invita à dîner, invitation que le mousquetaire refusa avec un air plein de majesté. 

Mme Coquenard se rendit toute tremblante au cloître Saint-Magloire, car elle devinait les reproches qui l'y attendaient; mais elle était fascinée par les grandes façons de Porthos. 

Tout ce qu'un homme blessé dans son amour-propre peut laisser tomber d'imprécations et de reproches sur la tête d'une femme, Porthos le laissa tomber sur la tête courbée de la procureuse. 

«Hélas! dit-elle, j'ai fait pour le mieux. Un de nos clients est marchand de chevaux, il devait de l'argent à l'étude, et s'est montré récalcitrant. J'ai pris ce mulet et ce cheval pour ce qu'il nous devait; il m'avait promis deux montures royales. 

\speak  Eh bien, madame, dit Porthos, s'il vous devait plus de cinq écus, votre maquignon est un voleur. 

\speak  Il n'est pas défendu de chercher le bon marché, monsieur Porthos, dit la procureuse cherchant à s'exprimer. 

\speak  Non, madame, mais ceux qui cherchent le bon marché doivent permettre aux autres de chercher des amis plus généreux.» 

Et Porthos, tournant sur ses talons, fit un pas pour se retirer. 

«Monsieur Porthos! monsieur Porthos! s'écria la procureuse, j'ai tort, je le reconnais, je n'aurais pas dû marchander quand il s'agissait d'équiper un cavalier comme vous!» 

Porthos, sans répondre, fit un second pas de retraite. 

La procureuse crut le voir dans un nuage étincelant tout entouré de duchesses et de marquises qui lui jetaient des sacs d'or sous les pieds. 

«Arrêtez, au nom du Ciel! monsieur Porthos, s'écria-t-elle, arrêtez et causons. 

\speak  Causer avec vous me porte malheur, dit Porthos. 

\speak  Mais, dites-moi, que demandez-vous? 

\speak  Rien, car cela revient au même que si je vous demandais quelque chose.» 

La procureuse se pendit au bras de Porthos, et, dans l'élan de sa douleur, elle s'écria: 

«Monsieur Porthos, je suis ignorante de tout cela, moi; sais-je ce que c'est qu'un cheval? sais-je ce que c'est que des harnais? 

\speak  Il fallait vous en rapporter à moi, qui m'y connais, madame; mais vous avez voulu ménager, et, par conséquent, prêter à usure. 

\speak  C'est un tort, monsieur Porthos, et je le réparerai sur ma parole d'honneur. 

\speak  Et comment cela? demanda le mousquetaire. 

\speak  Écoutez. Ce soir M. Coquenard va chez M. le duc de Chaulnes, qui l'a mandé. C'est pour une consultation qui durera deux heures au moins, venez, nous serons seuls, et nous ferons nos comptes. 

\speak  À la bonne heure! voilà qui est parler, ma chère! 

\speak  Vous me pardonnez? 

\speak  Nous verrons», dit majestueusement Porthos. 

Et tous deux se séparèrent en se disant: «À ce soir.» 

«Diable! pensa Porthos en s'éloignant, il me semble que je me rapproche enfin du bahut de maître Coquenard.» 