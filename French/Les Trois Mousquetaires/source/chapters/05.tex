%!TeX root=../musketeersfr.tex 


\chapter[Les Mousquetaires et les Gardes]{Les Mousquetaires Du Roi et les Gardes De M. Le Cardinal} 
	
\lettrine{D}{'Artagnan} ne connaissait personne à Paris. Il alla donc au rendez-vous d'Athos sans amener de second, résolu de se contenter de ceux qu'aurait choisis son adversaire. D'ailleurs son intention était formelle de faire au brave mousquetaire toutes les excuses convenables, mais sans faiblesse, craignant qu'il ne résultât de ce duel ce qui résulte toujours de fâcheux, dans une affaire de ce genre, quand un homme jeune et vigoureux se bat contre un adversaire blessé et affaibli: vaincu, il double le triomphe de son antagoniste; vainqueur, il est accusé de forfaiture et de facile audace. 

Au reste, ou nous avons mal exposé le caractère de notre chercheur d'aventures, ou notre lecteur a déjà dû remarquer que d'Artagnan n'était point un homme ordinaire. Aussi, tout en se répétant à lui-même que sa mort était inévitable, il ne se résigna point à mourir tout doucettement, comme un autre moins courageux et moins modéré que lui eût fait à sa place. Il réfléchit aux différents caractères de ceux avec lesquels il allait se battre, et commença à voir plus clair dans sa situation. Il espérait, grâce aux excuses loyales qu'il lui réservait, se faire un ami d'Athos, dont l'air grand seigneur et la mine austère lui agréaient fort. Il se flattait de faire peur à Porthos avec l'aventure du baudrier, qu'il pouvait, s'il n'était pas tué sur le coup, raconter à tout le monde, récit qui, poussé adroitement à l'effet, devait couvrir Porthos de ridicule; enfin, quant au sournois Aramis, il n'en avait pas très grand-peur, et en supposant qu'il arrivât jusqu'à lui, il se chargeait de l'expédier bel et bien, ou du moins en le frappant au visage, comme César avait recommandé de faire aux soldats de Pompée, d'endommager à tout jamais cette beauté dont il était si fier. 

Ensuite il y avait chez d'Artagnan ce fonds inébranlable de résolution qu'avaient déposé dans son cœur les conseils de son père, conseils dont la substance était: «Ne rien souffrir de personne que du roi, du cardinal et de M. de Tréville.» Il vola donc plutôt qu'il ne marcha vers le couvent des Carmes Déchaussés, ou plutôt Deschaux, comme on disait à cette époque, sorte de bâtiment sans fenêtres, bordé de prés arides, succursale du Pré-aux-Clercs, et qui servait d'ordinaire aux rencontres des gens qui n'avaient pas de temps à perdre. 

Lorsque d'Artagnan arriva en vue du petit terrain vague qui s'étendait au pied de ce monastère, Athos attendait depuis cinq minutes seulement, et midi sonnait. Il était donc ponctuel comme la Samaritaine, et le plus rigoureux casuiste à l'égard des duels n'avait rien a dire. 

Athos, qui souffrait toujours cruellement de sa blessure, quoiqu'elle eût été pansée à neuf par le chirurgien de M. de Tréville, s'était assis sur une borne et attendait son adversaire avec cette contenance paisible et cet air digne qui ne l'abandonnaient jamais. À l'aspect de d'Artagnan, il se leva et fit poliment quelques pas au-devant de lui. Celui-ci, de son côté, n'aborda son adversaire que le chapeau à la main et sa plume traînant jusqu'à terre. 

«Monsieur, dit Athos, j'ai fait prévenir deux de mes amis qui me serviront de seconds, mais ces deux amis ne sont point encore arrivés. Je m'étonne qu'ils tardent: ce n'est pas leur habitude. 

\speak  Je n'ai pas de seconds, moi, monsieur, dit d'Artagnan, car arrivé d'hier seulement à Paris, je n'y connais encore personne que M. de Tréville, auquel j'ai été recommandé par mon père qui a l'honneur d'être quelque peu de ses amis.» 

Athos réfléchit un instant. 

«Vous ne connaissez que M. de Tréville? demanda-t-il. 

\speak  Oui, monsieur, je ne connais que lui. 

\speak  Ah çà, mais\dots, continua Athos parlant moitié à lui-même, moitié à d'Artagnan, ah\dots çà, mais si je vous tue, j'aurai l'air d'un mangeur d'enfants, moi! 

\speak  Pas trop, monsieur, répondit d'Artagnan avec un salut qui ne manquait pas de dignité; pas trop, puisque vous me faites l'honneur de tirer l'épée contre moi avec une blessure dont vous devez être fort incommodé. 

\speak  Très incommodé, sur ma parole, et vous m'avez fait un mal du diable, je dois le dire; mais je prendrai la main gauche, c'est mon habitude en pareille circonstance. Ne croyez donc pas que je vous fasse une grâce, je tire proprement des deux mains; et il y aura même désavantage pour vous: un gaucher est très gênant pour les gens qui ne sont pas prévenus. Je regrette de ne pas vous avoir fait part plus tôt de cette circonstance. 

\speak  Vous êtes vraiment, monsieur, dit d'Artagnan en s'inclinant de nouveau, d'une courtoisie dont je vous suis on ne peut plus reconnaissant. 

\speak  Vous me rendez confus, répondit Athos avec son air de gentilhomme; causons donc d'autre chose, je vous prie, à moins que cela ne vous soit désagréable. Ah! sangbleu! que vous m'avez fait mal! l'épaule me brûle. 

\speak  Si vous vouliez permettre\dots, dit d'Artagnan avec timidité. 

\speak  Quoi, monsieur? 

\speak  J'ai un baume miraculeux pour les blessures, un baume qui me vient de ma mère, et dont j'ai fait l'épreuve sur moi-même. 

\speak  Eh bien? 

\speak  Eh bien, je suis sûr qu'en moins de trois jours ce baume vous guérirait, et au bout de trois jours, quand vous seriez guéri: eh bien, monsieur, ce me serait toujours un grand honneur d'être votre homme.» 

D'Artagnan dit ces mots avec une simplicité qui faisait honneur à sa courtoisie, sans porter aucunement atteinte à son courage. 

«Pardieu, monsieur, dit Athos, voici une proposition qui me plaît, non pas que je l'accepte, mais elle sent son gentilhomme d'une lieue. C'est ainsi que parlaient et faisaient ces preux du temps de Charlemagne, sur lesquels tout cavalier doit chercher à se modeler. Malheureusement, nous ne sommes plus au temps du grand empereur. Nous sommes au temps de M. le cardinal, et d'ici à trois jours on saurait, si bien gardé que soit le secret, on saurait, dis-je, que nous devons nous battre, et l'on s'opposerait à notre combat. Ah çà, mais! ces flâneurs ne viendront donc pas? 

\speak  Si vous êtes pressé, monsieur, dit d'Artagnan à Athos avec la même simplicité qu'un instant auparavant il lui avait proposé de remettre le duel à trois jours, si vous êtes pressé et qu'il vous plaise de m'expédier tout de suite, ne vous gênez pas, je vous en prie. 

\speak  Voilà encore un mot qui me plaît, dit Athos en faisant un gracieux signe de tête à d'Artagnan, il n'est point d'un homme sans cervelle, et il est à coup sûr d'un homme de cœur. Monsieur, j'aime les hommes de votre trempe, et je vois que si nous ne nous tuons pas l'un l'autre, j'aurai plus tard un vrai plaisir dans votre conversation. Attendons ces messieurs, je vous prie, j'ai tout le temps, et cela sera plus correct. Ah! en voici un, je crois.» 

En effet, au bout de la rue de Vaugirard commençait à apparaître le gigantesque Porthos. 

«Quoi! s'écria d'Artagnan, votre premier témoin est M. Porthos? 

\speak  Oui, cela vous contrarie-t-il? 

\speak  Non, aucunement. 

\speak  Et voici le second.» 

D'Artagnan se retourna du côté indiqué par Athos, et reconnut Aramis. 

«Quoi! s'écria-t-il d'un accent plus étonné que la première fois, votre second témoin est M. Aramis? 

\speak  Sans doute, ne savez-vous pas qu'on ne nous voit jamais l'un sans l'autre, et qu'on nous appelle, dans les mousquetaires et dans les gardes, à la cour et à la ville, Athos, Porthos et Aramis ou les trois inséparables? Après cela, comme vous arrivez de Dax ou de Pau\dots 

\speak  De Tarbes, dit d'Artagnan. 

\speak \dots Il vous est permis d'ignorer ce détail, dit Athos. 

\speak  Ma foi, dit d'Artagnan, vous êtes bien nommés, messieurs, et mon aventure, si elle fait quelque bruit, prouvera du moins que votre union n'est point fondée sur les contrastes.» 

Pendant ce temps, Porthos s'était rapproché, avait salué de la main Athos; puis, se retournant vers d'Artagnan, il était resté tout étonné. 

Disons, en passant, qu'il avait changé de baudrier et quitté son manteau. 

«Ah! ah! fit-il, qu'est-ce que cela? 

\speak  C'est avec monsieur que je me bats, dit Athos en montrant de la main d'Artagnan, et en le saluant du même geste. 

\speak  C'est avec lui que je me bats aussi, dit Porthos. 

\speak  Mais à une heure seulement, répondit d'Artagnan. 

\speak  Et moi aussi, c'est avec monsieur que je me bats, dit Aramis en arrivant à son tour sur le terrain. 

\speak  Mais à deux heures seulement, fit d'Artagnan avec le même calme. 

\speak  Mais à propos de quoi te bats-tu, toi, Athos? demanda Aramis. 

\speak  Ma foi, je ne sais pas trop, il m'a fait mal à l'épaule; et toi, Porthos? 

\speak  Ma foi, je me bats parce que je me bats», répondit Porthos en rougissant. 

Athos, qui ne perdait rien, vit passer un fin sourire sur les lèvres du Gascon. 

«Nous avons eu une discussion sur la toilette, dit le jeune homme. 

\speak  Et toi, Aramis? demanda Athos. 

\speak  Moi, je me bats pour cause de théologie», répondit Aramis tout en faisant signe à d'Artagnan qu'il le priait de tenir secrète la cause de son duel. 

Athos vit passer un second sourire sur les lèvres de d'Artagnan. 

«Vraiment, dit Athos. 

\speak  Oui, un point de saint Augustin sur lequel nous ne sommes pas d'accord, dit le Gascon. 

\speak  Décidément c'est un homme d'esprit, murmura Athos. 

\speak  Et maintenant que vous êtes rassemblés, messieurs, dit d'Artagnan, permettez-moi de vous faire mes excuses.» 

À ce mot d'\textit{excuses}, un nuage passa sur le front d'Athos, un sourire hautain glissa sur les lèvres de Porthos, et un signe négatif fut la réponse d'Aramis. 

«Vous ne me comprenez pas, messieurs, dit d'Artagnan en relevant sa tête, sur laquelle jouait en ce moment un rayon de soleil qui en dorait les lignes fines et hardies: je vous demande excuse dans le cas où je ne pourrais vous payer ma dette à tous trois, car M. Athos a le droit de me tuer le premier, ce qui ôte beaucoup de sa valeur à votre créance, monsieur Porthos, et ce qui rend la vôtre à peu près nulle, monsieur Aramis. Et maintenant, messieurs, je vous le répète, excusez-moi, mais de cela seulement, et en garde!» 

À ces mots, du geste le plus cavalier qui se puisse voir, d'Artagnan tira son épée. 

Le sang était monté à la tête de d'Artagnan, et dans ce moment il eût tiré son épée contre tous les mousquetaires du royaume, comme il venait de faire contre Athos, Porthos et Aramis. 

Il était midi et un quart. Le soleil était à son zénith et l'emplacement choisi pour être le théâtre du duel se trouvait exposé à toute son ardeur. 

«Il fait très chaud, dit Athos en tirant son épée à son tour, et cependant je ne saurais ôter mon pourpoint; car, tout à l'heure encore, j'ai senti que ma blessure saignait, et je craindrais de gêner monsieur en lui montrant du sang qu'il ne m'aurait pas tiré lui-même. 

\speak  C'est vrai, monsieur, dit d'Artagnan, et tiré par un autre ou par moi, je vous assure que je verrai toujours avec bien du regret le sang d'un aussi brave gentilhomme; je me battrai donc en pourpoint comme vous. 

\speak  Voyons, voyons, dit Porthos, assez de compliments comme cela, et songez que nous attendons notre tour. 

\speak  Parlez pour vous seul, Porthos, quand vous aurez à dire de pareilles incongruités, interrompit Aramis. Quant à moi, je trouve les choses que ces messieurs se disent fort bien dites et tout à fait dignes de deux gentilshommes. 

\speak  Quand vous voudrez, monsieur, dit Athos en se mettant en garde. 

\speak  J'attendais vos ordres», dit d'Artagnan en croisant le fer. 

Mais les deux rapières avaient à peine résonné en se touchant, qu'une escouade des gardes de Son Éminence, commandée par M. de Jussac, se montra à l'angle du couvent. 

«Les gardes du cardinal! s'écrièrent à la fois Porthos et Aramis. L'épée au fourreau, messieurs! l'épée au fourreau! 

Mais il était trop tard. Les deux combattants avaient été vus dans une pose qui ne permettait pas de douter de leurs intentions. 

«Holà! cria Jussac en s'avançant vers eux et en faisant signe à ses hommes d'en faire autant, holà! mousquetaires, on se bat donc ici? Et les édits, qu'en faisons-nous? 

\speak  Vous êtes bien généreux, messieurs les gardes, dit Athos plein de rancune, car Jussac était l'un des agresseurs de l'avant-veille. Si nous vous voyions battre, je vous réponds, moi, que nous nous garderions bien de vous en empêcher. Laissez-nous donc faire, et vous allez avoir du plaisir sans prendre aucune peine. 

\speak  Messieurs, dit Jussac, c'est avec grand regret que je vous déclare que la chose est impossible. Notre devoir avant tout. Rengainez donc, s'il vous plaît, et nous suivez. 

\speak  Monsieur, dit Aramis parodiant Jussac, ce serait avec un grand plaisir que nous obéirions à votre gracieuse invitation, si cela dépendait de nous; mais malheureusement la chose est impossible: M. de Tréville nous l'a défendu. Passez donc votre chemin, c'est ce que vous avez de mieux à faire.» 

Cette raillerie exaspéra Jussac. 

«Nous vous chargerons donc, dit-il, si vous désobéissez. 

\speak  Ils sont cinq, dit Athos à demi-voix, et nous ne sommes que trois; nous serons encore battus, et il nous faudra mourir ici, car je le déclare, je ne reparais pas vaincu devant le capitaine.» 

Alors Porthos et Aramis se rapprochèrent à l'instant les uns des autres, pendant que Jussac alignait ses soldats. 

Ce seul moment suffit à d'Artagnan pour prendre son parti: c'était là un de ces événements qui décident de la vie d'un homme, c'était un choix à faire entre le roi et le cardinal; ce choix fait, il allait y persévérer. Se battre, c'est-à-dire désobéir à la loi, c'est-à-dire risquer sa tête, c'est-à-dire se faire d'un seul coup l'ennemi d'un ministre plus puissant que le roi lui-même: voilà ce qu'entrevit le jeune homme, et, disons-le à sa louange, il n'hésita point une seconde. Se tournant donc vers Athos et ses amis: 

«Messieurs, dit-il, je reprendrai, s'il vous plaît, quelque chose à vos paroles. Vous avez dit que vous n'étiez que trois, mais il me semble, à moi, que nous sommes quatre. 

\speak  Mais vous n'êtes pas des nôtres, dit Porthos. 

\speak  C'est vrai, répondit d'Artagnan; je n'ai pas l'habit, mais j'ai l'âme. Mon cœur est mousquetaire, je le sens bien, monsieur, et cela m'entraîne. 

\speak  Écartez-vous, jeune homme, cria Jussac, qui sans doute à ses gestes et à l'expression de son visage avait deviné le dessein de d'Artagnan. Vous pouvez vous retirer, nous y consentons. Sauvez votre peau; allez vite.» 

D'Artagnan ne bougea point. 

«Décidément vous êtes un joli garçon, dit Athos en serrant la main du jeune homme. 

\speak  Allons! allons! prenons un parti, reprit Jussac. 

\speak  Voyons, dirent Porthos et Aramis, faisons quelque chose. 

\speak  Monsieur est plein de générosité», dit Athos. 

Mais tous trois pensaient à la jeunesse de d'Artagnan et redoutaient son inexpérience. 

«Nous ne serons que trois, dont un blessé, plus un enfant, reprit Athos, et l'on n'en dira pas moins que nous étions quatre hommes. 

\speak  Oui, mais reculer! dit Porthos. 

\speak  C'est difficile», reprit Athos. 

D'Artagnan comprit leur irrésolution. 

«Messieurs, essayez-moi toujours, dit-il, et je vous jure sur l'honneur que je ne veux pas m'en aller d'ici si nous sommes vaincus. 

\speak  Comment vous appelle-t-on, mon brave? dit Athos. 

\speak  D'Artagnan, monsieur. 

\speak  Eh bien, Athos, Porthos, Aramis et d'Artagnan, en avant! cria Athos. 

\speak  Eh bien, voyons, messieurs, vous décidez-vous à vous décider? cria pour la troisième fois Jussac. 

\speak  C'est fait, messieurs, dit Athos. 

\speak  Et quel parti prenez-vous? demanda Jussac. 

Nous allons avoir l'honneur de vous charger, répondit Aramis en levant son chapeau d'une main et tirant son épée de l'autre. 

\speak  Ah! vous résistez! s'écria Jussac. 

\speak  Sangdieu! cela vous étonne?» 

Et les neuf combattants se précipitèrent les uns sur les autres avec une furie qui n'excluait pas une certaine méthode. 

Athos prit un certain Cahusac, favori du cardinal; Porthos eut Biscarat, et Aramis se vit en face de deux adversaires. 

Quant à d'Artagnan, il se trouva lancé contre Jussac lui-même. 

Le cœur du jeune Gascon battait à lui briser la poitrine, non pas de peur, Dieu merci! il n'en avait pas l'ombre, mais d'émulation; il se battait comme un tigre en fureur, tournant dix fois autour de son adversaire, changeant vingt fois ses gardes et son terrain. Jussac était, comme on le disait alors, friand de la lame, et avait fort pratiqué; cependant il avait toutes les peines du monde à se défendre contre un adversaire qui, agile et bondissant, s'écartait à tout moment des règles reçues, attaquant de tous côtés à la fois, et tout cela en parant en homme qui a le plus grand respect pour son épiderme. 

Enfin cette lutte finit par faire perdre patience à Jussac. Furieux d'être tenu en échec par celui qu'il avait regardé comme un enfant, il s'échauffa et commença à faire des fautes. D'Artagnan, qui, à défaut de la pratique, avait une profonde théorie, redoubla d'agilité. Jussac, voulant en finir, porta un coup terrible à son adversaire en se fendant à fond; mais celui-ci para prime, et tandis que Jussac se relevait, se glissant comme un serpent sous son fer, il lui passa son épée au travers du corps. Jussac tomba comme une masse. 

D'Artagnan jeta alors un coup d'œil inquiet et rapide sur le champ de bataille. 

Aramis avait déjà tué un de ses adversaires; mais l'autre le pressait vivement. Cependant Aramis était en bonne situation et pouvait encore se défendre. 

Biscarat et Porthos venaient de faire coup fourré: Porthos avait reçu un coup d'épée au travers du bras, et Biscarat au travers de la cuisse. Mais comme ni l'une ni l'autre des deux blessures n'était grave, ils ne s'en escrimaient qu'avec plus d'acharnement. 

Athos, blessé de nouveau par Cahusac, pâlissait à vue d'œil, mais il ne reculait pas d'une semelle: il avait seulement changé son épée de main, et se battait de la main gauche. 

D'Artagnan, selon les lois du duel de cette époque, pouvait secourir quelqu'un; pendant qu'il cherchait du regard celui de ses compagnons qui avait besoin de son aide, il surprit un coup d'œil d'Athos. Ce coup d'œil était d'une éloquence sublime. Athos serait mort plutôt que d'appeler au secours; mais il pouvait regarder, et du regard demander un appui. D'Artagnan le devina, fit un bond terrible et tomba sur le flanc de Cahusac en criant: 

«À moi, monsieur le garde, je vous tue!» 

Cahusac se retourna; il était temps. Athos, que son extrême courage soutenait seul, tomba sur un genou. 

«Sangdieu! criait-il à d'Artagnan, ne le tuez pas, jeune homme, je vous en prie; j'ai une vieille affaire à terminer avec lui, quand je serai guéri et bien portant. Désarmez-le seulement, liez-lui l'épée. C'est cela. Bien! très bien!» 

Cette exclamation était arrachée à Athos par l'épée de Cahusac qui sautait à vingt pas de lui. D'Artagnan et Cahusac s'élancèrent ensemble, l'un pour la ressaisir, l'autre pour s'en emparer; mais d'Artagnan, plus leste, arriva le premier et mit le pied dessus. 

Cahusac courut à celui des gardes qu'avait tué Aramis, s'empara de sa rapière, et voulut revenir à d'Artagnan; mais sur son chemin il rencontra Athos, qui, pendant cette pause d'un instant que lui avait procurée d'Artagnan, avait repris haleine, et qui, de crainte que d'Artagnan ne lui tuât son ennemi, voulait recommencer le combat. 

D'Artagnan comprit que ce serait désobliger Athos que de ne pas le laisser faire. En effet, quelques secondes après, Cahusac tomba la gorge traversée d'un coup d'épée. 

Au même instant, Aramis appuyait son épée contre la poitrine de son adversaire renversé, et le forçait à demander merci. 

Restaient Porthos et Biscarat. Porthos faisait mille fanfaronnades, demandant à Biscarat quelle heure il pouvait bien être, et lui faisait ses compliments sur la compagnie que venait d'obtenir son frère dans le régiment de Navarre; mais tout en raillant, il ne gagnait rien. Biscarat était un de ces hommes de fer qui ne tombent que morts. 

Cependant il fallait en finir. Le guet pouvait arriver et prendre tous les combattants, blessés ou non, royalistes ou cardinalistes. Athos, Aramis et d'Artagnan entourèrent Biscarat et le sommèrent de se rendre. Quoique seul contre tous, et avec un coup d'épée qui lui traversait la cuisse, Biscarat voulait tenir; mais Jussac, qui s'était élevé sur son coude, lui cria de se rendre. Biscarat était un Gascon comme d'Artagnan; il fit la sourde oreille et se contenta de rire, et entre deux parades, trouvant le temps de désigner, du bout de son épée, une place à terre: 

«Ici, dit-il, parodiant un verset de la Bible, ici mourra Biscarat, seul de ceux qui sont avec lui. 

\speak  Mais ils sont quatre contre toi; finis-en, je te l'ordonne. 

\speak  Ah! si tu l'ordonnes, c'est autre chose, dit Biscarat, comme tu es mon brigadier, je dois obéir.» 

Et, faisant un bond en arrière, il cassa son épée sur son genou pour ne pas la rendre, en jeta les morceaux pardessus le mur du couvent et se croisa les bras en sifflant un air cardinaliste. 

La bravoure est toujours respectée, même dans un ennemi. Les mousquetaires saluèrent Biscarat de leurs épées et les remirent au fourreau. D'Artagnan en fit autant, puis, aidé de Biscarat, le seul qui fut resté debout, il porta sous le porche du couvent Jussac, Cahusac et celui des adversaires d'Aramis qui n'était que blessé. Le quatrième, comme nous l'avons dit, était mort. Puis ils sonnèrent la cloche, et, emportant quatre épées sur cinq, ils s'acheminèrent ivres de joie vers l'hôtel de M. de Tréville. On les voyait entrelacés, tenant toute la largeur de la rue, et accostant chaque mousquetaire qu'ils rencontraient, si bien qu'à la fin ce fut une marche triomphale. Le cœur de d'Artagnan nageait dans l'ivresse, il marchait entre Athos et Porthos en les étreignant tendrement. 

«Si je ne suis pas encore mousquetaire, dit-il à ses nouveaux amis en franchissant la porte de l'hôtel de M. de Tréville, au moins me voilà reçu apprenti, n'est-ce pas?»