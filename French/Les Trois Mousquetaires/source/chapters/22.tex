%!TeX root=../musketeersfr.tex 

\chapter{Le Ballet De La Merlaison}

\lettrine{L}{e} lendemain, il n'était bruit dans tout Paris que du bal que MM. les échevins de la ville donnaient au roi et à la reine, et dans lequel Leurs Majestés devaient danser le fameux ballet de la Merlaison, qui était le ballet favori du roi. 

Depuis huit jours on préparait, en effet, toutes choses à l'Hôtel de Ville pour cette solennelle soirée. Le menuisier de la ville avait dressé des échafauds sur lesquels devaient se tenir les dames invitées; l'épicier de la ville avait garni les salles de deux cents flambeaux de cire blanche, ce qui était un luxe inouï pour cette époque; enfin vingt violons avaient été prévenus, et le prix qu'on leur accordait avait été fixé au double du prix ordinaire, attendu, dit ce rapport, qu'ils devaient sonner toute la nuit. 

À dix heures du matin, le sieur de La Coste, enseigne des gardes du roi, suivi de deux exempts et de plusieurs archers du corps, vint demander au greffier de la ville, nommé Clément, toutes les clefs des portes, des chambres et bureaux de l'Hôtel. Ces clefs lui furent remises à l'instant même; chacune d'elles portait un billet qui devait servir à la faire reconnaître, et à partir de ce moment le sieur de La Coste fut chargé de la garde de toutes les portes et de toutes les avenues. 

À onze heures vint à son tour Duhallier, capitaine des gardes, amenant avec lui cinquante archers qui se répartirent aussitôt dans l'Hôtel de Ville, aux portes qui leur avaient été assignées. 

À trois heures arrivèrent deux compagnies des gardes, l'une française l'autre suisse. La compagnie des gardes françaises était composée moitié des hommes de M. Duhallier, moitié des hommes de M. des Essarts. 

À six heures du soir les invités commencèrent à entrer. À mesure qu'ils entraient, ils étaient placés dans la grande salle, sur les échafauds préparés. 

À neuf heures arriva Mme la Première présidente. Comme c'était, après la reine, la personne la plus considérable de la fête, elle fut reçue par messieurs de la ville et placée dans la loge en face de celle que devait occuper la reine. 

À dix heures on dressa la collation des confitures pour le roi, dans la petite salle du côté de l'église Saint-Jean, et cela en face du buffet d'argent de la ville, qui était gardé par quatre archers. 

À minuit on entendit de grands cris et de nombreuses acclamations: c'était le roi qui s'avançait à travers les rues qui conduisent du Louvre à l'Hôtel de Ville, et qui étaient toutes illuminées avec des lanternes de couleur. 

Aussitôt MM. les échevins, vêtus de leurs robes de drap et précédés de six sergents tenant chacun un flambeau à la main, allèrent au-devant du roi, qu'ils rencontrèrent sur les degrés, où le prévôt des marchands lui fit compliment sur sa bienvenue, compliment auquel Sa Majesté répondit en s'excusant d'être venue si tard, mais en rejetant la faute sur M. le cardinal, lequel l'avait retenue jusqu'à onze heures pour parler des affaires de l'État. 

Sa Majesté, en habit de cérémonie, était accompagnée de S.A.R. Monsieur, du comte de Soissons, du grand prieur, du duc de Longueville, du duc d'Elbeuf, du comte d'Harcourt, du comte de La Roche-Guyon, de M. de Liancourt, de M. de Baradas, du comte de Cramail et du chevalier de Souveray. 

Chacun remarqua que le roi avait l'air triste et préoccupé. 

Un cabinet avait été préparé pour le roi, et un autre pour Monsieur. Dans chacun de ces cabinets étaient déposés des habits de masques. Autant avait été fait pour la reine et pour Mme la présidente. Les seigneurs et les dames de la suite de Leurs Majestés devaient s'habiller deux par deux dans des chambres préparées à cet effet. 

Avant d'entrer dans le cabinet, le roi recommanda qu'on le vînt prévenir aussitôt que paraîtrait le cardinal. 

Une demi-heure après l'entrée du roi, de nouvelles acclamations retentirent: celles-là annonçaient l'arrivée de la reine: les échevins firent ainsi qu'ils avaient fait déjà et, précédés des sergents, ils s'avancèrent au devant de leur illustre convive. 

La reine entra dans la salle: on remarqua que, comme le roi, elle avait l'air triste et surtout fatigué. 

Au moment où elle entrait, le rideau d'une petite tribune qui jusque-là était resté fermé s'ouvrit, et l'on vit apparaître la tête pâle du cardinal vêtu en cavalier espagnol. Ses yeux se fixèrent sur ceux de la reine, et un sourire de joie terrible passa sur ses lèvres: la reine n'avait pas ses ferrets de diamants. 

La reine resta quelque temps à recevoir les compliments de messieurs de la ville et à répondre aux saluts des dames. 

Tout à coup, le roi apparut avec le cardinal à l'une des portes de la salle. Le cardinal lui parlait tout bas, et le roi était très pâle. 

Le roi fendit la foule et, sans masque, les rubans de son pourpoint à peine noués, il s'approcha de la reine, et d'une voix altérée: 

«Madame, lui dit-il, pourquoi donc, s'il vous plaît, n'avez-vous point vos ferrets de diamants, quand vous savez qu'il m'eût été agréable de les voir?» 

La reine étendit son regard autour d'elle, et vit derrière le roi le cardinal qui souriait d'un sourire diabolique. 

«Sire, répondit la reine d'une voix altérée, parce qu'au milieu de cette grande foule j'ai craint qu'il ne leur arrivât malheur. 

\speak  Et vous avez eu tort, madame! Si je vous ai fait ce cadeau, c'était pour que vous vous en pariez. Je vous dis que vous avez eu tort.» 

Et la voix du roi était tremblante de colère; chacun regardait et écoutait avec étonnement, ne comprenant rien à ce qui se passait. 

«Sire, dit la reine, je puis les envoyer chercher au Louvre, où ils sont, et ainsi les désirs de Votre Majesté seront accomplis. 

\speak  Faites, madame, faites, et cela au plus tôt: car dans une heure le ballet va commencer.» 

La reine salua en signe de soumission et suivit les dames qui devaient la conduire à son cabinet. 

De son côté, le roi regagna le sien. 

Il y eut dans la salle un moment de trouble et de confusion. 

Tout le monde avait pu remarquer qu'il s'était passé quelque chose entre le roi et la reine; mais tous deux avaient parlé si bas, que, chacun par respect s'étant éloigné de quelques pas, personne n'avait rien entendu. Les violons sonnaient de toutes leurs forces, mais on ne les écoutait pas. 

Le roi sortit le premier de son cabinet; il était en costume de chasse des plus élégants, et Monsieur et les autres seigneurs étaient habillés comme lui. C'était le costume que le roi portait le mieux, et vêtu ainsi il semblait véritablement le premier gentilhomme de son royaume. 

Le cardinal s'approcha du roi et lui remit une boîte. Le roi l'ouvrit et y trouva deux ferrets de diamants. 

«Que veut dire cela? demanda-t-il au cardinal. 

\speak  Rien, répondit celui-ci; seulement si la reine a les ferrets, ce dont je doute, comptez-les, Sire, et si vous n'en trouvez que dix, demandez à Sa Majesté qui peut lui avoir dérobé les deux ferrets que voici.» 

Le roi regarda le cardinal comme pour l'interroger; mais il n'eut le temps de lui adresser aucune question: un cri d'admiration sortit de toutes les bouches. Si le roi semblait le premier gentilhomme de son royaume, la reine était à coup sûr la plus belle femme de France. 

Il est vrai que sa toilette de chasseresse lui allait à merveille; elle avait un chapeau de feutre avec des plumes bleues, un surtout en velours gris perle rattaché avec des agrafes de diamants, et une jupe de satin bleu toute brodée d'argent. Sur son épaule gauche étincelaient les ferrets soutenus par un noeud de même couleur que les plumes et la jupe. 

Le roi tressaillit de joie et le cardinal de colère; cependant, distants comme ils l'étaient de la reine, ils ne pouvaient compter les ferrets; la reine les avait, seulement en avait-elle dix ou en avait-elle douze? 

En ce moment, les violons sonnèrent le signal du ballet. Le roi s'avança vers Mme la présidente, avec laquelle il devait danser, et S.A.R. Monsieur avec la reine. On se mit en place, et le ballet commença. 

Le roi figurait en face de la reine, et chaque fois qu'il passait près d'elle, il dévorait du regard ces ferrets, dont il ne pouvait savoir le compte. Une sueur froide couvrait le front du cardinal. 

Le ballet dura une heure; il avait seize entrées. 

Le ballet finit au milieu des applaudissements de toute la salle, chacun reconduisit sa dame à sa place; mais le roi profita du privilège qu'il avait de laisser la sienne où il se trouvait, pour s'avancer vivement vers la reine. 

«Je vous remercie, madame, lui dit-il, de la déférence que vous avez montrée pour mes désirs, mais je crois qu'il vous manque deux ferrets, et je vous les rapporte.» 

À ces mots, il tendit à la reine les deux ferrets que lui avait remis le cardinal. 

«Comment, Sire! s'écria la jeune reine jouant la surprise, vous m'en donnez encore deux autres; mais alors cela m'en fera donc quatorze?» 

En effet, le roi compta, et les douze ferrets se trouvèrent sur l'épaule de Sa Majesté. 

Le roi appela le cardinal: 

«Eh bien, que signifie cela, monsieur le cardinal? demanda le roi d'un ton sévère. 

\speak  Cela signifie, Sire, répondit le cardinal, que je désirais faire accepter ces deux ferrets à Sa Majesté, et que n'osant les lui offrir moi-même, j'ai adopté ce moyen. 

\speak  Et j'en suis d'autant plus reconnaissante à Votre Éminence, répondit Anne d'Autriche avec un sourire qui prouvait qu'elle n'était pas dupe de cette ingénieuse galanterie, que je suis certaine que ces deux ferrets vous coûtent aussi cher à eux seuls que les douze autres ont coûté à Sa Majesté.» 

Puis, ayant salué le roi et le cardinal, la reine reprit le chemin de la chambre où elle s'était habillée et où elle devait se dévêtir. 

L'attention que nous avons été obligés de donner pendant le commencement de ce chapitre aux personnages illustres que nous y avons introduits nous a écartés un instant de celui à qui Anne d'Autriche devait le triomphe inouï qu'elle venait de remporter sur le cardinal, et qui, confondu, ignoré, perdu dans la foule entassée à l'une des portes, regardait de là cette scène compréhensible seulement pour quatre personnes: le roi, la reine, Son Éminence et lui. 

La reine venait de regagner sa chambre, et d'Artagnan s'apprêtait à se retirer, lorsqu'il sentit qu'on lui touchait légèrement l'épaule; il se retourna, et vit une jeune femme qui lui faisait signe de la suivre. Cette jeune femme avait le visage couvert d'un loup de velours noir, mais malgré cette précaution, qui, au reste, était bien plutôt prise pour les autres que pour lui, il reconnut à l'instant même son guide ordinaire, la légère et spirituelle Mme Bonacieux. 

La veille ils s'étaient vus à peine chez le suisse Germain, où d'Artagnan l'avait fait demander. La hâte qu'avait la jeune femme de porter à la reine cette excellente nouvelle de l'heureux retour de son messager fit que les deux amants échangèrent à peine quelques paroles. D'Artagnan suivit donc Mme Bonacieux, mû par un double sentiment, l'amour et la curiosité. Pendant toute la route, et à mesure que les corridors devenaient plus déserts, d'Artagnan voulait arrêter la jeune femme, la saisir, la contempler, ne fût-ce qu'un instant; mais, vive comme un oiseau, elle glissait toujours entre ses mains, et lorsqu'il voulait parler, son doigt ramené sur sa bouche avec un petit geste impératif plein de charme lui rappelait qu'il était sous l'empire d'une puissance à laquelle il devait aveuglément obéir, et qui lui interdisait jusqu'à la plus légère plainte; enfin, après une minute ou deux de tours et de détours, Mme Bonacieux ouvrit une porte et introduisit le jeune homme dans un cabinet tout à fait obscur. Là elle lui fit un nouveau signe de mutisme, et ouvrant une seconde porte cachée par une tapisserie dont les ouvertures répandirent tout à coup une vive lumière, elle disparut. 

D'Artagnan demeura un instant immobile et se demandant où il était, mais bientôt un rayon de lumière qui pénétrait par cette chambre, l'air chaud et parfumé qui arrivait jusqu'à lui, la conversation de deux ou trois femmes, au langage à la fois respectueux et élégant, le mot de Majesté plusieurs fois répété, lui indiquèrent clairement qu'il était dans un cabinet attenant à la chambre de la reine. 

Le jeune homme se tint dans l'ombre et attendit. 

La reine paraissait gaie et heureuse, ce qui semblait fort étonner les personnes qui l'entouraient, et qui avaient au contraire l'habitude de la voir presque toujours soucieuse. La reine rejetait ce sentiment joyeux sur la beauté de la fête, sur le plaisir que lui avait fait éprouver le ballet, et comme il n'est pas permis de contredire une reine, qu'elle sourie ou qu'elle pleure, chacun renchérissait sur la galanterie de MM. les échevins de la ville de Paris. 

Quoique d'Artagnan ne connût point la reine, il distingua sa voix des autres voix, d'abord à un léger accent étranger, puis à ce sentiment de domination naturellement empreint dans toutes les paroles souveraines. Il l'entendait s'approcher et s'éloigner de cette porte ouverte, et deux ou trois fois il vit même l'ombre d'un corps intercepter la lumière. 

Enfin, tout à coup une main et un bras adorables de forme et de blancheur passèrent à travers la tapisserie; d'Artagnan comprit que c'était sa récompense: il se jeta à genoux, saisit cette main et appuya respectueusement ses lèvres; puis cette main se retira laissant dans les siennes un objet qu'il reconnut pour être une bague; aussitôt la porte se referma, et d'Artagnan se retrouva dans la plus complète obscurité. 

D'Artagnan mit la bague à son doigt et attendit de nouveau; il était évident que tout n'était pas fini encore. 

Après la récompense de son dévouement venait la récompense de son amour. D'ailleurs, le ballet était dansé, mais la soirée était à peine commencée: on soupait à trois heures, et l'horloge Saint-Jean, depuis quelque temps déjà, avait sonné deux heures trois quarts. 

En effet, peu à peu le bruit des voix diminua dans la chambre voisine; puis on l'entendit s'éloigner; puis la porte du cabinet où était d'Artagnan se rouvrit, et Mme Bonacieux s'y élança. 

«Vous, enfin! s'écria d'Artagnan. 

\speak  Silence! dit la jeune femme en appuyant sa main sur les lèvres du jeune homme: silence! et allez-vous-en par où vous êtes venu. 

\speak  Mais où et quand vous reverrai-je? s'écria d'Artagnan. 

\speak  Un billet que vous trouverez en rentrant vous le dira. Partez, partez!» 

Et à ces mots elle ouvrit la porte du corridor et poussa d'Artagnan hors du cabinet. 

D'Artagnan obéit comme un enfant, sans résistance et sans objection aucune, ce qui prouve qu'il était bien réellement amoureux.