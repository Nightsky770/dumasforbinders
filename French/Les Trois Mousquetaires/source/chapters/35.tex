%!TeX root=../musketeersfr.tex 

\chapter{La Nuit Tous Les Chats Sont Gris}

\lettrine{C}{e} soir, attendu si impatiemment par Porthos et par d'Artagnan, arriva enfin. 

\zz
D'Artagnan, comme d'habitude, se présenta vers les neuf heures chez Milady. Il la trouva d'une humeur charmante; jamais elle ne l'avait si bien reçu. Notre Gascon vit du premier coup d'œil que son billet avait été remis, et ce billet faisait son effet. 

Ketty entra pour apporter des sorbets. Sa maîtresse lui fit une mine charmante, lui sourit de son plus gracieux sourire; mais, hélas! la pauvre fille était si triste, qu'elle ne s'aperçut même pas de la bienveillance de Milady. 

D'Artagnan regardait l'une après l'autre ces deux femmes, et il était forcé de s'avouer que la nature s'était trompée en les formant; à la grande dame elle avait donné une âme vénale et vile, à la soubrette elle avait donné le cœur d'une duchesse. 

À dix heures Milady commença à paraître inquiète, d'Artagnan comprit ce que cela voulait dire; elle regardait la pendule, se levait, se rasseyait, souriait à d'Artagnan d'un air qui voulait dire: Vous êtes fort aimable sans doute, mais vous seriez charmant si vous partiez! 

D'Artagnan se leva et prit son chapeau; Milady lui donna sa main à baiser; le jeune homme sentit qu'elle la lui serrait et comprit que c'était par un sentiment non pas de coquetterie, mais de reconnaissance à cause de son départ. 

«Elle l'aime diablement», murmura-t-il. Puis il sortit. 

Cette fois Ketty ne l'attendait aucunement, ni dans l'antichambre, ni dans le corridor, ni sous la grande porte. Il fallut que d'Artagnan trouvât tout seul l'escalier et la petite chambre. 

Ketty était assise la tête cachée dans ses mains, et pleurait. 

Elle entendit entrer d'Artagnan, mais elle ne releva point la tête; le jeune homme alla à elle et lui prit les mains, alors elle éclata en sanglots. 

Comme l'avait présumé d'Artagnan, Milady, en recevant la lettre, avait, dans le délire de sa joie, tout dit à sa suivante; puis, en récompense de la manière dont cette fois elle avait fait la commission, elle lui avait donné une bourse. Ketty, en rentrant chez elle, avait jeté la bourse dans un coin, où elle était restée tout ouverte, dégorgeant trois ou quatre pièces d'or sur le tapis. 

La pauvre fille, à la voix de d'Artagnan, releva la tête. D'Artagnan lui-même fut effrayé du bouleversement de son visage; elle joignit les mains d'un air suppliant, mais sans oser dire une parole. 

Si peu sensible que fût le cœur de d'Artagnan, il se sentit attendri par cette douleur muette; mais il tenait trop à ses projets et surtout à celui-ci, pour rien changer au programme qu'il avait fait d'avance. Il ne laissa donc à Ketty aucun espoir de le fléchir, seulement il lui présenta son action comme une simple vengeance. 

Cette vengeance, au reste, devenait d'autant plus facile, que Milady, sans doute pour cacher sa rougeur à son amant, avait recommandé à Ketty d'éteindre toutes les lumières dans l'appartement, et même dans sa chambre, à elle. Avant le jour, M. de Wardes devait sortir, toujours dans l'obscurité. 

Au bout d'un instant on entendit Milady qui rentrait dans sa chambre. D'Artagnan s'élança aussitôt dans son armoire. À peine y était-il blotti que la sonnette se fit entendre. 

Ketty entra chez sa maîtresse, et ne laissa point la porte ouverte; mais la cloison était si mince, que l'on entendait à peu près tout ce qui se disait entre les deux femmes. 

Milady semblait ivre de joie, elle se faisait répéter par Ketty les moindres détails de la prétendue entrevue de la soubrette avec de Wardes, comment il avait reçu sa lettre, comment il avait répondu, quelle était l'expression de son visage, s'il paraissait bien amoureux; et à toutes ces questions la pauvre Ketty, forcée de faire bonne contenance, répondait d'une voix étouffée dont sa maîtresse ne remarquait même pas l'accent douloureux, tant le bonheur est égoïste. 

Enfin, comme l'heure de son entretien avec le comte approchait, Milady fit en effet tout éteindre chez elle, et ordonna à Ketty de rentrer dans sa chambre, et d'introduire de Wardes aussitôt qu'il se présenterait. 

L'attente de Ketty ne fut pas longue. À peine d'Artagnan eut-il vu par le trou de la serrure de son armoire que tout l'appartement était dans l'obscurité, qu'il s'élança de sa cachette au moment même où Ketty refermait la porte de communication. 

«Qu'est-ce que ce bruit? demanda Milady. 

\speak  C'est moi, dit d'Artagnan à demi-voix; moi, le comte de Wardes. 

\speak  Oh! mon Dieu, mon Dieu! murmura Ketty, il n'a pas même pu attendre l'heure qu'il avait fixée lui-même! 

\speak  Eh bien, dit Milady d'une voix tremblante, pourquoi n'entre-t-il pas? Comte, comte, ajouta-t-elle, vous savez bien que je vous attends!» 

À cet appel, d'Artagnan éloigna doucement Ketty et s'élança dans la chambre de Milady. 

Si la rage et la douleur doivent torturer une âme, c'est celle de l'amant qui reçoit sous un nom qui n'est pas le sien des protestations d'amour qui s'adressent à son heureux rival. 

D'Artagnan était dans une situation douloureuse qu'il n'avait pas prévue, la jalousie le mordait au cœur, et il souffrait presque autant que la pauvre Ketty, qui pleurait en ce même moment dans la chambre voisine. 

«Oui, comte, disait Milady de sa plus douce voix en lui serrant tendrement la main dans les siennes; oui, je suis heureuse de l'amour que vos regards et vos paroles m'ont exprimé chaque fois que nous nous sommes rencontrés. Moi aussi, je vous aime. Oh! demain, demain, je veux quelque gage de vous qui me prouve que vous pensez à moi, et comme vous pourriez m'oublier, tenez.» 

Et elle passa une bague de son doigt à celui de d'Artagnan. 

D'Artagnan se rappela avoir vu cette bague à la main de Milady: c'était un magnifique saphir entouré de brillants. 

Le premier mouvement de d'Artagnan fut de le lui rendre, mais Milady ajouta: 

«Non, non; gardez cette bague pour l'amour de moi. Vous me rendez d'ailleurs, en l'acceptant, ajouta-t-elle d'une voix émue, un service bien plus grand que vous ne sauriez l'imaginer.» 

«Cette femme est pleine de mystères», murmura en lui-même d'Artagnan. 

En ce moment il se sentit prêt à tout révéler. Il ouvrit la bouche pour dire à Milady qui il était, et dans quel but de vengeance il était venu, mais elle ajouta: 

«Pauvre ange, que ce monstre de Gascon a failli tuer!» 

Le monstre, c'était lui. 

«Oh! continua Milady, est-ce que vos blessures vous font encore souffrir? 

\speak  Oui, beaucoup, dit d'Artagnan, qui ne savait trop que répondre. 

\speak  Soyez tranquille, murmura Milady, je vous vengerai, moi, et cruellement!» 

«Peste! se dit d'Artagnan, le moment des confidences n'est pas encore venu.» 

Il fallut quelque temps à d'Artagnan pour se remettre de ce petit dialogue: mais toutes les idées de vengeance qu'il avait apportées s'étaient complètement évanouies. Cette femme exerçait sur lui une incroyable puissance, il la haïssait et l'adorait à la fois, il n'avait jamais cru que deux sentiments si contraires pussent habiter dans le même cœur, et en se réunissant, former un amour étrange et en quelque sorte diabolique. 

Cependant une heure venait de sonner; il fallut se séparer; d'Artagnan, au moment de quitter Milady, ne sentit plus qu'un vif regret de s'éloigner, et, dans l'adieu passionné qu'ils s'adressèrent réciproquement, une nouvelle entrevue fut convenue pour la semaine suivante. La pauvre Ketty espérait pouvoir adresser quelques mots à d'Artagnan lorsqu'il passerait dans sa chambre; mais Milady le reconduisit elle-même dans l'obscurité et ne le quitta que sur l'escalier. 

Le lendemain au matin, d'Artagnan courut chez Athos. Il était engagé dans une si singulière aventure qu'il voulait lui demander conseil. Il lui raconta tout: Athos fronça plusieurs fois le sourcil. 

«Votre Milady, lui dit-il, me paraît une créature infâme, mais vous n'en avez pas moins eu tort de la tromper: vous voilà d'une façon ou d'une autre une ennemie terrible sur les bras.» 

Et tout en lui parlant, Athos regardait avec attention le saphir entouré de diamants qui avait pris au doigt de d'Artagnan la place de la bague de la reine, soigneusement remise dans un écrin. 

«Vous regardez cette bague? dit le Gascon tout glorieux d'étaler aux regards de ses amis un si riche présent. 

\speak  Oui, dit Athos, elle me rappelle un bijou de famille. 

\speak  Elle est belle, n'est-ce pas? dit d'Artagnan. 

\speak  Magnifique! répondit Athos; je ne croyais pas qu'il existât deux saphirs d'une si belle eau. L'avez-vous donc troquée contre votre diamant? 

\speak  Non, dit d'Artagnan; c'est un cadeau de ma belle Anglaise, ou plutôt de ma belle Française: car, quoique je ne le lui aie point demandé, je suis convaincu qu'elle est née en France. 

\speak  Cette bague vous vient de Milady? s'écria Athos avec une voix dans laquelle il était facile de distinguer une grande émotion. 

\speak  D'elle-même; elle me l'a donnée cette nuit. 

\speak  Montrez-moi donc cette bague, dit Athos. 

\speak  La voici», répondit d'Artagnan en la tirant de son doigt. 

Athos l'examina et devint très pâle, puis il l'essaya à l'annulaire de sa main gauche; elle allait à ce doigt comme si elle eût été faite pour lui. Un nuage de colère et de vengeance passa sur le front ordinairement calme du gentilhomme. 

«Il est impossible que ce soit la même, dit-il; comment cette bague se trouverait-elle entre les mains de Milady Clarick? Et cependant il est bien difficile qu'il y ait entre deux bijoux une pareille ressemblance. 

\speak  Connaissez-vous cette bague? demanda d'Artagnan. 

\speak  J'avais cru la reconnaître, dit Athos, mais sans doute que je me trompais.» 

Et il la rendit à d'Artagnan, sans cesser cependant de la regarder. 

«Tenez, dit-il au bout d'un instant, d'Artagnan, ôtez cette bague de votre doigt ou tournez-en le chaton en dedans; elle me rappelle de si cruels souvenirs, que je n'aurais pas ma tête pour causer avec vous. Ne veniez-vous pas me demander des conseils, ne me disiez-vous point que vous étiez embarrassé sur ce que vous deviez faire?\dots Mais attendez\dots rendez-moi ce saphir: celui dont je voulais parler doit avoir une de ses faces éraillée par suite d'un accident.» 

D'Artagnan tira de nouveau la bague de son doigt et la rendit à Athos. 

Athos tressaillit: 

«Tenez, dit-il, voyez, n'est-ce pas étrange?» 

Et il montrait à d'Artagnan cette égratignure qu'il se rappelait devoir exister. 

«Mais de qui vous venait ce saphir, Athos? 

\speak  De ma mère, qui le tenait de sa mère à elle. Comme je vous le dis, c'est un vieux bijou\dots qui ne devait jamais sortir de la famille. 

\speak  Et vous l'avez\dots vendu? demanda avec hésitation d'Artagnan. 

\speak  Non, reprit Athos avec un singulier sourire; je l'ai donné pendant une nuit d'amour, comme il vous a été donné à vous.» 

D'Artagnan resta pensif à son tour, il lui semblait voir dans l'âme de Milady des abîmes dont les profondeurs étaient sombres et inconnues. 

Il remit la bague non pas à son doigt, mais dans sa poche. 

«Écoutez, lui dit Athos en lui prenant la main, vous savez si je vous aime, d'Artagnan; j'aurais un fils que je ne l'aimerais pas plus que vous. Eh bien, croyez-moi, renoncez à cette femme. Je ne la connais pas, mais une espèce d'intuition me dit que c'est une créature perdue, et qu'il y a quelque chose de fatal en elle. 

\speak  Et vous avez raison, dit d'Artagnan. Aussi, je m'en sépare; je vous avoue que cette femme m'effraie moi-même. 

\speak  Aurez-vous ce courage? dit Athos. 

\speak  Je l'aurai, répondit d'Artagnan, et à l'instant même. 

\speak  Eh bien, vrai, mon enfant, vous avez raison, dit le gentilhomme en serrant la main du Gascon avec une affection presque paternelle; que Dieu veuille que cette femme, qui est à peine entrée dans votre vie, n'y laisse pas une trace funeste!» 

Et Athos salua d'Artagnan de la tête, en homme qui veut faire comprendre qu'il n'est pas fâché de rester seul avec ses pensées. 

En rentrant chez lui d'Artagnan trouva Ketty, qui l'attendait. Un mois de fièvre n'eût pas plus changé la pauvre enfant qu'elle ne l'était pour cette nuit d'insomnie et de douleur. 

Elle était envoyée par sa maîtresse au faux de Wardes. Sa maîtresse était folle d'amour, ivre de joie: elle voulait savoir quand le comte lui donnerait une seconde entrevue. 

Et la pauvre Ketty, pâle et tremblante, attendait la réponse de d'Artagnan. 

Athos avait une grande influence sur le jeune homme: les conseils de son ami joints aux cris de son propre cœur l'avaient déterminé, maintenant que son orgueil était sauvé et sa vengeance satisfaite, à ne plus revoir Milady. Pour toute réponse il prit donc une plume et écrivit la lettre suivante: 

\begin{mail}{}{}
Ne comptez pas sur moi, madame, pour le prochain rendez-vous: depuis ma convalescence j'ai tant d'occupations de ce genre qu'il m'a fallu y mettre un certain ordre. Quand votre tour viendra, j'aurai l'honneur de vous en faire part. 
\closeletter[Je vous baise les mains.]{Comte de Wardes }
\end{mail}

Du saphir pas un mot: le Gascon voulait-il garder une arme contre Milady? ou bien, soyons franc, ne conservait-il pas ce saphir comme une dernière ressource pour l'équipement? 

On aurait tort au reste de juger les actions d'une époque au point de vue d'une autre époque. Ce qui aujourd'hui serait regardé comme une honte pour un galant homme était dans ce temps une chose toute simple et toute naturelle, et les cadets des meilleures familles se faisaient en général entretenir par leurs maîtresses. 

D'Artagnan passa sa lettre tout ouverte à Ketty, qui la lut d'abord sans la comprendre et qui faillit devenir folle de joie en la relisant une seconde fois. 

Ketty ne pouvait croire à ce bonheur: d'Artagnan fut forcé de lui renouveler de vive voix les assurances que la lettre lui donnait par écrit; et quel que fût, avec le caractère emporté de Milady, le danger que courût la pauvre enfant à remettre ce billet à sa maîtresse, elle n'en revint pas moins place Royale de toute la vitesse de ses jambes. 

Le cœur de la meilleure femme est impitoyable pour les douleurs d'une rivale. 

Milady ouvrit la lettre avec un empressement égal à celui que Ketty avait mis à l'apporter, mais au premier mot qu'elle lut, elle devint livide; puis elle froissa le papier; puis elle se retourna avec un éclair dans les yeux du côté de Ketty. 

«Qu'est-ce que cette lettre? dit-elle. 

\speak  Mais c'est la réponse à celle de madame, répondit Ketty toute tremblante. 

\speak  Impossible! s'écria Milady; impossible qu'un gentilhomme ait écrit à une femme une pareille lettre!» 

Puis tout à coup tressaillant: 

«Mon Dieu! dit-elle, saurait-il\dots» Et elle s'arrêta. 

Ses dents grinçaient, elle était couleur de cendre: elle voulut faire un pas vers la fenêtre pour aller chercher de l'air; mais elle ne put qu'étendre les bras, les jambes lui manquèrent, et elle tomba sur un fauteuil. 

Ketty crut qu'elle se trouvait mal et se précipita pour ouvrir son corsage. Mais Milady se releva vivement: 

«Que me voulez-vous? dit-elle, et pourquoi portez-vous la main sur moi? 

\speak  J'ai pensé que madame se trouvait mal et j'ai voulu lui porter secours, répondit la suivante tout épouvantée de l'expression terrible qu'avait prise la figure de sa maîtresse. 

\speak  Me trouver mal, moi? moi? me prenez-vous pour une femmelette? Quand on m'insulte, je ne me trouve pas mal, je me venge, entendez-vous!» 

Et de la main elle fit signe à Ketty de sortir.