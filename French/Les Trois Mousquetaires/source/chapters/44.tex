%!TeX root=../musketeersfr.tex 

\chapter{De L'Utilité Des Tuyaux De Poêle}

\lettrine{I}{l} était évident que, sans s'en douter, et mus seulement par leur caractère chevaleresque et aventureux, nos trois amis venaient de rendre service à quelqu'un que le cardinal honorait de sa protection particulière. 

Maintenant quel était ce quelqu'un? C'est la question que se firent d'abord les trois mousquetaires; puis, voyant qu'aucune des réponses que pouvait leur faire leur intelligence n'était satisfaisante, Porthos appela l'hôte et demanda des dés. 

Porthos et Aramis se placèrent à une table et se mirent à jouer. Athos se promena en réfléchissant. 

En réfléchissant et en se promenant, Athos passait et repassait devant le tuyau du poêle rompu par la moitié et dont l'autre extrémité donnait dans la chambre supérieure, et à chaque fois qu'il passait et repassait, il entendait un murmure de paroles qui finit par fixer son attention. Athos s'approcha, et il distingua quelques mots qui lui parurent sans doute mériter un si grand intérêt qu'il fit signe à ses compagnons de se taire, restant lui-même courbé l'oreille tendue à la hauteur de l'orifice inférieur. 

«Écoutez, Milady, disait le cardinal, l'affaire est importante: asseyez-vous là et causons. 

\speak  Milady! murmura Athos. 

\speak  J'écoute Votre Éminence avec la plus grande attention, répondit une voix de femme qui fit tressaillir le mousquetaire. 

\speak  Un petit bâtiment avec équipage anglais, dont le capitaine est à moi, vous attend à l'embouchure de la Charente, au fort de La Pointe; il mettra à la voile demain matin. 

\speak  Il faut alors que je m'y rende cette nuit? 

\speak  À l'instant même, c'est-à-dire lorsque vous aurez reçu mes instructions. Deux hommes que vous trouverez à la porte en sortant vous serviront d'escorte; vous me laisserez sortir le premier, puis une demi-heure après moi, vous sortirez à votre tour. 

\speak  Oui, Monseigneur. Maintenant revenons à la mission dont vous voulez bien me charger; et comme je tiens à continuer de mériter la confiance de Votre Éminence, daignez me l'exposer en termes clairs et précis, afin que je ne commette aucune erreur.» 

Il y eut un instant de profond silence entre les deux interlocuteurs; il était évident que le cardinal mesurait d'avance les termes dans lesquels il allait parler, et que Milady recueillait toutes ses facultés intellectuelles pour comprendre les choses qu'il allait dire et les graver dans sa mémoire quand elles seraient dites. 

Athos profita de ce moment pour dire à ses deux compagnons de fermer la porte en dedans et pour leur faire signe de venir écouter avec lui. 

Les deux mousquetaires, qui aimaient leurs aises, apportèrent une chaise pour chacun d'eux, et une chaise pour Athos. Tous trois s'assirent alors, leurs têtes rapprochées et l'oreille au guet. 

«Vous allez partir pour Londres, continua le cardinal. Arrivée à Londres, vous irez trouver Buckingham. 

\speak  Je ferai observer à Son Éminence, dit Milady, que depuis l'affaire des ferrets de diamants, pour laquelle le duc m'a toujours soupçonnée, Sa Grâce se défie de moi. 

\speak  Aussi cette fois-ci, dit le cardinal, ne s'agit-il plus de capter sa confiance, mais de se présenter franchement et loyalement à lui comme négociatrice. 

\speak  Franchement et loyalement, répéta Milady avec une indicible expression de duplicité. 

\speak  Oui, franchement et loyalement, reprit le cardinal du même ton; toute cette négociation doit être faite à découvert. 

\speak  Je suivrai à la lettre les instructions de Son Éminence, et j'attends qu'elle me les donne. 

\speak  Vous irez trouver Buckingham de ma part, et vous lui direz que je sais tous les préparatifs qu'il fait mais que je ne m'en inquiète guère, attendu qu'au premier mouvement qu'il risquera, je perds la reine. 

\speak  Croira-t-il que Votre Éminence est en mesure d'accomplir la menace qu'elle lui fait? 

\speak  Oui, car j'ai des preuves. 

\speak  Il faut que je puisse présenter ces preuves à son appréciation. 

\speak  Sans doute, et vous lui direz que je publie le rapport de Bois-Robert et du marquis de Beautru sur l'entrevue que le duc a eu chez Mme la connétable avec la reine, le soir que Mme la connétable a donné une fête masquée; vous lui direz, afin qu'il ne doute de rien, qu'il y est venu sous le costume du grand mogol que devait porter le chevalier de Guise, et qu'il a acheté à ce dernier moyennant la somme de trois mille pistoles. 

\speak  Bien, Monseigneur. 

\speak  Tous les détails de son entrée au Louvre et de sa sortie pendant la nuit où il s'est introduit au palais sous le costume d'un diseur de bonne aventure italien me sont connus; vous lui direz, pour qu'il ne doute pas encore de l'authenticité de mes renseignements, qu'il avait sous son manteau une grande robe blanche semée de larmes noires, de têtes de mort et d'os en sautoir: car, en cas de surprise, il devait se faire passer pour le fantôme de la Dame blanche qui, comme chacun le sait, revient au Louvre chaque fois que quelque grand événement va s'accomplir. 

\speak  Est-ce tout, Monseigneur? 

\speak  Dites-lui que je sais encore tous les détails de l'aventure d'Amiens, que j'en ferai faire un petit roman, spirituellement tourné, avec un plan du jardin et les portraits des principaux acteurs de cette scène nocturne. 

\speak  Je lui dirai cela. 

\speak  Dites-lui encore que je tiens Montaigu, que Montaigu est à la Bastille, qu'on n'a surpris aucune lettre sur lui, c'est vrai, mais que la torture peut lui faire dire ce qu'il sait, et même\dots ce qu'il ne sait pas. 

\speak  À merveille. 

\speak  Enfin ajoutez que Sa Grâce, dans la précipitation qu'elle a mise à quitter l'île de Ré, oublia dans son logis certaine lettre de Mme de Chevreuse qui compromet singulièrement la reine, en ce qu'elle prouve non seulement que Sa Majesté peut aimer les ennemis du roi, mais encore qu'elle conspire avec ceux de la France. Vous avez bien retenu tout ce que je vous ai dit, n'est-ce pas? 

\speak  Votre Éminence va en juger: le bal de Mme la connétable; la nuit du Louvre; la soirée d'Amiens; l'arrestation de Montaigu; la lettre de Mme de Chevreuse. 

\speak  C'est cela, dit le cardinal, c'est cela: vous avez une bien heureuse mémoire, Milady. 

\speak  Mais, reprit celle à qui le cardinal venait d'adresser ce compliment flatteur, si malgré toutes ces raisons le duc ne se rend pas et continue de menacer la France? 

\speak  Le duc est amoureux comme un fou, ou plutôt comme un niais, reprit Richelieu avec une profonde amertume; comme les anciens paladins, il n'a entrepris cette guerre que pour obtenir un regard de sa belle. S'il sait que cette guerre peut coûter l'honneur et peut-être la liberté à la dame de ses pensées, comme il dit, je vous réponds qu'il y regardera à deux fois. 

\speak  Et cependant, dit Milady avec une persistance qui prouvait qu'elle voulait voir clair jusqu'au bout, dans la mission dont elle allait être chargée, cependant s'il persiste? 

\speak  S'il persiste, dit le cardinal\dots, ce n'est pas probable. 

\speak  C'est possible, dit Milady. 

\speak  S'il persiste\dots» 

Son Éminence fit une pause et reprit\dots 

«S'il persiste, eh bien, j'espérerai dans un de ces événements qui changent la face des États. 

\speak  Si Son Éminence voulait me citer dans l'histoire quelques-uns de ces événements, dit Milady, peut-être partagerais-je sa confiance dans l'avenir. 

\speak  Eh bien, tenez! par exemple, dit Richelieu, lorsqu'en 1610, pour une cause à peu près pareille à celle qui fait mouvoir le duc, le roi Henri IV, de glorieuse mémoire, allait à la fois envahir les Flandres et l'Italie pour frapper à la fois l'Autriche des deux côtés, eh bien, n'est-il pas arrivé un événement qui a sauvé l'Autriche? Pourquoi le roi de France n'aurait-il pas la même chance que l'empereur? 

\speak  Votre Éminence veut parler du coup de couteau de la rue de la Ferronnerie? 

\speak  Justement, dit le cardinal. 

\speak  Votre Éminence ne craint-elle pas que le supplice de Ravaillac épouvante ceux qui auraient un instant l'idée de l'imiter? 

\speak  Il y aura en tout temps et dans tous les pays, surtout si ces pays sont divisés de religion, des fanatiques qui ne demanderont pas mieux que de se faire martyrs. Et tenez, justement il me revient à cette heure que les puritains sont furieux contre le duc de Buckingham et que leurs prédicateurs le désignent comme l'Antéchrist. 

\speak  Eh bien? fit Milady. 

\speak  Eh bien, continua le cardinal d'un air indifférent, il ne s'agirait, pour le moment, par exemple, que de trouver une femme, belle, jeune, adroite, qui eût à se venger elle-même du duc. Une pareille femme peut se rencontrer: le duc est homme à bonnes fortunes, et, s'il a semé bien des amours par ses promesses de constance éternelle, il a dû semer bien des haines aussi par ses éternelles infidélités. 

\speak  Sans doute, dit froidement Milady, une pareille femme peut se rencontrer. 

\speak  Eh bien, une pareille femme, qui mettrait le couteau de Jacques Clément ou de Ravaillac aux mains d'un fanatique, sauverait la France. 

\speak  Oui, mais elle serait complice d'un assassinat. 

\speak  A-t-on jamais connu les complices de Ravaillac ou de Jacques Clément? 

\speak  Non, car peut-être étaient-ils placés trop haut pour qu'on osât les aller chercher là où ils étaient: on ne brûlerait pas le Palais de Justice pour tout le monde, Monseigneur. 

\speak  Vous croyez donc que l'incendie du Palais de Justice a une cause autre que celle du hasard? demanda Richelieu du ton dont il eût fait une question sans aucune importance. 

\speak  Moi, Monseigneur, répondit Milady, je ne crois rien, je cite un fait, voilà tout, seulement, je dis que si je m'appelais Mlle de Monpensier ou la reine Marie de Médicis, je prendrais moins de précautions que j'en prends, m'appelant tout simplement Lady Clarick. 

\speak  C'est juste, dit Richelieu, et que voudriez-vous donc? 

\speak  Je voudrais un ordre qui ratifiât d'avance tout ce que je croirai devoir faire pour le plus grand bien de la France. 

\speak  Mais il faudrait d'abord trouver la femme que j'ai dit, et qui aurait à se venger du duc. 

\speak  Elle est trouvée, dit Milady. 

\speak  Puis il faudrait trouver ce misérable fanatique qui servira d'instrument à la justice de Dieu. 

\speak  On le trouvera. 

\speak  Eh bien, dit le duc, alors il sera temps de réclamer l'ordre que vous demandiez tout à l'heure. 

\speak  Votre Éminence a raison, dit Milady, et c'est moi qui ai eu tort de voir dans la mission dont elle m'honore autre chose que ce qui est réellement, c'est-à-dire d'annoncer à Sa Grâce, de la part de Son Éminence, que vous connaissez les différents déguisements à l'aide desquels il est parvenu à se rapprocher de la reine pendant la fête donnée par Mme la connétable; que vous avez les preuves de l'entrevue accordée au Louvre par la reine à certain astrologue italien qui n'est autre que le duc de Buckingham; que vous avez commandé un petit roman, des plus spirituels, sur l'aventure d'Amiens, avec plan du jardin où cette aventure s'est passée et portraits des acteurs qui y ont figuré; que Montaigu est à la Bastille, et que la torture peut lui faire dire des choses dont il se souvient et même des choses qu'il aurait oubliées; enfin, que vous possédez certaine lettre de Mme de Chevreuse, trouvée dans le logis de Sa Grâce, qui compromet singulièrement, non seulement celle qui l'a écrite, mais encore celle au nom de qui elle a été écrite. Puis, s'il persiste malgré tout cela, comme c'est à ce que je viens de dire que se borne ma mission, je n'aurai plus qu'à prier Dieu de faire un miracle pour sauver la France. C'est bien cela, n'est-ce pas, Monseigneur, et je n'ai pas autre chose à faire? 

\speak  C'est bien cela, reprit sèchement le cardinal. 

\speak  Et maintenant, dit Milady sans paraître remarquer le changement de ton du duc à son égard, maintenant que j'ai reçu les instructions de Votre Éminence à propos de ses ennemis, Monseigneur me permettra-t-il de lui dire deux mots des miens? 

\speak  Vous avez donc des ennemis? demanda Richelieu. 

\speak  Oui, Monseigneur; des ennemis contre lesquels vous me devez tout votre appui, car je me les suis faits en servant Votre Éminence. 

\speak  Et lesquels? répliqua le duc. 

\speak  D'abord une petite intrigante du nom de Bonacieux. 

\speak  Elle est dans la prison de Mantes. 

\speak  C'est-à-dire qu'elle y était, reprit Milady, mais la reine a surpris un ordre du roi, à l'aide duquel elle l'a fait transporter dans un couvent. 

\speak  Dans un couvent? dit le duc. 

\speak  Oui, dans un couvent. 

\speak  Et dans lequel? 

\speak  Je l'ignore, le secret a été bien gardé\dots 

\speak  Je le saurai, moi! 

\speak  Et Votre Éminence me dira dans quel couvent est cette femme? 

\speak  Je n'y vois pas d'inconvénient, dit le cardinal. 

\speak  Bien; maintenant j'ai un autre ennemi bien autrement à craindre pour moi que cette petite Mme Bonacieux. 

\speak  Et lequel? 

\speak  Son amant. 

\speak  Comment s'appelle-t-il? 

\speak  Oh! Votre Éminence le connaît bien, s'écria Milady emportée par la colère, c'est notre mauvais génie à tous deux; c'est celui qui, dans une rencontre avec les gardes de Votre Éminence, a décidé la victoire en faveur des mousquetaires du roi; c'est celui qui a donné trois coups d'épée à de Wardes, votre émissaire, et qui a fait échouer l'affaire des ferrets; c'est celui enfin qui, sachant que c'était moi qui lui avais enlevé Mme Bonacieux, a juré ma mort. 

\speak  Ah! ah! dit le cardinal, je sais de qui vous voulez parler. 

\speak  Je veux parler de ce misérable d'Artagnan. 

\speak  C'est un hardi compagnon, dit le cardinal. 

\speak  Et c'est justement parce que c'est un hardi compagnon qu'il n'en est que plus à craindre. 

\speak  Il faudrait, dit le duc, avoir une preuve de ses intelligences avec Buckingham. 

\speak  Une preuve, s'écria Milady, j'en aurai dix. 

\speak  Eh bien, alors! c'est la chose la plus simple du monde, ayez-moi cette preuve et je l'envoie à la Bastille. 

\speak  Bien, Monseigneur! mais ensuite? 

\speak  Quand on est à la Bastille, il n'y a pas d'ensuite, dit le cardinal d'une voix sourde. Ah! pardieu, continua-t-il, s'il m'était aussi facile de me débarrasser de mon ennemi qu'il m'est facile de me débarrasser des vôtres, et si c'était contre de pareilles gens que vous me demandiez l'impunité!\dots 

\speak  Monseigneur, reprit Milady, troc pour troc, existence pour existence, homme pour homme; donnez-moi celui-là, je vous donne l'autre. 

\speak  Je ne sais pas ce que vous voulez dire, reprit le cardinal, et ne veux même pas le savoir, mais j'ai le désir de vous être agréable et ne vois aucun inconvénient à vous donner ce que vous demandez à l'égard d'une si infime créature; d'autant plus, comme vous me le dites, que ce petit d'Artagnan est un libertin, un duelliste, un traître. 

\speak  Un infâme, Monseigneur, un infâme! 

\speak  Donnez-moi donc du papier, une plume et de l'encre, dit le cardinal. 

\speak  En voici, Monseigneur.» 

Il se fit un instant de silence qui prouvait que le cardinal était occupé à chercher les termes dans lesquels devait être écrit le billet, ou même à l'écrire. Athos, qui n'avait pas perdu un mot de la conversation, prit ses deux compagnons chacun par une main et les conduisit à l'autre bout de la chambre. 

«Eh bien, dit Porthos, que veux-tu, et pourquoi ne nous laisses-tu pas écouter la fin de la conversation? 

\speak  Chut! dit Athos parlant à voix basse, nous en avons entendu tout ce qu'il est nécessaire que nous entendions; d'ailleurs je ne vous empêche pas d'écouter le reste, mais il faut que je sorte. 

\speak  Il faut que tu sortes! dit Porthos; mais si le cardinal te demande, que répondrons-nous? 

\speak  Vous n'attendrez pas qu'il me demande, vous lui direz les premiers que je suis parti en éclaireur parce que certaines paroles de notre hôte m'ont donné à penser que le chemin n'était pas sûr; j'en toucherai d'abord deux mots à l'écuyer du cardinal; le reste me regarde, ne vous en inquiétez pas. 

\speak  Soyez prudent, Athos! dit Aramis. 

\speak  Soyez tranquille, répondit Athos, vous le savez, j'ai du sang-froid.» 

Porthos et Aramis allèrent reprendre leur place près du tuyau de poêle. 

Quant à Athos, il sortit sans aucun mystère, alla prendre son cheval attaché avec ceux de ses deux amis aux tourniquets des contrevents, convainquit en quatre mots l'écuyer de la nécessité d'une avant-garde pour le retour, visita avec affectation l'amorce de ses pistolets, mit l'épée aux dents et suivit, en enfant perdu, la route qui conduisait au camp.