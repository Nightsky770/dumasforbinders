%!TeX root=../musketeersfr.tex 

\chapter[Portsmouth, le 23 Août 1628]{Ce Qui Se Passait À Portsmouth Le 23 Août 1628}

\lettrine{F}{elton} prit congé de Milady comme un frère qui va faire une simple promenade prend congé de sa sœur en lui baisant la main. 

\zz
Toute sa personne paraissait dans son état de calme ordinaire: seulement une lueur inaccoutumée brillait dans ses yeux, pareille à un reflet de fièvre; son front était plus pâle encore que de coutume; ses dents étaient serrées, et sa parole avait un accent bref et saccadé qui indiquait que quelque chose de sombre s'agitait en lui. 

Tant qu'il resta sur la barque qui le conduisait à terre, il demeura le visage tourné du côté de Milady, qui, debout sur le pont, le suivait des yeux. Tous deux étaient assez rassurés sur la crainte d'être poursuivis: on n'entrait jamais dans la chambre de Milady avant neuf heures; et il fallait trois heures pour venir du château à Londres. 

Felton mit pied à terre, gravit la petite crête qui conduisait au haut de la falaise, salua Milady une dernière fois, et prit sa course vers la ville. 

Au bout de cent pas, comme le terrain allait en descendant, il ne pouvait plus voir que le mât du sloop. 

Il courut aussitôt dans la direction de Portsmouth, dont il voyait en face de lui, à un demi-mille à peu près, se dessiner dans la brume du matin les tours et les maisons. 

Au-delà de Portsmouth, la mer était couverte de vaisseaux dont on voyait les mâts, pareils à une forêt de peupliers dépouillés par l'hiver, se balancer sous le souffle du vent. 

Felton, dans sa marche rapide, repassait ce que dix années de méditations ascétiques et un long séjour au milieu des puritains lui avaient fourni d'accusations vraies ou fausses contre le favori de Jacques VI et de Charles I\ier. 

Lorsqu'il comparait les crimes publics de ce ministre, crimes éclatants, crimes européens, si on pouvait le dire, avec les crimes privés et inconnus dont l'avait chargé Milady, Felton trouvait que le plus coupable des deux hommes que renfermait Buckingham était celui dont le public ne connaissait pas la vie. C'est que son amour si étrange, si nouveau, si ardent, lui faisait voir les accusations infâmes et imaginaires de Lady de Winter, comme on voit au travers d'un verre grossissant, à l'état de monstres effroyables, des atomes imperceptibles en réalité auprès d'une fourmi. 

La rapidité de sa course allumait encore son sang: l'idée qu'il laissait derrière lui, exposée à une vengeance effroyable, la femme qu'il aimait ou plutôt qu'il adorait comme une sainte, l'émotion passée, sa fatigue présente, tout exaltait encore son âme au-dessus des sentiments humains. 

Il entra à Portsmouth vers les huit heures du matin; toute la population était sur pied; le tambour battait dans les rues et sur le port; les troupes d'embarquement descendaient vers la mer. 

Felton arriva au palais de l'Amirauté, couvert de poussière et ruisselant de sueur; son visage, ordinairement si pâle, était pourpre de chaleur et de colère. La sentinelle voulut le repousser; mais Felton appela le chef du poste, et tirant de sa poche la lettre dont il était porteur: 

«Message pressé de la part de Lord de Winter», dit-il. 

Au nom de Lord de Winter, qu'on savait l'un des plus intimes de Sa Grâce, le chef de poste donna l'ordre de laisser passer Felton, qui, du reste, portait lui-même l'uniforme d'officier de marine. 

Felton s'élança dans le palais. 

Au moment où il entrait dans le vestibule un homme entrait aussi, poudreux, hors d'haleine, laissant à la porte un cheval de poste qui en arrivant tomba sur les deux genoux. 

Felton et lui s'adressèrent en même temps à Patrick, le valet de chambre de confiance du duc. Felton nomma le baron de Winter, l'inconnu ne voulut nommer personne, et prétendit que c'était au duc seul qu'il pouvait se faire connaître. Tous deux insistaient pour passer l'un avant l'autre. 

Patrick, qui savait que Lord de Winter était en affaires de service et en relations d'amitié avec le duc, donna la préférence à celui qui venait en son nom. L'autre fut forcé d'attendre, et il fut facile de voir combien il maudissait ce retard. 

Le valet de chambre fit traverser à Felton une grande salle dans laquelle attendaient les députés de La Rochelle conduits par le prince de Soubise, et l'introduisit dans un cabinet où Buckingham, sortant du bain, achevait sa toilette, à laquelle, cette fois comme toujours, il accordait une attention extraordinaire. 

«Le lieutenant Felton, dit Patrick, de la part de Lord de Winter. 

\speak  De la part de Lord de Winter! répéta Buckingham, faites entrer.» 

Felton entra. En ce moment Buckingham jetait sur un canapé une riche robe de chambre brochée d'or, pour endosser un pourpoint de velours bleu tout brodé de perles. 

«Pourquoi le baron n'est-il pas venu lui-même? demanda Buckingham, je l'attendais ce matin. 

\speak  Il m'a chargé de dire à Votre Grâce, répondit Felton, qu'il regrettait fort de ne pas avoir cet honneur, mais qu'il en était empêché par la garde qu'il est obligé de faire au château. 

\speak  Oui, oui, dit Buckingham, je sais cela, il a une prisonnière. 

\speak  C'est justement de cette prisonnière que je voulais parler à Votre Grâce, reprit Felton. 

\speak  Eh bien, parlez. 

\speak  Ce que j'ai à vous dire ne peut être entendu que de vous, Milord. 

\speak  Laissez-nous, Patrick, dit Buckingham, mais tenez-vous à portée de la sonnette; je vous appellerai tout à l'heure.» 

Patrick sortit. 

«Nous sommes seuls, monsieur, dit Buckingham, parlez. 

\speak  Milord, dit Felton, le baron de Winter vous a écrit l'autre jour pour vous prier de signer un ordre d'embarquement relatif à une jeune femme nommée Charlotte Backson. 

\speak  Oui, monsieur, et je lui ai répondu de m'apporter ou de m'envoyer cet ordre et que je le signerais. 

\speak  Le voici, Milord. 

\speak  Donnez», dit le duc. 

Et, le prenant des mains de Felton, il jeta sur le papier un coup d'œil rapide. Alors, s'apercevant que c'était bien celui qui lui était annoncé, il le posa sur la table, prit une plume et s'apprêta à signer. 

«Pardon, Milord, dit Felton arrêtant le duc, mais Votre Grâce sait-elle que le nom de Charlotte Backson n'est pas le véritable nom de cette jeune femme? 

\speak  Oui, monsieur, je le sais, répondit le duc en trempant la plume dans l'encrier. 

\speak  Alors, Votre Grâce connaît son véritable nom? demanda Felton d'une voix brève. 

\speak  Je le connais.» 

Le duc approcha la plume du papier. 

«Et, connaissant ce véritable nom, reprit Felton, Monseigneur signera tout de même? 

\speak  Sans doute, dit Buckingham, et plutôt deux fois qu'une. 

\speak  Je ne puis croire, continua Felton d'une voix qui devenait de plus en plus brève et saccadée, que Sa Grâce sache qu'il s'agit de Lady de Winter\dots 

\speak  Je le sais parfaitement, quoique je sois étonné que vous le sachiez, vous! 

\speak  Et Votre Grâce signera cet ordre sans remords?» 

Buckingham regarda le jeune homme avec hauteur. 

«Ah çà, monsieur, savez-vous bien, lui dit-il, que vous me faites là d'étranges questions, et que je suis bien simple d'y répondre? 

\speak  Répondez-y, Monseigneur, dit Felton, la situation est plus grave que vous ne le croyez peut-être.» 

Buckingham pensa que le jeune homme, venant de la part de Lord de Winter, parlait sans doute en son nom et se radoucit. 

«Sans remords aucun, dit-il, et le baron sait comme moi que Milady de Winter est une grande coupable, et que c'est presque lui faire grâce que de borner sa peine à l'extradition.» 

Le duc posa sa plume sur le papier. 

«Vous ne signerez pas cet ordre, Milord! dit Felton en faisant un pas vers le duc. 

\speak  Je ne signerai pas cet ordre, dit Buckingham, et pourquoi? 

\speak  Parce que vous descendrez en vous-même, et que vous rendrez justice à Milady. 

\speak  On lui rendra justice en l'envoyant à Tyburn, dit Buckingham; Milady est une infâme. 

\speak  Monseigneur, Milady est un ange, vous le savez bien, et je vous demande sa liberté. 

\speak  Ah çà, dit Buckingham, êtes-vous fou de me parler ainsi? 

\speak  Milord, excusez-moi! je parle comme je puis; je me contiens. Cependant, Milord, songez à ce que vous allez faire, et craignez d'outrepasser la mesure! 

\speak  Plaît-il?\dots Dieu me pardonne! s'écria Buckingham, mais je crois qu'il me menace! 

\speak  Non, Milord, je prie encore, et je vous dis: une goutte d'eau suffit pour faire déborder le vase plein, une faute légère peut attirer le châtiment sur la tête épargnée malgré tant de crimes. 

\speak  Monsieur Felton, dit Buckingham, vous allez sortir d'ici et vous rendre aux arrêts sur-le-champ. 

\speak  Vous allez m'écouter jusqu'au bout, Milord. Vous avez séduit cette jeune fille, vous l'avez outragée, souillée; réparez vos crimes envers elle, laissez-la partir librement, et je n'exigerai pas autre chose de vous. 

\speak  Vous n'exigerez pas? dit Buckingham regardant Felton avec étonnement et appuyant sur chacune des syllabes des trois mots qu'il venait de prononcer. 

\speak  Milord, continua Felton s'exaltant à mesure qu'il parlait, Milord, prenez garde, toute l'Angleterre est lasse de vos iniquités; Milord, vous avez abusé de la puissance royale que vous avez presque usurpée; Milord, vous êtes en horreur aux hommes et à Dieu; Dieu vous punira plus tard, mais, moi, je vous punirai aujourd'hui. 

\speak  Ah! ceci est trop fort!» cria Buckingham en faisant un pas vers la porte. 

Felton lui barra le passage. 

«Je vous le demande humblement, dit-il, signez l'ordre de mise en liberté de Lady de Winter; songez que c'est la femme que vous avez déshonorée. 

\speak  Retirez-vous, monsieur, dit Buckingham, ou j'appelle et vous fais mettre aux fers. 

\speak  Vous n'appellerez pas, dit Felton en se jetant entre le duc et la sonnette placée sur un guéridon incrusté d'argent; prenez garde, Milord, vous voilà entre les mains de Dieu. 

\speak  Dans les mains du diable, vous voulez dire, s'écria Buckingham en élevant la voix pour attirer du monde, sans cependant appeler directement. 

\speak  Signez, Milord, signez la liberté de Lady de Winter, dit Felton en poussant un papier vers le duc. 

\speak  De force! vous moquez-vous? holà, Patrick! 

\speak  Signez, Milord! 

\speak  Jamais! 

\speak  Jamais! 

\speak  À moi», cria le duc, et en même temps il sauta sur son épée. 

Mais Felton ne lui donna pas le temps de la tirer: il tenait tout ouvert et caché dans son pourpoint le couteau dont s'était frappée Milady; d'un bond il fut sur le duc. 

En ce moment Patrick entrait dans la salle en criant: 

«Milord, une lettre de France! 

\speak  De France!» s'écria Buckingham, oubliant tout en pensant de qui lui venait cette lettre. 

Felton profita du moment et lui enfonça dans le flanc le couteau jusqu'au manche. 

«Ah! traître! cria Buckingham, tu m'as tué\dots 

\speak  Au meurtre!» hurla Patrick. 

Felton jeta les yeux autour de lui pour fuir, et, voyant la porte libre, s'élança dans la chambre voisine, qui était celle où attendaient, comme nous l'avons dit, les députés de La Rochelle, la traversa tout en courant et se précipita vers l'escalier; mais, sur la première marche, il rencontra Lord de Winter, qui, le voyant pâle, égaré, livide, taché de sang à la main et à la figure, lui sauta au cou en s'écriant: 

«Je le savais, je l'avais deviné et j'arrive trop tard d'une minute! oh! malheureux que je suis!» 

Felton ne fit aucune résistance; Lord de Winter le remit aux mains des gardes, qui le conduisirent, en attendant de nouveaux ordres, sur une petite terrasse dominant la mer, et il s'élança dans le cabinet de Buckingham. 

Au cri poussé par le duc, à l'appel de Patrick, l'homme que Felton avait rencontré dans l'antichambre se précipita dans le cabinet. 

Il trouva le duc couché sur un sofa, serrant sa blessure dans sa main crispée. 

«La Porte, dit le duc d'une voix mourante, La Porte, viens-tu de sa part? 

\speak  Oui, Monseigneur, répondit le fidèle serviteur d'Anne d'Autriche, mais trop tard peut-être. 

\speak  Silence, La Porte! on pourrait vous entendre; Patrick, ne laissez entrer personne: oh! je ne saurai pas ce qu'elle me fait dire! mon Dieu, je me meurs!» 

Et le duc s'évanouit. 

Cependant, Lord de Winter, les députés, les chefs de l'expédition, les officiers de la maison de Buckingham, avaient fait irruption dans sa chambre; partout des cris de désespoir retentissaient. La nouvelle qui emplissait le palais de plaintes et de gémissements en déborda bientôt partout et se répandit par la ville. 

Un coup de canon annonça qu'il venait de se passer quelque chose de nouveau et d'inattendu. 

Lord de Winter s'arrachait les cheveux. 

«Trop tard d'une minute! s'écriait-il, trop tard d'une minute! oh! mon Dieu, mon Dieu, quel malheur!» 

En effet, on était venu lui dire à sept heures du matin qu'une échelle de corde flottait à une des fenêtres du château; il avait couru aussitôt à la chambre de Milady, avait trouvé la chambre vide et la fenêtre ouverte, les barreaux sciés, il s'était rappelé la recommandation verbale que lui avait fait transmettre d'Artagnan par son messager, il avait tremblé pour le duc, et, courant à l'écurie, sans prendre le temps de faire seller son cheval, avait sauté sur le premier venu, était accouru ventre à terre, et sautant à bas dans la cour, avait monté précipitamment l'escalier, et, sur le premier degré, avait, comme nous l'avons dit, rencontré Felton. 

Cependant le duc n'était pas mort: il revint à lui, rouvrit les yeux, et l'espoir rentra dans tous les cœurs. 

«Messieurs, dit-il, laissez-moi seul avec Patrick et La Porte. 

«Ah! c'est vous, de Winter! vous m'avez envoyé ce matin un singulier fou, voyez l'état dans lequel il m'a mis! 

\speak  Oh! Milord! s'écria le baron, je ne m'en consolerai jamais. 

\speak  Et tu aurais tort, mon cher de Winter, dit Buckingham en lui tendant la main, je ne connais pas d'homme qui mérite d'être regretté pendant toute la vie d'un autre homme; mais laisse-nous, je t'en prie.» 

Le baron sortit en sanglotant. 

Il ne resta dans le cabinet que le duc blessé, La Porte et Patrick. 

On cherchait un médecin, qu'on ne pouvait trouver. 

«Vous vivrez, Milord, vous vivrez, répétait, à genoux devant le sofa du duc, le messager d'Anne d'Autriche. 

\speak  Que m'écrivait-elle? dit faiblement Buckingham tout ruisselant de sang et domptant, pour parler de celle qu'il aimait, d'atroces douleurs, que m'écrivait-elle? Lis-moi sa lettre. 

\speak  Oh! Milord! fit La Porte. 

\speak  Obéis, La Porte; ne vois-tu pas que je n'ai pas de temps à perdre?» 

La Porte rompit le cachet et plaça le parchemin sous les yeux du duc; mais Buckingham essaya vainement de distinguer l'écriture. 

«Lis donc, dit-il, lis donc, je n'y vois plus; lis donc! car bientôt peut-être je n'entendrai plus, et je mourrai sans savoir ce qu'elle m'a écrit.» 

La Porte ne fit plus de difficulté, et lut: 

\begin{mail}{}{Milord,}
	
Par ce que j'ai, depuis que je vous connais, souffert par vous et pour vous, je vous conjure, si vous avez souci de mon repos, d'interrompre les grands armements que vous faites contre la France et de cesser une guerre dont on dit tout haut que la religion est la cause visible, et tout bas que votre amour pour moi est la cause cachée. Cette guerre peut non seulement amener pour la France et pour l'Angleterre de grandes catastrophes, mais encore pour vous, Milord, des malheurs dont je ne me consolerais pas.

Veillez sur votre vie, que l'on menace et qui me sera chère du moment où je ne serai pas obligée de voir en vous un ennemi.

\closeletter[Votre affectionnée,]{Anne}
\end{mail}

Buckingham rappela tous les restes de sa vie pour écouter cette lecture; puis, lorsqu'elle fut finie, comme s'il eût trouvé dans cette lettre un amer désappointement: 

«N'avez-vous donc pas autre chose à me dire de vive voix, La Porte? demanda-t-il. 

\speak  Si fait, Monseigneur: la reine m'avait chargé de vous dire de veiller sur vous, car elle avait eu avis qu'on voulait vous assassiner. 

\speak  Et c'est tout, c'est tout? reprit Buckingham avec impatience. 

\speak  Elle m'avait encore chargé de vous dire qu'elle vous aimait toujours. 

\speak  Ah! fit Buckingham, Dieu soit loué! ma mort ne sera donc pas pour elle la mort d'un étranger!\dots» 

La Porte fondit en larmes. 

«Patrick, dit le duc, apportez-moi le coffret où étaient les ferrets de diamants.» 

Patrick apporta l'objet demandé, que La Porte reconnut pour avoir appartenu à la reine. 

«Maintenant le sachet de satin blanc, où son chiffre est brodé en perles.» 

Patrick obéit encore. 

«Tenez, La Porte, dit Buckingham, voici les seuls gages que j'eusse à elle, ce coffret d'argent, et ces deux lettres. Vous les rendrez à Sa Majesté; et pour dernier souvenir\dots (il chercha autour de lui quelque objet précieux)\dots vous y joindrez\dots» 

Il chercha encore; mais ses regards obscurcis par la mort ne rencontrèrent que le couteau tombé des mains de Felton, et fumant encore du sang vermeil étendu sur la lame. 

«Et vous y joindrez ce couteau», dit le duc en serrant la main de La Porte. 

Il put encore mettre le sachet au fond du coffret d'argent, y laissa tomber le couteau en faisant signe à La Porte qu'il ne pouvait plus parler; puis, dans une dernière convulsion, que cette fois il n'avait plus la force de combattre, il glissa du sofa sur le parquet. 

Patrick poussa un grand cri. 

Buckingham voulut sourire une dernière fois; mais la mort arrêta sa pensée, qui resta gravée sur son front comme un dernier baiser d'amour. 

En ce moment le médecin du duc arriva tout effaré; il était déjà à bord du vaisseau amiral, on avait été obligé d'aller le chercher là. 

Il s'approcha du duc, prit sa main, la garda un instant dans la sienne, et la laissa retomber. 

«Tout est inutile, dit-il, il est mort. 

\speak  Mort, mort!» s'écria Patrick. 

À ce cri toute la foule rentra dans la salle, et partout ce ne fut que consternation et que tumulte. 

Aussitôt que Lord de Winter vit Buckingham expiré, il courut à Felton, que les soldats gardaient toujours sur la terrasse du palais. 

«Misérable! dit-il au jeune homme qui, depuis la mort de Buckingham, avait retrouvé ce calme et ce sang-froid qui ne devaient plus l'abandonner; misérable! qu'as-tu fait? 

\speak  Je me suis vengé, dit-il. 

\speak  Toi! dit le baron; dis que tu as servi d'instrument à cette femme maudite; mais je te le jure, ce crime sera son dernier crime. 

\speak  Je ne sais ce que vous voulez dire, reprit tranquillement Felton, et j'ignore de qui vous voulez parler, Milord; j'ai tué M. de Buckingham parce qu'il a refusé deux fois à vous-même de me nommer capitaine: je l'ai puni de son injustice, voilà tout.» 

De Winter, stupéfait, regardait les gens qui liaient Felton, et ne savait que penser d'une pareille insensibilité. 

Une seule chose jetait cependant un nuage sur le front pur de Felton. À chaque bruit qu'il entendait, le naïf puritain croyait reconnaître les pas et la voix de Milady venant se jeter dans ses bras pour s'accuser et se perdre avec lui. 

Tout à coup il tressaillit, son regard se fixa sur un point de la mer, que de la terrasse où il se trouvait on dominait tout entière; avec ce regard d'aigle du marin, il avait reconnu, là où un autre n'aurait vu qu'un goéland se balançant sur les flots, la voile du sloop qui se dirigeait vers les côtes de France. 

Il pâlit, porta la main à son cœur, qui se brisait, et comprit toute la trahison. 

«Une dernière grâce, Milord! dit-il au baron. 

\speak  Laquelle? demanda celui-ci. 

\speak  Quelle heure est-il?» 

Le baron tira sa montre. 

«Neuf heures moins dix minutes», dit-il. 

Milady avait avancé son départ d'une heure et demie dès qu'elle avait entendu le coup de canon qui annonçait le fatal événement, elle avait donné l'ordre de lever l'ancre. 

La barque voguait sous un ciel bleu à une grande distance de la côte. 

«Dieu l'a voulu», dit Felton avec la résignation du fanatique, mais cependant sans pouvoir détacher les yeux de cet esquif à bord duquel il croyait sans doute distinguer le blanc fantôme de celle à qui sa vie allait être sacrifiée. 

De Winter suivit son regard, interrogea sa souffrance et devina tout. 

«Sois puni seul d'abord, misérable, dit Lord de Winter à Felton, qui se laissait entraîner les yeux tournés vers la mer; mais je te jure, sur la mémoire de mon frère que j'aimais tant, que ta complice n'est pas sauvée.» 

Felton baissa la tête sans prononcer une syllabe. 

Quant à de Winter, il descendit rapidement l'escalier et se rendit au port. 