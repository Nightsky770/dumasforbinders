%!TeX root=../musketeersfr.tex 

\chapter{Scène Conjugale}

\lettrine{C}{omme} l'avait prévu Athos, le cardinal ne tarda point à descendre; il ouvrit la porte de la chambre où étaient entrés les mousquetaires, et trouva Porthos faisant une partie de dés acharnée avec Aramis. D'un coup d'œil rapide, il fouilla tous les coins de la salle, et vit qu'un de ses hommes lui manquait. 

«Qu'est devenu M. Athos? demanda-t-il. 

\speak  Monseigneur, répondit Porthos, il est parti en éclaireur sur quelques propos de notre hôte, qui lui ont fait croire que la route n'était pas sûre. 

\speak  Et vous, qu'avez-vous fait, monsieur Porthos? 

\speak  J'ai gagné cinq pistoles à Aramis. 

\speak  Et maintenant, vous pouvez revenir avec moi? 

\speak  Nous sommes aux ordres de Votre Éminence. 

\speak  À cheval donc, messieurs, car il se fait tard.» 

L'écuyer était à la porte, et tenait en bride le cheval du cardinal. Un peu plus loin, un groupe de deux hommes et de trois chevaux apparaissait dans l'ombre; ces deux hommes étaient ceux qui devaient conduire Milady au fort de La Pointe, et veiller à son embarquement. 

L'écuyer confirma au cardinal ce que les deux mousquetaires lui avaient déjà dit à propos d'Athos. Le cardinal fit un geste approbateur, et reprit la route, s'entourant au retour des mêmes précautions qu'il avait prises au départ. 

Laissons-le suivre le chemin du camp, protégé par l'écuyer et les deux mousquetaires, et revenons à Athos. 

Pendant une centaine de pas, il avait marché de la même allure; mais, une fois hors de vue, il avait lancé son cheval à droite, avait fait un détour, et était revenu à une vingtaine de pas, dans le taillis, guetter le passage de la petite troupe; ayant reconnu les chapeaux bordés de ses compagnons et la frange dorée du manteau de M. le cardinal, il attendit que les cavaliers eussent tourné l'angle de la route, et, les ayant perdus de vue, il revint au galop à l'auberge, qu'on lui ouvrit sans difficulté. 

L'hôte le reconnut. 

«Mon officier, dit Athos, a oublié de faire à la dame du premier une recommandation importante, il m'envoie pour réparer son oubli. 

\speak  Montez, dit l'hôte, elle est encore dans sa chambre.» 

Athos profita de la permission, monta l'escalier de son pas le plus léger, arriva sur le carré, et, à travers la porte entrouverte, il vit Milady qui attachait son chapeau. 

Il entra dans la chambre, et referma la porte derrière lui. 

Au bruit qu'il fit en repoussant le verrou, Milady se retourna. 

Athos était debout devant la porte, enveloppé dans son manteau, son chapeau rabattu sur ses yeux. 

En voyant cette figure muette et immobile comme une statue, Milady eut peur. 

«Qui êtes-vous? et que demandez-vous?» s'écria-t-elle. «Allons, c'est bien elle!» murmura Athos. 

Et, laissant tomber son manteau, et relevant son feutre, il s'avança vers Milady. 

«Me reconnaissez-vous, madame?» dit-il. 

Milady fit un pas en avant, puis recula comme à la vue d'un serpent. 

«Allons, dit Athos, c'est bien, je vois que vous me reconnaissez. 

\speak  Le comte de La Fère! murmura Milady en pâlissant et en reculant jusqu'à ce que la muraille l'empêchât d'aller plus loin. 

\speak  Oui, Milady, répondit Athos, le comte de La Fère en personne, qui revient tout exprès de l'autre monde pour avoir le plaisir de vous voir. Asseyons-nous donc, et causons, comme dit Monseigneur le cardinal.» 

Milady, dominée par une terreur inexprimable, s'assit sans proférer une seule parole. 

«Vous êtes donc un démon envoyé sur la terre? dit Athos. Votre puissance est grande, je le sais; mais vous savez aussi qu'avec l'aide de Dieu les hommes ont souvent vaincu les démons les plus terribles. Vous vous êtes déjà trouvée sur mon chemin, je croyais vous avoir terrassée, madame; mais, ou je me trompai, ou l'enfer vous a ressuscitée.» 

Milady, à ces paroles qui lui rappelaient des souvenirs effroyables, baissa la tête avec un gémissement sourd. 

«Oui, l'enfer vous a ressuscitée, reprit Athos, l'enfer vous a faite riche, l'enfer vous a donné un autre nom, l'enfer vous a presque refait même un autre visage; mais il n'a effacé ni les souillures de votre âme, ni la flétrissure de votre corps.» 

Milady se leva comme mue par un ressort, et ses yeux lancèrent des éclairs. Athos resta assis. 

«Vous me croyiez mort, n'est-ce pas, comme je vous croyais morte? et ce nom d'Athos avait caché le comte de La Fère, comme le nom de Milady Clarick avait caché Anne de Breuil! N'était-ce pas ainsi que vous vous appeliez quand votre honoré frère nous a mariés? Notre position est vraiment étrange, poursuivit Athos en riant; nous n'avons vécu jusqu'à présent l'un et l'autre que parce que nous nous croyions morts, et qu'un souvenir gêne moins qu'une créature, quoique ce soit chose dévorante parfois qu'un souvenir! 

\speak  Mais enfin, dit Milady d'une voix sourde, qui vous ramène vers moi? et que me voulez-vous? 

\speak  Je veux vous dire que, tout en restant invisible à vos yeux, je ne vous ai pas perdue de vue, moi! 

\speak  Vous savez ce que j'ai fait? 

\speak  Je puis vous raconter jour par jour vos actions, depuis votre entrée au service du cardinal jusqu'à ce soir.» 

Un sourire d'incrédulité passa sur les lèvres pâles de Milady. 

«Écoutez: c'est vous qui avez coupé les deux ferrets de diamants sur l'épaule du duc de Buckingham; c'est vous qui avez fait enlever Mme Bonacieux; c'est vous qui, amoureuse de de Wardes, et croyant passer la nuit avec lui, avez ouvert votre porte à M. d'Artagnan; c'est vous qui, croyant que de Wardes vous avait trompée, avez voulu le faire tuer par son rival; c'est vous qui, lorsque ce rival eut découvert votre infâme secret, avez voulu le faire tuer à son tour par deux assassins que vous avez envoyés à sa poursuite; c'est vous qui, voyant que les balles avaient manqué leur coup, avez envoyé du vin empoisonné avec une fausse lettre, pour faire croire à votre victime que ce vin venait de ses amis; c'est vous, enfin, qui venez là, dans cette chambre, assise sur cette chaise où je suis, de prendre avec le cardinal de Richelieu l'engagement de faire assassiner le duc de Buckingham, en échange de la promesse qu'il vous a faite de vous laisser assassiner d'Artagnan.» 

Milady était livide. 

«Mais vous êtes donc Satan? dit-elle. 

\speak  Peut-être, dit Athos; mais, en tout cas, écoutez bien ceci: Assassinez ou faites assassiner le duc de Buckingham, peu m'importe! je ne le connais pas: d'ailleurs c'est un Anglais; mais ne touchez pas du bout du doigt à un seul cheveu de d'Artagnan, qui est un fidèle ami que j'aime et que je défends, ou, je vous le jure par la tête de mon père, le crime que vous aurez commis sera le dernier. 

\speak  M. d'Artagnan m'a cruellement offensée, dit Milady d'une voix sourde, M. d'Artagnan mourra. 

\speak  En vérité, cela est-il possible qu'on vous offense, madame? dit en riant Athos; il vous a offensée, et il mourra? 

\speak  Il mourra, reprit Milady; elle d'abord, lui ensuite.» 

Athos fut saisi comme d'un vertige: la vue de cette créature, qui n'avait rien d'une femme, lui rappelait des souvenirs terribles; il pensa qu'un jour, dans une situation moins dangereuse que celle où il se trouvait, il avait déjà voulu la sacrifier à son honneur; son désir de meurtre lui revint brûlant et l'envahit comme une fièvre ardente: il se leva à son tour, porta la main à sa ceinture, en tira un pistolet et l'arma. 

Milady, pâle comme un cadavre, voulut crier, mais sa langue glacée ne put proférer qu'un son rauque qui n'avait rien de la parole humaine et qui semblait le râle d'une bête fauve; collée contre la sombre tapisserie, elle apparaissait, les cheveux épars, comme l'image effrayante de la terreur. 

Athos leva lentement son pistolet, étendit le bras de manière que l'arme touchât presque le front de Milady puis, d'une voix d'autant plus terrible qu'elle avait le calme suprême d'une inflexible résolution: 

«Madame, dit-il, vous allez à l'instant même me remettre le papier que vous a signé le cardinal, ou, sur mon âme, je vous fais sauter la cervelle.» 

Avec un autre homme Milady aurait pu conserver quelque doute, mais elle connaissait Athos; cependant elle resta immobile. 

«Vous avez une seconde pour vous décider», dit-il. 

Milady vit à la contraction de son visage que le coup allait partir; elle porta vivement la main à sa poitrine, en tira un papier et le tendit à Athos. 

«Tenez, dit-elle, et soyez maudit!» 

Athos prit le papier, repassa le pistolet à sa ceinture, s'approcha de la lampe pour s'assurer que c'était bien celui-là, le déplia et lut: 

\begin{mail}{3 \textit{décembre} 1627.}

C'est par mon ordre et pour le bien de l'État que le porteur du présent a fait ce qu'il a fait.
\closeletter{Richelieu}
\end{mail}

«Et maintenant, dit Athos en reprenant son manteau et en replaçant son feutre sur sa tête, maintenant que je t'ai arraché les dents, vipère, mords si tu peux.» 

Et il sortit de la chambre sans même regarder en arrière. 

À la porte il trouva les deux hommes et le cheval qu'ils tenaient en main. 

«Messieurs, dit-il, l'ordre de Monseigneur, vous le savez, est de conduire cette femme, sans perdre de temps, au fort de La Pointe et de ne la quitter que lorsqu'elle sera à bord.» 

Comme ces paroles s'accordaient effectivement avec l'ordre qu'ils avaient reçu, ils inclinèrent la tête en signe d'assentiment. 

Quant à Athos, il se mit légèrement en selle et partit au galop; seulement, au lieu de suivre la route, il prit à travers champs, piquant avec vigueur son cheval et de temps en temps s'arrêtant pour écouter. 

Dans une de ces haltes, il entendit sur la route le pas de plusieurs chevaux. Il ne douta point que ce ne fût le cardinal et son escorte. Aussitôt il fit une nouvelle pointe en avant, bouchonna son cheval avec de la bruyère et des feuilles d'arbres, et vint se mettre en travers de la route à deux cents pas du camp à peu près. 

«Qui vive? cria-t-il de loin quand il aperçut les cavaliers. 

\speak  C'est notre brave mousquetaire, je crois, dit le cardinal. 

\speak  Oui, Monseigneur, répondit Athos. C'est lui-même. 

\speak  Monsieur Athos, dit Richelieu, recevez tous mes remerciements pour la bonne garde que vous nous avez faite; messieurs, nous voici arrivés: prenez la porte à gauche, le mot d'ordre est \textit{Roi} et \textit{Ré}.» 

En disant ces mots, le cardinal salua de la tête les trois amis, et prit à droite suivi de son écuyer; car, cette nuit-là, lui-même couchait au camp. 

«Eh bien! dirent ensemble Porthos et Aramis lorsque le cardinal fut hors de la portée de la voix, eh bien il a signé le papier qu'elle demandait. 

\speak  Je le sais, dit tranquillement Athos, puisque le voici.» 

Et les trois amis n'échangèrent plus une seule parole jusqu'à leur quartier, excepté pour donner le mot d'ordre aux sentinelles. 

Seulement, on envoya Mousqueton dire à Planchet que son maître était prié, en relevant de tranchée, de se rendre à l'instant même au logis des mousquetaires. 

D'un autre côté, comme l'avait prévu Athos, Milady, en retrouvant à la porte les hommes qui l'attendaient, ne fit aucune difficulté de les suivre; elle avait bien eu l'envie un instant de se faire reconduire devant le cardinal et de lui tout raconter, mais une révélation de sa part amenait une révélation de la part d'Athos: elle dirait bien qu'Athos l'avait pendue, mais Athos dirait qu'elle était marquée; elle pensa qu'il valait donc encore mieux garder le silence, partir discrètement, accomplir avec son habileté ordinaire la mission difficile dont elle s'était chargée, puis, toutes les choses accomplies à la satisfaction du cardinal, venir lui réclamer sa vengeance. 

En conséquence, après avoir voyagé toute la nuit, à sept heures du matin elle était au fort de La Pointe, à huit heures elle était embarquée, et à neuf heures le bâtiment, qui, avec des lettres de marque du cardinal, était censé être en partance pour Bayonne, levait l'ancre et faisait voile pour l'Angleterre.