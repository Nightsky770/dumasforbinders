%!TeX root=../musketeersfr.tex 

\chapter{Deuxième Journée De Captivité}

\lettrine{M}{ilady} rêvait qu'elle tenait enfin d'Artagnan, qu'elle assistait à son supplice, et c'était la vue de son sang odieux, coulant sous la hache du bourreau, qui dessinait ce charmant sourire sur les lèvres. 

\zz
Elle dormait comme dort un prisonnier bercé par sa première espérance. 

Le lendemain, lorsqu'on entra dans sa chambre, elle était encore au lit. Felton était dans le corridor: il amenait la femme dont il avait parlé la veille, et qui venait d'arriver; cette femme entra et s'approcha du lit de Milady en lui offrant ses services. 

Milady était habituellement pâle; son teint pouvait donc tromper une personne qui la voyait pour la première fois. 

«J'ai la fièvre, dit-elle; je n'ai pas dormi un seul instant pendant toute cette longue nuit, je souffre horriblement: serez-vous plus humaine qu'on ne l'a été hier avec moi? Tout ce que je demande, au reste, c'est la permission de rester couchée. 

\speak  Voulez-vous qu'on appelle un médecin?» dit la femme. 

Felton écoutait ce dialogue sans dire une parole. 

Milady réfléchissait que plus on l'entourerait de monde, plus elle aurait de monde à apitoyer, et plus la surveillance de Lord de Winter redoublerait; d'ailleurs le médecin pourrait déclarer que la maladie était feinte, et Milady après avoir perdu la première partie ne voulait pas perdre la seconde. 

«Aller chercher un médecin, dit-elle, à quoi bon? ces messieurs ont déclaré hier que mon mal était une comédie, il en serait sans doute de même aujourd'hui; car depuis hier soir, on a eu le temps de prévenir le docteur. 

\speak  Alors, dit Felton impatienté, dites vous-même, madame, quel traitement vous voulez suivre. 

\speak  Eh! le sais-je, moi? mon Dieu! je sens que je souffre, voilà tout, que l'on me donne ce que l'on voudra, peu m'importe. 

\speak  Allez chercher Lord de Winter, dit Felton fatigué de ces plaintes éternelles. 

\speak  Oh! non, non! s'écria Milady, non, monsieur, ne l'appelez pas, je vous en conjure, je suis bien, je n'ai besoin de rien, ne l'appelez pas.» 

Elle mit une véhémence si prodigieuse, une éloquence si entraînante dans cette exclamation, que Felton, entraîné, fit quelques pas dans la chambre. 

«Il est ému», pensa Milady. 

«Cependant, madame, dit Felton, si vous souffrez \textit{réellement}, on enverra chercher un médecin, et si vous nous trompez, eh bien, ce sera tant pis pour vous, mais du moins, de notre côté, nous n'aurons rien à nous reprocher.» 

Milady ne répondit point; mais renversant sa belle tête sur son oreiller, elle fondit en larmes et éclata en sanglots. 

Felton la regarda un instant avec son impassibilité ordinaire; puis voyant que la crise menaçait de se prolonger, il sortit; la femme le suivit. Lord de Winter ne parut pas. 

«Je crois que je commence à voir clair», murmura Milady avec une joie sauvage, en s'ensevelissant sous les draps pour cacher à tous ceux qui pourraient l'épier cet élan de satisfaction intérieure. 

Deux heures s'écoulèrent. 

«Maintenant il est temps que la maladie cesse, dit-elle: levons-nous et obtenons quelque succès dès aujourd'hui; je n'ai que dix jours, et ce soir il y en aura deux d'écoulés. 

En entrant, le matin, dans la chambre de Milady, on lui avait apporté son déjeuner; or elle avait pensé qu'on ne tarderait pas à venir enlever la table, et qu'en ce moment elle reverrait Felton. 

Milady ne se trompait pas. Felton reparut, et, sans faire attention si Milady avait ou non touché au repas, fit un signe pour qu'on emportât hors de la chambre la table, que l'on apportait ordinairement toute servie. 

Felton resta le dernier, il tenait un livre à la main. 

Milady, couchée dans un fauteuil près de la cheminée, belle, pâle et résignée, ressemblait à une vierge sainte attendant le martyre. 

Felton s'approcha d'elle et dit: 

«Lord de Winter, qui est catholique comme vous, madame, a pensé que la privation des rites et des cérémonies de votre religion peut vous être pénible: il consent donc à ce que vous lisiez chaque jour l'ordinaire de \textit{votre messe}, et voici un livre qui en contient le rituel.» 

À l'air dont Felton déposa ce livre sur la petite table près de laquelle était Milady, au ton dont il prononça ces deux mots, \textit{votre messe}, au sourire dédaigneux dont il les accompagna, Milady leva la tête et regarda plus attentivement l'officier. 

Alors, à cette coiffure sévère, à ce costume d'une simplicité exagérée, à ce front poli comme le marbre, mais dur et impénétrable comme lui, elle reconnut un de ces sombres puritains qu'elle avait rencontrés si souvent tant à la cour du roi Jacques qu'à celle du roi de France, où, malgré le souvenir de la Saint-Barthélémy, ils venaient parfois chercher un refuge. 

Elle eut donc une de ces inspirations subites comme les gens de génie seuls en reçoivent dans les grandes crises, dans les moments suprêmes qui doivent décider de leur fortune ou de leur vie. 

Ces deux mots, \textit{votre messe}, et un simple coup d'œil jeté sur Felton, lui avaient en effet révélé toute l'importance de la réponse qu'elle allait faire. 

Mais avec cette rapidité d'intelligence qui lui était particulière, cette réponse toute formulée se présenta sur ses lèvres: 

«Moi! dit-elle avec un accent de dédain monté à l'unisson de celui qu'elle avait remarqué dans la voix du jeune officier, moi, monsieur, \textit{ma messe!} Lord de Winter, le catholique corrompu, sait bien que je ne suis pas de sa religion, et c'est un piège qu'il veut me tendre! 

\speak  Et de quelle religion êtes-vous donc, madame? demanda Felton avec un étonnement que, malgré son empire sur lui-même, il ne put cacher entièrement. 

\speak  Je le dirai, s'écria Milady avec une exaltation feinte, le jour où j'aurai assez souffert pour ma foi.» 

Le regard de Felton découvrit à Milady toute l'étendue de l'espace qu'elle venait de s'ouvrir par cette seule parole. 

Cependant le jeune officier demeura muet et immobile, son regard seul avait parlé. 

«Je suis aux mains de mes ennemis, continua-t-elle avec ce ton d'enthousiasme qu'elle savait familier aux puritains; eh bien, que mon Dieu me sauve ou que je périsse pour mon Dieu! voilà la réponse que je vous prie de faire à Lord de Winter. Et quant à ce livre, ajouta-t-elle en montrant le rituel du bout du doigt, mais sans le toucher, comme si elle eût dû être souillée par cet attouchement, vous pouvez le remporter et vous en servir pour vous-même, car sans doute vous êtes doublement complice de Lord de Winter, complice dans sa persécution, complice dans son hérésie.» 

Felton ne répondit rien, prit le livre avec le même sentiment de répugnance qu'il avait déjà manifesté et se retira pensif. Lord de Winter vint vers les cinq heures du soir; Milady avait eu le temps pendant toute la journée de se tracer son plan de conduite; elle le reçut en femme qui a déjà repris tous ses avantages. 

«Il paraît, dit le baron en s'asseyant dans un fauteuil en face de celui qu'occupait Milady et en étendant nonchalamment ses pieds sur le foyer, il paraît que nous avons fait une petite apostasie! 

\speak  Que voulez-vous dire, monsieur? 

\speak  Je veux dire que depuis la dernière fois que nous nous sommes vus, nous avons changé de religion; auriez-vous épousé un troisième mari protestant, par hasard? 

\speak  Expliquez-vous, Milord, reprit la prisonnière avec majesté, car je vous déclare que j'entends vos paroles, mais que je ne les comprends pas. 

\speak  Alors, c'est que vous n'avez pas de religion du tout; j'aime mieux cela, reprit en ricanant Lord de Winter. 

\speak  Il est certain que cela est plus selon vos principes, reprit froidement Milady. 

\speak  Oh! je vous avoue que cela m'est parfaitement égal. 

\speak  Oh! vous n'avoueriez pas cette indifférence religieuse, Milord, que vos débauches et vos crimes en feraient foi. 

\speak  Hein! vous parlez de débauches, madame Messaline, vous parlez de crimes, Lady Macbeth! Ou j'ai mal entendu, ou vous êtes, pardieu, bien impudente. 

\speak  Vous parlez ainsi parce que vous savez qu'on nous écoute, monsieur, répondit froidement Milady, et que vous voulez intéresser vos geôliers et vos bourreaux contre moi. 

\speak  Mes geôliers! mes bourreaux! Ouais, madame, vous le prenez sur un ton poétique, et la comédie d'hier tourne ce soir à la tragédie. Au reste, dans huit jours vous serez où vous devez être et ma tâche sera achevée. 

\speak  Tâche infâme! tâche impie! reprit Milady avec l'exaltation de la victime qui provoque son juge. 

\speak  Je crois, ma parole d'honneur, dit de Winter en se levant, que la drôlesse devient folle. Allons, allons, calmez-vous, madame la puritaine, ou je vous fais mettre au cachot. Pardieu! c'est mon vin d'Espagne qui vous monte à la tête, n'est-ce pas? mais, soyez tranquille, cette ivresse-là n'est pas dangereuse et n'aura pas de suites.» 

Et Lord de Winter se retira en jurant, ce qui à cette époque était une habitude toute cavalière. 

Felton était en effet derrière la porte et n'avait pas perdu un mot de toute cette scène. 

Milady avait deviné juste. 

«Oui, va! va! dit-elle à son frère, les suites approchent, au contraire, mais tu ne les verras, imbécile, que lorsqu'il ne sera plus temps de les éviter.» 

Le silence se rétablit, deux heures s'écoulèrent; on apporta le souper, et l'on trouva Milady occupée à faire tout haut ses prières, prières qu'elle avait apprises d'un vieux serviteur de son second mari, puritain des plus austères. Elle semblait en extase et ne parut pas même faire attention à ce qui se passait autour d'elle. Felton fit signe qu'on ne la dérangeât point, et lorsque tout fut en état il sortit sans bruit avec les soldats. 

Milady savait qu'elle pouvait être épiée, elle continua donc ses prières jusqu'à la fin, et il lui sembla que le soldat qui était de sentinelle à sa porte ne marchait plus du même pas et paraissait écouter. 

Pour le moment, elle n'en voulait pas davantage, elle se releva, se mit à table, mangea peu et ne but que de l'eau. 

Une heure après on vint enlever la table, mais Milady remarqua que cette fois Felton n'accompagnait point les soldats. 

Il craignait donc de la voir trop souvent. 

Elle se retourna vers le mur pour sourire, car il y avait dans ce sourire une telle expression de triomphe que ce seul sourire l'eût dénoncée. 

Elle laissa encore s'écouler une demi-heure, et comme en ce moment tout faisait silence dans le vieux château, comme on n'entendait que l'éternel murmure de la houle, cette respiration immense de l'océan, de sa voix pure, harmonieuse et vibrante, elle commença le premier couplet de ce psaume alors en entière faveur près des puritains: 

\begin{a4}
	\clearpage
\end{a4}

\begin{verse}
Seigneur, si tu nous abandonnes,\\
C'est pour voir si nous sommes forts;\\
Mais ensuite c'est toi qui donnes\\
De ta céleste main la palme à nos efforts. 
\end{verse}

Ces vers n'étaient pas excellents, il s'en fallait même de beaucoup; mais, comme on le sait, les protestants ne se piquaient pas de poésie. 

Tout en chantant, Milady écoutait: le soldat de garde à sa porte s'était arrêté comme s'il eût été changé en pierre. Milady put donc juger de l'effet qu'elle avait produit. 

Alors elle continua son chant avec une ferveur et un sentiment inexprimables; il lui sembla que les sons se répandaient au loin sous les voûtes et allaient comme un charme magique adoucir le cœur de ses geôliers. Cependant il paraît que le soldat en sentinelle, zélé catholique sans doute, secoua le charme, car à travers la porte: 

«Taisez-vous donc madame, dit-il, votre chanson est triste comme un \textit{De profondis}, et si, outre l'agrément d'être en garnison ici, il faut encore y entendre de pareilles choses, ce sera à n'y point tenir. 

\speak  Silence! dit alors une voix grave, que Milady reconnut pour celle de Felton; de quoi vous mêlez-vous, drôle? Vous a-t-on ordonné d'empêcher cette femme de chanter? Non. On vous a dit de la garder, de tirer sur elle si elle essayait de fuir. Gardez-la; si elle fuit, tuez-la, mais ne changez rien à la consigne.» 

Une expression de joie indicible illumina le visage de Milady, mais cette expression fut fugitive comme le reflet d'un éclair, et, sans paraître avoir entendu le dialogue dont elle n'avait pas perdu un mot, elle reprit en donnant à sa voix tout le charme, toute l'étendue et toute la séduction que le démon y avait mis:
\begin{verse}
Pour tant de pleurs et de misère,\\
Pour mon exil et pour mes fers,\\
J'ai ma jeunesse, ma prière,\\
Et Dieu, qui comptera les maux que j'ai soufferts. 
\end{verse}

Cette voix, d'une étendue inouïe et d'une passion sublime, donnait à la poésie rude et inculte de ces psaumes une magie et une expression que les puritains les plus exaltés trouvaient rarement dans les chants de leurs frères et qu'ils étaient forcés d'orner de toutes les ressources de leur imagination: Felton crut entendre chanter l'ange qui consolait les trois Hébreux dans la fournaise. 

Milady continua: 

\begin{verse}
	Mais le jour de la délivrance\\
	Viendra pour nous, Dieu juste et fort;\\
	Et s'il trompe notre espérance,\\
	Il nous reste toujours le martyre et la mort. 
\end{verse}

Ce couplet, dans lequel la terrible enchanteresse s'efforça de mettre toute son âme, acheva de porter le désordre dans le cœur du jeune officier: il ouvrit brusquement la porte, et Milady le vit apparaître pâle comme toujours, mais les yeux ardents et presque égarés. 

«Pourquoi chantez-vous ainsi, dit-il, et avec une pareille voix? 

\speak  Pardon, monsieur, dit Milady avec douceur, j'oubliais que mes chants ne sont pas de mise dans cette maison. Je vous ai sans doute offensé dans vos croyances; mais c'était sans le vouloir, je vous jure; pardonnez-moi donc une faute qui est peut-être grande, mais qui certainement est involontaire.» 

Milady était si belle dans ce moment, l'extase religieuse dans laquelle elle semblait plongée donnait une telle expression à sa physionomie, que Felton, ébloui, crut voir l'ange que tout à l'heure il croyait seulement entendre. 

«Oui, oui, répondit-il, oui: vous troublez, vous agitez les gens qui habitent ce château.» 

Et le pauvre insensé ne s'apercevait pas lui-même de l'incohérence de ses discours, tandis que Milady plongeait son œil de lynx au plus profond de son cœur. 

«Je me tairai, dit Milady en baissant les yeux avec toute la douceur qu'elle put donner à sa voix, avec toute la résignation qu'elle put imprimer à son maintien. 

\speak  Non, non, madame, dit Felton; seulement, chantez moins haut, la nuit surtout.» 

Et à ces mots, Felton, sentant qu'il ne pourrait pas conserver longtemps sa sévérité à l'égard de la prisonnière, s'élança hors de son appartement. 

«Vous avez bien fait, lieutenant, dit le soldat; ces chants bouleversent l'âme; cependant on finit par s'y accoutumer: sa voix est si belle!» 