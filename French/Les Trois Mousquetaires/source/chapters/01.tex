%!TeX root=../musketeersfr.tex 

\chapter{Les Trois Présents de M. D'Artagnan Père}

\lettrine{L}{e} premier lundi du mois d'avril 1625, le bourg de Meung, où naquit l'auteur du \textit{Roman de la Rose}, semblait être dans une révolution aussi entière que si les huguenots en fussent venus faire une seconde Rochelle. Plusieurs bourgeois, voyant s'enfuir les femmes du côté de la Grande-Rue, entendant les enfants crier sur le seuil des portes, se hâtaient d'endosser la cuirasse et, appuyant leur contenance quelque peu incertaine d'un mousquet ou d'une pertuisane, se dirigeaient vers l'hôtellerie du \textit{Franc-Meunier}, devant laquelle s'empressait, en grossissant de minute en minute, un groupe compact, bruyant et plein de curiosité. 

En ce temps-là les paniques étaient fréquentes, et peu de jours se passaient sans qu'une ville ou l'autre enregistrât sur ses archives quelque événement de ce genre. Il y avait les seigneurs qui guerroyaient entre eux; il y avait le roi qui faisait la guerre au cardinal; il y avait l'Espagnol qui faisait la guerre au roi. Puis, outre ces guerres sourdes ou publiques, secrètes ou patentes, il y avait encore les voleurs, les mendiants, les huguenots, les loups et les laquais, qui faisaient la guerre à tout le monde. Les bourgeois s'armaient toujours contre les voleurs, contre les loups, contre les laquais, --- souvent contre les seigneurs et les huguenots, --- quelquefois contre le roi, --- mais jamais contre le cardinal et l'Espagnol. Il résulta donc de cette habitude prise, que, ce susdit premier lundi du mois d'avril 1625, les bourgeois, entendant du bruit, et ne voyant ni le guidon jaune et rouge, ni la livrée du duc de Richelieu, se précipitèrent du côté de l'hôtel du \textit{Franc-Meunier}. 

Arrivé là, chacun put voir et reconnaître la cause de cette rumeur. 

Un jeune homme\dots --- traçons son portrait d'un seul trait de plume: figurez-vous don Quichotte à dix-huit ans, don Quichotte décorcelé, sans haubert et sans cuissards, don Quichotte revêtu d'un pourpoint de laine dont la couleur bleue s'était transformée en une nuance insaisissable de lie-de-vin et d'azur céleste. Visage long et brun; la pommette des joues saillante, signe d'astuce; les muscles maxillaires énormément développés, indice infaillible auquel on reconnaît le Gascon, même sans béret, et notre jeune homme portait un béret orné d'une espèce de plume; l'œil ouvert et intelligent; le nez crochu, mais finement dessiné; trop grand pour un adolescent, trop petit pour un homme fait, et qu'un œil peu exercé eût pris pour un fils de fermier en voyage, sans sa longue épée qui, pendue à un baudrier de peau, battait les mollets de son propriétaire quand il était à pied, et le poil hérissé de sa monture quand il était à cheval. 

Car notre jeune homme avait une monture, et cette monture était même si remarquable, qu'elle fut remarquée: c'était un bidet du Béarn, âgé de douze ou quatorze ans, jaune de robe, sans crins à la queue, mais non pas sans javarts aux jambes, et qui, tout en marchant la tête plus bas que les genoux, ce qui rendait inutile l'application de la martingale, faisait encore également ses huit lieues par jour. Malheureusement les qualités de ce cheval étaient si bien cachées sous son poil étrange et son allure incongrue, que dans un temps où tout le monde se connaissait en chevaux, l'apparition du susdit bidet à Meung, où il était entré il y avait un quart d'heure à peu près par la porte de Beaugency, produisit une sensation dont la défaveur rejaillit jusqu'à son cavalier. 

Et cette sensation avait été d'autant plus pénible au jeune d'Artagnan (ainsi s'appelait le don Quichotte de cette autre Rossinante), qu'il ne se cachait pas le côté ridicule que lui donnait, si bon cavalier qu'il fût, une pareille monture; aussi avait-il fort soupiré en acceptant le don que lui en avait fait M. d'Artagnan père. Il n'ignorait pas qu'une pareille bête valait au moins vingt livres: il est vrai que les paroles dont le présent avait été accompagné n'avaient pas de prix. 

\speak  Mon fils, avait dit le gentilhomme gascon --- dans ce pur patois de Béarn dont Henri IV n'avait jamais pu parvenir à se défaire ---, mon fils, ce cheval est né dans la maison de votre père, il y a tantôt treize ans, et y est resté depuis ce temps-là, ce qui doit vous porter à l'aimer. Ne le vendez jamais, laissez-le mourir tranquillement et honorablement de vieillesse, et si vous faites campagne avec lui, ménagez-le comme vous ménageriez un vieux serviteur. À la cour, continua M. d'Artagnan père, si toutefois vous avez l'honneur d'y aller, honneur auquel, du reste, votre vieille noblesse vous donne des droits, soutenez dignement votre nom de gentilhomme, qui a été porté dignement par vos ancêtres depuis plus de cinq cents ans. Pour vous et pour les vôtres --- par les vôtres, j'entends vos parents et vos amis ---, ne supportez jamais rien que de M. le cardinal et du roi. C'est par son courage, entendez-vous bien, par son courage seul, qu'un gentilhomme fait son chemin aujourd'hui. Quiconque tremble une seconde laisse peut-être échapper l'appât que, pendant cette seconde justement, la fortune lui tendait. Vous êtes jeune, vous devez être brave par deux raisons: la première, c'est que vous êtes Gascon, et la seconde, c'est que vous êtes mon fils. Ne craignez pas les occasions et cherchez les aventures. Je vous ai fait apprendre à manier l'épée; vous avez un jarret de fer, un poignet d'acier; battez-vous à tout propos; battez-vous d'autant plus que les duels sont défendus, et que, par conséquent, il y a deux fois du courage à se battre. Je n'ai, mon fils, à vous donner que quinze écus, mon cheval et les conseils que vous venez d'entendre. Votre mère y ajoutera la recette d'un certain baume qu'elle tient d'une bohémienne, et qui a une vertu miraculeuse pour guérir toute blessure qui n'atteint pas le cœur. Faites votre profit du tout, et vivez heureusement et longtemps. --- Je n'ai plus qu'un mot à ajouter, et c'est un exemple que je vous propose, non pas le mien, car je n'ai, moi, jamais paru à la cour et n'ai fait que les guerres de religion en volontaire; je veux parler de M. de Tréville, qui était mon voisin autrefois, et qui a eu l'honneur de jouer tout enfant avec notre roi Louis treizième, que Dieu conserve! Quelquefois leurs jeux dégénéraient en bataille et dans ces batailles le roi n'était pas toujours le plus fort. Les coups qu'il en reçut lui donnèrent beaucoup d'estime et d'amitié pour M. de Tréville. Plus tard, M. de Tréville se battit contre d'autres dans son premier voyage à Paris, cinq fois; depuis la mort du feu roi jusqu'à la majorité du jeune sans compter les guerres et les sièges, sept fois; et depuis cette majorité jusqu'aujourd'hui, cent fois peut-être! --- Aussi, malgré les édits, les ordonnances et les arrêts, le voilà capitaine des mousquetaires, c'est-à-dire chef d'une légion de Césars, dont le roi fait un très grand cas, et que M. le cardinal redoute, lui qui ne redoute pas grand-chose, comme chacun sait. De plus, M. de Tréville gagne dix mille écus par an; c'est donc un fort grand seigneur. --- Il a commencé comme vous, allez le voir avec cette lettre, et réglez-vous sur lui, afin de faire comme lui.» 

Sur quoi, M. d'Artagnan père ceignit à son fils sa propre épée, l'embrassa tendrement sur les deux joues et lui donna sa bénédiction. 

En sortant de la chambre paternelle, le jeune homme trouva sa mère qui l'attendait avec la fameuse recette dont les conseils que nous venons de rapporter devaient nécessiter un assez fréquent emploi. Les adieux furent de ce côté plus longs et plus tendres qu'ils ne l'avaient été de l'autre, non pas que M. d'Artagnan n'aimât son fils, qui était sa seule progéniture, mais M. d'Artagnan était un homme, et il eût regardé comme indigne d'un homme de se laisser aller à son émotion, tandis que Mme d'Artagnan était femme et, de plus, était mère. --- Elle pleura abondamment, et, disons-le à la louange de M. d'Artagnan fils, quelques efforts qu'il tentât pour rester ferme comme le devait être un futur mousquetaire, la nature l'emporta et il versa force larmes, dont il parvint à grand-peine à cacher la moitié. 

Le même jour le jeune homme se mit en route, muni des trois présents paternels et qui se composaient, comme nous l'avons dit, de quinze écus, du cheval et de la lettre pour M. de Tréville; comme on le pense bien, les conseils avaient été donnés par-dessus le marché. 

Avec un pareil \textit{vade mecum}, d'Artagnan se trouva, au moral comme au physique, une copie exacte du héros de Cervantes, auquel nous l'avons si heureusement comparé lorsque nos devoirs d'historien nous ont fait une nécessité de tracer son portrait. Don Quichotte prenait les moulins à vent pour des géants et les moutons pour des armées, d'Artagnan prit chaque sourire pour une insulte et chaque regard pour une provocation. Il en résulta qu'il eut toujours le poing fermé depuis Tarbes jusqu'à Meung, et que l'un dans l'autre il porta la main au pommeau de son épée dix fois par jour; toutefois le poing ne descendit sur aucune mâchoire, et l'épée ne sortit point de son fourreau. Ce n'est pas que la vue du malencontreux bidet jaune n'épanouît bien des sourires sur les visages des passants; mais, comme au-dessus du bidet sonnait une épée de taille respectable et qu'au-dessus de cette épée brillait un œil plutôt féroce que fier, les passants réprimaient leur hilarité, ou, si l'hilarité l'emportait sur la prudence, ils tâchaient au moins de ne rire que d'un seul côté, comme les masques antiques. D'Artagnan demeura donc majestueux et intact dans sa susceptibilité jusqu'à cette malheureuse ville de Meung. 

Mais là, comme il descendait de cheval à la porte du \textit{Franc-Meunier} sans que personne, hôte, garçon ou palefrenier, fût venu prendre l'étrier au montoir, d'Artagnan avisa à une fenêtre entrouverte du rez-de-chaussée un gentilhomme de belle taille et de haute mine, quoique au visage légèrement renfrogné, lequel causait avec deux personnes qui paraissaient l'écouter avec déférence. D'Artagnan crut tout naturellement, selon son habitude, être l'objet de la conversation et écouta. Cette fois, d'Artagnan ne s'était trompé qu'à moitié: ce n'était pas de lui qu'il était question, mais de son cheval. Le gentilhomme paraissait énumérer à ses auditeurs toutes ses qualités, et comme, ainsi que je l'ai dit, les auditeurs paraissaient avoir une grande déférence pour le narrateur, ils éclataient de rire à tout moment. Or, comme un demi-sourire suffisait pour éveiller l'irascibilité du jeune homme, on comprend quel effet produisit sur lui tant de bruyante hilarité. 

Cependant d'Artagnan voulut d'abord se rendre compte de la physionomie de l'impertinent qui se moquait de lui. Il fixa son regard fier sur l'étranger et reconnut un homme de quarante à quarante-cinq ans, aux yeux noirs et perçants, au teint pâle, au nez fortement accentué, à la moustache noire et parfaitement taillée; il était vêtu d'un pourpoint et d'un haut-de-chausses violet avec des aiguillettes de même couleur, sans aucun ornement que les crevés habituels par lesquels passait la chemise. Ce haut-de-chausses et ce pourpoint, quoique neufs, paraissaient froissés comme des habits de voyage longtemps renfermés dans un portemanteau. D'Artagnan fit toutes ces remarques avec la rapidité de l'observateur le plus minutieux, et sans doute par un sentiment instinctif qui lui disait que cet inconnu devait avoir une grande influence sur sa vie à venir. 

Or, comme au moment où d'Artagnan fixait son regard sur le gentilhomme au pourpoint violet, le gentilhomme faisait à l'endroit du bidet béarnais une de ses plus savantes et de ses plus profondes démonstrations, ses deux auditeurs éclatèrent de rire, et lui-même laissa visiblement, contre son habitude, errer, si l'on peut parler ainsi, un pâle sourire sur son visage. Cette fois, il n'y avait plus de doute, d'Artagnan était réellement insulté. Aussi, plein de cette conviction, enfonça-t-il son béret sur ses yeux, et, tâchant de copier quelques-uns des airs de cour qu'il avait surpris en Gascogne chez des seigneurs en voyage, il s'avança, une main sur la garde de son épée et l'autre appuyée sur la hanche. Malheureusement, au fur et à mesure qu'il avançait, la colère l'aveuglant de plus en plus, au lieu du discours digne et hautain qu'il avait préparé pour formuler sa provocation, il ne trouva plus au bout de sa langue qu'une personnalité grossière qu'il accompagna d'un geste furieux. 

\speak Eh! Monsieur, s'écria-t-il, monsieur, qui vous cachez derrière ce volet! oui, vous, dites-moi donc un peu de quoi vous riez, et nous rirons ensemble. 

Le gentilhomme ramena lentement les yeux de la monture au cavalier, comme s'il lui eût fallu un certain temps pour comprendre que c'était à lui que s'adressaient de si étranges reproches; puis, lorsqu'il ne put plus conserver aucun doute, ses sourcils se froncèrent légèrement, et après une assez longue pause, avec un accent d'ironie et d'insolence impossible à décrire, il répondit à d'Artagnan: 

\speak Je ne vous parle pas, monsieur. 

\speak Mais je vous parle, moi!» s'écria le jeune homme exaspéré de ce mélange d'insolence et de bonnes manières, de convenances et de dédains. 

L'inconnu le regarda encore un instant avec son léger sourire, et, se retirant de la fenêtre, sortit lentement de l'hôtellerie pour venir à deux pas de d'Artagnan se planter en face du cheval. Sa contenance tranquille et sa physionomie railleuse avaient redoublé l'hilarité de ceux avec lesquels il causait et qui, eux, étaient restés à la fenêtre. 

D'Artagnan, le voyant arriver, tira son épée d'un pied hors du fourreau. 

\speak  Ce cheval est décidément ou plutôt a été dans sa jeunesse bouton d'or, reprit l'inconnu continuant les investigations commencées et s'adressant à ses auditeurs de la fenêtre, sans paraître aucunement remarquer l'exaspération de d'Artagnan, qui cependant se redressait entre lui et eux. C'est une couleur fort connue en botanique, mais jusqu'à présent fort rare chez les chevaux. 

\speak  Tel rit du cheval qui n'oserait pas rire du maître! s'écria l'émule de Tréville, furieux. 

\speak  Je ne ris pas souvent, monsieur, reprit l'inconnu, ainsi que vous pouvez le voir vous-même à l'air de mon visage; mais je tiens cependant à conserver le privilège de rire quand il me plaît. 

\speak  Et moi, s'écria d'Artagnan, je ne veux pas qu'on rie quand il me déplaît! 

\speak  En vérité, monsieur? continua l'inconnu plus calme que jamais, eh bien, c'est parfaitement juste.» Et tournant sur ses talons, il s'apprêta à rentrer dans l'hôtellerie par la grande porte, sous laquelle d'Artagnan en arrivant avait remarqué un cheval tout sellé. 

Mais d'Artagnan n'était pas de caractère à lâcher ainsi un homme qui avait eu l'insolence de se moquer de lui. Il tira son épée entièrement du fourreau et se mit à sa poursuite en criant: 

\speak  Tournez, tournez donc, monsieur le railleur, que je ne vous frappe point par derrière. 

\speak  Me frapper, moi! dit l'autre en pivotant sur ses talons et en regardant le jeune homme avec autant d'étonnement que de mépris. Allons, allons donc, mon cher, vous êtes fou!» 

Puis, à demi-voix, et comme s'il se fût parlé à lui-même: 

\speak  C'est fâcheux, continua-t-il, quelle trouvaille pour Sa Majesté, qui cherche des braves de tous côtés pour recruter ses mousquetaires! 

Il achevait à peine, que d'Artagnan lui allongea un si furieux coup de pointe, que, s'il n'eût fait vivement un bond en arrière, il est probable qu'il eût plaisanté pour la dernière fois. L'inconnu vit alors que la chose passait la raillerie, tira son épée, salua son adversaire et se mit gravement en garde. Mais au même moment ses deux auditeurs, accompagnés de l'hôte, tombèrent sur d'Artagnan à grands coups de bâtons, de pelles et de pincettes. Cela fit une diversion si rapide et si complète à l'attaque, que l'adversaire de d'Artagnan, pendant que celui-ci se retournait pour faire face à cette grêle de coups, rengainait avec la même précision, et, d'acteur qu'il avait manqué d'être, redevenait spectateur du combat, rôle dont il s'acquitta avec son impassibilité ordinaire, tout en marmottant néanmoins: 

\speak  La peste soit des Gascons! Remettez-le sur son cheval orange, et qu'il s'en aille! 

\speak  Pas avant de t'avoir tué, lâche!» criait d'Artagnan tout en faisant face du mieux qu'il pouvait et sans reculer d'un pas à ses trois ennemis, qui le moulaient de coups. 

\speak  Encore une gasconnade, murmura le gentilhomme. Sur mon honneur, ces Gascons sont incorrigibles! Continuez donc la danse, puisqu'il le veut absolument. Quand il sera las, il dira qu'il en a assez. 

Mais l'inconnu ne savait pas encore à quel genre d'entêté il avait affaire; d'Artagnan n'était pas homme à jamais demander merci. Le combat continua donc quelques secondes encore; enfin d'Artagnan, épuisé, laissa échapper son épée qu'un coup de bâton brisa en deux morceaux. Un autre coup, qui lui entama le front, le renversa presque en même temps tout sanglant et presque évanoui. 

C'est à ce moment que de tous côtés on accourut sur le lieu de la scène. L'hôte, craignant du scandale, emporta, avec l'aide de ses garçons, le blessé dans la cuisine où quelques soins lui furent accordés. 

Quant au gentilhomme, il était revenu prendre sa place à la fenêtre et regardait avec une certaine impatience toute cette foule, qui semblait en demeurant là lui causer une vive contrariété. 

\speak  Eh bien, comment va cet enragé? reprit-il en se retournant au bruit de la porte qui s'ouvrit et en s'adressant à l'hôte qui venait s'informer de sa santé. 

\speak  Votre Excellence est saine et sauve? demanda l'hôte. 

\speak  Oui, parfaitement saine et sauve, mon cher hôtelier, et c'est moi qui vous demande ce qu'est devenu notre jeune homme. 

\speak  Il va mieux, dit l'hôte: il s'est évanoui tout à fait. 

\speak  Vraiment? fit le gentilhomme. 

\speak  Mais avant de s'évanouir il a rassemblé toutes ses forces pour vous appeler et vous défier en vous appelant. 

\speak  Mais c'est donc le diable en personne que ce gaillard-là! s'écria l'inconnu. 

\speak  Oh! non, Votre Excellence, ce n'est pas le diable, reprit l'hôte avec une grimace de mépris, car pendant son évanouissement nous l'avons fouillé, et il n'a dans son paquet qu'une chemise et dans sa bourse que onze écus, ce qui ne l'a pas empêché de dire en s'évanouissant que si pareille chose était arrivée à Paris, vous vous en repentiriez tout de suite, tandis qu'ici vous ne vous en repentirez que plus tard. 

\speak  Alors, dit froidement l'inconnu, c'est quelque prince du sang déguisé. 

\speak  Je vous dis cela, mon gentilhomme, reprit l'hôte, afin que vous vous teniez sur vos gardes. 

\speak  Et il n'a nommé personne dans sa colère? 

\speak  Si fait, il frappait sur sa poche, et il disait: «Nous verrons ce que M. de Tréville pensera de cette insulte faite à son protégé. 

\speak  M. de Tréville? dit l'inconnu en devenant attentif; il frappait sur sa poche en prononçant le nom de M. de Tréville?\dots Voyons, mon cher hôte, pendant que votre jeune homme était évanoui, vous n'avez pas été, j'en suis bien sûr, sans regarder aussi cette poche-là. Qu'y avait-il? 

\speak  Une lettre adressée à M. de Tréville, capitaine des mousquetaires. 

\speak  En vérité! 

\speak  C'est comme j'ai l'honneur de vous le dire, Excellence.» 

L'hôte, qui n'était pas doué d'une grande perspicacité, ne remarqua point l'expression que ses paroles avaient donnée à la physionomie de l'inconnu. Celui-ci quitta le rebord de la croisée sur lequel il était toujours resté appuyé du bout du coude, et fronça le sourcil en homme inquiet. 

\speak  Diable! murmura-t-il entre ses dents, Tréville m'aurait-il envoyé ce Gascon? il est bien jeune! Mais un coup d'épée est un coup d'épée, quel que soit l'âge de celui qui le donne, et l'on se défie moins d'un enfant que de tout autre; il suffit parfois d'un faible obstacle pour contrarier un grand dessein. 

Et l'inconnu tomba dans une réflexion qui dura quelques minutes. 

\speak  Voyons, l'hôte, dit-il, est-ce que vous ne me débarrasserez pas de ce frénétique? En conscience, je ne puis le tuer, et cependant, ajouta-t-il avec une expression froidement menaçante, cependant il me gêne. Où est-il? 

\speak  Dans la chambre de ma femme, où on le panse, au premier étage. 

\speak  Ses hardes et son sac sont avec lui? il n'a pas quitté son pourpoint? 

\speak  Tout cela, au contraire, est en bas dans la cuisine. Mais puisqu'il vous gêne, ce jeune fou\dots 

\speak  Sans doute. Il cause dans votre hôtellerie un scandale auquel d'honnêtes gens ne sauraient résister. Montez chez vous, faites mon compte et avertissez mon laquais. 

\speak  Quoi! Monsieur nous quitte déjà? 

\speak  Vous le savez bien, puisque je vous avais donné l'ordre de seller mon cheval. Ne m'a-t-on point obéi? 

\speak  Si fait, et comme Votre Excellence a pu le voir, son cheval est sous la grande porte, tout appareillé pour partir. 

\speak  C'est bien, faites ce que je vous ai dit alors.» 

\speak  Ouais! se dit l'hôte, aurait-il peur du petit garçon? 

Mais un coup d'œil impératif de l'inconnu vint l'arrêter court. Il salua humblement et sortit. 

\speak  Il ne faut pas que Milady soit aperçue de ce drôle, continua l'étranger: elle ne doit pas tarder à passer: déjà même elle est en retard. Décidément, mieux vaut que je monte à cheval et que j'aille au-devant d'elle\dots Si seulement je pouvais savoir ce que contient cette lettre adressée à Tréville! 

Et l'inconnu, tout en marmottant, se dirigea vers la cuisine. 

Pendant ce temps, l'hôte, qui ne doutait pas que ce ne fût la présence du jeune garçon qui chassât l'inconnu de son hôtellerie, était remonté chez sa femme et avait trouvé d'Artagnan maître enfin de ses esprits. Alors, tout en lui faisant comprendre que la police pourrait bien lui faire un mauvais parti pour avoir été chercher querelle à un grand seigneur --- car, à l'avis de l'hôte, l'inconnu ne pouvait être qu'un grand seigneur ---, il le détermina, malgré sa faiblesse, à se lever et à continuer son chemin. D'Artagnan à moitié abasourdi, sans pourpoint et la tête tout emmaillotée de linges, se leva donc et, poussé par l'hôte, commença de descendre; mais, en arrivant à la cuisine, la première chose qu'il aperçut fut son provocateur qui causait tranquillement au marchepied d'un lourd carrosse attelé de deux gros chevaux normands. 

Son interlocutrice, dont la tête apparaissait encadrée par la portière, était une femme de vingt à vingt-deux ans. Nous avons déjà dit avec quelle rapidité d'investigation d'Artagnan embrassait toute une physionomie; il vit donc du premier coup d'œil que la femme était jeune et belle. Or cette beauté le frappa d'autant plus qu'elle était parfaitement étrangère aux pays méridionaux que jusque-là d'Artagnan avait habités. C'était une pâle et blonde personne, aux longs cheveux bouclés tombant sur ses épaules, aux grands yeux bleus languissants, aux lèvres rosées et aux mains d'albâtre. Elle causait très vivement avec l'inconnu. 

\speak  Ainsi, Son Éminence m'ordonne\dots, disait la dame. 

\speak  De retourner à l'instant même en Angleterre, et de la prévenir directement si le duc quittait Londres. 

\speak  Et quant à mes autres instructions? demanda la belle voyageuse. 

\speak  Elles sont renfermées dans cette boîte, que vous n'ouvrirez que de l'autre côté de la Manche. 

\speak  Très bien; et vous, que faites-vous? 

\speak  Moi, je retourne à Paris. 

\speak  Sans châtier cet insolent petit garçon?» demanda la dame. 

L'inconnu allait répondre: mais, au moment où il ouvrait la bouche, d'Artagnan, qui avait tout entendu, s'élança sur le seuil de la porte. 

\speak  C'est cet insolent petit garçon qui châtie les autres, s'écria-t-il, et j'espère bien que cette fois-ci celui qu'il doit châtier ne lui échappera pas comme la première. 

\speak  Ne lui échappera pas? reprit l'inconnu en fronçant le sourcil. 

\speak  Non, devant une femme, vous n'oseriez pas fuir, je présume. 

\speak  Songez, s'écria Milady en voyant le gentilhomme porter la main à son épée, songez que le moindre retard peut tout perdre. 

\speak  Vous avez raison, s'écria le gentilhomme; partez donc de votre côté, moi, je pars du mien.» 

Et, saluant la dame d'un signe de tête, il s'élança sur son cheval, tandis que le cocher du carrosse fouettait vigoureusement son attelage. Les deux interlocuteurs partirent donc au galop, s'éloignant chacun par un côté opposé de la rue. 

\speak  Eh! votre dépense», vociféra l'hôte, dont l'affection pour son voyageur se changeait en un profond dédain en voyant qu'il s'éloignait sans solder ses comptes. 

\speak  Paie, maroufle», s'écria le voyageur toujours galopant à son laquais, lequel jeta aux pieds de l'hôte deux ou trois pièces d'argent et se mit à galoper après son maître. 

\speak  Ah! lâche, ah! misérable, ah! faux gentilhomme!» cria d'Artagnan s'élançant à son tour après le laquais. 

Mais le blessé était trop faible encore pour supporter une pareille secousse. À peine eut-il fait dix pas, que ses oreilles tintèrent, qu'un éblouissement le prit, qu'un nuage de sang passa sur ses yeux et qu'il tomba au milieu de la rue, en criant encore: 

\speak  Lâche! lâche! lâche! 

\speak  Il est en effet bien lâche», murmura l'hôte en s'approchant de d'Artagnan, et essayant par cette flatterie de se raccommoder avec le pauvre garçon, comme le héron de la fable avec son limaçon du soir. 

\speak  Oui, bien lâche, murmura d'Artagnan; mais elle, bien belle! 

\speak  Qui, elle? demanda l'hôte. 

\speak  Milady», balbutia d'Artagnan. 

Et il s'évanouit une seconde fois. 

\speak  C'est égal, dit l'hôte, j'en perds deux, mais il me reste celui-là, que je suis sûr de conserver au moins quelques jours. C'est toujours onze écus de gagnés. 

On sait que onze écus faisaient juste la somme qui restait dans la bourse de d'Artagnan. 

L'hôte avait compté sur onze jours de maladie à un écu par jour; mais il avait compté sans son voyageur. Le lendemain, dès cinq heures du matin, d'Artagnan se leva, descendit lui-même à la cuisine, demanda, outre quelques autres ingrédients dont la liste n'est pas parvenue jusqu'à nous, du vin, de l'huile, du romarin, et, la recette de sa mère à la main, se composa un baume dont il oignit ses nombreuses blessures, renouvelant ses compresses lui-même et ne voulant admettre l'adjonction d'aucun médecin. Grâce sans doute à l'efficacité du baume de Bohême, et peut-être aussi grâce à l'absence de tout docteur, d'Artagnan se trouva sur pied dès le soir même, et à peu près guéri le lendemain. 

Mais, au moment de payer ce romarin, cette huile et ce vin, seule dépense du maître qui avait gardé une diète absolue, tandis qu'au contraire le cheval jaune, au dire de l'hôtelier du moins, avait mangé trois fois plus qu'on n'eût raisonnablement pu le supposer pour sa taille, d'Artagnan ne trouva dans sa poche que sa petite bourse de velours râpé ainsi que les onze écus qu'elle contenait; mais quant à la lettre adressée à M. de Tréville, elle avait disparu. 

Le jeune homme commença par chercher cette lettre avec une grande patience, tournant et retournant vingt fois ses poches et ses goussets, fouillant et refouillant dans son sac, ouvrant et refermant sa bourse; mais lorsqu'il eut acquis la conviction que la lettre était introuvable, il entra dans un troisième accès de rage, qui faillit lui occasionner une nouvelle consommation de vin et d'huile aromatisés: car, en voyant cette jeune mauvaise tête s'échauffer et menacer de tout casser dans l'établissement si l'on ne retrouvait pas sa lettre, l'hôte s'était déjà saisi d'un épieu, sa femme d'un manche à balai, et ses garçons des mêmes bâtons qui avaient servi la surveille. 

\speak  Ma lettre de recommandation! s'écria d'Artagnan, ma lettre de recommandation, sangdieu! ou je vous embroche tous comme des ortolans! 

Malheureusement une circonstance s'opposait à ce que le jeune homme accomplît sa menace: c'est que, comme nous l'avons dit, son épée avait été, dans sa première lutte, brisée en deux morceaux, ce qu'il avait parfaitement oublié. Il en résulta que, lorsque d'Artagnan voulut en effet dégainer, il se trouva purement et simplement armé d'un tronçon d'épée de huit ou dix pouces à peu près, que l'hôte avait soigneusement renfoncé dans le fourreau. Quant au reste de la lame, le chef l'avait adroitement détourné pour s'en faire une lardoire. 

Cependant cette déception n'eût probablement pas arrêté notre fougueux jeune homme, si l'hôte n'avait réfléchi que la réclamation que lui adressait son voyageur était parfaitement juste. 

\speak  Mais, au fait, dit-il en abaissant son épieu, où est cette lettre? 

\speak  Oui, où est cette lettre? cria d'Artagnan. D'abord, je vous en préviens, cette lettre est pour M. de Tréville, et il faut qu'elle se retrouve; ou si elle ne se retrouve pas, il saura bien la faire retrouver, lui!» 

Cette menace acheva d'intimider l'hôte. Après le roi et M. le cardinal, M. de Tréville était l'homme dont le nom peut-être était le plus souvent répété par les militaires et même par les bourgeois. Il y avait bien le père Joseph, c'est vrai; mais son nom à lui n'était jamais prononcé que tout bas, tant était grande la terreur qu'inspirait l'Éminence grise, comme on appelait le familier du cardinal. 

Aussi, jetant son épieu loin de lui, et ordonnant à sa femme d'en faire autant de son manche à balai et à ses valets de leurs bâtons, il donna le premier l'exemple en se mettant lui-même à la recherche de la lettre perdue. 

\speak  Est-ce que cette lettre renfermait quelque chose de précieux? demanda l'hôte au bout d'un instant d'investigations inutiles. 

\speak  Sandis! je le crois bien! s'écria le Gascon qui comptait sur cette lettre pour faire son chemin à la cour; elle contenait ma fortune. 

\speak  Des bons sur l'épargne? demanda l'hôte inquiet. 

\speak  Des bons sur la trésorerie particulière de Sa Majesté», répondit d'Artagnan, qui, comptant entrer au service du roi grâce à cette recommandation, croyait pouvoir faire sans mentir cette réponse quelque peu hasardée. 

\speak  Diable! fit l'hôte tout à fait désespéré. 

\speak  Mais il n'importe, continua d'Artagnan avec l'aplomb national, il n'importe, et l'argent n'est rien: --- cette lettre était tout. J'eusse mieux aimé perdre mille pistoles que de la perdre.» 

Il ne risquait pas davantage à dire vingt mille, mais une certaine pudeur juvénile le retint. 

Un trait de lumière frappa tout à coup l'esprit de l'hôte qui se donnait au diable en ne trouvant rien. 

\speak  Cette lettre n'est point perdue, s'écria-t-il. 

\speak  Ah! fit d'Artagnan. 

\speak  Non; elle vous a été prise. 

\speak  Prise! et par qui? 

\speak  Par le gentilhomme d'hier. Il est descendu à la cuisine, où était votre pourpoint. Il y est resté seul. Je gagerais que c'est lui qui l'a volée. 

\speak  Vous croyez?» répondit d'Artagnan peu convaincu; car il savait mieux que personne l'importance toute personnelle de cette lettre, et n'y voyait rien qui pût tenter la cupidité. Le fait est qu'aucun des valets, aucun des voyageurs présents n'eût rien gagné à posséder ce papier. 

\speak  Vous dites donc, reprit d'Artagnan, que vous soupçonnez cet impertinent gentilhomme. 

\speak  Je vous dis que j'en suis sûr, continua l'hôte; lorsque je lui ai annoncé que Votre Seigneurie était le protégé de M. de Tréville, et que vous aviez même une lettre pour cet illustre gentilhomme, il a paru fort inquiet, m'a demandé où était cette lettre, et est descendu immédiatement à la cuisine où il savait qu'était votre pourpoint. 

\speak  Alors c'est mon voleur, répondit d'Artagnan; je m'en plaindrai à M. de Tréville, et M. de Tréville s'en plaindra au roi.» Puis il tira majestueusement deux écus de sa poche, les donna à l'hôte, qui l'accompagna, le chapeau à la main, jusqu'à la porte, remonta sur son cheval jaune, qui le conduisit sans autre incident jusqu'à la porte Saint-Antoine à Paris, où son propriétaire le vendit trois écus, ce qui était fort bien payé, attendu que d'Artagnan l'avait fort surmené pendant la dernière étape. Aussi le maquignon auquel d'Artagnan le céda moyennant les neuf livres susdites ne cacha-t-il point au jeune homme qu'il n'en donnait cette somme exorbitante qu'à cause de l'originalité de sa couleur. 

D'Artagnan entra donc dans Paris à pied, portant son petit paquet sous son bras, et marcha tant qu'il trouvât à louer une chambre qui convînt à l'exiguïté de ses ressources. Cette chambre fut une espèce de mansarde, sise rue des Fossoyeurs, près du Luxembourg. 

Aussitôt le denier à Dieu donné, d'Artagnan prit possession de son logement, passa le reste de la journée à coudre à son pourpoint et à ses chausses des passementeries que sa mère avait détachées d'un pourpoint presque neuf de M. d'Artagnan père, et qu'elle lui avait données en cachette; puis il alla quai de la Ferraille, faire remettre une lame à son épée; puis il revint au Louvre s'informer, au premier mousquetaire qu'il rencontra, de la situation de l'hôtel de M. de Tréville, lequel était situé rue du Vieux-Colombier, c'est-à-dire justement dans le voisinage de la chambre arrêtée par d'Artagnan: circonstance qui lui parut d'un heureux augure pour le succès de son voyage. 

Après quoi, content de la façon dont il s'était conduit à Meung, sans remords dans le passé, confiant dans le présent et plein d'espérance dans l'avenir, il se coucha et s'endormit du sommeil du brave. 

Ce sommeil, tout provincial encore, le conduisit jusqu'à neuf heures du matin, heure à laquelle il se leva pour se rendre chez ce fameux M. de Tréville, le troisième personnage du royaume d'après l'estimation paternelle. 
