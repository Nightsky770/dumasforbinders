%!TeX root=../musketeersfr.tex 

\chapter{Milady} 
	
\lettrine{D}{'Artagnan} avait suivi Milady sans être aperçu par elle: il la vit monter dans son carrosse, et il l'entendit donner à son cocher l'ordre d'aller à Saint-Germain. 

\zz
Il était inutile d'essayer de suivre à pied une voiture emportée au trot de deux vigoureux chevaux. D'Artagnan revint donc rue Férou. 

Dans la rue de Seine, il rencontra Planchet, qui était arrêté devant la boutique d'un pâtissier, et qui semblait en extase devant une brioche de la forme la plus appétissante. 

Il lui donna l'ordre d'aller seller deux chevaux dans les écuries de M. de Tréville, un pour lui d'Artagnan, l'autre pour lui Planchet, et de venir le joindre chez Athos, --- M. de Tréville, une fois pour toutes, ayant mis ses écuries au service de d'Artagnan. 

Planchet s'achemina vers la rue du Colombier, et d'Artagnan vers la rue Férou. Athos était chez lui, vidant tristement une des bouteilles de ce fameux vin d'Espagne qu'il avait rapporté de son voyage en Picardie. Il fit signe à Grimaud d'apporter un verre pour d'Artagnan, et Grimaud obéit comme d'habitude. 

D'Artagnan raconta alors à Athos tout ce qui s'était passé à l'église entre Porthos et la procureuse, et comment leur camarade était probablement, à cette heure, en voie de s'équiper. 

«Quant à moi, répondit Athos à tout ce récit, je suis bien tranquille, ce ne seront pas les femmes qui feront les frais de mon harnais. 

\speak  Et cependant, beau, poli, grand seigneur comme vous l'êtes, mon cher Athos, il n'y aurait ni princesses, ni reines à l'abri de vos traits amoureux. 

\speak  Que ce d'Artagnan est jeune!» dit Athos en haussant les épaules. 

Et il fit signe à Grimaud d'apporter une seconde bouteille. 

En ce moment, Planchet passa modestement la tête par la porte entrebâillée, et annonça à son maître que les deux chevaux étaient là. 

«Quels chevaux? demanda Athos. 

\speak  Deux que M. de Tréville me prête pour la promenade, et avec lesquels je vais aller faire un tour à Saint-Germain. 

\speak  Et qu'allez-vous faire à Saint-Germain?» demanda encore Athos. 

Alors d'Artagnan lui raconta la rencontre qu'il avait faite dans l'église, et comment il avait retrouvé cette femme qui, avec le seigneur au manteau noir et à la cicatrice près de la tempe, était sa préoccupation éternelle. 

«C'est-à-dire que vous êtes amoureux de celle-là, comme vous l'étiez de Mme Bonacieux, dit Athos en haussant dédaigneusement les épaules, comme s'il eût pris en pitié la faiblesse humaine. 

\speak  Moi, point du tout! s'écria d'Artagnan. Je suis seulement curieux d'éclaircir le mystère auquel elle se rattache. Je ne sais pourquoi, je me figure que cette femme, tout inconnue qu'elle m'est et tout inconnu que je lui suis, a une action sur ma vie. 

\speak  Au fait, vous avez raison, dit Athos, je ne connais pas une femme qui vaille la peine qu'on la cherche quand elle est perdue. Mme Bonacieux est perdue, tant pis pour elle! qu'elle se retrouve! 

\speak  Non, Athos, non, vous vous trompez, dit d'Artagnan; j'aime ma pauvre Constance plus que jamais, et si je savais le lieu où elle est, fût-elle au bout du monde, je partirais pour la tirer des mains de ses ennemis; mais je l'ignore, toutes mes recherches ont été inutiles. Que voulez-vous, il faut bien se distraire. 

\speak  Distrayez-vous donc avec Milady, mon cher d'Artagnan; je le souhaite de tout mon cœur, si cela peut vous amuser. 

\speak  Écoutez, Athos, dit d'Artagnan, au lieu de vous tenir enfermé ici comme si vous étiez aux arrêts, montez à cheval et venez vous promener avec moi à Saint-Germain. 

\speak  Mon cher, répliqua Athos, je monte mes chevaux quand j'en ai, sinon je vais à pied. 

\speak  Eh bien, moi, répondit d'Artagnan en souriant de la misanthropie d'Athos, qui dans un autre l'eût certainement blessé, moi, je suis moins fier que vous, je monte ce que je trouve. Ainsi, au revoir, mon cher Athos. 

\speak  Au revoir», dit le mousquetaire en faisant signe à Grimaud de déboucher la bouteille qu'il venait d'apporter. 

D'Artagnan et Planchet se mirent en selle et prirent le chemin de Saint-Germain. 

Tout le long de la route, ce qu'Athos avait dit au jeune homme de Mme Bonacieux lui revenait à l'esprit. Quoique d'Artagnan ne fût pas d'un caractère fort sentimental, la jolie mercière avait fait une impression réelle sur son cœur: comme il le disait, il était prêt à aller au bout du monde pour la chercher. Mais le monde a bien des bouts, par cela même qu'il est rond; de sorte qu'il ne savait de quel côté se tourner. 

En attendant, il allait tâcher de savoir ce que c'était que Milady. Milady avait parlé à l'homme au manteau noir, donc elle le connaissait. Or, dans l'esprit de d'Artagnan, c'était l'homme au manteau noir qui avait enlevé Mme Bonacieux une seconde fois, comme il l'avait enlevée une première. D'Artagnan ne mentait donc qu'à moitié, ce qui est bien peu mentir, quand il disait qu'en se mettant à la recherche de Milady, il se mettait en même temps à la recherche de Constance. 

Tout en songeant ainsi et en donnant de temps en temps un coup d'éperon à son cheval, d'Artagnan avait fait la route et était arrivé à Saint-Germain. Il venait de longer le pavillon où, dix ans plus tard, devait naître Louis XIV. Il traversait une rue fort déserte, regardant à droite et à gauche s'il ne reconnaîtrait pas quelque vestige de sa belle Anglaise, lorsque au rez-de-chaussée d'une jolie maison qui, selon l'usage du temps, n'avait aucune fenêtre sur la rue, il vit apparaître une figure de connaissance. Cette figure se promenait sur une sorte de terrasse garnie de fleurs. Planchet la reconnut le premier. «Eh! monsieur dit-il s'adressant à d'Artagnan, ne vous remettez-vous pas ce visage qui baye aux corneilles? 

\speak  Non, dit d'Artagnan; et cependant je suis certain que ce n'est point la première fois que je le vois, ce visage. 

\speak  Je le crois pardieu bien, dit Planchet: c'est ce pauvre Lubin, le laquais du comte de Wardes, celui que vous avez si bien accommodé il y a un mois, à Calais, sur la route de la maison de campagne du gouverneur. 

\speak  Ah! oui bien, dit d'Artagnan, et je le reconnais à cette heure. Crois-tu qu'il te reconnaisse, toi? 

\speak  Ma foi, monsieur, il était si fort troublé que je doute qu'il ait gardé de moi une mémoire bien nette. 

\speak  Eh bien, va donc causer avec ce garçon, dit d'Artagnan, et informe-toi dans la conversation si son maître est mort.» 

Planchet descendit de cheval, marcha droit à Lubin, qui en effet ne le reconnut pas, et les deux laquais se mirent à causer dans la meilleure intelligence du monde, tandis que d'Artagnan poussait les deux chevaux dans une ruelle et, faisant le tour d'une maison, s'en revenait assister à la conférence derrière une haie de coudriers. 

Au bout d'un instant d'observation derrière la haie, il entendit le bruit d'une voiture, et il vit s'arrêter en face de lui le carrosse de Milady. Il n'y avait pas à s'y tromper. Milady était dedans. D'Artagnan se coucha sur le cou de son cheval, afin de tout voir sans être vu. 

Milady sortit sa charmante tête blonde par la portière, et donna des ordres à sa femme de chambre. 

Cette dernière, jolie fille de vingt à vingt-deux ans, alerte et vive, véritable soubrette de grande dame, sauta en bas du marchepied, sur lequel elle était assise selon l'usage du temps, et se dirigea vers la terrasse où d'Artagnan avait aperçu Lubin. 

D'Artagnan suivit la soubrette des yeux, et la vit s'acheminer vers la terrasse. Mais, par hasard, un ordre de l'intérieur avait appelé Lubin, de sorte que Planchet était resté seul, regardant de tous côtés par quel chemin avait disparu d'Artagnan. 

La femme de chambre s'approcha de Planchet, qu'elle prit pour Lubin, et lui tendant un petit billet: 

«Pour votre maître, dit-elle. 

\speak  Pour mon maître? reprit Planchet étonné. 

\speak  Oui, et très pressé. Prenez donc vite.» 

Là-dessus elle s'enfuit vers le carrosse, retourné à l'avance du côté par lequel il était venu; elle s'élança sur le marchepied, et le carrosse repartit. 

Planchet tourna et retourna le billet, puis, accoutumé à l'obéissance passive, il sauta à bas de la terrasse, enfila la ruelle et rencontra au bout de vingt pas d'Artagnan qui, ayant tout vu, allait au-devant de lui. 

«Pour vous, monsieur, dit Planchet, présentant le billet au jeune homme. 

\speak  Pour moi? dit d'Artagnan; en es-tu bien sûr? 

\speak  Pardieu! si j'en suis sûr; la soubrette a dit: “Pour ton maître.” Je n'ai d'autre maître que vous; ainsi\dots Un joli brin de fille, ma foi, que cette soubrette!» 

D'Artagnan ouvrit la lettre, et lut ces mots: 

«Une personne qui s'intéresse à vous plus qu'elle ne peut le dire voudrait savoir quel jour vous serez en état de vous promener dans la forêt. Demain, à l'hôtel du Champ du Drap d'Or, un laquais noir et rouge attendra votre réponse.» 

«Oh! oh! se dit d'Artagnan, voilà qui est un peu vif. Il paraît que Milady et moi nous sommes en peine de la santé de la même personne. Eh bien, Planchet, comment se porte ce bon M. de Wardes? il n'est donc pas mort? 

\speak  Non, monsieur, il va aussi bien qu'on peut aller avec quatre coups d'épée dans le corps, car vous lui en avez, sans reproche, allongé quatre, à ce cher gentilhomme, et il est encore bien faible, ayant perdu presque tout son sang. Comme je l'avais dit à monsieur, Lubin ne m'a pas reconnu, et m'a raconté d'un bout à l'autre notre aventure. 

\speak  Fort bien, Planchet, tu es le roi des laquais; maintenant, remonte à cheval et rattrapons le carrosse.» 

Ce ne fut pas long; au bout de cinq minutes on aperçut le carrosse arrêté sur le revers de la route, un cavalier richement vêtu se tenait à la portière. 

La conversation entre Milady et le cavalier était tellement animée, que d'Artagnan s'arrêta de l'autre côté du carrosse sans que personne autre que la jolie soubrette s'aperçût de sa présence. 

La conversation avait lieu en anglais, langue que d'Artagnan ne comprenait pas; mais, à l'accent, le jeune homme crut deviner que la belle Anglaise était fort en colère; elle termina par un geste qui ne lui laissa point de doute sur la nature de cette conversation: c'était un coup d'éventail appliqué de telle force, que le petit meuble féminin vola en mille morceaux. 

Le cavalier poussa un éclat de rire qui parut exaspérer Milady. 

D'Artagnan pensa que c'était le moment d'intervenir; il s'approcha de l'autre portière, et se découvrant respectueusement: 

«Madame, dit-il, me permettez-vous de vous offrir mes services? Il me semble que ce cavalier vous a mise en colère. Dites un mot, madame, et je me charge de le punir de son manque de courtoisie.» 

Aux premières paroles, Milady s'était retournée, regardant le jeune homme avec étonnement, et lorsqu'il eut fini: 

«Monsieur, dit-elle en très bon français, ce serait de grand cœur que je me mettrais sous votre protection si la personne qui me querelle n'était point mon frère. 

\speak  Ah! excusez-moi, alors, dit d'Artagnan, vous comprenez que j'ignorais cela, madame. 

\speak  De quoi donc se mêle cet étourneau, s'écria en s'abaissant à la hauteur de la portière le cavalier que Milady avait désigné comme son parent, et pourquoi ne passe-t-il pas son chemin? 

\speak  Étourneau vous-même, dit d'Artagnan en se baissant à son tour sur le cou de son cheval, et en répondant de son côté par la portière; je ne passe pas mon chemin parce qu'il me plaît de m'arrêter ici.» 

Le cavalier adressa quelques mots en anglais à sa soeur. 

«Je vous parle français, moi, dit d'Artagnan; faites-moi donc, je vous prie, le plaisir de me répondre dans la même langue. Vous êtes le frère de madame, soit, mais vous n'êtes pas le mien, heureusement.» 

On eût pu croire que Milady, craintive comme l'est ordinairement une femme, allait s'interposer dans ce commencement de provocation, afin d'empêcher que la querelle n'allât plus loin; mais, tout au contraire, elle se rejeta au fond de son carrosse, et cria froidement au cocher: 

«Touche à l'hôtel!» 

La jolie soubrette jeta un regard d'inquiétude sur d'Artagnan, dont la bonne mine paraissait avoir produit son effet sur elle. 

Le carrosse partit et laissa les deux hommes en face l'un de l'autre, aucun obstacle matériel ne les séparant plus. 

Le cavalier fit un mouvement pour suivre la voiture; mais d'Artagnan, dont la colère déjà bouillante s'était encore augmentée en reconnaissant en lui l'Anglais qui, à Amiens, lui avait gagné son cheval et avait failli gagner à Athos son diamant, sauta à la bride et l'arrêta. 

«Eh! Monsieur, dit-il, vous me semblez encore plus étourneau que moi, car vous me faites l'effet d'oublier qu'il y a entre nous une petite querelle engagée. 

\speak  Ah! ah! dit l'Anglais, c'est vous, mon maître. Il faut donc toujours que vous jouiez un jeu ou un autre? 

\speak  Oui, et cela me rappelle que j'ai une revanche à prendre. Nous verrons, mon cher monsieur, si vous maniez aussi adroitement la rapière que le cornet. 

\speak  Vous voyez bien que je n'ai pas d'épée, dit l'Anglais; voulez-vous faire le brave contre un homme sans armes? 

\speak  J'espère bien que vous en avez chez vous, répondit d'Artagnan. En tout cas, j'en ai deux, et si vous le voulez, je vous en jouerai une. 

\speak  Inutile, dit l'Anglais, je suis muni suffisamment de ces sortes d'ustensiles. 

\speak  Eh bien, mon digne gentilhomme, reprit d'Artagnan choisissez la plus longue et venez me la montrer ce soir. 

\speak  Où cela, s'il vous plaît? 

\speak  Derrière le Luxembourg, c'est un charmant quartier pour les promenades dans le genre de celle que je vous propose. 

\speak  C'est bien, on y sera. 

\speak  Votre heure? 

\speak  Six heures. 

\speak  À propos, vous avez aussi probablement un ou deux amis? 

\speak  Mais j'en ai trois qui seront fort honorés de jouer la même partie que moi. 

\speak  Trois? à merveille! comme cela se rencontre! dit d'Artagnan, c'est juste mon compte. 

\speak  Maintenant, qui êtes-vous? demanda l'Anglais. 

\speak  Je suis M. d'Artagnan, gentilhomme gascon, servant aux gardes, compagnie de M. des Essarts. Et vous? 

\speak  Moi, je suis Lord de Winter, baron de Sheffield. 

\speak  Eh bien, je suis votre serviteur, monsieur le baron, dit d'Artagnan, quoique vous ayez des noms bien difficiles à retenir.» 

Et piquant son cheval, il le mit au galop, et reprit le chemin de Paris. 

Comme il avait l'habitude de le faire en pareille occasion, d'Artagnan descendit droit chez Athos. 

Il trouva Athos couché sur un grand canapé, où il attendait, comme il l'avait dit, que son équipement le vînt trouver. 

Il raconta à Athos tout ce qui venait de se passer, moins la lettre de M. de Wardes. 

Athos fut enchanté lorsqu'il sut qu'il allait se battre contre un Anglais. Nous avons dit que c'était son rêve. 

On envoya chercher à l'instant même Porthos et Aramis par les laquais, et on les mit au courant de la situation. 

Porthos tira son épée hors du fourreau et se mit à espadonner contre le mur en se reculant de temps en temps et en faisant des pliés comme un danseur. Aramis, qui travaillait toujours à son poème, s'enferma dans le cabinet d'Athos et pria qu'on ne le dérangeât plus qu'au moment de dégainer. 

Athos demanda par signe à Grimaud une bouteille. 

Quant à d'Artagnan, il arrangea en lui-même un petit plan dont nous verrons plus tard l'exécution, et qui lui promettait quelque gracieuse aventure, comme on pouvait le voir aux sourires qui, de temps en temps, passaient sur son visage dont ils éclairaient la rêverie.