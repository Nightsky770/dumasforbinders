\documentclass[
%paper=5.5in:8.5in,
a5paper,
]{scrbook} 
\title{Le comte de Monte-Cristo, tome~4}
\usepackage{montecristovf}


\begin{document}
\frontmatter
\pagestyle{plain}
\includepdf[width=\basicwidth]{titlepagevf4.jpg}
\KOMAoptions{headings=openleft}
\tableofcontents
 

 
\mainmatter
\pagestyle{headings}
\KOMAoptions{headings=openright}
\setcounter{chapter}{84}
\include{chapters/085.tex}
\include{chapters/086.tex}
\include{chapters/087.tex}
\include{chapters/088.tex}
\include{chapters/089.tex}
\include{chapters/090.tex}
\include{chapters/091.tex}
\include{chapters/092.tex}
\include{chapters/093.tex}
\include{chapters/094.tex}
\include{chapters/095.tex}
\include{chapters/096.tex}
\include{chapters/097.tex}
\include{chapters/098.tex}
\include{chapters/099.tex}
\include{chapters/100.tex}
\include{chapters/101.tex}
\include{chapters/102.tex}
\include{chapters/103.tex}
\include{chapters/104.tex}
\include{chapters/105.tex}
\include{chapters/106.tex}
\include{chapters/107.tex}
\include{chapters/108.tex}
\include{chapters/109.tex}
\include{chapters/110.tex}
\include{chapters/111.tex}
\include{chapters/112.tex}
\include{chapters/113.tex}
\include{chapters/114.tex}
\include{chapters/115.tex}
\include{chapters/116.tex}
\include{chapters/117.tex}



\KOMAoptions{headings=openleft}
\chapter*{Colophon}

\centering

\vfill
\begin{minipage}{\textwidth}
\textit{Le comte de Monte-Cristo} a été publié en feuilleton dans le \textit{Journal des Débats} entre août 1844 et janvier 1846.  Comme la plupart des œuvres de Alexandre Dumas \textit{père} (1802--1870), ce fut une collaboration avec son partenaire d'écriture Auguste Maquet (1813--1888).
\end{minipage}
\vfill
gutenberg.org/ebooks/17989\\
gutenberg.org/ebooks/17990\\
gutenberg.org/ebooks/17991\\
gutenberg.org/ebooks/17992
\vfill
\divider
\vfill
\begin{minipage}{\textwidth}
Le texte est composé en <EB Garamond,> l'implémentation libre et open source par Georg Mayr-Duffner des célèbres caractères humanistes de Claude Garamond du milieu du seizième siècle. Les pages de titre sont composés en <Bolton Light,> par Paul Lloyd. Les lettrines sont composées en <Floral Capitals,> par Vladimir Nikolic. Le testament de César Spada (chapitre 18) est composé en <Essays 1743,> par John Stracke. 
\end{minipage}
\vfill
github.com/georgd/EB-Garamond\\moorstation.org/typoasis/designers/lloyd/\\www.thibault.org/fonts/essays\\
\vfill
\divider
\vfill
\begin{minipage}{\textwidth}
L'illustration des pages de titre est tiré d'un catalogue, \textit{Katalog der Ausstellung für Buchgewerbe und Photographie,} qui fut publié en 1904. Illustration de la dernière page est un dessin de Albertus van Beest (1820--1860) entitré \textit{Afgemeerde zeilboot op een kalme zee}. L'original est retenu par le Rijksmuseum d'Amsterdam.
\end{minipage}
\vfill
\divider
\vfill
\begin{minipage}{\textwidth}
Cette composition typographique est dédiée au domaine public sous une licence Creative Commons CC0 1.0 Universal: creativecommons.org/publicdomain/zero/1.0/\
\end{minipage}
\vfill
\divider
\vfill
Composé en \LaTeX{}. Dernière révision \today.
\thispagestyle{empty}

\end{document}