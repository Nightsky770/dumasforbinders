\chapter{Éblouissement}

\lettrine{L}{e} soleil était arrivé au tiers de sa course à peu près, et ses rayons de mai donnaient, chauds et vivants, sur ces rochers, qui eux-mêmes semblaient sensibles à sa chaleur; des milliers de cigales, invisibles dans les bruyères, faisaient entendre leur murmure monotone et continu; les feuilles des myrtes et des oliviers s'agitaient frissonnantes, et rendaient un bruit presque métallique; à chaque pas que faisait Edmond sur le granit échauffé, il faisait fuir des lézards qui semblaient des émeraudes; on voyait bondir, sur les talus inclinés, les chèvres sauvages qui parfois y attirent les chasseurs: en un mot, l'île était habitée, vivante, animée, et cependant Edmond s'y sentait seul sous la main de Dieu.

Il éprouvait je ne sais quelle émotion assez semblable à de la crainte: c'était cette défiance du grand jour, qui fait supposer, même dans le désert, que des yeux inquisiteurs sont ouverts sur nous.

Ce sentiment fut si fort, qu'au moment de se mettre à la besogne, Edmond s'arrêta, déposa sa pioche, reprit son fusil, gravit une dernière fois le roc le plus élevé de l'île, et de là jeta un vaste regard sur tout ce qui l'entourait.

Mais, nous devons le dire, ce qui attira son attention, ce ne fut ni cette Corse poétique dont il pouvait distinguer jusqu'aux maisons, ni cette Sardaigne presque inconnue qui lui fait suite, ni l'île d'Elbe aux souvenirs gigantesques, ni enfin cette ligne imperceptible qui s'étendait à l'horizon et qui à l'œil exercé du marin révélait Gênes la superbe et Livourne la commerçante; non: ce fut le brigantin qui était parti au point du jour, et la tartane qui venait de partir. Le premier était sur le point de disparaître au détroit de Bonifacio; l'autre, suivant la route opposée, côtoyait la Corse, qu'elle s'apprêtait à doubler.

Cette vue rassura Edmond.

Il ramena alors les yeux sur les objets qui l'entouraient plus immédiatement; il se vit sur le point le plus élevé de l'île, conique, grêle statue de cet immense piédestal; au-dessous de lui, pas un homme; autour de lui, pas une barque: rien que la mer azurée qui venait battre la base de l'île, et que ce choc éternel bordait d'une frange d'argent.

Alors il descendit d'une marche rapide, mais cependant pleine de prudence: il craignait fort, en un pareil moment, un accident semblable à celui qu'il avait si habilement et si heureusement simulé.

Dantès, comme nous l'avons dit, avait repris le contre-pied des entailles laissées sur les rochers et il avait vu que cette ligne conduisait à une espèce de petite crique cachée comme un bain de nymphe antique; cette crique était assez large à son ouverture et assez profonde à son centre pour qu'un petit bâtiment du genre des spéronares pût y entrer et y demeurer caché. Alors, en suivant le fil des inductions, ce fil qu'aux mains de l'abbé Faria il avait vu guider l'esprit d'une façon si ingénieuse dans le dédale des probabilités, il songea que le cardinal Spada, dans son intérêt à ne pas être vu, avait abordé à cette crique, y avait caché son petit bâtiment, avait suivi la ligne indiquée par des entailles, et avait, à l'extrémité de cette ligne, enfoui son trésor.

C'était cette supposition qui avait ramené Dantès près du rocher circulaire.

Seulement, cette chose inquiétait Edmond et bouleversait toutes les idées qu'il avait en dynamique: comment avait-on pu, sans employer des forces considérables, hisser ce rocher, qui pesait peut-être cinq ou six milliers, sur l'espèce de base où il reposait?

Tout à coup, une idée vint à Dantès.

«Au lieu de le faire monter, se dit-il, on l'aura fait descendre.»

Et lui-même s'élança au-dessus du rocher, afin de chercher la place de sa base première.

En effet, bientôt il vit qu'une pente légère avait été pratiquée; le rocher avait glissé sur sa base et était venu s'arrêter à l'endroit; un autre rocher, gros comme une pierre de taille ordinaire, lui avait servi de cale; des pierres et des cailloux avaient été soigneusement rajustés pour faire disparaître toute solution de continuité; cette espèce de petit ouvrage en maçonnerie avait été recouvert de terre végétale, l'herbe y avait poussé, la mousse s'y était étendue, quelques semences de myrtes et de lentisques s'y étaient arrêtées, et le vieux rocher semblait soudée au sol.

Dantès enleva avec précaution la terre, et reconnut ou crut reconnaître tout cet ingénieux artifice.

Alors il se mit à attaquer avec sa pioche cette muraille intermédiaire cimentée par le temps.

Après un travail de dix minutes, la muraille céda, et un trou à y fourrer le bras fut ouvert.

Dantès alla couper l'olivier le plus fort qu'il put trouver, le dégarnit de ses branches, l'introduisit dans le trou et en fit un levier.

Mais le roc était à la fois trop lourd et calé trop solidement par le rocher inférieur, pour qu'une force humaine, fût-ce celle d'Hercule lui-même, pût l'ébranler.

Dantès réfléchit alors que c'était cette cale elle-même qu'il fallait attaquer.

Mais par quel moyen?

Dantès jeta les yeux autour de lui, comme font les hommes embarrassés; et son regard tomba sur une corne de mouflon pleine de poudre que lui avait laissée son ami Jacopo.

Il sourit: l'invention infernale allait faire son œuvre.

À l'aide de sa pioche Dantès creusa, entre le rocher supérieur et celui sur lequel il était posé, un conduit de mine comme ont l'habitude de faire les pionniers, lorsqu'ils veulent épargner au bras de l'homme une trop grande fatigue, puis il le bourra de poudre; puis, effilant son mouchoir et le roulant dans le salpêtre, il en fit une mèche.

Le feu mis à cette mèche, Dantès s'éloigna.

L'explosion ne se fit pas attendre: le rocher supérieur fut en un instant soulevé par l'incalculable force, le rocher inférieur vola en éclats; par la petite ouverture qu'avait d'abord pratiquée Dantès, s'échappa tout un monde d'insectes frémissants, et une couleuvre énorme, gardien de ce chemin mystérieux, roula sur ses volutes bleuâtres et disparut.

Dantès s'approcha: le rocher supérieur, désormais sans appui, inclinait vers l'abîme; l'intrépide chercheur en fit le tour, choisit l'endroit le plus vacillant, appuya son levier dans une de ses arêtes et, pareil à Sisyphe, se raidit de toute sa puissance contre le rocher.

Le rocher, déjà ébranlé par la commotion chancela; Dantès redoubla d'efforts: on eût dit un de ces Titans qui déracinaient des montagnes pour faire la guerre au maître des dieux. Enfin le rocher céda, roula, bondit, se précipita et disparut, s'engloutissant dans la mer.

Il laissait découverte une place circulaire, et mettait au jour un anneau de fer scellé au milieu d'une dalle de forme carrée.

Dantès poussa un cri de joie et d'étonnement: jamais plus magnifique résultat n'avait couronné une première tentative.

Il voulut continuer; mais ses jambes tremblaient si fort, mais son cœur battait si violemment, mais un nuage si brûlant passait devant ses yeux, qu'il fut forcé de s'arrêter.

Ce moment d'hésitation eut la durée de l'éclair. Edmond passa son levier dans l'anneau, leva vigoureusement, et la dalle descellée s'ouvrit, découvrant la pente rapide d'une sorte d'escalier qui allait s'enfonçant dans l'ombre d'une grotte de plus en plus obscure.

Un autre se fût précipité, eût poussé des exclamations de joie; Dantès s'arrêta, pâlit, douta.

«Voyons, se dit-il, soyons homme! accoutumé à l'adversité, ne nous laissons pas abattre par une déception; ou sans cela ce serait donc pour rien que j'aurais souffert! Le cœur se brise, lorsque après avoir été dilaté outre mesure par l'espérance à la tiède haleine il rentre et se renferme dans la froide réalité! Faria a fait un rêve: le cardinal Spada n'a rien enfoui dans cette grotte, peut-être même n'y est-il jamais venu, ou, s'il y est venu, César Borgia l'intrépide aventurier, l'infatigable et sombre larron, y est venu après lui, a découvert sa trace, a suivi les mêmes brisées que moi, comme moi a soulevé cette pierre, et, descendu avant moi, ne m'a rien laissé à prendre après lui.»

Il resta un moment immobile, pensif, les yeux fixés sur cette ouverture sombre et continue.

«Or, maintenant que je ne compte plus sur rien, maintenant que je me suis dit qu'il serait insensé de conserver quelque espoir, la suite de cette aventure est pour moi une chose de curiosité, voilà tout.»

Et il demeura encore immobile et méditant.

«Oui, oui, ceci est une aventure à trouver sa place dans la vie mêlée d'ombre et de lumière de ce royal bandit, dans ce tissu d'événements étranges qui composent la trame diaprée de son existence; ce fabuleux événement a dû s'enchaîner invinciblement aux autres choses; oui, Borgia est venu quelque nuit ici, un flambeau d'une main, une épée de l'autre, tandis qu'à vingt pas de lui, au pied de cette roche peut-être, se tenaient, sombres et menaçants, deux sbires interrogeant la terre, l'air et la mer, pendant que leur maître entrait comme je vais le faire, secouant les ténèbres de son bras redoutable et flamboyant.

«Oui; mais des sbires auxquels il aura livré ainsi son secret, qu'en aura fait César? se demanda Dantès.

«Ce qu'on fit, se répondit-il en souriant, des ensevelisseurs d'Alaric, que l'on enterra avec l'enseveli.

«Cependant s'il y était venu, reprit Dantès, il eût retrouvé et pris le trésor; Borgia, l'homme qui comparait l'Italie à un artichaut et qui la mangeait feuille à feuille, Borgia savait trop bien l'emploi du temps pour avoir perdu le sien à replacer ce rocher sur sa base.

«Descendons.»

Alors il descendit, le sourire du doute sur les lèvres, en murmurant ce dernier mot de la sagesse humaine: Peut-être!\dots

Mais, au lieu des ténèbres qu'il s'était attendu trouver, au lieu d'une atmosphère opaque et viciée, Dantès ne vit qu'une douce lueur décomposée en jour bleuâtre; l'air et la lumière filtraient non seulement par l'ouverture qui venait d'être pratiquée, mais encore par des gerçures de rochers invisibles du sol extérieur, et à travers lesquels on voyait l'azur du ciel où se jouaient les branches tremblotantes des chênes verts et des ligaments épineux et rampants des ronces.

Après quelques secondes de séjour dans cette grotte, dont l'atmosphère plutôt tiède qu'humide, plutôt odorante que fade, était à la température de l'île ce que la lueur bleue était au soleil, le regard de Dantès, habitué, comme nous l'avons dit, aux ténèbres, put sonder les angles les plus reculés de la caverne: elle était de granit dont les facettes pailletées étincelaient comme des diamants.

«Hélas! se dit Edmond en souriant, voilà sans doute tous les trésors qu'aura laissés le cardinal; et ce bon abbé, en voyant en rêve ces murs tout resplendissants, se sera entretenu dans ses riches espérances.» Mais Dantès se rappela les termes du testament, qu'il savait par cœur: «Dans l'angle le plus éloigné de la seconde ouverture», disait ce testament.

Dantès avait pénétré seulement dans la première grotte, il fallait chercher maintenant l'entrée de la seconde.

Dantès s'orienta: cette seconde grotte devait naturellement s'enfoncer dans l'intérieur de l'île; il examina les souches des pierres, et il alla frapper à une des parois qui lui parut celle où devait être cette ouverture, masquée sans doute pour plus grande précaution.

La pioche résonna pendant un instant, tirant du rocher un son mat, dont la compacité faisait germer la sueur au front de Dantès; enfin il sembla au mineur persévérant qu'une portion de la muraille granitique répondait par un écho plus sourd et plus profond à l'appel qui lui était fait; il rapprocha son regard ardent de la muraille et reconnut, avec le tact du prisonnier, ce que nul autre n'eût reconnu peut-être: c'est qu'il devait y avoir là une ouverture.

Cependant, pour ne pas faire une besogne inutile, Dantès, qui, comme César Borgia, avait étudié le prix du temps, sonda les autres parois avec sa pioche, interrogea le sol avec la crosse de son fusil, ouvrit le sable aux endroits suspects, et n'ayant rien trouvé rien reconnu, revint à la portion de la muraille qui rendait ce son consolateur.

Il frappa de nouveau et avec plus de force.

Alors il vit une chose singulière, c'est que, sous les coups de l'instrument, une espèce d'enduit, pareil à celui qu'on applique sur les murailles pour peindre à fresque, se soulevait et tombait en écailles découvrant une pierre blanchâtre et molle, pareille à nos pierres de taille ordinaires. On avait fermé l'ouverture du rocher avec des pierres d'une autre nature, puis on avait étendu sur ces pierres cet enduit, puis sur cet enduit on avait imité la teinte et le cristallin du granit.

Dantès frappa alors par le bout aigu de la pioche, qui entra d'un pouce dans la porte-muraille.

C'était là qu'il fallait fouiller.

Par un mystère étrange de l'organisation humaine, plus les preuves que Faria ne s'était pas trompé devaient, en s'accumulant, rassurer Dantès, plus son cœur défaillant se laissait aller au doute et presque au découragement: cette nouvelle expérience, qui aurait dû lui donner une force nouvelle, lui ôta la force qui lui restait: la pioche descendit, s'échappant presque de ses mains; il la posa sur le sol, s'essuya le front et remonta vers le jour, se donnant à lui-même le prétexte de voir si personne ne l'épiait, mais, en réalité, parce qu'il avait besoin d'air, parce qu'il sentait qu'il allait s'évanouir.

L'île était déserte, et le soleil à son zénith semblait la couvrir de son œil de feu; au loin, de petites barques de pécheurs ouvraient leurs ailes sur la mer d'un bleu de saphir.

Dantès n'avait encore rien pris: mais c'était bien long de manger dans un pareil moment; il avala une gorgée de rhum et rentra dans la grotte le cœur raffermi.

La pioche qui lui avait semblé si lourde était redevenue légère; il la souleva comme il eût fait d'une plume, et se remit vigoureusement à la besogne.

Après quelques coups, il s'aperçut que les pierres n'étaient point scellées, mais seulement posées les unes sur les autres et recouvertes de l'enduit dont nous avons parlé; il introduisit dans une des fissures la pointe de la pioche, pesa sur le manche et vit avec joie la pierre tomber à ses pieds.

Dès lors, Dantès n'eut plus qu'à tirer chaque pierre à lui avec la dent de fer de la pioche, et chaque pierre à son tour tomba près de la première.

Dès la première ouverture, Dantès eût pu entrer; mais en tardant de quelques instants, c'était retarder la certitude en se cramponnant à l'espérance.

Enfin, après une nouvelle hésitation d'un instant, Dantès passa de cette première grotte dans la seconde.

Cette seconde grotte était plus basse, plus sombre et d'un aspect plus effrayant que la première; l'air, qui n'y pénétrait que par l'ouverture pratiquée à l'instant même, avait cette odeur méphitique que Dantès s'était étonné de ne pas trouver dans la première.

Dantès donna le temps à l'air extérieur d'aller raviver cette atmosphère morte, et entra.

À gauche de l'ouverture, était un angle profond et sombre.

Mais, nous l'avons dit, pour l'œil de Dantès il n'y avait pas de ténèbres.

Il sonda du regard la seconde grotte: elle était vide comme la première.

Le trésor, s'il existait, était enterré dans cet angle sombre.

L'heure de l'angoisse était arrivée; deux pieds de terre à fouiller, c'était tout ce qui restait à Dantès entre la suprême joie et le suprême désespoir.

Il s'avança vers l'angle, et, comme pris d'une résolution subite, il attaqua le sol hardiment.

Au cinquième ou sixième coup de pioche, le fer résonna sur du fer.

Jamais tocsin funèbre, jamais glas frémissant ne produisit pareil effet sur celui qui l'entendit. Dantès n'aurait rien rencontré qu'il ne fût certes pas devenu plus pâle.

Il sonda à côté de l'endroit où il avait sondé déjà, et rencontra la même résistance mais non pas le même son.

«C'est un coffre de bois, cerclé de fer», dit-il.

En ce moment, une ombre rapide passa interceptant le jour.

Dantès laissa tomber sa pioche, saisit son fusil, repassa par l'ouverture, et s'élança vers le jour.

Une chèvre sauvage avait bondi par-dessus la première entrée de la grotte et broutait à quelques pas de là.

C'était une belle occasion de s'assurer son dîner, mais Dantès eut peur que la détonation du fusil n'attirât quelqu'un.

Il réfléchit un instant, coupa un arbre résineux, alla l'allumer au feu encore fumant où les contrebandiers avaient fait cuire leur déjeuner, et revint avec cette torche.

Il ne voulait perdre aucun détail de ce qu'il allait voir.

Il approcha la torche du trou informe et inachevé, et reconnut qu'il ne s'était pas trompé: ses coups avaient alternativement frappé sur le fer et sur le bois.

Il planta sa torche dans la terre et se remit à l'œuvre.

En un instant, un emplacement de trois pieds de long sur deux pieds de large à peu près fut déblayé, et Dantès put reconnaître un coffre de bois de chêne cerclé de fer ciselé. Au milieu du couvercle resplendissaient, sur une plaque d'argent que la terre n'avait pu ternir, les armes de la famille Spada, c'est-à-dire une épée posée en pal sur un écusson ovale, comme sont les écussons italiens, et surmonté d'un chapeau de cardinal.

Dantès les reconnut facilement: l'abbé Faria les lui avait tant de fois dessinées!

Dès lors, il n'y avait plus de doute, le trésor était bien là; on n'eût pas pris tant de précautions pour remettre à cette place un coffre vide.

En un instant, tous les alentours du coffre furent déblayés, et Dantès vit tour à tour apparaître la serrure du milieu, placée entre deux cadenas, et les anses des faces latérales; tout cela était ciselé comme on ciselait à cette époque, où l'art rendait précieux les plus vils métaux.

Dantès prit le coffre par les anses et essaya de le soulever: c'était chose impossible.

Dantès essaya de l'ouvrir: serrure et cadenas étaient fermés; les fidèles gardiens semblaient ne pas vouloir rendre leur trésor.

Dantès introduisit le côté tranchant de sa pioche entre le coffre et le couvercle, pesa sur le manche de la pioche, et le couvercle, après avoir crié, éclata. Une large ouverture des ais rendit les ferrures inutiles, elles tombèrent à leur tour, serrant encore de leurs ongles tenaces les planches entamées par leur chute, et le coffre fut découvert.

Une fièvre vertigineuse s'empara de Dantès; il saisit son fusil, l'arma et le plaça près de lui. D'abord il ferma les yeux, comme font les enfants, pour apercevoir, dans la nuit étincelante de leur imagination, plus d'étoiles qu'ils n'en peuvent compter dans un ciel encore éclairé, puis il les rouvrit et demeura ébloui.

Trois compartiments scindaient le coffre.

Dans le premier brillaient de rutilants écus d'or aux fauves reflets.

Dans le second, des lingots mal polis et rangés en bon ordre, mais qui n'avaient de l'or que le poids et la valeur.

Dans le troisième enfin, à demi plein, Edmond remua à poignée les diamants, les perles, les rubis, qui, cascade étincelante, faisaient, en retombant les uns sur les autres, le bruit de la grêle sur les vitres.

Après avoir touché, palpé, enfoncé ses mains frémissantes dans l'or et les pierreries, Edmond se releva et prit sa course à travers les cavernes avec la tremblante exaltation d'un homme qui touche à la folie. Il sauta sur un rocher d'où il pouvait découvrir la mer, et n'aperçut rien; il était seul, bien seul, avec ces richesses incalculables, inouïes, fabuleuses, qui lui appartenaient: seulement rêvait-il ou était-il éveillé? faisait-il un songe fugitif ou étreignait-il corps à corps une réalité?

Il avait besoin de revoir son or, et cependant il sentait qu'il n'aurait pas la force, en ce moment, d'en soutenir la vue. Un instant, il appuya ses deux mains sur le haut de sa tête, comme pour empêcher sa raison de s'enfuir; puis il s'élança tout au travers de l'île, sans suivre, non pas de chemin, il n'y en a pas dans l'île de Monte-Cristo, mais de ligne arrêtée, faisant fuir les chèvres sauvages et effrayant les oiseaux de mer par ses cris et ses gesticulations. Puis, par un détour, il revint, doutant encore, se précipitant de la première grotte dans la seconde, et se retrouvant en face cette mine d'or et de diamants.

Cette fois, il tomba à genoux, comprimant de ses deux mains convulsives son cœur bondissant, et murmurant une prière intelligible pour Dieu seul.

Bientôt, il se sentit plus calme et partant plus heureux, car de cette heure seulement il commençait à croire à sa félicité.

Il se mit alors à compter sa fortune; il y avait mille lingots d'or de deux à trois livres chacun; ensuite, il empila vingt-cinq mille écus d'or, pouvant valoir chacun quatre-vingts francs de notre monnaie actuelle, tous à l'effigie du pape Alexandre VI et de ses prédécesseurs, et il s'aperçut que le compartiment n'était qu'à moitié vide; enfin, il mesura dix fois la capacité de ses deux mains en perles, en pierreries, en diamants, dont beaucoup, montés par les meilleurs orfèvres de l'époque, offraient une valeur d'exécution remarquable, même à côté de leur valeur intrinsèque.

Dantès vit le jour baisser et s'éteindre peu à peu. Il craignit d'être surpris s'il restait dans la caverne, et sortit son fusil à la main. Un morceau de biscuit et quelques gorgées de vin furent son souper. Puis il replaça la pierre, se coucha dessus, et dormit à peine quelques heures, couvrant de son corps l'entrée de la grotte.

Cette nuit fut à la fois une de ces nuits délicieuses et terribles, comme cet homme aux foudroyantes émotions en avait déjà passé deux ou trois dans la vie.



