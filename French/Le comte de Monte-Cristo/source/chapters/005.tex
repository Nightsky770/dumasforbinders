\chapter{Le repas des fiançailles}

\lettrine{L}{e} lendemain fut un beau jour. Le soleil se leva pur et brillant, et les premiers rayons d'un rouge pourpre diaprèrent de leurs rubis les pointes écumeuses des vagues.

\zz
Le repas avait été préparé au premier étage de cette même Réserve, avec la tonnelle de laquelle nous avons déjà fait connaissance. C'était une grande salle éclairée par cinq ou six fenêtres, au-dessus de chacune desquelles (explique le phénomène qui pourra!) était écrit le nom d'une des grandes villes de France.

Une balustrade en bois, comme le reste du bâtiment, régnait tout le long de ces fenêtres.

Quoique le repas ne fût indiqué que pour midi, dès onze heures du matin, cette balustrade était chargée de promeneurs impatients. C'étaient les marins privilégiés du \textit{Pharaon} et quelques soldats, amis de Dantès. Tous avaient, pour faire honneur aux fiancés, fait voir le jour à leurs plus belles toilettes.

Le bruit circulait, parmi les futurs convives, que les armateurs du \textit{Pharaon} devaient honorer de leur présence le repas de noces de leur second; mais c'était de leur part un si grand honneur accordé à Dantès que personne n'osait encore y croire.

Cependant Danglars, en arrivant avec Caderousse, confirma à son tour cette nouvelle. Il avait vu le matin M. Morrel lui-même, et M. Morrel lui avait dit qu'il viendrait dîner à la Réserve.

En effet, un instant après eux, M. Morrel fit à son tour son entrée dans la chambre et fut salué par les matelots du \textit{Pharaon} d'un hourra unanime d'applaudissements. La présence de l'armateur était pour eux la confirmation du bruit qui courait déjà que Dantès serait nommé capitaine; et comme Dantès était fort aimé à bord, ces braves gens remerciaient ainsi l'armateur de ce qu'une fois par hasard son choix était en harmonie avec leurs désirs. À peine M. Morrel fut-il entré qu'on dépêcha unanimement Danglars et Caderousse vers le fiancé: ils avaient mission de le prévenir de l'arrivée du personnage important dont la vue avait produit une si vive sensation, et de lui dire de se hâter.

Danglars et Caderousse partirent tout courant mais ils n'eurent pas fait cent pas, qu'à la hauteur du magasin à poudre ils aperçurent la petite troupe qui venait.

Cette petite troupe se composait de quatre jeunes filles amies de Mercédès et Catalanes comme elle, et qui accompagnaient la fiancée à laquelle Edmond donnait le bras. Près de la future marchait le père Dantès, et derrière eux venait Fernand avec son mauvais sourire.

Ni Mercédès ni Edmond ne voyaient ce mauvais sourire de Fernand. Les pauvres enfants étaient si heureux qu'ils ne voyaient qu'eux seuls et ce beau ciel pur qui les bénissait.

Danglars et Caderousse s'acquittèrent de leur mission d'ambassadeurs; puis après avoir échangé une poignée de main bien vigoureuse et bien amicale avec Edmond, ils allèrent, Danglars prendre place près de Fernand, Caderousse se ranger aux côtés du père Dantès, centre de l'attention générale.

Ce vieillard était vêtu de son bel habit de taffetas épinglé, orné de larges boutons d'acier, taillés à facettes. Ses jambes grêles, mais nerveuses, s'épanouissaient dans de magnifiques bas de coton mouchetés, qui sentaient d'une lieue la contrebande anglaise. À son chapeau à trois cornes pendait un flot de rubans blancs et bleus.

Enfin, il s'appuyait sur un bâton de bois tordu et recourbé par le haut comme un pedum antique. On eût dit un de ces muscadins qui paradaient en 1796 dans les jardins nouvellement rouverts du Luxembourg et des Tuileries.

Près de lui, nous l'avons dit, s'était glissé Caderousse, Caderousse que l'espérance d'un bon repas avait achevé de réconcilier avec les Dantès, Caderousse à qui il restait dans la mémoire un vague souvenir de ce qui s'était passé la veille, comme en se réveillant le matin on trouve dans son esprit l'ombre du rêve qu'on a fait pendant le sommeil.

Danglars, en s'approchant de Fernand, avait jeté sur l'amant désappointé un regard profond. Fernand, marchant derrière les futurs époux, complètement oublié par Mercédès, qui dans cet égoïsme juvénile et charmant de l'amour n'avait d'yeux que pour son Edmond. Fernand était pâle, puis rouge par bouffées subites qui disparaissaient pour faire place chaque fois à une pâleur croissante. De temps en temps, il regardait du côté de Marseille, et alors un tremblement nerveux et involontaire faisait frissonner ses membres. Fernand semblait attendre ou tout au moins prévoir quelque grand événement.

Dantès était simplement vêtu. Appartenant à la marine marchande, il avait un habit qui tenait le milieu entre l'uniforme militaire et le costume civil; et sous cet habit, sa bonne mine, que rehaussaient encore la joie et la beauté de sa fiancée, était parfaite.

Mercédès était belle comme une de ces Grecques de Chypre ou de Céos, aux yeux d'ébène et aux lèvres de corail. Elle marchait de ce pas libre et franc dont marchent les Arlésiennes et les Andalouses. Une fille des villes eût peut-être essayé de cacher sa joie sous un voile ou tout au moins sous le velours de ses paupières, mais Mercédès souriait et regardait tous ceux qui l'entouraient, et son sourire et son regard disaient aussi franchement qu'auraient pu le dire ses paroles: Si vous êtes mes amis, réjouissez-vous avec moi, car, en vérité, je suis bien heureuse!

Dès que les fiancés et ceux qui les accompagnaient furent en vue de la Réserve, M. Morrel descendit et s'avança à son tour au-devant d'eux, suivi des matelots et des soldats avec lesquels il était resté, et auxquels il avait renouvelé la promesse déjà faite à Dantès qu'il succéderait au capitaine Leclère. En le voyant venir, Edmond quitta le bras de sa fiancée et le passa sous celui de M. Morrel. L'armateur et la jeune fille donnèrent alors l'exemple en montant les premiers l'escalier de bois qui conduisait à la chambre où le dîner était servi, et qui cria pendant cinq minutes sous les pas pesants des convives.

«Mon père, dit Mercédès en s'arrêtant au milieu de la table, vous à ma droite, je vous prie; quant à ma gauche, j'y mettrai celui qui m'a servi de frère», fit-elle avec une douceur qui pénétra au plus profond du cœur de Fernand comme un coup de poignard.

Ses lèvres blêmirent, et sous la teinte bistrée de son mâle visage on put voir encore une fois le sang se retirer peu à peu pour affluer au cœur.

Pendant ce temps, Dantès avait exécuté la même manœuvre; à sa droite il avait mis M. Morrel, à sa gauche Danglars; puis de la main il avait fait signe à chacun de se placer à sa fantaisie.

Déjà couraient autour de la table les saucissons d'Arles à la chair brune et au fumet accentué, les langoustes à la cuirasse éblouissante, les prayres à la coquille rosée, les oursins, qui semblent des châtaignes entourées de leur enveloppe piquante, les clovisses, qui ont la prétention de remplacer avec supériorité, pour les gourmets du Midi, les huîtres du Nord; enfin tous ces hors-d'œuvre délicats que la vague roule sur sa rive sablonneuse, et que les pêcheurs reconnaissants désignent sous le nom générique de fruits de mer.

«Un beau silence! dit le vieillard en savourant un verre de vin jaune comme la topaze, que le père Pamphile en personne venait d'apporter devant Mercédès. Dirait-on qu'il y a ici trente personnes qui ne demandent qu'à rire.

—Eh! un mari n'est pas toujours gai, dit Caderousse.

—Le fait est, dit Dantès, que je suis trop heureux en ce moment pour être gai. Si c'est comme cela que vous l'entendez, voisin, vous avez raison! La joie fait quelquefois un effet étrange, elle oppresse comme la douleur.»

Danglars observa Fernand, dont la nature impressionnable absorbait et renvoyait chaque émotion.

«Allons donc, dit-il, est-ce que vous craindriez quelque chose? il me semble, au contraire, que tout va selon vos désirs!

—Et c'est justement cela qui m'épouvante, dit Dantès, il me semble que l'homme n'est pas fait pour être si facilement heureux! Le bonheur est comme ces palais des îles enchantées dont les dragons gardent les portes. Il faut combattre pour le conquérir, et moi, en vérité, je ne sais en quoi j'ai mérité le bonheur d'être le mari de Mercédès.

—Le mari, le mari, dit Caderousse en riant, pas encore, mon capitaine; essaie un peu de faire le mari, et tu verras comme tu seras reçu!»

Mercédès rougit. Fernand se tourmentait sur sa chaise, tressaillait au moindre bruit, et de temps en temps essuyait de larges plaques de sueur qui perlaient sur son front, comme les premières gouttes d'une pluie d'orage.

«Ma foi, dit Dantès, voisin Caderousse, ce n'est point la peine de me démentir pour si peu. Mercédès n'est point encore ma femme, c'est vrai\dots (il tira sa montre). Mais, dans une heure et demie elle le sera!»

Chacun poussa un cri de surprise, à l'exception du père Dantès, dont le large rire montra les dents encore belles. Mercédès sourit et ne rougit plus. Fernand saisit convulsivement le manche de son couteau.

«Dans une heure! dit Danglars pâlissant lui-même; et comment cela?

—Oui, mes amis, répondit Dantès, grâce au crédit de M. Morrel, l'homme après mon père auquel je dois le plus au monde, toutes les difficultés sont aplanies. Nous avons acheté les bans, et à deux heures et demie le maire de Marseille nous attend à l'hôtel de ville. Or, comme une heure et un quart viennent de sonner, je ne crois pas me tromper de beaucoup en disant que dans une heure trente minutes Mercédès s'appellera Mme Dantès.»

Fernand ferma les yeux: un nuage de feu brûla ses paupières; il s'appuya à la table pour ne pas défaillir, et, malgré tous ses efforts, ne put retenir un gémissement sourd qui se perdit dans le bruit des rires et des félicitations de l'assemblée.

«C'est bien agir, cela, hein, dit le père Dantès. Cela s'appelle-t-il perdre son temps, à votre avis? Arrivé d'hier au matin, marié aujourd'hui à trois heures! Parlez-moi des marins pour aller rondement en besogne.

—Mais les autres formalités, objecta timidement Danglars: le contrat, les écritures?\dots

—Le contrat, dit Dantès en riant, le contrat est tout fait: Mercédès n'a rien, ni moi non plus! Nous nous marions sous le régime de la communauté, et voilà! Ça n'a pas été long à écrire et ce ne sera pas cher à payer.»

Cette plaisanterie excita une nouvelle explosion de joie et de bravos.

«Ainsi, ce que nous prenions pour un repas de fiançailles, dit Danglars, est tout bonnement un repas de noces.

—Non pas, dit Dantès; vous n'y perdrez rien, soyez tranquilles. Demain matin, je pars pour Paris. Quatre jours pour aller, quatre jours pour revenir, un jour pour faire en conscience la commission dont je suis chargé, et le 1\ier{} mars je suis de retour; au 2 mars donc le véritable repas de noces.»

Cette perspective d'un nouveau festin redoubla l'hilarité au point que le père Dantès, qui au commencement du dîner se plaignait du silence, faisait maintenant, au milieu de la conversation générale, de vains efforts pour placer son vœu de prospérité en faveur des futurs époux.

Dantès devina la pensée de son père et y répondit par un sourire plein d'amour. Mercédès commença de regarder l'heure au coucou de la salle et fit un petit signe à Edmond.

Il y avait autour de la table cette hilarité bruyante et cette liberté individuelle qui accompagnent, chez les gens de condition inférieure, la fin des repas. Ceux qui étaient mécontents de leur place s'étaient levés de table et avaient été chercher d'autres voisins. Tout le monde commençait à parler à la fois, et personne ne s'occupait de répondre à ce que son interlocuteur lui disait, mais seulement à ses propres pensées.

La pâleur de Fernand était presque passée sur les joues de Danglars; quant à Fernand lui-même, il ne vivait plus et semblait un damné dans le lac de feu. Un des premiers, il s'était levé et se promenait de long en large dans la salle, essayant d'isoler son oreille du bruit des chansons et du choc des verres.

Caderousse s'approcha de lui au moment où Danglars, qu'il semblait fuir, venait de le rejoindre dans un angle de la salle.

«En vérité, dit Caderousse, à qui les bonnes façons de Dantès et surtout le bon vin du père Pamphile avaient enlevé tous les restes de la haine dont le bonheur inattendu de Dantès avait jeté les germes dans son âme, en vérité, Dantès est un gentil garçon; et quand je le vois assis près de sa fiancée, je me dis que ç'eût été dommage de lui faire la mauvaise plaisanterie que vous complotiez hier.

—Aussi, dit Danglars, tu as vu que la chose n'a pas eu de suite; ce pauvre M. Fernand était si bouleversé qu'il m'avait fait de la peine d'abord; mais du moment qu'il en a pris son parti, au point de s'être fait le premier garçon de noces de son rival, il n'y a plus rien à dire.»

Caderousse regarda Fernand, il était livide.

«Le sacrifice est d'autant plus grand, continua Danglars, qu'en vérité la fille est belle. Peste! l'heureux coquin que mon futur capitaine; je voudrais m'appeler Dantès douze heures seulement.

—Partons-nous? demanda la douce voix de Mercédès; voici deux heures qui sonnent, et l'on nous attend à deux heures un quart.

—Oui, oui, partons! dit Dantès en se levant vivement.

—Partons!» répétèrent en chœur tous les convives.

Au même instant, Danglars, qui ne perdait pas de vue Fernand assis sur le rebord de la fenêtre, le vit ouvrir des yeux hagards, se lever comme par un mouvement convulsif, et retomber assis sur l'appui de cette croisée; presque au même instant un bruit sourd retentit dans l'escalier; le retentissement d'un pas pesant, une rumeur confuse de voix mêlées à un cliquetis d'armes couvrirent les exclamations des convives, si bruyantes qu'elles fussent, et attirèrent l'attention générale, qui se manifesta à l'instant même par un silence inquiet. Le bruit s'approcha: trois coups retentirent dans le panneau de la porte; chacun regarda son voisin d'un air étonné.

«Au nom de la loi!» cria une voix vibrante, à laquelle aucune voix ne répondit.

Aussitôt la porte s'ouvrit, et un commissaire, ceint de son écharpe, entra dans la salle, suivi de quatre soldats armés, conduits par un caporal.

L'inquiétude fit place à la terreur.

«Qu'y a-t-il? demanda l'armateur en s'avançant au-devant du commissaire qu'il connaissait; bien certainement, monsieur, il y a méprise.

—S'il y a méprise, monsieur Morrel, répondit le commissaire croyez que la méprise sera promptement réparée; en attendant, je suis porteur d'un mandat d'arrêt; et quoique ce soit avec regret que je remplisse ma mission, il ne faut pas moins que je la remplisse: lequel de vous, messieurs, est Edmond Dantès?»

Tous les regards se tournèrent vers le jeune homme qui, fort ému, mais conservant sa dignité, fit un pas en avant et dit:

«C'est moi, monsieur, que me voulez-vous?

—Edmond Dantès, reprit le commissaire, au nom de la loi, je vous arrête!

—Vous m'arrêtez! dit Edmond avec une légère pâleur, mais pourquoi m'arrêtez-vous?

—Je l'ignore, monsieur, mais votre premier interrogatoire vous l'apprendra.»

M. Morrel comprit qu'il n'y avait rien à faire contre l'inflexibilité de la situation: un commissaire ceint de son écharpe n'est plus un homme, c'est la statue de la loi, froide, sourde, muette.

Le vieillard, au contraire, se précipita vers l'officier; il y a des choses que le cœur d'un père ou d'une mère ne comprendra jamais.

Il pria et supplia: larmes et prières ne pouvaient rien; cependant son désespoir était si grand, que le commissaire en fut touché.

«Monsieur, dit-il, tranquillisez-vous; peut-être votre fils a-t-il négligé quelque formalité de douane ou de santé, et, selon toute probabilité, lorsqu'on aura reçu de lui les renseignements qu'on désire en tirer, il sera remis en liberté.

—Ah çà! qu'est-ce que cela signifie? demanda en fronçant le sourcil Caderousse à Danglars, qui jouait la surprise.

—Le sais-je, moi? dit Danglars; je suis comme toi: je vois ce qui se passe, je n'y comprends rien, et je reste confondu.»

Caderousse chercha des yeux Fernand: il avait disparu. Toute la scène de la veille se représenta alors à son esprit avec une effrayante lucidité. On eût dit que la catastrophe venait de tirer le voile que l'ivresse de la veille avait jeté entre lui et sa mémoire.

«Oh! oh! dit-il d'une voix rauque, serait-ce la suite de la plaisanterie dont vous parliez hier, Danglars? En ce cas, malheur à celui qui l'aurait faite, car elle est bien triste.

—Pas du tout! s'écria Danglars, tu sais bien, au contraire, que j'ai déchiré le papier.

—Tu ne l'as pas déchiré, dit Caderousse; tu l'as jeté dans un coin, voilà tout.

—Tais-toi, tu n'as rien vu, tu étais ivre.

—Où est Fernand? demanda Caderousse.

—Le sais-je, moi! répondit Danglars, à ses affaires probablement: mais, au lieu de nous occuper de cela, allons donc porter du secours à ces pauvres affligés.»

En effet, pendant cette conversation, Dantès avait en souriant, serré la main à tous ses amis, et s'était constitué prisonnier en disant:

«Soyez tranquilles, l'erreur va s'expliquer, et probablement que je n'irai même pas jusqu'à la prison.

—Oh! bien certainement, j'en répondrais», dit Danglars qui, en ce moment, s'approchait, comme nous l'avons dit, du groupe principal.

Dantès descendit l'escalier, précédé du commissaire de police et entouré par les soldats. Une voiture, dont la portière était tout ouverte, attendait à la porte, il y monta, deux soldats et le commissaire montèrent après lui; la portière se referma, et la voiture reprit le chemin de Marseille.

«Adieu, Dantès! adieu, Edmond!» s'écria Mercédès en s'élançant sur la balustrade.

Le prisonnier entendit ce dernier cri, sorti comme un sanglot du cœur déchiré de sa fiancée; il passa la tête par la portière, cria: «Au revoir, Mercédès!» et disparut à l'un des angles du fort Saint-Nicolas.

«Attendez-moi ici, dit l'armateur, je prends la première voiture que je rencontre, je cours à Marseille, et je vous rapporte des nouvelles.

—Allez! crièrent toutes les voix, allez! et revenez bien vite!»

Il y eut, après ce double départ, un moment de stupeur terrible parmi tous ceux qui étaient restés.

Le vieillard et Mercédès restèrent quelque temps isolés, chacun dans sa propre douleur; mais enfin leurs yeux se rencontrèrent; ils se reconnurent comme deux victimes frappées du même coup, et se jetèrent dans les bras l'un de l'autre.

Pendant ce temps, Fernand rentra, se versa un verre d'eau qu'il but, et alla s'asseoir sur une chaise.

Le hasard fit que ce fut sur une chaise voisine que vint tomber Mercédès en sortant des bras du vieillard.

Fernand, par un mouvement instinctif, recula sa chaise.

«C'est lui, dit à Danglars Caderousse, qui n'avait pas perdu de vue le Catalan.

—Je ne crois pas, répondit Danglars, il était trop bête; en tout cas, que le coup retombe sur celui qui l'a fait.

—Tu ne me parles pas de celui qui l'a conseillé, dit Caderousse.

—Ah! ma foi, dit Danglars, si l'on était responsable de tout ce que l'on dit en l'air!

—Oui, lorsque ce que l'on dit en l'air retombe par la pointe.»

Pendant ce temps, les groupes commentaient l'arrestation de toutes les manières.

«Et vous, Danglars, dit une voix, que pensez-vous de cet événement?

—Moi, dit Danglars, je crois qu'il aura rapporté quelques ballots de marchandises prohibées.

—Mais si c'était cela, vous devriez le savoir, Danglars, vous qui étiez agent comptable.

—Oui, c'est vrai; mais l'agent comptable ne connaît que les colis qu'on lui déclare: je sais que nous sommes chargés de coton, voilà tout; que nous avons pris le chargement à Alexandrie, chez M. Pastret, et à Smyrne, chez M. Pascal; ne m'en demandez pas davantage.

—Oh! je me rappelle maintenant, murmura le pauvre père, se rattachant à ce débris, qu'il m'a dit hier qu'il avait pour moi une caisse de café et une caisse de tabac.

—Voyez-vous, dit Danglars, c'est cela: en notre absence, la douane aura fait une visite à bord du \textit{Pharaon}, et elle aura découvert le pot aux roses.»

Mercédès ne croyait point à tout cela; car, comprimée jusqu'à ce moment, sa douleur éclata tout à coup en sanglots.

«Allons, allons, espoir! dit, sans trop savoir ce qu'il disait, le père Dantès.

—Espoir! répéta Danglars.

—Espoir», essaya de murmurer Fernand.

Mais ce mot l'étouffait; ses lèvres s'agitèrent, aucun son ne sortit de sa bouche.

«Messieurs, cria un des convives resté en vedette sur la balustrade; messieurs, une voiture! Ah! c'est M. Morrel! courage, courage! sans doute qu'il nous apporte de bonnes nouvelles.»

Mercédès et le vieux père coururent au-devant de l'armateur, qu'ils rencontrèrent à la porte. M. Morrel était fort pâle.

«Eh bien? s'écrièrent-ils d'une même voix.

—Eh bien, mes amis! répondit l'armateur en secouant la tête, la chose est plus grave que nous ne le pensions.

—Oh! monsieur, s'écria Mercédès, il est innocent!

—Je le crois, répondit M. Morrel, mais on l'accuse\dots.

—De quoi donc? demanda le vieux Dantès.

—D'être un agent bonapartiste.»

Ceux de mes lecteurs qui ont vécu dans l'époque où se passe cette histoire se rappelleront quelle terrible accusation c'était alors, que celle que venait de formuler M. Morrel. Mercédès poussa un cri; le vieillard se laissa tomber sur une chaise.

«Ah! murmura Caderousse, vous m'avez trompé, Danglars, et la plaisanterie a été faite; mais je ne veux pas laisser mourir de douleur ce vieillard et cette jeune fille, et je vais tout leur dire.

—Tais-toi, malheureux! s'écria Danglars en saisissant la main de Caderousse, ou je ne réponds pas de toi-même; qui te dit que Dantès n'est pas véritablement coupable? Le bâtiment a touché à l'île d'Elbe, il y est descendu, il est resté tout un jour à Porto-Ferrajo; si l'on trouvait sur lui quelque lettre qui le compromette, ceux qui l'auraient soutenu passeraient pour ses complices.»

Caderousse, avec l'instinct rapide de l'égoïsme, comprit toute la solidité de ce raisonnement; il regarda Danglars avec des yeux hébétés par la crainte et la douleur, et, pour un pas qu'il avait fait en avant, il en fit deux en arrière.

«Attendons, alors, murmura-t-il.

—Oui, attendons, dit Danglars; s'il est innocent, on le mettra en liberté; s'il est coupable, il est inutile de se compromettre pour un conspirateur.

—Alors, partons, je ne puis rester plus longtemps ici.

—Oui, viens, dit Danglars enchanté de trouver un compagnon de retraite, viens, et laissons-les se retirer de là comme ils pourront.»

Ils partirent: Fernand, redevenu l'appui de la jeune fille, prit Mercédès par la main et la ramena aux Catalans. Les amis de Dantès ramenèrent, de leur côté, aux allées de Meilhan, ce vieillard presque évanoui.

Bientôt cette rumeur, que Dantès venait d'être arrêté comme agent bonapartiste, se répandit par toute la ville.

«Eussiez-vous cru cela, mon cher Danglars? dit M. Morrel en rejoignant son agent comptable et Caderousse, car il regagnait lui-même la ville en toute hâte pour avoir quelque nouvelle directe d'Edmond par le substitut du procureur du roi, M. de Villefort, qu'il connaissait un peu; auriez-vous cru cela?

—Dame, monsieur! répondit Danglars, je vous avais dit que Dantès, sans aucun motif, avait relâché à l'île d'Elbe, et cette relâche, vous le savez, m'avait paru suspecte.

—Mais aviez-vous fait part de vos soupçons à d'autres qu'à moi?

—Je m'en serais bien gardé, monsieur, ajouta tout bas Danglars; vous savez bien qu'à cause de votre oncle, M. Policar Morrel, qui a servi sous l'autre et qui ne cache pas sa pensée, on vous soupçonne de regretter Napoléon; j'aurais eu peur de faire tort à Edmond et ensuite à vous; il y a de ces choses qu'il est du devoir d'un subordonné de dire à son armateur et de cacher sévèrement aux autres.

—Bien, Danglars, bien, dit l'armateur, vous êtes un brave garçon; aussi j'avais d'avance pensé à vous, dans le cas où ce pauvre Dantès fût devenu le capitaine du \textit{Pharaon}.

—Comment cela, monsieur?

—Oui, j'avais d'avance demandé à Dantès ce qu'il pensait de vous, et s'il aurait quelque répugnance à vous garder à votre poste; car, je ne sais pourquoi, j'avais cru remarquer qu'il y avait du froid entre vous.

—Et que vous a-t-il répondu?

—Qu'il croyait effectivement avoir eu dans une circonstance qu'il ne m'a pas dite, quelques torts envers vous, mais que toute personne qui avait la confiance de l'armateur avait la sienne.

—L'hypocrite! murmura Danglars.

—Pauvre Dantès! dit Caderousse, c'est un fait qu'il était excellent garçon.

—Oui, mais en attendant, dit M. Morrel, voilà le \textit{Pharaon} sans capitaine.

—Oh! dit Danglars, il faut espérer, puisque nous ne pouvons repartir que dans trois mois, que d'ici à cette époque Dantès sera mis en liberté.

—Sans doute, mais jusque-là?

—Eh bien, jusque-là me voici, monsieur Morrel, dit Danglars; vous savez que je connais le maniement d'un navire aussi bien que le premier capitaine au long cours venu, cela vous offrira même un avantage, de vous servir de moi, car lorsque Edmond sortira de prison, vous n'aurez personne à remercier: il reprendra sa place et moi la mienne, voilà tout.

—Merci, Danglars, dit l'armateur; voilà en effet qui concilie tout. Prenez donc le commandement, je vous y autorise, et surveillez le débarquement: il ne faut jamais, quelque catastrophe qui arrive aux individus, que les affaires souffrent.

—Soyez tranquille, monsieur; mais pourra-t-on le voir au moins, ce bon Edmond?

—Je vous dirai cela tout à l'heure, Danglars; je vais tâcher de parler à M. de Villefort et d'intercéder près de lui en faveur du prisonnier. Je sais bien que c'est un royaliste enragé, mais, que diable! tout royaliste et procureur du roi qu'il est, il est un homme aussi, et je ne le crois pas méchant.

—Non, dit Danglars, mais j'ai entendu dire qu'il était ambitieux, et cela se ressemble beaucoup.

—Enfin, dit M. Morrel avec un soupir, nous verrons; allez à bord, je vous y rejoins.»

Et il quitta les deux amis pour prendre le chemin du palais de justice.

«Tu vois, dit Danglars à Caderousse, la tournure que prend l'affaire. As-tu encore envie d'aller soutenir Dantès maintenant?

—Non, sans doute; mais c'est cependant une terrible chose qu'une plaisanterie qui a de pareilles suites.

—Dame! qui l'a faite? ce n'est ni toi ni moi, n'est-ce pas? c'est Fernand. Tu sais bien que quant à moi j'ai jeté le papier dans un coin: je croyais même l'avoir déchiré.

—Non, non, dit Caderousse. Oh! quant à cela, j'en suis sûr; je le vois au coin de la tonnelle, tout froissé, tout roulé, et je voudrais même bien qu'il fût encore où je le vois!

—Que veux-tu? Fernand l'aura ramassé, Fernand l'aura copié ou fait copier, Fernand n'aura peut-être même pas pris cette peine; et, j'y pense\dots mon Dieu! il aura peut-être envoyé ma propre lettre! Heureusement que j'avais déguisé mon écriture.

—Mais tu savais donc que Dantès conspirait?

—Moi, je ne savais rien au monde. Comme je l'ai dit, j'ai cru faire une plaisanterie, pas autre chose. Il paraît que, comme Arlequin, j'ai dit la vérité en riant.

—C'est égal, reprit Caderousse, je donnerais bien des choses pour que toute cette affaire ne fût pas arrivée, ou du moins pour n'y être mêlé en rien. Tu verras qu'elle nous portera malheur, Danglars!

—Si elle doit porter malheur à quelqu'un, c'est au vrai coupable, et le vrai coupable c'est Fernand et non pas nous. Quel malheur veux-tu qu'il nous arrive à nous? Nous n'avons qu'à nous tenir tranquilles, sans souffler le mot de tout cela, et l'orage passera sans que le tonnerre tombe.

—Amen! dit Caderousse en faisant un signe d'adieu à Danglars et en se dirigeant vers les allées de Meilhan, tout en secouant la tête et en se parlant à lui-même, comme ont l'habitude de faire les gens fort préoccupés.

—Bon! dit Danglars, les choses prennent la tournure que j'avais prévue: me voilà capitaine par intérim, et si cet imbécile de Caderousse peut se taire, capitaine tout de bon. Il n'y a donc que le cas où la justice relâcherait Dantès? Oh! mais, ajouta-t-il avec un sourire, la justice est la justice, et je m'en rapporte à elle.»

Et sur ce, il sauta dans une barque en donnant l'ordre au batelier de le conduire à bord du \textit{Pharaon}, où l'armateur, on se le rappelle, lui avait donné rendez-vous.



