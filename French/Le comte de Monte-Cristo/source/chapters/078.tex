\chapter{On nous écrit de Janina}

\lettrine{F}{ranz} était sorti de la chambre de Noirtier si chancelant et si égaré, que Valentine elle-même avait eu pitié de lui. 

\zz
Villefort, qui n'avait articulé que quelques mots sans suite, et qui s'était enfui dans son cabinet, reçut, deux heures après, la lettre suivante: 

«Après ce qui a été révélé ce matin, M. Noirtier de Villefort ne peut supposer qu'une alliance soit possible entre sa famille et celle de M. Franz d'Épinay. M. Franz d'Épinay a horreur de songer que M. de Villefort, qui paraissait connaître les événements racontés ce matin, ne l'ait pas prévenu dans cette pensée.» 

Quiconque eût vu en ce moment le magistrat ployé sous le coup n'eût pas cru qu'il le prévoyait; en effet, jamais il n'eût pensé que son père eût poussé la franchise, ou plutôt la rudesse, jusqu'à raconter une pareille histoire. Il est vrai que jamais M. Noirtier, assez dédaigneux qu'il était de l'opinion de son fils, ne s'était préoccupé d'éclaircir le fait aux yeux de Villefort, et que celui-ci avait toujours cru que le général de Quesnel, ou le baron d'Épinay, selon qu'on voudra l'appeler, ou du nom qu'il s'était fait, ou du nom qu'on lui avait fait, était mort assassiné et non tué loyalement en duel. 

Cette lettre si dure d'un jeune homme si respectueux jusqu'alors était mortelle pour l'orgueil d'un homme comme Villefort. 

À peine était-il dans son cabinet que sa femme entra. 

La sortie de Franz, appelé par M. Noirtier, avait tellement étonné tout le monde que la position de Mme de Villefort, restée seule avec le notaire et les témoins, devint de moment en moment plus embarrassante. Alors Mme de Villefort avait pris son parti, et elle était sortie en annonçant qu'elle allait aux nouvelles. 

M. de Villefort se contenta de lui dire qu'à la suite d'une explication entre lui, M. Noirtier et M. d'Épinay, le mariage de Valentine avec Franz était rompu. 

C'était difficile à rapporter à ceux qui attendaient; aussi Mme de Villefort, en rentrant, se contenta-t-elle de dire que M. Noirtier, ayant eu, au commencement de la conférence, une espèce d'attaque d'apoplexie, le contrat était naturellement remis à quelques jours. 

Cette nouvelle, toute fausse qu'elle était, arrivait si singulièrement à la suite de deux malheurs du même genre, que les auditeurs se regardèrent étonnés et se retirèrent sans dire une parole. 

Pendant ce temps, Valentine, heureuse et épouvantée à la fois, après avoir embrassé et remercié le faible vieillard, qui venait de briser ainsi d'un seul coup une chaîne qu'elle regardait déjà comme indissoluble, avait demandé à se retirer chez elle pour se remettre et Noirtier lui avait, de l'œil, accordé la permission qu'elle sollicitait. 

Mais, au lieu de remonter chez elle, Valentine, une fois sortie, prit le corridor, et, sortant par la petite porte, s'élança dans le jardin. Au milieu de tous les événements qui venaient de s'entasser les uns sur les autres, une terreur sourde avait constamment comprimé son cœur. Elle s'attendait d'un moment à l'autre à voir apparaître Morrel pâle et menaçant comme le laird de Ravenswood au contrat de Lucie de Lammermoor. 

En effet, il était temps qu'elle arrivât à la grille. Maximilien, qui s'était douté de ce qui allait se passer en voyant Franz quitter le cimetière avec M. de Villefort, l'avait suivi; puis, après l'avoir vu entrer, l'avait vu sortir encore et rentrer de nouveau avec Albert et Château-Renaud. Pour lui, il n'y avait donc plus de doute. Il s'était alors jeté dans son enclos, prêt à tout événement, et bien certain qu'au premier moment de liberté qu'elle pourrait saisir, Valentine accourrait à lui. 

Il ne s'était point trompé; son œil, collé aux planches, vit en effet apparaître la jeune fille, qui, sans prendre aucune précaution d'usage, accourait à la grille. Au premier coup d'œil qu'il jeta sur elle, Maximilien fut rassuré; au premier mot qu'elle prononça il bondit de joie. 

«Sauvés! dit Valentine. 

—Sauvés! répéta Morrel, ne pouvant croire à un pareil bonheur: mais par qui sauvés? 

—Par mon grand-père. Oh! aimez-le bien, Morrel.» 

Morrel jura d'aimer le vieillard de toute son âme, et ce serment ne lui coûtait point à faire, car, dans ce moment, il ne se contentait pas de l'aimer comme un ami ou comme un père, il l'adorait comme un dieu. 

«Mais comment cela s'est-il fait? demanda Morrel; quel moyen étrange a-t-il employé?» 

Valentine ouvrait la bouche pour tout raconter; mais elle songea qu'il y avait au fond de tout cela un secret terrible qui n'était point à son grand-père seulement. 

«Plus tard, dit-elle, je vous raconterai tout cela.  

—Mais quand? 

—Quand je serai votre femme.» 

C'était mettre la conversation sur un chapitre qui rendait Morrel facile à tout entendre: aussi il entendit même qu'il devait se contenter de ce qu'il savait, et que c'était assez pour un jour. Cependant il ne consentit à se retirer que sur la promesse qu'il verrait Valentine le lendemain soir. 

Valentine promit ce que voulut Morrel. Tout était changé à ses yeux, et certes il lui était moins difficile de croire maintenant qu'elle épouserait Maximilien, que de croire une heure auparavant qu'elle n'épouserait pas Franz. 

Pendant ce temps, Mme de Villefort était montée chez Noirtier. 

Noirtier la regarda de cet œil sombre et sévère avec lequel il avait coutume de la recevoir. 

«Monsieur, lui dit-elle, je n'ai pas besoin de vous apprendre que le mariage de Valentine est rompu, puisque c'est ici que cette rupture a eu lieu.» 

Noirtier resta impassible. 

«Mais, continua Mme de Villefort, ce que vous ne savez pas, monsieur, c'est que j'ai toujours été opposée à ce mariage, qui se faisait malgré moi.» 

Noirtier regarda sa belle-fille en homme qui attend une explication. 

«Or, maintenant que ce mariage, pour lequel je connaissais votre répugnance, est rompu, je viens faire près de vous une démarche que ni M. de Villefort ni Valentine ne peuvent faire.» 

Les yeux de Noirtier demandèrent quelle était cette démarche. 

«Je viens vous prier, monsieur, continua Mme de Villefort, comme la seule qui en ait le droit, car je suis la seule à qui il n'en reviendra rien; je viens vous prier de rendre, je ne dirai pas vos bonnes grâces, elle les a toujours eues, mais votre fortune, à votre petite-fille.» 

Les yeux de Noirtier demeurèrent un instant incertains: il cherchait évidemment les motifs de cette démarche et ne les pouvait trouver. 

«Puis-je espérer, monsieur, dit Mme de Villefort que vos intentions étaient en harmonie avec la prière que je venais vous faire? 

—Oui, fit Noirtier. 

—En ce cas, monsieur, dit Mme de Villefort, je me retire à la fois reconnaissante et heureuse.» 

Et saluant M. Noirtier, elle se retira. 

En effet, dès le lendemain, Noirtier fit venir le notaire: le premier testament fut déchiré, et un nouveau fut fait, dans lequel il laissa toute sa fortune à Valentine, à la condition qu'on ne la séparerait pas de lui. 

Quelques personnes alors calculèrent de par le monde que Mlle de Villefort, héritière du marquis et de la marquise de Saint-Méran, et rentrée en la grâce de son grand-père, aurait un jour bien près de trois cent mille livres de rente. 

Tandis que ce mariage se rompait chez les Villefort, M. le comte de Morcerf avait reçu la visite de Monte-Cristo, et, pour montrer son empressement à Danglars, il endossait son grand uniforme de lieutenant général, qu'il avait fait orner de toutes ses croix, et demandait ses meilleurs chevaux. Ainsi paré, il se rendit rue de la Chaussée-d'Antin, et se fit annoncer à Danglars, qui faisait son relevé de fin de mois. 

Ce n'était pas le moment où, depuis quelque temps il fallait prendre le banquier pour le trouver de bonne humeur. 

Aussi, à l'aspect de son ancien ami, Danglars prit son air majestueux et s'établit carrément dans son fauteuil. 

Morcerf, si empesé d'habitude, avait emprunté au contraire un air riant et affable; en conséquence, à peu près sûr qu'il était que son ouverture allait recevoir un bon accueil, il ne fit point de diplomatie, et arrivant au but d'un seul coup: 

«Baron, dit-il, me voici. Depuis longtemps nous tournons autour de nos paroles d'autrefois\dots.» 

Morcerf s'attendait, à ces mots, à voir s'épanouir la figure du banquier, dont il attribuait le rembrunissement à son silence; mais, au contraire, cette figure devint, ce qui était presque incroyable, plus impassible et plus froide encore. 

Voilà pourquoi Morcerf s'était arrêté au milieu de sa phrase. 

«Quelles paroles, monsieur le comte? demanda le banquier, comme s'il cherchait vainement dans son esprit l'explication de ce que le général voulait dire. 

—Oh! dit le comte, vous êtes formaliste, mon cher monsieur, et vous me rappelez que le cérémonial doit se faire selon tous les rites. Très bien! ma foi. Pardonnez-moi, comme je n'ai qu'un fils, et que c'est la première fois que je songe à le marier, j'en suis encore à mon apprentissage: allons, je m'exécute.» 

Et Morcerf, avec un sourire forcé, se leva, fit une profonde révérence à Danglars, et lui dit: 

«Monsieur le baron, j'ai l'honneur de vous demander la main de Mlle Eugénie Danglars, votre fille, pour mon fils le vicomte Albert de Morcerf.» 

Mais Danglars, au lieu d'accueillir ces paroles avec une faveur que Morcerf pouvait espérer de lui, fronça le sourcil, et, sans inviter le comte, qui était resté debout, à s'asseoir: 

«Monsieur le comte, dit-il, avant de vous répondre, j'aurai besoin de réfléchir. 

—De réfléchir! reprit Morcerf de plus en plus étonné, n'avez-vous pas eu le temps de réfléchir depuis tantôt huit ans que nous causâmes de ce mariage pour la première fois? 

—Monsieur le comte, dit Danglars, tous les jours il arrive des choses qui font que les réflexions que l'on croyait faites sont à refaire. 

—Comment cela? demanda Morcerf; je ne vous comprends plus, baron! 

—Je veux dire, monsieur, que depuis quinze jours de nouvelles circonstances\dots. 

—Permettez, dit Morcerf; est-ce ou n'est-ce pas une comédie que nous jouons? 

—Comment cela, une comédie? 

—Oui, expliquons-nous catégoriquement. 

—Je ne demande pas mieux. 

—Vous avez vu M. de Monte-Cristo! 

—Je le vois très souvent, dit Danglars en secouant son jabot, c'est un de mes amis. 

—Eh bien, une des dernières fois que vous l'avez vu, vous lui avez dit que je semblais oublieux, irrésolu, à l'endroit de ce mariage. 

—C'est vrai. 

—Eh bien, me voici. Je ne suis ni oublieux ni irrésolu, vous le voyez, puisque je viens vous sommer de tenir votre promesse.»  

Danglars ne répondit pas. 

«Avez-vous si tôt changé d'avis, ajouta Morcerf, ou n'avez-vous provoqué ma demande que pour vous donner le plaisir de m'humilier?» 

Danglars comprit que, s'il continuait la conversation sur le ton qu'il l'avait entreprise, la chose pourrait mal tourner pour lui. 

«Monsieur le comte, dit-il, vous devez être à bon droit surpris de ma réserve, je comprends cela: aussi, croyez bien que moi, tout le premier, je m'en afflige; croyez bien qu'elle m'est commandée par des circonstances impérieuses. 

—Ce sont là des propos en l'air, mon cher monsieur, dit le comte, et dont pourrait peut-être se contenter le premier venu; mais le comte de Morcerf n'est pas le premier venu; et quand un homme comme lui vient trouver un autre homme, lui rappelle la parole donnée, et que cet homme manque à sa parole, il a le droit d'exiger en place qu'on lui donne au moins une bonne raison.» 

Danglars était lâche, mais il ne le voulait point paraître: il fut piqué du ton que Morcerf venait de prendre. 

«Aussi n'est-ce pas la bonne raison qui me manque, répliqua-t-il. 

—Que prétendez-vous dire? 

—Que la bonne raison, je l'ai, mais qu'elle est difficile à donner. 

—Vous sentez cependant, dit Morcerf, que je ne puis me payer de vos réticences; et une chose, en tout cas, me paraît claire, c'est que vous refusez mon alliance. 

—Non, monsieur, dit Danglars, je suspends ma résolution, voilà tout. 

—Mais vous n'avez cependant pas la prétention, je le suppose, de croire que je souscrive à vos caprices, au point d'attendre tranquillement et humblement le retour de vos bonnes grâces? 

—Alors, monsieur le comte, si vous ne pouvez attendre, regardons nos projets comme non avenus.» 

Le comte se mordit les lèvres jusqu'au sang pour ne pas faire l'éclat que son caractère superbe et irritable le portait à faire; cependant, comprenant qu'en pareille circonstance le ridicule serait de son côté, il avait déjà commencé à gagner la porte du salon, lorsque, se ravisant, il revint sur ses pas. 

Un nuage venait de passer sur son front, y laissant, au lieu de l'orgueil offensé, la trace d'une vague inquiétude. 

«Voyons, dit-il, mon cher Danglars, nous nous connaissons depuis de longues années, et, par conséquent, nous devons avoir quelques ménagements l'un pour l'autre. Vous me devez une explication, et c'est bien le moins que je sache à quel malheureux événement mon fils doit la perte de vos bonnes intentions à son égard. 

—Ce n'est point personnel au vicomte, voilà tout ce que je puis vous dire, monsieur, répondit Danglars, qui redevenait impertinent en voyant que Morcerf s'adoucissait.  

—Et à qui donc est-ce personnel?» demanda d'une voix altérée Morcerf, dont le front se couvrit de pâleur. 

Danglars, à qui aucun de ces symptômes n'échappait, fixa sur lui un regard plus assuré qu'il n'avait coutume de le faire. 

«Remerciez-moi de ne pas m'expliquer davantage», dit-il. 

Un tremblement nerveux, qui venait sans doute d'une colère contenue, agitait Morcerf. 

«J'ai le droit, répondit-il en faisant un violent effort sur lui-même, j'ai le projet d'exiger que vous vous expliquiez; est-ce donc contre Mme de Morcerf que vous avez quelque chose? Est-ce ma fortune qui n'est pas suffisante? Sont-ce mes opinions qui, étant contraires aux vôtres\dots. 

—Rien de tout cela, monsieur, dit Danglars; je serais impardonnable, car je me suis engagé connaissant tout cela. Non, ne cherchez plus, je suis vraiment honteux de vous faire faire cet examen de conscience; restons-en là, croyez-moi. Prenons le terme moyen du délai, qui n'est ni une rupture, ni un engagement. Rien ne presse, mon Dieu! Ma fille a dix-sept ans, et votre fils vingt et un. Pendant notre halte, le temps marchera, lui; il amènera les événements; les choses qui paraissent obscures la veille sont parfois trop claires le lendemain; parfois ainsi, en un jour, tombent les plus cruelles calomnies. 

—Des calomnies, avez-vous dit, monsieur! s'écria Morcerf en devenant livide. On me calomnie, moi! 

—Monsieur le comte, ne nous expliquons pas, vous dis-je.  

—Ainsi, monsieur, il me faudra subir tranquillement ce refus? 

—Pénible surtout pour moi, monsieur. Oui, plus pénible pour moi que pour vous, car je comptais sur l'honneur de votre alliance, et un mariage manqué fait toujours plus de tort à la fiancée qu'au fiancé. 

—C'est bien, monsieur, n'en parlons plus», dit Morcerf. 

Et froissant ses gants avec rage, il sortit de l'appartement. 

Danglars remarqua que, pas une seule fois, Morcerf n'avait osé demander si c'était à cause de lui, Morcerf, que Danglars retirait sa parole. 

Le soir il eut une longue conférence avec plusieurs amis, et M. Cavalcanti, qui s'était constamment tenu dans le salon des dames, sortit le dernier de la maison du banquier. 

Le lendemain, en se réveillant, Danglars demanda les journaux, on les lui apporta aussitôt: il en écarta trois ou quatre et prit \textit{l'Impartial}. 

C'était celui dont Beauchamp était le rédacteur-gérant. 

Il brisa rapidement l'enveloppe, l'ouvrit avec une précipitation nerveuse, passa dédaigneusement sur le \textit{Premier Paris}, et, arrivant aux faits divers, s'arrêta avec son méchant sourire sur un entrefilet commençant par ces mots: \textit{On nous écrit de Janina}. 

«Bon, dit-il après avoir lu, voici un petit bout d'article sur le colonel Fernand qui, selon toute probabilité, me dispensera de donner des explications à M. le comte de Morcerf.» 

Au même moment, c'est-à-dire comme neuf heures du matin sonnaient, Albert de Morcerf, vêtu de noir, boutonné méthodiquement, la démarche agitée et la parole brève, se présentait à la maison des Champs-Élysées. 

«M. le comte vient de sortir il y a une demi-heure à peu près, dit le concierge. 

—A-t-il emmené Baptistin? demanda Morcerf. 

—Non, monsieur le vicomte. 

—Appelez Baptistin, je veux lui parler.» 

Le concierge alla chercher le valet de chambre lui-même, et un instant après revint avec lui. 

«Mon ami, dit Albert, je vous demande pardon de mon indiscrétion, mais j'ai voulu vous demander à vous-même si votre maître était bien réellement sorti? 

—Oui, monsieur, répondit Baptistin. 

—Même pour moi? 

—Je sais combien mon maître est heureux de recevoir monsieur, et je me garderais bien de confondre monsieur dans une mesure générale.  

—Tu as raison, car j'ai à lui parler d'une affaire sérieuse. Crois-tu qu'il tardera à rentrer? 

—Non, car il a commandé son déjeuner pour dix heures. 

—Bien, je vais faire un tour aux Champs-Élysées, à dix heures je serai ici; si M. le comte rentre avant moi, dis-lui que je le prie d'attendre. 

—Je n'y manquerai pas, monsieur peut en être sûr.» 

Albert laissa à la porte du comte le cabriolet de place qu'il avait pris et alla se promener à pied. 

En passant devant l'allée des Veuves, il crut reconnaître les chevaux du comte qui stationnaient à la porte du tir de Gosset; il s'approcha et, après avoir reconnu les chevaux, reconnut le cocher. 

«M. le comte est au tir? demanda Morcerf à celui-ci. 

—Oui, monsieur», répondit le cocher. 

En effet, plusieurs coups réguliers s'étaient fait entendre depuis que Morcerf était aux environs du tir. 

Il entra. 

Dans le petit jardin se tenait le garçon. 

«Pardon, dit-il, mais monsieur le vicomte voudrait-il attendre un instant? 

—Pourquoi cela, Philippe? demanda Albert, qui, étant un habitué, s'étonnait de cet obstacle qu'il ne comprenait pas. 

—Parce que la personne qui s'exerce en ce moment prend le tir à elle seule, et ne tire jamais devant quelqu'un. 

—Pas même devant vous, Philippe? 

—Vous voyez, monsieur, je suis à la porte de ma loge. 

—Et qui lui charge ses pistolets? 

—Son domestique. 

—Un Nubien? 

—Un nègre. 

—C'est cela. 

—Vous connaissez donc ce seigneur? 

—Je viens le chercher; c'est mon ami. 

—Oh! alors, c'est autre chose. Je vais entrer pour le prévenir.» 

Et Philippe, poussé par sa propre curiosité, entra dans la cabane de planches. Une seconde après, Monte-Cristo parut sur le seuil. 

«Pardon de vous poursuivre jusqu'ici, mon cher comte, dit Albert; mais je commence par vous dire que ce n'est point la faute de vos gens, et que moi seul suis indiscret. Je me suis présenté chez vous; on m'a dit que vous étiez en promenade, mais que vous rentreriez à dix heures pour déjeuner. Je me suis promené à mon tour en attendant dix heures, et, en me promenant, j'ai aperçu vos chevaux et votre voiture. 

—Ce que vous me dites là me donne l'espoir que vous venez me demander à déjeuner. 

—Non pas, merci, il ne s'agit pas de déjeuner à cette heure; peut-être déjeunerons-nous plus tard, mais en mauvaise compagnie, pardieu! 

—Que diable contez-vous là? 

—Mon cher, je me bats aujourd'hui. 

—Vous? et pour quoi faire? 

—Pour me battre, pardieu! 

—Oui, j'entends bien, mais à cause de quoi? On se bat pour toute espèce de choses, vous comprenez bien. 

—À cause de l'honneur. 

—Ah! ceci, c'est sérieux. 

—Si sérieux, que je viens vous prier de me rendre un service. 

—Lequel? 

—Celui d'être mon témoin. 

—Alors cela devient grave; ne parlons de rien ici, et rentrons chez moi. Ali, donne-moi de l'eau.» 

Le comte retroussa ses manches et passa dans le petit vestibule qui précède les tirs, et où les tireurs ont l'habitude de se laver les mains. 

«Entrez donc, monsieur le vicomte, dit tout bas Philippe, vous verrez quelque chose de drôle.» 

Morcerf entra. Au lieu de mouches, des cartes à jouer étaient collées sur la plaque. 

De loin, Morcerf crut que c'était le jeu complet; il y avait depuis l'as jusqu'au dix. 

«Ah! ah! fit Albert, vous étiez en train de jouer au piquet? 

—Non, dit le comte, j'étais en train de faire un jeu de cartes. 

—Comment cela? 

—Oui, ce sont des as et des deux que vous voyez; seulement mes balles en ont fait des trois, des cinq, des sept, des huit, des neuf et des dix.» 

Albert s'approcha. 

En effet, les balles avaient, avec des lignes parfaitement exactes et des distances parfaitement égales, remplacé les signes absents et troué le carton aux endroits où il aurait dû être peint. En allant à la plaque, Morcerf ramassa, en outre, deux ou trois hirondelles qui avaient eu l'imprudence de passer à portée du pistolet du comte, et que le comte avait abattues. 

«Diable! fit Morcerf. 

—Que voulez-vous, mon cher vicomte, dit Monte-Cristo en s'essuyant les mains avec du linge apporté par Ali, il faut bien que j'occupe mes instants d'oisiveté, mais venez, je vous attends.» 

Tous deux montèrent dans le coupé de Monte-Cristo qui, au bout de quelques instants, les eut déposés à la porte du n°30. 

Monte-Cristo conduisit Morcerf dans son cabinet, et lui montra un siège. Tous deux s'assirent. 

«Maintenant, causons tranquillement, dit le comte. 

—Vous voyez que je suis parfaitement tranquille. 

—Avec qui voulez-vous vous battre? 

—Avec Beauchamp. 

—Un de vos amis! 

—C'est toujours avec des amis qu'on se bat. 

—Au moins faut-il une raison. 

—J'en ai une. 

—Que vous a-t-il fait? 

—Il y a, dans un journal d'hier soir\dots mais tenez, lisez. 

Albert tendit à Monte-Cristo un journal où il lut ces mots: 

«On nous écrit de Janina: 

«Un fait jusqu'alors ignoré, ou tout au moins inédit, est parvenu à notre connaissance; les châteaux qui défendaient la ville ont été livrés aux Turcs par un officier français dans lequel le vizir Ali-Tebelin avait mis toute sa confiance, et qui s'appelait Fernand.» 

«Eh bien, demanda Monte-Cristo, que voyez-vous là-dedans qui vous choque? 

—Comment! ce que je vois? 

—Oui. Que vous importe à vous que les châteaux de Janina aient été livrés par un officier nommé Fernand? 

—Il m'importe que mon père, le comte de Morcerf, s'appelle Fernand de son nom de baptême. 

—Et votre père servait Ali-Pacha? 

—C'est-à-dire qu'il combattait pour l'indépendance des Grecs; voilà où est la calomnie. 

—Ah çà! mon cher vicomte, parlons raison. 

—Je ne demande pas mieux. 

—Dites-moi un peu: qui diable sait en France que l'officier Fernand est le même homme que le comte de Morcerf et qui s'occupe à cette heure de Janina, qui a été prise en 1822 ou 1823, je crois? 

—Voilà justement où est la perfidie: on a laissé le temps passer là-dessus, puis aujourd'hui on revient sur des événements oubliés pour en faire sortir un scandale qui peut ternir une haute position. Eh bien, moi, héritier du nom de mon père, je ne veux pas même que sur ce nom flotte l'ombre d'un doute. Je vais envoyer à Beauchamp, dont le journal a publié cette note, deux témoins, et il la rétractera. 

—Beauchamp ne rétractera rien. 

—Alors, nous nous battrons. 

—Non, vous ne vous battrez pas, car il vous répondra qu'il y avait peut-être dans l'armée grecque cinquante officiers qui s'appelaient Fernand. 

—Nous nous battrons malgré cette réponse. Oh! je veux que cela disparaisse\dots. Mon père, un si noble soldat, une si illustre carrière\dots. 

—Ou bien il mettra: Nous sommes fondés à croire que ce Fernand n'a rien de commun avec M. le comte de Morcerf, dont le nom de baptême est aussi Fernand. 

—Il me faut une rétractation pleine et entière; je ne me contenterai point de celle-là! 

—Et vous allez lui envoyer vos témoins? 

—Oui. 

—Vous avez tort. 

—Cela veut dire que vous me refusez le service que je venais vous demander. 

—Ah! vous savez ma théorie à l'égard du duel; je vous ai fait ma profession de foi à Rome, vous vous la rappelez? 

—Cependant, mon cher comte, je vous ai trouvé ce matin, tout à l'heure, exerçant une occupation peu en harmonie avec cette théorie. 

—Parce que, mon cher ami, vous comprenez, il ne faut jamais être exclusif. Quand on vit avec des fous, il faut faire aussi son apprentissage d'insensé, d'un moment à l'autre quelque cerveau brûlé, qui n'aura pas plus de motif de me chercher querelle que vous n'en avez d'aller chercher querelle à Beauchamp, me viendra trouver pour la première niaiserie venue, ou m'enverra ses témoins, ou m'insultera dans un endroit public: eh bien, ce cerveau brûlé, il faudra bien que je le tue. 

—Vous admettez donc que, vous-même, vous vous battriez? 

—Pardieu! 

—Eh bien, alors, pourquoi voulez-vous que, moi, je ne me batte pas? 

—Je ne dis point que vous ne devez point vous battre; je dis seulement qu'un duel est une chose grave et à laquelle il faut réfléchir. 

—A-t-il réfléchi, lui, pour insulter mon père? 

—S'il n'a pas réfléchi, et qu'il vous l'avoue; il ne faut pas lui en vouloir. 

—Oh! mon cher comte, vous êtes beaucoup trop indulgent! 

—Et vous, beaucoup trop rigoureux. Voyons, je suppose\dots écoutez bien ceci: je suppose\dots. N'allez pas vous fâcher de ce que je vous dis! 

—J'écoute. 

—Je suppose que le fait rapporté soit vrai\dots. 

—Un fils ne doit pas admettre une pareille supposition sur l'honneur de son père. 

—Eh! mon Dieu! nous sommes dans une époque où l'on admet tant de choses! 

—C'est justement le vice de l'époque. 

—Avez-vous la prétention de le réformer? 

—Oui, à l'endroit de ce qui me regarde. 

—Mon Dieu! quel rigoriste vous faites, mon cher ami! 

—Je suis ainsi. 

—Êtes-vous inaccessible aux bons conseils? 

—Non, quand ils viennent d'un ami. 

—Me croyez-vous le vôtre? 

—Oui. 

—Eh bien, avant d'envoyer vos témoins à Beauchamp, informez-vous. 

—Auprès de qui? 

—Eh pardieu! auprès d'Haydée, par exemple. 

—Mêler une femme dans tout cela, que peut-elle y faire? 

—Vous déclarer que votre père n'est pour rien dans la défaite ou la mort du sien, par exemple, ou vous éclairer à ce sujet, si par hasard votre père avait eu le malheur\dots. 

—Je vous ai déjà dit, mon cher comte, que je ne pouvais admettre une pareille supposition. 

—Vous refusez donc ce moyen?  

—Je le refuse. 

—Absolument? 

—Absolument! 

—Alors, un dernier conseil. 

—Soit, mais le dernier. 

—Ne le voulez-vous point? 

—Au contraire, je vous le demande. 

—N'envoyez point de témoins à Beauchamp. 

—Comment? 

—Allez le trouver vous-même. 

—C'est contre toutes les habitudes. 

—Votre affaire est en dehors des affaires ordinaires. 

—Et pourquoi dois-je y aller moi-même, voyons? 

—Parce qu'ainsi l'affaire reste entre vous et Beauchamp. 

—Expliquez-vous. 

—Sans doute; si Beauchamp est disposé à se rétracter, il faut lui laisser le mérite de la bonne volonté: la rétraction n'en sera pas moins faite. S'il refuse, au contraire, il sera temps de mettre deux étrangers dans votre secret. 

—Ce ne seront pas deux étrangers, ce seront deux amis. 

—Les amis d'aujourd'hui sont les ennemis de demain. 

—Oh! par exemple! 

—Témoin Beauchamp. 

—Ainsi\dots. 

—Ainsi, je vous recommande la prudence. 

—Ainsi, vous croyez que je dois aller trouver Beauchamp moi-même? 

—Oui. 

—Seul? 

—Seul. Quand on veut obtenir quelque chose de l'amour-propre d'un homme, il faut sauver à l'amour-propre de cet homme jusqu'à l'apparence de la souffrance. 

—Je crois que vous avez raison. 

—Ah! c'est bien heureux! 

—J'irai seul.  

—Allez; mais vous feriez encore mieux de n'y point aller du tout. 

—C'est impossible. 

—Faites donc ainsi; ce sera toujours mieux ce que vous que vouliez faire. 

—Mais en ce cas, voyons, si malgré toutes mes précautions, tous mes procédés, si j'ai un duel, me servirez-vous de témoin? 

—Mon cher vicomte; dit Monte-Cristo avec une gravité suprême, vous avez dû voir qu'en temps et lieu j'étais tout à votre dévotion; mais le service que vous me demanderez là sort du cercle de ceux que je puis vous rendre. 

—Pourquoi cela? 

—Peut-être le saurez-vous un jour. 

—Mais en attendant? 

—Je demande votre indulgence pour mon secret. 

—C'est bien. Je prendrai Franz et Château-Renaud. 

—Prenez Franz et Château-Renaud, ce sera à merveille. 

—Mais enfin, si je me bats, vous me donnerez bien une petite leçon d'épée ou de pistolet? 

—Non, c'est encore une chose impossible. 

—Singulier homme que vous faites, allez! Alors vous ne voulez vous mêler de rien? 

—De rien absolument. 

—Alors n'en parlons plus. Adieu, comte. 

—Adieu, vicomte.» 

Morcerf prit son chapeau et sortit. 

À la porte, il retrouva son cabriolet, et, contenant du mieux qu'il put sa colère, il se fit conduire chez Beauchamp; Beauchamp était à son journal. 

Albert se fit conduire au journal. 

Beauchamp était dans un cabinet sombre et poudreux, comme sont de fondation les bureaux de journaux. 

On lui annonça Albert de Morcerf. Il fit répéter deux fois l'annonce; puis, mal convaincu encore, il cria: 

«Entrez!» 

Albert parut. Beauchamp poussa une exclamation en voyant son ami franchir les liasses de papier et fouler d'un pied mal exercé les journaux de toutes grandeurs qui jonchaient non point le parquet, mais le carreau rougi de son bureau. 

«Par ici, par ici, mon cher Albert, dit-il en tendant la main au jeune homme; qui diable vous amène? êtes-vous perdu comme le petit Poucet, ou venez-vous tout bonnement me demander à déjeuner? Tâchez de trouver une chaise; tenez, là-bas, près de ce géranium qui, seul ici, me rappelle qu'il y a au monde des feuilles qui ne sont pas des feuilles de papier. 

—Beauchamp; dit Albert, c'est de votre journal que je viens vous parler. 

—Vous, Morcerf? que désirez-vous? 

—Je désire une rectification. 

—Vous, une rectification? À propos de quoi, Albert? mais asseyez-vous donc! 

—Merci, répondit Albert pour la seconde fois, et avec un léger signe de tête. 

—Expliquez-vous. 

—Une rectification sur un fait qui porte atteinte à l'honneur d'un membre de ma famille. 

—Allons donc! dit Beauchamp, surpris. Quel fait? Cela ne se peut pas. 

—Le fait qu'on vous a écrit de Janina. 

—De Janina? 

—Oui, de Janina. En vérité vous avez l'air d'ignorer ce qui m'amène? 

—Sur mon honneur\dots Baptiste! un journal d'hier! cria Beauchamp. 

—C'est inutile, je vous apporte le mien.» 

Beauchamp lut en bredouillant: 

«On nous écrit de Janina, etc.» 

«Vous comprenez que le fait est grave, dit Morcerf, quand Beauchamp eut fini. 

—Cet officier est donc votre parent? demanda le journaliste. 

—Oui, dit Albert en rougissant. 

—Eh bien, que voulez-vous que je fasse pour vous être agréable? dit Beauchamp avec douceur. 

—Je voudrais, mon cher Beauchamp, que vous rétractassiez ce fait.» 

Beauchamp regarda Albert avec une attention qui annonçait assurément beaucoup de bienveillance. 

«Voyons, dit-il, cela va nous entraîner dans une longue causerie; car c'est toujours une chose grave qu'une rétractation. Asseyez-vous; je vais relire ces trois ou quatre lignes.» 

Albert s'assit, et Beauchamp relut les lignes incriminées par son ami avec plus d'attention que la première fois. 

«Eh bien, vous le voyez, dit Albert avec fermeté, avec rudesse même, on a insulté dans votre journal quelqu'un de ma famille, et je veux une rétractation. 

—Vous\dots voulez\dots. 

—Oui, je veux! 

—Permettez-moi de vous dire que vous n'êtes point parlementaire, mon cher vicomte. 

—Je ne veux point l'être, répliqua le jeune homme en se levant; je poursuis la rétractation d'un fait que vous avez énoncé hier, et je l'obtiendrai. Vous êtes assez mon ami, continua Albert les lèvres serrées, voyant que Beauchamp, de son côté, commençait à relever sa tête dédaigneuse; vous êtes assez mon ami et, comme tel, vous me connaissez assez, je l'espère pour comprendre ma ténacité en pareille circonstance. 

—Si je suis votre ami, Morcerf, vous finirez par me le faire oublier avec des mots pareils à ceux de tout à l'heure\dots. Mais voyons, ne nous fâchons pas, ou du moins, pas encore\dots. Vous êtes inquiet, irrité, piqué\dots. Voyons, quel est ce parent qu'on appelle Fernand? 

—C'est mon père, tout simplement, dit Albert; M. Fernand Mondego, comte de Morcerf, un vieux militaire qui a vu vingt champs de bataille, et dont on voudrait couvrir les nobles cicatrices avec la fange impure ramassée dans le ruisseau. 

—C'est votre père? dit Beauchamp: alors c'est autre chose; je conçois votre indignation, mon cher Albert\dots Relisons donc\dots.» 

Et il relut la note, en pesant cette fois sur chaque mot. 

«Mais où voyez-vous, demanda Beauchamp, que le Fernand du journal soit votre père? 

—Nulle part, je le sais bien; mais d'autres le verront. C'est pour cela que je veux que le fait soit démenti.» 

Aux mots \textit{je veux}, Beauchamp leva les yeux sur Morcerf, et les baissant presque aussitôt, il demeura un instant pensif. 

«Vous démentirez ce fait, n'est-ce pas, Beauchamp? répéta Morcerf avec une colère croissante, quoique toujours concentrée. 

—Oui, dit Beauchamp. 

—À la bonne heure! dit Albert. 

—Mais quand je me serai assuré que le fait est faux. 

—Comment! 

—Oui, la chose vaut la peine d'être éclaircie, et je l'éclaircirai. 

—Mais que voyez-vous donc à éclaircir dans tout cela, monsieur? dit Albert, hors de toute mesure. Si vous ne croyez pas que ce soit mon père, dites-le tout de suite; si vous croyez que ce soit lui, rendez-moi raison de cette opinion.»  

Beauchamp regarda Albert avec ce sourire qui lui était particulier, et qui savait prendre la nuance de toutes les passions. 

«Monsieur, reprit-il, puisque monsieur il y a, si c'est pour me demander raison que vous êtes venu, il fallait le faire d'abord et ne point venir me parler d'amitié et d'autres choses oiseuses comme celles que j'ai la patience d'entendre depuis une demi-heure. Est-ce bien sur ce terrain que nous allons marcher désormais, voyons! 

—Oui, si vous ne rétractez pas l'infâme calomnie! 

—Un moment! pas de menaces, s'il vous plaît, monsieur Albert Mondego, vicomte de Morcerf, je n'en souffre pas de mes ennemis, à plus forte raison de mes amis. Donc, vous voulez que je démente le fait sur le colonel Fernand, fait auquel je n'ai, sur mon honneur pris aucune part? 

—Oui, je le veux! dit Albert, dont la tête commençait à s'égarer. 

—Sans quoi, nous nous battrons? continua Beauchamp avec le même calme. 

—Oui! reprit Albert, en haussant la voix. 

—Eh bien, dit Beauchamp, voici ma réponse, mon cher monsieur: ce fait n'a pas été inséré par moi, je ne le connaissais pas; mais vous avez, par votre démarche, attiré mon attention sur ce fait, elle s'y cramponne; il subsistera donc jusqu'à ce qu'il soit démenti ou confirmé par qui de droit.  

—Monsieur, dit Albert en se levant, je vais donc avoir l'honneur de vous envoyer mes témoins, vous discuterez avec eux le lieu et les armes. 

—Parfaitement, mon cher monsieur. 

—Et ce soir, s'il vous plaît ou demain au plus tard, nous nous rencontrerons. 

—Non pas! non pas! Je serai sur le terrain quand il le faudra, et, à mon avis (j'ai le droit de le donner, puisque c'est moi qui reçois la provocation), et, à mon avis, dis-je, l'heure n'est pas encore venue. Je sais que vous tirez très bien l'épée, je la tire passablement; je sais que vous faites trois mouches sur six, c'est ma force à peu près; je sais qu'un duel entre nous sera un duel sérieux, parce que vous êtes brave et que\dots je le suis aussi. Je ne veux donc pas m'exposer à vous tuer ou à être tué moi-même par vous, sans cause. C'est moi qui vais à mon tour poser la question et ca-té-go-ri-que-ment. 

«Tenez-vous à cette rétractation au point de me tuer si je ne le fais pas, bien que je vous aie dit, bien que je vous répète, bien que je vous affirme sur l'honneur que je ne connaissais pas le fait; bien que je vous déclare enfin qu'il est impossible à tout autre qu'à un don Japhet comme vous de deviner M. le comte de Morcerf sous ce nom de Fernand? 

—J'y tiens absolument. 

—Eh bien, mon cher monsieur, je consens à me couper la gorge avec vous, mais je veux trois semaines; dans trois semaines vous me retrouverez pour vous dire: Oui, le fait est faux, je l'efface; ou bien: Oui, le fait est vrai, et je sors les épées du fourreau, ou les pistolets de la boîte, à votre choix. 

—Trois semaines! s'écria Albert; mais trois semaines, c'est trois siècles pendant lesquels je suis déshonoré! 

—Si vous étiez resté mon ami, je vous eusse dit: Patience, ami; vous vous êtes fait mon ennemi et je vous dis: Que m'importe, à moi, monsieur! 

—Eh bien, dans trois semaines, soit, dit Morcerf. Mais songez-y, dans trois semaines il n'y aura plus ni délai ni subterfuge qui puisse vous dispenser\dots. 

—Monsieur Albert de Morcerf, dit Beauchamp en se levant à son tour, je ne puis vous jeter par les fenêtres que dans trois semaines, c'est-à-dire dans vingt-quatre jours, et vous, vous n'avez le droit de me pourfendre qu'à cette époque. Nous sommes le 29 du mois d'août, donc au 21 du mois de septembre. Jusque-là, croyez-moi, et c'est un conseil de gentilhomme que je vous donne, épargnons-nous les aboiements de deux dogues enchaînés à distance.» 

Et Beauchamp, saluant gravement le jeune homme, lui tourna le dos et passa dans son imprimerie. 

Albert se vengea sur une pile de journaux qu'il dispersa en les cinglant à grands coups de badine, après quoi il partit, non sans s'être retourné deux ou trois fois vers la porte de l'imprimerie. 

Tandis qu'Albert fouettait le devant de son cabriolet après avoir fouetté les innocents papiers noircis qui n'en pouvaient mais de sa déconvenue, il aperçut en traversant le boulevard, Morrel qui, le nez au vent, l'œil éveillé et les bras dégagés, passait devant les bains Chinois, venant du côté de la porte Saint-Martin, et allant du côté de la Madeleine. 

«Ah! dit-il en soupirant, voilà un homme heureux!» 

Par hasard, Albert ne se trompait point. 