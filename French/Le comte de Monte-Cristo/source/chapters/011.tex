\chapter{L'Ogre de Corse}

\lettrine{L}{ouis} XVIII, à l'aspect de ce visage bouleversé, repoussa violemment la table devant laquelle il se trouvait.

\zz
«Qu'avez-vous donc, monsieur le baron? s'écria-t-il, vous paraissez tout bouleversé: ce trouble, cette hésitation, ont-ils rapport à ce que disait M. de Blacas, et à ce que vient de me confirmer M. de Villefort?»


De son côté, M. de Blacas s'approchait vivement du baron, mais la terreur du courtisan empêchait de triompher l'orgueil de l'homme d'État; en effet, en pareille circonstance, il était bien autrement avantageux pour lui d'être humilié par le préfet de police que de l'humilier sur un pareil sujet.

«Sire\dots balbutia le baron.

—Eh bien, voyons!» dit Louis XVIII.

Le ministre de la Police, cédant alors à un mouvement de désespoir, alla se précipiter aux pieds de Louis XVIII, qui recula d'un pas, en fronçant le sourcil.

«Parlerez-vous? dit-il.

—Oh! Sire, quel affreux malheur! suis-je assez à plaindre? je ne m'en consolerai jamais!

—Monsieur, dit Louis XVIII, je vous ordonne de parler.

—Eh bien, Sire, l'usurpateur a quitté l'île d'Elbe le 28 février et a débarqué le 1\ier{} mars.

—Où cela? demanda vivement le roi.

—En France, Sire, dans un petit port; près d'Antibes, au golfe Juan.

—L'usurpateur a débarqué en France, près d'Antibes, au golfe Juan, à deux cent cinquante lieues de Paris, le 1\ier{} mars, et vous apprenez cette nouvelle aujourd'hui seulement 3 mars!\dots Eh! monsieur, ce que vous me dites là est impossible: on vous aura fait un faux rapport, ou vous êtes fou.

—Hélas! Sire, ce n'est que trop vrai!»

Louis XVIII fit un geste indicible de colère et d'effroi, et se dressa tout debout, comme si un coup imprévu l'avait frappé en même temps au cœur et au visage.

«En France! s'écria-t-il, l'usurpateur en France! Mais on ne veillait donc pas sur cet homme? mais qui sait? on était donc d'accord avec lui?

—Oh! Sire, s'écria le duc de Blacas, ce n'est pas un homme comme M. Dandré que l'on peut accuser de trahison. Sire, nous étions tous aveugles, et le ministre de la Police a partagé l'aveuglement général; voilà tout.

—Mais\dots dit Villefort; puis s'arrêtant tout à coup: Ah! pardon, pardon, Sire, fit-il en s'inclinant, mon zèle m'emporte, que Votre Majesté daigne m'excuser.

—Parlez, monsieur, parlez hardiment, dit le roi; vous seul nous avez prévenu du mal, aidez-nous à y chercher le remède.

—Sire, dit Villefort, l'usurpateur est détesté dans le Midi; il me semble que s'il se hasarde dans le Midi, on peut facilement soulever contre lui la Provence et le Languedoc.

—Oui, sans doute, dit le ministre, mais il s'avance par Gap et Sisteron.

—Il s'avance, il s'avance, dit Louis XVIII; il marche donc sur Paris?»

Le ministre de la Police garda un silence qui équivalait au plus complet aveu.

«Et le Dauphiné, monsieur, demanda le roi à Villefort, croyez-vous qu'on puisse le soulever comme la Provence?

—Sire, je suis fâché de dire à Votre Majesté une vérité cruelle; mais l'esprit du Dauphiné est loin de valoir celui de la Provence et du Languedoc. Les montagnards sont bonapartistes, Sire.

—Allons, murmura Louis XVIII, il était bien renseigné. Et combien d'hommes a-t-il avec lui?

—Sire, je ne sais, dit le ministre de la Police.

—Comment, vous ne savez! Vous avez oublié de vous informer de cette circonstance? Il est vrai qu'elle est de peu d'importance, ajouta-t-il avec un sourire écrasant.

—Sire, je ne pouvais m'en informer; la dépêche portait simplement l'annonce du débarquement et de la route prise par l'usurpateur.

—Et comment donc vous est parvenue cette dépêche?» demanda le roi.

Le ministre baissa la tête, et une vive rougeur envahit son front.

«Par le télégraphe, Sire», balbutia-t-il.

Louis XVIII fait un pas en avant et croisa les bras comme eût fait Napoléon.

«Ainsi, dit-il, pâlissant de colère, sept armées coalisées auront renversé cet homme; un miracle du ciel m'aura replacé sur le trône de mes pères après vingt-cinq ans d'exil; j'aurai, pendant ces vingt-cinq ans étudié, sondé, analysé les hommes et les choses de cette France qui m'était promise, pour qu'arrivé au but de tous mes vœux, une force que je tenais entre mes mains éclate et me brise!

—Sire, c'est de la fatalité, murmura le ministre, sentant qu'un pareil poids, léger pour le destin, suffisait à écraser un homme.

—Mais ce que disaient de nous nos ennemis est donc vrai: Rien appris, rien oublié? Si j'étais trahi comme lui, encore, je me consolerais; mais être au milieu de gens élevés par moi aux dignités, qui devaient veiller sur moi plus précieusement que sur eux-mêmes, car ma fortune c'est la leur, avant moi ils n'étaient rien, après moi ils ne seront rien, et périr misérablement par incapacité, par ineptie! Ah! oui, monsieur, vous avez bien raison, c'est de la fatalité.»

Le ministre se tenait courbé sous cet effrayant anathème.

M. de Blacas essuyait son front couvert de sueur; Villefort souriait intérieurement, car il sentait grandir son importance.

«Tomber, continuait Louis XVIII, qui du premier coup d'œil avait sondé le précipice où penchait la monarchie, tomber et apprendre sa chute par le télégraphe! Oh! j'aimerais mieux monter sur l'échafaud de mon frère Louis XVI, que de descendre ainsi l'escalier des Tuileries, chassé par le ridicule\dots. Le ridicule, monsieur, vous ne savez pas ce que c'est, en France, et cependant vous devriez le savoir.

—Sire, Sire, murmura le ministre, par pitié!\dots

—Approchez, monsieur de Villefort, continua le roi s'adressant au jeune homme, qui, debout, immobile et en arrière, considérait la marche de cette conversation où flottait éperdu le destin d'un royaume, approchez et dites à monsieur qu'on pouvait savoir d'avance tout ce qu'il n'a pas su.

—Sire, il était matériellement impossible de deviner les projets que cet homme cachait à tout le monde.

—Matériellement impossible! oui, voilà un grand mot, monsieur; malheureusement, il en est des grands mots comme des grands hommes, je les ai mesurés. Matériellement impossible à un ministre, qui a une administration, des bureaux, des agents, des mouchards, des espions et quinze cent mille francs de fonds secrets, de savoir ce qui se passe à soixante lieues des côtes de France! Eh bien, tenez, voici monsieur, qui n'avait aucune de ces ressources à sa disposition, voici monsieur, simple magistrat, qui en savait plus que vous avec toute votre police, et qui eût sauvé ma couronne s'il eût eu comme vous le droit de diriger un télégraphe.»

Le regard du ministre de la Police se tourna avec une expression de profond dépit sur Villefort, qui inclina la tête avec la modestie du triomphe.

«Je ne dis pas cela pour vous, Blacas, continua Louis XVIII, car si vous n'avez rien découvert, vous, au moins avez-vous eu le bon esprit de persévérer dans votre soupçon: un autre que vous eût peut-être considéré la révélation de M. de Villefort comme insignifiante, ou bien encore suggérée par une ambition vénale.»

Ces mots faisaient allusion à ceux que le ministre de la Police avait prononcés avec tant de confiance une heure auparavant.

Villefort comprit le jeu du roi. Un autre peut-être se serait laissé emporter par l'ivresse de la louange; mais il craignit de se faire un ennemi mortel du ministre de la Police, bien qu'il sentît que celui-ci était irrévocablement perdu. En effet, le ministre qui n'avait pas, dans la plénitude de sa puissance, su deviner le secret de Napoléon, pouvait, dans les convulsions de son agonie, pénétrer celui de Villefort: il ne lui fallait, pour cela, qu'interroger Dantès. Il vint donc en aide au ministre au lieu de l'accabler.

«Sire, dit Villefort, la rapidité de l'événement doit prouver à Votre Majesté que Dieu seul pouvait l'empêcher en soulevant une tempête; ce que Votre Majesté croit de ma part l'effet d'une profonde perspicacité est dû, purement et simplement, au hasard; j'ai profité de ce hasard en serviteur dévoué, voilà tout. Ne m'accordez pas plus que je ne mérite, Sire, pour ne revenir jamais sur la première idée que vous aurez conçue de moi.»

Le ministre de la Police remercia le jeune homme par un regard éloquent, et Villefort comprit qu'il avait réussi dans son projet, c'est-à-dire que, sans rien perdre de la reconnaissance du roi, il venait de se faire un ami sur lequel, le cas échéant, il pouvait compter.

«C'est bien, dit le roi. Et maintenant, messieurs, continua-t-il en se retournant vers M. de Blacas et vers le ministre de la Police, je n'ai plus besoin de vous, et vous pouvez vous retirer: ce qui reste à faire est du ressort du ministre de la Guerre.

—Heureusement, Sire, dit M. de Blacas, que nous pouvons compter sur l'armée. Votre Majesté sait combien tous les rapports nous la peignent dévouée à votre gouvernement.

—Ne me parlez pas de rapports: maintenant, duc, je sais la confiance que l'on peut avoir en eux. Eh! mais, à propos de rapports, monsieur le baron, qu'avez-vous appris de nouveau sur l'affaire de la rue Saint-Jacques?

—Sur l'affaire de la rue Saint-Jacques!» s'écria Villefort, ne pouvant retenir une exclamation.

Mais s'arrêtant tout à coup:

«Pardon, Sire, dit-il, mon dévouement à Votre Majesté me fait sans cesse oublier, non le respect que j'ai pour elle, ce respect est trop profondément gravé dans mon cœur, mais les règles de l'étiquette.

—Dites et faites, monsieur, reprit Louis XVIII; vous avez acquis aujourd'hui le droit d'interroger.

—Sire, répondit le ministre de la Police, je venais justement aujourd'hui donner à Votre Majesté les nouveaux renseignements que j'avais recueillis sur cet événement, lorsque l'attention de Votre Majesté a été détournée par la terrible catastrophe du golfe; maintenant, ces renseignements n'auraient plus aucun intérêt pour le roi.

—Au contraire, monsieur, au contraire, dit Louis XVIII, cette affaire me semble avoir un rapport direct avec celle qui nous occupe, et la mort du général Quesnel va peut-être nous mettre sur la voie d'un grand complot intérieur.»

À ce nom du général Quesnel, Villefort frissonna.

«En effet, Sire, reprit le ministre de la Police, tout porterait à croire que cette mort est le résultat, non pas d'un suicide, comme on l'avait cru d'abord, mais d'un assassinat: le général Quesnel sortait, à ce qu'il paraît, d'un club bonapartiste lorsqu'il a disparu. Un homme inconnu était venu le chercher le matin même, et lui avait donné rendez-vous rue Saint-Jacques; malheureusement, le valet de chambre du général, qui le coiffait au moment où cet inconnu a été introduit dans le cabinet, a bien entendu qu'il désignait la rue Saint-Jacques, mais n'a pas retenu le numéro.»

À mesure que le ministre de la Police donnait au roi Louis XVIII ces renseignements, Villefort, qui semblait suspendu à ses lèvres, rougissait et pâlissait.

Le roi se retourna de son côté.

«N'est-ce pas votre avis, comme c'est le mien, monsieur de Villefort, que le général Quesnel, que l'on pouvait croire attaché à l'usurpateur, mais qui, réellement, était tout entier à moi, a péri victime d'un guet-apens bonapartiste?

—C'est probable, Sire, répondit Villefort; mais ne sait-on rien de plus?

—On est sur les traces de l'homme qui avait donné le rendez-vous.

—On est sur ses traces? répéta Villefort.

—Oui, le domestique a donné son signalement: c'est un homme de cinquante à cinquante-deux ans, brun, avec des yeux noirs couverts d'épais sourcils, et portant moustaches; il était vêtu d'une redingote bleue, et portait à sa boutonnière une rosette d'officier de la Légion d'honneur. Hier on a suivi un individu dont le signalement répond exactement à celui que je viens de dire, et on l'a perdu au coin de la rue de la Jussienne et de la rue Coq-Héron.»

Villefort s'était appuyé au dossier d'un fauteuil car à mesure que le ministre de la Police parlait, il sentait ses jambes se dérober sous lui; mais lorsqu'il vit que l'inconnu avait échappé aux recherches de l'agent qui le suivait, il respira.

«Vous chercherez cet homme, monsieur, dit le roi au ministre de la Police; car, si, comme tout me porte à le croire, le général Quesnel, qui nous eût été si utile en ce moment, a été victime d'un meurtre, bonapartistes ou non, je veux que ses assassins soient cruellement punis.»

Villefort eut besoin de tout son sang-froid pour ne point trahir la terreur que lui inspirait cette recommandation du roi.

«Chose étrange! continua le roi avec un mouvement d'humeur, la police croit avoir tout dit lorsqu'elle a dit: un meurtre a été commis, et tout fait lorsqu'elle a ajouté: on est sur la trace des coupables.

—Sire, Votre Majesté, sur ce point du moins, sera satisfaite, je l'espère.

—C'est bien, nous verrons; je ne vous retiens pas plus longtemps, baron; monsieur de Villefort, vous devez être fatigué de ce long voyage, allez vous reposer. Vous êtes sans doute descendu chez votre père?»

Un éblouissement passa sur les yeux de Villefort.

«Non, Sire, dit-il, je suis descendu hôtel de Madrid, rue de Tournon.

—Mais vous l'avez vu?

—Sire, je me suis fait tout d'abord conduire chez M. le duc de Blacas.

—Mais vous le verrez, du moins?

—Je ne le pense pas, Sire.

—Ah! c'est juste, dit Louis XVIII en souriant de manière à prouver que toutes ces questions réitérées n'avaient pas été faites sans intention, j'oubliais que vous êtes en froid avec M. Noirtier, et que c'est un nouveau sacrifice fait à la cause royale, et dont il faut que je vous dédommage.

—Sire, la bonté que me témoigne Votre Majesté est une récompense qui dépasse de si loin toutes mes ambitions, que je n'ai rien à demander de plus au roi.

—N'importe, monsieur, et nous ne vous oublierons pas, soyez tranquille; en attendant (le roi détacha la croix de la Légion d'honneur qu'il portait d'ordinaire sur son habit bleu, près de la croix de Saint-Louis, au-dessus de la plaque de l'ordre de Notre-Dame du mont Carmel et de Saint-Lazare, et la donnant à Villefort), en attendant, dit-il, prenez toujours cette croix.

—Sire, dit Villefort, Votre Majesté, se trompe, cette croix est celle d'officier.

—Ma foi, monsieur, dit Louis XVIII, prenez-la telle qu'elle est; je n'ai pas le temps d'en faire demander une autre. Blacas, vous veillerez à ce que le brevet soit délivré à M. de Villefort.»

Les yeux de Villefort se mouillèrent d'une larme d'orgueilleuse joie; il prit la croix et la baisa.

«Et maintenant, demanda-t-il, quels sont les ordres que me fait l'honneur de me donner Votre Majesté?

—Prenez le repos qui vous est nécessaire et songez que, sans force à Paris pour me servir, vous pouvez m'être à Marseille de la plus grande utilité.

—Sire, répondit Villefort en s'inclinant, dans une heure j'aurai quitté Paris.

—Allez, monsieur, dit le roi, et si je vous oubliais—la mémoire des rois est courte—ne craignez pas de vous rappeler à mon souvenir\dots. Monsieur le baron, donnez l'ordre qu'on aille chercher le ministre de la Guerre. Blacas, restez.

—Ah! monsieur, dit le ministre de la Police à Villefort en sortant des Tuileries, vous entrez par la bonne porte et votre fortune est faite.

—Sera-t-elle longue?» murmura Villefort en saluant le ministre, dont la carrière était finie, et en cherchant des yeux une voiture pour rentrer chez lui.

Un fiacre passait sur le quai, Villefort lui fit un signe, le fiacre s'approcha; Villefort donna son adresse et se jeta dans le fond de la voiture, se laissant aller à ses rêves d'ambition. Dix minutes après, Villefort était rentré chez lui; il commanda ses chevaux pour dans deux heures, et ordonna qu'on lui servît à déjeuner.

Il allait se mettre à table lorsque le timbre de la sonnette retentit sous une main franche et ferme: le valet de chambre alla ouvrir, et Villefort entendit une voix qui prononçait son nom.

«Qui peut déjà savoir que je suis ici?» se demanda le jeune homme.

En ce moment, le valet de chambre rentra.

«Eh bien, dit Villefort, qu'y a-t-il donc? qui a sonné? qui me demande?

—Un étranger qui ne veut pas dire son nom.

—Comment! un étranger qui ne veut pas dire son nom? et que me veut cet étranger?

—Il veut parler à monsieur.

—À moi?

—Oui.

—Il m'a nommé?

—Parfaitement.

—Et quelle apparence a cet étranger?

—Mais, monsieur, c'est un homme d'une cinquantaine d'années.

—Petit? grand?

—De la taille de monsieur à peu près.

—Brun ou blond?

—Brun, très brun: des cheveux noirs, des yeux noirs, des sourcils noirs.

—Et vêtu, demanda vivement Villefort, vêtu de quelle façon?

—D'une grande lévite bleue boutonnée du haut en bas; décoré de la Légion d'honneur.

—C'est lui, murmura Villefort en pâlissant.

—Eh pardieu! dit en paraissant sur la porte l'individu dont nous avons déjà donné deux fois le signalement, voilà bien des façons; est-ce l'habitude à Marseille que les fils fassent faire antichambre à leur père?

—Mon père! s'écria Villefort; je ne m'étais donc pas trompé\dots et je me doutais que c'était vous.

—Alors, si tu te doutais que c'était moi, reprit le nouveau venu, en posant sa canne dans un coin et son chapeau sur une chaise, permets-moi de te dire, mon cher Gérard, que ce n'est guère aimable à toi de me faire attendre ainsi.

—Laissez-nous, Germain», dit Villefort.

Le domestique sortit en donnant des marques visibles d'étonnement.



