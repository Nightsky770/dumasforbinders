\chapter{Le crédit illimité}

\lettrine{L}{e} lendemain, vers deux heures de l'après-midi une calèche attelée de deux magnifiques chevaux anglais s'arrêta devant la porte de Monte-Cristo; un homme vêtu d'un habit bleu, à boutons de soie de même couleur, d'un gilet blanc sillonné par une énorme chaîne d'or et d'un pantalon couleur noisette, coiffé de cheveux si noirs et descendant si bas sur les sourcils, qu'on eût pu hésiter à les croire naturels tant ils semblaient peu en harmonie avec celles des rides inférieures qu'ils ne parvenaient point à cacher; un homme enfin de cinquante à cinquante-cinq ans, et qui cherchait à en paraître quarante, passa sa tête par la portière d'un coupé sur le panneau duquel était peinte une couronne de baron, et envoya son groom demander au concierge si le comte de Monte-Cristo était chez lui. 

En attendant, cet homme considérait, avec une attention si minutieuse qu'elle devenait presque impertinente, l'extérieur de la maison, ce que l'on pouvait distinguer du jardin, et la livrée de quelques domestiques que l'on pouvait apercevoir allant et venant. L'œil de cet homme était vif, mais plutôt rusé que spirituel. Ses lèvres étaient si minces, qu'au lieu de saillir en dehors elles rentraient dans la bouche; enfin la largeur et la proéminence des pommettes, signe infaillible d'astuce, la dépression du front, le renflement de l'occiput, qui dépassait de beaucoup de larges oreilles des moins aristocratiques, contribuaient à donner, pour tout physionomiste, un caractère presque repoussant à la figure de ce personnage fort recommandable aux yeux du vulgaire par ses chevaux magnifiques, l'énorme diamant qu'il portait à sa chemise et le ruban rouge qui s'étendait d'une boutonnière à l'autre de son habit. 

Le groom frappa au carreau du concierge et demanda: 

«N'est-ce point ici que demeure M. le comte de Monte-Cristo? 

—C'est ici que demeure Son Excellence, répondit le concierge, mais\dots» 

Il consulta Ali du regard. 

Ali fit un signe négatif. 

«Mais?\dots demanda le groom. 

—Mais Son Excellence n'est pas visible, répondit le concierge. 

—En ce cas, voici la carte de mon maître, M. le baron Danglars. Vous la remettrez au comte de Monte-Cristo, et vous lui direz qu'en allant à la Chambre mon maître s'est détourné pour avoir l'honneur de le voir. 

—Je ne parle pas à Son Excellence, dit le concierge; le valet de chambre fera la commission.» 

Le groom retourna vers la voiture. 

«Eh bien?» demanda Danglars. 

L'enfant, assez honteux de la leçon qu'il venait de recevoir, apporta à son maître la réponse qu'il avait reçue du concierge. 

«Oh! fit celui-ci, c'est donc un prince que ce monsieur, qu'on l'appelle Excellence, et qu'il n'y ait que son valet de chambre qui ait le droit de lui parler; n'importe, puisqu'il a un crédit sur moi, il faudra bien que je le voie quand il voudra de l'argent.» 

Et Danglars se rejeta dans le fond de sa voiture en criant au cocher, de manière qu'on pût l'entendre de l'autre côté de la route: 

«À la Chambre des députés!» 

Au travers d'une jalousie de son pavillon, Monte-Cristo, prévenu à temps, avait vu le baron et l'avait étudié, à l'aide d'une excellente lorgnette, avec non moins d'attention que M. Danglars en avait mis lui-même à analyser la maison, le jardin et les livrées. 

«Décidément, fit-il avec un geste de dégoût et en faisant rentrer les tuyaux de sa lunette dans leur fourreau d'ivoire, décidément c'est une laide créature que cet homme; comment, dès la première fois qu'on le voit, ne reconnaît-on pas le serpent au front aplati, le vautour au crâne bombé et la buse au bec tranchant! 

«Ali!» cria-t-il, puis il frappa un coup sur le timbre de cuivre. Ali parut. «Appelez Bertuccio», dit-il. 

Au même moment Bertuccio entra. 

«Votre Excellence me faisait demander? dit l'intendant.  

—Oui, monsieur, dit le comte. Avez-vous vu les chevaux qui viennent de s'arrêter devant ma porte? 

—Certainement, Excellence, ils sont même fort beaux. 

—Comment se fait-il, dit Monte-Cristo en fronçant le sourcil, quand je vous ai demandé les deux plus beaux chevaux de Paris, qu'il y ait à Paris deux autres chevaux aussi beaux que les miens, et que ces chevaux ne soient pas dans mes écuries?» 

Au froncement de sourcil et à l'intonation sévère de cette voix, Ali baissa la tête. 

«Ce n'est pas ta faute, bon Ali, dit en arabe le comte avec une douceur qu'on n'aurait pas cru pouvoir rencontrer ni dans sa voix, ni sur son visage; tu ne te connais pas en chevaux anglais, toi.» 

La sérénité reparut sur les traits d'Ali. 

«Monsieur le comte, dit Bertuccio, les chevaux dont vous me parlez n'étaient pas à vendre. 

Monte-Cristo haussa les épaules: 

«Sachez, monsieur l'intendant, que tout est toujours à vendre pour qui sait y mettre le prix. 

—M. Danglars les a payés seize mille francs, monsieur le comte.  

—Eh bien, il fallait lui en offrir trente-deux mille; il est banquier, et un banquier ne manque jamais une occasion de doubler son capital. 

—Monsieur le comte parle-t-il sérieusement?» demanda Bertuccio. 

Monte-Cristo regarda l'intendant en homme étonné qu'on ose lui faire une question. 

«Ce soir, dit-il, j'ai une visite à rendre; je veux que ces deux chevaux soient attelés à ma voiture avec un harnais neuf.» 

Bertuccio se retira en saluant; près de la porte, il s'arrêta:  

«À quelle heure, dit-il, Son Excellence compte-t-elle faire cette visite? 

—À cinq heures, dit Monte-Cristo. 

—Je ferai observer à Votre Excellence qu'il est deux heures, hasarda l'intendant. 

—Je le sais», se contenta de répondre Monte-Cristo. 

Puis se retournant vers Ali: 

«Faites passer tous les chevaux devant madame, dit-il, qu'elle choisisse l'attelage qui lui conviendra le mieux, et qu'elle me fasse dire si elle veut dîner avec moi: dans ce cas on servira chez elle; allez; en descendant, vous m'enverrez le valet de chambre.» 

Ali venait à peine de disparaître, que le valet de chambre entra à son tour. 

«Monsieur Baptistin, dit le comte, depuis un an vous êtes à mon service; c'est le temps d'épreuve que j'impose d'ordinaire à mes gens: vous me convenez.» 

Baptistin s'inclina. 

«Reste à savoir si je vous conviens.  

—Oh! monsieur le comte! se hâta de dire Baptistin. 

—Écoutez jusqu'au bout, reprit le comte. Vous gagnez par an quinze cents francs, c'est-à-dire les appointements d'un bon et brave officier qui risque tous les jours sa vie; vous avez une table telle que beaucoup de chefs de bureau, malheureux serviteurs infiniment plus occupés que vous, en désireraient une pareille. Domestique, vous avez vous-même des domestiques qui ont soin de votre linge et de vos effets. Outre vos quinze cents francs de gages, vous me volez, sur les achats que vous faites pour ma toilette, à peu près quinze cents autres francs par an. 

—Oh! Excellence! 

—Je ne m'en plains pas, monsieur Baptistin, c'est raisonnable; cependant je désire que cela s'arrête là. Vous ne retrouveriez donc nulle part un poste pareil à celui que votre bonne fortune vous a donné. Je ne bats jamais mes gens, je ne jure jamais, je ne me mets jamais en colère, je pardonne toujours une erreur, jamais une négligence ou un oubli. Mes ordres sont d'ordinaire courts, mais clairs et précis; j'aime mieux les répéter à deux fois et même à trois, que de les voir mal interprétés. Je suis assez riche pour savoir tout ce que je veux savoir, et je suis fort curieux, je vous en préviens. Si j'apprenais donc que vous ayez parlé de moi en bien ou en mal, commenté mes actions, surveillé ma conduite, vous sortiriez de chez moi à l'instant même. Je n'avertis jamais mes domestiques qu'une seule fois; vous voilà averti, allez!»  

Baptistin s'inclina et fit trois ou quatre pas pour se retirer. 

«À propos, reprit le comte, j'oubliais de vous dire que, chaque année, je place une certaine somme sur la tête de mes gens. Ceux que je renvoie perdent nécessairement cet argent, qui profite à ceux qui restent et qui y auront droit après ma mort. Voilà un an que vous êtes chez moi, votre fortune est commencée, continuez-la.» 

Cette allocution, faite devant Ali, qui demeurait impassible, attendu qu'il n'entendait pas un mot de français, produisit sur M. Baptistin un effet que comprendront tous ceux qui ont étudié la psychologie du domestique français. 

«Je tâcherai de me conformer en tous points aux désirs de Votre Excellence, dit-il; d'ailleurs je me modèlerai sur M. Ali. 

—Oh! pas du tout, dit le comte avec une froideur de marbre. Ali a beaucoup de défauts mêlés à ses qualités; ne prenez donc pas exemple sur lui, car Ali est une exception; il n'a pas de gages, ce n'est pas un domestique, c'est mon esclave, c'est mon chien; s'il manquait à son devoir, je ne le chasserais pas, lui, je le tuerais.» 

Baptistin ouvrit de grands yeux. 

«Vous doutez?» dit Monte-Cristo. 

Et il répéta à Ali les mêmes paroles qu'il venait de dire en français à Baptistin. 

Ali écouta, sourit, s'approcha de son maître, mit un genou à terre, et lui baisa respectueusement la main. 

Ce petit corollaire de la leçon mit le comble à la stupéfaction de M. Baptistin. 

Le comte fit signe à Baptistin de sortir, et à Ali de le suivre. Tous deux passèrent dans son cabinet, et là ils causèrent longtemps. 

À cinq heures, le comte frappa trois coups sur son timbre. Un coup appelait Ali, deux coups Baptistin, trois coups Bertuccio.  

L'intendant entra. 

«Mes chevaux! dit Monte-Cristo. 

—Ils sont à la voiture, Excellence, répliqua Bertuccio. Accompagnerai-je monsieur le comte? 

—Non, le cocher, Baptistin et Ali, voilà tout.» 

Le comte descendit et vit attelés à sa voiture, les chevaux qu'il avait admirés le matin à la voiture de Danglars. 

En passant près d'eux il leur jeta un coup d'œil. 

«Ils sont beaux, en effet, dit-il, et vous avez bien fait de les acheter, seulement c'était un peu tard. 

—Excellence, dit Bertuccio, j'ai eu bien de la peine à les avoir, et ils ont coûté bien cher. 

—Les chevaux en sont-ils moins beaux? demanda le comte en haussant les épaules. 

—Si Votre Excellence est satisfaite, dit Bertuccio, tout est bien. Où va Votre Excellence? 

—Rue de la Chaussée-d'Antin, chez M. le baron Danglars.» 

Cette conversation se passait sur le haut du perron. Bertuccio fit un pas pour descendre la première marche. 

«Attendez, monsieur, dit Monte-Cristo en l'arrêtant. J'ai besoin d'une terre sur le bord de la mer, en Normandie, par exemple, entre le Havre et Boulogne. Je vous donne de l'espace, comme vous voyez. Il faudrait que, dans cette acquisition, il y eût un petit port, une petite crique, une petite baie, où puisse entrer et se tenir ma corvette; elle ne tire que quinze pieds d'eau. Le bâtiment sera toujours prêt à mettre à la mer, à quelque heure du jour ou de la nuit qu'il me plaise de lui donner le signal. Vous vous informerez chez tous les notaires d'une propriété dans les conditions que je vous explique; quand vous en aurez connaissance, vous irez la visiter, et si vous êtes content, vous l'achèterez à votre nom. La corvette doit être en route pour Fécamp, n'est-ce pas? 

—Le soir même où nous avons quitté Marseille, je l'ai vu mettre à la mer. 

—Et le yacht? 

—Le yacht a ordre de demeurer aux Martigues. 

—Bien! Vous correspondrez de temps en temps avec les deux patrons qui les commandent, afin qu'ils ne s'endorment pas. 

—Et pour le bateau à vapeur? 

—Qui est à Chalons? 

—Oui. 

—Même ordres que pour les deux navires à voiles. 

—Bien! 

—Aussitôt cette propriété achetée, j'aurai des relais de dix lieues en dix lieues sur la route du Nord et sur la route du Midi. 

—Votre Excellence peut compter sur moi.» 

Le comte fit un signe de satisfaction, descendit les degrés, sauta dans sa voiture, qui, entraînée au trot du magnifique attelage, ne s'arrêta que devant l'hôtel du banquier. Danglars présidait une commission nommée pour un chemin de fer, lorsqu'on vint lui annoncer la visite du comte de Monte-Cristo. La séance, au reste, était presque finie. 

Au nom du comte, il se leva. 

«Messieurs, dit-il en s'adressant à ses collègues, dont plusieurs étaient des honorables membres de l'une ou l'autre Chambre, pardonnez-moi si je vous quitte ainsi; mais imaginez-vous que la maison Thomson et French, de Rome, m'adresse un certain comte de Monte-Cristo, en lui ouvrant chez moi un crédit illimité. C'est la plaisanterie la plus drôle que mes correspondants de l'étranger se soient encore permise vis-à-vis de moi. Ma foi, vous le comprenez, la curiosité m'a saisi et me tient encore; je suis passé ce matin chez le prétendu comte. Si c'était un vrai comte, vous comprenez qu'il ne serait pas si riche. Monsieur n'était pas visible. Que vous en semble? ne sont-ce point des façons d'altesse ou de jolie femme que se donne là maître Monte-Cristo? Au reste, la maison située aux Champs-Élysées et qui est à lui, je m'en suis informé, m'a paru propre. Mais un crédit illimité, reprit Danglars en riant de son vilain sourire, rend bien exigeant le banquier chez qui le crédit est ouvert. J'ai donc hâte de voir notre homme. Je me crois mystifié. Mais ils ne savent point là-bas à qui ils ont affaire; rira bien qui rira le dernier.» 

En achevant ces mots et en leur donnant une emphase qui gonfla les narines de M. le baron, celui-ci quitta ses hôtes et passa dans un salon blanc et or qui faisait grand bruit dans la Chaussée-d'Antin. 

C'est là qu'il avait ordonné d'introduire le visiteur pour l'éblouir du premier coup. 

Le comte était debout, considérant quelques copies de l'Albane et du Fattore qu'on avait fait passer au banquier pour des originaux, et qui, toutes copies qu'elles étaient, juraient fort avec les chicorées d'or de toutes couleurs qui garnissaient les plafonds. 

Au bruit que fit Danglars en entrant, le comte se retourna. 

Danglars salua légèrement de la tête, et fit signe au comte de s'asseoir dans un fauteuil de bois doré garni de satin blanc broché d'or. 

Le comte s'assit. 

«C'est à monsieur de Monte-Cristo que j'ai l'honneur de parler? 

—Et moi, répondit le comte, à monsieur le baron Danglars, chevalier de la Légion d'honneur, membre de la Chambre des députés?» 

Monte-Cristo redisait tous les titres qu'il avait trouvés sur la carte du baron. 

Danglars sentit la botte et se mordit les lèvres. 

«Excusez-moi, monsieur, dit-il, de ne pas vous avoir donné du premier coup le titre sous lequel vous m'avez été annoncé; mais, vous le savez, nous vivons sous un gouvernement populaire, et moi, je suis un représentant des intérêts du peuple. 

—De sorte, répondit Monte-Cristo, que, tout en conservant l'habitude de vous faire appeler baron, vous avez perdu celle d'appeler les autres, comte. 

—Ah! je n'y tiens pas même pour moi, monsieur, répondit négligemment Danglars; ils m'ont nommé baron et fait chevalier de la Légion d'honneur pour quelques services rendus, mais\dots. 

—Mais vous avez abdiqué vos titres, comme ont fait autrefois MM. de Montmorency et de Lafayette? C'était un bel exemple à suivre, monsieur. 

—Pas tout à fait, cependant, reprit Danglars embarrassé; pour les domestiques, vous comprenez\dots. 

—Oui, vous vous appelez monseigneur pour vos gens; pour les journalistes, vous vous appelez monsieur; et pour vos commettants, citoyen. Ce sont des nuances très applicables au gouvernement constitutionnel. Je comprends parfaitement.» 

Danglars se pinça les lèvres: il vit que, sur ce terrain-là, il n'était pas de force avec Monte-Cristo, il essaya donc de revenir sur un terrain qui lui était plus familier. 

«Monsieur le comte, dit-il en s'inclinant, j'ai reçu une lettre d'avis de la maison Thomson et French.  

—J'en suis charmé, monsieur le baron. Permettez-moi de vous traiter comme vous traitent vos gens, c'est une mauvaise habitude prise dans des pays où il y a encore des barons, justement parce qu'on n'en fait plus. J'en suis charmé, dis-je; je n'aurai pas besoin de me présenter moi-même, ce qui est toujours assez embarrassant. Vous aviez donc, disiez-vous, reçu une lettre d'avis? 

—Oui, dit Danglars; mais je vous avoue que je n'en ai pas parfaitement compris le sens. 

—Bah! 

—Et j'avais même eu l'honneur de passer chez vous pour vous demander quelques explications. 

—Faites, monsieur, me voilà, j'écoute et suis prêt à vous entendre. 

—Cette lettre, dit Danglars, je l'ai sur moi, je crois (il fouilla dans sa poche). Oui, la voici: cette lettre ouvre à M. le comte de Monte-Cristo un crédit illimité sur ma maison. 

—Eh bien, monsieur le baron, que voyez-vous d'obscur là-dedans? 

—Rien, monsieur; seulement le mot \textit{illimité}\dots 

—Eh bien, ce mot n'est-il pas français?\dots Vous comprenez, ce sont des Anglo-Allemands qui écrivent. 

—Oh! si fait, monsieur, et du côté de la syntaxe il n'y a rien à redire, mais il n'en est pas de même du côté de la comptabilité. 

—Est-ce que la maison Thomson et French, demanda Monte-Cristo de l'air le plus naïf qu'il put prendre, n'est point parfaitement sûre, à votre avis, monsieur le baron? diable! cela me contrarierait, car j'ai quelques fonds placés chez elle. 

—Ah! parfaitement sûre, répondit Danglars avec un sourire presque railleur; mais le sens du mot illimité, en matière de finances, est tellement vague\dots. 

—Qu'il est illimité, n'est-ce pas? dit Monte-Cristo. 

—C'est justement cela, monsieur, que je voulais dire. Or, le vague, c'est le doute, et, dit le sage, dans le doute abstiens-toi. 

—Ce qui signifie, reprit Monte-Cristo, que si la maison Thomson et French est disposée à faire des folies, la maison Danglars ne l'est pas à suivre son exemple. 

—Comment cela, monsieur le comte? 

—Oui, sans doute, MM. Thomson et French font les affaires sans chiffres; mais M. Danglars a une limite aux siennes; c'est un homme sage, comme il disait tout à l'heure.  

—Monsieur, répondit orgueilleusement le banquier, personne n'a encore compté avec ma caisse. 

—Alors, répondit froidement Monte-Cristo, il paraît que c'est moi qui commencerai. 

—Qui vous dit cela? 

—Les explications que vous me demandez, monsieur, et qui ressemblent fort à des hésitations\dots» 

Danglars se mordit les lèvres; c'était la seconde fois qu'il était battu par cet homme et cette fois sur un terrain qui était le sien. Sa politesse railleuse n'était qu'affectée, et touchait à cet extrême si voisin qui est l'impertinence. 

Monte-Cristo, au contraire, souriait de la meilleure grâce du monde, et possédait, quand il le voulait, un certain air naïf qui lui donnait bien des avantages. 

«Enfin, monsieur, dit Danglars après un moment de silence, je vais essayer de me faire comprendre en vous priant de fixer vous-même la somme que vous comptez toucher chez moi. 

—Mais, monsieur, reprit Monte-Cristo décidé à ne pas perdre un pouce de terrain dans la discussion, si j'ai demandé un crédit illimité sur vous, c'est que je ne savais justement pas de quelles sommes j'aurais besoin.» 

Le banquier crut que le moment était venu enfin de prendre le dessus; il se renversa dans son fauteuil, et avec un lourd et orgueilleux sourire: 

«Oh! monsieur, dit-il, ne craignez pas de désirer; vous pourrez vous convaincre alors que le chiffre de la maison Danglars, tout limité qu'il est, peut satisfaire les plus larges exigences, et dussiez-vous demander un million\dots. 

—Plaît-il? fit Monte-Cristo. 

—Je dis un million, répéta Danglars avec l'aplomb de la sottise. 

—Et que ferais-je d'un million? dit le comte. Bon Dieu! monsieur, s'il ne m'eût fallu qu'un million, je ne me serais pas fait ouvrir un crédit pour une pareille misère. Un million? mais j'ai toujours un million dans mon portefeuille ou dans mon nécessaire de voyage.» 

Et Monte-Cristo retira d'un petit carnet où étaient ses cartes de visite deux bons de cinq cent mille francs chacun, payables au porteur, sur le Trésor. 

Il fallait assommer et non piquer un homme comme Danglars. Le coup de massue fit son effet: le banquier chancela et eut le vertige; il ouvrit sur Monte-Cristo deux yeux hébétés dont la prunelle se dilata effroyablement. 

«Voyons, avouez-moi, dit Monte-Cristo, que vous vous défiez de la maison Thomson et French. Mon Dieu! c'est tout simple; j'ai prévu le cas, et, quoique assez étranger aux affaires, j'ai pris mes précautions. Voici donc deux autres lettres pareilles à celle qui vous est adressée, l'une est de la maison Arestein et Eskoles, de Vienne, sur M. le baron de Rothschild, l'autre est de la maison Baring, de Londres, sur M. Laffitte. Dites un mot, monsieur, et je vous ôterai toute préoccupation, en me présentant dans l'une ou l'autre de ces deux maisons.» 

C'en était fait, Danglars était vaincu; il ouvrit avec un tremblement visible la lettre de Vienne et la lettre de Londres, que lui tendait du bout des doigts le comte, vérifia l'authenticité des signatures avec une minutie qui eût été insultante pour Monte-Cristo, s'il n'eût pas fait la part de l'égarement du banquier. 

«Oh! monsieur, voilà trois signatures qui valent bien des millions, dit Danglars en se levant comme pour saluer la puissance de l'or personnifiée en cet homme qu'il avait devant lui. Trois crédits illimités sur nos maisons! Pardonnez-moi, monsieur le comte, mais tout en cessant d'être défiant, on peut demeurer encore étonné. 

—Oh! ce n'est pas une maison comme la vôtre qui s'étonnerait ainsi, dit Monte-Cristo avec toute sa politesse; ainsi, vous pourrez donc m'envoyer quelque argent, n'est-ce pas? 

—Parlez, monsieur le comte; je suis à vos ordres.  

—Eh bien, reprit Monte-Cristo, à présent que nous nous entendons, car nous nous entendons, n'est-ce pas?» 

Danglars fit un signe de tête affirmatif. 

«Et vous n'avez plus aucune défiance? continua Monte-Cristo. 

—Oh! monsieur le comte! s'écria le banquier, je n'en ai jamais eu. 

—Non; vous désiriez une preuve, voilà tout. Eh bien, répéta le comte, maintenant que nous nous entendons, maintenant que vous n'avez plus aucune défiance, fixons, si vous le voulez bien, une somme générale pour la première année: six millions, par exemple.  

—Six millions, soit! dit Danglars suffoqué. 

—S'il me faut plus, reprit machinalement Monte-Cristo, nous mettrons plus; mais je ne compte rester qu'une année en France, et pendant cette année je ne crois pas dépasser ce chiffre\dots enfin nous verrons\dots. Veuillez, pour commencer, me faire porter cinq cent mille francs demain, je serai chez moi jusqu'à midi, et d'ailleurs, si je n'y étais pas, je laisserais un reçu à mon intendant. 

—L'argent sera chez vous demain à dix heures du matin, monsieur le comte, répondit Danglars. Voulez-vous de l'or, ou des billets de banque, ou de l'argent?  

—Or et billets par moitié, s'il vous plaît. 

Et le comte se leva. 

«Je dois vous confesser une chose, monsieur le comte, dit Danglars à son tour; je croyais avoir des notions exactes sur toutes les belles fortunes de l'Europe, et cependant la vôtre, qui me paraît considérable, m'était, je l'avoue, tout à fait inconnue; elle est récente? 

—Non, monsieur, répondit Monte-Cristo, elle est, au contraire, de fort vieille date: c'était une espèce de trésor de famille auquel il était défendu de toucher, et dont les intérêts accumulés ont triplé le capital; l'époque fixée par le testateur est révolue depuis quelques années seulement: ce n'est donc que depuis quelques années que j'en use, et votre ignorance à ce sujet n'a rien que de naturel; au reste, vous la connaîtrez mieux dans quelque temps.» 

Et le comte accompagna ces mots d'un de ces sourires pâles qui faisaient si grand-peur à Franz d'Épinay. 

«Avec vos goûts et vos intentions, monsieur, continua Danglars, vous allez déployer dans la capitale un luxe qui va nous écraser tous, nous autres pauvres petits millionnaires: cependant comme vous me paraissez amateur, car lorsque je suis entré vous regardiez mes tableaux, je vous demande la permission de vous faire voir ma galerie: tous tableaux anciens, tous tableaux de maîtres garantis comme tels; je n'aime pas les modernes. 

—Vous avez raison, monsieur, car ils ont en général un grand défaut: c'est celui de n'avoir pas encore eu le temps de devenir des anciens. 

—Puis-je vous montrer quelques statues de Thorwaldsen, de Bartoloni, de Canova, tous artistes étrangers? Comme vous voyez, je n'apprécie pas les artistes français. 

—Vous avez le droit d'être injuste avec eux, monsieur, ce sont vos compatriotes. 

—Mais tout cela sera pour plus tard, quand nous aurons fait meilleure connaissance, pour aujourd'hui, je me contenterai, si vous le permettez toutefois, de vous présenter à Mme la baronne Danglars; excusez mon empressement, monsieur le comte, mais un client comme vous fait presque partie de la famille.» 

Monte-Cristo s'inclina, en signe qu'il acceptait l'honneur que le financier voulait bien lui faire. 

Danglars sonna; un laquais, vêtu d'une livrée éclatante, parut. 

«Mme la baronne est-elle chez elle? demanda Danglars. 

—Oui, monsieur le baron, répondit le laquais. 

—Seule?  

—Non, madame a du monde. 

—Ce ne sera pas indiscret de vous présenter devant quelqu'un, n'est-ce pas, monsieur le comte? Vous ne gardez pas l'incognito? 

—Non, monsieur le baron, dit en souriant Monte-Cristo, je ne me reconnais pas ce droit-là. 

—Et qui est près de madame? M. Debray?» demanda Danglars avec une bonhomie qui fit sourire intérieurement Monte-Cristo, déjà renseigné sur les transparents secrets d'intérieur du financier. 

«M. Debray, oui, monsieur le baron», répondit le laquais. 

Danglars fit un signe de tête. 

Puis se tournant vers Monte-Cristo: 

«M. Lucien Debray, dit-il, est un ancien ami à nous, secrétaire intime du ministre de l'intérieur; quant à ma femme, elle a dérogé en m'épousant, car elle appartient à une ancienne famille, c'est une demoiselle de Servières, veuve en premières noces de M. le colonel marquis de Nargonne. 

—Je n'ai pas l'honneur de connaître Mme Danglars; mais j'ai déjà rencontré M. Lucien Debray. 

—Bah! dit Danglars, où donc cela?  

—Chez M. de Morcerf. 

—Ah! vous connaissez le petit vicomte, dit Danglars. 

—Nous nous sommes trouvés ensemble à Rome à l'époque du carnaval. 

—Ah! oui, dit Danglars; n'ai-je pas entendu parler de quelque chose comme une aventure singulière avec des bandits, des voleurs dans les ruines? Il a été tiré de là miraculeusement. Je crois qu'il a raconté quelque chose de tout cela à ma femme et à ma fille à son retour d'Italie. 

—Mme la baronne attend ces messieurs, revint dire le laquais.  

—Je passe devant pour vous montrer le chemin, fit Danglars en saluant. 

—Et moi, je vous suis», dit Monte-Cristo. 