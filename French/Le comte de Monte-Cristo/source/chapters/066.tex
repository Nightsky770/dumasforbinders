\chapter{Projets de mariage}

\lettrine{L}{e} lendemain de cette scène, à l'heure que Debray avait coutume de choisir pour venir faire, en allant à son bureau, une petite visite à Mme Danglars, son coupé ne parut pas dans la cour. 

\zz
À cette heure-là, c'est-à-dire vers midi et demi, Mme Danglars demanda sa voiture et sortit. 

Danglars, placé derrière un rideau, avait guetté cette sortie qu'il attendait. Il donna l'ordre qu'on le prévînt aussitôt que madame reparaîtrait; mais à deux heures, elle n'était pas rentrée. 

À deux heures il demanda ses chevaux, se rendit à la Chambre et se fit inscrire pour parler contre le budget. 

De midi à deux heures, Danglars était resté à son cabinet, décachetant ses dépêches, s'assombrissant de plus en plus, entassant chiffres sur chiffres et recevant entre autres visites celle du major Cavalcanti qui, toujours aussi bleu, aussi raide et aussi exact, se présenta à l'heure annoncée la veille pour terminer son affaire avec le banquier.  

En sortant de la Chambre, Danglars, qui avait donné de violentes marques d'agitation pendant la séance et qui surtout avait été plus acerbe que jamais contre le ministère, remonta dans sa voiture et ordonna au cocher de le conduire avenue des Champs-Élysées, n°30. 

Monte-Cristo était chez lui; seulement il était avec quelqu'un, et il priait Danglars d'attendre un instant au salon. 

Pendant que le banquier attendait, la porte s'ouvrit, et il vit entrer un homme habillé en abbé, qui, au lieu d'attendre comme lui, plus familier que lui sans doute dans la maison, le salua, entra dans l'intérieur des appartements et disparut. 

Un instant après, la porte par laquelle le prêtre était entré se rouvrit, et Monte-Cristo parut. 

«Pardon, dit-il, cher baron, mais un de mes bons amis, l'abbé Busoni, que vous avez pu voir passer, vient d'arriver à Paris; il y avait fort longtemps que nous étions séparés, et je n'ai pas eu le courage de le quitter tout aussitôt. J'espère qu'en faveur du motif vous m'excuserez de vous avoir fait attendre. 

—Comment donc, dit Danglars, c'est tout simple; c'est moi qui ai mal pris mon moment, et je vais me retirer. 

—Point du tout; asseyez-vous donc, au contraire. Mais, bon Dieu! qu'avez-vous donc? vous avez l'air tout soucieux; en vérité vous m'effrayez. Un capitaliste chagrin est comme les comètes, il présage toujours quelque grand malheur au monde. 

—J'ai, mon cher monsieur, dit Danglars, que la mauvaise chance est sur moi depuis plusieurs jours, et que je n'apprends que des sinistres. 

—Ah! mon Dieu! dit Monte-Cristo, est-ce que vous avez eu une rechute à la Bourse? 

—Non, j'en suis guéri, pour quelques jours du moins; il s'agit tout bonnement pour moi d'une banqueroute à Trieste. 

—Vraiment? Est-ce que votre banqueroutier serait par hasard Jacopo Manfredi? 

—Justement! Figurez-vous un homme qui faisait, depuis je ne sais combien de temps, pour huit ou neuf cent mille francs par an d'affaires avec moi. Jamais un mécompte, jamais un retard; un gaillard qui payait comme un prince\dots qui paie. Je me mets en avance d'un million avec lui, et ne voilà-t-il pas mon diable de Jacopo Manfredi qui suspend ses paiements! 

—En vérité? 

—C'est une fatalité inouïe. Je tire sur lui six cent mille livres, qui me reviennent impayées, et de plus je suis encore porteur de quatre cent mille francs de lettres de change signées par lui et payables fin courant chez son correspondant de Paris. Nous sommes le 30, j'envoie toucher; ah! bien oui, le correspondant a disparu. Avec mon affaire d'Espagne, cela me fait une gentille fin de mois. 

—Mais est-ce vraiment une perte, votre affaire d'Espagne? 

—Certainement, sept cent mille francs hors de ma caisse, rien que cela. 

—Comment diable avez-vous fait une pareille école, vous un vieux loup-cervier? 

—Eh! c'est la faute de ma femme. Elle a rêvé que don Carlos était rentré en Espagne; elle croit aux rêves. C'est du magnétisme, dit-elle, et quand elle rêve une chose, cette chose, à ce qu'elle assure, doit infailliblement arriver. Sur sa conviction, je lui permets de jouer: elle a sa cassette et son agent de change: elle joue et elle perd. Il est vrai que ce n'est pas mon argent, mais le sien qu'elle joue. Cependant, n'importe, vous comprendrez que lorsque sept cent mille francs sortent de la poche de la femme, le mari s'en aperçoit toujours bien un peu. Comment! vous ne saviez pas cela? Mais la chose a fait un bruit énorme. 

—Si fait, j'en avais entendu parler, mais j'ignorais les détails; puis je suis on ne peut plus ignorant de toutes ces affaires de Bourse. 

—Vous ne jouez donc pas? 

—Moi! et comment voulez-vous que je joue? Moi qui ai déjà tant de peine à régler mes revenus, je serais forcé, outre mon intendant, de prendre encore un commis et un garçon de caisse. Mais, à propos d'Espagne, il me semble que la baronne n'avait pas tout à fait rêvé l'histoire de la rentrée de don Carlos. Les journaux n'ont-ils pas dit quelque chose de cela? 

—Vous croyez donc aux journaux, vous? 

—Moi, pas le moins du monde; mais il me semble que cet honnête \textit{Messager} faisait exception à la règle, et qu'il n'annonçait que les nouvelles certaines, les nouvelles télégraphiques. 

—Eh bien, voilà ce qui est inexplicable, reprit Danglars, c'est que cette rentrée de don Carlos était effectivement une nouvelle télégraphique. 

—En sorte, dit Monte-Cristo, que c'est dix-sept cent mille francs à peu près que vous perdez ce mois-ci? 

—Il n'y a pas d'à peu près, c'est juste mon chiffre. 

—Diable! pour une fortune de troisième ordre, dit Monte-Cristo avec compassion, c'est un rude coup. 

—De troisième ordre! dit Danglars un peu humilié; que diable entendez-vous par là? 

—Sans doute, continua Monte-Cristo, je fais trois catégories dans les fortunes: fortune de premier ordre, fortune de deuxième ordre, fortune de troisième ordre. J'appelle fortune de premier ordre celle qui se compose de trésors que l'on a sous la main, les terres, les mines, les revenus sur des États comme la France, l'Autriche et l'Angleterre, pourvu que ces trésors, ces mines, ces revenus, forment un total d'une centaine de millions; j'appelle fortune de second ordre les exploitations manufacturières, les entreprises par association, les vice-royautés et les principautés ne dépassant pas quinze cent mille francs de revenu, le tout formant un capital d'une cinquantaine de millions; j'appelle enfin fortune de troisième ordre les capitaux fructifiant par intérêts composés, les gains dépendant de la volonté d'autrui ou des chances du hasard, qu'une banqueroute entame, qu'une nouvelle télégraphique ébranle; les spéculations éventuelles, les opérations soumises enfin aux chances de cette fatalité qu'on pourrait appeler force mineure, en la comparant à la force majeure, qui est la force naturelle; le tout formant un capital fictif ou réel d'une quinzaine de millions. N'est-ce point là votre position à peu près, dites? 

—Mais dame, oui! répondit Danglars. 

—Il en résulte qu'avec six fins de mois comme celle-là, continua imperturbablement Monte-Cristo, une maison de troisième ordre serait à l'agonie. 

—Oh! dit Danglars avec un sourire fort pâle, comme vous y allez! 

—Mettons sept mois, répliqua Monte-Cristo du même ton. Dites-moi, avez-vous pensé à cela quelquefois, que sept fois dix-sept cent mille francs font douze millions ou à peu près?\dots Non? Eh bien, vous avez raison, car avec des réflexions pareilles on n'engagerait jamais ses capitaux, qui sont au financier ce que la peau est à l'homme civilisé. Nous avons nos habits plus ou moins somptueux, c'est notre crédit; mais quand l'homme meurt, il n'a que sa peau, de même qu'en sortant des affaires, vous n'avez que votre bien réel, cinq ou six millions tout au plus; car les fortunes de troisième ordre ne représentent guère que le tiers ou le quart de leur apparence, comme la locomotive d'un chemin de fer n'est toujours, au milieu de la fumée qui l'enveloppe et qui la grossit, qu'une machine plus ou moins forte. Eh bien, sur ces cinq millions qui forment votre actif réel, vous venez d'en perdre à peu près deux, qui diminuent d'autant votre fortune fictive ou votre crédit; c'est-à-dire, mon cher monsieur Danglars, que votre peau vient d'être ouverte par une saignée qui, réitérée quatre fois, entraînerait la mort. Eh! eh! faites attention, mon cher monsieur Danglars. Avez-vous besoin d'argent? Voulez-vous que je vous en prête? 

—Que vous êtes un mauvais calculateur! s'écria Danglars en appelant à son aide toute la philosophie et toute la dissimulation de l'apparence: à l'heure qu'il est, l'argent est rentré dans mes coffres par d'autres spéculations qui ont réussi. Le sang sorti par la saignée est rentré par la nutrition. J'ai perdu une bataille en Espagne, j'ai été battu à Trieste; mais mon armée navale de l'Inde aura pris quelques galions; mes pionniers du Mexique auront découvert quelque mine. 

—Fort bien, fort bien! mais la cicatrice reste, et à la première perte elle se rouvrira. 

—Non, car je marche sur des certitudes, poursuivit Danglars avec la faconde banale du charlatan, dont l'état est de prôner son crédit; il faudrait pour me renverser, que trois gouvernements croulassent. 

—Dame! cela s'est vu. 

—Que la terre manquât de récoltes. 

—Rappelez-vous les sept vaches grasses et les sept vaches maigres. 

—Ou que la mer se retirât, comme du temps de \textit{Pharaon}; encore il y a plusieurs mers, et les vaisseaux en seraient quittes pour se faire caravanes.  

—Tant mieux, mille fois tant mieux, cher monsieur Danglars, dit Monte-Cristo; et je vois que je m'étais trompé, et que vous rentrez dans les fortunes du second ordre. 

—Je crois pouvoir aspirer à cet honneur, dit Danglars avec un de ces sourires stéréotypés qui faisaient à Monte-Cristo l'effet d'une de ces lunes pâteuses dont les mauvais peintres badigeonnent leurs ruines; mais, puisque nous en sommes à parler d'affaires, ajouta-t-il, enchanté de trouver ce motif de changer de conversation, dites-moi donc un peu ce que je puis faire pour M. Cavalcanti. 

—Mais, lui donner de l'argent, s'il a un crédit sur vous et que ce crédit vous paraisse bon. 

—Excellent! il s'est présenté ce matin avec un bon de quarante mille francs, payable à vue sur vous, signé Busoni, et renvoyé par vous à moi avec votre endos. Vous comprenez que je lui ai compté à l'instant même ses quarante billets carrés.» 

Monte-Cristo fit un signe de tête qui indiquait toute son adhésion. 

«Mais ce n'est pas tout, continua Danglars; il a ouvert à son fils un crédit chez moi. 

—Combien, sans indiscrétion, donne-t-il au jeune homme? 

—Cinq mille francs par mois. 

—Soixante mille francs par an. Je m'en doutais bien, dit Monte-Cristo en haussant les épaules; ce sont des pleutres que les Cavalcanti. Que veut-il qu'un jeune homme fasse avec cinq mille francs par mois? 

—Mais vous comprenez que si le jeune homme a besoin de quelques mille de francs de plus\dots. 

—N'en faites rien, le père vous les laisserait pour votre compte; vous ne connaissez pas tous les millionnaires ultramontains: ce sont de véritables harpagons. Et par qui lui est ouvert ce crédit? 

—Oh! par la maison Fenzi, une des meilleures de Florence. 

—Je ne veux pas dire que vous perdrez, tant s'en faut; mais tenez-vous cependant dans les termes de la lettre. 

—Vous n'auriez donc pas confiance dans ce Cavalcanti? 

—Moi! je lui donnerais dix millions sur sa signature. Cela rentre dans les fortunes de second ordre, dont je vous parlais tout à l'heure, mon cher monsieur Danglars. 

—Et avec cela comme il est simple! Je l'aurais pris pour un major, rien de plus. 

—Et vous lui eussiez fait honneur; car, vous avez raison, il ne paie pas de mine. Quand je l'ai vu pour la première fois, il m'a fait l'effet d'un vieux lieutenant moisi sous la contre épaulette. Mais tous les Italiens sont comme cela, ils ressemblent à de vieux juifs quand ils n'éblouissent pas comme des mages d'Orient. 

—Le jeune homme est mieux, dit Danglars. 

—Oui, un peu timide, peut-être; mais, en somme, il m'a paru convenable. J'en étais inquiet. 

—Pourquoi cela? 

—Parce que vous l'avez vu chez moi à peu près à son entrée dans le monde, à ce que l'on m'a dit du moins. Il a voyagé avec un précepteur très sévère et n'était jamais venu à Paris. 

—Tous ces Italiens de qualité ont l'habitude de se marier entre eux, n'est-ce pas? demanda négligemment Danglars; ils aiment à associer leurs fortunes. 

—D'habitude ils font ainsi, c'est vrai; mais Cavalcanti est un original qui ne fait rien comme les autres. On ne m'ôtera pas de l'idée qu'il envoie son fils en France pour qu'il y trouve une femme. 

—Vous croyez? 

—J'en suis sûr. 

—Et vous avez entendu parler de sa fortune? 

—Il n'est question que de cela; seulement les uns lui accordent des millions, les autres prétendent qu'il ne possède pas un paul. 

—Et votre opinion à vous? 

—Il ne faudra pas vous fonder dessus; elle est toute personnelle. 

—Mais, enfin\dots. 

—Mon opinion, à moi, est que tous ces vieux podestats, tous ces anciens condottieri, car ces Cavalcanti ont commandé des armées, ont gouverné des provinces; mon opinion, dis-je, est qu'ils ont enterré des millions dans des coins que leurs aînés seuls connaissent et font connaître à leurs aînés de génération en génération; et la preuve, c'est qu'ils sont tous jaunes et secs comme leurs florins du temps de la République, dont ils conservent un reflet à force de les regarder. 

—Parfait, dit Danglars; et c'est d'autant plus vrai qu'on ne leur connaît pas un pouce de terre, à tous ces gens-là. 

—Fort peu, du moins; moi, je sais bien que je ne connais à Cavalcanti que son palais de Lucques. 

—Ah! il a un palais! dit en riant Danglars; c'est déjà quelque chose. 

—Oui, et encore le loue-t-il au ministre des Finances, tandis qu'il habite lui, dans une maisonnette. Oh! je vous l'ai déjà dit, je crois le bonhomme serré. 

—Allons, allons, vous ne le flattez pas. 

—Écoutez, je le connais à peine: je crois l'avoir vu trois fois dans ma vie. Ce que j'en sais, c'est par l'abbé Busoni et par lui-même; il me parlait ce matin de ses projets sur son fils, et me laissait entrevoir que, las de voir dormir des fonds considérables en Italie, qui est un pays mort, il voudrait trouver un moyen, soit en France, soit en Angleterre, de faire fructifier ses millions. Mais remarquez bien toujours que, quoique j'aie la plus grande confiance dans l'abbé Busoni personnellement, moi, je ne réponds de rien. 

—N'importe, merci du client que vous m'avez envoyé; c'est un fort beau nom à inscrire sur mes registres, et mon caissier, à qui j'ai expliqué ce que c'étaient que les Cavalcanti, en est tout fier. À propos, et ceci est un simple détail de touriste, quand ces gens-là marient leurs fils, leur donnent-ils des dots? 

—Eh, mon Dieu! c'est selon. J'ai connu un prince italien, riche comme une mine d'or, un des premiers noms de Toscane, qui, lorsque ses fils se mariaient à sa guise, leur donnait des millions, et, quand ils se mariaient malgré lui, se contentait de leur faire une rente de trente écus par mois. Admettons qu'Andrea se marie selon les vues de son père, il lui donnera peut-être un, deux, trois millions. Si c'était avec la fille d'un banquier, par exemple, peut-être prendrait-il un intérêt dans la maison du beau-père de son fils; puis, supposez à côté de cela que sa bru lui déplaise: bonsoir, le père Cavalcanti met la main sur la clef de son coffre-fort, donne un double tour à la serrure, et voilà maître Andrea obligé de vivre comme un fils de famille parisien, en bizeautant des cartes ou en pipant des dés. 

—Ce garçon-là trouvera une princesse bavaroise ou péruvienne; il voudra une couronne fermée, un Eldorado traversé par le Potose. 

—Non, tous ces grands seigneurs de l'autre côté des monts épousent fréquemment de simples mortelles; ils sont comme Jupiter, ils aiment à croiser les races. Ah çà! est-ce que vous voulez marier Andrea, mon cher monsieur Danglars, que vous me faites toutes ces questions-là?  

—Ma foi, dit Danglars, cela ne me paraîtrait pas une mauvaise spéculation; et je suis un spéculateur. 

—Ce n'est pas avec Mlle Danglars, je présume? vous ne voudriez pas faire égorger ce pauvre Andrea par Albert? 

—Albert? dit Danglars en haussant les épaules; ah! bien oui, il se soucie pas mal de cela. 

—Mais il est fiancé avec votre fille, je crois? 

—C'est-à-dire que M. de Morcerf et moi, nous avons quelquefois causé de ce mariage; mais Mme de Morcerf et Albert\dots. 

—N'allez-vous pas me dire que celui-ci n'est pas un bon parti? 

—Eh! eh! Mlle Danglars vaut bien M. de Morcerf, ce me semble! 

—La dot de Mlle Danglars sera belle, en effet, et je n'en doute pas, surtout si le télégraphe ne fait plus de nouvelles folies. 

—Oh! ce n'est pas seulement la dot. Mais, dites-moi donc, à propos? 

—Eh bien! 

—Pourquoi donc n'avez-vous pas invité Morcerf et sa famille à votre dîner? 

—Je l'avais fait aussi, mais il a objecté un voyage à Dieppe avec Mme de Morcerf, à qui on a recommandé l'air de la mer.  

—Oui, oui, dit Danglars en riant, il doit lui être bon. 

—Pourquoi cela? 

—Parce que c'est l'air qu'elle a respiré dans sa jeunesse.» 

Monte-Cristo laissa passer l'épigramme sans paraître y faire attention. 

«Mais enfin, dit le comte, si Albert n'est point aussi riche que Mlle Danglars, vous ne pouvez nier qu'il porte un beau nom. 

—Soit, mais j'aime autant le mien, dit Danglars. 

—Certainement, votre nom est populaire, et il a orné le titre dont on a cru l'orner; mais vous êtes un homme trop intelligent pour n'avoir point compris que, selon certains préjugés trop puissamment enracinés pour qu'on les extirpe, noblesse de cinq siècles vaut mieux que noblesse de vingt ans. 

—Et voilà justement pourquoi, dit Danglars avec un sourire qu'il essayait de rendre sardonique, voilà pourquoi je préférerais M. Andrea Cavalcanti à M. Albert de Morcerf. 

—Mais cependant, dit Monte-Cristo, je suppose que les Morcerf ne le cèdent pas aux Cavalcanti? 

—Les Morcerf!\dots Tenez, mon cher comte, reprit Danglars, vous êtes un galant homme, n'est-ce pas? 

—Je le crois. 

—Et, de plus, connaisseur en blason? 

—Un peu. 

—Eh bien, regardez la couleur du mien; elle est plus solide que celle du blason de Morcerf. 

—Pourquoi cela? 

—Parce que, moi, si je ne suis pas baron de naissance, je m'appelle Danglars au moins. 

—Après? 

—Tandis que lui ne s'appelle pas Morcerf. 

—Comment, il ne s'appelle pas Morcerf? 

—Pas le moins du monde. 

—Allons donc! 

—Moi, quelqu'un m'a fait baron, de sorte que je le suis; lui s'est fait comte tout seul, de sorte qu'il ne l'est pas. 

—Impossible. 

—Écoutez, mon cher comte, continua Danglars, M. de Morcerf est mon ami, ou plutôt ma connaissance depuis trente ans; moi, vous savez que je fais bon marché de mes armoiries, attendu que je n'ai jamais oublié d'où je suis parti. 

—C'est la preuve d'une grande humilité ou d'un grand orgueil, dit Monte-Cristo. 

—Eh bien, quand j'étais petit commis, moi, Morcerf était simple pêcheur. 

—Et alors on l'appelait? 

—Fernand. 

—Tout court? 

—Fernand Mondego. 

—Vous en êtes sûr? 

—Pardieu! il m'a vendu assez de poisson pour que je le connaisse. 

—Alors, pourquoi lui donniez-vous votre fille? 

—Parce que Fernand et Danglars étant deux parvenus, tous deux anoblis, tous deux enrichis, se valent au fond, sauf certaines choses, cependant, qu'on a dites de lui et qu'on n'a jamais dites de moi. 

—Quoi donc? 

—Rien. 

—Ah! oui, je comprends; ce que vous me dites là me rafraîchit la mémoire à propos du nom de Fernand Mondego; j'ai entendu prononcer ce nom-là en Grèce. 

—À propos de l'affaire d'Ali-Pacha? 

—Justement. 

—Voilà le mystère, reprit Danglars, et j'avoue que j'eusse donné bien des choses pour le découvrir. 

—Ce n'était pas difficile, si vous en aviez eu grande envie. 

—Comment cela? 

—Sans doute, vous avez bien quelque correspondant en Grèce? 

—Pardieu! 

—À Janina? 

—J'en ai partout\dots. 

—Eh bien, écrivez à votre correspondant de Janina, et demandez-lui quel rôle a joué dans la catastrophe d'Ali-Tebelin un Français nommé Fernand. 

—Vous avez raison! s'écria Danglars en se levant vivement, j'écrirai aujourd'hui même! 

—Faites. 

—Je vais le faire.  

—Et si vous avez quelque nouvelle bien scandaleuse\dots. 

—Je vous la communiquerai. 

—Vous me ferez plaisir.» 

Danglars s'élança hors de l'appartement, et ne fit qu'un bond jusqu'à sa voiture. 