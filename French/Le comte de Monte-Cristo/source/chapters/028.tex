\chapter{Les registres des prisons}

\lettrine{L}{e} lendemain du jour où s'était passée, sur la route de Bellegarde à Beaucaire, la scène que nous venons de raconter, un homme de trente à trente-deux ans, vêtu d'un frac bleu barbeau, d'un pantalon de nankin et d'un gilet blanc, ayant à la fois la tournure et l'accent britanniques, se présenta chez le maire de Marseille.

«Monsieur, lui dit-il, je suis le premier commis de la maison Thomson et French de Rome. Nous sommes depuis dix ans en relations avec la maison Morrel et fils de Marseille. Nous avons une centaine de mille francs à peu près engagés dans ces relations, et nous ne sommes pas sans inquiétudes, attendu que l'on dit que la maison menace ruine: j'arrive donc tout exprès de Rome pour vous demander des renseignements sur cette maison.

—Monsieur, répondit le maire, je sais effectivement que depuis quatre ou cinq ans le malheur semble poursuivre M. Morrel: il a successivement perdu quatre ou cinq bâtiments, essuyé trois ou quatre banqueroutes; mais il ne m'appartient pas, quoique son créancier moi-même pour une dizaine de mille francs, de donner aucun renseignement sur l'état de sa fortune. Demandez-moi comme maire ce que je pense de M. Morrel, et je vous répondrai que c'est un homme probe jusqu'à la rigidité, et qui jusqu'à présent a rempli tous ses engagements avec une parfaite exactitude. Voilà tout ce que je puis vous dire, monsieur; si vous voulez en savoir davantage, adressez-vous à M. de Boville, inspecteur des prisons, rue de Noailles, nº 15; il a, je crois, deux cent mille francs placés dans la maison Morrel, et s'il y a réellement quelque chose à craindre, comme cette somme est plus considérable que la mienne, vous le trouverez probablement sur ce point mieux renseigné que moi.»

L'Anglais parut apprécier cette suprême délicatesse, salua, sortit et s'achemina de ce pas particulier aux fils de la Grande-Bretagne vers la rue indiquée.

M. de Boville était dans son cabinet. En l'apercevant, l'Anglais fit un mouvement de surprise qui semblait indiquer que ce n'était point la première fois qu'il se trouvait devant celui auquel il venait faire une visite. Quand à M. de Boville, il était si désespéré, qu'il était évident que toutes les facultés de son esprit, absorbées dans la pensée qui l'occupait en ce moment, ne laissaient ni à sa mémoire ni à son imagination le loisir de s'égarer dans le passé.

L'Anglais, avec le flegme de sa nation, lui posa à peu près dans les mêmes termes la même question qu'il venait de poser au maire de Marseille.

«Oh! monsieur, s'écria M. de Boville, vos craintes sont malheureusement on ne peut plus fondées, et vous voyez un homme désespéré. J'avais deux cent mille francs placés dans la maison Morrel: ces deux cent mille francs étaient la dot de ma fille que je comptais marier dans quinze jours; ces deux cent mille francs étaient remboursables, cent mille le 15 de ce mois-ci, cent mille le 15 du mois prochain. J'avais donné avis à M. Morrel du désir que j'avais que ce remboursement fût fait exactement, et voilà qu'il est venu ici, monsieur, il y a à peine une demi-heure, pour me dire que si son bâtiment le \textit{Pharaon} n'était pas rentré d'ici au 15, il se trouverait dans l'impossibilité de me faire ce paiement.

—Mais, dit l'Anglais, cela ressemble fort à un atermoiement.

—Dites monsieur, que cela ressemble à une banqueroute!» s'écria M. de Boville désespéré.

L'Anglais parut réfléchir un instant, puis il dit:

«Ainsi, monsieur, cette créance vous inspire des craintes?

—C'est-à-dire que je la regarde comme perdue.

—Eh bien, moi, je vous l'achète.

—Vous?

—Oui, moi.

—Mais à un rabais énorme, sans doute?

—Non, moyennant deux cent mille francs; notre maison, ajouta l'Anglais en riant, ne fait pas de ces sortes d'affaires.

—Et vous payez?

—Comptant.»

Et l'Anglais tira de sa poche une liasse de billets de banque qui pouvait faire le double de la somme que M. de Boville craignait de perdre. Un éclair de joie passa sur le visage de M. de Boville; mais cependant il fit un effort sur lui-même et dit:

«Monsieur, je dois vous prévenir que, selon toute probabilité, vous n'aurez pas six du cent de cette somme.

—Cela ne me regarde pas, répondit l'Anglais; cela regarde la maison Thomson et French, au nom de laquelle j'agis. Peut-être a-t-elle intérêt à hâter la ruine d'une maison rivale. Mais ce que je sais, monsieur, c'est que je suis prêt à vous compter cette somme contre le transport que vous m'en ferez; seulement je demanderai un droit de courtage.

—Comment, monsieur, c'est trop juste! s'écria M. de Boville. La commission est ordinairement de un et demi: voulez-vous deux? voulez-vous trois? voulez-vous cinq? voulez-vous plus, enfin? Parlez?

—Monsieur, reprit l'Anglais en riant, je suis comme ma maison, je ne fais pas de ces sortes d'affaires; non: mon droit de courtage est de tout autre nature.

—Parlez donc, monsieur, je vous écoute.

—Vous êtes inspecteur des prisons?

—Depuis plus de quatorze ans.

—Vous tenez des registres d'entrée et de sortie?

—Sans doute.

—À ces registres doivent être jointes des notes relatives aux prisonniers?

—Chaque prisonnier a son dossier.

—Eh bien, monsieur, j'ai été élevé à Rome par un pauvre diable d'abbé qui a disparu tout à coup. J'ai appris, depuis, qu'il avait été détenu au château d'If, et je voudrais avoir quelques détails sur sa mort.

—Comment le nommiez-vous?

—L'abbé Faria.

—Oh! je me le rappelle parfaitement! s'écria M. de Boville, il était fou.

—On le disait.

—Oh! il l'était bien certainement.

—C'est possible; et quel était son genre de folie?

—Il prétendait avoir la connaissance d'un trésor immense, et offrait des sommes folles au gouvernement si on voulait le mettre en liberté.

—Pauvre diable! et il est mort?

—Oui, monsieur, il y a cinq ou six mois à peu près, en février dernier.

—Vous avez une heureuse mémoire, monsieur, pour vous rappeler ainsi les dates.

—Je me rappelle celle-ci, parce que la mort du pauvre diable fut accompagnée d'une circonstance singulière.

—Peut on connaître cette circonstance? demanda l'Anglais avec une expression de curiosité qu'un profond observateur eût été étonné de trouver sur son flegmatique visage.

—Oh! mon Dieu! oui, monsieur: le cachot de l'abbé était éloigné de quarante-cinq à cinquante pieds à peu près de celui d'un ancien agent bonapartiste, un de ceux qui avaient le plus contribué au retour de l'usurpateur en 1815, homme très résolu et très dangereux.

—Vraiment? dit l'Anglais.

—Oui, répondit M. de Boville; j'ai eu l'occasion moi-même de voir cet homme en 1816 ou 1817, et l'on ne descendait dans son cachot qu'avec un piquet de soldats: cet homme m'a fait une profonde impression, et je n'oublierai jamais son visage.»

L'Anglais sourit imperceptiblement.

«Et vous dites donc, monsieur, reprit-il, que les deux cachots\dots.

—Étaient séparés par une distance de cinquante pieds; mais il paraît que cet Edmond Dantès\dots.

—Cet homme dangereux s'appelait\dots.

—Edmond Dantès. Oui, monsieur; il paraît que cet Edmond Dantès s'était procuré des outils ou en avait fabriqué, car on trouva un couloir à l'aide duquel les prisonniers communiquaient.

—Ce couloir avait sans doute été pratiqué dans un but d'évasion?

—Justement; mais malheureusement pour les prisonniers, l'abbé Faria fut atteint d'une attaque de catalepsie et mourut.

—Je comprends; cela dut arrêter court les projets d'évasion.

—Pour le mort, oui, répondit M. de Boville, mais pas pour le vivant; au contraire, ce Dantès y vit un moyen de hâter sa fuite; il pensait sans doute que les prisonniers morts au château d'If étaient enterrés dans un cimetière ordinaire; il transporta le défunt dans sa chambre, prit sa place dans le sac où on l'avait cousu et attendit le moment de l'enterrement.

—C'était un moyen hasardeux et qui indiquait quelque courage, reprit l'Anglais.

—Oh! je vous ai dit, monsieur, que c'était un homme fort dangereux; par bonheur il a débarrassé lui-même le gouvernement des craintes qu'il avait à son sujet.

—Comment cela?

—Comment? vous ne comprenez pas?

—Non.

—Le château d'If n'a pas de cimetière; on jette tout simplement les morts à la mer, après leur avoir attaché aux pieds un boulet de trente-six.

—Eh bien? fit l'Anglais, comme s'il avait la conception difficile.

—Eh bien, on lui attacha un boulet de trente-six aux pieds et on le jeta à la mer.

—En vérité? s'écria l'Anglais.

—Oui monsieur, continua l'inspecteur. Vous comprenez quel dut être l'étonnement du fugitif lorsqu'il se sentit précipité du haut en bas des rochers. J'aurais voulu voir sa figure en ce moment-là.

—Ç'eût été difficile.

—N'importe! dit M. de Boville, que la certitude de rentrer dans ses deux cent mille francs mettait de belle humeur, n'importe! je me la représente.»

Et il éclata de rire.

«Et moi aussi», dit l'Anglais.

Et il se mit à rire de son côté, mais comme rient les Anglais, c'est-à-dire du bout des dents.

«Ainsi, continua l'Anglais, qui reprit le premier son sang-froid, ainsi le fugitif fut noyé?

—Bel et bien.

—De sorte que le gouverneur du château fut débarrassé à la fois du furieux et du fou?

—Mais une espèce d'acte a dû être dressé de cet événement? demanda l'Anglais.

—Oui, oui, acte mortuaire. Vous comprenez, les parents de Dantès, s'il en a, pouvaient avoir intérêt à s'assurer s'il était mort ou vivant.

—De sorte que maintenant ils peuvent être tranquilles s'ils héritent de lui. Il est mort et bien mort?

—Oh! mon Dieu, oui. Et on leur délivrera attestation quand ils voudront.

—Ainsi soit-il, dit l'Anglais. Mais revenons aux registres.

—C'est vrai. Cette histoire nous en avait éloignés. Pardon.

—Pardon, de quoi? de l'histoire? Pas du tout, elle m'a paru curieuse.

—Elle l'est en effet. Ainsi, vous désirez voir, monsieur, tout ce qui est relatif à votre pauvre abbé, qui était bien la douceur même, lui?

—Cela me fera plaisir.

—Passez dans mon cabinet et je vais vous montrer cela.»

Et tous deux passèrent dans le cabinet de M. de Boville. Tout y était effectivement dans un ordre parfait: chaque registre était à son numéro, chaque dossier à sa case. L'inspecteur fit asseoir l'Anglais dans son fauteuil, et posa devant lui le registre et le dossier relatifs au château d'If, lui donnant tout le loisir de feuilleter, tandis que lui-même, assis dans un coin, lisait son journal.

L'Anglais trouva facilement le dossier relatif à l'abbé Faria; mais il paraît que l'histoire que lui avait racontée M. de Boville l'avait vivement intéressé, car après avoir pris connaissance de ces premières pièces, il continua de feuilleter jusqu'à ce qu'il fût arrivé à la liasse d'Edmond Dantès. Là, il retrouva chaque chose à sa place: dénonciation, interrogatoire, pétition de Morrel, apostille de M. de Villefort. Il plia tout doucement la dénonciation, la mit dans sa poche, lut l'interrogatoire, et vit que le nom de Noirtier n'y était pas prononcé, parcourut la demande en date du 10 avril 1815, dans laquelle Morrel, d'après le conseil du substitut, exagérait dans une excellente intention, puisque Napoléon régnait alors, les services que Dantès avait rendus à la cause impériale, services que le certificat de Villefort rendait incontestables. Alors, il comprit tout. Cette demande à Napoléon, gardée par Villefort, était devenue sous la seconde Restauration une arme terrible entre les mains du procureur du roi. Il ne s'étonna donc plus en feuilletant le registre, de cette note mise en accolade en regard de son nom:

\begin{quote}\itshape
Edmond Dantès: Bonapartiste enragé: a pris une part active au retour de l'île d'Elbe. À tenir au plus grand secret et sous la plus stricte surveillance.
\end{quote}

Au-dessous de ces lignes, était écrit d'une autre écriture:

«Vu la note ci-dessus, \textit{rien à faire}.»

Seulement, en comparant l'écriture de l'accolade avec celle du certificat placé au bas de la demande de Morrel, il acquit la certitude que la note de l'accolade était de la même écriture que le certificat, c'est-à-dire tracée par la main de Villefort.

Quant à la note qui accompagnait la note, l'Anglais comprit qu'elle avait dû être consignée par quelque inspecteur qui avait pris un intérêt passager à la situation de Dantès, mais que le renseignement que nous venons de citer avait mis dans l'impossibilité de donner suite à cet intérêt.

Comme nous l'avons dit, l'inspecteur, par discrétion et pour ne pas gêner l'élève de l'abbé Faria dans ses recherches, s'était éloigné et lisait \textit{Le Drapeau blanc}.

Il ne vit donc pas l'Anglais plier et mettre dans sa poche la dénonciation écrite par Danglars sous la tonnelle de la Réserve, et portant le timbre de la poste de Marseille, 27 février, levée de 6 heures du soir.

Mais, il faut le dire, il l'eût vu, qu'il attachait trop peu d'importance à ce papier et trop d'importance à ses deux cent mille francs, pour s'opposer à ce que faisait l'Anglais, si incorrect que cela fût.

«Merci, dit celui-ci en refermant bruyamment le registre. J'ai ce qu'il me faut; maintenant, c'est à moi de tenir ma promesse: faites-moi un simple transport de votre créance; reconnaissez dans ce transport en avoir reçu le montant, et je vais vous compter la somme.»

Et il céda sa place au bureau à M. de Boville, qui s'y assit sans façon et s'empressa de faire le transport demandé, tandis que l'Anglais comptait les billets de banque sur le rebord du casier.




