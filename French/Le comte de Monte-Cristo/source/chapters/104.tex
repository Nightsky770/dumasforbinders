\chapter{La signature Danglars}

\lettrine{L}{e} jour du lendemain se leva triste et nuageux. 

\zz
Les ensevelisseurs avaient pendant la nuit accompli leur funèbre office, et cousu le corps déposé sur le lit dans le suaire qui drape lugubrement les trépassés en leur prêtant, quelque chose qu'on dise de l'égalité devant la mort, un dernier témoignage du luxe qu'ils aimaient pendant leur vie. 

Ce suaire n'était autre chose qu'une pièce de magnifique batiste que la jeune fille avait achetée quinze jours auparavant. 

Dans la soirée, des hommes appelés à cet effet avaient transporté Noirtier de la chambre de Valentine dans la sienne, et, contre toute attente, le vieillard n'avait fait aucune difficulté de s'éloigner du corps de son enfant. 

L'abbé Busoni avait veillé jusqu'au jour, et, au jour, il s'était retiré chez lui, sans appeler personne. 

Vers huit heures du matin, d'Avrigny était revenu; il avait rencontré Villefort qui passait chez Noirtier, et il l'avait accompagné pour savoir comment le vieillard avait passé la nuit. 

Ils le trouvèrent dans le grand fauteuil qui lui servait de lit, reposant d'un sommeil doux et presque souriant. 

Tous deux s'arrêtèrent étonnés sur le seuil. 

«Voyez, dit d'Avrigny à Villefort, qui regardait son père endormi; voyez, la nature sait calmer les plus vives douleurs; certes, on ne dira pas que M. Noirtier n'aimait pas sa petite-fille; il dort cependant. 

—Oui, et vous avez raison, répondit Villefort avec surprise; il dort, et c'est bien étrange, car la moindre contrariété le tient éveillé des nuits entières. 

—La douleur l'a terrassé», répliqua d'Avrigny. 

Et tous deux regagnèrent pensifs le cabinet du procureur du roi. 

«Tenez, moi, je n'ai pas dormi, dit Villefort en montrant à d'Avrigny son lit intact; la douleur ne me terrasse pas, moi, il y a deux nuits que je ne me suis couché; mais, en échange, voyez mon bureau; ai-je écrit, mon Dieu! pendant ces deux jours et ces deux nuits!\dots ai-je fouillé ce dossier, ai-je annoté cet acte d'accusation de l'assassin Benedetto!\dots Ô travail, travail! ma passion, ma joie, ma rage, c'est à toi de terrasser toutes mes douleurs!» 

Et il serra convulsivement la main de d'Avrigny. 

«Avez-vous besoin de moi? demanda le docteur. 

—Non, dit Villefort; seulement revenez à onze heures, je vous prie; c'est à midi qu'a lieu\dots le départ\dots Mon Dieu! ma pauvre enfant! ma pauvre enfant!» 

Et le procureur du roi, redevenant homme, leva les yeux au ciel et poussa un soupir. 

«Vous tiendrez-vous donc au salon de réception? 

—Non, j'ai un cousin qui se charge de ce triste honneur. Moi, je travaillerai, docteur; quand je travaille, tout disparaît.» 

En effet, le docteur n'était point à la porte que déjà le procureur du roi s'était remis au travail. 

Sur le perron, d'Avrigny rencontra ce parent dont lui avait parlé Villefort, personnage insignifiant dans cette histoire comme dans la famille, un de ces êtres voués en naissant à jouer le rôle d'utilité dans le monde. 

Il était ponctuel, vêtu de noir, avait un crêpe au bras, et s'était rendu chez son cousin avec une figure qu'il s'était faite, qu'il comptait garder tant que besoin serait, et quitter ensuite. 

À onze heures, les voitures funèbres roulèrent sur le pavé de la cour, et la rue du Faubourg-Saint-Honoré s'emplit des murmures de la foule, également avide des joies ou du deuil des riches, et qui court à un enterrement pompeux avec la même hâte qu'à un mariage de duchesse. 

Peu à peu le salon mortuaire s'emplit et l'on vit arriver d'abord une partie de nos anciennes connaissances, c'est-à-dire Debray, Château-Renaud, Beauchamp, puis toutes les illustrations du parquet, de la littérature et de l'armée; car M. de Villefort occupait moins encore par sa position sociale que par son mérite personnel, un des premiers rangs dans le monde parisien. 

Le cousin se tenait à la porte et faisait entrer tout le monde, et c'était pour les indifférents un grand soulagement, il faut le dire, que de voir là une figure indifférente qui n'exigeait point des conviés une physionomie menteuse ou de fausses larmes, comme eussent fait un père, un frère ou un fiancé. 

Ceux qui se connaissaient s'appelaient du regard et se réunissaient en groupes. 

Un de ces groupes était composé de Debray, de Château-Renaud et de Beauchamp. 

«Pauvre jeune fille! dit Debray, payant, comme chacun au reste le faisait malgré soi, un tribut à ce douloureux événement; pauvre jeune fille! si riche, si belle! Eussiez-vous pensé cela, Château-Renaud, quand nous vînmes, il y a combien?\dots trois semaines ou un mois tout au plus, pour signer ce contrat qui ne fut pas signé? 

—Ma foi, non, dit Château-Renaud. 

—La connaissiez-vous? 

—J'avais causé une fois ou deux avec elle au bal de Mme de Morcerf, elle m'avait paru charmante quoique d'un esprit un peu mélancolique. Où est la belle-mère? savez-vous? 

—Elle est allée passer la journée avec la femme de ce digne monsieur qui nous reçoit. 

—Qu'est-ce que c'est que ça? 

—Qui ça? 

—Le monsieur qui nous reçoit. Un député? 

—Non, dit Beauchamp; je suis condamné à voir nos honorables tous les jours, et sa tête m'est inconnue. 

—Avez-vous parlé de cette mort dans votre journal? 

—L'article n'est pas de moi, mais on en a parlé; je doute même qu'il soit agréable à M. de Villefort. Il est dit, je crois, que si quatre morts successives avaient eu lieu autre part que dans la maison de M. le procureur du roi, M. le procureur du roi s'en fût certes plus ému. 

—Au reste, dit Château-Renaud, le docteur d'Avrigny, qui est le médecin de ma mère, le prétend fort désespéré. 

—Mais qui cherchez-vous donc, Debray? 

—Je cherche M. de Monte-Cristo, répondit le jeune homme. 

—Je l'ai rencontré sur le boulevard en venant ici. Je le crois sur son départ, il allait chez son banquier, dit Beauchamp. 

—Chez son banquier? Son banquier, n'est-ce pas Danglars? demanda Château-Renaud à Debray. 

—Je crois que oui, répondit le secrétaire intime avec un léger trouble; mais M. de Monte-Cristo n'est pas le seul qui manque ici. Je ne vois pas Morrel. 

—Morrel! est-ce qu'il les connaissait? demanda Château-Renaud. 

—Je crois qu'il avait été présenté à Mme de Villefort seulement. 

—N'importe, il aurait dû venir, dit Debray; de quoi causera-t-il, ce soir? cet enterrement, c'est la nouvelle de la journée; mais, chut, taisons-nous, voici M. le ministre de la Justice et des Cultes, il va se croire obligé de faire son petit \textit{speech} au cousin larmoyant.» 

Et les trois jeunes gens se rapprochèrent de la porte pour entendre le petit \textit{speech} de M. le ministre de la Justice et des Cultes. 

Beauchamp avait dit vrai; en se rendant à l'invitation mortuaire, il avait rencontré Monte-Cristo, qui, de son côté, se dirigeait vers l'hôtel de Danglars, rue de la Chaussée-d'Antin. 

Le banquier avait, de sa fenêtre, aperçu la voiture du comte entrant dans la cour, et il était venu au-devant de lui avec un visage attristé, mais affable. 

«Eh bien, comte, dit-il en tendant la main à Monte-Cristo, vous venez me faire vos compliments de condoléance. En vérité, le malheur est dans ma maison; c'est au point que, lorsque je vous ai aperçu, je m'interrogeais moi-même pour savoir si je n'avais pas souhaité malheur à ces pauvres Morcerf, ce qui eût justifié le proverbe: Qui mal veut, mal lui arrive. Eh bien, sur ma parole, non, je ne souhaitais pas de mal à Morcerf; il était peut-être un peu orgueilleux pour un homme parti de rien, comme moi, se devant tout à lui-même, comme moi, mais chacun a ses défauts. Ah, tenez-vous bien, comte, les gens de notre génération\dots Mais, pardon, vous n'êtes pas de notre génération, vous, vous êtes un jeune homme\dots Les gens de notre génération ne sont point heureux cette année: témoin notre puritain de procureur du roi, témoin Villefort, qui vient encore de perdre sa fille. Ainsi, récapitulez: Villefort, comme nous disions, perdant toute sa famille d'une façon étrange; Morcerf déshonoré et tué; moi, couvert de ridicule par la scélératesse de ce Benedetto, et puis\dots 

—Puis, quoi? demanda le comte. 

—Hélas! vous l'ignorez donc? 

—Quelque nouveau malheur? 

—Ma fille\dots 

—Mlle Danglars? 

—Eugénie nous quitte. 

—Oh! mon Dieu! que me dites-vous là! 

—La vérité, mon cher comte. Mon Dieu! que vous êtes heureux de n'avoir ni femme ni enfant, vous! 

—Vous trouvez? 

—Ah! mon Dieu! 

—Et vous dites que Mlle Eugénie\dots 

—Elle n'a pu supporter l'affront que nous a fait ce misérable, et m'a demandé la permission de voyager. 

—Et elle est partie? 

—L'autre nuit. 

—Avec Mme Danglars? 

—Non, avec une parente\dots Mais nous ne la perdons pas moins, cette chère Eugénie; car je doute qu'avec le caractère que je lui connais, elle consente jamais à revenir en France! 

—Que voulez-vous, mon cher baron, dit Monte-Cristo, chagrins de famille, chagrins qui seraient écrasants pour un pauvre diable dont l'enfant serait toute la fortune, mais supportables pour un millionnaire. Les philosophes ont beau dire, les hommes pratiques leur donneront toujours un démenti là-dessus: l'argent console de bien des choses; et vous, vous devez être plus vite consolé que qui que ce soit, si vous admettez la vertu de ce baume souverain: vous, le roi de la finance, le point d'intersection de tous les pouvoirs.» 

Danglars lança un coup d'œil oblique au comte, pour voir s'il raillait ou s'il parlait sérieusement. 

«Oui, dit-il, le fait est que si la fortune console, je dois être consolé: je suis riche. 

—Si riche, mon cher baron, que votre fortune ressemble aux Pyramides; voulût-on les démolir, on n'oserait; osât-on, on ne pourrait.» 

Danglars sourit de cette confiante bonhomie du comte. 

«Cela me rappelle, dit-il, que lorsque vous êtes entré, j'étais en train de faire cinq petits bons; j'en avais déjà signé deux; voulez-vous me permettre de faire les trois autres? 

—Faites, mon cher baron, faites.» 

Il y eut un instant de silence, pendant lequel on entendit crier la plume du banquier, tandis que Monte-Cristo regardait les moulures dorées au plafond. 

«Des bons d'Espagne, dit Monte-Cristo, des bons d'Haïti, des bons de Naples? 

—Non, dit Danglars en riant de son rire suffisant, des bons au porteur, des bons sur la Banque de France. Tenez, ajouta-t-il, monsieur le comte, vous qui êtes l'empereur de la finance, comme j'en suis le roi, avez-vous vu beaucoup de chiffons de papier de cette grandeur-là valoir chacun un million?» 

Monte-Cristo prit dans sa main, comme pour les peser, les cinq chiffons de papier que lui présentait orgueilleusement Danglars, et lut: 


\begin{mail}{}{}
Plaise à M. le Régent de la Banque de faire payer à mon ordre, et sur les fonds déposés par moi, la somme d'un million, valeur en compte. 

\closeletter{Baron Danglars.}
\end{mail}


—Un, deux, trois, quatre, cinq, fit Monte-Cristo; cinq millions! peste! comme vous y allez, seigneur Crésus! 

—Voilà comme je fais les affaires, moi, dit Danglars. 

—C'est merveilleux, si surtout, comme je n'en doute pas, cette somme est payée comptant. 

—Elle le sera, dit Danglars. 

—C'est beau d'avoir un pareil crédit; en vérité il n'y a qu'en France qu'on voie ces choses-là: cinq chiffons de papier valant cinq millions; et il faut le voir pour le croire. 

—Vous en doutez? 

—Non. 

—Vous dites cela avec un accent\dots Tenez, donnez-vous-en le plaisir: conduisez mon commis à la banque, et vous l'en verrez sortir avec des bons sur le trésor pour la même somme. 

—Non, dit Monte-Cristo pliant les cinq billets, ma foi non, la chose est trop curieuse, et j'en ferai l'expérience moi-même. Mon crédit chez vous était de six millions, j'ai pris neuf cent mille francs, c'est cinq millions cent mille francs que vous restez me devoir. Je prends vos cinq chiffons de papier que je tiens pour bons à la seule vue de votre signature, et voici un reçu général de six millions qui régularise notre compte. Je l'avais préparé d'avance, car il faut vous dire que j'ai fort besoin d'argent aujourd'hui.» 

Et d'une main Monte-Cristo mit les cinq billets dans sa poche, tandis que de l'autre il tendait son reçu au banquier. 

La foudre tombant aux pieds de Danglars ne l'eût pas écrasé d'une terreur plus grande. 

«Quoi! balbutia-t-il, quoi! monsieur le comte, vous prenez cet argent? Mais, pardon, pardon, c'est de l'argent que je dois aux hospices, un dépôt, et j'avais promis de payer ce matin. 

—Ah! dit Monte-Cristo, c'est différent. Je ne tiens pas précisément à ces cinq billets, payez-moi en autres valeurs; c'était par curiosité que j'avais pris celles-ci, afin de pouvoir dire de par le monde que, sans avis aucun, sans me demander cinq minutes de délai, la maison Danglars m'avait payé cinq millions comptant! c'eût été remarquable! Mais voici vos valeurs; je vous le répète, donnez-m'en d'autres.» 

Et il tendait les cinq effets à Danglars qui, livide, allongea d'abord la main, ainsi que le vautour allonge la griffe par les barreaux de sa cage pour retenir la chair qu'on lui enlève. 

Tout à coup il se ravisa, fit un effort violent et se contint. 

Puis on le vit sourire, arrondir peu à peu les traits de son visage bouleversé. 

«Au fait, dit-il, votre reçu, c'est de l'argent. 

—Oh! mon Dieu, oui! et si vous étiez à Rome, sur mon reçu, la maison Thomson et French ne ferait pas plus de difficulté de vous payer que vous n'en avez fait vous-même. 

—Pardon, monsieur le comte, pardon. 

—Je puis donc garder cet argent? 

—Oui, dit Danglars en essuyant la sueur qui perlait à la racine de ses cheveux, gardez, gardez.» 

Monte-Cristo remit les cinq billets dans sa poche avec cet intraduisible mouvement de physionomie qui veut dire: 

«Dame! réfléchissez; si vous vous repentez, il est encore temps. 

—Non, dit Danglars, non; décidément, gardez mes signatures. Mais, vous le savez, rien n'est formaliste comme un homme d'argent; je destinais cet argent aux hospices et j'eusse cru les voler en ne leur donnant pas précisément celui-là, comme si un écu n'en valait pas un autre. Excusez!» 

Et il se mit à rire bruyamment, mais des nerfs. 

«J'excuse, répondit gracieusement Monte-Cristo, et j'empoche.» 

Et il plaça les bons dans son portefeuille. 

«Mais, dit Danglars, nous avons une somme de cent mille francs? 

—Oh! bagatelle, dit Monte-Cristo. L'agio doit monter à peu près à cette somme; gardez-la, et nous serons quittes. 

—Comte, dit Danglars, parlez-vous sérieusement? 

—Je ne ris jamais avec les banquiers», répliqua Monte-Cristo avec un sérieux qui frisait l'impertinence. 

Et il s'achemina vers la porte, juste au moment où le valet de chambre annonçait: 

«M. de Boville, receveur général des hospices. 

—Ma foi, dit Monte-Cristo, il paraît que je suis arrivé à temps pour jouir de vos signatures, on se les dispute.» 

Danglars pâlit une seconde fois, et se hâta de prendre congé du comte. 

Le comte de Monte-Cristo échangea un cérémonieux salut avec M. de Boville, qui se tenait debout dans le salon d'attente, et qui, M. de Monte-Cristo passé, fut immédiatement introduit dans le cabinet de M. Danglars. 

On eût pu voir le visage si sérieux du comte s'illuminer d'un éphémère sourire à l'aspect du portefeuille que tenait à la main M. le receveur des hospices. 

À la porte, il retrouva sa voiture, et se fit conduire sur-le-champ à la Banque. 

Pendant ce temps, Danglars, comprimant toute émotion, venait à la rencontre du receveur général. 

Il va sans dire que le sourire et la gracieuseté étaient stéréotypés sur ses lèvres. 

«Bonjour, dit-il, mon cher créancier, car je gagerais que c'est le créancier qui m'arrive. 

—Vous avez deviné juste, monsieur le baron, dit M. de Boville, les hospices se présentent à vous dans ma personne; les veuves et les orphelins viennent par mes mains vous demander une aumône de cinq millions. 

—Et l'on dit que les orphelins sont à plaindre! dit Danglars en prolongeant la plaisanterie; pauvres enfants! 

—Me voici donc venu en leur nom, dit M. de Boville. Vous avez dû recevoir ma lettre hier? 

—Oui. 

—Me voici avec mon reçu. 

—Mon cher monsieur de Boville, dit Danglars, vos veuves et vos orphelins auront, si vous le voulez bien, la bonté d'attendre vingt-quatre heures, attendu que M. de Monte-Cristo, que vous venez de voir sortir d'ici\dots Vous l'avez vu, n'est-ce pas? 

—Oui; eh bien? 

—Eh bien, M. de Monte-Cristo emportait leur cinq millions! 

—Comment cela? 

—Le comte avait un crédit illimité sur moi, crédit ouvert par la maison Thomson et French, de Rome. Il est venu me demander une somme de cinq millions d'un seul coup; je lui ai donné un bon sur la Banque: c'est là que sont déposés mes fonds; et vous comprenez, je craindrais, en retirant des mains de M. le régent dix millions le même jour, que cela ne lui parût bien étrange. 

«En deux jours, ajouta Danglars en souriant, je ne dis pas. 

—Allons donc! s'écria M. de Boville avec le ton de la plus complète incrédulité; cinq millions à ce monsieur qui sortait tout à l'heure, et qui m'a salué en sortant comme si je le connaissais? 

—Peut-être vous connaît-il sans que vous le connaissiez, vous. M. de Monte-Cristo connaît tout le monde. 

—Cinq millions! 

—Voilà son reçu. Faites comme saint Thomas: voyez et touchez.» 

M. de Boville prit le papier que lui présentait Danglars, et lut: 

«Reçu de M. le baron Danglars la somme de cinq millions cent mille francs, dont il se remboursera à volonté sur la maison Thomson et French, de Rome.» 

«C'est ma foi vrai! dit celui-ci. 

—Connaissez-vous la maison Thomson et French? 

—Oui, dit M. de Boville, j'ai fait autrefois une affaire de deux cent mille francs avec elle; mais je n'en ai pas entendu parler depuis. 

—C'est une des meilleures maisons d'Europe, dit Danglars en rejetant négligemment sur son bureau le reçu qu'il venait de prendre des mains de M. de Boville. 

—Et il avait comme cela cinq millions, rien que sur vous? Ah çà! mais c'est donc un nabab que ce comte de Monte-Cristo? 

—Ma foi! je ne sais pas ce que c'est, mais il avait trois crédits illimités: un sur moi, un sur Rothschild, un sur Laffitte, et, ajouta négligemment Danglars, comme vous voyez, il m'a donné la préférence en me laissant cent mille francs pour l'agio.» 

M. de Boville donna tous les signes de la plus grande admiration. 

«Il faudra que je l'aille visiter, dit-il, et que j'obtienne quelque fondation pieuse pour nous. 

—Oh! c'est comme si vous la teniez; ses aumônes seules montent à plus de vingt mille francs par mois. 

—C'est magnifique; d'ailleurs, je lui citerai l'exemple de Mme de Morcerf et de son fils. 

—Quel exemple? 

—Ils ont donné toute leur fortune aux hospices. 

—Quelle fortune? 

—Leur fortune, celle du général de Morcerf, du défunt. 

—Et à quel propos? 

—À propos qu'ils ne voulaient pas d'un bien si misérablement acquis. 

—De quoi vont-ils vivre? 

—La mère se retire en province et le fils s'engage. 

—Tiens, tiens, dit Danglars, en voilà des scrupules! 

—J'ai fait enregistrer l'acte de donation hier. 

—Et combien possédaient-ils? 

—Oh! pas grand-chose: douze à treize cent mille francs. Mais revenons à nos millions. 

—Volontiers, dit Danglars le plus naturellement du monde; vous êtes donc bien pressé de cet argent? 

—Mais oui; la vérification de nos caisses se fait demain. 

—Demain! que ne disiez-vous cela tout de suite? Mais c'est un siècle, demain! À quelle heure cette vérification? 

—À deux heures. 

—Envoyez à midi, dit Danglars avec son sourire. 

M. de Boville ne répondait pas grand-chose; il faisait oui de la tête et remuait son portefeuille. 

—Eh! mais j'y songe, dit Danglars, faites mieux. 

—Que voulez-vous que je fasse? 

—Le reçu de M. de Monte-Cristo vaut de l'argent; passez ce reçu chez Rothschild ou chez Laffitte; ils vous le prendront à l'instant même. 

—Quoique remboursable sur Rome? 

—Certainement; il vous en coûtera seulement un escompte de cinq à six mille francs. 

Le receveur fit un bond en arrière. 

«Ma foi! non, j'aime mieux attendre à demain. Comme vous y allez! 

—J'ai cru un instant, pardonnez-moi, dit Danglars avec une suprême impudence, j'ai cru que vous aviez un petit déficit à combler. 

—Ah! fit le receveur. 

—Écoutez, cela s'est vu, et dans ce cas on fait un sacrifice. 

—Dieu merci! non, dit M. de Boville. 

—Alors, à demain; mais sans faute? 

—Ah çà! mais, vous riez! Envoyez à midi, et la Banque sera prévenue. 

—Je viendrai moi-même. 

—Mieux encore, puisque cela me procurera le plaisir de vous voir.» 

Ils se serrèrent la main. 

«À propos, dit M. de Boville, n'allez-vous donc point à l'enterrement de cette pauvre Mlle de Villefort, que j'ai rencontré sur le boulevard? 

—Non, dit le banquier, je suis encore un peu ridicule depuis l'affaire de Benedetto, et je fais un plongeon. 

—Bah! vous avez tort; est-ce qu'il y a de votre faute dans tout cela? 

—Écoutez, mon cher receveur, quand on porte un nom sans tache comme le mien, on est susceptible. 

—Tout le monde vous plaint, soyez-en persuadé, et, surtout, tout le monde plaint mademoiselle votre fille. 

—Pauvre Eugénie! fit Danglars avec un profond soupir. Vous savez qu'elle entre en religion, monsieur? 

—Non. 

—Hélas! ce n'est que malheureusement trop vrai. Le lendemain de l'événement, elle s'est décidée à partir avec une religieuse de ses amies; elle va chercher un couvent bien sévère en Italie ou en Espagne. 

—Oh! c'est terrible!» 

Et M. de Boville se retira sur cette exclamation en faisant au père mille compliments de condoléance. Mais il ne fut pas plus tôt dehors, que Danglars, avec une énergie de geste que comprendront ceux-là seulement qui ont vu représenter \textit{Robert Macaire}, par Frédérick, s'écria: 

«Imbécile!» 

Et serrant la quittance de Monte-Cristo dans un petit portefeuille: 

«Viens à midi, ajouta-t-il, à midi, je serai loin.» 

Puis il s'enferma à double tour, vida tous les tiroirs de sa caisse, réunit une cinquantaine de mille francs en billets de banque, brûla différents papiers, en mit d'autres en évidence, et commença d'écrire une lettre qu'il cacheta, et sur laquelle il mit pour suscription: 

«À madame la baronne Danglars.» 

«Ce soir, murmura-t-il, je la placerai moi-même sur sa toilette.» 

Puis, tirant un passeport de son tiroir. 

«Bon, dit-il, il est encore valable pour deux mois.» 