\chapter{La limonade}

\lettrine{E}{n} effet, Morrel était bien heureux. 

\zz
M. Noirtier venait de l'envoyer chercher, et il avait si grande hâte de savoir pour quelle cause, qu'il n'avait pas pris de cabriolet, se fiant bien plus à ses deux jambes qu'aux jambes d'un cheval de place; il était donc parti tout courant de la rue Meslay et se rendait au faubourg Saint-Honoré. 

Morrel marchait au pas gymnastique, et le pauvre Barrois le suivait de son mieux. Morrel avait trente et un ans, Barrois en avait soixante; Morrel était ivre d'amour, Barrois était altéré par la grande chaleur. Ces deux hommes, ainsi divisés d'intérêts et d'âge, ressemblaient aux deux lignes que forme un triangle: écartées par la base, elles se rejoignent au sommet. 

Le sommet, c'était Noirtier, lequel avait envoyé chercher Morrel en lui recommandant de faire diligence, recommandation que Morrel suivait à la lettre, au grand désespoir de Barrois. 

En arrivant, Morrel n'était pas même essoufflé: l'amour donne des ailes, mais Barrois, qui depuis longtemps n'était plus amoureux, Barrois était en nage. 

Le vieux serviteur fit entrer Morrel par la porte particulière, ferma la porte du cabinet, et bientôt un froissement de robe sur le parquet annonça la visite de Valentine. 

Valentine était belle à ravir sous ses vêtements de deuil. 

Le rêve devenait si doux que Morrel se fût presque passé de converser avec Noirtier; mais le fauteuil du vieillard roula bientôt sur le parquet, et il entra. 

Noirtier accueillit par un regard bienveillant les remerciements que Morrel lui prodiguait pour cette merveilleuse intervention qui les avait sauvés, Valentine et lui, du désespoir. Puis le regard de Morrel alla provoquer, sur la nouvelle faveur qui lui était accordée, la jeune fille, qui, timide et assise loin de Morrel, attendait d'être forcée à parler. 

Noirtier la regarda à son tour. 

«Il faut donc que je dise ce dont vous m'avez chargée? demanda-t-elle. 

—Oui, fit Noirtier.  

—Monsieur Morrel, dit alors Valentine au jeune homme qui la dévorait des yeux, mon bon papa Noirtier avait mille choses à vous dire, que depuis trois jours il m'a dites. Aujourd'hui, il vous envoie chercher pour que je vous les répète; je vous les répéterai donc, puisqu'il m'a choisie pour son interprète, sans changer un mot à ses intentions. 

—Oh! j'écoute bien impatiemment, répondit le jeune homme; parlez, mademoiselle, parlez.» 

Valentine baissa les yeux: ce fut un présage qui parut doux à Morrel. Valentine n'était faible que dans le bonheur. 

«Mon père veut quitter cette maison, dit-elle. Barrois s'occupe de lui chercher un appartement convenable. 

—Mais vous, mademoiselle, dit Morrel vous qui êtes si chère et si nécessaire à M. Noirtier? 

—Moi, reprit la jeune fille, je ne quitterai point mon grand-père, c'est chose convenue entre lui et moi. Mon appartement sera près du sien. Ou j'aurai le consentement de M. de Villefort pour aller habiter avec bon papa Noirtier, ou on me le refusera: dans le premier cas, je pars dès à présent; dans le second, j'attends ma majorité, qui arrive dans dix-huit mois. Alors je serai libre, j'aurai une fortune indépendante, et\dots. 

—Et?\dots demanda Morrel. 

—Et, avec l'autorisation de bon papa, je tiendrai la promesse que je vous ai faite.»  

Valentine prononça ces derniers mots si bas, que Morrel n'eût pu les entendre sans l'intérêt qu'il avait à les dévorer. 

«N'est-ce point votre pensée que j'ai exprimée là, bon papa? ajouta Valentine en s'adressant à Noirtier. 

—Oui, fit le vieillard. 

—Une fois chez mon grand-père, ajouta Valentine, M. Morrel pourra me venir voir en présence de ce bon et digne protecteur. Si ce lien que nos cœurs, peut-être ignorants ou capricieux, avaient commencé de former paraît convenable et offre des garanties de bonheur futur à notre expérience (hélas! dit-on, les cœurs enflammés par les obstacles se refroidissent dans la sécurité!) alors M. Morrel pourra me demander à moi-même, je l'attendrai. 

—Oh! s'écria Morrel, tenté de s'agenouiller devant le vieillard comme devant Dieu, devant Valentine comme devant un ange; oh! qu'ai-je donc fait de bien dans ma vie pour mériter tant de bonheur? 

—Jusque-là, continua la jeune fille de sa voix pure et sévère, nous respectons les convenances, la volonté même de nos parents, pourvu que cette volonté ne tende pas à nous séparer toujours; en un mot, et je répète ce mot parce qu'il dit tout, nous attendrons. 

—Et les sacrifices que ce mot impose, monsieur, dit Morrel, je vous jure de les accomplir, non pas avec résignation, mais avec bonheur. 

—Ainsi, continua Valentine avec un regard bien doux au cœur de Maximilien, plus d'imprudences, mon ami, ne compromettez pas celle qui, à partir d'aujourd'hui, se regarde comme destinée à porter purement et dignement votre nom.» 

Morrel appuya sa main sur son cœur. 

Cependant Noirtier les regardait tous deux avec tendresse. Barrois, qui était resté au fond comme un homme à qui l'on n'a rien à cacher, souriait en essuyant les grosses gouttes d'eau qui tombaient de son front chauve. 

«Oh! mon Dieu, comme il a chaud, ce bon Barrois, dit Valentine. 

—Ah! dit Barrois, c'est que j'ai bien couru, allez, mademoiselle; mais M. Morrel, je dois lui rendre cette justice-là, courait encore plus vite que moi.» 

Noirtier indiqua de l'œil un plateau sur lequel étaient servis une carafe de limonade et un verre. Ce qui manquait dans la carafe avait été bu une demi-heure auparavant par Noirtier. 

«Tiens, bon Barrois, dit la jeune fille, prends, car je vois que tu couves des yeux cette carafe entamée. 

—Le fait est, dit Barrois, que je meurs de soif, et que je boirai bien volontiers un verre de limonade à votre santé. 

—Bois donc, dit Valentine, et reviens dans un instant.» 

Barrois emporta le plateau, et à peine était-il dans le corridor, qu'à travers la porte qu'il avait oublié de fermer, on le voyait pencher la tête en arrière pour vider le verre que Valentine avait rempli. 

Valentine et Morrel échangeaient leurs adieux en présence de Noirtier, quand on entendit la sonnette retentir dans l'escalier de Villefort. 

C'était le signal d'une visite. 

Valentine regarda la pendule. 

«Il est midi, dit-elle, c'est aujourd'hui samedi, bon papa, c'est sans doute le docteur.» 

Noirtier fit signe qu'en effet ce devait être lui. 

«Il va venir ici, il faut que M. Morrel s'en aille, n'est-ce pas, bon papa? 

—Oui, répondit le vieillard. Barrois! appela Valentine; Barrois, venez!» 

On entendit la voix du vieux serviteur qui répondait: 

«J'y vais, mademoiselle. 

—Barrois va vous reconduire jusqu'à la porte, dit Valentine à Morrel; et maintenant, rappelez-vous une chose, monsieur l'officier, c'est que mon bon papa vous recommande de ne risquer aucune démarche capable de compromettre notre bonheur. 

—J'ai promis d'attendre, dit Morrel, et j'attendrai.» 

En ce moment, Barrois entra. 

«Qui a sonné? demanda Valentine. 

—Monsieur le docteur d'Avrigny, dit Barrois en chancelant sur ses jambes. 

—Eh bien, qu'avez-vous donc, Barrois?» demanda Valentine. 

Le vieillard ne répondit pas; il regardait son maître avec des yeux effarés, tandis que de sa main crispée il cherchait un appui pour demeurer debout. 

«Mais il va tomber!» s'écria Morrel. 

En effet, le tremblement dont Barrois était saisi augmentait par degrés; les traits du visage, altérés par les mouvements convulsifs des muscles de la face, annonçaient une attaque nerveuse des plus intenses. 

Noirtier, voyant Barrois ainsi troublé, multipliait ses regards dans lesquels se peignaient, intelligibles et palpitantes, toutes les émotions qui agitent le cœur de l'homme. 

Barrois fit quelques pas vers son maître. 

«Ah! mon Dieu! mon Dieu! Seigneur, dit-il, mais qu'ai-je donc?\dots Je souffre\dots je n'y vois plus. Mille pointes de feu me traversent le crâne. Oh! ne me touchez pas, ne me touchez pas!» 

En effet, les yeux devenaient saillants et hagards, et la tête se renversait en arrière, tandis que le reste du corps se raidissait. 

Valentine épouvantée poussa un cri; Morrel la prit dans ses bras comme pour la défendre contre quelque danger inconnu. 

«Monsieur d'Avrigny! monsieur d'Avrigny! cria Valentine d'une voix étouffée, à nous! au secours!» 

Barrois tourna sur lui-même, fit trois pas en arrière, trébucha et vint tomber aux pieds de Noirtier, sur le genou duquel il appuya sa main en criant: 

«Mon maître! mon bon maître!» 

En ce moment M. de Villefort, attiré par les cris, parut sur le seuil de la chambre. 

Morrel lâcha Valentine à moitié évanouie, et se rejetant en arrière, s'enfonça dans l'angle de la chambre et disparut presque derrière un rideau. 

Pâle comme s'il eût vu un serpent se dresser devant lui, il attachait un regard glacé sur le malheureux agonisant. 

Noirtier bouillait d'impatience et de terreur; son âme volait au secours du pauvre vieillard, son ami plutôt que son domestique. On voyait le combat terrible de la vie et de la mort se traduire sur son front par le gonflement des veines et la contraction de quelques muscles restés vivants autour de ses yeux. 

Barrois, la face agitée, les yeux injectés de sang, le cou renversé en arrière, gisait battant le parquet de ses mains, tandis qu'au contraire ses jambes raides semblaient devoir rompre plutôt que plier. 

Une légère écume montait à ses lèvres, et il haletait douloureusement. 

Villefort, stupéfait, demeura un instant les yeux fixés sur ce tableau, qui, dès son entrée dans la chambre, attira ses regards. 

Il n'avait pas vu Morrel. 

Après un instant de contemplation muette pendant lequel on put voir son visage pâlir et ses cheveux se dresser sur sa tête: 

«Docteur! docteur! s'écria-t-il en s'élançant vers la porte, venez! venez! 

—Madame! madame! cria Valentine appelant sa belle-mère en se heurtant aux parois de l'escalier, venez! venez vite et apportez votre flacon de sels! 

—Qu'y a-t-il? demanda la voix métallique et contenue de Mme de Villefort. 

—Oh! venez! venez! 

—Mais où donc est le docteur! criait Villefort; où est-il?» 

Mme de Villefort descendit lentement; on entendait craquer les planches sous ses pieds. D'une main elle tenait le mouchoir avec lequel elle s'essuyait le visage, de l'autre un flacon de sels anglais.  

Son premier regard, en arrivant à la porte, fut pour Noirtier, dont le visage, sauf l'émotion bien naturelle dans une semblable circonstance, annonçait une santé égale; son second coup d'œil rencontra le moribond. 

Elle pâlit, et son œil rebondit pour ainsi dire du serviteur sur le maître. 

«Mais au nom du Ciel, madame, où est le docteur? il est entré chez vous. C'est une apoplexie, vous le voyez bien, avec une saignée on le sauvera. 

—A-t-il mangé depuis peu? demanda Mme de Villefort éludant la question. 

—Madame, dit Valentine, il n'a pas déjeuné, mais il a fort couru ce matin pour faire une commission dont l'avait chargé bon papa. Au retour seulement il a pris un verre de limonade. 

—Ah! fit Mme de Villefort, pourquoi pas du vin? C'est très mauvais, la limonade. 

—La limonade était là sous sa main, dans la carafe de bon papa; le pauvre Barrois avait soif, il a bu ce qu'il a trouvé.» 

Mme de Villefort tressaillit. Noirtier l'enveloppa de son regard profond. 

«Il a le cou si court! dit-elle. 

—Madame, dit Villefort, je vous demande où est M. d'Avrigny; au nom du Ciel, répondez! 

—Il est dans la chambre d'Édouard qui est un peu souffrant», dit Mme de Villefort, qui ne pouvait éluder plus longtemps. 

Villefort s'élança dans l'escalier pour l'aller chercher lui-même. 

«Tenez, dit la jeune femme en donnant son flacon à Valentine, on va le saigner sans doute. Je remonte chez moi, car je ne puis supporter la vue du sang.» 

Et elle suivit son mari. 

Morrel sortit de l'angle sombre où il s'était retiré, et où personne ne l'avait vu, tant la préoccupation était grande. 

«Partez vite, Maximilien, lui dit Valentine, et attendez que je vous appelle. Allez.» 

Morrel consulta Noirtier par un geste. Noirtier, qui avait conservé tout son sang-froid, lui fit signe que oui. 

Il serra la main de Valentine contre son cœur et sortit par le corridor dérobé. 

En même temps Villefort et le docteur rentraient par la porte opposée. 

Barrois commençait à revenir à lui: la crise était passée, sa parole revenait gémissante, et il se soulevait sur un genou. 

D'Avrigny et Villefort portèrent Barrois sur une chaise longue.  

«Qu'ordonnez-vous, docteur? demanda Villefort. 

—Qu'on m'apporte de l'eau et de l'éther. Vous en avez dans la maison? 

—Oui. 

—Qu'on coure me chercher de l'huile de térébenthine et de l'émétique. 

—Allez! dit Villefort. 

—Et maintenant que tout le monde se retire. 

—Moi aussi? demanda timidement Valentine. 

—Oui, mademoiselle, vous surtout», dit rudement le docteur. 

Valentine regarda M. d'Avrigny avec étonnement, embrassa M. Noirtier au front et sortit. 

Derrière elle le docteur ferma la porte d'un air sombre. 

«Tenez, tenez, docteur, le voilà qui revient; ce n'était qu'une attaque sans importance. 

M. d'Avrigny sourit d'un air sombre. 

«Comment vous sentez-vous, Barrois? demanda le docteur. 

—Un peu mieux, monsieur. 

—Pouvez-vous boire ce verre d'eau éthérée? 

—Je vais essayer, mais ne me touchez pas. 

—Pourquoi? 

—Parce qu'il me semble que si vous me touchiez, ne fût-ce que du bout du doigt, l'accès me reprendrait. 

—Buvez.» 

Barrois prit le verre, l'approcha de ses lèvres violettes et le vida à moitié à peu près. 

«Où souffrez-vous? demanda le docteur. 

—Partout; j'éprouve comme d'effroyables crampes. 

—Avez-vous des éblouissements? 

—Oui. 

—Des tintements d'oreille? 

—Affreux. 

—Quand cela vous a-t-il pris? 

—Tout à l'heure. 

—Rapidement? 

—Comme la foudre.  

—Rien hier? rien avant-hier? 

—Rien. 

—Pas de somnolence? pas de pesanteurs? 

—Non. 

—Qu'avez-vous mangé aujourd'hui? 

—Je n'ai rien mangé; j'ai bu seulement un verre de la limonade de monsieur, voilà tout.» 

Et Barrois fit de la tête un signe pour désigner Noirtier qui immobile dans son fauteuil, contemplait cette terrible scène sans en perdre un mouvement, sans laisser échapper une parole. 

«Où est cette limonade? demanda vivement le docteur. 

—Dans la carafe, en bas. 

—Où cela, en bas! 

—Dans la cuisine. 

—Voulez-vous que j'aille la chercher, docteur? demanda Villefort. 

—Non, restez ici, et tâchez de faire boire au malade le reste de ce verre d'eau. 

—Mais cette limonade\dots. 

—J'y vais moi-même.» 

D'Avrigny fit un bond, ouvrit la porte, s'élança dans l'escalier de service et faillit renverser madame de Villefort, qui, elle aussi, descendait à la cuisine. 

Elle poussa un cri. 

D'Avrigny n'y fit même pas attention; emporté par la puissance d'une seule idée, il sauta les trois ou quatre dernières marches, se précipita dans la cuisine, et aperçut le carafon aux trois quarts vide sur un plateau. 

Il fondit dessus comme un aigle sur sa proie. 

Haletant, il remonta au rez-de-chaussée et rentra dans la chambre. Mme de Villefort remontait lentement l'escalier qui conduisait chez elle. 

«Est-ce bien cette carafe qui était ici? demanda d'Avrigny. 

—Oui, monsieur le docteur. 

—Cette limonade est la même que vous avez bue? 

—Je le crois. 

—Quel goût lui avez-vous trouvé? 

—Un goût amer.» 

Le docteur versa quelques gouttes de limonade dans le creux de sa main, les aspira avec ses lèvres, et, après s'en être rincé la bouche comme on fait avec le vin que l'on veut goûter, il cracha la liqueur dans la cheminée. 

«C'est bien la même, dit-il. Et vous en avez bu aussi, monsieur Noirtier? 

—Oui, fit le vieillard. 

—Et vous lui avez trouvé ce même goût amer? 

—Oui. 

—Ah! monsieur le docteur! cria Barrois, voilà que cela me reprend! Mon Dieu, Seigneur, ayez pitié de moi!» 

Le docteur courut au malade. 

«Cet émétique, Villefort, voyez s'il vient.» 

Villefort s'élança en criant: 

«L'émétique! l'émétique! l'a-t-on apporté?» 

Personne ne répondit. La terreur la plus profonde régnait dans la maison. 

«Si j'avais un moyen de lui insuffler de l'air dans les poumons, dit d'Avrigny en regardant autour de lui, peut-être y aurait-il possibilité de prévenir l'asphyxie. Mais non, rien, rien! 

—Oh! monsieur, criait Barrois, me laisserez-vous mourir ainsi sans secours? Oh! je me meurs, mon Dieu! je me meurs! 

—Une plume! une plume!» demanda le docteur. 

Il en aperçut une sur la table. 

Il essaya d'introduire la plume dans la bouche du malade, qui faisait, au milieu de ses convulsions, d'inutiles efforts pour vomir; mais les mâchoires étaient tellement serrées, que la plume ne put passer. 

Barrois était atteint d'une attaque nerveuse encore plus intense que la première. Il avait glissé de la chaise longue à terre, et se raidissait sur le parquet. 

Le docteur le laissa en proie à cet accès, auquel il ne pouvait apporter aucun soulagement, et alla à Noirtier. 

«Comment vous trouvez-vous? lui dit-il précipitamment et à voix basse; bien? 

—Oui. 

—Léger d'estomac ou lourd? léger? 

—Oui. 

—Comme lorsque vous avez pris la pilule que je fais donner chaque dimanche? 

—Oui. 

—Est-ce Barrois qui a fait votre limonade? 

—Oui.  

—Est-ce vous qui l'avez engagé à en boire? 

—Non. 

—Est-ce M. de Villefort? 

—Non. 

—Madame? 

—Non. 

—C'est donc Valentine, alors? 

—Oui.» 

Un soupir de Barrois, un bâillement qui faisait craquer des os de sa mâchoire, appelèrent l'attention de d'Avrigny: il quitta M. Noirtier et courut près du malade. 

«Barrois, dit le docteur, pouvez-vous parler?» 

Barrois balbutia quelques paroles inintelligibles. 

«Essayez un effort, mon ami.» 

Barrois rouvrit des yeux sanglants. 

«Qui a fait la limonade? 

—Moi. 

—L'avez-vous apportée à votre maître aussitôt après l'avoir faite? 

—Non. 

—Vous l'avez laissée quelque part, alors? 

—À l'office, on m'appelait. 

—Qui l'a apportée ici? 

—Mlle Valentine.» 

D'Avrigny se frappa le front. 

«Ô mon Dieu! mon Dieu! murmura-t-il. 

—Docteur! docteur! cria Barrois, qui sentait un troisième accès arriver. 

—Mais n'apportera-t-on pas cet émétique, s'écria le docteur. 

—Voilà un verre tout préparé, dit Villefort en rentrant. 

—Par qui? 

—Par le garçon pharmacien qui est venu avec moi. 

—Buvez. 

—Impossible, docteur, il est trop tard; j'ai la gorge qui se serre, j'étouffe! Oh! mon cœur! Oh! ma tête\dots. Oh! quel enfer!\dots Est-ce que je vais souffrir longtemps comme cela?  

—Non, non, mon ami, dit le docteur, bientôt vous ne souffrirez plus. 

—Ah je vous comprends! s'écria le malheureux; mon Dieu! prenez pitié de moi!» 

Et, jetant un cri, il tomba renversé en arrière, comme s'il eût été foudroyé. D'Avrigny posa une main sur son cœur, approcha une glace de ses lèvres. 

«Eh bien? demanda Villefort. 

—Allez dire à la cuisine que l'on m'apporte bien vite du sirop de violettes.» 

Villefort descendit à l'instant même. 

«Ne vous effrayez pas, monsieur Noirtier, dit d'Avrigny, j'emporte le malade dans une autre chambre pour le saigner; en vérité, ces sortes d'attaques sont un affreux spectacle à voir.» 

Et prenant Barrois par-dessous les bras, il le traîna dans une chambre voisine; mais presque aussitôt il rentra chez Noirtier pour prendre le reste de la limonade. 

Noirtier fermait l'œil droit. 

«Valentine, n'est-ce pas? vous voulez Valentine? Je vais dire qu'on vous l'envoie.» 

Villefort remontait; d'Avrigny le rencontra dans le corridor. 

«Eh bien? demanda-t-il. 

—Venez», dit d'Avrigny. 

Et il l'emmena dans la chambre. 

«Toujours évanoui? demanda le procureur du roi. 

—Il est mort.» 

Villefort recula de trois pas, joignit les mains au-dessus de sa tête, et avec une commisération non équivoque: 

«Mort si promptement! dit-il en regardant le cadavre. 

—Oui, bien promptement, n'est-ce pas? dit d'Avrigny; mais cela ne doit pas vous étonner: M. et Mme de Saint-Méran sont morts tout aussi promptement. Oh! l'on meurt vite dans votre maison, monsieur de Villefort. 

—Quoi! s'écria le magistrat avec un accent d'horreur et de consternation, vous en revenez à cette terrible idée! 

—Toujours, monsieur, toujours! dit d'Avrigny avec solennité, car elle ne m'a pas quitté un instant, et pour que vous soyez bien convaincu que je ne me trompe pas cette fois, écoutez bien, monsieur de Villefort.» 

Villefort tremblait convulsivement. 

«Il y a un poison qui tue sans presque laisser de trace. Ce poison, je le connais bien: je l'ai étudié dans tous les accidents qu'il amène, dans tous les phénomènes qu'il produit. Ce poison, je l'ai reconnu tout à l'heure chez le pauvre Barrois, comme je l'avais reconnu chez Mme de Saint-Méran. Ce poison, il y a une manière de reconnaître sa présence: il rétablit la couleur bleue du papier de tournesol rougi par un acide, et il teint en vert le sirop de violettes. Nous n'avons pas de papier de tournesol; mais, tenez, voilà qu'on apporte le sirop de violettes que j'ai demandé.» 

En effet, on entendait des pas dans le corridor, le docteur entrebâilla la porte, prit des mains de la femme de chambre un vase au fond duquel il y avait deux ou trois cuillerées de sirop, et referma la porte. 

«Regardez, dit-il au procureur du roi, dont le cœur battait si fort qu'on eût pu l'entendre, voici dans cette tasse du sirop de violettes, et dans cette carafe le reste de la limonade dont M. Noirtier et Barrois ont bu une partie. Si la limonade est pure et inoffensive, le sirop va garder sa couleur; si la limonade est empoisonnée, le sirop va devenir vert. Regardez!» 

Le docteur versa lentement quelques gouttes de limonade de la carafe dans la tasse, et l'on vit à l'instant même un nuage se former au fond de la tasse, ce nuage prit d'abord une nuance bleue; puis du saphir il passa à l'opale et de l'opale à l'émeraude. 

Arrivé à cette dernière couleur, il s'y fixa, pour ainsi dire, l'expérience ne laissait aucun doute. 

«Le malheureux Barrois a été empoisonné avec de la fausse angusture et de la noix de Saint-Ignace, dit d'Avrigny; maintenant j'en répondrais devant les hommes et devant Dieu.» 

Villefort ne dit rien, lui, mais il leva les bras au ciel, ouvrit des yeux hagards, et tomba foudroyé sur un fauteuil. 