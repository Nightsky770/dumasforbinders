\chapter{La chambre du boulanger retiré}

\lettrine{L}{e} soir même du jour où le comte de Morcerf était sorti de chez Danglars avec une honte et une fureur que rend concevables la froideur du banquier, M. Andrea Cavalcanti, les cheveux frisés et luisants, les moustaches aiguisées, les gants blancs dessinant les ongles, était entré, presque debout sur son phaéton, dans la cour du banquier de la Chaussée-d'Antin. 

Au bout de dix minutes de conversation au salon, il avait trouvé le moyen de conduire Danglars dans une embrasure de fenêtre, et là, après un adroit préambule, il avait exposé les tourments de sa vie, depuis le départ de son noble père. Depuis le départ, il avait, disait-il, dans la famille du banquier, où l'on avait bien voulu le recevoir comme un fils, il avait trouvé toutes les garanties de bonheur qu'un homme doit toujours rechercher avant les caprices de la passion, et, quant à la passion elle-même, il avait eu le bonheur de la rencontrer dans les beaux yeux de Mlle Danglars. 

Danglars écoutait avec l'attention la plus profonde, il y avait déjà deux ou trois jours qu'il attendait cette déclaration, et lorsqu'elle arriva enfin, son œil se dilata autant qu'il s'était couvert et assombri en écoutant Morcerf. 

Cependant, il ne voulut point accueillir ainsi la proposition du jeune homme sans lui faire quelques observations de conscience. 

«Monsieur Andrea, lui dit-il, n'êtes-vous pas un peu jeune pour songer au mariage? 

—Mais non, monsieur, reprit Cavalcanti, je ne trouve pas, du moins: en Italie, les grands seigneurs se marient jeunes, en général; c'est une coutume logique. La vie est si chanceuse que l'on doit saisir le bonheur aussitôt qu'il passe à notre portée. 

—Maintenant, monsieur, dit Danglars, en admettant que vos propositions, qui m'honorent, soient agréées de ma femme et de ma fille, avec qui débattrions-nous les intérêts? C'est, il me semble, une négociation importante que les pères seuls savent traiter convenablement pour le bonheur de leurs enfants. 

—Monsieur, mon père est un homme sage, plein de convenance et de raison. Il a prévu la circonstance probable où j'éprouverais le désir de m'établir en France: il m'a donc laissé en partant, avec tous les papiers qui constatent mon identité, une lettre par laquelle il m'assure, dans le cas où je ferais un choix qui lui soit agréable, cent cinquante mille livres de rente, à partir du jour de mon mariage. C'est, autant que je puis juger, le quart du revenu de mon père. 

—Moi, dit Danglars, j'ai toujours eu l'intention de donner à ma fille cinq cent mille francs en la mariant; c'est d'ailleurs ma seule héritière. 

—Eh bien, dit Andrea, vous voyez, la chose serait pour le mieux, en supposant que ma demande ne soit pas repoussée par Mme la baronne Danglars et par Mlle Eugénie. Nous voilà à la tête de cent soixante-quinze mille livres de rente. Supposons une chose, que j'obtienne du marquis qu'au lieu de me payer la rente il me donne le capital (ce ne serait pas facile, je le sais bien, mais enfin cela se peut), vous nous feriez valoir ces deux ou trois millions, et deux ou trois millions entre des mains habiles peuvent toujours rapporter dix pour cent. 

—Je ne prends jamais qu'à quatre, dit le banquier, et même à trois et demi. Mais à mon gendre, je prendrais à cinq, et nous partagerions les bénéfices. 

—Eh bien, à merveille, beau-père», dit Cavalcanti, se laissant entraîner à la nature quelque peu vulgaire qui, de temps en temps, malgré ses efforts, faisait éclater le vernis d'aristocratie dont il essayait de les couvrir. 

Mais aussitôt se reprenant: 

«Oh! pardon, monsieur, dit-il, vous voyez, l'espérance seule me rend presque fou, que serait-ce donc de la réalité? 

—Mais, dit Danglars, qui, de son côté, ne s'apercevait pas combien cette conversation, désintéressée d'abord, tournait promptement à l'agence d'affaires, il y a sans doute une portion de votre fortune que votre père ne peut vous refuser? 

—Laquelle? demanda le jeune homme. 

—Celle qui vient de votre mère. 

—Eh! certainement, celle qui vient de ma mère, Leonora Corsinari. 

—Et à combien peut monter cette portion de fortune? 

—Ma foi, dit Andrea, je vous assure, monsieur, que je n'ai jamais arrêté mon esprit sur ce sujet, mais je l'estime à deux millions pour le moins.» 

Danglars ressentit cette espèce d'étouffement joyeux que ressentent, ou l'avare qui retrouve un trésor perdu, ou l'homme prêt à se noyer qui rencontre sous ses pieds la terre solide au lieu du vide dans lequel il allait s'engloutir. 

«Eh bien, monsieur, dit Andrea en saluant le banquier avec un tendre respect, puis-je espérer\dots. 

—Monsieur Andrea, dit Danglars, espérez, et croyez bien que si nul obstacle de votre part n'arrête la marche de cette affaire, elle est conclue. Mais, dit Danglars réfléchissant, comment se fait-il que M. le comte de Monte-Cristo, votre patron en ce monde parisien, ne soit pas venu avec vous nous faire cette demande?» 

Andrea rougit imperceptiblement. 

«Je viens de chez le comte, monsieur, dit-il, c'est incontestablement un homme charmant, mais d'une originalité inconcevable; il m'a fort approuvé, il m'a dit même qu'il ne croyait pas que mon père hésitât un instant à me donner le capital au lieu de la rente; il m'a promis son influence pour m'aider à obtenir cela de lui, mais il m'a déclaré que, personnellement, il n'avait jamais pris et ne prendrait jamais sur lui cette responsabilité de faire une demande en mariage. Mais je dois lui rendre cette justice, il a daigné ajouter que, s'il avait jamais déploré cette répugnance, c'était à mon sujet, puisqu'il pensait que l'union projetée serait heureuse et assortie. Du reste, s'il ne veut rien faire officiellement, il se réserve de vous répondre, m'a-t-il dit, quand vous lui parlerez. 

—Ah! fort bien. 

—Maintenant, dit Andrea avec son plus charmant sourire, j'ai fini de parler au beau-père et je m'adresse au banquier. 

—Que lui voulez-vous, voyons? dit en riant Danglars à son tour. 

—C'est après-demain que j'ai quelque chose comme quatre mille francs à toucher chez vous; mais le comte a compris que le mois dans lequel j'allais entrer amènerait peut-être un surcroît de dépenses auquel mon petit revenu de garçon ne saurait suffire, et voici un bon de vingt mille francs qu'il m'a, je ne dirai pas donné, mais offert. Il est signé de sa main, comme vous voyez; cela vous convient-il? 

—Apportez-m'en comme celui-là pour un million, je vous les prends, dit Danglars en mettant le bon dans sa poche. Dites-moi votre heure pour demain, et mon garçon de caisse passera chez vous avec un reçu de vingt-quatre mille francs. 

—Mais à dix heures du matin, si vous voulez bien; le plus tôt sera le mieux: je voudrais aller demain à la campagne. 

—Soit, à dix heures, à l'hôtel des Princes, toujours? 

—Oui.» 

Le lendemain, avec une exactitude qui faisait honneur à la ponctualité du banquier, les vingt-quatre mille francs étaient chez le jeune homme, qui sortit effectivement, laissant deux cents francs pour Caderousse. Cette sortie avait, de la part d'Andrea, pour but principal d'éviter son dangereux ami; aussi rentra-t-il le soir le plus tard possible. 

Mais à peine eut-il mis le pied sur le pavé de la cour qu'il trouva devant lui le concierge de l'hôtel, qui l'attendait, la casquette à la main. 

«Monsieur, dit-il, cet homme est venu. 

—Quel homme? demanda négligemment Andrea comme s'il eût oublié celui dont, au contraire, il se souvenait trop bien. 

—Celui à qui Votre Excellence fait cette petite rente. 

—Ah! oui, dit Andrea, cet ancien serviteur de mon père. Eh bien, vous lui avez donné les deux cents francs que j'avais laissés pour lui. 

—Oui, Excellence, précisément.» 

Andrea se faisait appeler Excellence. 

«Mais, continua le concierge, il n'a pas voulu les prendre.» 

Andrea pâlit; seulement, comme il faisait nuit, personne ne le vit pâlir. 

«Comment! il n'a pas voulu les prendre? dit-il d'une voix légèrement émue. 

—Non! il voulait parler à Votre Excellence. J'ai répondu que vous étiez sorti; il a insisté. Mais enfin il a paru se laisser convaincre, et m'a donné cette lettre qu'il avait apportée toute cachetée. 

—Voyons», dit Andrea. 

Il lut à la lanterne de son phaéton: 

«Tu sais où je demeure; je t'attends demain à neuf heures du matin.» 

Andrea interrogea le cachet pour voir s'il avait été forcé et si des regards indiscrets avaient pu pénétrer dans l'intérieur de la lettre; mais elle était pliée de telle sorte, avec un tel luxe de losanges et d'angles, que pour la lire il eût fallu rompre le cachet; or, le cachet était parfaitement intact. 

«Très bien, dit-il. Pauvre homme! c'est une bien excellente créature.» 

Et il laissa le concierge édifié par ces paroles, et ne sachant pas lequel il devait le plus admirer, du jeune maître ou du vieux serviteur. 

«Dételez vite, et montez chez moi», dit Andrea à son groom. 

En deux bonds, le jeune homme fut dans sa chambre et eut brûlé la lettre de Caderousse, dont il fit disparaître jusqu'aux cendres. 

Il achevait cette opération lorsque le domestique entra. 

«Tu es de la même taille que moi, Pierre, lui dit-il. 

—J'ai cet honneur-là, Excellence, répondit le valet. 

—Tu dois avoir une livrée neuve qu'on t'a apportée hier? 

—Oui, monsieur. 

—J'ai affaire à une petite grisette à qui je ne veux dire ni mon titre ni ma condition. Prête-moi ta livrée et apporte-moi tes papiers, afin que je puisse, si besoin est, coucher dans une auberge.» 

Pierre obéit. 

Cinq minutes après, Andrea, complètement déguisé, sortait de l'hôtel sans être reconnu, prenait un cabriolet et se faisait conduire à l'auberge du Cheval-Rouge, à Picpus. 

Le lendemain, il sortit de l'auberge du Cheval-Rouge comme il était sorti de l'hôtel des Princes, c'est-à-dire sans être remarqué, descendit le faubourg Saint-Antoine, prit le boulevard jusqu'à la rue Ménilmontant, et, s'arrêtant à la porte de la troisième maison a gauche, chercha à qui il pouvait, en l'absence du concierge, demander des renseignements. 

«Que cherchez-vous, mon joli garçon? demanda la fruitière d'en face. 

—M. Pailletin, s'il vous plaît, ma grosse maman? répondit Andrea. 

—Un boulanger retiré? demanda la fruitière. 

—Justement, c'est cela. 

—Au fond de la cour, à gauche, au troisième.» 

Andrea prit le chemin indiqué, et au troisième trouva une patte de lièvre qu'il agita avec un sentiment de mauvaise humeur dont le mouvement précipité de la sonnette se ressentit. 

Une seconde après, la figure de Caderousse apparut au grillage pratiqué dans la porte. 

«Ah! tu es exact», dit-il. 

Et il tira les verrous. 

«Parbleu!» dit Andrea en entrant. 

Et il lança devant lui sa casquette de livrée qui, manquant la chaise, tomba à terre et fit le tour de la chambre en roulant sur sa circonférence. 

«Allons, allons, dit Caderousse, ne te fâche pas, le petit! Voyons, tiens, j'ai pensé à toi, regarde un peu le bon déjeuner que nous aurons: rien que des choses que tu aimes, tron de l'air!» 

Andrea sentit en effet, en respirant, une odeur de cuisine dont les arômes grossiers ne manquaient pas d'un certain charme pour un estomac affamé, c'était ce mélange de graisse fraîche et d'ail qui signale la cuisine provençale d'un ordre inférieur; c'était en outre un goût de poisson gratiné, puis, par-dessus tout, l'âpre parfum de la muscade et du girofle. Tout cela s'exhalait de deux plats creux et couverts, posés sur deux fourneaux, et d'une casserole qui bruissait dans le four d'un poêle de fonte. 

Dans la chambre voisine, Andrea vit en outre une table assez propre ornée de deux couverts, de deux bouteilles de vin cachetées, l'une de vert, l'autre de jaune, d'une bonne mesure d'eau-de-vie dans un carafon, et d'une macédoine de fruits dans une large feuille de chou posée avec art sur une assiette de faïence. 

«Que t'en semble? le petit, dit Caderousse; hein, comme cela embaume! Ah! dame! tu sais, j'étais bon cuisinier là-bas! te rappelles-tu comme on se léchait les doigts de ma cuisine? Et toi tout le premier, tu en as goûté de mes sauces, et tu ne les méprisais pas, que je crois.» 

Et Caderousse se mit à éplucher un supplément d'oignons. 

«C'est bon, c'est bon, dit Andrea avec humeur, pardieu!, si c'est pour déjeuner avec toi que tu m'as dérangé, que le diable t'emporte! 

—Mon fils, dit sentencieusement Caderousse, en mangeant l'on cause; et puis, ingrat que tu es, tu n'as donc pas de plaisir à voir un peu ton ami? Moi, j'en pleure de joie.» 

Caderousse, en effet, pleurait réellement; seulement, il eût été difficile de dire si c'était la joie ou les oignons qui opéraient sur la glande lacrymale de l'ancien aubergiste du pont du Gard. 

«Tais-toi donc, hypocrite, dit Andrea; tu m'aimes, toi? 

—Oui, je t'aime, ou le diable m'emporte; c'est une faiblesse, dit Caderousse, je le sais bien, mais c'est plus fort que moi. 

—Ce qui ne t'empêche pas de m'avoir fait venir pour quelque perfidie. 

—Allons donc! dit Caderousse en essuyant son large couteau à son tablier, si je ne t'aimais pas, est-ce que je supporterais la vie misérable que tu me fais? Regarde un peu, tu as sur le dos l'habit de ton domestique, donc tu as un domestique; moi, je n'en ai pas, et je suis forcé d'éplucher mes légumes moi-même: tu fais fi de ma cuisine, parce que tu dînes à la table d'hôte de l'hôtel des Princes ou au Café de Paris. Eh bien, moi aussi, je pourrais avoir un domestique; moi aussi, je pourrais avoir un tilbury; moi aussi, je pourrais dîner où je voudrais: eh bien, pourquoi est-ce que je m'en prive? pour ne pas faire de peine à mon petit Benedetto. Voyons, avoue seulement que je le pourrais, hein?» 

Et un regard parfaitement clair de Caderousse termina le sens de la phrase. 

«Bon, dit Andrea, mettons que tu m'aimes: alors pourquoi exiges-tu que je vienne déjeuner avec toi? 

—Mais pour te voir, le petit. 

—Pour me voir, à quoi bon? puisque nous avons fait d'avance toutes nos conditions. 

—Eh! cher ami, dit Caderousse, est-ce qu'il y a des testaments sans codicilles? Mais tu es venu pour déjeuner d'abord, n'est-ce pas? Eh bien, voyons, assieds-toi, et commençons par ces sardines et ce beurre frais, que j'ai mis sur des feuilles de vigne à ton intention, méchant. Ah! oui, tu regardes ma chambre, mes quatre chaises de paille, mes images à trois francs le cadre. Dame! que veux-tu, ça n'est pas l'hôtel des Princes. 

—Allons, te voilà dégoûté à présent; tu n'es plus heureux, toi qui ne demandais qu'à avoir l'air d'un boulanger retiré.» 

Caderousse poussa un soupir. 

«Eh bien, qu'as-tu à dire? tu as vu ton rêve réalisé. 

—J'ai à dire que c'est un rêve, un boulanger retiré, mon pauvre Benedetto, c'est riche, cela a des rentes. 

—Pardieu! tu en as des rentes. 

—Moi? 

—Oui, toi, puisque je t'apporte tes deux cents francs.» 

Caderousse haussa les épaules. 

«C'est humiliant, dit-il, de recevoir ainsi de l'argent donné à contrecœur, de l'argent éphémère, qui peut me manquer du jour au lendemain. Tu vois bien que je suis obligé de faire des économies pour le cas où ta prospérité ne durerait pas. Eh! mon ami, la fortune est inconstante, comme disait l'aumônier\dots du régiment. Je sais bien qu'elle est immense, ta prospérité, scélérat; tu vas épouser la fille de Danglars. 

—Comment! de Danglars? 

—Et certainement, de Danglars! Ne faut-il pas que je dise du baron Danglars? C'est comme si je disais du comte Benedetto. C'était un ami, Danglars, et s'il n'avait pas la mémoire si mauvaise, il devrait m'inviter à ta noce\dots attendu qu'il est venu à la mienne\dots oui, oui, oui, à la mienne! Dame! il n'était pas si fier dans ce temps-là; il était petit commis chez ce bon M. Morrel. J'ai dîné plus d'une fois avec lui et le comte de Morcerf\dots. Va, tu vois que j'ai de belles connaissances et que si je voulais les cultiver un petit peu, nous nous rencontrerions dans les mêmes salons. 

—Allons donc, ta jalousie te fait voir des arcs-en-ciel, Caderousse. 

—C'est bon, Benedetto mio, on sait ce que l'on dit. Peut-être qu'un jour aussi l'on mettra son habit des dimanches, et qu'on ira dire à une porte cochère: «Le cordon, s'il vous plaît!» En attendant, assieds-toi et mangeons.» 

Caderousse donna l'exemple et se mit à déjeuner de bon appétit, et en faisant l'éloge de tous les mets qu'il servait à son hôte. 

Celui-ci sembla prendre son parti, déboucha bravement les bouteilles et attaqua la bouillabaisse et la morue gratinée à l'ail et à l'huile. 

«Ah! compère, dit Caderousse, il paraît que tu te raccommodes avec ton ancien maître d'hôtel? 

—Ma foi, oui, répondit Andrea, chez lequel, jeune et vigoureux qu'il était, l'appétit l'emportait pour le moment sur toute autre chose. 

—Et tu trouves cela bon, coquin? 

—Si bon, que je ne comprends pas comment un homme qui fricasse et qui mange de si bonnes choses peut trouver que la vie est mauvaise. 

—Vois-tu, dit Caderousse, c'est que tout mon bonheur est gâté par une seule pensée. 

—Laquelle? 

—C'est que je vis aux dépens d'un ami, moi qui ai toujours bravement gagné ma vie moi-même. 

—Oh! oh! qu'à cela ne tienne, dit Andrea, j'ai assez pour deux, ne te gêne pas. 

—Non, vraiment; tu me croiras si tu veux, à la fin de chaque mois, j'ai des remords. 

—Bon Caderousse! 

—C'est au point qu'hier je n'ai pas voulu prendre les deux cents francs. 

—Oui, tu voulais me parler; mais est-ce bien le remords, voyons? 

—Le vrai remords; et puis il m'était venu une idée.» 

Andrea frémit; il frémissait toujours aux idées de Caderousse. 

«C'est misérable, vois-tu, continua celui-ci, d'être toujours à attendre la fin d'un mois. 

—Eh! dit philosophiquement Andrea, décidé à voir venir son compagnon, la vie ne se passe-t-elle pas à attendre? Moi, par exemple, est-ce que je fais autre chose? Eh bien, je prends patience, n'est-ce pas? 

—Oui, parce qu'au lieu d'attendre deux cents misérables francs, tu en attends cinq ou six mille, peut-être dix, peut-être douze même; car tu es un cachottier: là-bas, tu avais toujours des boursicots, des tirelires que tu essayais de soustraire à ce pauvre ami Caderousse. Heureusement qu'il avait le nez fin, l'ami Caderousse en question. 

—Allons, voilà que tu vas te remettre à divaguer, dit Andrea, à parler et à reparler du passé toujours! Mais à quoi bon rabâcher comme cela, je te le demande? 

—Ah! c'est que tu as vingt et un ans, toi, et que tu peux oublier le passé; j'en ai cinquante, et je suis bien forcé de m'en souvenir. Mais n'importe, revenons aux affaires. 

—Oui. 

—Je voulais dire que si j'étais à ta place\dots. 

—Eh bien? 

—Je réaliserais\dots. 

—Comment! tu réaliserais\dots. 

—Oui, je demanderais un semestre d'avance, sous prétexte que je veux devenir éligible et que je vais acheter une ferme; puis avec mon semestre je décamperais. 

—Tiens, tiens, tiens, fit Andrea, ce n'est pas si mal pensé, cela, peut-être! 

—Mon cher ami, dit Caderousse, mange de ma cuisine et suis mes conseils; tu ne t'en trouveras pas plus mal, physiquement et moralement. 

—Eh bien, mais, dit Andrea, pourquoi ne suis-tu pas toi-même le conseil que tu donnes? pourquoi ne réalises-tu pas un semestre, une année même et ne te retires-tu pas à Bruxelles? Au lieu d'avoir l'air d'un boulanger retiré, tu aurais l'air d'un banqueroutier dans l'exercice de ses fonctions: cela est bien porté. 

—Mais comment diable veux-tu que je me retire avec douze cents francs? 

—Ah! Caderousse, dit Andrea, comme tu te fais exigeant! Il y a deux mois, tu mourais de faim. 

—L'appétit vient en mangeant, dit Caderousse en montrant ses dents comme un singe qui rit ou comme un tigre qui gronde. Aussi, ajouta-t-il en coupant avec ces mêmes dents, si blanches et si aiguës, malgré l'âge, une énorme bouchée de pain, j'ai fait un plan.» 

Les plans de Caderousse épouvantaient Andrea encore plus que ses idées; les idées n'étaient que le germe, le plan, c'était la réalisation. 

«Voyons ce plan, dit-il; ce doit être joli! 

—Pourquoi pas? Le plan grâce auquel nous avons quitté l'établissement de M. Chose, de qui venait-il, hein? de moi, je présuppose; il n'en était pas plus mauvais, ce me semble, puisque nous voilà ici! 

—Je ne dis pas, répondit Andrea, tu as quelquefois du bon; mais enfin, voyons ton plan. 

—Voyons, poursuivit Caderousse, peux-tu, toi, sans débourser un sou, me faire avoir une quinzaine de mille francs\dots non, ce n'est pas assez de quinze mille francs, je ne veux pas devenir honnête homme à moins de trente mille francs? 

—Non, répondit sèchement Andrea, non, je ne le puis pas. 

—Tu ne m'as pas compris, à ce qu'il paraît, répondit froidement Caderousse d'un air calme; je t'ai dit sans débourser un sou. 

—Ne veux-tu pas que je vole pour gâter toute mon affaire, et la tienne avec la mienne, et qu'on nous reconduise là-bas? 

—Oh! moi, dit Caderousse, ça m'est bien égal qu'on me reprenne; je suis un drôle de corps, sais-tu: je m'ennuie parfois des camarades; ce n'est pas comme toi, sans cœur, qui voudrais ne jamais les revoir!» 

Andrea fit plus que frémir cette fois, il pâlit. 

«Voyons Caderousse, pas de bêtises, dit-il. 

—Eh! non, sois donc tranquille, mon petit Benedetto; mais indique-moi donc un petit moyen de gagner ces trente mille francs sans te mêler de rien; tu me laisseras faire, voilà tout! 

—Eh bien, je verrai, je chercherai, dit Andrea. 

—Mais, en attendant, tu pousseras mon mois à cinq cents francs, j'ai une manie, je voudrais prendre une bonne! 

—Eh bien, tu auras tes cinq cents francs, dit Andrea: mais c'est lourd pour moi, mon pauvre Caderousse\dots tu abuses\dots. 

—Bah! dit Caderousse; puisque tu puises dans des coffres qui n'ont point de fond.» 

On eût dit qu'Andrea attendait là son compagnon, tant son œil brilla d'un rapide éclair qui, il est vrai, s'éteignit aussitôt. 

«Ça, c'est la vérité, répondit Andrea, et mon protecteur est excellent pour moi. 

—Ce cher protecteur! dit Caderousse; ainsi donc il te fait par mois?\dots 

—Cinq mille francs, dit Andrea. 

—Autant de mille que tu me fais de cents, reprit Caderousse; en vérité, il n'y a que des bâtards pour avoir du bonheur. Cinq mille francs par mois\dots. Que diable peut-on faire de tout cela? 

—Eh, mon Dieu! c'est bien vite dépensé; aussi, je suis comme toi, je voudrais bien avoir un capital. 

—Un capital\dots oui\dots je comprends, tout le monde voudrait bien avoir un capital. 

—Eh bien, moi, j'en aurai un. 

—Et qui est-ce qui te le fera? ton prince? 

—Oui, mon prince; malheureusement il faut que j'attende. 

—Que tu attendes quoi? demanda Caderousse. 

—Sa mort.  

—La mort de ton prince? 

—Oui. 

—Comment cela? 

—Parce qu'il m'a porté sur son testament. 

—Vrai? 

—Parole d'honneur! 

—Pour combien? 

—Pour cinq cent mille! 

—Rien que cela; merci du peu. 

—C'est comme je te le dis. 

—Allons donc, pas possible! 

—Caderousse, tu es mon ami? 

—Comment donc! à la vie, à la mort. 

—Eh bien, je vais te dire un secret. 

—Dis. 

—Mais écoute. 

—Oh! pardieu! muet comme une carpe. 

—Eh bien, je crois\dots.» 

Andrea s'arrêta en regardant autour de lui. 

«Tu crois?\dots N'aie pas peur, pardieu! nous sommes seuls. 

—Je crois que j'ai retrouvé mon père. 

—Ton vrai père? 

—Oui. 

—Pas le père Cavalcanti. 

—Non, puisque celui-là est reparti; le vrai, comme tu dis. 

—Et ce père, c'est\dots. 

—Eh bien, Caderousse, c'est le comte de Monte-Cristo. 

—Bah! 

—Oui; tu comprends, alors tout s'explique. Il ne peut pas m'avouer tout haut, à ce qu'il paraît, mais il me fait reconnaître par M. Cavalcanti, à qui il donne cinquante mille francs pour ça. 

—Cinquante mille francs pour être ton père! Moi, j'aurais accepté pour moitié prix, pour vingt mille, pour quinze mille! Comment, tu n'as pas pensé à moi? 

—Est-ce que je savais cela, puisque tout s'est fait tandis que nous étions là-bas? 

—Ah! c'est vrai. Et tu dis que, par son testament\dots? 

—Il me laisse cinq cent mille livres. 

—Tu en es sûr? 

—Il me l'a montré; mais ce n'est pas le tout. 

—Il y a un codicille, comme je disais tout à l'heure! 

—Probablement. 

—Et dans ce codicille?\dots 

—Il me reconnaît. 

—Oh! le bon homme de père, le brave homme de père, l'honnêtissime homme de père! dit Caderousse en faisant tourner en l'air une assiette qu'il retint entre ses deux mains. 

—Voilà! dis encore que j'ai des secrets pour toi! 

—Non, et ta confiance t'honore à mes yeux. Et ton prince de père, il est donc riche, richissime? 

—Je crois bien. Il ne connaît pas sa fortune. 

—Est-ce possible? 

—Dame! je le vois bien, moi qui suis reçu chez lui à toute heure. L'autre jour, c'était un garçon de banque qui lui apportait cinquante mille francs dans un portefeuille gros comme ta serviette; hier, c'est un banquier qui lui apportait cent mille francs en or.» 

Caderousse était abasourdi; il lui semblait que les paroles du jeune homme avaient le son du métal, et qu'il entendait rouler des cascades de louis. 

«Et tu vas dans cette maison-là? s'écria-t-il avec naïveté. 

—Quand je veux.» 

Caderousse demeura pensif un instant. Il était facile de voir qu'il retournait dans son esprit quelque profonde pensée. 

Puis soudain: 

«Que j'aimerais à voir tout cela! s'écria-t-il, et comme tout cela doit être beau! 

—Le fait est, dit Andrea, que c'est magnifique! 

—Et ne demeure-t-il pas avenue des Champs-Élysées? 

—Numéro trente. 

—Ah! dit Caderousse, numéro trente? 

—Oui, une belle maison isolée, entre cour et jardin, tu ne connais que cela. 

—C'est possible; mais ce n'est pas l'extérieur qui m'occupe, c'est l'intérieur: les beaux meubles, hein! qu'il doit y avoir là-dedans? 

—As-tu vu quelquefois les Tuileries? 

—Non. 

—Eh bien, c'est plus beau. 

—Dis donc, Andrea, il doit faire bon à se baisser quand ce bon Monte-Cristo laisse tomber sa bourse? 

—Oh! mon Dieu! ce n'est pas la peine d'attendre ce moment-là, dit Andrea, l'argent traîne dans cette maison-là comme les fruits dans un verger. 

—Dis donc, tu devrais m'y conduire un jour avec toi. 

—Est-ce que c'est possible! et à quel titre? 

—Tu as raison; mais tu m'as fait venir l'eau à la bouche; faut absolument que je voie cela; je trouverai un moyen. 

—Pas de bêtises, Caderousse! 

—Je me présenterai comme frotteur. 

—Il y a des tapis partout. 

—Ah! pécaïre! alors il faut que je me contente de voir cela en imagination. 

—C'est ce qu'il y a de mieux, crois-moi. 

—Tâche au moins de me faire comprendre ce que cela peut être. 

—Comment veux-tu?\dots 

—Rien de plus facile. Est-ce grand? 

—Ni trop grand ni trop petit. 

—Mais comment est-ce distribué? 

—Dame! il me faudrait de l'encre et du papier pour faire un plan. 

—En voilà!» dit vivement Caderousse. 

Et il alla chercher sur un vieux secrétaire une feuille de papier blanc, de l'encre et une plume. 

«Tiens, dit Caderousse, trace-moi tout cela sur du papier, mon fils.» 

Andrea prit la plume avec un imperceptible sourire et commença. 

«La maison, comme je te l'ai dit, est entre cour et jardin, vois-tu, comme cela?» 

Et Andrea fit le tracé du jardin, de la cour et de la maison. 

«Des grands murs? 

—Non, huit ou dix pieds tout au plus. 

—Ce n'est pas prudent, dit Caderousse. 

—Dans la cour, des caisses d'orangers, des pelouses, des massifs de fleurs.  

—Et pas de pièges à loups? 

—Non. 

—Les écuries? 

—Aux deux côtés de la grille, où tu vois, là.» 

Andrea continua son plan. 

«Voyons le rez-de-chaussée, dit Caderousse. 

—Au rez-de-chaussée, salle à manger, deux salons, salle de billard, escalier dans le vestibule, et petit escalier dérobé. 

—Des fenêtres?\dots 

—Des fenêtres magnifiques, si belles, si larges que, ma foi, oui, je crois qu'un homme de ta taille passerait par chaque carreau. 

—Pourquoi diable a-t-on des escaliers, quand on a des fenêtres pareilles? 

—Que veux-tu! le luxe. 

—Mais des volets? 

—Oui, des volets, mais dont on ne se sert jamais. Un original, ce comte de Monte-Cristo, qui aime à voir le ciel même pendant la nuit! 

—Et les domestiques, où couchent-ils? 

—Oh! ils ont leur maison à eux. Figure-toi un joli hangar à droite en entrant, où l'on serre les échelles. Eh bien, il y a sur ce hangar une collection de chambres pour les domestiques, avec des sonnettes correspondant aux chambres. 

—Ah! diable! des sonnettes! 

—Tu dis?\dots 

—Moi, rien. Je dis que cela coûte très cher à poser les sonnettes; et à quoi cela sert-il, je te le demande? 

—Autrefois il y avait un chien qui se promenait la nuit dans la cour, mais on l'a fait conduire à la maison d'Auteuil, tu sais, à celle où tu es venu? 

—Oui. 

—Moi, je lui disais encore hier: «C'est imprudent de votre part, monsieur le comte, car, lorsque vous allez à Auteuil et que vous emmenez vos domestiques, la maison reste seule.» 

—Eh bien, a-t-il démandé, après? 

—Eh bien, après, quelque beau jour on vous volera. 

—Qu'a-t-il répondu? 

—Ce qu'il a répondu? 

—Oui. 

—Il a répondu: «Eh bien qu'est-ce que cela me fait qu'on me vole?» 

—Andrea, il y a quelque secrétaire à mécanique. 

—Comment cela? 

—Oui, qui prend le voleur dans une grille et qui joue un air. On m'a dit qu'il y en avait comme cela à la dernière exposition. 

—Il a tout bonnement un secrétaire en acajou auquel j'ai toujours vu la clef. 

—Et on ne le vole pas? 

—Non, les gens qui le servent lui sont tout dévoués. 

—Il doit y en avoir dans ce secrétaire-là, hein! de la monnaie? 

—Il y a peut-être\dots on ne peut pas savoir ce qu'il y a. 

—Et où est-il? 

—Au premier. 

—Fais-moi donc un peu le plan du premier, le petit, comme tu m'as fait celui du rez-de-chaussée. 

—C'est facile.» 

Et Andrea reprit la plume. 

«Au premier, vois-tu, il y a antichambre, salon; à droite du salon, bibliothèque et cabinet de travail; à gauche du salon, une chambre à coucher et un cabinet de toilette. C'est dans le cabinet de toilette qu'est le fameux secrétaire. 

—Et une fenêtre au cabinet de toilette? 

—Deux, là et là.» 

Et Andrea dessina deux fenêtres à la pièce qui, sur le plan, faisait l'angle et figurait comme un carré moins grand ajouté au carré long de la chambre à coucher. 

Caderousse devint rêveur. 

«Et va-t-il souvent à Auteuil? demanda-t-il. 

—Deux ou trois fois par semaine; demain, par exemple, il doit y aller passer la journée et la nuit. 

—Tu en es sûr? 

—Il m'a invité à y aller dîner. 

—À la bonne heure! voilà une existence, dit Caderousse: maison à la ville, maison à la campagne! 

—Voilà ce que c'est que d'être riche. 

—Et iras-tu dîner? 

—Probablement. 

—Quand tu y dînes, y couches-tu?  

—Quand cela me fait plaisir. Je suis chez le comte comme chez moi.» 

Caderousse regarda le jeune homme comme pour arracher la vérité du fond de son cœur. Mais Andrea tira une boîte à cigares de sa poche, y prit un havane, l'alluma tranquillement et commença à le fumer sans affectation. 

«Quand veux-tu les cinq cents francs? demanda-t-il à Caderousse. 

—Mais tout de suite, si tu les as.» 

Andrea tira vingt-cinq louis de sa poche. 

«Des jaunets, dit Caderousse; non, merci! 

—Eh bien, tu les méprises? 

—Je les estime, au contraire, mais je n'en veux pas. 

—Tu gagneras le change, imbécile: l'or vaut cinq sous. 

—C'est ça, et puis le changeur fera suivre l'ami Caderousse, et puis on lui mettra la main dessus, et puis il faudra qu'il dise quels sont les fermiers qui lui paient ses redevances en or. Pas de bêtises, le petit: de l'argent tout simplement, des pièces rondes à l'effigie d'un monarque quelconque. Tout le monde peut atteindre à une pièce de cinq francs. 

—Tu comprends bien que je n'ai pas cinq cents francs sur moi: il m'aurait fallu prendre un commissionnaire. 

—Eh bien, laisse-les chez toi, à ton concierge, c'est un brave homme, j'irai les prendre. 

—Aujourd'hui? 

—Non, demain; aujourd'hui je n'ai pas le temps. 

—Eh bien, soit; demain, en partant pour Auteuil, je les laisserai. 

—Je peux compter dessus? 

—Parfaitement. 

—C'est que je vais arrêter d'avance ma bonne, vois-tu. 

—Arrête. Mais ce sera fini, hein? tu ne me tourmenteras plus? 

—Jamais.» 

Caderousse était devenu si sombre, qu'Andrea craignit d'être forcé de s'apercevoir de ce changement. Il redoubla donc de gaieté et d'insouciance. 

«Comme tu es guilleret, dit Caderousse; on dirait que tu tiens déjà ton héritage! 

—Non pas, malheureusement!\dots Mais le jour où je le tiendrai\dots. 

—Eh bien? 

—Eh bien, on se souviendra des amis; je ne te dis que ça. 

—Oui, comme tu as bonne mémoire, justement!  

—Que veux-tu? je croyais que tu voulais me rançonner. 

—Moi! oh! quelle idée! moi qui, au contraire, vais encore te donner un conseil d'ami. 

—Lequel? 

—C'est de laisser ici le diamant que tu as à ton doigt. Ah çà! mais tu veux donc nous faire prendre? tu veux donc nous perdre tous les deux, que tu fais de pareilles bêtises? 

—Pourquoi cela? dit Andrea. 

—Comment! tu prends une livrée, tu te déguises en domestique, et tu gardes à ton doigt un diamant de quatre à cinq mille francs! 

—Peste! tu estimes juste! Pourquoi ne te fais-tu pas commissaire-priseur? 

—C'est que je m'y connais en diamants; j'en ai eu. 

—Je te conseille de t'en vanter», dit Andrea, qui, sans se courroucer, comme le craignait Caderousse, de cette nouvelle extorsion, livra complaisamment la bague. 

Caderousse la regarda de si près qu'il fut clair pour Andrea qu'il examinait si les arêtes de la coupe étaient bien vives. 

«C'est un faux diamant, dit Caderousse. 

—Allons donc, fit Andrea, plaisantes-tu? 

—Oh! ne te fâche pas, on peut voir.» 

Et Caderousse alla à la fenêtre, fit glisser le diamant sur le carreau; on entendit crier la vitre. 

«\textit{Confiteor}! dit Caderousse en passant le diamant à son petit doigt, je me trompais; mais ces voleurs de joailliers imitent si bien les pierres, qu'on n'ose plus aller voler dans les boutiques de bijouterie. C'est encore une branche d'industrie paralysée. 

—Eh bien, dit Andrea, est-ce fini? as-tu encore quelque chose à me demander? Ne te gêne pas pendant que tu y es. 

—Non, tu es un bon compagnon au fond. Je ne te retiens plus, et je tâcherai de me guérir de mon ambition. 

—Mais prends garde qu'en vendant ce diamant, il ne t'arrive ce que tu craignais qu'il ne t'arrivât pour l'or. 

—Je ne le vendrai pas, sois tranquille. 

—Non, pas d'ici à après-demain, du moins, pensa le jeune homme. 

—Heureux coquin! dit Caderousse, tu t'en vas retrouver tes laquais, tes chevaux, ta voiture et ta fiancée. 

—Mais oui, dit Andrea. 

—Dis donc, j'espère que tu me feras un joli cadeau de noces le jour où tu épouseras la fille de mon ami Danglars. 

—Je t'ai déjà dit que c'était une imagination que tu t'étais mise en tête.  

—Combien de dot? 

—Mais je te dis\dots. 

—Un million?» 

Andrea haussa les épaules. 

«Va pour un million, dit Caderousse, tu n'en auras jamais autant que je t'en désire. 

—Merci, dit le jeune homme. 

—Oh! c'est de bon cœur, ajouta Caderousse en riant de son gros rire. Attends, que je te reconduise. 

—Ce n'est pas la peine. 

—Si fait. 

—Pourquoi cela? 

—Oh! parce qu'il y a un petit secret à la porte; c'est une mesure de précaution que j'ai cru devoir adopter; serrure Huret et Fichet, revue et corrigée par Gaspard Caderousse. Je t'en confectionnerai une pareille quand tu seras capitaliste. 

—Merci, dit Andrea; je te ferai prévenir huit jours d'avance.» 

Ils se séparèrent. Caderousse resta sur le palier jusqu'à ce qu'il eût vu Andrea non seulement descendre les trois étages, mais encore traverser la cour. Alors il rentra précipitamment, ferma la porte avec soin, et se mit à étudier, en profond architecte, le plan que lui avait laissé Andrea. 

«Ce cher Benedetto, dit-il, je crois qu'il ne serait pas fâché d'hériter, et que celui qui avancera le jour où il doit palper ses cinq cent mille francs ne sera pas son plus méchant ami.» 