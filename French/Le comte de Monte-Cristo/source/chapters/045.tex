\chapter{La pluie de sang}

\lettrine{E}{n} entrant, le bijoutier jeta un regard interrogateur autour de lui; mais rien ne semblait faire naître les soupçons s'il n'en avait pas, rien ne semblait les confirmer s'il en avait. 

\zz
«Caderousse tenait toujours des deux mains ses billets et son or. La Carconte souriait à son hôte le plus agréablement qu'elle pouvait. 

«—Ah! ah! dit le bijoutier, il paraît que vous aviez peur de ne pas avoir votre compte, que vous repassiez votre trésor après mon départ. 

«—Non pas, dit Caderousse; mais l'événement qui nous en fait possesseur est si inattendu que nous n'y pouvons croire, et que, lorsque nous n'avons pas la preuve matérielle sous les yeux, nous croyons faire encore un rêve.» 

«Le bijoutier sourit. 

«—Est-ce que vous avez des voyageurs dans votre auberge? demanda-t-il. 

«—Non, répondit Caderousse, nous ne donnons point à coucher; nous sommes trop près de la ville, et personne ne s'arrête. 

«—Alors, je vais vous gêner horriblement? 

«—Nous gêner, vous! mon cher monsieur! dit gracieusement la Carconte, pas du tout, je vous jure. 

«—Voyons, où me mettez-vous? 

«—Dans la chambre là-haut. 

«—Mais n'est-ce pas votre chambre? 

«—Oh! n'importe; nous avons un second lit dans la pièce à côté de celle-ci. 

«Caderousse regarda avec étonnement sa femme. Le bijoutier chantonna un petit air en se chauffant le dos à un fagot que la Carconte venait d'allumer dans la cheminée pour sécher son hôte. 

«Pendant ce temps, elle apportait sur un coin de la table où elle avait étendu une serviette les maigres restes d'un dîner, auxquels elle joignit deux ou trois œufs frais. 

«Caderousse avait renfermé de nouveau les billets dans son portefeuille, son or dans un sac, et le tout dans son armoire. Il se promenait de long en large, sombre et pensif, levant de temps en temps la tête sur le bijoutier, qui se tenait tout fumant devant l'âtre, et qui, à mesure qu'il se séchait d'un côté, se tournait de l'autre. 

«—Là, dit la Carconte en posant une bouteille de vin sur la table, quand vous voudrez souper tout est prêt. 

«—Et vous? demanda Joannès. 

«—Moi, je ne souperai pas, répondit Caderousse. 

«—Nous avons dîné très tard, se hâta de dire la Carconte. 

«—Je vais donc souper seul? fit le bijoutier. 

«—Nous vous servirons, répondit la Carconte avec un empressement qui ne lui était pas habituel, même envers ses hôtes payants. 

«De temps en temps Caderousse lançait sur elle un regard rapide comme un éclair. 

«L'orage continuait. 

«—Entendez-vous, entendez-vous? dit la Carconte; vous avez, ma foi, bien fait de revenir. 

«—Ce qui n'empêche pas, dit le bijoutier, que si, pendant mon souper, l'ouragan s'apaise, je me remettrai en route. 

«—C'est le mistral, dit Caderousse en secouant la tête; nous en avons pour jusqu'à demain. 

«Et il poussa un soupir. 

«—Ma foi, dit le bijoutier en se mettant à table, tant pis pour ceux qui sont dehors. 

«—Oui, reprit la Carconte, ils passeront une mauvaise nuit. 

«Le bijoutier commença de souper, et la Carconte continua d'avoir pour lui tous les petits soins d'une hôtesse attentive; elle d'ordinaire si quinteuse et si revêche, elle était devenue un modèle de prévenance et de politesse. Si le bijoutier l'eût connue auparavant, un si grand changement l'eût certes étonné et n'eût pas manqué de lui inspirer quelque soupçon. Quant à Caderousse, il ne disait pas une parole, continuant sa promenade et paraissant hésiter même à regarder son hôte. 

«Lorsque le souper fut terminé, Caderousse alla lui-même ouvrir la porte. 

«—Je crois que l'orage se calme, dit-il. 

«Mais en ce moment, comme pour lui donner un démenti, un coup de tonnerre terrible ébranla la maison, et une bouffée de vent mêlée de pluie entra, qui éteignit la lampe. 

«Caderousse referma la porte; sa femme alluma une chandelle au brasier mourant.  

«—Tenez, dit-elle au bijoutier, vous devez être fatigué; j'ai mis des draps blancs au lit, montez vous coucher et dormez bien. 

«Joannès resta encore un instant pour s'assurer que l'ouragan ne se calmait point, et lorsqu'il eut acquis la certitude que le tonnerre et la pluie ne faisaient qu'aller en augmentant, il souhaita le bonjour à ses hôtes et monta l'escalier. 

«Il passait au-dessus de ma tête, et j'entendais chaque marche craquer sous ses pas. 

«La Carconte le suivit d'un œil avide, tandis qu'au contraire Caderousse lui tournait le dos et ne regardait pas même de son côté.  

«Tous ces détails, qui sont revenus à mon esprit depuis ce temps-là, ne me frappèrent point au moment où ils se passaient sous mes yeux; il n'y avait, à tout prendre, rien que de naturel dans ce qui arrivait, et, à part l'histoire du diamant qui me paraissait un peu invraisemblable, tout allait de source. Aussi comme j'étais écrasé de fatigue, que je comptais profiter moi-même du premier répit que la tempête donnerait aux éléments, je résolus de dormir quelques heures et de m'éloigner au milieu de la nuit. 

«J'entendais dans la pièce au-dessus le bijoutier, qui prenait de son côté toutes ses dispositions pour passer la meilleure nuit possible. Bientôt son lit craqua sous lui; il venait de se coucher.  

«Je sentais mes yeux qui se fermaient malgré moi, et comme je n'avais conçu aucun soupçon, je ne tentai point de lutter contre le sommeil; je jetai un dernier regard sur l'intérieur de la cuisine. Caderousse était assis à côté d'une longue table, sur un de ces bancs de bois qui, dans les auberges de village, remplacent les chaises; il me tournait le dos, de sorte que je ne pouvais voir sa physionomie; d'ailleurs eût-il été dans la position contraire, la chose m'eût encore été impossible, attendu qu'il tenait sa tête ensevelie dans ses deux mains. 

«La Carconte le regarda quelque temps, haussa les épaules et vint s'asseoir en face de lui. 

«En ce moment la flamme mourante gagna un reste de bois sec oublié par elle; une lueur un peu plus vive éclaira le sombre intérieur\dots. La Carconte tenait ses yeux fixés sur son mari, et comme celui-ci restait toujours dans la même position, je la vis étendre vers lui sa main crochue, et elle le toucha au front. 

«Caderousse tressaillit. Il me sembla que la femme remuait les lèvres, mais, soit qu'elle parlât tout à fait bas, soit que mes sens fussent déjà engourdis par le sommeil, le bruit de sa parole n'arriva point jusqu'à moi. Je ne voyais même plus qu'à travers un brouillard et avec ce doute précurseur du sommeil pendant lequel on croit que l'on commence un rêve. Enfin mes yeux se fermèrent, et je perdis conscience de moi-même. 

«J'étais au plus profond de mon sommeil, lorsque je fus réveillé par un coup de pistolet, suivi d'un cri terrible. Quelques pas chancelants retentirent sur le plancher de la chambre, et une masse inerte vint s'abattre dans l'escalier, juste au-dessus de ma tête. 

«Je n'étais pas encore bien maître de moi. J'entendais des gémissements, puis des cris étouffés comme ceux qui accompagnent une lutte. 

«Un dernier cri, plus prolongé que les autres et qui dégénéra en gémissements, vint me tirer complètement de ma léthargie. 

«Je me soulevai sur un bras, j'ouvris les yeux, qui ne virent rien dans les ténèbres, et je portai la main à mon front, sur lequel il me semblait que dégouttait à travers les planches de l'escalier une pluie tiède et abondante. 

«Le plus profond silence avait succédé à ce bruit affreux. J'entendis les pas d'un homme qui marchait au-dessus de ma tête, ses pas firent craquer l'escalier. L'homme descendit dans la salle inférieure, s'approcha de la cheminée et alluma une chandelle. 

«Cet homme, c'était Caderousse; il avait le visage pâle, et sa chemise était tout ensanglantée. 

«La chandelle allumée, il remonta rapidement l'escalier, et j'entendis de nouveau ses pas rapides et inquiets. 

«Un instant après il redescendit. Il tenait à la main l'écrin; il s'assura que le diamant était bien dedans, chercha un instant dans laquelle de ses poches il le mettrait; puis, sans doute, ne considérant point sa poche comme une cachette assez sûre, il le roula dans son mouchoir rouge, qu'il tourna autour de son cou. 

«Puis il courut à l'armoire, en tira ses billets et son or, mit les uns dans le gousset de son pantalon, l'autre dans la poche de sa veste, prit deux ou trois chemises, et, s'élançant vers la porte, il disparut dans l'obscurité. Alors tout devint clair et lucide pour moi; je me reprochai ce qui venait d'arriver, comme si j'eusse été le vrai coupable. Il me sembla entendre des gémissements: le malheureux bijoutier pouvait n'être pas mort; peut-être était-il en mon pouvoir, en lui portant secours, de réparer une partie du mal non pas que j'avais fait, mais que j'avais laissé faire. J'appuyai mes épaules contre une de ces planches mal jointes qui séparaient l'espèce de tambour dans lequel j'étais couché de la salle inférieure; les planches cédèrent, et je me trouvai dans la maison. 

«Je courus à la chandelle, et je m'élançai dans l'escalier; un corps le barrait en travers, c'était le cadavre de la Carconte. 

«Le coup de pistolet que j'avais entendu avait été tiré sur elle: elle avait la gorge traversée de part en part, et outre sa double blessure qui coulait à flots, elle vomissait le sang par la bouche. Elle était tout à fait morte. J'enjambai par-dessus son corps, et je passai. 

«La chambre offrait l'aspect du plus affreux désordre. Deux ou trois meubles étaient renversés; les draps, auxquels le malheureux bijoutier s'était cramponné, traînaient par la chambre: lui-même était couché à terre, la tête appuyée contre le mur, nageant dans une mare de sang qui s'échappait de trois larges blessures reçues dans la poitrine. 

«Dans la quatrième était resté un long couteau de cuisine, dont on ne voyait que le manche. 

«Je marchai sur le second pistolet qui n'était point parti, la poudre étant probablement mouillée. 

«Je m'approchai du bijoutier; il n'était pas mort effectivement: au bruit que je fis, à l'ébranlement du plancher surtout, il rouvrit des yeux hagards, parvint à les fixer un instant sur moi, remua les lèvres comme s'il voulait parler, et expira. 

«Cet affreux spectacle m'avait rendu presque insensé; du moment où je ne pouvais plus porter de secours à personne je n'éprouvais plus qu'un besoin, celui de fuir. Je me précipitai dans l'escalier, en enfonçant mes mains dans mes cheveux et en poussant un rugissement de terreur. 

«Dans la salle inférieure, il y avait cinq ou six douaniers et deux ou trois gendarmes, toute une troupe armée. 

«On s'empara de moi; je n'essayai même pas de faire résistance, je n'étais plus le maître de mes sens. J'essayai de parler, je poussai quelques cris inarticulés, voilà tout. 

«Je vis que les douaniers et les gendarmes me montraient du doigt; j'abaissai les yeux sur moi-même, j'étais tout couvert de sang. Cette pluie tiède que j'avais sentie tomber sur moi à travers les planches de l'escalier, c'était le sang de la Carconte. 

«Je montrai du doigt l'endroit où j'étais caché. 

«—Que veut-il dire? demanda un gendarme. 

«Un douanier alla voir. 

«—Il veut dire qu'il est passé par là, répondit-il. 

«Et il montra le trou par lequel j'avais passé effectivement.  

«Alors, je compris qu'on me prenait pour l'assassin. Je retrouvai la voix, je retrouvai la force; je me dégageai des mains des deux hommes qui me tenaient, en m'écriant: 

«—Ce n'est pas moi! ce n'est pas moi! 

«Deux gendarmes me mirent en joue avec leurs carabines. 

«—Si tu fais un mouvement, dirent-ils, tu es mort. 

«—Mais, m'écriai-je, puisque je vous répète que ce n'est pas moi! 

«—Tu conteras ta petite histoire aux juges de Nîmes, répondirent-ils. En attendant, suis-nous; et si nous avons un conseil à te donner, c'est de ne pas faire résistance. 

«Ce n'était point mon intention, j'étais brisé par l'étonnement et par la terreur. On me mit les menottes, on m'attacha à la queue d'un cheval, et l'on me conduisit à Nîmes. 

«J'avais été suivi par un douanier; il m'avait perdu de vue aux environs de la maison, il s'était douté que j'y passerais la nuit; il avait été prévenir ses compagnons, et ils étaient arrivés juste pour entendre le coup de pistolet et pour me prendre au milieu de telles preuves de culpabilité, que je compris tout de suite la peine que j'aurais à faire reconnaître mon innocence. 

«Aussi, ne m'attachai-je qu'à une chose: ma première demande au juge d'instruction fut pour le prier de faire chercher partout un certain abbé Busoni, qui s'était arrêté dans la journée à l'auberge du Pont-du-Gard. Si Caderousse avait inventé une histoire, si cet abbé n'existait pas, il était évident que j'étais perdu, à moins que Caderousse ne fût pris à son tour et n'avouât tout. 

«Deux mois s'écoulèrent pendant lesquels, je dois le dire à la louange de mon juge, toutes les recherches furent faites pour retrouver celui que je lui demandais. J'avais déjà perdu tout espoir. Caderousse n'avait point été pris. J'allais être jugé à la première session, lorsque le 8 septembre, c'est-à-dire trois mois et cinq jours après l'événement, l'abbé Busoni, sur lequel je n'espérais plus, se présenta à la geôle, disant qu'il avait appris qu'un prisonnier désirait lui parler. Il avait su, disait-il, la chose à Marseille, et il s'empressait de se rendre à mon désir. 

«Vous comprenez avec quelle ardeur je le reçus; je lui racontai tout ce dont j'avais été témoin, j'abordai avec inquiétude l'histoire du diamant; contre mon attente elle était vraie de point en point; contre mon attente encore, il ajouta une foi entière à tout ce que je lui dis. Ce fut alors qu'entraîné par sa douce charité, reconnaissant en lui une profonde connaissance des mœurs de mon pays, pensant que le pardon du seul crime que j'eusse commis pouvait peut-être descendre de ses lèvres si charitables, je lui racontai, sous le sceau de la confession, l'aventure d'Auteuil dans tous ses détails. Ce que j'avais fait par entraînement obtint le même résultat que si je l'eusse fait par calcul, l'aveu de ce premier assassinat, que rien ne me forçait de lui révéler, lui prouva que je n'avais pas commis le second, et il me quitta en m'ordonnant d'espérer, et en promettant de faire tout ce qui serait en son pouvoir pour convaincre mes juges de mon innocence. 

«J'eus la preuve qu'en effet il s'était occupé de moi quand je vis ma prison s'adoucir graduellement, et quand j'appris qu'on attendrait pour me juger les assises qui devaient suivre celles pour lesquelles on se rassemblait. 

«Dans cet intervalle, la Providence permit que Caderousse fût pris à l'étranger et ramené en France. Il avoua tout, rejetant la préméditation et surtout l'instigation sur sa femme. Il fut condamné aux galères perpétuelles, et moi mis en liberté.  

—Et ce fut alors, dit Monte-Cristo, que vous vous présentâtes chez moi porteur d'une lettre de l'abbé Busoni? 

—Oui, Excellence, il avait pris à moi un intérêt visible. 

«—Votre état de contrebandier vous perdra, me dit-il; si vous sortez d'ici, quittez-le. 

«—Mais mon père, demandai-je, comment voulez-vous que je vive et que je fasse vivre ma pauvre sœur? 

«—Un de mes pénitents, me répondit-il, a une grande estime pour moi, et m'a chargé de lui chercher un homme de confiance. Voulez-vous être cet homme? je vous adresserai à lui. 

«—Ô mon père! m'écriai-je, que de bonté! 

«—Mais vous me jurez que je n'aurai jamais à me repentir.» 

«J'étendis la main pour faire serment. 

«—C'est inutile, dit-il, je connais et j'aime les Corses, voici ma recommandation. 

«Et il écrivit les quelques lignes que je vous remis, et sur lesquelles Votre Excellence eut la bonté de me prendre à son service. Maintenant je le demande avec orgueil à Votre Excellence, a-t-elle jamais eu à se plaindre de moi?  

—Non, répondit le comte; et, je le confesse avec plaisir, vous êtes un bon serviteur, Bertuccio, quoique vous manquiez de confiance. 

—Moi, monsieur le comte! 

—Oui, vous. Comment se fait-il que vous ayez une sœur et un fils adoptif, et que, cependant vous ne m'ayez jamais parlé ni de l'une ni de l'autre! 

—Hélas! Excellence, c'est qu'il me reste à vous dire la partie la plus triste de ma vie. Je partis pour la Corse. J'avais hâte, vous le comprenez bien, de revoir et de consoler ma pauvre sœur; mais quand j'arrivai à Rogliano, je trouvai la maison en deuil; il y avait eu une scène horrible et dont les voisins gardent encore le souvenir! Ma pauvre sœur, selon mes conseils, résistait aux exigences de Benedetto, qui, à chaque instant, voulait se faire donner tout l'argent qu'il y avait à la maison. Un matin, il la menaça, et disparut pendant toute la journée. Elle pleura, car cette chère Assunta avait pour le misérable un cœur de mère. Le soir vint, elle l'attendit sans se coucher. Lorsque, à onze heures, il rentra avec deux de ses amis, compagnons ordinaires de toutes ses folies, alors elle lui tendit les bras; mais eux s'emparèrent d'elle, et l'un des trois, je tremble que ce ne soit cet infernal enfant, l'un des trois s'écria: 

«—Jouons à la question, et il faudra bien qu'elle avoue où est son argent.  

«Justement le voisin Wasilio était à Bastia; sa femme seule était restée à la maison. Nul, excepté elle, ne pouvait ni voir ni entendre ce qui se passait chez ma sœur. Deux retinrent la pauvre Assunta, qui ne pouvant croire à la possibilité d'un pareil crime, souriait à ceux qui allaient devenir ses bourreaux, le troisième alla barricader portes et fenêtres, puis il revint, et tous trois réunis, étouffant les cris que la terreur lui arrachait devant ces préparatifs plus sérieux, approchèrent les pieds d'Assunta du brasier sur lequel ils comptaient pour lui faire avouer où était caché notre petit trésor; mais, dans la lutte, le feu prit à ses vêtements: ils lâchèrent alors la patiente, pour ne pas être brûlés eux-mêmes. Tout en flammes elle courut à la porte, mais la porte était fermée. 

«Elle s'élança vers la fenêtre, mais la fenêtre était barricadée. Alors la voisine entendit des cris affreux: c'était Assunta qui appelait au secours. Bientôt sa voix fut étouffée; les cris devinrent des gémissements, et le lendemain, après une nuit de terreur et d'angoisses quand la femme de Wasilio se hasarda de sortir de chez elle et fit ouvrir la porte de notre maison par le juge, on trouva Assunta à moitié brûlée, mais respirant encore, les armoires forcées, l'argent disparu. Quant à Benedetto, il avait quitté Rogliano pour n'y plus revenir; depuis ce jour je ne l'ai pas revu, et je n'ai pas même entendu parler de lui. 

«Ce fut, reprit Bertuccio, après avoir appris ces tristes nouvelles, que j'allai à Votre Excellence. Je n'avais plus à vous parler de Benedetto, puisqu'il avait disparu, ni de ma sœur, puisqu'elle était morte. 

—Et qu'avez-vous pensé de cet événement? demanda Monte-Cristo. 

—Que c'était le châtiment du crime que j'avais commis, répondit Bertuccio. Ah! ces Villefort, c'était une race maudite. 

—Je le crois, murmura le comte avec un accent lugubre. 

—Et maintenant, n'est-ce pas, reprit Bertuccio, Votre Excellence comprend que cette maison que je n'ai pas revue depuis, que ce jardin où je me suis retrouvé tout à coup, que cette place où j'ai tué un homme, ont pu me causer ces sombres émotions dont vous avez voulu connaître la source; car enfin je ne suis pas bien sûr que devant moi, là, à mes pieds, M. de Villefort ne soit pas couché dans la fosse qu'il avait creusé pour son enfant. 

—En effet, tout est possible, dit Monte-Cristo en se levant du banc où il était assis; même, ajouta-t-il tout bas, que le procureur du roi ne soit pas mort. L'abbé Busoni a bien fait de vous envoyer à moi. Vous avez bien fait de me raconter votre histoire, car je n'aurai pas de mauvaises pensées à votre sujet. Quant à ce Benedetto si mal nommé, n'avez-vous jamais essayé de retrouver sa trace? n'avez-vous jamais cherché à savoir ce qu'il était devenu? 

—Jamais, si j'avais su où il était, au lieu d'aller à lui, j'aurais fui comme devant un monstre. Non heureusement, jamais je n'en ai entendu parler par qui que ce soit au monde, j'espère qu'il est mort. 

—N'espérez pas, Bertuccio, dit le comte; les méchants ne meurent pas ainsi, car Dieu semble les prendre sous sa garde pour en faire l'instrument de ses vengeances. 

—Soit, dit Bertuccio. Tout ce que je demande au ciel seulement, c'est de ne le revoir jamais. Maintenant, continua l'intendant en baissant la tête, vous savez tout, monsieur le comte; vous êtes mon juge ici-bas comme Dieu le sera là-haut; ne me direz-vous point quelques paroles de consolation? 

—Vous avez raison, en effet, et je puis vous dire ce que vous dirait l'abbé Busoni: celui que vous avez frappé, ce Villefort, méritait un châtiment pour ce qu'il avait fait à vous et peut-être pour autre chose encore. Benedetto, s'il vit, servira, comme je vous l'ai dit, à quelque vengeance divine, puis sera puni à son tour. Quant à vous, vous n'avez en réalité qu'un reproche à vous adresser: demandez-vous pourquoi, ayant enlevé cet enfant à la mort, vous ne l'avez pas rendu à sa mère: là est le crime, Bertuccio. 

—Oui, monsieur, là est le crime et le véritable crime, car en cela j'ai été un lâche. Une fois que j'eus rappelé l'enfant à la vie, je n'avais qu'une chose à faire, vous l'avez dit, c'était de le renvoyer à sa mère. Mais, pour cela, il me fallait faire des recherches, attirer l'attention, me livrer peut-être; je n'ai pas voulu mourir, je tenais à la vie par ma sœur, par l'amour-propre inné chez nous autres de rester entiers et victorieux dans notre vengeance; et puis enfin, peut-être, tenais-je simplement à la vie par l'amour même de la vie. Oh! moi, je ne suis pas un brave comme mon pauvre frère!» 

Bertuccio cacha son visage dans ses deux mains, et Monte-Cristo attacha sur lui un long et indéfinissable regard. 

Puis, après un instant de silence, rendu plus solennel encore par l'heure et par le lieu: 

«Pour terminer dignement cet entretien, qui sera le dernier sur ces aventures, monsieur Bertuccio, dit le comte avec un accent de mélancolie qui ne lui était pas habituel, retenez bien mes paroles, je les ai souvent entendu prononcer par l'abbé Busoni lui-même: À tous maux il est deux remèdes: le temps et le silence. Maintenant, monsieur Bertuccio, laissez-moi me promener un instant dans ce jardin. Ce qui est une émotion poignante pour vous, acteur dans cette scène, sera pour moi une sensation presque douce et qui donnera un double prix à cette propriété. Les arbres, voyez-vous, monsieur Bertuccio ne plaisent que parce qu'ils font de l'ombre, et l'ombre elle-même ne plaît que parce qu'elle est pleine de rêveries et de visions. Voilà que j'ai acheté un jardin croyant acheter un simple enclos fermé de murs, et point du tout, tout à coup cet enclos se trouve être un jardin tout plein de fantômes, qui n'étaient point portés sur le contrat. Or, j'aime les fantômes; je n'ai jamais entendu dire que les morts eussent fait en six mille ans autant de mal que les vivants en font en un jour. Rentrez donc, monsieur Bertuccio, et allez dormir en paix. Si votre confesseur, au moment suprême, est moins indulgent que ne le fut l'abbé Busoni, faites-moi venir si je suis encore de ce monde, je vous trouverai des paroles qui berceront doucement votre âme au moment où elle sera prête à se mettre en route pour faire ce rude voyage qu'on appelle l'éternité.» 

Bertuccio s'inclina respectueusement devant le comte, et s'éloigna en poussant un soupir. 

Monte-Cristo resta seul; et, faisant quatre pas en avant: 

«Ici, près de ce platane, murmura-t-il, la fosse où l'enfant fut déposé: là-bas, la petite porte par laquelle on entrait dans le jardin; à cet angle, l'escalier dérobé qui conduit à la chambre à coucher. Je ne crois pas avoir besoin d'inscrire tout cela sur mes tablettes, car voilà devant mes yeux, autour de moi, sous mes pieds, le plan en relief, le plan vivant.» 

Et le comte, après un dernier tour dans ce jardin, alla retrouver sa voiture. Bertuccio, qui le voyait rêveur, monta sans rien dire sur le siège auprès du cocher. 

La voiture reprit le chemin de Paris. 

Le soir même, à son arrivée à la maison des Champs-Élysées, le comte de Monte-Cristo visita toute l'habitation comme eût pu le faire un homme familiarisé avec elle depuis de longues années; pas une seule fois, quoiqu'il marchât le premier, il n'ouvrit une porte pour une autre, et ne prit un escalier ou un corridor qui ne le conduisît pas directement où il comptait aller. Ali l'accompagnait dans cette revue nocturne. Le comte donna à Bertuccio plusieurs ordres pour l'embellissement ou la distribution nouvelle du logis, et tirant sa montre, il dit au Nubien attentif: 

«Il est onze heures et demie, Haydée ne peut tarder à arriver. A-t-on prévenu les femmes françaises?» 

Ali étendit la main vers l'appartement destiné à la belle Grecque, et qui était tellement isolé qu'en cachant la porte derrière une tapisserie on pouvait visiter toute la maison sans se douter qu'il y eût là un salon et deux chambres habités; Ali, disons-nous donc, étendit la main vers l'appartement, montra le nombre trois avec les doigts de sa main gauche, et sur cette même main, mise à plat, appuyant sa tête, ferma les yeux en guise de sommeil.  

«Ah! fit Monte-Cristo, habitué à ce langage, elles sont trois qui attendent dans la chambre à coucher, n'est-ce pas? 

—Oui, fit Ali en agitant la tête de haut en bas. 

—Madame sera fatiguée ce soir, continua Monte-Cristo, et sans doute elle voudra dormir; qu'on ne la fasse pas parler: les suivantes françaises doivent seulement saluer leur nouvelle maîtresse et se retirer; vous veillerez à ce que la suivante grecque ne communique pas avec les suivantes françaises.» 

Ali s'inclina. Bientôt on entendit héler le concierge; la grille s'ouvrit, une voiture roula dans l'allée et s'arrêta devant le perron. Le comte descendit; la portière était déjà ouverte; il tendit la main à une jeune femme enveloppée d'une mante de soie verte toute brodée d'or qui lui couvrait la tête. 

La jeune femme prit la main qu'on lui tendait, la baisa avec un certain amour mêlé de respect, et quelques mots furent échangés, tendrement de la part de la jeune femme et avec une douce gravité de la part du comte, dans cette langue sonore que le vieil Homère a mise dans la bouche de ses dieux. 

Alors, précédé d'Ali qui portait un flambeau de cire rose, la jeune femme, laquelle n'était autre que cette belle Grecque, compagne ordinaire de Monte-Cristo en Italie, fut conduite à son appartement, puis le comte se retira dans le pavillon qu'il s'était réservé.  

À minuit et demi, toutes les lumières étaient éteintes dans la maison, et l'on eût pu croire que tout le monde dormait. 