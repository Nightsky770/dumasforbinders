\chapter{Le voyage}

\lettrine{M}{onte-Cristo} poussa un cri de joie en voyant les deux jeunes gens ensemble. 

\zz
«Ah! ah! dit-il. Eh bien, j'espère que tout est fini, éclairci, arrangé? 

—Oui, dit Beauchamp, des bruits absurdes qui sont tombés d'eux-mêmes, et, qui maintenant, s'ils se renouvelaient, m'auraient pour premier antagoniste. Ainsi donc, ne parlons plus de cela. 

—Albert vous dira, reprit le comte, que c'est le conseil que je lui avais donné. Tenez, ajouta-t-il, vous me voyez au reste achevant la plus exécrable matinée que j'aie jamais passée, je crois. 

—Que faites-vous? dit Albert, vous mettez de l'ordre dans vos papiers, ce me semble? 

—Dans mes papiers, Dieu merci non! il y a toujours dans mes papiers un ordre merveilleux, attendu que je n'ai pas de papiers, mais dans les papiers de M. Cavalcanti. 

—De M. Cavalcanti? demanda Beauchamp. 

—Eh oui! ne savez-vous pas que c'est un jeune homme que lance le comte? dit Morcerf. 

—Non pas, entendons-nous bien, répondit Monte-Cristo, je ne lance personne, et M. Cavalcanti moins que tout autre. 

—Et qui va épouser Mlle Danglars en mon lieu et place; ce qui, continua Albert en essayant de sourire, comme vous pouvez bien vous en douter, mon cher Beauchamp, m'affecte cruellement. 

—Comment! Cavalcanti épouse Mlle Danglars? demanda Beauchamp. 

—Ah çà! mais vous venez donc du bout du monde? dit Monte-Cristo; vous, un journaliste, le mari de la Renommée! Tout Paris ne parle que de cela. 

—Et c'est vous, comte, qui avez fait ce mariage? demanda Beauchamp. 

—Moi? Oh! silence, monsieur le nouvelliste, n'allez pas dire de pareilles choses! Moi, bon Dieu! faire un mariage? Non, vous ne me connaissez pas; je m'y suis au contraire opposé de tout mon pouvoir, j'ai refusé de faire la demande. 

—Ah! je comprends, dit Beauchamp: à cause de notre ami Albert? 

—À cause de moi, dit le jeune homme; oh! non, par ma foi! Le comte me rendra la justice d'attester que je l'ai toujours prié, au contraire, de rompre ce projet, qui heureusement est rompu. Le comte prétend que ce n'est pas lui que je dois remercier; soit, j'élèverai, comme les anciens, un autel \textit{Deo ignoto}. 

—Écoutez, dit Monte-Cristo, c'est si peu moi, que je suis en froid avec le beau-père et avec le jeune homme; il n'y a que Mlle Eugénie, laquelle ne me paraît pas avoir une profonde vocation pour le mariage, qui, en voyant à quel point j'étais peu disposé à la faire renoncer à sa chère liberté, m'ait conservé son affection. 

—Et vous dites que ce mariage est sur le point de se faire? 

—Oh! mon Dieu! oui, malgré tout ce que j'ai pu dire. Moi, je ne connais pas le jeune homme, on le prétend riche et de bonne famille, mais pour moi ces choses sont de simples \textit{on dit}. J'ai répété tout cela à satiété à M. Danglars; mais il est entiché de son Lucquois. J'ai été jusqu'à lui faire part d'une circonstance qui, pour moi, était plus grave: le jeune homme a été changé en nourrice, enlevé par des Bohémiens ou égaré par son précepteur, je ne sais pas trop. Mais ce que je sais, c'est que son père l'a perdu de vue depuis plus de dix années; ce qu'il a fait pendant ces dix années de vie errante, Dieu seul le sait. Eh bien, rien de tout cela n'y a fait. On m'a chargé d'écrire au major, de lui demander des papiers; ces papiers, les voilà. Je les leur envoie, mais, comme Pilate, en me lavant les mains. 

—Et Mlle d'Armilly, demanda Beauchamp, quelle mine vous fait-elle à vous, qui lui enlevez son élève? 

—Dame! je ne sais pas trop: mais il paraît qu'elle part pour l'Italie. Mme Danglars m'a parlé d'elle et m'a demandé des lettres de recommandation pour les impresarii; je lui ai donné un mot pour le directeur du théâtre Valle, qui m'a quelques obligations. Mais qu'avez-vous donc, Albert? vous avez l'air tout attristé; est-ce que, sans vous en douter vous êtes amoureux de Mlle Danglars, par exemple? 

—Pas que je sache», dit Albert en souriant tristement. 

Beauchamp se mit à regarder les tableaux. 

«Mais enfin, continua Monte-Cristo, vous n'êtes pas dans votre état ordinaire. Voyons, qu'avez-vous? dites. 

—J'ai la migraine, dit Albert. 

—Eh bien, mon cher vicomte, dit Monte-Cristo, j'ai en ce cas un remède infaillible à vous proposer, remède qui m'a réussi à moi chaque fois que j'ai éprouvé quelque contrariété. 

—Lequel? demanda le jeune homme. 

—Le déplacement. 

—En vérité? dit Albert. 

—Oui; et tenez, comme en ce moment-ci je suis excessivement contrarié, je me déplace. Voulez-vous que nous nous déplacions ensemble? 

—Vous, contrarié, comte! dit Beauchamp, et de quoi donc? 

—Pardieu! vous en parlez fort à votre aise, vous; je voudrais bien vous voir avec une instruction se poursuivant dans votre maison! 

—Une instruction! quelle instruction? 

—Eh! celle que M. de Villefort dresse contre mon aimable assassin donc, une espèce de brigand échappé du bagne, à ce qu'il paraît. 

—Ah! c'est vrai, dit Beauchamp, j'ai lu le fait dans les journaux. Qu'est-ce que c'est que ce Caderousse? 

—Eh bien\dots mais il paraît que c'est un Provençal. M. de Villefort en a entendu parler quand il était à Marseille, et M. Danglars se rappelle l'avoir vu. Il en résulte que M. le procureur du roi prend l'affaire fort à cœur, qu'elle a, à ce qu'il paraît, intéressé au plus haut degré le préfet de police, et que, grâce à cet intérêt dont je suis on ne peut plus reconnaissant, on m'envoie ici depuis quinze jours tous les bandits qu'on peut se procurer dans Paris et dans la banlieue, sous prétexte que ce sont les assassins de M. Caderousse; d'où il résulte que, dans trois mois, si cela continue, il n'y aura pas un voleur ni un assassin dans ce beau royaume de France qui ne connaisse le plan de ma maison sur le bout de son doigt; aussi je prends le parti de la leur abandonner tout entière, et de m'en aller aussi loin que la terre pourra me porter. Venez avec moi, vicomte, je vous emmène. 

—Volontiers. 

—Alors, c'est convenu? 

—Oui, mais où cela? 

—Je vous l'ai dit, où l'air est pur, où le bruit endort, où, si orgueilleux que l'on soit, on se sent humble et l'on se trouve petit. J'aime cet abaissement, moi, que l'on dit maître de l'univers comme Auguste. 

—Où allez-vous, enfin? 

—À la mer, vicomte, à la mer. Je suis un marin, voyez-vous, tout enfant, j'ai été bercé dans les bras du vieil Océan et sur le sein de la belle Amphitrite; j'ai joué avec le manteau vert de l'un et la robe azurée de l'autre; j'aime la mer comme on aime une maîtresse, et quand il y a longtemps que je ne l'ai vue, je m'ennuie d'elle. 

—Allons, comte, allons! 

—À la mer? 

—Oui. 

—Vous acceptez? 

—J'accepte. 

—Eh bien, vicomte, il y aura ce soir dans ma cour un briska de voyage, dans lequel on peut s'étendre comme dans son lit; ce briska sera attelé de quatre chevaux de poste. Monsieur Beauchamp, on y tient quatre très facilement. Voulez-vous venir avec nous? je vous emmène! 

—Merci, je viens de la mer. 

—Comment! vous venez de la mer? 

—Oui, ou à peu près. Je viens de faire un petit voyage aux îles Borromées. 

—Qu'importe! venez toujours, dit Albert. 

—Non, cher Morcerf, vous devez comprendre que du moment où je refuse, c'est que la chose est impossible. D'ailleurs, il est important, ajouta-t-il en baissant la voix, que je reste à Paris, ne fût-ce que pour surveiller la boîte du journal. 

—Ah! vous êtes un bon et excellent ami, dit Albert; oui, vous avez raison, veillez, surveillez, Beauchamp, et tâchez de découvrir l'ennemi à qui cette révélation a dû le jour.» 

Albert et Beauchamp se séparèrent: leur dernière poignée de main renfermait tous les sens que leurs lèvres ne pouvaient exprimer devant un étranger. 

«Excellent garçon que Beauchamp! dit Monte-Cristo après le départ du journaliste; n'est-ce pas, Albert? 

—Oh! oui, un homme de cœur, je vous en réponds; aussi je l'aime de toute mon âme. Mais, maintenant que nous voilà seuls, quoique la chose me soit à peu près égale, où allons-nous? 

—En Normandie, si vous voulez bien. 

—À merveille. Nous sommes tout à fait à la campagne, n'est-ce pas? point de société, point de voisins? 

—Nous sommes tête à tête avec des chevaux pour courir, des chiens pour chasser, et une barque pour pêcher, voilà tout. 

—C'est ce qu'il me faut; je préviens ma mère, et je suis à vos ordres. 

—Mais, dit Monte-Cristo, vous permettra-t-on? 

—Quoi? 

—De venir en Normandie. 

—À moi? est-ce que je ne suis pas libre? 

—D'aller où vous voulez, seul, je le sais bien, puisque je vous ai rencontré échappé par l'Italie. 

—Eh bien? 

—Mais de venir avec l'homme qu'on appelle le comte de Monte-Cristo? 

—Vous avez peu de mémoire, comte. 

—Comment cela? 

—Ne vous ai-je pas dit toute la sympathie que ma mère avait pour vous? 

—Souvent femme varie, a dit François I\ier; la femme, c'est l'onde, a dit Shakespeare; l'un était un grand roi et l'autre un grand poète, et chacun d'eux devait connaître la femme. 

—Oui, la femme; mais ma mère n'est point la femme, c'est une femme. 

—Permettez-vous à un pauvre étranger de ne point comprendre parfaitement toutes les subtilités de votre langue? 

—Je veux dire que ma mère est avare de ses sentiments, mais qu'une fois qu'elle les a accordés, c'est pour toujours. 

—Ah! vraiment, dit en soupirant Monte-Cristo; et vous croyez qu'elle me fait l'honneur de m'accorder un sentiment autre que la plus parfaite indifférence? 

—Écoutez! je vous l'ai déjà dit et je vous le répète, reprit Morcerf, il faut que vous soyez réellement un homme bien étrange et bien supérieur. 

—Oh! 

—Oui, car ma mère s'est laissée prendre, je ne dirai pas à la curiosité, mais à l'intérêt que vous inspirez. Quand nous sommes seuls, nous ne causons que de vous. 

—Et elle vous a dit de vous méfier de ce Manfred? 

—Au contraire, elle me dit: «Morcerf, je crois le comte une noble nature; tâche de te faire aimer de lui.» 

Monte-Cristo détourna les yeux et poussa un soupir. 

«Ah! vraiment? dit-il. 

—De sorte, vous comprenez, continua Albert, qu'au lieu de s'opposer à mon voyage, elle l'approuvera de tout son cœur, puisqu'il rentre dans les recommandations qu'elle me fait chaque jour. 

—Allez donc, dit Monte-Cristo; à ce soir. Soyez ici à cinq heures; nous arriverons là-bas à minuit ou une heure. 

—Comment! au Tréport?\dots 

—Au Tréport ou dans les environs. 

—Il ne vous faut que huit heures pour faire quarante-huit lieues? 

—C'est encore beaucoup, dit Monte-Cristo. 

—Décidément vous êtes l'homme des prodiges, et vous arriverez non seulement à dépasser les chemins de fer, ce qui n'est pas bien difficile en France surtout, mais encore à aller plus vite que le télégraphe. 

—En attendant, vicomte, comme il nous faut toujours sept ou huit heures pour arriver là-bas, soyez exact. 

—Soyez tranquille, je n'ai rien autre chose à faire d'ici là que de m'apprêter. 

—À cinq heures, alors? 

—À cinq heures.» 

Albert sortit. Monte-Cristo, après lui avoir en souriant fait un signe de la tête, demeura un instant pensif et comme absorbé dans une profonde méditation. Enfin, passant la main sur son front, comme pour écarter sa rêverie, il alla au timbre et frappa deux coups. 

Au bruit des deux coups frappés par Monte-Cristo sur le timbre, Bertuccio entra. 

«Maître Bertuccio, dit-il, ce n'est pas demain, ce n'est pas après-demain, comme je l'avais pensé d'abord, c'est ce soir que je pars pour la Normandie; d'ici à cinq heures, c'est plus de temps qu'il ne vous en faut; vous ferez prévenir les palefreniers du premier relais; M. de Morcerf m'accompagne. Allez!» 

Bertuccio obéit, et un piqueur courut à Pontoise annoncer que la chaise de poste passerait à six heures précises. Le palefrenier de Pontoise envoya au relais suivant un exprès, qui en envoya un autre; et, six heures après, tous les relais disposés sur la route étaient prévenus. 

Avant de partir, le comte monta chez Haydée, lui annonça son départ, lui dit le lieu où il allait, et mit toute sa maison à ses ordres. 

Albert fut exact. Le voyage, sombre à son commencement, s'éclaircit bientôt par l'effet physique de la rapidité. Morcerf n'avait pas idée d'une pareille vitesse. 

«En effet, dit Monte-Cristo, avec votre poste faisant ses deux lieues à l'heure, avec cette loi stupide qui défend à un voyageur de dépasser l'autre sans lui demander la permission, et qui fait qu'un voyageur malade ou quinteux a le droit d'enchaîner à sa suite les voyageurs allègres et bien portants, il n'y a pas de locomotion possible; moi, j'évite cet inconvénient en voyageant avec mon propre postillon et mes propres chevaux, n'est-ce pas, Ali?» 

Et le comte, passant la tête par la portière, poussait un petit cri d'excitation qui donnait des ailes aux chevaux; ils ne couraient plus, ils volaient. La voiture roulait comme un tonnerre sur ce pavé royal, et chacun se détournait pour voir passer ce météore flamboyant. Ali, répétant ce cri, souriait, montrant ses dents blanches, serrant dans ses mains robustes les rênes écumantes, aiguillonnant les chevaux, dont les belles crinières s'éparpillaient au vent; Ali, l'enfant du désert, se retrouvait dans son élément, et avec son visage noir, ses yeux ardents, son burnous de neige, il semblait, au milieu de la poussière qu'il soulevait, le génie du simoun et le dieu de l'ouragan. 

«Voilà, dit Morcerf, une volupté que je ne connaissais pas, c'est la volupté de la vitesse.» 

Et les derniers nuages de son front se dissipaient, comme si l'air qu'il fendait emportait ces nuages avec lui. 

«Mais où diable trouvez-vous de pareils chevaux? demanda Albert. Vous les faites donc faire exprès? 

—Justement, dit le comte. Il y a six ans, je trouvai en Hongrie un fameux étalon renommé pour sa vitesse; je l'achetai je ne sais plus combien: ce fut Bertuccio qui paya. Dans la même année, il eut trente-deux enfants. C'est toute cette progéniture du même père que nous allons passer en revue; ils sont tous pareils, noirs, sans une seule tache, excepté une étoile au front, car à ce privilégié du haras on a choisi des juments, comme aux pachas on choisit des favorites. 

—C'est admirable!\dots Mais dites-moi, comte, que faites-vous de tous ces chevaux? 

—Vous le voyez, je voyage avec eux. 

—Mais vous ne voyagerez pas toujours? 

—Quand je n'en aurai plus besoin, Bertuccio les vendra, et il prétend qu'il gagnera trente ou quarante mille francs sur eux. 

—Mais il n'y aura pas de roi d'Europe assez riche pour vous les acheter. 

—Alors il les vendra à quelque simple vizir d'Orient, qui videra son trésor pour les payer et qui remplira son trésor en administrant des coups de bâton sous la plante des pieds de ses sujets. 

—Comte, voulez-vous que je vous communique une pensée qui m'est venue? 

—Faites. 

—C'est qu'après vous, M. Bertuccio doit être le plus riche particulier de l'Europe. 

—Eh bien, vous vous trompez, vicomte. Je suis sûr que si vous retourniez les poches de Bertuccio, vous n'y trouveriez pas dix sous vaillant. 

—Pourquoi cela? demanda le jeune homme. C'est donc un phénomène que M. Bertuccio? Ah! mon cher comte, ne me poussez pas trop loin dans le merveilleux, ou je ne vous croirai plus, je vous préviens. 

—Jamais de merveilleux avec moi, Albert; des chiffres et de la raison, voilà tout. Or, écoutez ce dilemme: Un intendant vole, mais pourquoi vole-t-il? 

—Dame! parce que c'est dans sa nature, ce me semble, dit Albert, il vole pour voler. 

—Eh bien, non, vous vous trompez: il vole parce qu'il a une femme, des enfants, des désirs ambitieux pour lui et pour sa famille; il vole surtout parce qu'il n'est pas sûr de ne jamais quitter son maître et qu'il veut se faire un avenir. Eh bien, M. Bertuccio est seul au monde, il puise dans ma bourse sans me rendre compte, il est sûr de ne jamais me quitter. 

—Pourquoi cela? 

—Parce que je n'en trouverais pas un meilleur. 

—Vous tournez dans un cercle vicieux, celui des probabilités. 

—Oh! non pas; je suis dans les certitudes. Le bon serviteur pour moi, c'est celui sur lequel j'ai droit de vie ou de mort. 

—Et vous avez droit de vie ou de mort sur Bertuccio? demanda Albert. 

—Oui», répondit froidement le comte. 

Il y a des mots qui ferment la conversation comme une porte de fer. Le \textit{oui} du comte était un de ces mots-là. 

Le reste du voyage s'accomplit avec la même rapidité, les trente-deux chevaux, divisés en huit relais, firent leurs quarante-huit lieues en huit heures. 

On arriva au milieu de la nuit, à la porte d'un beau parc. Le concierge était debout et tenait la grille ouverte. Il avait été prévenu par le palefrenier du dernier relais. 

Il était deux heures et demie du matin. On conduisit Morcerf à son appartement. Il trouva un bain et un souper prêts. Le domestique, qui avait fait la route sur le siège de derrière de la voiture, était à ses ordres; Baptistin, qui avait fait la route sur le siège de devant, était à ceux du comte. 

Albert prit son bain, soupa et se coucha. Toute la nuit, il fut bercé par le bruit mélancolique de la houle. En se levant, il alla droit à la fenêtre, l'ouvrit et se trouva sur une petite terrasse, où l'on avait devant soi la mer, c'est-à-dire l'immensité, et derrière soi un joli parc donnant sur une petite forêt. 

Dans une anse d'une certaine grandeur se balançait une petite corvette à la carène étroite, à la mâture élancée, et portant à la corne un pavillon aux armes de Monte-Cristo, armes représentant une montagne d'or posant sur une mer d'azur, avec une croix de gueules au chef, ce qui pouvait aussi bien être une allusion à son nom rappelant le Calvaire, que la passion de Notre-Seigneur a fait une montagne plus précieuse que l'or, et la croix infâme que son sang divin a faite sainte, qu'à quelque souvenir personnel de souffrance et de régénération enseveli dans la nuit du passé mystérieux de cet homme. Autour de la goélette étaient plusieurs petits chasse-marée appartenant aux pêcheurs des villages voisins, et qui semblaient d'humbles sujets attendant les ordres de leur reine. 

Là, comme dans tous les endroits où s'arrêtait Monte-Cristo, ne fût-ce que pour y passer deux jours, la vie y était organisée au thermomètre du plus haut confortable; aussi la vie, à l'instant même, y devenait-elle facile. 

Albert trouva dans son antichambre deux fusils et tous les ustensiles nécessaires à un chasseur; une pièce plus haute, et placée au rez-de-chaussée, était consacrée à toutes les ingénieuses machines que les Anglais, grands pêcheurs, parce qu'ils sont patients et oisifs, n'ont pas encore pu faire adopter aux routiniers pêcheurs de France. 

Toute la journée se passa à ces exercices divers auxquels, d'ailleurs, Monte-Cristo excellait: on tua une douzaine de faisans dans le parc, on pêcha autant de truites dans les ruisseaux, on dîna dans un kiosque donnant sur la mer, et l'on servit le thé dans la bibliothèque. 

Vers le soir du troisième jour, Albert, brisé de fatigue à l'user de cette vie qui semblait être un jeu pour Monte-Cristo, dormait près de la fenêtre tandis que le comte faisait avec son architecte le plan d'une serre qu'il voulait établir dans sa maison, lorsque le bruit d'un cheval écrasant les cailloux de la route fit lever la tête au jeune homme; il regarda par la fenêtre et, avec une surprise des plus désagréables, aperçut dans la cour son valet de chambre, dont il n'avait pas voulu se faire suivre pour moins embarrasser Monte-Cristo. 

«Florentin ici! s'écria-t-il en bondissant sur son fauteuil; est-ce que ma mère est malade?» 

Et il se précipita vers la porte de la chambre. 

Monte-Cristo le suivit des yeux, et le vit aborder le valet qui, tout essoufflé encore, tira de sa poche un petit paquet cacheté. Le petit paquet contenait un journal et une lettre. 

«De qui cette lettre? demanda vivement Albert. 

—De M. Beauchamp, répondit Florentin. 

—C'est Beauchamp qui vous envoie alors? 

—Oui, monsieur. Il m'a fait venir chez lui, m'a donné l'argent nécessaire à mon voyage, m'a fait venir un cheval de poste, et m'a fait promettre de ne point m'arrêter que je n'aie rejoint monsieur: j'ai fait la route en quinze heures.» 

Albert ouvrit la lettre en frissonnant: aux premières lignes, il poussa un cri, et saisit le journal avec un tremblement visible. 

Tout à coup ses yeux s'obscurcirent, ses jambes semblèrent se dérober sous lui, et, prêt à tomber, il s'appuya sur Florentin, qui étendait le bras pour le soutenir. 

«Pauvre jeune homme! murmura Monte-Cristo, si bas que lui-même n'eût pu entendre le bruit des paroles de compassion qu'il prononçait; il est donc dit que la faute des pères retombera sur les enfants jusqu'à la troisième et quatrième génération.» 

Pendant ce temps Albert avait repris sa force, et, continuant de lire, il secoua ses cheveux sur sa tête mouillée de sueur, et, froissant lettre et journal: 

«Florentin, dit-il, votre cheval est-il en état de reprendre le chemin de Paris? 

—C'est un mauvais bidet de poste éclopé. 

—Oh! mon Dieu! et comment était la maison quand vous l'avez quittée? 

—Assez calme; mais en revenant de chez M. Beauchamp, j'ai trouvé madame dans les larmes; elle m'avait fait demander pour savoir quand vous reviendriez. Alors je lui ai dit que j'allais vous chercher de la part de M. Beauchamp. Son premier mouvement a été d'étendre le bras comme pour m'arrêter; mais après un instant de réflexion: 

«Oui, allez, Florentin, a-t-elle dit, et qu'il revienne.» 

—Oui, ma mère, oui, dit Albert, je reviens, sois tranquille, et malheur à l'infâme!\dots Mais, avant tout, il faut que je parte.» 

Il reprit le chemin de la chambre où il avait laissé Monte-Cristo. 

Ce n'était plus le même homme et cinq minutes avaient suffi pour opérer chez Albert une triste métamorphose; il était sorti dans son état ordinaire, il rentrait avec la voix altérée, le visage sillonné de rougeurs fébriles, l'œil étincelant sous des paupières veinées de bleu, et la démarche chancelante comme celle d'un homme ivre. 

«Comte, dit-il, merci de votre bonne hospitalité dont j'aurais voulu jouir plus longtemps, mais il faut que je retourne à Paris. 

—Qu'est-il donc arrivé? 

—Un grand malheur; mais permettez-moi de partir, il s'agit d'une chose bien autrement précieuse que ma vie. Pas de question, comte, je vous en supplie, mais un cheval! 

—Mes écuries sont à votre service, vicomte, dit Monte-Cristo; mais vous allez vous tuer de fatigue en courant la poste à cheval; prenez une calèche, un coupé, quelque voiture. 

—Non, ce serait trop long, et puis j'ai besoin de cette fatigue que vous craignez pour moi, elle me fera du bien.» 

Albert fit quelques pas en tournoyant comme un homme frappé d'une balle, et alla tomber sur une chaise près de la porte. 

Monte-Cristo ne vit pas cette seconde faiblesse, il était à la fenêtre et criait: 

«Ali, un cheval pour M. de Morcerf! qu'on se hâte! il est pressé!» 

Ces paroles rendirent la vie à Albert; il s'élança hors de la chambre, le comte le suivit. 

«Merci! murmura le jeune homme en s'élançant en selle. Vous reviendrez aussi vite que vous pourrez, Florentin. Y a-t-il un mot d'ordre pour qu'on me donne des chevaux? 

—Pas d'autre que de rendre celui que vous montez; on vous en sellera à l'instant un autre.» 

Albert allait s'élancer, il s'arrêta. 

«Vous trouverez peut-être mon départ étrange, insensé, dit le jeune homme. Vous ne comprenez pas comment quelques lignes écrites sur un journal peuvent mettre un homme au désespoir; eh bien, ajouta-t-il en lui jetant le journal, lisez ceci, mais quand je serai parti seulement, afin que vous ne voyiez pas ma rougeur.» 

Et tandis que le comte ramassait le journal, il enfonça les éperons, qu'on venait d'attacher à ses bottes, dans le ventre du cheval, qui, étonné qu'il existât un cavalier qui crût avoir besoin vis-à-vis de lui d'un pareil stimulant, partit comme un trait d'arbalète. 

Le comte suivit des yeux avec un sentiment de compassion infinie le jeune homme, et ce ne fut que lorsqu'il eut complètement disparu que, reportant ses regards sur le journal, il lut ce qui suit: 

«Cet officier français au service d'Ali, pacha de Janina, dont parlait, il y a trois semaines, le journal \textit{L'Impartial} et qui non seulement livra les châteaux de Janina, mais encore vendit son bienfaiteur aux Turcs, s'appelait en effet à cette époque Fernand, comme l'a dit notre honorable confrère; mais, depuis, il a ajouté à son nom de baptême un titre de noblesse et un nom de terre. 

«Il s'appelle aujourd'hui M. le comte de Morcerf, et fait partie de la Chambre des pairs.» 

Ainsi donc ce secret terrible, que Beauchamp avait enseveli avec tant de générosité, reparaissait comme un fantôme armé, et un autre journal, cruellement renseigné, avait publié, le surlendemain du départ d'Albert pour la Normandie, les quelques lignes qui avaient failli rendre fou le malheureux jeune homme. 