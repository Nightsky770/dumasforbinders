\chapter{Robert le diable}

\lettrine{L}{a} raison de l'Opéra était d'autant meilleure à donner qu'il y avait ce soir-là solennité à l'Académie royale de musique. Levasseur, après une longue indisposition, rentrait par le rôle de Bertram, et, comme toujours, l'œuvre du maestro à la mode avait attiré la plus brillante société de Paris. 

Morcerf, comme la plupart des jeunes gens riches, avait sa stalle d'orchestre, plus dix loges de personnes de sa connaissance auxquelles il pouvait aller demander une place sans compter celle à laquelle il avait droit dans la loge des lions. 

Château-Renaud avait la stalle voisine de la sienne. 

Beauchamp, en sa qualité de journaliste, était roi de la salle et avait sa place partout. 

Ce soir-là, Lucien Debray avait la disposition de la loge du ministre, et il l'avait offerte au comte de Morcerf, lequel, sur le refus de Mercédès, l'avait envoyée à Danglars, en lui faisant dire qu'il irait probablement faire dans la soirée une visite à la baronne et à sa fille, si ces dames voulaient bien accepter la loge qu'il leur proposait. Ces dames n'avaient eu garde de refuser. Nul n'est friand de loges qui ne coûtent rien comme un millionnaire. 

Quant à Danglars, il avait déclaré que ses principes politiques et sa qualité de député de l'opposition ne lui permettaient pas d'aller dans la loge du ministre. En conséquence, la baronne avait écrit à Lucien de la venir prendre, attendu qu'elle ne pouvait pas aller à l'Opéra seule avec Eugénie. 

En effet, si les deux femmes y eussent été seules, on eût, certes, trouvé cela fort mauvais; tandis que Mlle Danglars allant à l'Opéra avec sa mère et l'amant de sa mère il n'y avait rien à dire: il faut bien prendre le monde comme il est fait. 

La toile se leva, comme d'habitude, sur une salle à peu près vide. C'est encore une habitude de notre fashion parisienne, d'arriver au spectacle quand le spectacle est commencé: il en résulte que le premier acte se passe, de la part des spectateurs arrivés, non pas à regarder ou à écouter la pièce, mais à regarder entrer les spectateurs qui arrivent, et à ne rien entendre que le bruit des portes et celui des conversations. 

«Tiens! dit tout à coup Albert en voyant s'ouvrir une loge de côté de premier rang, tiens! la comtesse G\dots» 

—Qu'est-ce que c'est que la comtesse G\dots? demanda Château-Renaud. 

—Oh! par exemple, baron, voici une question que je ne vous pardonne pas; vous demandez ce que c'est que la comtesse G\dots? 

—Ah! c'est vrai, dit Château-Renaud, n'est-ce pas cette charmante Vénitienne? 

—Justement.» 

En ce moment la comtesse G\dots aperçut Albert et échangea avec lui un salut accompagné d'un sourire. 

«Vous la connaissez? dit Château-Renaud. 

—Oui, fit Albert; je lui ai été présenté à Rome par Franz. 

—Voudrez-vous me rendre à Paris le même service que Franz vous a rendu à Rome? 

—Bien volontiers.  

—Chut!» cria le public. 

Les deux jeunes gens continuèrent leur conversation, sans paraître s'inquiéter le moins du monde du désir que paraissait éprouver le parterre d'entendre la musique. 

«Elle était aux courses du Champ-de-Mars, dit Château-Renaud. 

—Aujourd'hui? 

—Oui. 

—Tiens! au fait, il y avait courses. Étiez-vous engagé? 

—Oh! pour une misère, pour cinquante louis. 

—Et qui a gagné? 

—Nautilus; je pariais pour lui. 

—Mais il y avait trois courses? 

—Oui. Il y avait le prix du Jockey-Club, une coupe d'or. Il s'est même passé une chose assez bizarre. 

—Laquelle? 

—Chut donc! cria le public. 

—Laquelle? répéta Albert. 

—C'est un cheval et un jockey complètement inconnus qui ont gagné cette course. 

—Comment? 

—Oh! mon Dieu, oui, personne n'avait fait attention à un cheval inscrit sous le nom de \textit{Vampa} et à un jockey inscrit sous le nom de \textit{Job}, quand on a vu s'avancer tout à coup un admirable alezan et un jockey gros comme le poing; on a été obligé de lui fourrer vingt livres de plomb dans ses poches, ce qui ne l'a pas empêché d'arriver au but trois longueurs de cheval avant \textit{Ariel et Barbaro}, qui couraient avec lui. 

—Et l'on n'a pas su à qui appartenaient le cheval et le jockey? 

—Non. 

—Vous dites que ce cheval était inscrit sous le nom de\dots. 

—\textit{Vampa}. 

—Alors, dit Albert, je suis plus avancé que vous, je sais à qui il appartenait, moi. 

—Silence donc!» cria pour la troisième fois le parterre. 

Cette fois la levée de boucliers était si grande, que les deux jeunes gens s'aperçurent enfin que c'était à eux que le public s'adressait. Ils se retournèrent un instant, cherchant dans cette foule un homme qui prit la responsabilité de ce qu'ils regardaient comme une impertinence; mais personne ne réitéra l'invitation, et ils se retournèrent vers la scène. En ce moment la loge du ministre s'ouvrait, et Mme Danglars, sa fille et Lucien Debray prenaient leurs places. 

«Ah! ah! dit Château-Renaud, voilà des personnes de votre connaissance, vicomte. Que diable regardez-vous donc à droite? On vous cherche.» 

Albert se retourna et ses yeux rencontrèrent effectivement ceux de la baronne Danglars, qui lui fit avec son éventail un petit salut. Quant à Mlle Eugénie, ce fut à peine si ses grands yeux noirs daignèrent s'abaisser jusqu'à l'orchestre. 

«En vérité, mon cher, dit Château-Renaud, je ne comprends point, à part la mésalliance, et je ne crois point que ce soit cela qui vous préoccupe beaucoup; je ne comprends pas, dis-je, à part la mésalliance, ce que vous pouvez avoir contre Mlle Danglars; c'est en vérité une fort belle personne. 

—Fort belle, certainement, dit Albert; mais je vous avoue qu'en fait de beauté j'aimerais mieux quelque chose de plus doux, de plus suave, de plus féminin, enfin. 

—Voilà bien les jeunes gens, dit Château-Renaud qui, en sa qualité d'homme de trente ans, prenait avec Morcerf des airs paternels; ils ne sont jamais satisfaits. Comment, mon cher! on vous trouve une fiancée bâtie sur le modèle de la Diane chasseresse et vous n'êtes pas content! 

—Eh bien, justement, j'aurais mieux aimé quelque chose dans le genre de la Vénus de Milo ou de Capoue. Cette Diane chasseresse, toujours au milieu de ses nymphes, m'épouvante un peu, j'ai peur qu'elle ne me traite en Actéon.» 

En effet, un coup d'œil jeté sur la jeune fille pouvait presque expliquer le sentiment que venait d'avouer Morcerf. Mlle Danglars était belle, mais, comme l'avait dit Albert, d'une beauté un peu arrêtée: ses cheveux étaient d'un beau noir, mais dans leurs ondes naturelles on remarquait une certaine rébellion à la main qui voulait leur imposer sa volonté; ses yeux, noirs comme ses cheveux, encadrés sous de magnifiques sourcils qui n'avaient qu'un défaut, celui de se froncer quelquefois, étaient surtout remarquables par une expression de fermeté qu'on était étonné de trouver dans le regard d'une femme; son nez avait les proportions exactes qu'un statuaire eût données à celui de Junon: sa bouche seule était trop grande, mais garnie de belles dents que faisaient ressortir encore des lèvres dont le carmin trop vif tranchait avec la pâleur de son teint; enfin un signe noir placé au coin de la bouche, et plus large que ne le sont d'ordinaire ces sortes de caprices de la nature, achevait de donner à cette physionomie ce caractère décidé qui effrayait quelque peu Morcerf. 

D'ailleurs, tout le reste de la personne d'Eugénie s'alliait avec cette tête que nous venons d'essayer de décrire. C'était, comme l'avait dit Château-Renaud, la Diane chasseresse, mais avec quelque chose encore de plus ferme et de plus musculeux dans sa beauté. 

Quant à l'éducation, qu'elle avait reçue, s'il y avait un reproche à lui faire, c'est que, comme certains points de sa physionomie, elle semblait un peu appartenir à un autre sexe. En effet, elle parlait deux ou trois langues, dessinait facilement, faisait des vers et composait de la musique; elle était surtout passionnée pour ce dernier art, qu'elle étudiait avec une de ses amies de pension, jeune personne sans fortune, mais ayant toutes les dispositions possibles pour devenir, à ce que l'on assurait, une excellente cantatrice. Un grand compositeur portait, disait-on, à cette dernière, un intérêt presque paternel, et la faisait travailler avec l'espoir qu'elle trouverait un jour une fortune dans sa voix. 

Cette possibilité que Mlle Louise d'Armilly, c'était le nom de la jeune virtuose, entrât un jour au théâtre faisait que Mlle Danglars, quoique la recevant chez elle, ne se montrait point en public en sa compagnie. Du reste, sans avoir dans la maison du banquier la position indépendante d'une amie, Louise avait une position supérieure à celle des institutrices ordinaires. 

Quelques secondes après l'entrée de Mme Danglars dans sa loge, la toile avait baissé et, grâce à cette faculté, laissée par la longueur des entractes, de se promener au foyer ou de faire des visites pendant une demi-heure, l'orchestre s'était à peu près dégarni. 

Morcerf et Château-Renaud étaient sortis des premiers. Un instant Mme Danglars avait pensé que cet empressement d'Albert avait pour but de lui venir présenter ses compliments, et elle s'était penchée à l'oreille de sa fille pour lui annoncer cette visite, mais celle-ci s'était contentée de secouer la tête en souriant; et en même temps, comme pour prouver combien la dénégation d'Eugénie était fondée, Morcerf apparut dans une loge de côté du premier rang. Cette loge était celle de la comtesse G\dots 

«Ah! vous voilà, monsieur le voyageur, dit celle-ci en lui tendant la main avec toute la cordialité d'une vieille connaissance; c'est bien aimable à vous de m'avoir reconnue, et surtout de m'avoir donné la préférence pour votre première visite. 

—Croyez, madame, répondit Albert, que si j'eusse su votre arrivée à Paris et connu votre adresse, je n'eusse point attendu si tard. Mais veuillez me permettre de vous présenter M. le baron de Château-Renaud, mon ami, un des rares gentilshommes qui restent encore en France, et par lequel je viens d'apprendre que vous étiez aux courses du Champ-de-Mars.» 

Château-Renaud salua. 

«Ah! vous étiez aux courses, monsieur? dit vivement la comtesse. 

—Oui, madame. 

—Eh bien, reprit vivement Mme G\dots, pouvez-vous me dire à qui appartenait le cheval qui a gagné le prix du Jockey-Club? 

—Non, madame, dit Château-Renaud, et je faisais tout à l'heure la même question à Albert. 

—Y tenez-vous beaucoup, madame la comtesse? demanda Albert. 

—À quoi? 

—À connaître le maître du cheval? 

—Infiniment. Imaginez-vous\dots. Mais sauriez-vous qui, par hasard, vicomte? 

—Madame, vous alliez raconter une histoire: imaginez-vous, avez-vous dit. 

—Eh bien, imaginez-vous que ce charmant cheval alezan et ce joli petit jockey à casaque rose m'avaient, à la première vue, inspiré une si vive sympathie, que je faisais des vœux pour l'un et pour l'autre, exactement comme si j'avais engagé sur eux la moitié de ma fortune; aussi, lorsque je les vis arriver au but, devançant les autres coureurs de trois longueurs de cheval, je fus si joyeuse que je me mis à battre des mains comme une folle. Figurez-vous mon étonnement lorsque, en rentrant chez moi, je rencontrai sur mon escalier le petit jockey rose! Je crus que le vainqueur de la course demeurait par hasard dans la même maison que moi, lorsque, en ouvrant la porte de mon salon, la première chose que je vis fut la coupe d'or qui formait le prix gagné par le cheval et le jockey inconnus. Dans la coupe il y avait un petit papier sur lequel étaient écrits ces mots: «À la comtesse G\dots, Lord Ruthwen.» 

—C'est justement cela, dit Morcerf. 

—Comment! c'est justement cela; que voulez-vous dire? 

—Je veux dire que c'est Lord Ruthwen en personne. 

—Quel Lord Ruthwen? 

—Le nôtre, le vampire, celui du théâtre Argentina. 

—Vraiment! s'écria la comtesse; il est donc ici? 

—Parfaitement. 

—Et vous le voyez? vous le recevez? vous allez chez lui? 

—C'est mon ami intime, et M. de Château-Renaud lui-même a l'honneur de le connaître. 

—Qui peut vous faire croire que c'est lui qui a gagné? 

—Son cheval inscrit sous le nom de \textit{Vampa}\dots 

—Eh bien, après? 

—Eh bien, vous ne vous rappelez pas le nom du fameux bandit qui m'avait fait prisonnier? 

—Ah! c'est vrai. 

—Et des mains duquel le comte m'a miraculeusement tiré? 

—Si fait. 

—Il s'appelait \textit{Vampa}. Vous voyez bien que c'est lui. 

—Mais pourquoi m'a-t-il envoyé cette coupe, à moi? 

—D'abord, madame la comtesse, parce que je lui avais fort parlé de vous, comme vous pouvez le croire; ensuite parce qu'il aura été enchanté de retrouver une compatriote, et heureux de l'intérêt que cette compatriote prenait à lui. 

—J'espère bien que vous ne lui avez jamais raconté les folies que nous avons dites à son sujet! 

—Ma foi, je n'en jurerais pas, et cette façon de vous offrir cette coupe sous le nom de Lord Ruthwen\dots. 

—Mais c'est affreux, il va m'en vouloir mortellement. 

—Son procédé est-il celui d'un ennemi? 

—Non, je l'avoue. 

—Eh bien! 

—Ainsi, il est à Paris? 

—Oui. 

—Et quelle sensation a-t-il faite? 

—Mais, dit Albert, on en a parlé huit jours, puis sont arrivés le couronnement de la reine d'Angleterre et le vol des diamants de Mlle Mars, et l'on n'a plus parlé que de cela. 

—Mon cher, dit Château-Renaud, on voit bien que le comte est votre ami, vous le traitez en conséquence. Ne croyez pas ce que vous dit Albert, madame la comtesse, il n'est au contraire question que du comte de Monte-Cristo à Paris. Il a d'abord débuté par envoyer à Mme Danglars des chevaux de trente mille francs; puis il a sauvé la vie à Mme de Villefort; puis il a gagné la course du Jockey-Club à ce qu'il paraît. Je maintiens au contraire, moi, quoi qu'en dise Morcerf, qu'on s'occupe encore du comte en ce moment, et qu'on ne s'occupera même plus que de lui dans un mois, s'il veut continuer de faire de l'excentricité, ce qui, au reste, paraît être sa manière de vivre ordinaire. 

—C'est possible, dit Morcerf; en attendant, qui donc a repris la loge de l'ambassadeur de Russie? 

—Laquelle? demanda la comtesse. 

—L'entre-colonne du premier rang; elle me semble parfaitement remise à neuf. 

—En effet, dit Château-Renaud. Est-ce qu'il y avait quelqu'un pendant le premier acte? 

—Où? 

—Dans cette loge? 

—Non, reprit la comtesse, je n'ai vu personne; ainsi, continua-t-elle, revenant à la première conversation, vous croyez que c'est votre comte de Monte-Cristo qui a gagné le prix? 

—J'en suis sûr. 

—Et qui m'a envoyé cette coupe? 

—Sans aucun doute. 

—Mais je ne le connais pas, moi, dit la comtesse, et j'ai fort envie de la lui renvoyer. 

—Oh! n'en faites rien; il vous en enverrait une autre, taillée dans quelque saphir ou creusée dans quelque rubis. Ce sont ses manières d'agir; que voulez-vous, il faut le prendre comme il est.» 

En ce moment on entendit la sonnette qui annonçait que le deuxième acte allait commencer. Albert se leva pour regagner sa place. 

«Vous verrai-je? demanda la comtesse. 

—Dans les entractes, si vous le permettez, je viendrai m'informer si je puis vous être bon à quelque chose à Paris. 

—Messieurs, dit la comtesse, tous les samedis soir, rue de Rivoli, 22, je suis chez moi pour mes amis. Vous voilà prévenus.» 

Les jeunes gens saluèrent et sortirent. 

En entrant dans la salle, ils virent le parterre debout et les yeux fixés sur un seul point de la salle; leurs regards suivirent la direction générale, et s'arrêtèrent sur l'ancienne loge de l'ambassadeur de Russie. Un homme habillé de noir, de trente-cinq à quarante ans, venait d'y entrer avec une femme vêtue d'un costume oriental. La femme était de la plus grande beauté, et le costume d'une telle richesse que comme nous l'avons dit, tous les yeux s'étaient à l'instant tournés vers elle. 

«Eh! dit Albert, c'est Monte-Cristo et sa Grecque.» 

En effet, c'était le comte et Haydée. 

Au bout d'un instant, la jeune femme était l'objet de l'attention non seulement du parterre, mais de toute la salle; les femmes se penchaient hors des loges pour voir ruisseler sous les feux des lustres cette cascade de diamants. 

Le second acte se passa au milieu de cette rumeur sourde qui indique dans les masses assemblées un grand événement. Personne ne songea à crier silence. Cette femme si jeune, si belle, si éblouissante, était le plus curieux spectacle qu'on pût voir. 

Cette fois, un signe de Mme Danglars indiqua clairement à Albert que la baronne désirait avoir sa visite dans l'entracte suivant. 

Morcerf était de trop bon goût pour se faire attendre quand on lui indiquait clairement qu'il était attendu. L'acte fini, il se hâta donc de monter dans l'avant-scène. 

Il salua les deux dames et tendit la main à Debray. 

La baronne l'accueillit avec un charmant sourire et Eugénie avec sa froideur habituelle. 

«Ma foi, mon cher, dit Debray, vous voyez un homme à bout, et qui vous appelle en aide pour le relayer. Voici madame qui m'écrase de questions sur le comte, et qui veut que je sache d'où il est, d'où il vient, où il va; ma foi, je ne suis pas Cagliostro, moi, et pour me tirer d'affaire, j'ai dit: «Demandez tout cela à Morcerf, il connaît son Monte-Cristo sur le bout du doigt»; alors on vous a fait signe. 

—N'est-il pas incroyable, dit la baronne, que lorsqu'on a un demi-million de fonds secrets à sa disposition on ne soit pas mieux instruit que cela? 

—Madame, dit Lucien, je vous prie de croire que si j'avais un demi-million à ma disposition, je l'emploierais à autre chose qu'à prendre des informations sur M. de Monte-Cristo, qui n'a d'autre mérite à mes yeux que d'être deux fois riche comme un nabab; mais j'ai passé la parole à mon ami Morcerf; arrangez-vous avec lui, cela ne me regarde plus. 

—Un nabab ne m'eût certainement pas envoyé une paire de chevaux de trente mille francs, avec quatre diamants aux oreilles, de cinq mille francs chacun. 

—Oh! les diamants, dit en riant Morcerf, c'est sa manie. Je crois que, pareil à Potemkin, il en a toujours dans ses poches, et qu'il en sème sur son chemin comme le petit Poucet faisait de ses cailloux. 

—Il aura trouvé quelque mine, dit Mme Danglars; vous savez qu'il a un crédit illimité sur la maison du baron? 

—Non, je ne le savais pas, répondit Albert, mais cela doit être. 

—Et qu'il a annoncé à M. Danglars qu'il comptait rester un an à Paris et y dépenser six millions? 

—C'est le schah de Perse qui voyage incognito. 

—Et cette femme, monsieur Lucien, dit Eugénie, avez-vous remarqué comme elle est belle? 

—En vérité, mademoiselle, je ne connais que vous pour faire si bonne justice aux personnes de votre sexe.» 

Lucien approcha son lorgnon de son œil. 

«Charmante! dit-il. 

—Et cette femme, M. de Morcerf sait-il qui elle est? 

—Mademoiselle, dit Albert, répondant à cette interpellation presque directe, je le sais à peu près, comme tout ce qui regarde le personnage mystérieux dont nous nous occupons. Cette femme est une Grecque. 

—Cela se voit facilement à son costume, et vous ne m'apprenez là que ce que toute la salle sait déjà comme nous. 

—Je suis fâché, dit Morcerf, d'être un cicérone si ignorant, mais je dois avouer que là se bornent mes connaissances; je sais, en outre qu'elle est musicienne, car un jour que j'ai déjeuné chez le comte, j'ai entendu les sons d'une guzla qui ne pouvaient venir certainement que d'elle. 

—Il reçoit donc, votre comte? demanda Mme Danglars. 

—Et d'une façon splendide, je vous le jure. 

—Il faut que je pousse Danglars à lui offrir quelque dîner, quelque bal, afin qu'il nous les rende. 

—Comment, vous irez chez lui? dit Debray en riant. 

—Pourquoi pas? avec mon mari! 

—Mais il est garçon, ce mystérieux comte. 

—Vous voyez bien que non, dit en riant à son tour la baronne, en montrant la belle Grecque. 

—Cette femme est une esclave, à ce qu'il nous a dit lui-même, vous rappelez-vous, Morcerf? à votre déjeuner? 

—Convenez, mon cher Lucien, dit la baronne qu'elle a bien plutôt l'air d'une princesse. 

—Des \textit{Mille et une Nuits}. 

—Des \textit{Mille et une Nuits}, je ne dis pas; mais qu'est-ce qui fait les princesses, mon cher? ce sont les diamants, et celle-ci en est couverte. 

—Elle en a même trop, dit Eugénie; elle serait plus belle sans cela, car on verrait son cou et ses poignets, qui sont charmants de forme. 

—Oh! l'artiste. Tenez, dit Mme Danglars, la voyez-vous qui se passionne? 

—J'aime tout ce qui est beau, dit Eugénie. 

—Mais que dites-vous du comte alors? dit Debray, il me semble qu'il n'est pas mal non plus. 

—Le comte? dit Eugénie, comme si elle n'eût point encore pensé à le regarder, le comte, il est bien pâle. 

—Justement, dit Morcerf, c'est dans cette pâleur qu'est le secret que nous cherchons. La comtesse G\dots prétend, vous le savez, que c'est un vampire. 

—Elle est donc de retour, la comtesse G\dots? demanda la baronne. 

—Dans cette loge de côté, dit Eugénie, presque en face de nous, ma mère; cette femme, avec ces admirables cheveux blonds, c'est elle. 

—Oh! oui, dit Mme Danglars; vous ne savez pas ce que vous devriez faire, Morcerf? 

—Ordonnez, madame. 

—Vous devriez aller faire une visite à votre comte de Monte-Cristo et nous l'amener. 

—Pourquoi faire? dit Eugénie. 

—Mais pour que nous lui parlions; n'es-tu pas curieuse de le voir? 

—Pas le moins du monde. 

—Étrange enfant! murmura la baronne. 

—Oh! dit Morcerf, il viendra probablement de lui-même. Tenez, il vous a vue, madame, et il vous salue.» 

La baronne rendit au comte son salut, accompagné d'un charmant sourire. 

«Allons, dit Morcerf, je me sacrifie; je vous quitte et vais voir s'il n'y a pas moyen de lui parler. 

—Allez dans sa loge; c'est bien simple. 

—Mais je ne suis pas présenté. 

—À qui? 

—À la belle Grecque. 

—C'est une esclave, dites-vous? 

—Oui, mais vous prétendez, vous, que c'est une princesse\dots. Non. J'espère que lorsqu'il me verra sortir il sortira. 

—C'est possible. Allez! 

—J'y vais.» 

Morcerf salua et sortit. Effectivement, au moment où il passait devant la loge du comte, la porte s'ouvrit; le comte dit quelques mots en arabe à Ali, qui se tenait dans le corridor, et prit le bras de Morcerf. 

Ali referma la porte, et se tint debout devant elle; il y avait dans le corridor un rassemblement autour du Nubien. 

«En vérité, dit Monte-Cristo, votre Paris est une étrange ville, et vos Parisiens un singulier peuple. On dirait que c'est la première fois qu'ils voient un Nubien. Regardez-les donc se presser autour de ce pauvre Ali, qui ne sait pas ce que cela veut dire. Je vous réponds d'une chose, par exemple, c'est qu'un Parisien peut aller à Tunis, à Constantinople, à Bagdad ou au Caire, on ne fera pas cercle autour de lui. 

—C'est que vos Orientaux sont des gens sensés, et qu'ils ne regardent que ce qui vaut la peine d'être vu; mais croyez-moi, Ali ne jouit de cette popularité que parce qu'il vous appartient, et qu'en ce moment vous êtes l'homme à la mode. 

—Vraiment! et qui me vaut cette faveur? 

—Parbleu! vous-même. Vous donnez des attelages de mille louis; vous sauvez la vie à des femmes de procureur du roi; vous faites courir, sous le nom de major Brack, des chevaux pur sang et des jockeys gros comme des ouistitis; enfin, vous gagnez des coupes d'or, et vous les envoyez aux jolies femmes. 

—Et qui diable vous a conté toutes ces folies? 

—Dame! la première, Mme Danglars, qui meurt d'envie de vous voir dans sa loge, ou plutôt qu'on vous y voie; la seconde, le journal de Beauchamp, et la troisième, ma propre imaginative. Pourquoi appelez-vous votre cheval \textit{Vampa}, si vous voulez garder l'incognito? 

—Ah! c'est vrai! dit le comte, c'est une imprudence. Mais dites-moi donc, le comte de Morcerf ne vient-il point quelquefois à l'Opéra? Je l'ai cherché des yeux, et je ne l'ai aperçu nulle part. 

—Il viendra ce soir. 

—Où cela? 

—Dans la loge de la baronne, je crois. 

—Cette charmante personne qui est avec elle, c'est sa fille? 

—Oui. 

—Je vous en fais mon compliment.» 

Morcerf sourit. 

«Nous reparlerons de cela plus tard et en détail, dit-il. Que dites-vous de la musique? 

—De quelle musique? 

—Mais de celle que vous venez d'entendre. 

—Je dis que c'est de fort belle musique pour de la musique composée par un compositeur humain, et chantée par des oiseaux à deux pieds et sans plumes, comme disait feu Diogène. 

—Ah çà! mais, mon cher comte, il semblerait que vous pourriez entendre à votre caprice les sept chœurs du paradis? 

—Mais c'est un peu de cela. Quand je veux entendre d'admirable musique, vicomte, de la musique comme jamais l'oreille mortelle n'en a entendu, je dors. 

—Eh bien, mais, vous êtes à merveille ici; dormez, mon cher comte, dormez, l'Opéra n'a pas été inventé pour autre chose. 

—Non, en vérité, votre orchestre fait trop de bruit. Pour que je dorme du sommeil dont je vous parle, il me faut le calme et le silence, et puis une certaine préparation\dots. 

—Ah! le fameux haschich? 

—Justement, vicomte, quand vous voudrez entendre de la musique, venez souper avec moi. 

—Mais j'en ai déjà entendu en y allant déjeuner, dit Morcerf. 

—À Rome? 

—Oui. 

—Ah! c'était la guzla d'Haydée. Oui, la pauvre exilée s'amuse quelquefois à me jouer des airs de son pays.» 

Morcerf n'insista pas davantage; de son côté, le comte se tut. 

En ce moment la sonnette retentit. 

«Vous m'excusez? dit le comte en reprenant le chemin de sa loge. 

—Comment donc! 

—Emportez bien des choses pour la comtesse G\dots de la part de son vampire. 

—Et à la baronne? 

—Dites-lui que j'aurai l'honneur, si elle le permet, d'aller lui présenter mes hommages dans la soirée.» 

Le troisième acte commença. Pendant le troisième acte le comte de Morcerf vint, comme il l'avait promis, rejoindre Mme Danglars. 

Le comte n'était point un de ces hommes qui font révolution dans une salle; aussi personne ne s'aperçut-il de son arrivée que ceux dans la loge desquels il venait prendre une place. 

Monte-Cristo le vit cependant, et un léger sourire effleura ses lèvres. 

Quant à Haydée, elle ne voyait rien tant que la toile était levée; comme toutes les natures primitives, elle adorait tout ce qui parle à l'oreille et à la vue. 

Le troisième acte s'écoula comme d'habitude; Mlles Noblet, Julia et Leroux exécutèrent leurs entrechats ordinaires; le prince de Grenade fut défié par Robert-Mario; enfin ce majestueux roi que vous savez fit le tour de la salle pour montrer son manteau de velours, en tenant sa fille par la main; puis la toile tomba, et la salle se dégorgea aussitôt dans le foyer et les corridors. 

Le comte sortit de sa loge, et un instant après apparut dans celle de la baronne Danglars. 

La baronne ne put s'empêcher de jeter un cri de surprise légèrement mêlé de joie. 

«Ah! venez donc, monsieur le comte! s'écria-t-elle, car, en vérité, j'avais hâte de joindre mes grâces verbales aux remerciements écrits que je vous ai déjà faits. 

—Oh! madame, dit le comte, vous vous rappelez encore cette misère? je l'avais déjà oubliée, moi. 

—Oui, mais ce qu'on n'oublie pas, monsieur le comte, c'est que vous avez le lendemain sauvé ma bonne amie Mme de Villefort du danger que lui faisaient courir ces mêmes chevaux. 

—Cette fois encore, madame, je ne mérite pas vos remerciements; c'est Ali, mon Nubien, qui a eu le bonheur de rendre à Mme de Villefort cet éminent service. 

—Et est-ce aussi Ali, dit le comte de Morcerf, qui a tiré mon fils des bandits romains? 

—Non, monsieur le comte, dit Monte-Cristo en serrant la main que le général lui tendait, non; cette fois je prends les remerciements pour mon compte; mais vous me les avez déjà faits, je les ai déjà reçus, et, en vérité, je suis honteux de vous retrouver encore si reconnaissant. Faites-moi donc l'honneur, je vous prie, madame la baronne, de me présenter à mademoiselle votre fille. 

—Oh! vous êtes tout présenté, de nom du moins, car il y a deux ou trois jours que nous ne parlons que de vous. Eugénie, continua la baronne en se retournant vers sa fille, monsieur le comte de Monte-Cristo!» 

Le comte s'inclina: Mlle Danglars fit un léger mouvement de tête. 

«Vous êtes là avec une admirable personne, monsieur le comte, dit Eugénie; est-ce votre fille? 

—Non, mademoiselle, dit Monte-Cristo étonné de cette extrême ingénuité ou de cet étonnant aplomb, c'est une pauvre Grecque dont je suis le tuteur.  

—Et qui se nomme?\dots 

—Haydée, répondit Monte-Cristo. 

—Une Grecque! murmura le comte de Morcerf. 

—Oui, comte, dit Mme Danglars; et dites-moi si vous avez jamais vu à la cour d'Ali-Tebelin, que vous avez si glorieusement servi, un aussi admirable costume que celui que nous avons là devant les yeux. 

—Ah! dit Monte-Cristo, vous avez servi à Janina, monsieur le comte? 

—J'ai été général-inspecteur des troupes du pacha, répondit Morcerf, et mon peu de fortune, je ne le cache pas, vient des libéralités de l'illustre chef albanais. 

—Regardez donc! insista Mme Danglars. 

—Où cela? balbutia Morcerf. 

—Tenez!» dit Monte-Cristo. 

Et, enveloppant le comte de son bras, il se pencha avec lui hors la loge. 

En ce moment, Haydée, qui cherchait le comte des yeux, aperçut sa tête pâle près de celle de M. de Morcerf, qu'il tenait embrassé. 

Cette vue produisit sur la jeune fille l'effet de la tête de Méduse; elle fit un mouvement en avant comme pour les dévorer tous deux du regard, puis, presque aussitôt, elle se rejeta en arrière en poussant un faible cri, qui fut cependant entendu des personnes qui étaient les plus proches d'elle et d'Ali, qui aussitôt ouvrit la porte. 

«Tiens, dit Eugénie, que vient-il donc d'arriver à votre pupille, monsieur le comte? On dirait qu'elle se trouve mal. 

—En effet, dit le comte, mais ne vous effrayez point, mademoiselle: Haydée est très nerveuse et par conséquent très sensible aux odeurs: un parfum qui lui est antipathique suffit pour la faire évanouir; mais, ajouta le comte en tirant un flacon de sa poche, j'ai là le remède.» 

Et, après avoir salué la baronne et sa fille d'un seul et même salut, il échangea une dernière poignée de main avec le comte et avec Debray, et sortit de la loge de Mme Danglars.  

Quand il entra dans la sienne, Haydée était encore fort pâle; à peine parut-il qu'elle lui saisit la main. Monte-Cristo s'aperçut que les mains de la jeune fille étaient humides et glacées à la fois. 

«Avec qui donc causais-tu là, seigneur? demanda la jeune fille. 

—Mais, répondit Monte-Cristo, avec le comte de Morcerf, qui a été au service de ton illustre père, et qui avoue lui devoir sa fortune. 

—Ah! le misérable! s'écria Haydée, c'est lui qui l'a vendu aux Turcs; et cette fortune, c'est le prix de sa trahison. Ne savais-tu donc pas cela, mon cher seigneur? 

—J'avais bien déjà entendu dire quelques mots de cette histoire en Épire, dit Monte-Cristo, mais j'en ignore les détails. Viens, ma fille, tu me les donneras, ce doit être curieux. 

—Oh! oui, viens, viens; il me semble que je mourrais si je restais plus longtemps en face de cet homme.» 

Et Haydée, se levant vivement, s'enveloppa de son burnous de cachemire blanc brodé de perles et de corail, et sortit vivement au moment où la toile se levait. 

«Voyez si cet homme fait rien comme un autre! dit la comtesse G\dots à Albert, qui était retourné près d'elle; il écoute religieusement le troisième acte de \textit{Robert}, et il s'en va au moment où le quatrième va commencer. 