\chapter{L'île de Monte-Cristo}

\lettrine{E}{nfin} Dantès, par un de ces bonheurs inespérés qui arrivent parfois à ceux sur lesquels la rigueur du sort s'est longtemps lassée, Dantès allait arriver à son but par un moyen simple et naturel, et mettre le pied dans l'île sans inspirer à personne aucun soupçon.

Une nuit le séparait seulement de ce départ tant attendu.

Cette nuit fut une des plus fiévreuses que passa Dantès. Pendant cette nuit, toutes les chances bonnes et mauvaises se présentèrent tour à tour à son esprit: s'il fermait les yeux, il voyait la lettre du cardinal Spada écrite en caractères flamboyants sur la muraille; s'il s'endormait un instant, les rêves le plus insensés venaient tourbillonner dans son cerveau. Il descendait dans les grottes aux pavés d'émeraudes, aux parois de rubis, aux stalactites de diamants. Les perles tombaient goutte à goutte comme filtre d'ordinaire l'eau souterraine.

Edmond, ravi, émerveillé, remplissait ses poche de pierreries; puis il revenait au jour, et ces pierreries s'étaient changées en simples cailloux. Alors il essayait de rentrer dans ces grottes merveilleuses, entrevues seulement; mais le chemin se tordait en spirales infinies: l'entrée était redevenue invisible. Il cherchait inutilement dans sa mémoire fatiguée ce mot magique et mystérieux qui ouvrait pour le pêcheur arabe les cavernes splendides d'Ali-Baba. Tout était inutile; le trésor disparu était redevenu la propriété des génies de la terre, auxquels il avait eu un instant l'espoir de l'enlever.

Le jour vint presque aussi fébrile que l'avait été la nuit; mais il amena la logique à l'aide de l'imagination, et Dantès put arrêter un plan jusqu'alors vague et flottant dans son cerveau.

Le soir vint, et avec le soir les préparatifs du départ. Ces préparatifs étaient un moyen pour Dantès de cacher son agitation. Peu à peu, il avait pris cette autorité sur ses compagnons, de commander comme s'il était le maître du bâtiment; et comme ses ordres étaient toujours clairs, précis et faciles à exécuter, ses compagnons lui obéissaient non seulement avec promptitude, mais encore avec plaisir.

Le vieux marin le laissait faire: lui aussi avait reconnu la supériorité de Dantès sur ses autres matelots et sur lui-même. Il voyait dans le jeune homme son successeur naturel, et il regrettait de n'avoir pas une fille pour enchaîner Edmond par cette haute alliance.

À sept heures du soir tout fut prêt; à sept heures dix minutes on doublait le phare, juste au moment où le phare s'allumait.

La mer était calme, avec un vent frais venant du sud-est; on naviguait sous un ciel d'azur, où Dieu allumait aussi tour à tour ses phares, dont chacun est un monde. Dantès déclara que tout le monde pouvait se coucher et qu'il se chargeait du gouvernail.

Quand le Maltais (c'est ainsi que l'on appelait Dantès) avait fait une pareille déclaration, cela suffisait, et chacun s'en allait coucher tranquille.

Cela arrivait quelquefois: Dantès, rejeté de la solitude dans le monde, éprouvait de temps en temps d'impérieux besoins de solitude. Or, quelle solitude à la fois plus immense et plus poétique que celle d'un bâtiment qui flotte isolé sur la mer, pendant l'obscurité de la nuit, dans le silence de l'immensité et sous le regard du Seigneur?

Cette fois, la solitude fut peuplée de ses pensées, la nuit éclairée par ses illusions, le silence animé par ses promesses.

Quand le patron se réveilla, le navire marchait sous toutes voiles: il n'y avait pas un lambeau de toile qui ne fût gonflé par le vent; on faisait plus de deux lieues et demie à l'heure.

L'île de Monte-Cristo grandissait à l'horizon.

Edmond rendit le bâtiment à son maître et alla s'étendre à son tour dans son hamac: mais, malgré sa nuit d'insomnie, il ne put fermer l'œil un seul instant.

Deux heures après, il remonta sur le pont; le bâtiment était en train de doubler l'île d'Elbe. On était à la hauteur de Mareciana et au-dessus de l'île plate et verte de la Pianosa. On voyait s'élancer dans l'azur du ciel le sommet flamboyant de Monte-Cristo.

Dantès ordonna au timonier de mettre la barre à bâbord, afin de laisser la Pianosa à droite; il avait calculé que cette manœuvre devrait raccourcir la route de deux ou trois nœuds.

Vers cinq heures du soir, on eut la vue complète de l'île. On en apercevait les moindres détails, grâce à cette limpidité atmosphérique qui est particulière à la lumière que versent les rayons du soleil à son déclin.

Edmond dévorait des yeux cette masse de rochers qui passait par toutes les couleurs crépusculaires, depuis le rose vif jusqu'au bleu foncé; de temps en temps, des bouffées ardentes lui montaient au visage; son front s'empourprait, un nuage pourpre passait devant ses yeux.

Jamais joueur dont toute la fortune est en jeu n'eut, sur un coup de dés, les angoisses que ressentait Edmond dans ses paroxysmes d'espérance.

La nuit vint: à dix heures du soir on aborda; la \textit{Jeune-Amélie} était la première au rendez-vous.

Dantès, malgré son empire ordinaire sur lui-même, ne put se contenir: il sauta le premier sur le rivage; s'il l'eût osé comme Brutus, il eût baisé la terre.

Il faisait nuit close; mais à onze heures la lune se leva du milieu de la mer, dont elle argenta chaque frémissement; puis ses rayons, à mesure qu'elle se leva, commencèrent à se jouer, en blanches cascades de lumière, sur les roches entassées de cet autre Pélion.

L'île était familière à l'équipage de la \textit{Jeune-Amélie}: c'était une de ses stations ordinaires. Quant à Dantès, il l'avait reconnue à chacun de ses voyages dans le Levant, mais jamais il n'y était descendu.

Il interrogea Jacopo.

«Où allons-nous passer la nuit? demanda-t-il.

—Mais à bord de la tartane, répondit le marin.

—Ne serions-nous pas mieux dans les grottes?

—Dans quelles grottes?

—Mais dans les grottes de l'île.

—Je ne connais pas de grottes», dit Jacopo.

Une sueur froide passa sur le front de Dantès.

«Il n'y a pas de grottes à Monte-Cristo? demanda-t-il.

—Non.»

Dantès demeura un instant étourdi; puis il songea que ces grottes pouvaient avoir été comblées depuis par un accident quelconque, ou même bouchées, pour plus grandes précautions, par le cardinal Spada. Le tout, dans ce cas, était donc de retrouver cette ouverture perdue. Il était inutile de la chercher pendant la nuit. Dantès remit donc l'investigation au lendemain. D'ailleurs, un signal arboré à une demi-lieue en mer, et auquel la \textit{Jeune-Amélie} répondit aussitôt par un signal pareil, indiqua que le moment était venu de se mettre à la besogne. Le bâtiment retardataire, rassuré par le signal qui devait faire connaître au dernier arrivé qu'il y avait toute sécurité à s'aboucher, apparut bientôt blanc et silencieux comme un fantôme, et vint jeter l'ancre à une encablure du rivage.

Aussitôt le transport commença.

Dantès songeait, tout en travaillant, au hourra de joie que d'un seul mot il pourrait provoquer parmi tous ces hommes s'il disait tout haut l'incessante pensée qui bourdonnait tout bas à son oreille et à son cœur. Mais, tout au contraire de révéler le magnifique secret, il craignait d'en avoir déjà trop dit et d'avoir, par ses allées et venues, ses demandes répétées, ses observations minutieuses et sa préoccupation continuelle, éveillé les soupçons. Heureusement, pour cette circonstance du moins, que chez lui un passé bien douloureux reflétait sur son visage une tristesse indélébile, et que les lueurs de gaieté entrevues sous ce nuage n'étaient réellement que des éclairs.

Personne ne se doutait donc de rien, et lorsque le lendemain, en prenant un fusil, du plomb et de la poudre, Dantès manifesta le désir d'aller tuer quelqu'une de ces nombreuses chèvres sauvages que l'on voyait sauter de rocher en rocher, on n'attribua cette excursion de Dantès qu'à l'amour de la chasse ou au désir de la solitude. Il n'y eut que Jacopo qui insista pour le suivre. Dantès ne voulut pas s'y opposer, craignant par cette répugnance à être accompagné d'inspirer quelques soupçons. Mais à peine eut-il fait un quart de lieue, qu'ayant trouvé l'occasion de tirer et de tuer un chevreau, il envoya Jacopo le porter à ses compagnons, les invitant à le faire cuire et à lui donner lorsqu'il serait cuit, le signal d'en manger sa part en tirant un coup de fusil; quelques fruits secs et un fiasco de vin de Monte-Pulciano devaient compléter l'ordonnance du repas.

Dantès continua son chemin en se retournant de temps en temps. Arrivé au sommet d'une roche, il vit à mille pieds au-dessous de lui ses compagnons que venait de rejoindre Jacopo et qui s'occupaient déjà activement des apprêts du déjeuner, augmenté, grâce à l'adresse d'Edmond, d'une pièce capitale.

Edmond les regarda un instant avec ce sourire doux et triste de l'homme supérieur.

«Dans deux heures, dit-il, ces gens-là repartiront, riches de cinquante piastres, pour aller, en risquant leur vie, essayer d'en gagner cinquante autres; puis reviendront, riches de six cents livres, dilapider ce trésor dans une ville quelconque, avec la fierté des sultans et la confiance des nababs. Aujourd'hui, l'espérance fait que je méprise leur richesse, qui me paraît la plus profonde misère; demain, la déception fera peut-être que je serai forcé de regarder cette profonde misère comme le suprême bonheur\dots. Oh! non, s'écria Edmond, cela ne sera pas; le savant, l'infaillible Faria ne se serait pas trompé sur cette seule chose. D'ailleurs autant vaudrait mourir que de continuer de mener cette vie misérable et inférieure.»

Ainsi Dantès, qui, il y a trois mois, n'aspirait qu'à la liberté, n'avait déjà plus assez de la liberté et aspirait à la richesse; la faute n'en était pas à Dantès, mais à Dieu, qui, en bornant la puissance de l'homme, lui a fait des désirs infinis! Cependant par une route perdue entre deux murailles de roches, suivant un sentier creusé par le torrent et que, selon toute probabilité, jamais pied humain n'avait foulé, Dantès s'était approché de l'endroit où il supposait que les grottes avaient dû exister. Tout en suivant le rivage de la mer et en examinant les moindres objets avec une attention sérieuse, il crut remarquer sur certains rochers des entailles creusées par la main de l'homme.

Le temps, qui jette sur toute chose physique son manteau de mousse, comme sur les choses morales son manteau d'oubli, semblait avoir respecté ces signes tracés avec une certaine régularité, et dans le but probablement d'indiquer une trace; de temps en temps cependant, ces signes disparaissaient sous des touffes de myrtes, qui s'épanouissaient en gros bouquets chargés de fleurs, ou sous des lichens parasites. Il fallait alors qu'Edmond écartât les branches ou soulevât les mousses pour retrouver les signes indicateurs qui le conduisaient dans cet autre labyrinthe. Ces signes avaient, au reste, donné bon espoir à Edmond. Pourquoi ne serait-ce pas le cardinal qui les aurait tracés pour qu'ils pussent, en cas d'une catastrophe qu'il n'avait pas pu prévoir si complète, servir de guide à son neveu? Ce lieu solitaire était bien celui qui convenait à un homme qui voulait enfouir un trésor. Seulement, ces signes infidèles n'avaient-ils pas attiré d'autres yeux que ceux pour lesquels ils étaient tracés, et l'île aux sombres merveilles avait-elle fidèlement gardé son magnifique secret?

Cependant, à soixante pas du port à peu près, il sembla à Edmond, toujours caché à ses compagnons par les accidents du terrain, que les entailles s'arrêtaient; seulement, elles n'aboutissaient à aucune grotte. Un gros rocher rond posé sur une base solide était le seul but auquel elles semblassent conduire. Edmond pensa qu'au lieu d'être arrivé à la fin, il n'était peut-être, tout au contraire, qu'au commencement; il prit en conséquence le contre-pied et retourna sur ses pas.

Pendant ce temps, ses compagnons préparaient le déjeuner, allaient puiser de l'eau, à la source, transportaient le pain et les fruits à terre et faisaient cuire le chevreau. Juste au moment où ils le tiraient de sa broche improvisée, ils aperçurent Edmond qui, léger et hardi comme un chamois, sautait de rocher en rocher: ils tirèrent un coup de fusil pour lui donner le signal. Le chasseur changea aussitôt de direction, et revint tout courant à eux. Mais au moment où tous le suivaient des yeux dans l'espèce de vol qu'il exécutait, taxant son adresse de témérité, comme pour donner raison à leurs craintes, le pied manqua à Edmond; on le vit chanceler à la cime d'un rocher, pousser un cri et disparaître.

Tous bondirent d'un seul élan, car tous aimaient Edmond, malgré sa supériorité; cependant, ce fut Jacopo qui arriva le premier.

Il trouva Edmond étendu sanglant et presque sans connaissance: il avait dû rouler d'une hauteur de douze ou quinze pieds. On lui introduisit dans la bouche quelques gouttes de rhum, et ce remède qui avait déjà eu tant d'efficacité sur lui, produisit le même effet que la première fois.

Edmond rouvrit les yeux, se plaignit de souffrir une vive douleur au genou, une grande pesanteur à la tête et des élancements insupportables dans les reins. On voulut le transporter jusqu'au rivage; mais lorsqu'on le toucha, quoique ce fût Jacopo qui dirigeât l'opération, il déclara en gémissant qu'il ne se sentait point la force de supporter le transport.

On comprend qu'il ne fut point question de déjeuner pour Dantès; mais il exigea que ses camarades, qui n'avaient pas les mêmes raisons que lui pour faire diète, retournassent à leur poste. Quant à lui, il prétendit qu'il n'avait besoin que d'un peu de repos, et qu'à leur retour ils le trouveraient soulagé.

Les marins ne se firent pas trop prier: les marins avaient faim, l'odeur du chevreau arrivait jusqu'à eux et l'on n'est point cérémonieux entre loups de mer.

Une heure après, ils revinrent. Tout ce qu'Edmond avait pu faire, c'était de se traîner pendant un espace d'une dizaine de pas pour s'appuyer à une roche moussue.

Mais, loin de se calmer, les douleurs de Dantès avaient semblé croître en violence. Le vieux patron, qui était forcé de partir dans la matinée pour aller déposer son chargement sur les frontières du Piémont et de la France, entre Nice et Fréjus, insista pour que Dantès essayât de se lever. Dantès fit des efforts surhumains pour se rendre à cette invitation mais à chaque effort, il retombait plaintif et pâlissant.

«Il a les reins cassés, dit tout bas le patron: n'importe! c'est un bon compagnon, et il ne faut pas l'abandonner; tâchons de le transporter jusqu'à la tartane.»

Mais Dantès déclara qu'il aimait mieux mourir où il était que de supporter les douleurs atroces que lui occasionnerait le mouvement, si faible qu'il fût.

«Eh bien, dit le patron, advienne que pourra, mais il ne sera pas dit que nous avons laissé sans secours un brave compagnon comme vous. Nous ne partirons que ce soir.»

Cette proposition étonna fort les matelots, quoique aucun d'eux ne la combattît, au contraire. Le patron était un homme si rigide, que c'était la première fois qu'on le voyait renoncer à une entreprise, ou même retarder son exécution.

Aussi Dantès ne voulut-il pas souffrir qu'on fit en sa faveur une si grave infraction aux règles de la discipline établie à bord.

«Non, dit-il au patron, j'ai été un maladroit, et il est juste que je porte la peine de ma maladresse. Laissez-moi une petite provision de biscuit, un fusil, de la poudre et des balles pour tuer des chevreaux, ou même pour me défendre, et une pioche pour me construire, si vous tardiez trop à me venir prendre, une espèce de maison.

—Mais tu mourras de faim, dit le patron.

—J'aime mieux cela, répondit Edmond, que de souffrir les douleurs inouïes qu'un seul mouvement me fait endurer.»

Le patron se retournait du côté du bâtiment, qui se balançait avec un commencement d'appareillage dans le petit port, prêt à reprendre la mer dès que sa toilette serait achevée.

«Que veux-tu donc que nous fassions, Maltais, dit-il, nous ne pouvons t'abandonner ainsi, et nous ne pouvons rester, cependant?

—Partez, partez! s'écria Dantès.

—Nous serons au moins huit jours absents, dit le patron, et encore faudra-t-il que nous nous détournions de notre route pour te venir prendre.

—Écoutez, dit Dantès: si d'ici deux ou trois jours, vous rencontrez quelque bâtiment pêcheur ou autre qui vienne dans ces parages, recommandez-moi à lui, je donnerai vingt-cinq piastres pour mon retour à Livourne. Si vous n'en trouvez pas, revenez.»

Le patron secoua la tête.

«Écoutez, patron Baldi, il y a un moyen de tout concilier, dit Jacopo; partez; moi, je resterai avec le blessé pour le soigner.

—Et tu renonceras à ta part de partage, dit Edmond, pour rester avec moi?

—Oui, dit Jacopo, et sans regret.

—Allons, tu es un brave garçon, Jacopo, dit Edmond, Dieu te récompensera de ta bonne volonté; mais je n'ai besoin de personne, merci: un jour ou deux de repos me remettront et j'espère trouver dans ces rochers certaines herbes excellentes contre les contusions.»

Et un sourire étrange passa sur les lèvres de Dantès; il serra la main de Jacopo avec effusion, mais il demeura inébranlable dans sa résolution de rester, et de rester seul.

Les contrebandiers laissèrent à Edmond ce qu'il demandait et s'éloignèrent non sans se retourner plusieurs fois, lui faisant à chaque fois qu'ils détournaient tous les signes d'un cordial adieu, auquel Edmond répondait de la main seulement, comme s'il ne pouvait remuer le reste du corps.

Puis, lorsqu'ils eurent disparu:

«C'est étrange, murmura Dantès en riant, que ce soit parmi de pareils hommes que l'on trouve des preuves d'amitié et des actes de dévouement.»

Alors il se traîna avec précaution jusqu'au sommet d'un rocher qui lui dérobait l'aspect de la mer, et de là il vit la tartane achever son appareillage, lever l'ancre, se balancer gracieusement comme une mouette qui va prendre son vol, et partir.

Au bout d'une heure, elle avait complètement disparu: du moins, de l'endroit où était demeuré le blessé, il était impossible de la voir.

Alors Dantès se releva, plus souple et plus léger qu'un des chevreaux qui bondissaient parmi les myrtes et les lentisques sur ces rochers sauvages, prit son fusil d'une main, sa pioche de l'autre, et courut à cette roche à laquelle aboutissaient les entailles qu'il avait remarquées sur les rochers.

«Et maintenant, s'écria-t-il en se rappelant cette histoire du pêcheur arabe que lui avait racontée Faria, maintenant, Sésame, ouvre-toi!»



