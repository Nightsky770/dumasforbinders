\chapter{La main de Dieu}

\lettrine{C}{aderousse} continuait de crier d'une voix lamentable: 

\zz
«Monsieur l'abbé, au secours! au secours! 

\zz
—Qu'y a-t-il? demanda Monte-Cristo. 

\zz
—À mon secours! répéta Caderousse; on m'a assassiné! 

—Nous voici! Du courage! 

—Ah! c'est fini. Vous arrivez trop tard; vous arrivez pour me voir mourir. Quels coups! que de sang!» 

Et il s'évanouit. 

Ali et son maître prirent le blessé et le transportèrent dans une chambre. Là, Monte-Cristo fit signe à Ali de le déshabiller, et il reconnut les trois terribles blessures dont il était atteint. 

«Mon Dieu! dit-il, votre vengeance se fait parfois attendre; mais je crois qu'alors elle ne descend du ciel que plus complète.» 

Ali regarda son maître comme pour lui demander ce qu'il y avait à faire. 

«Va chercher M. le procureur du roi Villefort, qui demeure faubourg Saint-Honoré, et amène-le ici. En passant, tu réveilleras le concierge, et tu lui diras d'aller chercher un médecin.» 

Ali obéit et laissa le faux abbé seul avec Caderousse, toujours évanoui. Lorsque le malheureux rouvrit les yeux, le comte, assis à quelques pas de lui, le regardait avec une sombre expression de pitié, et ses lèvres, qui s'agitaient, semblaient murmurer une prière. 

«Un chirurgien, monsieur l'abbé, un chirurgien! dit Caderousse. 

—On en est allé chercher un, répondit l'abbé. 

—Je sais bien que c'est inutile, quant à la vie, mais il pourra me donner des forces peut-être, et je veux avoir le temps de faire ma déclaration. 

—Sur quoi? 

—Sur mon assassin. 

—Vous le connaissez donc? 

—Si je le connais! oui, je le connais, c'est Benedetto. 

—Ce jeune Corse? 

—Lui-même. 

—Votre compagnon? 

—Oui. Après m'avoir donné le plan de la maison du comte, espérant sans doute que je le tuerais et qu'il deviendrait ainsi son héritier, ou qu'il me tuerait et qu'il serait ainsi débarrassé de moi, il m'a attendu dans la rue et m'a assassiné. 

—En même temps que j'ai envoyé chercher le médecin, j'ai envoyé chercher le procureur du roi. 

—Il arrivera trop tard, il arrivera trop tard, dit Caderousse, je sens tout mon sang qui s'en va. 

—Attendez», dit Monte-Cristo. 

Il sortit et rentra cinq minutes après avec un flacon. 

Les yeux du moribond, effrayants de fixité, n'avaient point en son absence quitté cette porte par laquelle il devinait instinctivement qu'un secours allait lui venir. 

«Dépêchez-vous! monsieur l'abbé, dépêchez-vous! dit-il, je sens que je m'évanouis encore.» 

Monte-Cristo s'approcha et versa sur les lèvres violettes du blessé trois ou quatre gouttes de la liqueur que contenait le flacon. 

Caderousse poussa un soupir. 

«Oh! dit-il, c'est la vie que vous me versez là; encore\dots encore\dots. 

—Deux gouttes de plus vous tueraient, répondit l'abbé. 

—Oh! qu'il vienne donc quelqu'un à qui je puisse dénoncer le misérable. 

—Voulez-vous que j'écrive votre déposition? vous la signerez. 

—Oui\dots oui\dots» dit Caderousse, dont les yeux brillaient à l'idée de cette vengeance posthume. 

Monte-Cristo écrivit: 

«Je meurs assassiné par le Corse Benedetto, mon compagnon de chaîne à Toulon sous le n°59.» 

«Dépêchez-vous! dépêchez-vous! dit Caderousse, je ne pourrais plus signer.» 

Monte-Cristo présenta la plume à Caderousse, qui rassembla ses forces, signa et retomba sur son lit en disant: 

«Vous raconterez le reste, monsieur l'abbé; vous direz qu'il se fait appeler Andrea Cavalcanti, qu'il loge à l'hôtel des Princes, que\dots. Ah! ah! mon Dieu! mon Dieu! voilà que je meurs!» 

Et Caderousse s'évanouit pour la seconde fois. 

L'abbé lui fit respirer l'odeur du flacon; le blessé rouvrit les yeux. 

Son désir de vengeance ne l'avait pas abandonné pendant son évanouissement. 

«Ah! vous direz tout cela, n'est-ce pas, monsieur l'abbé? 

—Tout cela, oui, et bien d'autres choses encore. 

—Que direz-vous? 

—Je dirai qu'il vous avait sans doute donné le plan de cette maison dans l'espérance que le comte vous tuerait. Je dirai qu'il avait prévenu le comte par un billet; je dirai que, le comte étant absent, c'est moi qui ai reçu ce billet et qui ai veillé pour vous attendre. 

—Et il sera guillotiné, n'est-ce pas? dit Caderousse, il sera guillotiné, vous me le promettez? Je meurs avec cet espoir-là, cela va m'aider à mourir. 

—Je dirai, continua le comte, qu'il est arrivé derrière vous, qu'il vous a guetté tout le temps; que lorsqu'il vous a vu sortir, il a couru à l'angle du mur et s'est caché. 

—Vous avez donc vu tout cela, vous? 

—Rappelez-vous mes paroles: «Si tu rentres chez toi sain et sauf, je croirai que Dieu t'a pardonné, et je te pardonnerai aussi.»  

—Et vous ne m'avez pas averti? s'écria Caderousse en essayant de se soulever sur son coude; vous saviez que j'allais être tué en sortant d'ici, et vous ne m'avez pas averti! 

—Non, car dans la main de Benedetto je voyais la justice de Dieu, et j'aurais cru commettre un sacrilège en m'opposant aux intentions de la Providence. 

—La justice de Dieu! ne m'en parlez pas, monsieur l'abbé: s'il y avait une justice de Dieu, vous savez mieux que personne qu'il y a des gens qui seraient punis et qui ne le sont pas. 

—Patience, dit l'abbé d'un ton qui fit frémir le moribond, patience!» 

Caderousse le regarda avec étonnement. 

«Et puis, dit l'abbé, Dieu est plein de miséricorde pour tous, comme il a été pour toi: il est père avant d'être juge. 

—Ah! vous croyez donc à Dieu, vous? dit Caderousse. 

—Si j'avais le malheur de n'y pas avoir cru jusqu'à présent, dit Monte-Cristo, j'y croirais en te voyant. 

Caderousse leva les poings crispés au ciel. 

«Écoute, dit l'abbé en étendant la main sur le blessé comme pour lui commander la foi, voilà ce qu'il a fait pour toi, ce Dieu que tu refuses de reconnaître à ton dernier moment: il t'avait donné la santé, la force, un travail assuré, des amis même, la vie enfin telle qu'elle doit se présenter à l'homme pour être douce avec le calme de la conscience et la satisfaction des désirs naturels; au lieu d'exploiter ces dons du Seigneur, si rarement accordés par lui dans leur plénitude, voilà ce que tu as fait, toi: tu t'es adonné à la fainéantise, à l'ivresse, et dans l'ivresse tu as trahi un de tes meilleurs amis. 

—Au secours! s'écria Caderousse, je n'ai pas besoin d'un prêtre, mais d'un médecin; peut-être que je ne suis pas blessé à mort, peut-être que je ne vais pas encore mourir, peut-être qu'on peut me sauver! 

—Tu es si bien blessé à mort que, sans les trois gouttes de liqueur que je t'ai données tout à l'heure, tu aurais déjà expiré. Écoute donc! 

—Ah! murmura Caderousse, quel étrange prêtre vous faites, qui désespérez les mourants au lieu de les consoler. 

—Écoute, continua l'abbé: quand tu as eu trahi ton ami, Dieu a commencé, non pas de te frapper, mais de t'avertir; tu es tombé dans la misère et tu as eu faim; tu avais passé à envier la moitié d'une vie que tu pouvais passer à acquérir, et déjà tu songeais au crime en te donnant à toi-même l'excuse de la nécessité, quand Dieu fit pour toi un miracle, quand Dieu, par mes mains, t'envoya au sein de ta misère une fortune, brillante pour toi, malheureux, qui n'avais jamais rien possédé. Mais cette fortune inattendue, inespérée, inouïe, ne te suffit plus du moment où tu la possèdes, tu veux la doubler: par quel moyen? par un meurtre. Tu la doubles, et alors Dieu te l'arrache en te conduisant devant la justice humaine. 

—Ce n'est pas moi, dit Caderousse, qui ai voulu tuer le juif, c'est la Carconte.  

—Oui, dit Monte-Cristo. Aussi Dieu toujours, je ne dirai pas juste cette fois, car sa justice t'eût donné la mort, mais Dieu, toujours miséricordieux, permit que tes juges fussent touchés à tes paroles et te laissassent la vie. 

—Pardieu! pour m'envoyer au bagne à perpétuité: la belle grâce! 

—Cette grâce, misérable! tu la regardas cependant comme une grâce quand elle te fut faite; ton lâche cœur, qui tremblait devant la mort, bondit de joie à l'annonce d'une honte perpétuelle, car tu t'es dit, comme tous les forçats: Il y a une porte au bagne, il n'y en a pas à la tombe. Et tu avais raison, car cette porte du bagne s'est ouverte pour toi d'une manière inespérée: un Anglais visite Toulon, il avait fait le vœu de tirer deux hommes de l'infamie: son choix tombe sur toi et sur ton compagnon; une seconde fortune descend pour toi du ciel, tu retrouves à la fois l'argent et la tranquillité, tu peux recommencer à vivre de la vie de tous les hommes, toi qui avais été condamné à vivre de celle des forçats; alors, misérable, alors tu te mets à tenter Dieu une troisième fois. Je n'ai pas assez, dis-tu, quand tu avais plus que tu n'avais possédé jamais, et tu commets un troisième crime, sans raison, sans excuse. Dieu s'est fatigué. Dieu t'a puni.» 

Caderousse s'affaiblissait à vue d'œil. 

«À boire, dit-il; j'ai soif\dots je brûle!» 

Monte-Cristo lui donna un verre d'eau. 

«Scélérat de Benedetto, dit Caderousse en rendant le verre: il échappera cependant, lui! 

—Personne n'échappera, c'est moi qui te le dis, Caderousse\dots Benedetto sera puni! 

—Alors vous serez puni, vous aussi, dit Caderousse; car vous n'avez pas fait votre devoir de prêtre\dots vous deviez empêcher Benedetto de me tuer. 

—Moi! dit le comte avec un sourire qui glaça d'effroi le mourant, moi empêcher Benedetto de te tuer, au moment où tu venais de briser ton couteau contre la cotte de mailles qui me couvrait la poitrine!\dots Oui, peut-être si je t'eusse trouvé humble et repentant, j'eusse empêché Benedetto de te tuer, mais je t'ai trouvé orgueilleux et sanguinaire, et j'ai laissé s'accomplir la volonté de Dieu! 

—Je ne crois pas à Dieu! hurla Caderousse, tu n'y crois pas non plus\dots tu mens\dots tu mens!\dots 

—Tais-toi, dit l'abbé, car tu fais jaillir hors de ton corps les dernières gouttes de ton sang\dots. Ah! tu ne crois pas en Dieu, et tu meurs frappé par Dieu!\dots Ah! tu ne crois pas en Dieu, et Dieu qui cependant ne demande qu'une prière, qu'un mot, qu'une larme pour pardonner\dots. Dieu qui pouvait diriger le poignard de l'assassin de manière que tu expirasses sur le coup\dots. Dieu t'a donné un quart d'heure pour te repentir\dots. Rentre donc en toi-même, malheureux, et repens-toi! 

—Non, dit Caderousse, non, je ne me repens pas; il n'y a pas de Dieu, il n'y a pas de Providence, il n'y a que du hasard. 

—Il y a une Providence, il y a un Dieu, dit Monte-Cristo, et la preuve, c'est que tu es là gisant, désespéré, reniant Dieu, et que, moi, je suis debout devant toi riche, heureux, sain et sauf, et joignant les mains devant Dieu auquel tu essaies de ne pas croire, et auquel cependant tu crois au fond du cœur. 

—Mais qui donc êtes-vous, alors? demanda Caderousse en fixant ses yeux mourants sur le comte. 

—Regarde-moi bien, dit Monte-Cristo en prenant la bougie et l'approchant de son visage. 

—Eh bien, l'abbé\dots l'abbé Busoni\dots.» 

Monte-Cristo enleva la perruque qui le défigurait, et laissa retomber les beaux cheveux noirs qui encadraient si harmonieusement son pâle visage. 

«Oh! dit Caderousse épouvanté, si ce n'étaient ces cheveux noirs, je dirais que vous êtes l'Anglais, je dirais que vous êtes Lord Wilmore. 

—Je ne suis ni l'abbé Busoni ni Lord Wilmore, dit Monte-Cristo: regarde mieux, regarde plus loin, regarde dans tes premiers souvenirs.» 

Il y avait dans cette parole du comte une vibration magnétique dont les sens épuisés du misérable furent ravivés une dernière fois. 

«Oh! en effet, dit-il, il me semble que je vous ai vu, que je vous ai connu autrefois. 

—Oui, Caderousse, oui, tu m'as vu, oui, tu m'as connu. 

—Mais qui donc êtes-vous, alors? et pourquoi, si vous m'avez vu, si vous m'avez connu, pourquoi me laissez-vous mourir? 

—Parce que rien ne peut te sauver, Caderousse, parce que tes blessures sont mortelles. Si tu avais pu être sauvé, j'aurais vu là une dernière miséricorde du Seigneur, et j'eusse encore, je te le jure par la tombe de mon père, essayé de te rendre à la vie et au repentir. 

—Par la tombe de ton père! dit Caderousse, ranimé par une suprême étincelle et se soulevant pour voir de plus près l'homme qui venait de lui faire ce serment sacré à tous les hommes: Eh! qui es-tu donc?» 

Le comte n'avait pas cessé de suivre le progrès de l'agonie. Il comprit que cet élan de vie était le dernier; il s'approcha du moribond, et le couvrant d'un regard calme et triste à la fois: 

«Je suis\dots lui dit-il à l'oreille, je suis\dots.» 

Et ses lèvres, à peine ouvertes, donnèrent passage à un nom prononcé si bas, que le comte semblait craindre de l'entendre lui-même. 

Caderousse, qui s'était soulevé sur ses genoux, étendit les bras, fit un effort pour se reculer, puis joignant les mains et les levant avec un suprême effort: 

«Ô mon Dieu, mon Dieu, dit-il, pardon de vous avoir renié; vous existez bien, vous êtes bien le père des hommes au ciel et le juge des hommes sur la terre. Mon Dieu, seigneur, je vous ai longtemps méconnu! mon Dieu, Seigneur, pardonnez-moi! mon Dieu, Seigneur, recevez-moi!» 

Et Caderousse, fermant les yeux, tomba renversé en arrière avec un dernier cri et avec un dernier soupir. 

Le sang s'arrêta aussitôt aux lèvres de ses larges blessures. 

Il était mort. 

«\textit{Un}!» dit mystérieusement le comte, les yeux fixés sur le cadavre déjà défiguré par cette horrible mort. 

Dix minutes après, le médecin et le procureur du roi arrivèrent, amenés, l'un par le concierge, l'autre par Ali, et furent reçus par l'abbé Busoni, qui priait près du mort. 