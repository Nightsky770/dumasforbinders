\chapter{La provocation}

\lettrine[ante=«]{A}{lors,} continua Beauchamp, je profitai du silence et de l'obscurité de la salle pour sortir sans être vu. L'huissier qui m'avait introduit m'attendait à la porte. Il me conduisit, à travers les corridors, jusqu'à une petite porte donnant sur la rue de Vaugirard. Je sortis l'âme brisée et ravie tout à la fois, pardonnez-moi cette expression, Albert, brisée par rapport à vous, ravie de la noblesse de cette jeune fille poursuivant la vengeance paternelle. Oui, je vous le jure, Albert, de quelque part que vienne cette révélation, je dis, moi, qu'elle peut venir d'un ennemi, mais que cet ennemi n'est que l'agent de la Providence.» 

Albert tenait sa tête entre ses deux mains; il releva son visage, rouge de honte et baigné de larmes, et saisissant le bras de Beauchamp. 

«Ami, lui dit-il, ma vie est finie: il me reste, non pas à dire comme vous que la Providence m'a porté le coup, mais à chercher quel homme me poursuit de son inimitié; puis, quand je le connaîtrai, je tuerai cet homme, ou cet homme me tuera; or, je compte sur votre amitié pour m'aider, Beauchamp, si toutefois le mépris ne l'a pas tuée dans votre cœur. 

—Le mépris, mon ami? et en quoi ce malheur vous touchera-t-il? Non! Dieu merci! nous n'en sommes plus au temps où un injuste préjugé rendait les fils responsables des actions des pères. Repassez toute votre vie, Albert, elle date d'hier, il est vrai, mais jamais aurore d'un beau jour fut-elle plus pure que votre orient? non, Albert, croyez-moi, vous êtes jeune, vous êtes riche, quittez la France: tout s'oublie vite dans cette grande Babylone à l'existence agitée et aux goûts changeants; vous viendrez dans trois ou quatre ans, vous aurez épousé quelque princesse russe, et personne ne songera plus à ce qui s'est passé hier, à plus forte raison à ce qui s'est passé il y a seize ans. 

—Merci, mon cher Beauchamp, merci de l'excellente intention qui vous dicte vos paroles, mais cela ne peut être ainsi, je vous ai dit mon désir, et maintenant, s'il le faut, je changerai le mot désir en celui de volonté. Vous comprenez qu'intéressé comme je le suis dans cette affaire, je ne puis voir la chose du même point de vue que vous. Ce qui vous semble venir à vous d'une source céleste me semble venir à moi d'une source moins pure. La Providence me paraît, je vous l'avoue, fort étrangère à tout ceci, et cela heureusement, car au lieu de l'invisible et de l'impalpable messagère des récompenses et punitions célestes, je trouverai un être palpable et visible, sur lequel je me vengerai, oh! oui, je vous le jure, de tout ce que je souffre depuis un mois. Maintenant, je vous le répète, Beauchamp, je tiens à rentrer dans la vie humaine et matérielle, et, si vous êtes encore mon ami comme vous le dites, aidez-moi à retrouver la main qui a porté le coup. 

—Alors, soit! dit Beauchamp; et si vous tenez absolument à ce que je descende sur la terre je le ferai; si vous tenez à vous mettre à la recherche d'un ennemi, je m'y mettrai avec vous. Et je le trouverai, car mon honneur est presque aussi intéressé que le vôtre à ce que nous le retrouvions. 

—Eh bien, alors, Beauchamp, vous comprenez, à l'instant même, sans retard, commençons nos investigations. Chaque minute de retard est une éternité pour moi; le dénonciateur n'est pas encore puni, il peut donc espérer qu'il ne le sera pas; et, sur mon honneur, s'il l'espère, il se trompe! 

—Eh bien, écoutez-moi, Morcerf. 

—Ah! Beauchamp, je vois que vous savez quelque chose; tenez, vous me rendez la vie! 

—Je ne dis pas que ce soit réalité, Albert, mais c'est au moins une lumière dans la nuit: en suivant cette lumière, peut-être nous conduira-t-elle au but. 

—Dites! vous voyez bien que je bous d'impatience. 

—Eh bien, je vais vous raconter ce que je n'ai pas voulu vous dire en revenant de Janina. 

—Parlez. 

—Voilà ce qui s'est passé, Albert; j'ai été tout naturellement chez le premier banquier de la ville pour prendre des informations; au premier mot que j'ai dit de l'affaire, avant même que le nom de votre père eût été prononcé: 

«—Ah! dit-il, très bien, je devine ce qui vous amène. 

«—Comment cela, et pourquoi? 

«—Parce qu'il y a quinze jours à peine j'ai été interrogé sur le même sujet. 

«—Par qui? 

«—Par un banquier de Paris, mon correspondant. 

«—Que vous nommez? 

«—M. Danglars.» 

—Lui! s'écria Albert; en effet, c'est bien lui qui depuis si longtemps poursuit mon pauvre père de sa haine jalouse; lui, l'homme prétendu populaire, qui ne peut pardonner au comte de Morcerf d'être pair de France. Et, tenez, cette rupture de mariage sans raison donnée; oui, c'est bien cela. 

—Informez-vous, Albert (mais ne vous emportez pas d'avance), informez-vous, vous dis-je, et si la chose est vraie\dots 

—Oh! oui, si la chose est vraie! s'écria le jeune homme, il me paiera tout ce que j'ai souffert. 

—Prenez garde, Morcerf, c'est un homme déjà vieux. 

—J'aurai égard à son âge comme il a eu égard à l'honneur de ma famille; s'il en voulait à mon père, que ne frappait-il mon père? Oh! non, il a eu peur de se trouver en face d'un homme! 

—Albert, je ne vous condamne pas, je ne fais que vous retenir; Albert, agissez prudemment. 

—Oh! n'ayez pas peur; d'ailleurs, vous m'accompagnerez, Beauchamp, les choses solennelles doivent être traitées devant témoin. Avant la fin de cette journée, si M. Danglars est le coupable, M. Danglars aura cessé de vivre ou je serai mort. Pardieu, Beauchamp, je veux faire de belles funérailles à mon honneur! 

—Eh bien, alors, quand de pareilles résolutions sont prises, Albert, il faut les mettre à exécution à l'instant même. Vous voulez aller chez M. Danglars? partons.» 

On envoya chercher un cabriolet de place. En entrant dans l'hôtel du banquier, on aperçut le phaéton et le domestique de M. Andrea Cavalcanti à la porte. 

«Ah! parbleu! voilà qui va bien, dit Albert avec une voix sombre. Si M. Danglars ne veut pas se battre avec moi, je lui tuerai son gendre. Cela doit se battre, un Cavalcanti.» 

On annonça le jeune homme au banquier, qui, au nom d'Albert, sachant ce qui s'était passé la veille, fit défendre sa porte. Mais il était trop tard, il avait suivi le laquais; il entendit l'ordre donné, força la porte et pénétra, suivi de Beauchamp, jusque dans le cabinet du banquier. 

«Mais, monsieur! s'écria celui-ci, n'est-on plus maître de recevoir chez soi qui l'on veut, ou qui l'on ne veut pas? Il me semble que vous vous oubliez étrangement. 

—Non, monsieur, dit froidement Albert, il y a des circonstances, et vous êtes dans une de celles-là, où il faut, sauf lâcheté, je vous offre ce refuge, être chez soi pour certaines personnes du moins. 

—Alors, que me voulez-vous donc, monsieur? 

—Je veux, dit Morcerf, s'approchant sans paraître faire attention à Cavalcanti qui était adossé à la cheminée, je veux vous proposer un rendez-vous dans un coin écarté, où personne ne vous dérangera pendant dix minutes, je ne vous en demande pas davantage; où, des deux hommes qui se sont rencontrés, il en restera un sous les feuilles.» 

Danglars pâlit, Cavalcanti fit un mouvement. Albert se retourna vers le jeune homme: 

«Oh! mon Dieu! dit-il, venez si vous voulez, monsieur le comte, vous avez le droit d'y être, vous êtes presque de la famille, et je donne de ces sorties de rendez-vous à autant de gens qu'il s'en trouvera pour les accepter.» 

Cavalcanti regarda d'un air stupéfait Danglars lequel faisant un effort, se leva et s'avança entre les deux jeunes gens. L'attaque d'Albert à Andrea venait de le placer sur un autre terrain, et il espérait que la visite d'Albert avait une autre cause que celle qu'il lui avait supposée d'abord. 

«Ah çà! monsieur, dit-il à Albert, si vous venez ici chercher querelle à monsieur parce que je l'ai préféré à vous, je vous préviens que je ferai de cela une affaire de procureur du roi. 

—Vous vous trompez, monsieur, dit Morcerf avec un sombre sourire, je ne parle pas de mariage le moins du monde, et je ne m'adresse à M. Cavalcanti que parce qu'il m'a semblé avoir eu un instant l'intention d'intervenir dans notre discussion. Et puis, tenez, au reste, vous avez raison, dit-il, je cherche aujourd'hui querelle à tout le monde; mais soyez tranquille, monsieur Danglars, la priorité vous appartient. 

—Monsieur, répondit Danglars, pâle de colère et de peur, je vous avertis que lorsque j'ai le malheur de rencontrer sur mon chemin un dogue enragé, je le tue et que, loin de me croire coupable, je pense avoir rendu un service à la société. Or, si vous êtes enragé et que vous tendiez à me mordre, je vous en préviens, je vous tuerai sans pitié. Tiens! est-ce ma faute, à moi, si votre père est déshonoré? 

—Oui, misérable! s'écria Morcerf, c'est ta faute!» 

Danglars fit un pas en arrière. 

«Ma faute! à moi, dit-il; mais vous êtes fou! Est-ce que je sais l'histoire grecque, moi? Est-ce que j'ai voyagé dans tous ces pays-là? Est-ce que c'est moi qui ai conseillé à votre père de vendre les châteaux de Janina? de trahir\dots 

—Silence! dit Albert d'une voix sourde. Non, ce n'est pas vous qui directement avez fait cet éclat et causé ce malheur, mais c'est vous qui l'avez hypocritement provoqué. 

—Moi! 

—Oui, vous! d'où vient la révélation? 

—Mais il me semble que le journal vous l'a dit: de Janina, parbleu! 

—Qui a écrit à Janina? 

—À Janina? 

—Oui. Qui a écrit pour demander des renseignements sur mon père? 

—Il me semble que tout le monde peut écrire à Janina. 

—Une seule personne a écrit cependant. 

—Une seule? 

—Oui! et cette personne, c'est vous. 

—J'ai écrit, sans doute; il me semble que lorsqu'on marie sa fille à un jeune homme, on peut prendre des renseignements sur la famille de ce jeune homme; c'est non seulement un droit, mais encore un devoir. 

—Vous avez écrit, monsieur, dit Albert, sachant parfaitement la réponse qui vous viendrait. 

—Moi? Ah! je vous le jure bien, s'écria Danglars avec une confiance et une sécurité qui venaient encore moins de sa peur peut-être que de l'intérêt qu'il ressentait au fond pour le malheureux jeune homme; je vous jure que jamais je n'eusse pensé à écrire à Janina. Est-ce que je connaissais la catastrophe d'Ali-Pacha, moi? 

—Alors quelqu'un vous a donc poussé à écrire? 

—Certainement. 

—On vous a poussé? 

—Oui. 

—Qui cela?\dots achevez\dots dites\dots 

—Pardieu! rien de plus simple, je parlais du passé de votre père, je disais que la source de sa fortune était toujours restée obscure. La personne m'a demandé où votre père avait fait cette fortune. J'ai répondu: «En Grèce.» Alors elle m'a dit: «Eh bien, écrivez à Janina.» 

—Et qui vous a donné ce conseil? 

—Parbleu! le comte de Monte-Cristo, votre ami. 

—Le comte de Monte-Cristo vous a dit d'écrire à Janina? 

—Oui, et j'ai écrit. Voulez-vous voir ma correspondance? je vous la montrerai.» 

Albert et Beauchamp se regardèrent. 

«Monsieur, dit alors Beauchamp, qui n'avait point encore pris la parole, il me semble que vous accusez le comte, qui est absent de Paris, et qui ne peut se justifier en ce moment? 

—Je n'accuse personne, monsieur, dit Danglars, je raconte, et je répéterai devant M. le comte de Monte-Cristo ce que je viens de dire devant vous. 

—Et le comte sait quelle réponse vous avez reçue? 

—Je la lui ai montrée. 

—Savait-il que le nom de baptême de mon père était Fernand, et que son nom de famille était Mondego? 

—Oui, je le lui avais dit depuis longtemps au surplus, je n'ai fait là-dedans que ce que tout autre eût fait à ma place, et même peut-être beaucoup moins. Quand, le lendemain de cette réponse, poussé par M. de Monte-Cristo, votre père est venu me demander ma fille officiellement, comme cela se fait quand on veut en finir, j'ai refusé, j'ai refusé net, c'est vrai, mais sans explication, sans éclat. En effet, pourquoi aurais-je fait un éclat? En quoi l'honneur ou le déshonneur de M. de Morcerf m'importe-t-il? Cela ne faisait ni hausser ni baisser la rente.» 

Albert sentit la rougeur lui monter au front; il n'y avait plus de doute, Danglars se défendait avec la bassesse, mais avec l'assurance d'un homme qui dit, sinon toute la vérité, du moins une partie de la vérité, non point par conscience, il est vrai, mais par terreur. D'ailleurs, que cherchait Morcerf? ce n'était pas le plus ou moins de culpabilité de Danglars ou de Monte-Cristo, c'était un homme qui répondît de l'offense légère ou grave, c'était un homme qui se battît, et il était évident que Danglars ne se battrait pas. 

Et puis, chacune des choses oubliées ou inaperçues redevenait visible à ses yeux ou présente à son souvenir. Monte-Cristo savait tout, puisqu'il avait acheté la fille d'Ali-Pacha, or, sachant tout, il avait conseillé à Danglars d'écrire à Janina. Cette réponse connue, il avait accédé au désir manifesté par Albert d'être présenté à Haydée; une fois devant elle, il avait laissé l'entretien tomber sur la mort d'Ali, ne s'opposant pas au récit d'Haydée (mais ayant sans doute donné à la jeune fille dans les quelques mots romaïques qu'il avait prononcés des instructions qui n'avaient point permis à Morcerf de reconnaître son père); d'ailleurs n'avait-il pas prié Morcerf de ne pas prononcer le nom de son père devant Haydée? Enfin il avait mené Albert en Normandie au moment où il savait que le grand éclat devait se faire. Il n'y avait pas à en douter, tout cela était un calcul, et, sans aucun doute, Monte-Cristo s'entendait avec les ennemis de son père. 

Albert prit Beauchamp dans un coin et lui communiqua toutes ses idées. 

«Vous avez raison, dit celui-ci; M. Danglars n'est, dans ce qui est arrivé, que pour la partie brutale et matérielle; c'est à M. de Monte-Cristo que vous devez demander une explication.» 

Albert se retourna. 

«Monsieur, dit-il à Danglars, vous comprenez que je ne prends pas encore de vous un congé définitif; il me reste à savoir si vos inculpations sont justes, et je vais de ce pas m'en assurer chez M. le comte de Monte-Cristo.» 

Et, saluant le banquier, il sortit avec Beauchamp sans paraître autrement s'occuper de Cavalcanti. 

Danglars les reconduisit jusqu'à la porte, et, à la porte, renouvela à Albert l'assurance qu'aucun motif de haine personnel ne l'animait contre M. le comte de Morcerf. 