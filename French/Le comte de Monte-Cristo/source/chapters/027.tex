\chapter{Le récit}

\lettrine[ante=«]{A}{vant} tout, dit Caderousse, je dois, monsieur, vous prier de me promettre une chose.

\zz
—Laquelle? demanda l'abbé.

\zz
—C'est que jamais, si vous faites un usage quelconque des détails que je vais vous donner, on ne saura que ces détails viennent de moi, car ceux dont je vais vous parler sont riches et puissants, et, s'ils me touchaient seulement du bout du doigt, ils me briseraient comme verre.

—Soyez tranquille, mon ami, dit l'abbé, je suis prêtre, et les confessions meurent dans mon sein; rappelez-vous que nous n'avons d'autre but que d'accomplir dignement les dernières volontés de notre ami; parlez donc sans ménagement comme sans haine; dites la vérité, toute la vérité: je ne connais pas et ne connaîtrai probablement jamais les personnes dont vous allez me parler; d'ailleurs, je suis Italien et non pas Français; j'appartiens à Dieu et non pas aux hommes, et je vais rentrer dans mon couvent, dont je ne suis sorti que pour remplir les dernières volontés d'un mourant.»

Cette promesse positive parut donner à Caderousse un peu d'assurance.

«Eh bien, en ce cas, dit Caderousse, je veux, je dirai même plus, je dois vous détromper sur ces amitiés que le pauvre Edmond croyait sincères et dévouées.

—Commençons par son père, s'il vous plaît, dit l'abbé. Edmond m'a beaucoup parlé de ce vieillard, pour lequel il avait un profond amour.

—L'histoire est triste, monsieur, dit Caderousse en hochant la tête; vous en connaissez probablement les commencements.

—Oui, répondit l'abbé, Edmond m'a raconté les choses jusqu'au moment où il a été arrêté, dans un petit cabaret près de Marseille.

—À la Réserve! ô mon Dieu, oui! je vois encore la chose comme si j'y étais.

—N'était-ce pas au repas même de ses fiançailles?

—Oui, et le repas qui avait eu un gai commencement eut une triste fin: un commissaire de police suivi de quatre fusiliers entra, et Dantès fut arrêté.

—Voilà où s'arrête ce que je sais, monsieur, dit le prêtre; Dantès lui-même ne savait rien autre que ce qui lui était absolument personnel, car il n'a jamais revu aucune des cinq personnes que je vous ai nommées, ni entendu parler d'elles.

—Eh bien, Dantès une fois arrêté, M. Morrel courut prendre des informations: elles furent bien tristes. Le vieillard retourna seul dans sa maison, ploya son habit de noces en pleurant, passa toute la journée à aller et venir dans sa chambre, et le soir ne se coucha point, car je demeurais au-dessous de lui et je l'entendis marcher toute la nuit; moi-même, je dois le dire, je ne dormis pas non plus, car la douleur de ce pauvre père me faisait grand mal, et chacun de ses pas me broyait le cœur, comme s'il eût réellement posé son pied sur ma poitrine.

«Le lendemain, Mercédès vint à Marseille pour implorer la protection de M. de Villefort: elle n'obtint rien; mais, du même coup, elle alla rendre visite au vieillard. Quand elle le vit si morne et abattu, qu'il avait passé la nuit sans se mettre au lit, qu'il n'avait pas mangé depuis la veille, elle voulut l'emmener pour en prendre soin, mais le vieillard ne voulut jamais y consentir.

«—Non, disait-il, je ne quitterai pas la maison, car c'est moi que mon pauvre enfant aime avant toutes choses, et, s'il sort de prison, c'est moi qu'il accourra voir d'abord. Que dirait-il si je n'étais point là à l'attendre?

«J'écoutais tout cela du carré, car j'aurais voulu que Mercédès déterminât le vieillard à la suivre; ce pas retentissant tous les jours sur ma tête ne me laissait pas un instant de repos.

—Mais ne montiez-vous pas vous-même près du vieillard pour le consoler? demanda le prêtre.

—Ah! monsieur! répondit Caderousse, on ne console que ceux qui veulent être consolés, et lui ne voulait pas l'être: d'ailleurs, je ne sais pourquoi, mais il me semblait qu'il avait de la répugnance à me voir. Une nuit cependant que j'entendais ses sanglots, je n'y pus résister et je montai; mais quand j'arrivai à la porte, il ne sanglotait plus, il priait. Ce qu'il trouvait d'éloquentes paroles et de pitoyables supplications, je ne saurais vous le redire, monsieur: c'était plus que de la piété, c'était plus que de la douleur; aussi, moi qui ne suis pas cagot et qui n'aime pas les jésuites, je me dis ce jour-là: C'est bien heureux, en vérité, que je sois seul, et que le Bon Dieu ne m'ait pas envoyé d'enfants, car si j'étais père et que je ressentisse une douleur semblable à celle du pauvre vieillard, ne pouvant trouver dans ma mémoire ni dans mon cœur tout ce qu'il dit au Bon Dieu, j'irais tout droit me précipiter dans la mer pour ne pas souffrir plus longtemps.

—Pauvre père! murmura le prêtre.

—De jour en jour, il vivait plus seul et plus isolé: souvent M. Morrel et Mercédès venaient pour le voir, mais sa porte était fermée; et, quoique je fusse bien sûr qu'il était chez lui, il ne répondait pas. Un jour que, contre son habitude, il avait reçu Mercédès, et que la pauvre enfant, au désespoir elle-même, tentait de le réconforter:

«—Crois-moi, ma fille, lui dit-il, il est mort; et, au lieu que nous l'attendions, c'est lui qui nous attend: je suis bien heureux, c'est moi qui suis le plus vieux et qui, par conséquent, le reverrai le premier.

«Si bon que l'on soit, voyez-vous, on cesse bientôt de voir les gens qui vous attristent; le vieux Dantès finit par demeurer tout à fait seul: je ne voyais plus monter de temps en temps chez lui que des gens inconnus, qui descendaient avec quelque paquet mal dissimulé; j'ai compris depuis ce que c'était que ces paquets: il vendait peu à peu ce qu'il avait pour vivre. Enfin, le bonhomme arriva au bout de ses pauvres hardes; il devait trois termes: on menaça de le renvoyer; il demanda huit jours encore, on les lui accorda. Je sus ce détail parce que le propriétaire entra chez moi en sortant de chez lui.

«Pendant les trois premiers jours, je l'entendis marcher comme d'habitude; mais le quatrième, je n'entendis plus rien. Je me hasardai à monter: la porte était fermée; mais à travers la serrure je l'aperçu si pâle et si défait, que, le jugeant bien malade, je fis prévenir M. Morrel et courus chez Mercédès. Tous deux s'empressèrent de venir. M. Morrel amenait un médecin; le médecin reconnut une gastro-entérite et ordonna la diète. J'étais là, monsieur, et je n'oublierai jamais le sourire du vieillard à cette ordonnance.

«Dès lors, il ouvrit sa porte: il avait une excuse pour ne plus manger; le médecin avait ordonné la diète.»

L'abbé poussa une espèce de gémissement.

«Cette histoire vous intéresse, n'est-ce pas, monsieur? dit Caderousse.

—Oui, répondit l'abbé; elle est attendrissante.

—Mercédès revint; elle le trouva si changé, que, comme la première fois, elle voulut le faire transporter chez elle. C'était aussi l'avis de M. Morrel, qui voulait opérer le transport de force; mais le vieillard cria tant, qu'ils eurent peur. Mercédès resta au chevet de son lit. M. Morrel s'éloigna en faisant signe à la Catalane qu'il laissait une bourse sur la cheminée. Mais, armé de l'ordonnance du médecin, le vieillard ne voulut rien prendre. Enfin, après neuf jours de désespoir et d'abstinence, le vieillard expira en maudissant ceux qui avaient causé son malheur et disant à Mercédès:

«—Si vous revoyez mon Edmond, dites-lui que je meurs en le bénissant.»

L'abbé se leva, fit deux tours dans la chambre en portant une main frémissante à sa gorge aride.

«Et vous croyez qu'il est mort\dots.

—De faim\dots monsieur, de faim, dit Caderousse; j'en réponds aussi vrai que nous sommes ici deux chrétiens.»

L'abbé, d'une main convulsive, saisit le verre d'eau encore à moitié plein, le vida d'un trait et se rassit les yeux rougis et les joues pâles.

«Avouez que voilà un grand malheur! dit-il d'une voix rauque.

—D'autant plus grand, monsieur, que Dieu n'y est pour rien, et que les hommes seuls en sont cause.

—Passons donc à ces hommes, dit l'abbé; mais songez-y, continua-t-il d'un air presque menaçant, vous vous êtes engagé à me tout dire: voyons, quels sont ces hommes qui ont fait mourir le fils de désespoir, et le père de faim?

—Deux hommes jaloux de lui, monsieur, l'un par amour, l'autre par ambition: Fernand et Danglars.

—Et de quelle façon se manifesta cette jalousie, dites?

—Ils dénoncèrent Edmond comme agent bonapartiste.

—Mais lequel des deux le dénonça, lequel des deux fut le vrai coupable.

—Tous deux, monsieur, l'un écrivit la lettre, l'autre la mit à la poste.

—Et où cette lettre fut-elle écrite?

—À la Réserve même, la veille du mariage.

—C'est bien cela, c'est bien cela, murmura l'abbé. Ô Faria! Faria! comme tu connaissais les hommes et les choses!

—Vous dites, monsieur? demanda Caderousse.

—Rien, reprit le prêtre; continuez.

—Ce fut Danglars qui écrivit la dénonciation de la main gauche pour que son écriture ne fût pas reconnue, et Fernand qui l'envoya.

—Mais, s'écria tout à coup l'abbé, vous étiez là, vous!

—Moi! dit Caderousse étonné; qui vous a dit que j'y étais?»

L'abbé vit qu'il s'était lancé trop avant.

«Personne, dit-il, mais pour être si bien au fait de tous ces détails, il faut que vous en ayez été le témoin.

—C'est vrai, dit Caderousse d'une voix étouffée, j'y étais.

—Et vous ne vous êtes pas opposé à cette infamie? dit l'abbé; alors vous êtes leur complice.

—Monsieur, dit Caderousse, ils m'avaient fait boire tous deux au point que j'en avais à peu près perdu la raison. Je ne voyais plus qu'à travers un nuage. Je dis tout ce que peut dire un homme dans cet état; mais ils me répondirent tous deux que c'était une plaisanterie qu'ils avaient voulu faire, et que cette plaisanterie n'aurait pas de suite.

—Le lendemain, monsieur, le lendemain, vous vîtes bien qu'elle en avait; cependant vous ne dîtes rien; vous étiez là cependant lorsqu'il fut arrêté.

—Oui, monsieur, j'étais là et je voulus parler, je voulus tout dire, mais Danglars me retint.

—«Et s'il est coupable, par hasard, me dit-il, s'il a véritablement relâché à l'île d'Elbe, s'il est véritablement chargé d'une lettre pour le comité bonapartiste de Paris, si on trouve cette lettre sur lui, ceux qui l'auront soutenu passeront pour ses complices.»

«J'eus peur de la politique telle qu'elle se faisait alors, je l'avoue; je me tus, ce fut une lâcheté, j'en conviens, mais ce ne fut pas un crime.

—Je comprends; vous laissâtes faire, voilà tout.

—Oui, monsieur, répondit Caderousse, et c'est mon remords de la nuit et du jour. J'en demande bien souvent pardon à Dieu, je vous le jure, d'autant plus que cette action, la seule que j'aie sérieusement à me reprocher dans tout le cours de ma vie, est sans doute la cause de mes adversités. J'expie un instant d'égoïsme; aussi, c'est ce que je dis toujours à la Carconte lorsqu'elle se plaint: «Tais-toi, femme, c'est Dieu qui le veut ainsi.»

Et Caderousse baissa la tête avec tous les signes d'un vrai repentir.

«Bien, monsieur, dit l'abbé, vous avez parlé avec franchise; s'accuser ainsi, c'est mériter son pardon.

—Malheureusement, dit Caderousse, Edmond est mort et ne m'a pas pardonné, lui!

—Il ignorait, dit l'abbé\dots

—Mais il sait maintenant, peut-être, reprit Caderousse; on dit que les morts savent tout.»

Il se fit un instant de silence: l'abbé s'était levé et se promenait pensif; il revint à sa place et se rassit.

«Vous m'avez nommé déjà deux ou trois fois un certain M. Morrel, dit-il. Qu'était-ce que cet homme?

—C'était l'armateur du \textit{Pharaon}, le patron de Dantès.

—Et quel rôle a joué cet homme dans toute cette triste affaire? demanda l'abbé.

—Le rôle d'un homme honnête, courageux et affectionné, monsieur. Vingt fois il intercéda pour Edmond; quand l'empereur rentra, il écrivit, pria, menaça, si bien qu'à la seconde Restauration il fut fort persécuté comme bonapartiste. Dix fois, comme je vous l'ai dit, il était venu chez le père Dantès pour le retirer chez lui, et la veille ou la surveille de sa mort, je vous l'ai dit encore, il avait laissé sur la cheminée une bourse avec laquelle on paya les dettes du bonhomme et l'on subvint à son enterrement; de sorte que le pauvre vieillard put du moins mourir comme il avait vécu, sans faire de tort à personne. C'est encore moi qui ai la bourse, une grande bourse en filet rouge.

—Et, demanda l'abbé, ce M. Morrel vit-il encore?

—Oui, dit Caderousse.

—En ce cas, reprit l'abbé, ce doit être un homme béni de Dieu, il doit être riche\dots heureux?\dots»

Caderousse sourit amèrement.

«Oui, heureux, comme moi, dit-il.

—M. Morrel serait malheureux! s'écria l'abbé.

—Il touche à la misère, monsieur, et bien plus, il touche au déshonneur.

—Comment cela?

—Oui, reprit Caderousse, c'est comme cela; après vingt-cinq ans de travail, après avoir acquis la plus honorable place dans le commerce de Marseille, M. Morrel est ruiné de fond en comble. Il a perdu cinq vaisseaux en deux ans, a essuyé trois banqueroutes effroyables, et n'a plus d'espérance que dans ce même \textit{Pharaon} que commandait le pauvre Dantès, et qui doit revenir des Indes avec un chargement de cochenille et d'indigo. Si ce navire-là manque comme les autres, il est perdu.

—Et, dit l'abbé, a-t-il une femme, des enfants, le malheureux?

—Oui, il a une femme qui, dans tout cela, se conduit comme une sainte; il a une fille qui allait épouser un homme qu'elle aimait, et à qui sa famille ne veut plus laisser épouser une fille ruinée; il a un fils enfin, lieutenant dans l'armée; mais, vous le comprenez bien, tout cela double sa douleur au lieu de l'adoucir, à ce pauvre cher homme. S'il était seul, il se brûlerait la cervelle et tout serait dit.

—C'est affreux! murmura le prêtre.

—Voilà comme Dieu récompense la vertu, monsieur, dit Caderousse. Tenez, moi qui n'ai jamais fait une mauvaise action à part ce que je vous ai raconté, moi, je suis dans la misère; moi, après avoir vu mourir ma pauvre femme de la fièvre, sans pouvoir rien faire pour elle, je mourrai de faim comme est mort le père Dantès, tandis que Fernand et Danglars roulent sur l'or.

—Et comment cela?

—Parce que tout leur a tourné à bien, tandis qu'aux honnêtes gens tout tourne à mal.

—Qu'est devenu Danglars? le plus coupable, n'est-ce pas, l'instigateur?

—Ce qu'il est devenu? il a quitté Marseille; il est entré, sur la recommandation de M. Morrel, qui ignorait son crime comme commis d'ordre chez un banquier espagnol; à l'époque de la guerre d'Espagne il s'est chargé d'une part dans les fournitures de l'armée française et a fait fortune; alors, avec ce premier argent il a joué sur les fonds, et a triplé, quadruplé ses capitaux, et, veuf lui-même de la fille de son banquier, il a épousé une veuve, Mme de Nargonne, fille de M. Servieux, chambellan du roi actuel, et qui jouit de la plus grande faveur. Il s'était fait millionnaire, on l'a fait baron; de sorte qu'il est baron Danglars maintenant, qu'il a un hôtel rue du Mont-Blanc, dix chevaux dans ses écuries, six laquais dans son antichambre, et je ne sais combien de millions dans ses caisses.

—Ah! fit l'abbé avec un singulier accent; et il est heureux?

—Ah! heureux, qui peut dire cela? Le malheur ou le bonheur, c'est le secret des murailles; les murailles ont des oreilles, mais elles n'ont pas de langue; si l'on est heureux avec une grande fortune, Danglars est heureux.

—Et Fernand?

—Fernand, c'est bien autre chose encore.

—Mais comment a pu faire fortune un pauvre pêcheur catalan, sans ressources, sans éducation? Cela me passe, je vous l'avoue.

—Et cela passe tout le monde aussi; il faut qu'il y ait dans sa vie quelque étrange secret que personne ne sait.

—Mais enfin par quels échelons visibles a-t-il monté à cette haute fortune ou à cette haute position?

—À toutes deux, monsieur, à toutes deux! lui a fortune et position tout ensemble.

—C'est un conte que vous me faites là.

—Le fait est que la chose en a bien l'air; mais écoutez, et vous allez comprendre.

«Fernand, quelques jours avant le retour, était tombé à la conscription. Les Bourbons, le laissèrent bien tranquille aux Catalans, mais Napoléon revint, une levée extraordinaire fut décrétée, et Fernand fut forcé de partir. Moi aussi, je partis; mais comme j'étais plus vieux que Fernand et que je venais d'épouser ma pauvre femme, je fus envoyé sur les côtes seulement.

«Fernand, lui, fut enrégimenté dans les troupes actives, gagna la frontière avec son régiment, et assista à la bataille de Ligny.

«La nuit qui suivit la bataille, il était de planton à la porte du général qui avait des relations secrètes avec l'ennemi. Cette nuit même le général devait rejoindre les Anglais. Il proposa à Fernand de l'accompagner; Fernand accepta, quitta son poste et suivit le général.

«Ce qui eût fait passer Fernand à un conseil de guerre si Napoléon fût resté sur le trône lui servit de recommandation près des Bourbons. Il rentra en France avec l'épaulette de sous-lieutenant; et comme la protection du général, qui est en haute faveur, ne l'abandonna point, il était capitaine en 1823, lors de la guerre d'Espagne, c'est-à-dire au moment même où Danglars risquait ses premières spéculations. Fernand était Espagnol, il fut envoyé à Madrid pour y étudier l'esprit de ses compatriotes; il y retrouva Danglars, s'aboucha avec lui, promit à son général un appui parmi les royalistes de la capitale et des provinces, reçut des promesses, prit de son côté des engagements, guida son régiment par les chemins connus de lui seul dans des gorges gardées par des royalistes, et enfin rendit dans cette courte campagne de tels services, qu'après la prise du Trocadéro il fut nommé colonel et reçut la croix d'officier de la Légion d'honneur avec le titre de comte.

—Destinée! destinée! murmura l'abbé.

—Oui, mais écoutez, ce n'est pas le tout. La guerre d'Espagne finie, la carrière de Fernand se trouvait compromise par la longue paix qui promettait de régner en Europe. La Grèce seule était soulevée contre la Turquie, et venait de commencer la guerre de son indépendance; tous les yeux étaient tournés vers Athènes: c'était la mode de plaindre et de soutenir les Grecs. Le gouvernement français, sans les protéger ouvertement, comme vous savez, tolérait les migrations partielles. Fernand sollicita et obtint la permission d'aller servir en Grèce, en demeurant toujours porté néanmoins sur les contrôles de l'armée.

«Quelque temps après, on apprit que le comte de Morcerf, c'était le nom qu'il portait, était entré au service d'Ali-Pacha avec le grade de général instructeur.

«Ali-Pacha fut tué, comme vous savez; mais avant de mourir il récompensa les services de Fernand en lui laissant une somme considérable avec laquelle Fernand revint en France, où son grade de lieutenant général lui fut confirmé.

—De sorte qu'aujourd'hui?\dots demanda l'abbé.

—De sorte qu'aujourd'hui, poursuivit Caderousse, il possède un hôtel magnifique à Paris, rue du Helder, nº 27.»

L'abbé ouvrit la bouche, demeura un instant comme un homme qui hésite, mais faisant un effort sur lui-même:

«Et Mercédès, dit-il, on m'a assuré qu'elle avait disparu?

—Disparu, dit Caderousse, oui, comme disparaît le soleil pour se lever le lendemain plus éclatant.

—A-t-elle donc fait fortune aussi? demanda l'abbé avec un sourire ironique.

—Mercédès est à cette heure une des plus grandes dames de Paris, dit Caderousse.

—Continuez, dit l'abbé, il me semble que j'écoute le récit d'un rêve. Mais j'ai vu moi-même des choses si extraordinaires, que celles que vous me dites m'étonnent moins.

—Mercédès fut d'abord désespérée du coup qui lui enlevait Edmond. Je vous ai dit ses instances près de M. de Villefort et son dévouement pour le père de Dantès. Au milieu de son désespoir une nouvelle douleur vint l'atteindre, ce fut le départ de Fernand, de Fernand dont elle ignorait le crime, et qu'elle regardait comme son frère.

«Fernand partit, Mercédès demeura seule.

«Trois mois s'écoulèrent pour elle dans les larmes: pas de nouvelles d'Edmond, pas de nouvelles de Fernand; rien devant les yeux qu'un vieillard qui s'en allait mourant de désespoir.

«Un soir, après être restée toute la journée assise, comme c'était son habitude, à l'angle des deux chemins qui se rendent de Marseille aux Catalans, elle rentra chez elle plus abattue qu'elle ne l'avait encore été: ni son amant ni son ami ne revenaient par l'un ou l'autre de ces deux chemins, et elle n'avait de nouvelles ni de l'un ni de l'autre.

«Tout à coup il lui sembla entendre un pas connu; elle se retourna avec anxiété, la porte s'ouvrit, elle vit apparaître Fernand avec son uniforme de sous-lieutenant.

«Ce n'était pas la moitié de ce qu'elle pleurait, mais c'était une portion de sa vie passée qui revenait à elle.

«Mercédès saisit les mains de Fernand avec un transport que celui-ci prit pour de l'amour, et qui n'était que la joie de n'être plus seule au monde et de revoir enfin un ami, après de longues heures de la tristesse solitaire. Et puis, il faut le dire, Fernand n'avait jamais été haï, il n'était pas aimé, voilà tout; un autre tenait tout le cœur de Mercédès, cet autre était absent\dots était disparu\dots était mort peut-être. À cette dernière idée, Mercédès éclatait en sanglots et se tordait les bras de douleur; mais cette idée, qu'elle repoussait autrefois quand elle lui était suggérée par un autre lui revenait maintenant tout seule à l'esprit; d'ailleurs, de son côté, le vieux Dantès ne cessait de lui dire: «Notre Edmond est mort, car s'il n'était pas mort, il nous reviendrait.»

«Le vieillard mourut, comme je vous l'ai dit: s'il eût vécu, peut-être Mercédès ne fût-elle jamais devenue la femme d'un autre; car il eût été là pour lui reprocher son infidélité. Fernand comprit cela. Quand il connut la mort du vieillard, il revint. Cette fois, il était lieutenant. Au premier voyage, il n'avait pas dit à Mercédès un mot d'amour; au second, il lui rappela qu'il l'aimait.

«Mercédès lui demanda six mois encore pour attendre et pleurer Edmond.

—Au fait, dit l'abbé avec un sourire amer, cela faisait dix-huit mois en tout. Que peut demander davantage l'amant le plus adoré?»

Puis il murmura les paroles du poète anglais: \textit{Frailty, thy name is woman!}

«Six mois après, reprit Caderousse, le mariage eut lieu à l'église des Accoules.

—C'était la même église où elle devait épouser Edmond, murmura le prêtre; il n'y avait que le fiancé de changé, voilà tout.

—Mercédès se maria donc, continua Caderousse; mais, quoique aux yeux de tous elle parût calme, elle ne manqua pas moins de s'évanouir en passant devant la Réserve, où dix-huit mois auparavant avaient été célébrées ses fiançailles avec celui qu'elle eût vu qu'elle aimait encore, si elle eût oser regarder au fond de son cœur.

«Fernand, plus heureux, mais non pas plus tranquille, car je le vis à cette époque, et il craignait sans cesse le retour d'Edmond, Fernand s'occupa aussitôt de dépayser sa femme et de s'exiler lui-même; il y avait à la fois trop de dangers et de souvenirs à rester aux Catalans. Huit jours après la noce, ils partirent.

—Et revîtes-vous Mercédès? demanda le prêtre.

—Oui, au moment de la guerre d'Espagne, à Perpignan où Fernand l'avait laissée; elle faisait alors l'éducation de son fils.»

L'abbé tressaillit. «De son fils? dit-il.

—Oui, répondit Caderousse, du petit Albert.

—Mais pour instruire ce fils, continua l'abbé, elle avait donc reçu de l'éducation elle-même? Il me semblait avoir entendu dire à Edmond que c'était la fille d'un simple pêcheur, belle, mais inculte.

—Oh! dit Caderousse, connaissait-il donc si mal sa propre fiancée! Mercédès eût pu devenir reine, monsieur, si la couronne se devait poser seulement sur les têtes les plus belles et les plus intelligentes. Sa fortune grandissait déjà, et elle grandissait avec sa fortune. Elle apprenait le dessin, elle apprenait la musique, elle apprenait tout. D'ailleurs, je crois, entre nous, qu'elle ne faisait tout cela que pour se distraire, pour oublier, et qu'elle ne mettait tant de choses dans sa tête que pour combattre ce qu'elle avait dans le cœur. Mais maintenant tout doit être dit, continua Caderousse: la fortune et les honneurs l'ont consolée sans doute. Elle est riche, elle est comtesse, et cependant\dots»

Caderousse s'arrêta.

«Cependant quoi? demanda l'abbé.

—Cependant, je suis sûr qu'elle n'est pas heureuse, dit Caderousse.

—Et qui vous le fait croire?

—Eh bien, quand je me suis trouvé trop malheureux moi-même, j'ai pensé que mes anciens amis m'aideraient en quelque chose. Je me suis présenté chez Danglars, qui ne m'a pas même reçu. J'ai été chez Fernand, qui m'a fait remettre cent francs par son valet de chambre.

—Alors vous ne les vîtes ni l'un ni l'autre?

—Non; mais Mme de Morcerf m'a vu, elle.

—Comment cela?

—Lorsque je suis sorti, une bourse est tombée à mes pieds, elle contenait vingt-cinq louis: j'ai levé vivement la tête et j'ai vu Mercédès qui refermait la persienne.

—Et M. de Villefort? demanda l'abbé.

—Oh! lui n'avait pas été mon ami; je ne le connaissais pas; lui, je n'avais rien à lui demander.

—Mais ne savez-vous point ce qu'il est devenu, et la part qu'il a prise au malheur d'Edmond?

—Non, je sais seulement que, quelque temps après l'avoir fait arrêter, il a épousé Mlle de Saint-Méran, et bientôt a quitté Marseille. Sans doute que le bonheur lui aura souri comme aux autres, sans doute qu'il est riche comme Danglars, considéré comme Fernand; moi seul, vous le voyez, suis resté pauvre, misérable et oublié de Dieu.

—Vous vous trompez, mon ami, dit l'abbé: Dieu peut paraître oublier parfois, quand sa justice se repose; mais il vient toujours un moment où il se souvient, et en voici la preuve.»

À ces mots, l'abbé tira le diamant de sa poche, et le présentant à Caderousse:

«Tenez, mon ami, lui dit-il, prenez ce diamant, car il est à vous.

—Comment, à moi seul! s'écria Caderousse! Ah! monsieur, ne raillez-vous pas?

—Ce diamant devait être partagé entre ses amis: Edmond n'avait qu'un seul ami, le partage devient donc inutile. Prenez ce diamant et vendez-le; il vaut cinquante mille francs, je vous le répète, de cette somme, je l'espère, suffira pour vous tirer de la misère.

—Oh! monsieur, dit Caderousse en avançant timidement une main et en essuyant de l'autre la sueur qui perlait sur son front; oh! monsieur, ne faites pas une plaisanterie du bonheur ou du désespoir d'un homme!

—Je sais ce que c'est que le bonheur et ce que c'est que le désespoir, et je ne jouerai jamais à plaisir avec les sentiments. Prenez donc, mais en échange\dots»

Caderousse qui touchait déjà le diamant, retira sa main.

L'abbé sourit.

«En échange, continua-t-il, donnez-moi cette bourse de soie rouge que M. Morrel avait laissée sur la cheminée du vieux Dantès, et qui, me l'avez-vous dit, est encore entre vos mains.»

Caderousse, de plus en plus étonné, alla vers une grande armoire de chêne, l'ouvrit et donna à l'abbé une bourse longue, de soie rouge flétrie, et autour de laquelle glissaient deux anneaux de cuivre dorés autrefois.

L'abbé la prit, et en sa place donna le diamant à Caderousse.

«Oh! vous êtes un homme de Dieu, monsieur! s'écria Caderousse, car en vérité personne ne savait qu'Edmond vous avait donné ce diamant et vous auriez pu le garder.

—Bien, se dit tout bas l'abbé, tu l'eusses fait, à ce qu'il paraît, toi.»

L'abbé se leva, prit son chapeau et ses gants.

«Ah çà, dit-il, tout ce que vous m'avez dit est bien vrai, n'est-ce pas, et je puis y croire en tout point?

—Tenez, monsieur l'abbé; dit Caderousse, voici dans le coin de ce mur un christ de bois bénit; voici sur ce bahut le livre d'évangiles de ma femme: ouvrez ce livre, et je vais vous jurer dessus, la main étendue vers le christ, je vais vous jurer sur le salut de mon âme, sur ma foi de chrétien, que je vous ai dit toutes choses comme elles s'étaient passées, et comme l'ange des hommes le dira à l'oreille de Dieu le jour du jugement dernier!

—C'est bien, dit l'abbé, convaincu par cet accent que Caderousse disait la vérité, c'est bien; que cet argent vous profite! Adieu, je retourne loin des hommes qui se font tant de mal les uns aux autres.»

Et l'abbé, se délivrant à grand peine des enthousiastes élans de Caderousse, leva lui-même la barre de la porte, sortit, remonta à cheval, salua une dernière fois l'aubergiste qui se confondait en adieux bruyants, et partit, suivant la même direction qu'il avait déjà suivie pour venir.

Quand Caderousse se retourna, il vit derrière lui la Carconte plus pâle et plus tremblante que jamais.

«Est-ce bien vrai, ce que j'ai entendu? dit-elle.

—Quoi? qu'il nous donnait le diamant pour nous tout seuls? dit Caderousse, presque fou de joie.

—Oui.

—Rien de plus vrai, car le voilà.»

La femme le regarda un instant; puis, d'une voix sourde:

«Et s'il était faux?» dit-elle.

Caderousse pâlit et chancela.

«Faux, murmura-t-il, faux\dots et pourquoi cet homme m'aurait-il donné un diamant faux?

—Pour avoir ton secret sans le payer, imbécile!»

Caderousse resta un instant étourdi sous le poids de cette supposition.

«Oh! dit-il au bout d'un instant, et en prenant son chapeau qu'il posa sur le mouchoir rouge noué autour de sa tête, nous allons bien le savoir.

—Et comment cela?

—C'est la foire à Beaucaire; il y a des bijoutiers de Paris: je vais aller le leur montrer. Toi, garde la maison, femme; dans deux heures je serai de retour.»

Et Caderousse s'élança hors de la maison, et prit tout courant la route opposée à celle que venait de prendre l'inconnu.

«Cinquante mille francs! murmura la Carconte, restée seule, c'est de l'argent\dots mais ce n'est pas une fortune.»



