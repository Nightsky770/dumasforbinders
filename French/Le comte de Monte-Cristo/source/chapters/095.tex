\chapter{Le père et la fille}

\lettrine{N}{ous} avons vu, dans le chapitre précédent, Mme Danglars venir annoncer officiellement à Mme de Villefort le prochain mariage de Mlle Eugénie Danglars avec M. Andrea Cavalcanti. 

\zz
Cette annonce officielle, qui indiquait ou semblait indiquer une résolution prise par tous les intéressés à cette grande affaire, avait cependant été précédée d'une scène dont nous devons compte à nos lecteurs. 

Nous les prions donc de faire un pas en arrière et de se transporter, le matin même de cette journée aux grandes catastrophes, dans ce beau salon si bien doré que nous leur avons fait connaître, et qui faisait l'orgueil de son propriétaire, M. le baron Danglars. 

Dans ce salon, en effet, vers les dix heures du matin, se promenait depuis quelques minutes, tout pensif et visiblement inquiet, le baron lui-même, regardant à chaque porte et s'arrêtant à chaque bruit. 

Lorsque sa somme de patience fut épuisée, il appela le valet de chambre. 

«Étienne, lui dit-il, voyez donc pourquoi Mlle Eugénie m'a prié de l'attendre au salon, et informez-vous pourquoi elle m'y fait attendre si longtemps.» 

Cette bouffée de mauvaise humeur exhalée, le baron reprit un peu de calme. 

En effet, Mlle Danglars, après son réveil, avait fait demander une audience à son père, et avait désigné le salon doré comme le lieu de cette audience. La singularité de cette démarche, son caractère officiel surtout, n'avaient pas médiocrement surpris le banquier, qui avait immédiatement obtempéré au désir de sa fille en se rendant le premier au salon. 

Étienne revint bientôt de son ambassade. 

«La femme de chambre de mademoiselle, dit-il, m'a annoncé que mademoiselle achevait sa toilette et ne tarderait pas à venir.» 

Danglars fit un signe de tête indiquant qu'il était satisfait. Danglars, vis-à-vis du monde et même vis-à-vis de ses gens, affectait le bonhomme et le père faible: c'était une face du rôle qu'il s'était imposé dans la comédie populaire qu'il jouait; c'était une physionomie qu'il avait adoptée et qui lui semblait convenir comme il convenait aux profils droits des masques des pères du théâtre antique d'avoir la lèvre retroussée et riante, tandis que le côté gauche avait la lèvre abaissée et pleurnicheuse. 

Hâtons-nous de dire que, dans l'intimité, la lèvre retroussée et riante descendait au niveau de la lèvre abaissée et pleurnicheuse; de sorte que, pour la plupart du temps, le bonhomme disparaissait pour faire place au mari brutal et au père absolu. 

«Pourquoi diable cette folle qui veut me parler à ce qu'elle prétend, murmurait Danglars, ne vient-elle pas simplement dans mon cabinet; et pourquoi veut-elle me parler?» 

Il roulait pour la vingtième fois cette pensée inquiétante dans son cerveau, lorsque la porte s'ouvrit et qu'Eugénie parut, vêtue d'une robe de satin noir brochée de fleurs mates de la même couleur, coiffée en cheveux, et gantée comme s'il se fût agi d'aller s'asseoir dans son fauteuil du Théâtre-Italien. 

«Eh bien, Eugénie, qu'y a-t-il donc? s'écria le père, et pourquoi le salon solennel, tandis qu'on est si bien dans mon cabinet particulier? 

—Vous avez parfaitement raison, monsieur, répondit Eugénie en faisant signe à son père qu'il pouvait s'asseoir, et vous venez de poser là deux questions qui résument d'avance toute la conversation que nous allons avoir. Je vais donc répondre à toutes deux, et contre les lois de l'habitude, à la seconde d'abord comme étant la moins complexe. J'ai choisi le salon, monsieur, pour lieu de rendez-vous, afin d'éviter les impressions désagréables et les influences du cabinet d'un banquier. Ces livres de caisse, si bien dorés qu'ils soient, ces tiroirs fermés comme des portes de forteresses, ces masses de billets de banque qui viennent on ne sait d'où, et ces quantités de lettres qui viennent d'Angleterre, de Hollande, d'Espagne, des Indes, de la Chine et du Pérou, agissent en général étrangement sur l'esprit d'un père et lui font oublier qu'il est dans le monde un intérêt plus grand et plus sacré que celui de la position sociale et de l'opinion de ses commettants. J'ai donc choisi ce salon où vous voyez, souriants et heureux, dans leurs cadres magnifiques, votre portrait, le mien, celui de ma mère et toutes sortes de paysages pastoraux et de bergeries attendrissantes. Je me fie beaucoup à la puissance des impressions extérieures. Peut-être, vis-à-vis de vous surtout, est-ce une erreur; mais, que voulez-vous? je ne serais pas artiste s'il ne me restait pas quelques illusions. 

—Très bien, répondit M. Danglars, qui avait écouté la tirade avec un imperturbable sang-froid, mais sans en comprendre une parole, absorbé qu'il était, comme tout homme plein d'arrière-pensées, à chercher le fil de sa propre idée dans les idées de l'interlocuteur. 

—Voilà donc le second point éclairci ou à peu près, dit Eugénie sans le moindre trouble et avec cet aplomb tout masculin qui caractérisait son geste et sa parole, et vous me paraissez satisfait de l'explication. Maintenant revenons au premier. Vous me demandiez pourquoi j'avais sollicité cette audience; je vais vous le dire en deux mots; monsieur, le voici: Je ne veux pas épouser M. le comte Andrea Cavalcanti.» 

Danglars fit un bond sur son fauteuil, et, de la secousse, leva à la fois les yeux et les bras au ciel. 

«Mon Dieu, oui, monsieur, continua Eugénie toujours aussi calme. Vous êtes étonné, je le vois bien, car depuis que toute cette petite affaire est en train, je n'ai point manifesté la plus petite opposition, certaine que je suis toujours, le moment venu, d'opposer franchement aux gens qui ne m'ont point consultée et aux choses qui me déplaisent une volonté franche et absolue. Cependant cette fois cette tranquillité, cette passivité, comme disent les philosophes, venait d'une autre source; elle venait de ce que, fille soumise et dévouée\dots (un léger sourire se dessina sur les lèvres empourprées de la jeune fille), je m'essayais à l'obéissance. 

—Eh bien? demanda Danglars. 

—Eh bien, monsieur, reprit Eugénie, j'ai essayé jusqu'au bout de mes forces, et maintenant que le moment est arrivé, malgré tous les efforts que j'ai tentés sur moi-même, je me sens incapable d'obéir. 

—Mais enfin, dit Danglars, qui, esprit secondaire, semblait d'abord tout abasourdi du poids de cette impitoyable logique, dont le flegme accusait tant de préméditation et de force de volonté, la raison de ce refus, Eugénie, la raison? 

—La raison, répliqua la jeune fille, oh! mon Dieu, ce n'est point que l'homme soit plus laid, soit plus sot ou soit plus désagréable qu'un autre, non; M. Andrea Cavalcanti peut même passer, près de ceux qui regardent les hommes au visage et à la taille, pour être d'un assez beau modèle; ce n'est pas non plus parce que mon cœur est moins touché de celui-là que de tout autre: ceci serait une raison de pensionnaire que je regarde comme tout à fait au-dessous de moi, je n'aime absolument personne, monsieur, vous le savez bien, n'est-ce pas? Je ne vois donc pas pourquoi, sans nécessité absolue, j'irais embarrasser ma vie d'un éternel compagnon. Est-ce que le sage n'a point dit quelque part: «Rien de trop»; et ailleurs: «Portez tout avec vous-même»? On m'a même appris ces deux aphorismes en latin et en grec: l'un est, je crois, de Phèdre, et l'autre de Bias. Eh bien, mon cher père, dans le naufrage de la vie, car la vie est un naufrage éternel de nos espérances, je jette à la mer mon bagage inutile, voilà tout, et je reste avec ma volonté, disposée à vivre parfaitement seule et par conséquent parfaitement libre. 

—Malheureuse! malheureuse! murmura Danglars pâlissant, car il connaissait par une longue expérience la solidité de l'obstacle qu'il rencontrait si soudainement. 

—Malheureuse, reprit Eugénie, malheureuse, dites-vous, monsieur? Mais non pas, en vérité, et l'exclamation me paraît tout à fait théâtrale et affectée. Heureuse, au contraire, car je vous le demande, que me manque-t-il? Le monde me trouve belle, c'est quelque chose pour être accueilli favorablement. J'aime les bons accueils, moi: ils épanouissent les visages, et ceux qui m'entourent me paraissent encore moins laids. Je suis douée de quelque esprit et d'une certaine sensibilité relative qui me permet de tirer de l'existence générale, pour la faire entrer dans la mienne, ce que j'y trouve de bon, comme fait le singe lorsqu'il casse la noix verte pour en tirer ce qu'elle contient. Je suis riche, car vous avez une des belles fortunes de France, car je suis votre fille unique, et vous n'êtes point tenace au degré où le sont les pères de la Porte-Saint-Martin et de la Gaîté, qui déshéritent leurs filles parce qu'elles ne veulent pas leur donner de petits-enfants. D'ailleurs, la loi prévoyante vous a ôté le droit de me déshériter, du moins tout à fait, comme elle vous a ôté le pouvoir de me contraindre à épouser monsieur tel ou tel. Ainsi, belle, spirituelle, ornée de quelque talent comme on dit dans les opéras comiques, et riche! mais c'est le bonheur cela, monsieur! Pourquoi donc m'appelez-vous malheureuse? 

Danglars, voyant sa fille souriante et fière jusqu'à l'insolence, ne put réprimer un mouvement de brutalité qui se trahit par un éclat de voix, mais ce fut le seul. Sous le regard interrogateur de sa fille, en face de ce beau sourcil noir, froncé par l'interrogation, il se retourna avec prudence et se calma aussitôt, dompté par la main de fer de la circonspection. 

«En effet, ma fille, répondit-il avec un sourire, vous êtes tout ce que vous vous vantez d'être, hormis une seule chose, ma fille; je ne veux pas trop brusquement vous dire laquelle: j'aime mieux vous la laisser deviner.» 

Eugénie regarda Danglars, fort surprise qu'on lui contestât l'un des fleurons de la couronne d'orgueil qu'elle venait de poser si superbement sur sa tête. 

«Ma fille, continua le banquier, vous m'avez parfaitement expliqué quels étaient les sentiments qui présidaient aux résolutions d'une fille comme vous quand elle a décidé qu'elle ne se mariera point. Maintenant c'est à moi de vous dire quels sont les motifs d'un père comme moi quand il a décidé que sa fille se mariera.» 

Eugénie s'inclina, non pas en fille soumise qui écoute, mais en adversaire prêt à discuter, qui attend. 

«Ma fille, continua Danglars, quand un père demande à sa fille de prendre un époux, il a toujours une raison quelconque pour désirer son mariage. Les uns sont atteints de la manie que vous disiez tout à l'heure, c'est-à-dire de se voir revivre dans leurs petits-fils. Je n'ai pas cette faiblesse, je commence par vous le dire, les joies de la famille me sont à peu près indifférentes, à moi. Je puis avouer cela à une fille que je sais assez philosophe pour comprendre cette indifférence et pour ne pas m'en faire un crime. 

—À la bonne heure, dit Eugénie; parlons franc, monsieur, j'aime cela. 

—Oh! dit Danglars, vous voyez que sans partager, en thèse générale, votre sympathie pour la franchise, je m'y soumets, quand je crois que la circonstance m'y invite. Je continuerai donc. Je vous ai proposé un mari, non pas pour vous, car en vérité je ne pensais pas le moins du monde à vous en ce moment. Vous aimez la franchise, en voilà, j'espère; mais parce que j'avais besoin que vous prissiez cet époux le plus tôt possible, pour certaines combinaisons commerciales que je suis en train d'établir en ce moment. 

Eugénie fit un mouvement. 

«C'est comme j'ai l'honneur de vous le dire, ma fille, et il ne faut pas m'en vouloir, car c'est vous qui m'y forcez; c'est malgré moi, vous le comprenez bien, que j'entre dans ces explications arithmétiques, avec une artiste comme vous, qui craint d'entrer dans le cabinet d'un banquier pour y percevoir des impressions ou des sensations désagréables et antipoétiques. 

«Mais dans ce cabinet de banquier, dans lequel cependant vous avez bien voulu entrer avant-hier pour me demander les mille francs que je vous accorde chaque mois pour vos fantaisies, sachez, ma chère demoiselle, qu'on apprend beaucoup de choses à l'usage même des jeunes personnes qui ne veulent pas se marier. On y apprend, par exemple, et par égard pour votre susceptibilité nerveuse je vous l'apprendrai dans ce salon, on y apprend que le crédit d'un banquier est sa vie physique et morale, que le crédit soutient l'homme comme le souffle anime le corps, et M. de Monte-Cristo m'a fait un jour là-dessus un discours que je n'ai jamais oublié. On y apprend qu'à mesure que le crédit se retire le corps devient cadavre et que cela doit arriver dans fort peu de temps au banquier qui s'honore d'être le père d'une fille si bonne logicienne.» 

Mais Eugénie, au lieu de se courber, se redressa sous le coup. 

«Ruiné! dit-elle. 

—Vous avez trouvé l'expression juste, ma fille, la bonne expression, dit Danglars en fouillant sa poitrine avec ses ongles, tout en conservant sur sa rude figure le sourire de l'homme sans cœur, mais non sans esprit, ruiné! c'est cela. 

—Ah! fit Eugénie. 

—Oui, ruiné! Eh bien, le voilà donc connu, ce secret plein d'horreur, comme dit le poète tragique. 

«Maintenant, ma fille, apprenez de ma bouche comment ce malheur peut, par vous, devenir moindre; je ne dirai pas pour moi, mais pour vous. 

—Oh! s'écria Eugénie, vous êtes mauvais physionomiste, monsieur, si vous vous figurez que c'est pour moi que je déplore la catastrophe que vous m'exposez. 

«Moi ruinée! et que m'importe? Ne me reste-t-il pas mon talent? Ne puis-je pas, comme la Pasta, comme la Malibran, comme la Grisi, me faire ce que vous ne m'eussiez jamais donné, quelle que fût votre fortune, cent ou cent cinquante mille livres de rente que je ne devrai qu'à moi seule, et qui, au lieu de m'arriver comme m'arrivaient ces pauvres douze mille francs que vous me donniez avec des regards rechignés et des paroles de reproche sur ma prodigalité, me viendront accompagnés d'acclamations, de bravos et de fleurs? Et quand je n'aurais pas ce talent dont votre sourire me prouve que vous doutez, ne me resterait-il pas encore ce furieux amour de l'indépendance, qui me tiendra toujours lieu de tous les trésors, et qui domine en moi jusqu'à l'instinct de la conservation? 

«Non, ce n'est pas pour moi que je m'attriste, je saurai toujours bien me tirer d'affaire, moi; mes livres, mes crayons, mon piano, toutes choses qui ne coûtent pas cher et que je pourrai toujours me procurer, me resteront toujours. Vous pensez peut-être que je m'afflige pour Mme Danglars, détrompez-vous encore: ou je me trompe grossièrement, ou ma mère a pris toutes ses précautions contre la catastrophe qui vous menace et qui passera sans l'atteindre; elle s'est mise à l'abri, je l'espère, et ce n'est pas en veillant sur moi qu'elle a pu se distraire de ses préoccupations de fortune, car, Dieu merci, elle m'a laissé toute mon indépendance sous le prétexte que j'aimais ma liberté. 

«Oh! non, monsieur, depuis mon enfance, j'ai vu se passer trop de choses autour de moi; je les ai toutes trop bien comprises, pour que le malheur fasse sur moi plus d'impression qu'il ne mérite de le faire; depuis que je me connais, je n'ai été aimée de personne; tant pis! cela m'a conduite tout naturellement à n'aimer personne; tant mieux! Maintenant vous avez ma profession de foi. 

—Alors, dit Danglars, pâle d'un courroux qui ne prenait point sa source dans l'amour paternel offensé; alors, mademoiselle, vous persistez à vouloir consommer ma ruine? 

—Votre ruine! Moi, dit Eugénie, consommer votre ruine! que voulez-vous dire? je ne comprends pas. 

—Tant mieux, cela me laisse un rayon d'espoir; écoutez. 

—J'écoute, dit Eugénie en regardant si fixement son père, qu'il fallut à celui-ci un effort pour qu'il ne baissât point les yeux sous le regard puissant de la jeune fille. 

—M. Cavalcanti, continua Danglars, vous épouse et, en vous épousant, vous apporte trois millions de dot qu'il place chez moi. 

—Ah! fort bien, fit avec un souverain mépris Eugénie, tout en lissant ses gants l'un sur l'autre. 

—Vous pensez que je vous ferai tort de ces trois millions? dit Danglars; pas du tout, ces trois millions sont destinés à en produire au moins dix. J'ai obtenu avec un banquier, mon confrère, la concession d'un chemin de fer, seule industrie qui de nos jours présente ces chances fabuleuses de succès immédiat qu'autrefois Law appliqua pour les bons Parisiens, ces éternels badauds de la spéculation, à un Mississippi fantastique. Par mon calcul on doit posséder un millionième de rail comme on possédait autrefois un arpent de terre en friche sur les bords de l'Ohio. C'est un placement hypothécaire, ce qui est un progrès, comme vous voyez, puisqu'on aura au moins dix, quinze, vingt, cent livres de fer en échange de son argent. Eh bien, je dois d'ici à huit jours déposer pour mon compte quatre millions! Ces quatre millions, je vous le dis, en produiront dix ou douze. 

—Mais pendant cette visite que je vous ai faite avant-hier, monsieur, et dont vous voulez bien vous souvenir, reprit Eugénie, je vous ai vu encaisser, c'est le terme, n'est-ce pas? cinq millions et demi; vous m'avez même montré la chose en deux bons sur le trésor, et vous vous étonniez qu'un papier ayant une si grande valeur n'éblouît pas mes yeux comme ferait un éclair. 

—Oui, mais ces cinq millions et demi ne sont point à moi et sont seulement une preuve de la confiance que l'on a en moi; mon titre de banquier populaire m'a valu la confiance des hôpitaux, et les cinq millions et demi sont aux hôpitaux; dans tout autre temps je n'hésiterais pas à m'en servir, mais aujourd'hui l'on sait les grandes pertes que j'ai faites, et, comme je vous l'ai dit, le crédit commence à se retirer de moi. D'un moment à l'autre, l'administration peut réclamer le dépôt, et si je l'ai employé à autre chose, je suis forcé de faire une banqueroute honteuse. Je ne méprise pas les banqueroutes, croyez-le bien, mais les banqueroutes qui enrichissent et non celles qui ruinent. Ou que vous épousiez M. Cavalcanti, que je touche les trois millions de la dot, ou même que l'on croie que je vais les toucher, mon crédit se raffermit, et ma fortune, qui depuis un mois ou deux s'est engouffrée dans des abîmes creusés sous mes pas par une fatalité inconcevable, se rétablit. Me comprenez-vous? 

—Parfaitement; vous me mettez en gage pour trois millions, n'est-ce pas? 

—Plus la somme est forte, plus elle est flatteuse; elle vous donne une idée de votre valeur. 

—Merci. Un dernier mot, monsieur: me promettez-vous de vous servir tant que vous le voudrez du chiffre de cette dot que doit apporter M. Cavalcanti, mais de ne pas toucher à la somme? Ceci n'est point une affaire d'égoïsme, c'est une affaire de délicatesse. Je veux bien servir à réédifier votre fortune, mais je ne veux pas être votre complice dans la ruine des autres. 

—Mais puisque je vous dis, s'écria Danglars, qu'avec ces trois millions\dots 

—Croyez-vous vous tirer d'affaire, monsieur, sans avoir besoin de toucher à ces trois millions? 

—Je l'espère, mais à condition toujours que le mariage, en se faisant, consolidera mon crédit. 

—Pourrez-vous payer à M. Cavalcanti les cinq cent mille francs que vous me donnez pour mon contrat? 

—En revenant de la mairie, il les touchera. 

—Bien! 

—Comment, bien? Que voulez-vous dire? 

—Je veux dire qu'en me demandant ma signature n'est-ce pas, vous me laissez absolument libre de ma personne? 

—Absolument. 

—Alors, \textit{bien}; comme je vous disais, monsieur, je suis prête à épouser M. Cavalcanti. 

—Mais quels sont vos projets? 

—Ah! c'est mon secret. Où serait ma supériorité sur vous si, ayant le vôtre, je vous livrais le mien!» 

Danglars se mordit les lèvres. 

«Ainsi, dit-il, vous êtes prête à faire les quelques visites officielles qui sont absolument indispensables. 

—Oui, répondit Eugénie. 

—Et à signer le contrat dans trois jours? 

—Oui. 

—Alors, à mon tour, c'est moi qui vous dis: Bien!» 

Et Danglars prit la main de sa fille et la serra entre les siennes. Mais, chose extraordinaire, pendant ce serrement de main, le père n'osa pas dire: «Merci, mon enfant»; la fille n'eut pas un sourire pour son père. 

«La conférence est finie?» demanda Eugénie en se levant. 

Danglars fit signe de la tête qu'il n'avait plus rien à dire. 

Cinq minutes après, le piano retentissait sous les doigts de Mlle d'Armilly, et Mlle Danglars chantait la malédiction de Brabantio sur \textit{Desdemona}. 

À la fin du morceau, Étienne entra et annonça à Eugénie que les chevaux étaient à la voiture et que la baronne l'attendait pour faire ses visites. 

Nous avons vu les deux femmes passer chez Villefort, d'où elles sortirent pour continuer leurs courses. 