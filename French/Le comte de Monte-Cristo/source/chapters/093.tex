\chapter{Valentine}

\lettrine{O}{n} devine où Morrel avait affaire et chez qui était son rendez-vous. 

\zz
Aussi Morrel, en quittant Monte-Cristo, s'achemina-t-il lentement vers la maison de Villefort. 

\zz
Nous disons lentement: c'est que Morrel avait plus d'une demi-heure à lui pour faire cinq cents pas; mais, malgré ce temps plus que suffisant, il s'était empressé de quitter Monte-Cristo, ayant hâte d'être seul avec ses pensées. 

Il savait bien son heure, l'heure à laquelle Valentine, assistant au déjeuner de Noirtier, était sûre de ne pas être troublée dans ce pieux devoir. Noirtier et Valentine lui avaient accordé deux visites par semaine, et il venait profiter de son droit. 

Il arriva, Valentine l'attendait. Inquiète, presque égarée, elle lui saisit la main, et l'amena devant son grand-père. 

Cette inquiétude, poussée, comme nous le disons, presque jusqu'à l'égarement, venait du bruit que l'aventure de Morcerf avait fait dans le monde, on savait (le monde sait toujours) l'aventure de l'Opéra. Chez Villefort, personne ne doutait qu'un duel ne fût la conséquence forcée de cette aventure; Valentine, avec son instinct de femme, avait deviné que Morrel serait le témoin de Monte-Cristo, et avec le courage bien connu du jeune homme, avec cette amitié profonde qu'elle lui connaissait pour le comte, elle craignait qu'il n'eût point la force de se borner au rôle passif qui lui était assigné. 

On comprend donc avec quelle avidité les détails furent demandés, donnés et reçus, et Morrel put lire une indicible joie dans les yeux de sa bien-aimée quand elle sut que cette terrible affaire avait eu une issue non moins heureuse qu'inattendue. 

«Maintenant, dit Valentine en faisant signe à Morrel de s'asseoir à côté du vieillard et en s'asseyant elle-même sur le tabouret où reposaient ses pieds, maintenant, parlons un peu de nos affaires. Vous savez, Maximilien, que bon papa avait eu un instant l'idée de quitter la maison et de prendre un appartement hors de l'hôtel de M. de Villefort? 

—Oui, certes, dit Maximilien, je me rappelle ce projet, et j'y avais même fort applaudi. 

—Eh bien, dit Valentine, applaudissez encore, Maximilien, car bon papa y revient. 

—Bravo! dit Maximilien. 

—Et savez-vous, dit Valentine, quelle raison donne bon papa pour quitter la maison?» 

Noirtier regardait sa fille pour lui imposer silence de l'œil; mais Valentine ne regardait point Noirtier; ses yeux, son regard, son sourire, tout était pour Morrel. 

«Oh! quelle que soit la raison que donne M. Noirtier, s'écria Morrel, je déclare qu'elle est bonne. 

—Excellente, dit Valentine: il prétend que l'air du faubourg Saint-Honoré ne vaut rien pour moi. 

—En effet, dit Morrel; écoutez, Valentine, M. Noirtier pourrait bien avoir raison; depuis quinze jours, je trouve que votre santé s'altère. 

—Oui, un peu, c'est vrai, répondit Valentine; aussi bon papa s'est constitué mon médecin, et comme bon papa sait tout, j'ai la plus grande confiance en lui. 

—Mais enfin il est donc vrai que vous souffrez, Valentine? demanda vivement Morrel. 

—Oh! mon Dieu! cela ne s'appelle pas souffrir: je ressens un malaise général, voilà tout; j'ai perdu l'appétit, et il me semble que mon estomac soutient une lutte pour s'habituer à quelque chose.» 

Noirtier ne perdait pas une des paroles de Valentine. 

«Et quel est le traitement que vous suivez pour cette maladie inconnue? 

—Oh! bien simple, dit Valentine; j'avale tous les matins une cuillerée de la potion qu'on apporte pour mon grand-père; quand je dis une cuillerée, j'ai commencé par une, et maintenant j'en suis à quatre. Mon grand-père prétend que c'est une panacée.» 

Valentine souriait; mais il y avait quelque chose de triste et de souffrant dans son sourire. 

Maximilien, ivre d'amour, la regardait en silence; elle était bien belle, mais sa pâleur avait pris un ton plus mat, ses yeux brillaient d'un feu plus ardent que d'habitude, et ses mains, ordinairement d'un blanc de nacre, semblaient des mains de cire qu'une nuance jaunâtre envahit avec le temps. 

De Valentine, le jeune homme porta les yeux sur Noirtier; celui-ci considérait avec cette étrange et profonde intelligence la jeune fille absorbée dans son amour; mais lui aussi, comme Morrel, suivait ces traces d'une sourde souffrance, si peu visible d'ailleurs qu'elle avait échappé à l'œil de tous, excepté celui du père et de l'amant. 

«Mais, dit Morrel, cette potion dont vous êtes arrivée jusqu'à quatre cuillerées, je la voyais médicamentée pour M. Noirtier? 

—Je sais que c'est fort amer, dit Valentine, si amer que tout ce que je bois après cela me semble avoir le même goût.» 

Noirtier regarda sa fille d'un ton interrogateur. 

«Oui, bon papa, dit Valentine, c'est comme cela. Tout à l'heure, avant de descendre chez vous, j'ai bu un verre d'eau sucrée; eh bien, j'en ai laissé la moitié tant cette eau m'a paru amère.» 

Noirtier pâlit, et fit signe qu'il voulait parler. 

Valentine se leva pour aller chercher le dictionnaire. 

Noirtier la suivait des yeux avec une angoisse visible. 

En effet, le sang montait à la tête de la jeune fille, ses joues se colorèrent. 

«Tiens! s'écria-t-elle sans rien perdre de sa gaieté, c'est singulier: un éblouissement! Est-ce donc le soleil qui m'a frappé dans les yeux?\dots» 

Et elle s'appuya à l'espagnolette de la fenêtre. 

«Il n'y a pas de soleil», dit Morrel encore plus inquiet de l'expression du visage de Noirtier que de l'indisposition de Valentine. 

Et il courut à Valentine. 

La jeune fille sourit. 

«Rassure-toi, bon père, dit-elle à Noirtier: rassurez-vous, Maximilien, ce n'est rien, et la chose est déjà passée: mais, écoutez donc! n'est-ce pas le bruit d'une voiture que j'entends dans la cour?» 

Elle ouvrit la porte de Noirtier, courut à une fenêtre du corridor, et revint précipitamment. 

«Oui, dit-elle, c'est Mme Danglars et sa fille qui viennent nous faire une visite. Adieu, je me sauve, car on me viendrait chercher ici; ou plutôt, au revoir, restez près de bon papa, monsieur Maximilien, je vous promets de ne pas les retenir.» 

Morrel la suivit des yeux, la vit refermer la porte, et l'entendit monter le petit escalier qui conduisait à la fois chez Mme de Villefort et chez elle. 

Dès qu'elle eut disparu, Noirtier fit signe à Morrel de prendre le dictionnaire. Morrel obéit; il s'était, guidé par Valentine, promptement habitué à comprendre le vieillard. 

Cependant, quelque habitude qu'il eût, et comme il fallait passer en revue une partie des vingt-quatre lettres de l'alphabet, et trouver chaque mot dans le dictionnaire, ce ne fut qu'au bout de dix minutes que la pensée du vieillard fut traduite par ces paroles: 

«Cherchez le verre d'eau et la carafe qui sont dans la chambre de Valentine.» 

Morrel sonna aussitôt le domestique qui avait remplacé Barrois, et au nom de Noirtier lui donna cet ordre. 

Le domestique revint un instant après. 

La carafe et le verre étaient entièrement vides. 

Noirtier fit signe qu'il voulait parler. 

«Pourquoi le verre et la carafe sont-ils vides? demanda-t-il. Valentine a dit qu'elle n'avait bu que la moitié du verre.» 

La traduction de cette nouvelle demande prit encore cinq minutes. 

«Je ne sais, dit le domestique; mais la femme de chambre est dans l'appartement de Mlle Valentine: c'est peut-être elle qui les a vidés. 

—Demandez-le-lui», dit Morrel, traduisant cette fois la pensée de Noirtier par le regard. 

Le domestique sortit, et presque aussitôt rentra. 

«Mlle Valentine a passé par sa chambre pour se rendre dans celle de Mme de Villefort, dit-il; et, en passant, comme elle avait soif, elle a bu ce qui restait dans le verre; quant à la carafe, M. Édouard l'a vidée pour faire un étang à ses canards.» 

Noirtier leva les yeux au ciel comme fait un joueur qui joue sur un coup tout ce qu'il possède. 

Dès lors, les yeux du vieillard se fixèrent sur la porte et ne quittèrent plus cette direction. 

C'était, en effet, Mme Danglars et sa fille que Valentine avait vues; on les avait conduites à la chambre de Mme de Villefort, qui avait dit qu'elle recevrait chez elle; voilà pourquoi Valentine avait passé par son appartement: sa chambre étant de plain-pied avec celle de sa belle-mère, et les deux chambres n'étant séparées que par celle d'Édouard. 

Les deux femmes entrèrent au salon avec cette espèce de raideur officielle qui fait présager une communication. 

Entre gens du même monde, une nuance est bientôt saisie. Mme de Villefort répondit à cette solennité par de la solennité. 

En ce moment, Valentine entra, et les révérences recommencèrent. 

«Chère amie, dit la baronne, tandis que les deux jeunes filles se prenaient les mains, je venais avec Eugénie vous annoncer la première le très prochain mariage de ma fille avec le prince Cavalcanti.» 

Danglars avait maintenu le titre de prince. Le banquier populaire avait trouvé que cela faisait mieux que comte. 

«Alors, permettez que je vous fasse mes sincères compliments, répondit Mme de Villefort. M. le prince Cavalcanti paraît un jeune homme plein de rares qualités. 

—Écoutez, dit la baronne en souriant; si nous parlons comme deux amies, je dois vous dire que le prince ne nous paraît pas encore être ce qu'il sera. Il a en lui un peu de cette étrangeté qui nous fait, à nous autres Français, reconnaître du premier coup d'œil un gentilhomme italien ou allemand. Cependant il annonce un fort bon cœur, beaucoup de finesse d'esprit, et quant aux convenances, M. Danglars prétend que la fortune est majestueuse; c'est son mot. 

—Et puis, dit Eugénie en feuilletant l'album de Mme de Villefort, ajoutez, madame, que vous avez une inclination toute particulière pour ce jeune homme. 

—Et, dit Mme de Villefort, je n'ai pas besoin de vous demander si vous partagez cette inclination? 

—Moi! répondit Eugénie avec son aplomb ordinaire, oh! pas le moins du monde, madame; ma vocation, à moi, n'était pas de m'enchaîner aux soins d'un ménage ou aux caprices d'un homme, quel qu'il fût. Ma vocation était d'être artiste et libre par conséquent de mon cœur, de ma personne et de ma pensée.» 

Eugénie prononça ces paroles avec un accent si vibrant et si ferme, que le rouge en monta au visage de Valentine. La craintive jeune fille ne pouvait comprendre cette nature vigoureuse qui semblait n'avoir aucune des timidités de la femme. 

«Au reste, continua-t-elle, puisque je suis destinée à être mariée, bon gré, mal gré, je dois remercier la Providence qui m'a du moins procuré les dédains de M. Albert de Morcerf; sans cette Providence, je serais aujourd'hui la femme d'un homme perdu d'honneur. 

—C'est pourtant vrai, dit la baronne avec cette étrange naïveté que l'on trouve quelquefois chez les grandes dames, et que les fréquentations roturières ne peuvent leur faire perdre tout à fait, c'est pourtant vrai, sans cette hésitation des Morcerf, ma fille épousait ce M. Albert: le général y tenait beaucoup, il était même venu pour forcer la main à M. Danglars; nous l'avons échappé belle. 

—Mais, dit timidement Valentine, est-ce que toute cette honte du père rejaillit sur le fils? M. Albert me semble bien innocent de toutes ces trahisons du général. 

—Pardon, chère amie, dit l'implacable jeune fille; M. Albert en réclame et en mérite sa part: il paraît qu'après avoir provoqué hier M. de Monte-Cristo à l'Opéra, il lui a fait aujourd'hui des excuses sur le terrain. 

—Impossible! dit Mme de Villefort. 

—Ah! chère amie, dit Mme Danglars avec cette même naïveté que nous avons déjà signalée, la chose est certaine; je le sais de M. Debray, qui était présent à l'explication.» 

Valentine aussi savait la vérité, mais elle ne répondait pas. Repoussée par un mot dans ses souvenirs, elle se retrouvait en pensée dans la chambre de Noirtier, où l'attendait Morrel. 

Plongée dans cette espèce de contemplation intérieure, Valentine avait depuis un instant cessé de prendre part à la conversation; il lui eût même été impossible de répéter ce qui avait été dit depuis quelques minutes, quand tout à coup la main de Mme Danglars, en s'appuyant sur son bras, la tira de sa rêverie. 

«Qu'y a-t-il, madame? dit Valentine en tressaillant au contact des doigts de Mme Danglars, comme elle eût tressailli à un contact électrique. 

—Il y a, ma chère Valentine, dit la baronne, que vous souffrez sans doute? 

—Moi? fit la jeune fille en passant sa main sur son front brûlant. 

—Oui; regardez-vous dans cette glace; vous avez rougi et pâli successivement trois ou quatre fois dans l'espace d'une minute. 

—En effet, s'écria Eugénie, tu es bien pâle! 

—Oh! ne t'inquiète pas, Eugénie; je suis comme cela depuis quelques jours.» 

Et si peu rusée qu'elle fût, la jeune fille comprit que c'était une occasion de sortir. D'ailleurs, Mme de Villefort vint à son aide. 

«Retirez-vous, Valentine, dit-elle; vous souffrez réellement et ces dames voudront bien vous pardonner; buvez un verre d'eau pure et cela vous remettra.» 

Valentine embrassa Eugénie, salua Mme Danglars déjà levée pour se retirer, et sortit. 

«Cette pauvre enfant, dit Mme de Villefort quand Valentine eut disparu, elle m'inquiète sérieusement, et je ne serais pas étonnée quand il lui arriverait quelque accident grave.» 

Cependant Valentine, dans une espèce d'exaltation dont elle ne se rendait pas compte, avait traversé la chambre d'Édouard sans répondre à je ne sais quelle méchanceté de l'enfant, et par chez elle avait atteint le petit escalier. Elle en avait franchi tous les degrés moins les trois derniers; elle entendait déjà la voix de Morrel, lorsque tout à coup un nuage passa devant ses yeux, son pied raidi manqua la marche, ses mains n'eurent plus de force pour la retenir à la rampe, et froissant la cloison, elle roula du haut des trois derniers degrés plutôt qu'elle ne les descendit. 

Morrel ne fit qu'un bond; il ouvrit la porte, et trouva Valentine étendue sur le palier. 

Rapide comme l'éclair, il l'enleva entre ses bras et l'assit dans un fauteuil. Valentine rouvrit les yeux. 

«Oh! maladroite que je suis, dit-elle avec une fiévreuse volubilité; je ne sais donc plus me tenir? j'oublie qu'il y a trois marches avant le palier! 

—Vous vous êtes blessée peut-être, Valentine? s'écria Morrel. Oh! mon Dieu! mon Dieu!» 

Valentine regarda autour d'elle: elle vit le plus profond effroi peint dans les yeux de Noirtier. 

«Rassure-toi, bon père, dit-elle en essayant de sourire; ce n'est rien, ce n'est rien\dots la tête m'a tourné, voilà tout. 

—Encore un étourdissement! dit Morrel joignant les mains. Oh! faites-y attention, Valentine, je vous supplie. 

—Mais non, dit Valentine, mais non, je vous dis que tout est passé et que ce n'était rien. Maintenant, laissez-moi vous apprendre une nouvelle: dans huit jours, Eugénie se marie, et dans trois jours il y a une espèce de grand festin, un repas de fiançailles. Nous sommes tous invités, mon père, Mme de Villefort et moi\dots à ce que j'ai cru comprendre, du moins. 

—Quand sera-ce donc notre tour de nous occuper de ces détails? Oh! Valentine, vous qui pouvez tant de choses sur notre bon papa, tâchez qu'il vous réponde: \textit{bientôt}! 

—Ainsi, demanda Valentine, vous comptez sur moi pour stimuler la lenteur et réveiller la mémoire de bon papa? 

—Oui, s'écria Morrel. Mon Dieu! mon Dieu! faites vite. Tant que vous ne serez pas à moi, Valentine, il me semblera toujours que vous allez m'échapper. 

—Oh! répondit Valentine avec un mouvement convulsif, oh! en vérité, Maximilien, vous êtes trop craintif, pour un officier, pour un soldat qui, dit-on, n'a jamais connu la peur. Ha! ha! ha!» 

Et elle éclata d'un rire strident et douloureux; ses bras se raidirent et se tournèrent, sa tête se renversa sur son fauteuil et elle demeura sans mouvement. 

Le cri de terreur que Dieu enchaînait aux lèvres de Noirtier jaillit de son regard. 

Morrel comprit; il s'agissait d'appeler du secours. 

Le jeune homme se pendit à la sonnette; la femme de chambre qui était dans l'appartement de Valentine et le domestique qui avait remplacé Barrois accoururent simultanément. 

Valentine était si pâle, si froide, si inanimée, que, sans écouter ce qu'on leur disait, la peur qui veillait sans cesse dans cette maison maudite les prit, et qu'ils s'élancèrent par les corridors en criant au secours. 

Mme Danglars et Eugénie sortaient en ce moment même; elles purent encore apprendre la cause de toute cette rumeur. 

«Je vous l'avais bien dit! s'écria Mme de Villefort. Pauvre petite.» 