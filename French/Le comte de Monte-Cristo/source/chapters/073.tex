\chapter{La promesse} 

\lettrine{C}{'était} en effet Morrel, qui depuis la veille ne vivait plus. Avec cet instinct particulier aux amants et aux mères, il avait deviné qu'il allait, à la suite de ce retour de Mme de Saint-Méran et de la mort du marquis, se passer quelque chose chez Villefort qui intéresserait son amour pour Valentine. 

Comme on va le voir, ses pressentiments s'étaient réalisés, et ce n'était plus une simple inquiétude qui le conduisait si effaré et si tremblant à la grille des marronniers. 

Mais Valentine n'était pas prévenue de l'attente de Morrel, ce n'était pas l'heure où il venait ordinairement, et ce fut un pur hasard ou, si l'on aime mieux une heureuse sympathie qui la conduisit au jardin. Quand elle parut, Morrel l'appela; elle courut à la grille. 

«Vous, à cette heure! dit-elle. 

—Oui, pauvre amie, répondit Morrel, je viens chercher et apporter de mauvaises nouvelles. 

—C'est donc la maison du malheur, dit Valentine. Parlez, Maximilien. Mais, en vérité, la somme de douleurs est déjà bien suffisante.  

—Chère Valentine, dit Morrel, essayant de se remettre de sa propre émotion pour parler convenablement, écoutez-moi bien, je vous prie; car tout ce que je vais vous dire est solennel. À quelle époque compte-t-on vous marier? 

—Écoutez, dit à son tour Valentine, je ne veux rien vous cacher, Maximilien. Ce matin on a parlé de mon mariage, et ma grand-mère, sur laquelle j'avais compté comme sur un appui qui ne manquerait pas, non seulement s'est déclarée pour ce mariage, mais encore le désire à tel point que le retour seul de M. d'Épinay le retarde et que le lendemain de son arrivée le contrat sera signé.» 

Un pénible soupir ouvrit la poitrine du jeune homme, et il regarda longuement et tristement la jeune fille. 

«Hélas! reprit-il à voix basse, il est affreux d'entendre dire tranquillement par la femme qu'on aime: «Le moment de votre supplice est fixé: c'est dans quelques heures qu'il aura lieu; mais n'importe, il faut que cela soit ainsi, et de ma part, je n'y apporterai aucune opposition.» Eh bien, puisque, dites-vous, on n'attend plus que M. d'Épinay pour signer le contrat, puisque vous serez à lui le lendemain de son arrivée, c'est demain que vous serez engagée à M. d'Épinay, car il est arrivé à Paris ce matin.» 

Valentine poussa un cri. 

«J'étais chez le comte de Monte-Cristo il y a une heure, dit Morrel; nous causions, lui de la douleur de votre maison et moi de votre douleur, quand tout à coup une voiture roule dans la cour. Écoutez. Jusque-là je ne croyais pas aux pressentiments, Valentine; mais maintenant il faut bien que j'y croie. Au bruit de cette voiture, un frisson m'a pris; bientôt j'ai entendu des pas sur l'escalier. Les pas retentissants du commandeur n'ont pas plus épouvanté don Juan que ces pas ne m'ont épouvanté. Enfin la porte s'ouvre; Albert de Morcerf entre le premier, et j'allais douter de moi-même, j'allais croire que je m'étais trompé, quand derrière lui s'avance un autre jeune homme et que le comte s'est écrié: «Ah! M. le baron Franz d'Épinay!» Tout ce que j'ai de force et de courage dans le cœur, je l'ai appelé pour me contenir. Peut-être ai-je pâli, peut-être ai-je tremblé: mais à coup sûr je suis resté le sourire sur les lèvres. Mais cinq minutes après, je suis sorti sans avoir entendu un mot de ce qui s'est dit pendant ces cinq minutes; j'étais anéanti. 

—Pauvre Maximilien! murmura Valentine. 

—Me voilà, Valentine. Voyons, maintenant répondez-moi comme à un homme à qui votre réponse va donner la mort ou la vie. Que comptez-vous faire?» 

Valentine baissa la tête; elle était accablée. 

«Écoutez, dit Morrel, ce n'est pas la première fois que vous pensez à la situation où nous sommes arrivés: elle est grave, elle est pesante, suprême. Je ne pense pas que ce soit le moment de s'abandonner à une douleur stérile: cela est bon pour ceux qui veulent souffrir à l'aise et boire leurs larmes à loisir. Il y a des gens comme cela, et Dieu sans doute leur tiendra compte au ciel de leur résignation sur la terre; mais quiconque se sent la volonté de lutter ne perd pas un temps précieux et rend immédiatement à la fortune le coup qu'il en a reçu. Est-ce votre volonté de lutter contre la mauvaise fortune, Valentine? Dites, car c'est cela que je viens vous demander.»  

Valentine tressaillit et regarda Morrel avec de grands yeux effarés. Cette idée de résister à son père, à sa grand-mère, à toute sa famille enfin, ne lui était pas même venue. 

«Que me dites-vous, Maximilien? demanda Valentine, et qu'appelez-vous une lutte? Oh! dites un sacrilège. Quoi! moi, je lutterais contre l'ordre de mon père, contre le vœu de mon aïeule mourante! C'est impossible!» 

Morrel fit un mouvement. 

«Vous êtes un trop noble cœur pour ne pas me comprendre, et vous me comprenez si bien, cher Maximilien, que je vous vois réduit au silence. Lutter, moi! Dieu m'en préserve! Non, non; je garde toute ma force pour lutter contre moi-même et pour boire mes larmes, comme vous dites. Quant à affliger mon père, quant à troubler les derniers moments de mon aïeule, jamais! 

—Vous avez bien raison, dit flegmatiquement Morrel. 

—Comme vous me dites cela, mon Dieu! s'écria Valentine blessée. 

—Je vous dis cela comme un homme qui vous admire, mademoiselle, reprit Maximilien. 

—Mademoiselle! s'écria Valentine, mademoiselle! Oh! l'égoïste! il me voit au désespoir et feint de ne pas me comprendre. 

—Vous vous trompez, et je vous comprends parfaitement au contraire. Vous ne voulez pas contrarier M. de Villefort, vous ne voulez pas désobéir à la marquise, et demain vous signerez le contrat qui doit vous lier à votre mari.  

—Mais, mon Dieu! Puis-je donc faire autrement? 

—Il ne faut pas en appeler à moi, mademoiselle, car je suis un mauvais juge dans cette cause, et mon égoïsme m'aveuglera, répondit Morrel, dont la voix sourde et les poings fermés annonçaient l'exaspération croissante. 

—Que m'eussiez-vous donc proposé, Morrel, si vous m'aviez trouvée disposée à accepter votre proposition? Voyons, répondez. Il ne s'agit pas de dire vous faites mal, il faut donner un conseil. 

—Est-ce sérieusement que vous me dites cela, Valentine, et dois-je le donner, ce conseil? dites. 

—Certainement, cher Maximilien, car s'il est bon, je le suivrai; vous savez bien que je suis dévouée à vos affections. 

—Valentine, dit Morrel en achevant d'écarter une planche déjà disjointe, donnez-moi votre main en preuve que vous me pardonnez ma colère; c'est que j'ai la tête bouleversée, voyez-vous, et que depuis une heure les idées les plus insensées ont tour à tour traversé mon esprit. Oh! dans le cas où vous refuseriez mon conseil!\dots 

—Eh bien, ce conseil? 

—Le voici, Valentine.» 

La jeune fille leva les yeux au ciel et poussa un soupir. 

«Je suis libre, reprit Maximilien, je suis assez riche pour nous deux; je vous jure que vous serez ma femme avant que mes lèvres se soient posées sur votre front. 

—Vous me faites trembler, dit la jeune fille. 

—Suivez-moi, continua Morrel; je vous conduis chez ma sœur, qui est digne d'être votre sœur; nous nous embarquerons pour Alger, pour l'Angleterre ou pour l'Amérique, si vous n'aimez pas mieux nous retirer ensemble dans quelque province, où nous attendrons, pour revenir à Paris, que nos amis aient vaincu la résistance de votre famille.» 

Valentine secoua la tête. 

«Je m'y attendais, Maximilien, dit-elle: c'est un conseil d'insensé, et je serais encore plus insensée que vous si je ne vous arrêtais pas à l'instant avec ce seul mot: impossible, Morrel, impossible. 

—Vous suivrez donc votre fortune, telle que le sort vous le fera, et sans même essayer de la combattre? dit Morrel rembruni. 

—Oui, dussé-je en mourir! 

—Eh bien, Valentine, reprit Maximilien, je vous répéterai encore que vous avez raison. En effet, c'est moi qui suis un fou, et vous me prouvez que la passion aveugle les esprits les plus justes. Merci donc, à vous qui raisonnez sans passion. Soit donc, c'est une chose entendue; demain vous serez irrévocablement promise à M. Franz d'Épinay, non point par cette formalité de théâtre inventée pour dénouer les pièces de comédie, et qu'on appelle la signature du contrat, mais par votre propre volonté. 

—Encore une fois, vous me désespérez, Maximilien! dit Valentine; encore une fois, vous retournez le poignard dans la plaie! Que feriez-vous, si votre sœur écoutait un conseil comme celui que vous me donnez? 

—Mademoiselle, reprit Morrel avec un sourire amer, je suis un égoïste, vous l'avez dit, et dans ma qualité d'égoïste, je ne pense pas à ce que feraient les autres dans ma position, mais à ce que je compte faire, moi. Je pense que je vous connais depuis un an, que j'ai mis, du jour où je vous ai connue, toutes mes chances de bonheur sur votre amour, qu'un jour est venu où vous m'avez dit que vous m'aimiez; que de ce jour j'ai mis toutes mes chances d'avenir sur votre possession: c'était ma vie. Je ne pense plus rien maintenant; je me dis seulement que les chances ont tourné, que j'avais cru gagner le ciel et que je l'ai perdu. Cela arrive tous les jours qu'un joueur perd non seulement ce qu'il a, mais encore ce qu'il n'a pas.» 

Morrel prononça ces mots avec un calme parfait; Valentine le regarda un instant de ses grands yeux scrutateurs, essayant de ne pas laisser pénétrer ceux de Morrel jusqu'au trouble qui tourbillonnait déjà au fond de son cœur. 

«Mais enfin, qu'allez-vous faire? demanda Valentine. 

—Je vais avoir l'honneur de vous dire adieu, mademoiselle, en attestant Dieu, qui entend mes paroles et qui lit au fond de mon cœur, que je vous souhaite une vie assez calme, assez heureuse et assez remplie pour qu'il n'y ait pas place pour mon souvenir. 

—Oh! murmura Valentine. 

—Adieu, Valentine, adieu! dit Morrel en s'inclinant. 

—Où allez-vous? cria en allongeant sa main à travers la grille et en saisissant Maximilien par son habit la jeune fille qui comprenait, à son agitation intérieure, que le calme de son amant ne pouvait être réel; où allez-vous? 

—Je vais m'occuper de ne point apporter un trouble nouveau dans votre famille, et donner un exemple que pourront suivre tous les hommes honnêtes et dévoués qui se trouveront dans ma position. 

—Avant de me quitter, dites-moi ce que vous allez faire, Maximilien?» 

Le jeune homme sourit tristement. 

«Oh! parlez, parlez! dit Valentine, je vous en prie! 

—Votre résolution a-t-elle changé, Valentine? 

—Elle ne peut changer, malheureux! Vous le savez bien! s'écria la jeune fille. 

—Alors, adieu, Valentine!» 

Valentine secoua la grille avec une force dont on l'aurait crue incapable; et comme Morrel s'éloignait, elle passa ses deux mains à travers la grille, et les joignant en se tordant les bras: 

«Qu'allez-vous faire? je veux le savoir! s'écria-t-elle; où allez-vous? 

—Oh! soyez tranquille, dit Maximilien en s'arrêtant à trois pas de la porte, mon intention n'est pas de rendre un autre homme responsable des rigueurs que le sort garde pour moi. Un autre vous menacerait d'aller trouver M. Franz, de le provoquer, de se battre avec lui, tout cela serait insensé. Qu'a à faire M. Franz dans tout cela? Il m'a vu ce matin pour la première fois, il a déjà oublié qu'il m'a vu; il ne savait même pas que j'existais lorsque des conventions faites par vos deux familles ont décidé que vous seriez l'un à l'autre. Je n'ai donc point affaire à M. Franz, et, je vous le jure, je ne m'en prendrai point à lui. 

—Mais à qui vous en prendrez-vous? à moi? 

—À vous, Valentine! Oh! Dieu m'en garde! La femme est sacrée; la femme qu'on aime est sainte. 

—À vous-même alors, malheureux, à vous-même? 

—C'est moi le coupable, n'est-ce pas? dit Morrel. 

—Maximilien, dit Valentine, Maximilien, venez ici, je le veux!» 

Maximilien se rapprocha avec son doux sourire, et, n'était sa pâleur, on eût pu le croire dans son état ordinaire. 

«Écoutez-moi, ma chère, mon adorée Valentine, dit-il de sa voix mélodieuse et grave, les gens comme nous, qui n'ont jamais formé une pensée dont ils aient eu à rougir devant le monde, devant leurs parents et devant Dieu, les gens comme nous peuvent lire dans le cœur l'un de l'autre à livre ouvert. Je n'ai jamais fait de roman, je ne suis pas un héros mélancolique, je ne me pose ni en Manfred ni en Antony: mais sans paroles, sans protestations, sans serments, j'ai mis ma vie en vous; vous me manquez et vous avez raison d'agir ainsi, je vous l'ai dit et je vous le répète; mais enfin vous me manquez et ma vie est perdue. Du moment où vous vous éloignez de moi, Valentine, je reste seul au monde. Ma sœur est heureuse près de son mari; son mari n'est que mon beau-frère, c'est-à-dire un homme que les conventions sociales attachent seules à moi; personne n'a donc besoin sur la terre de mon existence devenue inutile. Voilà ce que je ferai: j'attendrai jusqu'à la dernière seconde que vous soyez mariée, car je ne veux pas perdre l'ombre d'une de ces chances inattendues que nous garde quelquefois le hasard, car enfin d'ici là M. Franz d'Épinay peut mourir, au moment où vous vous en approcherez, la foudre peut tomber sur l'autel: tout semble croyable au condamné à mort, et pour lui les miracles rentrent dans la classe du possible dès qu'il s'agit du salut de sa vie. J'attendrai donc, dis-je, jusqu'au dernier moment, et quand mon malheur sera certain, sans remède, sans espérance, j'écrirai une lettre confidentielle à mon beau-frère, une autre au préfet de Police pour lui donner avis de mon dessein, et au coin de quelque bois, sur le revers de quelque fossé, au bord de quelque rivière, je me ferai sauter la cervelle, aussi vrai que je suis le fils du plus honnête homme qui ait jamais vécu en France.» 

Un tremblement convulsif agita les membres de Valentine; elle lâcha la grille qu'elle tenait de ses deux mains, ses bras retombèrent à ses côtés, et deux grosses larmes roulèrent sur ses joues. 

Le jeune homme demeura devant elle, sombre et résolu. 

«Oh! par pitié, par pitié, dit-elle, vous vivrez, n'est-ce pas? 

—Non, sur mon honneur, dit Maximilien; mais que vous importe à vous? vous aurez fait votre devoir, et votre conscience vous restera.» 

Valentine tomba à genoux en étreignant son cœur qui se brisait. 

«Maximilien, dit-elle, Maximilien, mon ami, mon frère sur la terre, mon véritable époux au ciel, je t'en prie, fais comme moi, vis avec la souffrance: un jour peut-être nous serons réunis. 

—Adieu, Valentine! répéta Morrel. 

—Mon Dieu! dit Valentine en levant ses deux mains au ciel avec une expression sublime, vous le voyez, j'ai fait tout ce que j'ai pu pour rester fille soumise: j'ai prié, supplié, imploré; il n'a écouté ni mes prières, ni mes supplications, ni mes pleurs. Eh bien, continua-t-elle en essuyant ses larmes et en reprenant sa fermeté, eh bien, je ne veux pas mourir de remords, j'aime mieux mourir de honte. Vous vivrez, Maximilien, et je ne serai à personne qu'à vous. À quelle heure? à quel moment? est-ce tout de suite? parlez, ordonnez, je suis prête.» 

Morrel, qui avait de nouveau fait quelques pas pour s'éloigner, était revenu de nouveau, et, pâle de joie, le cœur épanoui, tendant à travers la grille ses deux mains à Valentine: 

«Valentine, dit-il, chère amie, ce n'est point ainsi qu'il faut me parler, ou sinon il faut me laisser mourir. Pourquoi donc vous devrais-je à la violence, si vous m'aimez comme je vous aime? Me forcez-vous à vivre par humanité, voilà tout? en ce cas j'aime mieux mourir. 

—Au fait, murmura Valentine, qui est-ce qui m'aime au monde? lui. Qui m'a consolée de toutes mes douleurs? lui. Sur qui reposent mes espérances, sur qui s'arrête ma vue égarée, sur qui repose mon cœur saignant? sur lui, lui, toujours lui. Eh bien, tu as raison à ton tour; Maximilien, je te suivrai, je quitterai la maison paternelle, tout. Ô ingrate que je suis! s'écria Valentine en sanglotant, tout!\dots même mon bon grand-père que j'oubliais! 

—Non, dit Maximilien, tu ne le quitteras pas. M. Noirtier a paru éprouver, dis-tu, de la sympathie pour moi: eh bien, avant de fuir tu lui diras tout; tu te feras une égide devant Dieu de son consentement; puis, aussitôt mariés, il viendra avec nous: au lieu d'un enfant, il en aura deux. Tu m'as dit comment il te parlait et comment tu lui répondais; j'apprendrai bien vite cette langue touchante des signes, va, Valentine. Oh! je te le jure, au lieu du désespoir qui nous attend, c'est le bonheur que je te promets! 

—Oh! regarde, Maximilien, regarde quelle est ta puissance sur moi, tu me fais presque croire à ce que tu me dis, et cependant ce que tu me dis est insensé, car mon père me maudira, lui; car je le connais lui, le cœur inflexible, jamais il ne pardonnera. Aussi écoutez-moi, Maximilien, si par artifice, par prière, par accident, que sais-je, moi? si enfin par un moyen quelconque je puis retarder le mariage, vous attendrez, n'est-ce pas? 

—Oui, je le jure, comme vous me jurez, vous, que cet affreux mariage ne se fera jamais, et que, vous traînât-on devant le magistrat, devant le prêtre, vous direz non. 

—Je te le jure, Maximilien, par ce que j'ai de plus sacré au monde, par ma mère! 

—Attendons alors, dit Morrel. 

—Oui, attendons, reprit Valentine, qui respirait à ce mot; il y a tant de choses qui peuvent sauver des malheureux comme nous. 

—Je me fie à vous, Valentine, dit Morrel, tout ce que vous ferez sera bien fait; seulement, si l'on passe outre à vos prières, si votre père, si Mme de Saint-Méran exigent que M. Franz d'Épinay soit appelé demain à signer le contrat\dots. 

—Alors, vous avez ma parole, Morrel. 

—Au lieu de signer\dots. 

—Je viens vous rejoindre et nous fuyons: mais d'ici là, ne tentons pas Dieu, Morrel; ne nous voyons pas: c'est un miracle, c'est une providence que nous n'ayons pas encore été surpris; si nous étions surpris, si l'on savait comment nous nous voyons, nous n'aurions plus aucune ressource. 

—Vous avez raison, Valentine; mais comment savoir\dots. 

—Par le notaire, M. Deschamps. 

—Je le connais. 

—Et par moi-même. Je vous écrirai, croyez-le donc bien. Mon Dieu! ce mariage, Maximilien, m'est aussi odieux qu'à vous! 

—Bien, bien! merci, ma Valentine adorée, reprit Morrel. Alors tout est dit, une fois que je sais l'heure, j'accours ici, vous franchissez ce mur dans mes bras: la chose vous sera facile, une voiture vous attendra à la porte de l'enclos, vous y montez avec moi, je vous conduis chez ma sœur, là, inconnus si cela vous convient, faisant éclat si vous le désirez, nous aurons la conscience de notre force et de notre volonté, et nous ne nous laisserons pas égorger comme l'agneau qui ne se défend qu'avec ses soupirs. 

—Soit, dit Valentine; à votre tour je vous dirai: Maximilien, ce que vous ferez sera bien fait. 

—Oh! 

—Eh bien, êtes-vous content de votre femme? dit tristement la jeune fille. 

—Ma Valentine adorée, c'est bien peu dire que dire oui. 

—Dites toujours.» 

Valentine s'était approchée, ou plutôt avait approché ses lèvres de la grille, et ses paroles glissaient, avec son souffle parfumé, jusqu'aux lèvres de Morrel, qui collait sa bouche de l'autre côté de la froide et inexorable clôture. 

«Au revoir, dit Valentine, s'arrachant à ce bonheur, au revoir! 

—J'aurai une lettre de vous? 

—Oui.  

—Merci, chère femme! au revoir.» 

Le bruit d'un baiser innocent et perdu retentit, et Valentine s'enfuit sous les tilleuls. 

Morrel écouta les derniers bruits de sa robe frôlant les charmilles, de ses pieds faisant crier le sable, leva les yeux au ciel avec un ineffable sourire pour remercier le ciel de ce qu'il permettait qu'il fût aimé ainsi, et disparut à son tour. 

Le jeune homme rentra chez lui et attendit pendant tout le reste de la soirée et pendant toute la journée du lendemain sans rien recevoir. Enfin, ce ne fut que le surlendemain, vers dix heures du matin, comme il allait s'acheminer vers M. Deschamps, notaire, qu'il reçut par la poste un petit billet qu'il reconnut pour être de Valentine, quoiqu'il n'eût jamais vu son écriture. 

Il était conçu en ces termes: 

\begin{mail}{}{}
Larmes, supplications, prières, n'ont rien fait. Hier, pendant deux heures, j'ai été à l'église Saint-Philippe-du-Roule, et pendant deux heures j'ai prié Dieu du fond de l'âme, Dieu est insensible comme les hommes, et la signature du contrat est fixée à ce soir, neuf heures. 

Je n'ai qu'une parole comme je n'ai qu'un cœur, Morrel, et cette parole vous est engagée: ce cœur est à vous! 


Ce soir donc, à neuf heures moins un quart, à la grille.

\addPS{Ma pauvre grand-mère va de plus mal en plus mal; hier, son exaltation est devenue du délire: aujourd'hui son délire est presque de la folie.\\Vous m'aimerez bien, n'est-ce pas, Morrel, pour me faire oublier que je l'aurai quittée en cet état?\\Je crois que l'on cache à grand-papa Noirtier que la signature du contrat doit avoir lieu ce soir.}


\closeletter[Votre femme,]{Valentine de Villefort.}
\end{mail}

Morrel ne se borna pas aux renseignements que lui donnait Valentine; il alla chez le notaire, qui lui confirma la nouvelle que la signature du contrat était pour neuf heures du soir. 

Puis il passa chez Monte-Cristo; ce fut encore là qu'il en sut le plus: Franz était venu lui annoncer cette solennité; de son côté, Mme de Villefort avait écrit au comte pour le prier de l'excuser si elle ne l'invitait point; mais la mort de M. de Saint-Méran et l'état où se trouvait sa veuve jetaient sur cette réunion un voile de tristesse dont elle ne voulait pas assombrir le front du comte, auquel elle souhaitait toute sorte de bonheur. 

La veille, Franz avait été présenté à Mme de Saint-Méran, qui avait quitté le lit pour cette présentation, et qui s'y était remise aussitôt. 

Morrel, la chose est facile à comprendre, était dans un état d'agitation qui ne pouvait échapper à un œil aussi perçant que l'était l'œil du comte, aussi Monte-Cristo fut-il pour lui plus affectueux que jamais; si affectueux, que deux ou trois fois Maximilien fut sur le point de lui tout dire. Mais il se rappela la promesse formelle donnée à Valentine, et son secret resta au fond de son cœur. 

Le jeune homme relut vingt fois dans la journée la lettre de Valentine. C'était la première fois qu'elle lui écrivait, et à quelle occasion! À chaque fois qu'il relisait cette lettre, Maximilien se renouvelait à lui-même le serment de rendre Valentine heureuse. En effet, quelle autorité n'a pas la jeune fille qui prend une résolution si courageuse! quel dévouement ne mérite-t-elle pas de la part de celui à qui elle a tout sacrifié! Comme elle doit être réellement pour son amant le premier et le plus digne objet de son culte! C'est à la fois la reine et la femme, et l'on n'a point assez d'une âme pour la remercier et l'aimer. 

Morrel songeait avec une agitation inexprimable à ce moment où Valentine arriverait en disant: 

«Me voici, Maximilien; prenez-moi.» 

Il avait organisé toute cette fuite; deux échelles avaient été cachées dans la luzerne du clos; un cabriolet, que devait conduire Maximilien lui-même, attendait; pas de domestique, pas de lumière; au détour de la première rue on allumerait des lanternes, car il ne fallait point, par un surcroît de précautions, tomber entre les mains de la police. 

De temps en temps des frissonnements passaient par tout le corps de Morrel; il songeait au moment où, du faîte de ce mur, il protégerait la descente de Valentine, et où il sentirait tremblante et abandonnée dans ses bras celle dont il n'avait jamais pressé que la main et baisé le bout du doigt. 

Mais quand vint l'après-midi, quand Morrel sentit l'heure s'approcher, il éprouva le besoin d'être seul; son sang bouillait, les simples questions, la seule voix d'un ami l'eussent irrité; il se renferma chez lui, essayant de lire; mais son regard glissa sur les pages sans y rien comprendre, et il finit par jeter son livre, pour en revenir à dessiner, pour la deuxième fois, son plan, ses échelles et son clos. 

Enfin l'heure s'approcha. 

Jamais l'homme bien amoureux n'a laissé les horloges faire paisiblement leur chemin; Morrel tourmenta si bien les siennes, qu'elles finirent par marquer huit heures et demie à six heures. Il se dit alors qu'il était temps de partir, que neuf heures était bien effectivement l'heure de la signature du contrat, mais que, selon toute probabilité, Valentine n'attendrait pas cette signature inutile; en conséquence, Morrel, après être parti de la rue Meslay à huit heures et demie à sa pendule, entrait dans le clos comme huit heures sonnèrent à Saint-Philippe-du-Roule. 

Le cheval et le cabriolet furent cachés derrière une petite masure en ruine dans laquelle Morrel avait l'habitude de se cacher. 

Peu à peu le jour tomba, et les feuillages du jardin se massèrent en grosses touffes d'un noir opaque. 

Alors Morrel sortit de la cachette et vint regarder, le cœur palpitant, au trou de la grille: il n'y avait encore personne.  

Huit heures et demie sonnèrent. 

Une demi-heure s'écoula à attendre; Morrel se promenait de long en large, puis, à des intervalles toujours plus rapprochés, venait appliquer son œil aux planches. Le jardin s'assombrissait de plus en plus; mais dans l'obscurité on cherchait vainement la robe blanche; dans le silence on écoutait inutilement le bruit des pas. 

La maison qu'on apercevait à travers les feuillages restait sombre, et ne présentait aucun des caractères d'une maison qui s'ouvre pour un événement aussi important que l'est une signature du contrat de mariage. 

Morrel consulta sa montre, qui sonna neuf heures trois quarts; mais presque aussitôt cette même voix de l'horloge, déjà entendue deux ou trois fois rectifia l'erreur de la montre en sonnant neuf heures et demie. 

C'était déjà une demi-heure d'attente de plus que Valentine n'avait fixée elle-même: elle avait dit neuf heures, même plutôt avant qu'après. 

Ce fut le moment le plus terrible pour le cœur du jeune homme, sur lequel chaque seconde tombait comme un marteau de plomb. 

Le plus faible bruit du feuillage, le moindre cri du vent appelaient son oreille et faisaient monter la sueur à son front; alors, tout frissonnant, il assujettissait son échelle et, pour ne pas perdre de temps, posait le pied sur le premier échelon. 

Au milieu de ces alternatives de crainte et d'espoir, au milieu de ces dilatations et de ces serrements de cœur, dix heures sonnèrent à l'église. 

«Oh! murmura Maximilien avec terreur, il est impossible que la signature d'un contrat dure aussi longtemps, à moins d'événements imprévus; j'ai pesé toutes les chances, calculé le temps que durent toutes les formalités, il s'est passé quelque chose.» 

Et alors, tantôt il se promenait avec agitation devant la grille, tantôt il revenait appuyer son front brûlant sur le fer glacé. Valentine s'était-elle évanouie après le contrat, ou Valentine avait-elle été arrêtée dans sa fuite? C'étaient là les deux seules hypothèses où le jeune homme pouvait s'arrêter, toutes deux désespérantes. 

L'idée à laquelle il s'arrêta fut qu'au milieu de sa fuite même la force avait manqué à Valentine, et qu'elle était tombée évanouie au milieu de quelque allée. 

«Oh! s'il en est ainsi, s'écria-t-il en s'élançant au haut de l'échelle, je la perdrais, et par ma faute!» 

Le démon qui lui avait soufflé cette pensée ne le quitta plus, et bourdonna à son oreille avec cette persistance qui fait que certains doutes, au bout d'un instant, par la force du raisonnement, deviennent des convictions. Ses yeux, qui cherchaient à percer l'obscurité croissante, croyaient, sous la sombre allée, apercevoir un objet gisant; Morrel se hasarda jusqu'à appeler, et il lui sembla que le vent apportait jusqu'à lui une plainte inarticulée. 

Enfin la demie avait sonné à son tour, il était impossible de se borner plus longtemps, tout était supposable; les tempes de Maximilien battaient avec force, des nuages passaient devant ses yeux; il enjamba le mur et sauta de l'autre côté. 

Il était chez Villefort, il venait d'y entrer par escalade; il songea aux suites que pouvait avoir une pareille action, mais il n'était pas venu jusque-là pour reculer. 

En un instant il fut à l'extrémité de ce massif. Du point où il était parvenu on découvrait la maison. 

Alors Morrel s'assura d'une chose qu'il avait déjà soupçonnée en essayant de glisser son regard à travers les arbres: c'est qu'au lieu des lumières qu'il pensait voir briller à chaque fenêtre, ainsi qu'il est naturel aux jours de cérémonie, il ne vit rien que la masse grise et voilée encore par un grand rideau d'ombre que projetait un nuage immense répandu sur la lune. 

Une lumière courait de temps en temps comme éperdue, et passait devant trois fenêtres du premier étage. Ces trois fenêtres étaient celles de l'appartement de Mme de Saint-Méran. 

Une autre lumière restait immobile derrière des rideaux rouges. Ces rideaux étaient ceux de la chambre à coucher de Mme de Villefort. 

Morrel devina tout cela. Tant de fois, pour suivre Valentine en pensée à toute heure du jour, tant de fois, disons-nous, il s'était fait faire le plan de cette maison, que, sans l'avoir vue, il la connaissait. 

Le jeune homme fut encore plus épouvanté de cette obscurité et de ce silence qu'il ne l'avait été de l'absence de Valentine. 

Éperdu, fou de douleur, décidé à tout braver pour revoir Valentine et s'assurer du malheur qu'il pressentait, quel qu'il fût, Morrel gagna la lisière du massif, et s'apprêtait à traverser le plus rapidement possible le parterre, complètement découvert, quand un son de voix encore assez éloigné, mais que le vent lui apportait, parvint jusqu'à lui. 

À ce bruit, il fit un pas en arrière, déjà à moitié sorti du feuillage, il s'y enfonça complètement et demeura immobile et muet, enfoui dans son obscurité. 

Sa résolution était prise: si c'était Valentine seule, il l'avertirait par un mot au passage; si Valentine était accompagnée, il la verrait au moins et s'assurerait qu'il ne lui était arrivé aucun malheur; si c'étaient des étrangers, il saisirait quelques mots de leur conversation et arriverait à comprendre ce mystère, incompréhensible jusque-là. 

La lune alors sortit du nuage qui la cachait, et, sur la porte du perron, Morrel vit apparaître Villefort, suivi d'un homme vêtu de noir. Ils descendirent les marches et s'avancèrent vers le massif. Ils n'avaient pas fait quatre pas que, dans cet homme vêtu de noir, Morrel avait reconnu le docteur d'Avrigny. 

Le jeune homme, en les voyant venir à lui, recula machinalement devant eux jusqu'à ce qu'il rencontrât le tronc d'un sycomore qui faisait le centre du massif; là il fut forcé de s'arrêter. 

Bientôt le sable cessa de crier sous les pas des deux promeneurs.  

«Ah! cher docteur, dit le procureur du roi, voici le Ciel qui se déclare décidément contre ma maison. Quelle horrible mort! quel coup de foudre! N'essayez pas de me consoler; hélas! la plaie est trop vive et trop profonde! Morte, morte!» 

Une sueur froide glaça le front du jeune homme et fit claquer ses dents. Qui donc était mort dans cette maison que Villefort lui-même disait maudite? 

«Mon cher monsieur de Villefort, répondit le médecin avec un accent qui redoubla la terreur du jeune homme, je ne vous ai point amené ici pour vous consoler, tout au contraire. 

—Que voulez-vous dire? demanda le procureur du roi, effrayé. 

—Je veux dire que, derrière le malheur qui vient de vous arriver, il en est un autre plus grand encore peut-être. 

—Oh! mon Dieu! murmura Villefort en joignant les mains, qu'allez-vous me dire encore? 

—Sommes-nous bien seuls, mon ami? 

—Oh! oui, bien seuls. Mais que signifient toutes ces précautions? 

—Elles signifient que j'ai une confidence terrible à vous faire, dit le docteur: asseyons-nous.» 

Villefort tomba plutôt qu'il ne s'assit sur un banc. Le docteur resta debout devant lui, une main posée sur son épaule. Morrel, glacé d'effroi, tenait d'une main son front, de l'autre comprimait son cœur, dont il craignait qu'on entendît les battements. 

«Morte, morte!» répétait-il dans sa pensée avec la voix de son cœur. 

Et lui-même se sentait mourir. 

«Parlez, docteur, j'écoute, dit Villefort; frappez, je suis préparé à tout. 

—Mme de Saint-Méran était bien âgée sans doute, mais elle jouissait d'une santé excellente.» 

Morrel respira pour la première fois depuis dix minutes. 

«Le chagrin l'a tuée, dit Villefort, oui, le chagrin, docteur! Cette habitude de vivre depuis quarante ans près du marquis!\dots 

—Ce n'est pas le chagrin, mon cher Villefort, dit le docteur. Le chagrin peut tuer, quoique les cas soient rares, mais il ne tue pas en un jour, mais il ne tue pas en une heure, mais il ne tue pas en dix minutes.» 

Villefort ne répondit rien; seulement il leva la tête qu'il avait tenue baissée jusque-là, et regarda le docteur avec des yeux effarés. 

«Vous êtes resté là pendant l'agonie? demanda M. d'Avrigny. 

—Sans doute, répondit le procureur du roi; vous m'avez dit tout bas de ne pas m'éloigner. 

—Avez-vous remarqué les symptômes du mal auquel Mme de Saint-Méran a succombé? 

—Certainement; Mme de Saint-Méran a eu trois attaques successives à quelques minutes les unes des autres, et à chaque fois plus rapprochées et plus graves. Lorsque vous êtes arrivé, déjà depuis quelques minutes Mme de Saint-Méran était haletante; elle eut alors une crise que je pris pour une simple attaque de nerfs; mais je ne commençai à m'effrayer réellement que lorsque je la vis se soulever sur son lit, les membres et le cou tendus. Alors, à votre visage, je compris que la chose était plus grave que je ne le croyais. La crise passée, je cherchai vos yeux, mais je ne les rencontrai pas. Vous teniez le pouls, vous en comptiez les battements, et la seconde crise parut, que vous ne vous étiez pas encore retourné de mon côté. Cette seconde crise fut plus terrible que la première: les mêmes mouvements nerveux se reproduisirent, et la bouche se contracta et devint violette. 

«À la troisième elle expira. 

«Déjà, depuis la fin de la première, j'avais reconnu le tétanos; vous me confirmâtes dans cette opinion. 

—Oui, devant tout le monde, reprit le docteur; mais maintenant nous sommes seuls. 

—Qu'allez-vous me dire, mon Dieu? 

—Que les symptômes du tétanos et de l'empoisonnement par les matières végétales sont absolument les mêmes.» 

M. de Villefort se dressa sur ses pieds; puis, après un instant d'immobilité et de silence, il retomba sur son banc.  

«Oh! mon Dieu! docteur, dit-il, songez-vous bien à ce que vous me dites là?» 

Morrel ne savait pas s'il faisait un rêve ou s'il veillait. 

«Écoutez, dit le docteur, je connais l'importance de ma déclaration et le caractère de l'homme à qui je la fais. 

—Est-ce au magistrat ou à l'ami que vous parlez? demanda Villefort. 

—À l'ami, à l'ami seul en ce moment; les rapports entre les symptômes du tétanos et les symptômes de l'empoisonnement par les substances végétales sont tellement identiques, que s'il me fallait signer ce que je dis là, je vous déclare que j'hésiterais. Aussi, je vous le répète, ce n'est point au magistrat que je m'adresse, c'est à l'ami. Eh bien, à l'ami je dis: Pendant les trois quarts d'heure qu'elle a duré, j'ai étudié l'agonie, les convulsions, la mort de Mme de Saint-Méran; eh bien, dans ma conviction, non seulement Mme de Saint-Méran est morte empoisonnée, mais encore je dirais, oui, je dirais quel poison l'a tuée. 

—Monsieur! monsieur! 

—Tout y est, voyez-vous: somnolence interrompue par des crises nerveuses, surexcitation du cerveau, torpeur des centres. Mme de Saint-Méran a succombé à une dose violente de brucine ou de strychnine, que par hasard sans doute, que par erreur peut-être, on lui a administrée.»  

Villefort saisit la main du docteur. 

«Oh! c'est impossible! dit-il, je rêve, mon Dieu! je rêve! C'est effroyable d'entendre dire des choses pareilles à un homme comme vous! Au nom du Ciel, je vous en supplie, cher docteur, dites-moi que vous pouvez vous tromper! 

—Sans doute, je le puis, mais\dots. 

—Mais?\dots 

—Mais, je ne le crois pas. 

—Docteur, prenez pitié de moi; depuis quelques jours il m'arrive tant de choses inouïes, que je crois à la possibilité de devenir fou. 

—Un autre que moi a-t-il vu Mme de Saint-Méran? 

—Personne. 

—A-t-on envoyé chez le pharmacien quelque ordonnance qu'on ne m'ait pas soumise? 

—Aucune. 

—Mme de Saint-Méran avait-elle des ennemis? 

—Je ne lui en connais pas. 

—Quelqu'un avait-il intérêt à sa mort?  

—Mais non, mon Dieu! mais non; ma fille est sa seule héritière, Valentine seule\dots. Oh! si une pareille pensée me pouvait venir, je me poignarderais pour punir mon cœur d'avoir pu un seul instant abriter une pareille pensée. 

—Oh! s'écria à son tour M. d'Avrigny, cher ami, à Dieu ne plaise que j'accuse quelqu'un, je ne parle que d'un accident, comprenez-vous bien, d'une erreur. Mais accident ou erreur, le fait est là qui parle tout bas à ma conscience, et qui veut que ma conscience vous parle tout haut. Informez-vous. 

—À qui? comment? de quoi? 

—Voyons: Barrois, le vieux domestique, ne se serait-il pas trompé, et n'aurait-il pas donné à Mme de Saint-Méran quelque potion préparée pour son maître? 

—Pour mon père? 

—Oui. 

—Mais comment une potion préparée pour M. Noirtier peut-elle empoisonner Mme de Saint-Méran? 

—Rien de plus simple: vous savez que dans certaines maladies les poisons deviennent un remède; la paralysie est une de ces maladies-là. À peu près depuis trois mois, après avoir tout employé pour rendre le mouvement et la parole à M. Noirtier, je me suis décidé à tenter un dernier moyen; depuis trois mois, dis-je, je le traite par la brucine; ainsi, dans la dernière potion que j'ai commandée pour lui il en entrait six centigrammes; six centigrammes sans action sur les organes paralysés de M. Noirtier, et auxquels d'ailleurs il s'est accoutumé par des doses successives, six centigrammes suffisent pour tuer toute autre personne que lui. 

—Mon cher docteur, il n'y a aucune communication entre l'appartement de M. Noirtier et celui de Mme de Saint-Méran, et jamais Barrois n'entrait chez ma belle-mère. Enfin, vous le dirai-je, docteur, quoique je vous sache homme le plus habile et surtout le plus consciencieux du monde, quoique en toute circonstance votre parole soit pour moi un flambeau qui me guide à l'égal de la lumière du soleil, eh bien! docteur, eh bien! j'ai besoin, malgré cette conviction de m'appuyer sur cet axiome, \textit{errare humanum est}. 

—Écoutez, Villefort, dit le docteur, existe-t-il un de mes confrères en qui vous ayez autant confiance qu'en moi? 

—Pourquoi cela, dites? où voulez-vous en venir? 

—Appelez-le, je lui dirai ce que j'ai vu, ce que j'ai remarqué, nous ferons l'autopsie. 

—Et vous trouverez des traces de poison? 

—Non, pas du poison, je n'ai pas dit cela, mais nous constaterons l'exaspération du système nerveux, nous reconnaîtrons l'asphyxie patente, incontestable et nous vous dirons: Cher Villefort, si c'est par négligence que la chose est arrivée, veillez sur vos serviteurs; si c'est par haine, veillez sur vos ennemis. 

—Oh! mon Dieu! que me proposez-vous là, d'Avrigny? répondit Villefort abattu; du moment où il y aura un autre que vous dans le secret, une enquête deviendra nécessaire, et une enquête chez moi, impossible! Pourtant, continua le procureur du roi en se reprenant et en regardant le médecin avec inquiétude, pourtant si vous le voulez, si vous l'exigez absolument, je le ferai. En effet, peut-être dois-je donner suite à cette affaire; mon caractère me le commande. Mais docteur, vous me voyez d'avance pénétré de tristesse: introduire dans ma maison tant de scandale après tant de douleur! Oh! ma femme et ma fille en mourront; et moi, moi, docteur, vous le savez, un homme n'en arrive pas où j'en suis, un homme n'a pas été procureur du roi pendant vingt-cinq ans sans s'être amassé bon nombre d'ennemis; les miens sont nombreux. Cette affaire ébruitée sera pour eux un triomphe qui les fera tressaillir de joie, et moi me couvrira de honte. Docteur, pardonnez-moi ces idées mondaines. Si vous étiez un prêtre, je n'oserais vous dire cela; mais vous êtes un homme, mais vous connaissez les autres hommes; docteur, docteur, vous ne m'avez rien dit, n'est-ce pas? 

—Mon cher monsieur de Villefort, répondit le docteur ébranlé, mon premier devoir est l'humanité. J'eusse sauvé Mme de Saint-Méran si la science eût eu le pouvoir de le faire, mais elle est morte, je me dois aux vivants. Ensevelissons au plus profond de nos cœurs ce terrible secret. Je permettrai, si les yeux de quelques-uns s'ouvrent là-dessus, qu'on impute à mon ignorance le silence que j'aurai gardé. Cependant, monsieur, cherchez toujours, cherchez activement, car peut-être cela ne s'arrêtera-t-il point là\dots. Et quand vous aurez trouvé le coupable, si vous le trouvez, c'est moi qui vous dirai: Vous êtes magistrat, faites ce que vous voudrez! 

—Oh! merci, merci, docteur! dit Villefort avec une joie indicible, je n'ai jamais eu de meilleur ami que vous.» 

Et comme s'il eût craint que le docteur d'Avrigny ne revînt sur cette concession, il se leva et entraîna le docteur du côté de la maison. 

Ils s'éloignèrent. 

Morrel, comme s'il eût besoin de respirer, sortit sa tête du taillis, et la lune éclaira ce visage si pâle qu'on eût pu le prendre pour un fantôme. 

«Dieu me protège d'une manifeste mais terrible façon, dit-il. Mais Valentine, Valentine! pauvre amie! résistera-t-elle à tant de douleurs?» 

En disant ces mots il regardait alternativement la fenêtre aux rideaux rouges et les trois fenêtres aux rideaux blancs. 

La lumière avait presque complètement disparu de la fenêtre aux rideaux rouges. Sans doute Mme de Villefort venait d'éteindre sa lampe, et la veilleuse seule envoyait son reflet aux vitres. 

À l'extrémité du bâtiment, au contraire, il vit s'ouvrir une des trois fenêtres aux rideaux blancs. Une bougie placée sur la cheminée jeta au-dehors quelques rayons de sa pâle lumière, et une ombre vint un instant s'accouder au balcon. 

Morrel frissonna; il lui semblait avoir entendu un sanglot. 

Il n'était pas étonnant que cette âme ordinairement si courageuse et si forte, maintenant troublée et exaltée par les deux plus fortes des passions humaines, l'amour et la peur, se fût affaiblie au point de subir des hallucinations superstitieuses.  

Quoiqu'il fût impossible, caché comme il l'était, que l'œil de Valentine le distinguât, il crut se voir appeler par l'ombre de la fenêtre; son esprit troublé le lui disait, son cœur ardent le lui répétait. Cette double erreur devenait une réalité irrésistible, et, par un de ces incompréhensibles élans de jeunesse, il bondit hors de sa cachette, et en deux enjambées, au risque d'être vu, au risque d'effrayer Valentine, au risque de donner l'éveil par quelque cri involontaire échappé à la jeune fille, il franchit ce parterre que la lune faisait large et blanc comme un lac, et, gagnant la rangée de caisses d'orangers qui s'étendait devant la maison, il atteignit les marches du perron, qu'il monta rapidement, et poussa la porte, qui s'ouvrit sans résistance devant lui. 

Valentine ne l'avait pas vu; ses yeux levés au ciel suivaient un nuage d'argent glissant sur l'azur, et dont la forme était celle d'une ombre qui monte au ciel; son esprit poétique et exalté lui disait que c'était l'âme de sa grand-mère. 

Cependant, Morrel avait traversé l'antichambre et trouvé la rampe de l'escalier; des tapis étendus sur les marches assourdissaient son pas; d'ailleurs Morrel en était arrivé à ce point d'exaltation que la présence de M. de Villefort lui-même ne l'eût pas effrayé. Si M. de Villefort se fût présenté à sa vue, sa résolution était prise: il s'approchait de lui et lui avouait tout, en le priant d'excuser et d'approuver cet amour qui l'unissait à sa fille, et sa fille à lui; Morrel était fou. 

Par bonheur il ne vit personne. 

Ce fut alors surtout que cette connaissance qu'il avait prise par Valentine du plan intérieur de la maison lui servit; il arriva sans accident au haut de l'escalier, et comme, arrivé là, il s'orientait, un sanglot dont il reconnut l'expression lui indiqua le chemin qu'il avait à suivre; il se retourna; une porte entrebâillée laissait arriver à lui le reflet d'une lumière et le son de la voix gémissante. Il poussa cette porte et entra. 

Au fond d'une alcôve, sous le drap blanc qui recouvrait sa tête et dessinait sa forme, gisait la morte, plus effrayante encore aux yeux de Morrel depuis la révélation du secret dont le hasard l'avait fait possesseur. 

À côté du lit, à genoux, la tête ensevelie dans les coussins d'une large bergère, Valentine, frissonnante et soulevée par les sanglots, étendait au-dessus de sa tête, qu'on ne voyait pas, ses deux mains jointes et raidies. 

Elle avait quitté la fenêtre restée ouverte, et priait tout haut avec des accents qui eussent touché le cœur le plus insensible, la parole s'échappait de ses lèvres, rapide, incohérente, inintelligible, tant la douleur serrait sa gorge de ses brûlantes étreintes. 

La lune, glissant à travers l'ouverture des persiennes, faisait pâlir la lueur de la bougie, et azurait de ses teintes funèbres ce tableau de désolation. 

Morrel ne put résister à ce spectacle; il n'était pas d'une piété exemplaire, il n'était pas facile à impressionner, mais Valentine souffrant, pleurant, se tordant les bras à sa vue, c'était plus qu'il n'en pouvait supporter en silence. Il poussa un soupir, murmura un nom, et la tête noyée dans les pleurs et marbrée sur le velours du fauteuil, une tête de Madeleine du Corrège, se releva et demeura tournée vers lui. 

Valentine le vit et ne témoigna point d'étonnement. Il n'y a plus d'émotions intermédiaires dans un cœur gonflé par un désespoir suprême. 

Morrel tendit la main à son amie. Valentine, pour toute excuse de ce qu'elle n'avait point été le trouver, lui montra le cadavre gisant sous le drap funèbre et recommença à sangloter. 

Ni l'un ni l'autre n'osait parler dans cette chambre. Chacun hésitait à rompre ce silence que semblait commander la Mort debout dans quelque coin et le doigt sur les lèvres. 

Enfin Valentine osa la première. 

«Ami, dit-elle, comment êtes-vous ici? Hélas! je vous dirais: soyez le bienvenu, si ce n'était pas la Mort qui vous eût ouvert la porte de cette maison. 

—Valentine, dit Morrel d'une voix tremblante et les mains jointes, j'étais là depuis huit heures et demie; je ne vous voyais point venir, l'inquiétude m'a pris, j'ai sauté par-dessus le mur, j'ai pénétré dans le jardin; alors des voix qui s'entretenaient du fatal accident\dots. 

—Quelles voix?» dit Valentine. 

Morrel frémit, car toute la conversation du docteur et de M. de Villefort lui revint à l'esprit, et, à travers le drap, il croyait voir ces bras tordus, ce cou raidi, ces lèvres violettes.  

«Les voix de vos domestiques, dit-il, m'ont tout appris. 

—Mais venir jusqu'ici, c'est nous perdre, mon ami, dit Valentine, sans effroi et sans colère. 

—Pardonnez-moi, répondit Morrel du même ton, je vais me retirer. 

—Non, dit Valentine, on vous rencontrerait, restez. 

—Mais si l'on venait?» 

La jeune fille secoua la tête. 

«Personne ne viendra, dit-elle, soyez tranquille, voilà notre sauvegarde.» 

Et elle montra la forme du cadavre moulée par le drap. 

«Mais qu'est-il arrivé à M. d'Épinay? dites-moi, je vous en supplie, reprit Morrel. 

—M. Franz est arrivé pour signer le contrat au moment où ma bonne grand-mère rendait le dernier soupir. 

—Hélas! dit Morrel avec un sentiment de joie égoïste, car il songeait en lui-même que cette mort retardait indéfiniment le mariage de Valentine. 

—Mais ce qui redouble ma douleur, continua la jeune fille, comme si ce sentiment eût dû recevoir à l'instant même sa punition, c'est que cette pauvre chère aïeule, en mourant, a ordonné qu'on terminât le mariage le plus tôt possible; elle aussi, mon Dieu! en croyant me protéger, elle aussi agissait contre moi. 

—Écoutez!» dit Morrel. 

Les deux jeunes gens firent silence. 

On entendit la porte qui s'ouvrit, et des pas firent craquer le parquet du corridor et les marches de l'escalier. 

«C'est mon père qui sort de son cabinet, dit Valentine. 

—Et qui reconduit le docteur, ajouta Morrel. 

—Comment savez-vous que c'est le docteur? demanda Valentine étonnée. 

—Je le présume» dit Morrel. 

Valentine regarda le jeune homme. 

Cependant, on entendit la porte de la rue se fermer. M. de Villefort alla donner en outre un tour de clef à celle du jardin puis il remonta l'escalier. 

Arrivé dans l'antichambre, il s'arrêta un instant, comme s'il hésitait s'il devait entrer chez lui ou dans la chambre de Mme de Saint-Méran. Morrel se jeta derrière une portière. Valentine ne fit pas un mouvement; on eût dit qu'une suprême douleur la plaçait au-dessus des craintes ordinaires. 

M. de Villefort rentra chez lui. 

«Maintenant, dit Valentine, vous ne pouvez plus sortir ni par la porte du jardin, ni par celle de la rue.» 

Morrel regarda la jeune fille avec étonnement. 

«Maintenant, dit-elle, il n'y a plus qu'une issue permise et sûre, c'est celle de l'appartement de mon grand-père.» 

Elle se leva. 

«Venez, dit-elle. 

—Où cela? demanda Maximilien. 

—Chez mon grand-père. 

—Moi, chez M. Noirtier? 

—Oui. 

—Y songez-vous, Valentine? 

—J'y songe, et depuis longtemps. Je n'ai plus que cet ami au monde, et nous avons tous deux besoin de lui\dots. Venez. 

—Prenez garde, Valentine, dit Morrel, hésitant à faire ce que lui ordonnait la jeune fille; prenez garde, le bandeau est tombé de mes yeux: en venant ici, j'ai accompli un acte de démence. Avez-vous bien vous-même toute votre raison, chère amie? 

—Oui, dit Valentine, et je n'ai aucun scrupule au monde, si ce n'est de laisser seuls les restes de ma pauvre grand-mère, que je me suis chargée de garder. 

—Valentine, dit Morrel, la mort est sacrée par elle-même. 

—Oui, répondit la jeune fille; d'ailleurs ce sera court, venez.» 

Valentine traversa le corridor et descendit un petit escalier qui conduisait chez Noirtier. Morrel la suivait sur la pointe du pied. Arrivés sur le palier de l'appartement, ils trouvèrent le vieux domestique. 

«Barrois, dit Valentine, fermez la porte et ne laissez entrer personne.» 

Elle passa la première. 

Noirtier, encore dans son fauteuil, attentif au moindre bruit, instruit par son vieux serviteur de tout ce qui se passait, fixait des regards avides sur l'entrée de la chambre; il vit Valentine, et son œil brilla. 

Il y avait dans la démarche et dans l'attitude de la jeune fille quelque chose de grave et de solennel qui frappa le vieillard. Aussi, de brillant qu'il était, son œil devint-il interrogateur. 

«Cher père, dit-elle d'une voix brève, écoute-moi bien: tu sais que bonne maman Saint-Méran est morte il y a une heure, et que maintenant, excepté toi je n'ai plus personne qui m'aime au monde?» 

Une expression de tendresse infinie passa dans les yeux du vieillard. 

«C'est donc à toi seul, n'est-ce pas, que je dois confier mes chagrins ou mes espérances?» 

Le paralytique fit signe que oui. 

Valentine prit Maximilien par la main. 

«Alors, lui dit-elle, regarde bien monsieur.» 

Le vieillard fixa son œil scrutateur et légèrement étonné sur Morrel. 

«C'est M. Maximilien Morrel, dit-elle, le fils de cet homme négociant de Marseille dont tu as sans doute entendu parler? 

—Oui, fit le vieillard. 

—C'est un nom irréprochable, que Maximilien est en train de rendre glorieux, car, à trente ans, il est capitaine de spahis, officier de la Légion d'honneur.» 

Le vieillard fit signe qu'il se le rappelait. 

«Eh bien, bon papa, dit Valentine en se mettant à deux genoux devant le vieillard et en montrant Maximilien d'une main, je l'aime et ne serai qu'à lui! Si l'on me force d'en épouser un autre, je me laisserai mourir ou je me tuerai.» 

Les yeux du paralytique exprimaient tout un monde de pensées tumultueuses. 

«Tu aimes M. Maximilien Morrel, n'est-ce pas, bon papa? demanda la jeune fille. 

—Oui, fit le vieillard immobile. 

—Et tu peux bien nous protéger, nous qui sommes aussi tes enfants, contre la volonté de mon père?» 

Noirtier attacha son regard intelligent sur Morrel, comme pour lui dire: 

«C'est selon.» 

Maximilien comprit. 

«Mademoiselle, dit-il, vous avez un devoir sacré à remplir dans la chambre de votre aïeule; voulez-vous me permettre d'avoir l'honneur de causer un instant avec M. Noirtier? 

—Oui, oui, c'est cela», fit l'œil du vieillard. 

Puis il regarda Valentine avec inquiétude. 

«Comment il fera pour te comprendre, veux-tu dire, bon père? 

—Oui. 

—Oh! sois tranquille; nous avons si souvent parlé de toi, qu'il sait bien comment je te parle.»  

Puis, se tournant vers Maximilien avec un adorable sourire, quoique ce sourire fût voilé par une profonde tristesse: 

«Il sait tout ce que je sais», dit-elle. 

Valentine se releva, approcha un siège pour Morrel, recommanda à Barrois de ne laisser entrer personne; et après avoir embrassé tendrement son grand-père et dit adieu tristement à Morrel, elle partit. Alors Morrel, pour prouver à Noirtier qu'il avait la confiance de Valentine et connaissait tous leurs secrets, prit le dictionnaire, la plume et le papier, et plaça le tout sur une table où il y avait une lampe. 

«Mais d'abord, dit Morrel, permettez-moi, monsieur, de vous raconter qui je suis, comment j'aime Mlle Valentine, et quels sont mes desseins à son égard. 

—J'écoute», fit Noirtier. 

C'était un spectacle assez imposant que ce vieillard, inutile fardeau en apparence, et qui était devenu le seul protecteur, le seul appui, le seul juge de deux amants jeunes, beaux, forts, et entrant dans la vie. 

Sa figure, empreinte d'une noblesse et d'une austérité remarquables, imposait à Morrel, qui commença son récit en tremblant. 

Il raconta alors comment il avait connu, comment il avait aimé Valentine et comment Valentine, dans son isolement et son malheur, avait accueilli l'offre de son dévouement. Il lui dit quelles étaient sa naissance, sa position, sa fortune; et plus d'une fois, lorsqu'il interrogea le regard du paralytique, ce regard lui répondit: 

«C'est bien, continuez. 

—Maintenant, dit Morrel quand il eut fini cette première partie de son récit, maintenant que je vous ai dit, monsieur, mon amour et mes espérances, dois-je vous dire nos projets? 

—Oui, fit le vieillard. 

—Eh bien, voilà ce que nous avions résolu.» 

Et alors il raconta tout à Noirtier: comment un cabriolet attendait dans l'enclos, comment il comptait enlever Valentine, la conduire chez sa sœur, l'épouser, et dans une respectueuse attente espérer le pardon de M. de Villefort. 

«Non, dit Noirtier. 

—Non? reprit Morrel, ce n'est pas ainsi qu'il faut faire? 

—Non. 

—Ainsi ce projet n'a point votre assentiment? 

—Non. 

—Eh bien, il y a un autre moyen», dit Morrel. 

Le regard interrogateur du vieillard demanda:  

«Lequel?» 

«J'irai, continua Maximilien, j'irai trouver M. Franz d'Épinay, je suis heureux de pouvoir vous dire cela en l'absence de Mlle de Villefort, et je me conduirai avec lui de manière à le forcer d'être un galant homme. 

Le regard de Noirtier continua d'interroger. 

«Ce que je ferai? 

—Oui. 

—Le voici. Je l'irai trouver, comme je vous le disais, je lui raconterai les liens qui m'unissent à Mlle Valentine; si c'est un homme délicat, il prouvera sa délicatesse en renonçant de lui-même à la main de sa fiancée, et mon amitié et mon dévouement lui sont de cette heure acquis jusqu'à la mort; s'il refuse, soit que l'intérêt le pousse, soit qu'un ridicule orgueil le fasse persister, après lui avoir prouvé qu'il contraindrait ma femme, que Valentine m'aime et ne peut aimer un autre que moi, je me battrai avec lui, en lui donnant tous les avantages, et je le tuerai ou il me tuera; si je le tue, il n'épousera pas Valentine; s'il me tue, je serai bien sûr que Valentine ne l'épousera pas.» 

Noirtier considérait avec un plaisir indicible cette noble et sincère physionomie sur laquelle se peignaient tous les sentiments que sa langue exprimait, en y ajoutant par l'expression d'un beau visage tout ce que la couleur ajoute à un dessin solide et vrai. 

Cependant, lorsque Morrel eut fini de parler, Noirtier ferma les yeux à plusieurs reprises, ce qui était, on le sait, sa manière de dire non. 

«Non? dit Morrel. Ainsi vous désapprouvez ce second projet, comme vous avez déjà désapprouvé le premier? 

—Oui, je le désapprouve, fit le vieillard. 

—Mais que faire alors, monsieur? demanda Morrel. Les dernières paroles de Mme de Saint-Méran ont été pour que le mariage de sa petite-fille ne se fît point attendre: dois-je laisser les choses s'accomplir?» 

Noirtier resta immobile. 

«Oui, je comprends, dit Morrel, je dois attendre. 

—Oui. 

—Mais tout délai nous perdra, monsieur, reprit le jeune homme. Seule, Valentine est sans force, et on la contraindra comme un enfant. Entré ici miraculeusement pour savoir ce qui s'y passe, admis miraculeusement devant vous, je ne puis raisonnablement espérer que ces bonnes chances se renouvellent. Croyez-moi, il n'y a que l'un ou l'autre des deux partis que je vous propose, pardonnez cette vanité à ma jeunesse, qui soit le bon; dites-moi celui des deux que vous préférez: autorisez-vous Mlle Valentine à se confier à mon honneur? 

—Non. 

—Préférez-vous que j'aille trouver M. d'Épinay? 

—Non. 

—Mais, mon Dieu! de qui nous viendra le secours que nous attendons du Ciel?» 

Le vieillard sourit des yeux comme il avait l'habitude de sourire quand on lui parlait du ciel. Il était toujours resté un peu d'athéisme dans les idées du vieux jacobin. 

«Du hasard? reprit Morrel. 

—Non. 

—De vous? 

—Oui. 

—De vous? 

—Oui, répéta le vieillard. 

—Vous comprenez bien ce que je vous demande, monsieur? Excusez mon insistance, car ma vie est dans votre réponse: notre salut nous viendra de vous? 

—Oui. 

—Vous en êtes sûr? 

—Oui.  

—Vous en répondez? 

—Oui.» 

Et il y avait dans le regard qui donnait cette affirmation une telle fermeté, qu'il n'y avait pas moyen de douter de la volonté, sinon de la puissance. 

«Oh! merci, monsieur, merci cent fois! Mais comment, à moins qu'un miracle du Seigneur ne vous rende la parole, le geste, le mouvement, comment pourrez-vous, vous, enchaîné dans ce fauteuil, vous, muet et immobile, comment pourrez-vous vous opposer à ce mariage?» 

Un sourire éclaira le visage du vieillard, sourire étrange que celui des yeux sur un visage immobile. 

«Ainsi, je dois attendre? demanda le jeune homme. 

—Oui. Mais le contrat?» 

Le même sourire reparut. 

«Voulez-vous donc me dire qu'il ne sera pas signé? 

—Oui, dit Noirtier. 

—Ainsi le contrat ne sera même pas signé! s'écria Morrel. Oh! pardonnez, monsieur! à l'annonce d'un grand bonheur, il est bien permis de douter; le contrat ne sera pas signé? 

—Non», dit le paralytique. 

Malgré cette assurance, Morrel hésitait à croire. Cette promesse d'un vieillard impotent était si étrange, qu'au lieu de venir d'une force de volonté, elle pouvait émaner d'un affaiblissement des organes; n'est-il pas naturel que l'insensé qui ignore sa folie prétende réaliser des choses au-dessus de sa puissance? Le faible parle des fardeaux qu'il soulève, le timide des géants qu'il affronte, le pauvre des trésors qu'il manie, le plus humble paysan, au compte de son orgueil, s'appelle Jupiter. 

Soit que Noirtier eût compris l'indécision du jeune homme, soit qu'il n'ajoutât pas complètement foi à la docilité qu'il avait montrée, il le regarda fixement. 

«Que voulez-vous, monsieur? demanda Morrel, que je vous renouvelle ma promesse de ne rien faire?» 

Le regard de Noirtier demeura fixe et ferme, comme pour dire qu'une promesse ne lui suffisait pas; puis il passa du visage à la main. 

«Voulez-vous que je jure, monsieur? demanda Maximilien. 

—Oui, fit le paralytique avec la même solennité, je le veux.» 

Morrel comprit que le vieillard attachait une grande importance à ce serment. 

Il étendit la main. 

«Sur mon honneur, dit-il, je vous jure d'attendre ce que vous aurez décidé pour agir contre M. d'Épinay. 

—Bien, fit des yeux le vieillard. 

—Maintenant, monsieur, demanda Morrel, ordonnez-vous que je me retire? 

—Oui. 

—Sans revoir Mlle Valentine? 

—Oui.» 

Morrel fit signe qu'il était prêt à obéir. 

«Maintenant, continua Morrel, permettez-vous monsieur, que votre fils vous embrasse comme l'a fait tout à l'heure votre fille!» 

Il n'y avait pas à se tromper à l'expression des yeux de Noirtier. 

Le jeune homme posa sur le front du vieillard ses lèvres au même endroit où la jeune fille avait posé les siennes. 

Puis il salua une seconde fois le vieillard et sortit. 

Sur le carré il trouva le vieux serviteur, prévenu par Valentine; celui-ci attendait Morrel, et le guida par les détours d'un corridor sombre qui conduisait à une petite porte donnant sur le jardin. 

Arrivé là, Morrel gagna la grille, par la charmille, il fut en un instant au haut du mur, et par son échelle en une seconde, il fut dans l'enclos à la luzerne, où son cabriolet l'attendait toujours. 

Il y remonta, et brisé par tant d'émotions, mais le cœur plus libre, il rentra vers minuit rue Meslay, se jeta sur son lit et dormit comme s'il eût été plongé dans une profonde ivresse. 