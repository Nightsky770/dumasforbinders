\chapter{Le télégraphe} 

\lettrine{M}{.} et Mme de Villefort apprirent, en rentrant chez eux, que M. le comte de Monte-Cristo, qui était venu pour leur faire visite, avait été introduit dans le salon, où il les attendait; Mme de Villefort, trop émotionnée pour entrer ainsi tout à coup, passa par sa chambre à coucher, tandis que le procureur du roi, plus sûr de lui-même, s'avança directement vers le salon. 

Mais si maître qu'il fût de ses sensations, si bien qu'il sût composer son visage, M. de Villefort ne put si bien écarter le nuage de son front que le comte, dont le sourire brillait radieux, ne remarquât cet air sombre et rêveur.  

«Oh! mon Dieu! dit Monte-Cristo après les premiers compliments, qu'avez-vous donc, monsieur de Villefort? et suis-je arrivé au moment où vous dressiez quelque accusation un peu trop capitale?» 

Villefort essaya de sourire. 

«Non, monsieur le comte, dit-il, il n'y a d'autre victime ici que moi. C'est moi qui perds mon procès, et c'est le hasard, l'entêtement, la folie qui a lancé le réquisitoire. 

—Que voulez-vous dire? demanda Monte-Cristo avec un intérêt parfaitement joué. Vous est-il, en réalité, arrivé quelque malheur grave? 

—Oh! monsieur le comte, dit Villefort avec un calme plein d'amertume, cela ne vaut pas la peine d'en parler; presque rien, une simple perte d'argent. 

—En effet, répondit Monte-Cristo, une perte d'argent est peu de chose avec une fortune comme celle que vous possédez et avec un esprit philosophique et élevé comme l'est le vôtre. 

—Aussi, répondit Villefort, n'est-ce point la question d'argent qui me préoccupe, quoique, après tout, neuf cent mille francs vaillent bien un regret, ou tout au moins un mouvement de dépit. Mais je me blesse surtout de cette disposition du sort, du hasard, de la fatalité, je ne sais comment nommer la puissance qui dirige le coup qui me frappe et qui renverse mes espérances de fortune et détruit peut-être l'avenir de ma fille par le caprice d'un vieillard tombé en enfance. 

—Eh! mon Dieu! qu'est-ce donc? s'écria le comte. Neuf cent mille francs, avez-vous dit? Mais, en vérité, comme vous le dites, la somme mérite d'être regrettée, même par un philosophe. Et qui vous donne ce chagrin. 

—Mon père, dont je vous ai parlé. 

—M. Noirtier; vraiment! Mais vous m'aviez dit, ce me semble, qu'il était en paralysie complète, et que toutes ses facultés étaient anéanties? 

—Oui, ses facultés physiques, car il ne peut pas remuer, il ne peut point parler, et avec tout cela, cependant, il pense, il veut, il agit comme vous voyez. Je le quitte il y a cinq minutes et, dans ce moment, il est occupé à dicter un testament à deux notaires. 

—Mais alors il a parlé? 

—Il a fait mieux, il s'est fait comprendre. 

—Comment cela? 

—À l'aide du regard; ses yeux ont continué de vivre, et vous voyez, ils tuent. 

—Mon ami, dit Mme de Villefort qui venait d'entrer à son tour, peut-être vous exagérez-vous la situation? 

—Madame\dots» dit le comte en s'inclinant. 

Mme de Villefort salua avec son plus gracieux sourire. 

«Mais que me dit donc là M. de Villefort? demanda Monte-Cristo; et quelle disgrâce incompréhensible?\dots 

—Incompréhensible, c'est le mot! reprit le procureur du roi en haussant les épaules, un caprice de vieillard! 

—Et il n'y a pas moyen de le faire revenir sur cette décision? 

—Si fait, dit Mme de Villefort; et il dépend même de mon mari que ce testament, au lieu d'être fait au détriment de Valentine, soit fait au contraire en sa faveur.» 

Le comte, voyant que les deux époux commençaient à parler par paraboles, prit l'air distrait, et regarda avec l'attention la plus profonde et l'approbation la plus marquée Édouard qui versait de l'encre dans l'abreuvoir des oiseaux. 

«Ma chère, dit Villefort répondant à sa femme, vous savez que j'aime peu me poser chez moi en patriarche, et que je n'ai jamais cru que le sort de l'univers dépendît d'un signe de ma tête. Cependant il importe que mes décisions soient respectées dans ma famille, et que la folie d'un vieillard et le caprice d'un enfant ne renversent pas un projet arrêté dans mon esprit depuis de longues années. Le baron d'Épinay était mon ami, vous le savez, et une alliance avec son fils était des plus convenables. 

—Vous croyez, dit Mme de Villefort, que Valentine est d'accord avec lui?\dots En effet, elle a toujours été opposée à ce mariage, et je ne serais pas étonnée que tout ce que nous venons de voir et d'entendre ne soit l'exécution d'un plan concerté entre eux.  

—Madame, dit Villefort, on ne renonce pas ainsi croyez-moi, à une fortune de neuf cent mille francs. 

—Elle renoncerait au monde, monsieur, puisqu'il y a un an elle voulait entrer dans un couvent. 

—N'importe, reprit de Villefort, je dis que ce mariage doit se faire, madame! 

—Malgré la volonté de votre père? dit Mme de Villefort, attaquant une autre corde: c'est bien grave!» 

Monte-Cristo faisait semblant de ne point écouter, et ne perdait point un mot de ce qui se disait. 

«Madame, reprit Villefort, je puis dire que j'ai toujours respecté mon père, parce qu'au sentiment naturel de la descendance se joignait chez moi la conscience de sa supériorité morale; parce qu'enfin un père est sacré à deux titres, sacré comme notre créateur, sacré comme notre maître; mais aujourd'hui je dois renoncer à reconnaître une intelligence dans le vieillard qui, sur un simple souvenir de haine pour le père, poursuit ainsi le fils; il serait donc ridicule à moi de conformer ma conduite à ses caprices. Je continuerai d'avoir le plus grand respect pour M. Noirtier; je subirai sans me plaindre la punition pécuniaire qu'il m'inflige, mais je resterai immuable dans ma volonté, et le monde appréciera de quel côté était la saine raison. En conséquence, je marierai ma fille au baron Franz d'Épinay, parce que ce mariage est, à mon sens, bon et honorable, et qu'en définitive je veux marier ma fille à qui me plaît. 

—Eh quoi! dit le comte, dont le procureur du roi avait constamment sollicité l'approbation du regard; eh quoi! M. Noirtier déshérite, dites-vous, Mlle Valentine, parce qu'elle va épouser M. le baron Franz d'Épinay? 

—Eh! mon Dieu! oui! oui, monsieur; voilà la raison, dit Villefort en haussant les épaules. 

—La raison visible du moins, ajouta Mme de Villefort. 

—La raison réelle, madame. Croyez-moi, je connais mon père. 

—Conçoit-on cela? répondit la jeune femme; en quoi, je vous le demande, M. d'Épinay déplaît-il plus qu'un autre à M. Noirtier? 

—En effet, dit le comte, j'ai connu M. Franz d'Épinay, le fils du général de Quesnel, n'est-ce pas, qui a été fait baron d'Épinay par le roi Charles X? 

—Justement, reprit Villefort. 

—Eh bien, mais c'est un jeune homme charmant, ce me semble! 

—Aussi n'est-ce qu'un prétexte, j'en suis certaine, dit Mme de Villefort; les vieillards sont tyrans de leurs affections; M. Noirtier ne veut pas que sa petite-fille se marie. 

—Mais, dit Monte-Cristo, ne connaissez-vous pas une cause à cette haine? 

—Eh! mon Dieu! qui peut savoir? 

—Quelque antipathie politique peut-être? 

—En effet, mon père et le père de M. d'Épinay ont vécu dans des temps orageux dont je n'ai vu que les derniers jours, dit Villefort. 

—Votre père n'était-il pas bonapartiste? demanda Monte-Cristo. Je crois me rappeler que vous m'avez dit quelque chose comme cela. 

—Mon père a été jacobin avant toutes choses, reprit Villefort, emporté par son émotion hors des bornes de la prudence, et la robe de sénateur que Napoléon lui avait jetée sur les épaules ne faisait que déguiser le vieil homme, mais sans l'avoir changé. Quand mon père conspirait, ce n'était pas pour l'Empereur, c'était contre les Bourbons; car mon père avait cela de terrible en lui, qu'il n'a jamais combattu pour les utopies irréalisables, mais pour les choses possibles, et qu'il a appliqué à la réussite de ces choses possibles ces terribles théories de la Montagne, qui ne reculaient devant aucun moyen. 

—Eh bien, dit Monte-Cristo, voyez-vous, c'est cela, M. Noirtier et M. d'Épinay se seront rencontrés sur le sol de la politique. M. le général d'Épinay, quoique ayant servi sous Napoléon, n'avait-il pas au fond du cœur gardé des sentiments royalistes, et n'est-ce pas le même qui fut assassiné un soir sortant d'un club napoléonien, où on l'avait attiré dans l'espérance de trouver en lui un frère?» 

Villefort regarda le comte presque avec terreur. 

«Est-ce que je me trompe? dit Monte-Cristo.  

—Non pas, monsieur, dit Mme de Villefort, et c'est bien cela, au contraire; et c'est justement à cause de ce que vous venez de dire que, pour voir s'éteindre de vieilles haines, M. de Villefort avait eu l'idée de faire aimer deux enfants dont les pères s'étaient haïs. 

—Idée sublime! dit Monte-Cristo, idée pleine de charité et à laquelle le monde devait applaudir. En effet, c'était beau de voir Mlle Noirtier de Villefort s'appeler Mme Franz d'Épinay.» 

Villefort tressaillit et regarda Monte-Cristo comme s'il eût voulu lire au fond de son cœur l'intention qui avait dicté les paroles qu'il venait de prononcer. 

Mais le comte garda le bienveillant sourire stéréotypé sur ses lèvres; et cette fois encore, malgré la profondeur de son regard, le procureur du roi ne vit pas au-delà de l'épiderme. 

«Aussi, reprit Villefort, quoique ce soit un grand malheur pour Valentine que de perdre la fortune de son grand-père, je ne crois pas cependant que pour cela le mariage manque; je ne crois pas que M. d'Épinay recule devant cet échec pécuniaire; il verra que je vaux peut-être mieux que la somme, moi qui la sacrifie au désir de lui tenir ma parole; il calculera que Valentine d'ailleurs, est riche du bien de sa mère, administré par M. et Mme de Saint-Méran, ses aïeuls maternels, qui la chérissent tous deux tendrement. 

—Et qui valent bien qu'on les aime et qu'on les soigne comme Valentine a fait pour M. Noirtier, dit Mme de Villefort; d'ailleurs, ils vont venir à Paris dans un mois au plus, et Valentine, après un tel affront, sera dispensée de s'enterrer comme elle l'a fait jusqu'ici auprès de M. Noirtier.» 

Le comte écoutait avec complaisance la voix discordante de ces amours-propres blessés et de ces intérêts meurtris. 

«Mais il me semble, dit Monte-Cristo après un instant de silence, et je vous demande pardon d'avance de ce que je vais dire, il me semble que si M. Noirtier déshérite Mlle de Villefort, coupable de se vouloir marier avec un jeune homme dont il a détesté le père, il n'a pas le même tort à reprocher à ce cher Édouard. 

—N'est-ce pas, monsieur? s'écria Mme de Villefort avec une intonation impossible à décrire: n'est-ce pas que c'est injuste, odieusement injuste? Ce pauvre Édouard, il est aussi bien le petit-fils de M. Noirtier que Valentine, et cependant si Valentine n'avait pas dû épouser M. Franz, M. Noirtier lui laissait tout son bien; et de plus, enfin, Édouard porte le nom de la famille, ce qui n'empêche pas que, même en supposant que Valentine soit effectivement déshéritée par son grand-père, elle sera encore trois fois plus riche que lui.» 

Ce coup porté, le comte écouta et ne parla plus. 

«Tenez, reprit Villefort, tenez, monsieur le comte, cessons, je vous prie, de nous entretenir de ces misères de famille, oui c'est vrai, ma fortune va grossir le revenu des pauvres, qui sont aujourd'hui les véritables riches. Oui, mon père m'aura frustré d'un espoir légitime, et cela sans raison; mais, moi, j'aurai agi comme un homme de sens, comme un homme de cœur. M. d'Épinay, à qui j'avais promis le revenu de cette somme, le recevra, dussé-je m'imposer les plus cruelles privations. 

—Cependant, reprit Mme de Villefort, revenant à la seule idée qui murmurât sans cesse au fond de son cœur, peut-être vaudrait-il mieux que l'on confiât cette mésaventure à M. d'Épinay, et qu'il rendît lui-même sa parole. 

—Oh! ce serait un grand malheur! s'écria Villefort. 

—Un grand malheur? répéta Monte-Cristo. 

—Sans doute, reprit Villefort en se radoucissant; un mariage manqué, même pour des raisons d'argent jette de la défaveur sur une jeune fille; puis, d'anciens bruits, que je voulais éteindre, reprendraient de la consistance. Mais non, il n'en sera rien. M. d'Épinay, s'il est honnête homme, se verra encore plus engagé par l'exhérédation de Valentine qu'auparavant; autrement il agirait donc dans un simple but d'avarice: non, c'est impossible. 

—Je pense comme M. de Villefort, dit Monte-Cristo en fixant son regard sur Mme de Villefort; et si j'étais assez de ses amis pour me permettre de lui donner un conseil, je l'inviterais, puisque M. d'Épinay va revenir, à ce que l'on m'a dit du moins, à nouer cette affaire si fortement qu'elle ne se pût dénouer; j'engagerais enfin une partie dont l'issue doit être si honorable pour M. de Villefort.» 

Ce dernier se leva, transporté d'une joie visible, tandis que sa femme pâlissait légèrement. 

«Bien, dit-il, voilà tout ce que je demandais et je me prévaudrai de l'opinion d'un conseiller tel que vous, dit-il en tendant la main à Monte-Cristo. Ainsi donc que tout le monde ici considère ce qui arrive aujourd'hui comme non avenu; il n'y a rien de changé à nos projets.  

—Monsieur, dit le comte, le monde tout injuste qu'il est, vous saura, je vous en réponds, gré de votre résolution; vos amis en seront fiers et M. d'Épinay, dût-il prendre Mlle de Villefort sans dot, ce qui ne saurait être, sera charmé d'entrer dans une famille où l'on sait s'élever à la hauteur de tels sacrifices pour tenir sa parole et remplir son devoir.» 

En disant ces mots, le comte s'était levé et s'apprêtait à partir. 

«Vous nous quittez, monsieur le comte? dit Mme de Villefort. 

—J'y suis forcé, madame, je venais seulement vous rappeler votre promesse pour samedi. 

—Craigniez-vous que nous ne l'oubliassions? 

—Vous êtes trop bonne, madame; mais M. de Villefort a de si graves et parfois de si urgentes occupations\dots. 

—Mon mari a donné sa parole, monsieur, dit Mme de Villefort, vous venez de voir qu'il la tient quand il a tout à perdre, à plus forte raison quand il a tout à gagner. 

—Et, demanda Villefort, est-ce à votre maison des Champs-Élysées que la réunion a lieu? 

—Non pas, dit Monte-Cristo, et c'est ce qui rend encore votre dévouement plus méritoire: c'est à la campagne. 

—À la campagne? 

—Oui. 

—Et où cela? près de Paris, n'est-ce pas? 

—Aux portes, à une demi-heure de la barrière, à Auteuil. 

—À Auteuil! s'écria Villefort. Ah! c'est vrai, madame m'a dit que vous demeuriez à Auteuil, puisque c'est chez vous qu'elle a été transportée. Et à quel endroit d'Auteuil? 

—Rue de la Fontaine! 

—Rue de la Fontaine! reprit Villefort d'une voix étranglée; et à quel numéro? 

—Au n°28. 

—Mais, s'écria Villefort, c'est donc à vous que l'on a vendu la maison de M. de Saint-Méran? 

—M. de Saint-Méran? demanda Monte-Cristo. Cette maison appartenait-elle donc à M. de Saint-Méran? 

—Oui, reprit Mme de Villefort, et croyez-vous une chose, monsieur le comte? 

—Laquelle? 

—Vous trouvez cette maison jolie, n'est-ce pas? 

—Charmante.  

—Eh bien, mon mari n'a jamais voulu l'habiter. 

—Oh! reprit Monte-Cristo, en vérité, monsieur, c'est une prévention dont je ne me rends pas compte. 

—Je n'aime pas Auteuil, monsieur, répondit le procureur du roi, en faisant un effort sur lui-même. 

—Mais je ne serai pas assez malheureux, je l'espère, dit avec inquiétude Monte-Cristo, pour que cette antipathie me prive du bonheur de vous recevoir? 

—Non, monsieur le comte\dots j'espère bien\dots croyez que je ferai tout ce que je pourrai, balbutia Villefort. 

—Oh! répondit Monte-Cristo, je n'admets pas d'excuse. Samedi, à six heures, je vous attends, et si vous ne veniez pas, je croirais, que sais-je, moi? qu'il y a sur cette maison inhabitée depuis plus de vingt ans quelque lugubre tradition, quelque sanglante légende. 

—J'irai, monsieur le comte, j'irai, dit vivement Villefort. 

—Merci, dit Monte-Cristo. Maintenant il faut que vous me permettiez de prendre congé de vous. 

—En effet, vous avez dit que vous étiez forcé de nous quitter, monsieur le comte, dit Mme de Villefort, et vous alliez même, je crois, nous dire pour quoi faire, quand vous vous êtes interrompu pour passer à une autre idée. 

—En vérité, madame, dit Monte-Cristo, je ne sais si j'oserai vous dire où je vais. 

—Bah! dites toujours. 

—Je vais, en véritable badaud que je suis, visiter une chose qui m'a bien souvent fait rêver des heures entières. 

—Laquelle? 

—Un télégraphe. Ma foi tant pis, voilà le mot lâché. 

—Un télégraphe! répéta Mme de Villefort. 

—Eh mon Dieu, oui, un télégraphe. J'ai vu parfois au bout d'un chemin, sur un tertre, par un beau soleil, se lever ces bras noirs et pliants pareils aux pattes d'un immense coléoptère, et jamais ce ne fut sans émotion, je vous jure, car je pensais que ces signes bizarres fendant l'air avec précision, et portant à trois cents lieues la volonté inconnue d'un homme assis devant une table, à un autre homme assis à l'extrémité de la ligne devant une autre table, se dessinaient sur le gris du nuage ou sur l'azur du ciel, par la seule force du vouloir de ce chef tout-puissant: je croyais alors aux génies, aux sylphes, aux gnomes, aux pouvoirs occultes enfin, et je riais. Or, jamais l'envie ne m'était venue de voir de près ces gros insectes au ventre blanc, aux pattes noires et maigres, car je craignais de trouver sous leurs ailes de pierre le petit génie humain, bien gourmé, bien pédant, bien bourré de science, de cabale ou de sorcellerie. Mais voilà qu'un beau matin j'ai appris que le moteur de chaque télégraphe était un pauvre diable d'employé à douze cents francs par an, occupé tout le jour à regarder, non pas le ciel comme l'astronome, non pas l'eau comme le pêcheur, non pas le paysage comme un cerveau vide, mais bien l'insecte au ventre blanc, aux pattes noires, son correspondant, placé à quelque quatre ou cinq lieues de lui. Alors je me suis senti pris d'un désir curieux de voir de près cette chrysalide vivante et d'assister à la comédie que du fond de sa coque elle donne à cette autre chrysalide, en tirant les uns après les autres quelques bouts de ficelle. 

—Et vous allez là? 

—J'y vais. 

—À quel télégraphe? À celui du ministère de l'Intérieur ou de l'Observatoire? 

—Oh! non pas, je trouverais là des gens qui voudraient me forcer de comprendre des choses que je veux ignorer, et qui m'expliqueraient malgré moi un mystère qu'ils ne connaissent pas. Peste! je veux garder les illusions que j'ai encore sur les insectes; c'est bien assez d'avoir déjà perdu celles que j'avais sur les hommes. Je n'irai donc ni au télégraphe du ministère de l'Intérieur, ni au télégraphe de l'Observatoire. Ce qu'il me faut, c'est le télégraphe en plein champ, pour y trouver le pur bonhomme pétrifié dans sa tour. 

—Vous êtes un singulier grand seigneur, dit Villefort. 

—Quelle ligne me conseillez-vous d'étudier? 

—Mais la plus occupée à cette heure. 

—Bon! celle d'Espagne, alors?  

—Justement. Voulez-vous une lettre du ministre pour qu'on vous explique\dots. 

—Mais non, dit Monte-Cristo, puisque je vous dis, au contraire, que je n'y veux rien comprendre. Du moment où j'y comprendrai quelque chose, il n'y aura plus de télégraphe, il n'y aura plus qu'un signe de M. Duchâtel ou de M. de Montalivet, transmis au préfet de Bayonne et travesti en deux mots grecs: Τηλε, γραφετω. C'est la bête aux pattes noires et le mot effrayant que je veux conserver dans toute leur pureté et dans toute ma vénération. 

—Allez donc, car dans deux heures il fera nuit, et vous ne verrez plus rien. 

—Diable, vous m'effrayez. Quel est le plus proche? Sur la route de Bayonne? 

—Oui, va pour la route de Bayonne. C'est celui de Châtillon. 

—Et après celui de Châtillon? 

—Celui de la tour de Montlhéry, je crois. 

—Merci, au revoir! Samedi je vous raconterai mes impressions.» 

À la porte, le comte se trouva avec les deux notaires qui venaient de déshériter Valentine, et qui se retiraient enchantés d'avoir fait un acte qui ne pouvait manquer de leur faire grand honneur.  