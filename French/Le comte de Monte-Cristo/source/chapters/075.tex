\chapter{Le procès-verbal}

\lettrine{N}{oirtier} attendait, vêtu de noir et installé dans son fauteuil. 

\zz
Lorsque les trois personnes qu'il comptait voir venir furent entrées, il regarda la porte, que son valet de chambre ferma aussitôt. 

«Faites attention, dit Villefort bas à Valentine qui ne pouvait celer sa joie, que si M. Noirtier veut vous communiquer des choses qui empêchent votre mariage, je vous défends de le comprendre.» 

Valentine rougit, mais ne répondit pas. 

Villefort s'approcha de Noirtier: 

«Voici M. Franz d'Épinay, lui dit-il, vous l'avez mandé, monsieur, et il se rend à vos désirs. Sans doute nous souhaitons cette entrevue depuis longtemps, et je serai charmé qu'elle vous prouve combien votre opposition au mariage de Valentine était peu fondée.» 

Noirtier ne répondit que par un regard qui fit courir le frisson dans les veines de Villefort. 

Il fit de l'œil signe à Valentine de s'approcher. 

En un moment, grâce aux moyens dont elle avait l'habitude de se servir dans les conversations avec son grand-père, elle eut trouvé le mot \textit{clef}. 

Alors elle consulta le regard du paralytique, qui se fixa sur le tiroir d'un petit meuble entre les deux fenêtres. 

Elle ouvrit le tiroir et trouva effectivement une clef. Quand elle eut cette clef et que le vieillard lui eut fait signe que c'était bien celle-là qu'il demandait, les yeux du paralytique se dirigèrent vers un vieux secrétaire oublié depuis bien des années, et qui ne renfermait, croyait-on, que des paperasses inutiles. 

«Faut-il que j'ouvre le secrétaire? demanda Valentine. 

—Oui, fit le vieillard. 

—Faut-il que j'ouvre les tiroirs? 

—Oui. 

—Ceux des côtés?  

—Non. 

—Celui du milieu? 

—Oui.» 

Valentine l'ouvrit et en tira une liasse. 

«Est-ce là ce que vous désirez, bon père? dit-elle. 

—Non.» 

Elle tira successivement tous les autres papiers, jusqu'à ce qu'il ne restât plus rien absolument dans le tiroir. 

«Mais le tiroir est vide maintenant», dit-elle. 

Les yeux de Noirtier étaient fixés sur le dictionnaire. 

«Oui, bon père, je vous comprends», dit la jeune fille. 

Et elle répéta l'une après l'autre, chaque lettre de l'alphabet; à l'S Noirtier l'arrêta. 

Elle ouvrit le dictionnaire, et chercha jusqu'au mot \textit{secret}. 

«Ah! il y a un secret? dit Valentine. 

—Oui, fit Noirtier. 

—Et qui connaît ce secret?»  

Noirtier regarda la porte par laquelle était sorti le domestique. 

«Barrois? dit-elle. 

—Oui, fit Noirtier. 

—Faut-il que je l'appelle? 

—Oui.» 

Valentine alla à la porte et appela Barrois. 

Pendant ce temps, la sueur de l'impatience ruisselait sur le front de Villefort, et Franz demeurait stupéfait d'étonnement. 

Le vieux serviteur parut. 

«Barrois, dit Valentine, mon grand-père m'a commandé de prendre la clef dans cette console, d'ouvrir ce secrétaire et de tirer ce tiroir; maintenant il y a un secret à ce tiroir, il paraît que vous le connaissez, ouvrez-le.» 

Barrois regarda le vieillard. 

«Obéissez», dit l'œil intelligent de Noirtier. 

Barrois obéit; un double fond s'ouvrit et présenta une liasse de papiers nouée avec un ruban noir. 

«Est-ce cela que vous désirez, monsieur? demanda Barrois. 

—Oui, fit Noirtier. 

—À qui faut-il remettre ces papiers? à M. de Villefort? 

—Non. 

—À Mlle Valentine? 

—Non. 

—À M. Franz d'Épinay? 

—Oui.» 

Franz, étonné, fit un pas en avant. 

«À moi, monsieur? dit-il. 

—Oui.» 

Franz reçut les papiers des mains de Barrois, et jetant les yeux sur la couverture, il lut: 

«Pour être déposé, après ma mort, chez mon ami le général Durand, qui lui-même en mourant léguera ce paquet à son fils, avec injonction de le conserver comme renfermant un papier de la plus grande importance.» 

«Eh bien, monsieur, demanda Franz, que voulez-vous que je fasse de ce papier? 

—Que vous le conserviez cacheté comme il est, sans doute, dit le procureur du roi. 

—Non, non, répondit vivement Noirtier. 

—Vous désirez peut-être que monsieur le lise? demanda Valentine. 

—Oui, répondit le vieillard. 

—Vous entendez, monsieur le baron, mon grand-père vous prie de lire ce papier, dit Valentine. 

—Alors asseyons-nous, fit Villefort avec impatience, car cela durera quelque temps. 

—Asseyez-vous», fit l'œil du vieillard. 

Villefort s'assit, mais Valentine resta debout à côté de son père appuyée à côté de son fauteuil, et Franz debout devant lui. Il tenait le mystérieux papier à la main. 

«Lisez», dirent les yeux du vieillard. 

Franz défit l'enveloppe, et un grand silence se fit dans la chambre. Au milieu de ce silence il lut: 

«\textit{Extrait des procès-verbaux d'une séance du club bonapartiste de la rue Saint-Jacques, tenue le 5 février 1815}.» 

Franz s'arrêta. 

«Le 5 février 1815! C'est le jour où mon père a été assassiné!»  

Valentine et Villefort restèrent muets; l'œil seul du vieillard dit clairement: «Continuez.» 

«Mais c'est en sortant de ce club, continua Franz, que mon père a disparu!» 

Le regard de Noirtier continua de dire: «Lisez.» 

Il reprit: 

«Les soussignés Louis-Jacques Beaurepaire, lieutenant-colonel d'artillerie, Étienne Duchampy, général de brigade, et Claude Lecharpal, directeur des eaux et forêts, 

«Déclarent que, le 4 février 1815, une lettre arriva de l'île d'Elbe, qui recommandait à la bienveillance et à la confiance des membres du club bonapartiste le général Flavien de Quesnel, qui, ayant servi l'Empereur depuis 1804 jusqu'en 1815, devait être tout dévoué à la dynastie napoléonienne, malgré le titre de baron que Louis XVIII venait d'attacher à sa terre d'Épinay. 

«En conséquence, un billet fut adressé au général de Quesnel, qui le priait d'assister à la séance du lendemain. Le billet n'indiquait ni la rue ni le numéro de la maison où devait se tenir la réunion; il ne portait aucune signature, mais il annonçait au général que, s'il voulait se tenir prêt, on le viendrait prendre à neuf heures du soir. 

«Les séances avaient lieu de neuf heures du soir à minuit. 

«À neuf heures, le président du club se présenta chez le général, le général était prêt; le président lui dit qu'une des conditions de son introduction était qu'il ignorerait éternellement le lieu de la réunion, et qu'il se laisserait bander les yeux en jurant de ne point chercher à soulever le bandeau. 

«Le général de Quesnel accepta la condition, et promit sur l'honneur de ne pas chercher à voir où on le conduirait. 

«Le général avait fait préparer sa voiture; mais le président lui dit qu'il était impossible que l'on s'en servît, attendu que ce n'était pas la peine qu'on bandât les yeux du maître si le cocher demeurait les yeux ouverts et reconnaissait les rues par lesquelles on passerait. 

«—Comment faire alors? demanda le général. 

«—J'ai ma voiture, dit le président. 

«—Êtes-vous donc si sûr de votre cocher, que vous lui confiez un secret que vous jugez imprudent de dire au mien? 

«—Notre cocher est un membre du club, dit le président; nous serons conduits par un conseiller d'État. 

«—Alors, dit en riant le général, nous courons un autre risque, celui de verser.» 

«Nous consignons cette plaisanterie comme preuve que le général n'a pas été le moins du monde forcé d'assister à la séance, et qu'il est venu de son plein gré.» 

«Une fois monté dans la voiture, le président rappela au général la promesse faite par lui de se laisser bander les yeux. Le général ne mit aucune opposition à cette formalité: un foulard, préparé à cet effet dans la voiture, fit l'affaire. 

«Pendant la route, le président crut s'apercevoir que le général cherchait à regarder sous son bandeau: il lui rappela son serment. 

«—Ah! c'est vrai», dit le général. 

«La voiture s'arrêta devant une allée de la rue Saint-Jacques. Le général descendit en s'appuyant au bras du président, dont il ignorait la dignité, et qu'il prenait pour un simple membre du club, on traversa l'allée, on monta un étage, et l'on entra dans la chambre des délibérations. 

«La séance était commencée. Les membres du club prévenus de l'espèce de présentation qui devait avoir lieu ce soir-là, se trouvaient au grand complet. Arrivé au milieu de la salle, le général fut invité à ôter son bandeau. Il se rendit aussitôt à l'invitation, et parut fort étonné de trouver un si grand nombre de figures de connaissance dans une société dont il n'avait pas même soupçonné l'existence jusqu'alors. 

«On l'interrogea sur ses sentiments, mais il se contenta de répondre que les lettres de l'île d'Elbe avaient dû les faire connaître\dots.» 

Franz s'interrompit. 

«Mon père était royaliste, dit-il; on n'avait pas besoin de l'interroger sur ses sentiments, ils étaient connus.  

—Et de là, dit Villefort, venait ma liaison avec votre père, mon cher monsieur Franz; on se lie facilement quand on partage les mêmes opinions.» 

«Lisez», continua de dire l'œil du vieillard. 

Franz continua: 

«Le président prit alors la parole pour engager le général à s'exprimer plus explicitement; mais M. de Quesnel répondit qu'il désirait avant tout savoir ce que l'on désirait de lui. 

«Il fut alors donné communication au général de cette même lettre de l'île d'Elbe qui le recommandait au club comme un homme sur le concours duquel on pouvait compter. Un paragraphe tout entier exposait le retour probable de l'île d'Elbe, et promettait une nouvelle lettre et de plus amples détails à l'arrivée du \textit{Pharaon}, bâtiment appartenant à l'armateur Morrel, de Marseille, et dont le capitaine était à l'entière dévotion de l'empereur. 

«Pendant toute cette lecture, le général, sur lequel on avait cru pouvoir compter comme sur un frère, donna au contraire des signes de mécontentement et de répugnance visibles. 

«La lecture terminée, il demeura silencieux et le sourcil froncé. 

«—Eh bien, demanda le président, que dites-vous de cette lettre, monsieur le général? 

«—Je dis qu'il y a bien peu de temps, répondit-il, qu'on a prêté serment au roi Louis XVIII, pour le violer déjà au bénéfice de l'ex-empereur.» 

«Cette fois la réponse était trop claire pour que l'on pût se tromper à ses sentiments. 

«—Général, dit le président, il n'y a pas plus pour nous de roi Louis XVIII qu'il n'y a d'ex-empereur. Il n'y a que Sa Majesté l'Empereur et roi, éloigné depuis dix mois de la France, son État, par la violence et la trahison. 

«—Pardon, messieurs, dit le général; il se peut qu'il n'y ait pas pour vous de roi Louis XVIII, mais il y en a un pour moi: attendu qu'il m'a fait baron et maréchal de camp, et que je n'oublierai jamais que c'est à son heureux retour en France que je dois ces deux titres. 

«—Monsieur, dit le président du ton le plus sérieux et en se levant, prenez garde à ce que vous dites; vos paroles nous démontrent clairement que l'on s'est trompé sur votre compte à l'île d'Elbe et qu'on nous a trompés. La communication qui vous a été faite tient à la confiance qu'on avait en vous, et par conséquent à un sentiment qui vous honore. Maintenant nous étions dans l'erreur: un titre et un grade vous ont rallié au nouveau gouvernement que nous voulons renverser. Nous ne vous contraindrons pas à nous prêter votre concours; nous n'enrôlerons personne contre sa conscience et sa volonté; mais nous vous contraindrons à agir comme un galant homme, même au cas où vous n'y seriez point disposé. 

«—Vous appelez être un galant homme connaître votre conspiration et ne pas la révéler! J'appelle cela être votre complice, moi. Vous voyez que je suis encore plus franc que vous\dots.  

«Ah! mon père, dit Franz, s'interrompant, je comprends maintenant pourquoi ils t'ont assassiné.» 

Valentine ne put s'empêcher de jeter un regard sur Franz; le jeune homme était vraiment beau dans son enthousiasme filial. 

Villefort se promenait de long en large derrière lui. 

Noirtier suivait des yeux l'expression de chacun, et conservait son attitude digne et sévère. 

Franz revint au manuscrit et continua: 

«—Monsieur, dit le président, on vous a prié de vous rendre au sein de l'assemblée, on ne vous y a point traîné de force; on vous a proposé de vous bander les yeux, vous avez accepté. Quand vous avez accédé à cette double demande vous saviez parfaitement que nous ne nous occupions pas d'assurer le trône de Louis XVIII, sans quoi nous n'eussions pas pris tant de soin de nous cacher à la police. Maintenant, vous le comprenez, il serait trop commode de mettre un masque à l'aide duquel on surprend le secret des gens, et de n'avoir ensuite qu'à ôter ce masque pour perdre ceux qui se sont fiés à vous. Non, non, vous allez d'abord dire franchement si vous êtes pour le roi de hasard qui règne en ce moment, ou pour S. M. l'Empereur. 

«—Je suis royaliste, répondit le général; j'ai fait serment à Louis XVIII, je tiendrai mon serment. 

«Ces mots furent suivis d'un murmure général, et l'on put voir, par les regards d'un grand nombre des membres du club, qu'ils agitaient la question de faire repentir M. d'Épinay de ces imprudentes paroles. 

«Le président se leva de nouveau et imposa silence. 

«—Monsieur, lui dit-il, vous êtes un homme trop grave et trop sensé pour ne pas comprendre les conséquences de la situation où nous nous trouvons les uns en face des autres, et votre franchise même nous dicte les conditions qu'il nous reste à vous faire: vous allez donc jurer sur l'honneur de ne rien révéler de ce que vous avez entendu. 

«Le général porta la main à son épée et s'écria: 

«—Si vous parlez d'honneur, commencez par ne pas méconnaître ses lois, et n'imposez rien par la violence. 

«—Et vous, monsieur, continua le président avec un calme plus terrible peut-être que la colère du général, ne touchez pas à votre épée, c'est un conseil que je vous donne. 

«Le général tourna autour de lui des regards qui décelaient un commencement d'inquiétude. Cependant il ne fléchit pas encore; au contraire, rappelant toute sa force: 

«—Je ne jurerai pas, dit-il. 

«—Alors, monsieur, vous mourrez, répondit tranquillement le président. 

«M. d'Épinay devint fort pâle: il regarda une seconde fois tout autour de lui; plusieurs membres du club chuchotaient et cherchaient des armes sous leurs manteaux. 

«—Général, dit le président, soyez tranquille; vous êtes parmi des gens d'honneur qui essaieront de tous les moyens de vous convaincre avant de se porter contre vous à la dernière extrémité, mais aussi, vous l'avez dit, vous êtes parmi des conspirateurs, vous tenez notre secret, il faut nous le rendre.» 

«Un silence plein de signification suivit ces paroles et comme le général ne répondait rien: 

«—Fermez les portes, dit le président aux huissiers. 

«Le même silence de mort succéda à ses paroles. 

«Alors le général s'avança, et faisant un violent effort sur lui-même: 

«—J'ai un fils, dit-il, et je dois songer à lui en me trouvant parmi des assassins. 

«—Général, dit avec noblesse le chef de l'assemblée, un seul homme a toujours le droit d'en insulter cinquante: c'est le privilège de la faiblesse. Seulement il a tort d'user de ce droit. Croyez-moi, général, jurez et ne nous insultez pas. 

«Le général, encore une fois dompté par cette supériorité du chef de l'assemblée, hésita un instant; mais enfin, s'avançant jusqu'au bureau du président: 

«—Quelle est la formule? demanda-t-il. 

«—La voici: 

«—Je jure sur l'honneur de ne jamais révéler à qui que ce soit au monde ce que j'ai vu et entendu le 5 février 1815, entre neuf et dix heures du soir, et je déclare mériter la mort si je viole mon serment. 

«Le général parut éprouver un frémissement nerveux qui l'empêcha de répondre pendant quelques secondes; enfin, surmontant une répugnance manifeste, il prononça le serment exigé, mais d'une voix si basse qu'à peine on l'entendit: aussi plusieurs membres exigèrent-ils qu'il le répétât à voix plus haute et plus distincte, ce qui fut fait. 

«—Maintenant, je désire me retirer, dit le général; suis-je enfin libre? 

«Le président se leva, désigna trois membres de l'assemblée pour l'accompagner, et monta en voiture avec le général, après lui avoir bandé les yeux. Au nombre de ces trois membres était le cocher qui l'avait amené. 

«Les autres membres du club se séparèrent en silence. 

«—Où voulez-vous que nous vous reconduisions? demanda le président. 

«—Partout où je pourrai être délivré de votre présence, répondit M. d'Épinay. 

«—Monsieur, reprit alors le président, prenez garde, vous n'êtes plus dans l'assemblée, vous n'avez plus affaire qu'à des hommes isolés; ne les insultez pas si vous ne voulez pas être rendu responsable de l'insulte. 

«Mais au lieu de comprendre ce langage, M. d'Épinay répondit: 

«—Vous êtes toujours aussi brave dans votre voiture que dans votre club, par la raison, monsieur, que quatre hommes sont toujours plus forts qu'un seul.» 

«Le président fit arrêter la voiture. 

«On était juste à l'entrée du quai des Ormes, où se trouve l'escalier qui descend à la rivière. 

«—Pourquoi faites-vous arrêter ici? demanda M. d'Épinay. 

«—Parce que, monsieur, dit le président, vous avez insulté un homme, et que cet homme ne veut pas faire un pas de plus sans vous demander loyalement séparation. 

«—Encore une manière d'assassiner, dit le général en haussant les épaules. 

«—Pas de bruit, répondit le président, si vous ne voulez pas que je vous regarde vous-même comme un de ces hommes que vous désigniez tout à l'heure, c'est-à-dire comme un lâche qui prend sa faiblesse pour bouclier. Vous êtes seul, un seul vous répondra; vous avez une épée au côté, j'en ai une dans cette canne; vous n'avez pas de témoin, un de ces messieurs sera le vôtre. Maintenant, si cela vous convient, vous pouvez ôter votre bandeau. 

«Le général arracha à l'instant même le mouchoir qu'il avait sur les yeux. 

«—Enfin, dit-il, je vais donc savoir à qui j'ai affaire.» 

«On ouvrit la voiture: les quatre hommes descendirent\dots.» 

Franz s'interrompit encore une fois. Il essuya une sueur froide qui coulait sur son front, il y avait quelque chose d'effrayant à voir le fils, tremblant et pâle, lisant tout haut les détails, ignorés jusqu'alors, de la mort de son père. 

Valentine joignait les mains comme si elle eût été en prières. 

Noirtier regardait Villefort avec une expression presque sublime de mépris et d'orgueil. 

Franz continua: 

«On était, comme nous l'avons dit, au 5 février. Depuis trois jours il gelait à cinq ou six degrés; l'escalier était tout raide de glaçons, le général était gros et grand, le président lui offrit le côté de la rampe pour descendre. 

«Les deux témoins suivaient par-derrière. 

«Il faisait une nuit sombre, le terrain de l'escalier à la rivière était humide de neige et de givre, on voyait l'eau s'écouler, noire, profonde et charriant quelques glaçons. 

«Un des témoins alla chercher une lanterne dans un bateau de charbon, et à la lueur de cette lanterne on examina les armes.  

«L'épée du président, qui était simplement, comme il l'avait dit, une épée qu'il portait dans une canne, était plus courte que celle de son adversaire, et n'avait pas de garde. 

«Le général d'Épinay proposa de tirer au sort les deux épées: mais le président répondit que c'était lui qui avait provoqué, et qu'en provoquant il avait prétendu que chacun se servit de ses armes. 

«Les témoins essayèrent d'insister; le président leur imposa silence. 

«On posa la lanterne à terre: les deux adversaires se mirent de chaque côté; le combat commença. 

«La lumière faisait des deux épées deux éclairs. Quant aux hommes, à peine si on les apercevait, tant l'ombre était épaisse. 

«M. le général passait pour une des meilleures lames de l'armée. Mais il fut pressé si vivement dès les premières bottes, qu'il rompit; en rompant il tomba. 

«Les témoins le crurent tué; mais son adversaire, qui savait ne l'avoir point touché, lui offrit la main pour l'aider à se relever. Cette circonstance, au lieu de le calmer, irrita le général, qui fondit à son tour sur son adversaire. 

«Mais son adversaire ne rompit pas d'une semelle, le recevant sur son épée. Trois fois le général recula, se trouvant trop engagé, et revint à la charge. 

«À la troisième fois, il tomba encore.  

«On crut qu'il glissait comme la première fois; cependant les témoins, voyant qu'il ne se relevait pas, s'approchèrent de lui et tentèrent de le remettre sur ses pieds; mais celui qui l'avait pris à bras-le-corps sentit sous sa main une chaleur humide. C'était du sang. 

«Le général, qui était à peu près évanoui, reprit ses sens. 

«—Ah! dit-il, on m'a dépêché quelque spadassin, quelque maître d'armes du régiment. 

«Le président, sans répondre, s'approcha de celui des deux témoins qui tenait la lanterne et, relevant sa manche, il montra son bras percé de deux coups d'épée; puis, ouvrant son habit et déboutonnant son gilet, il fit voir son flanc entamé par une troisième blessure. 

«Cependant il n'avait pas même poussé un soupir. 

«Le général d'Épinay entra en agonie et expira cinq minutes après\dots.» 


Franz lut ces derniers mots d'une voix si étranglée, qu'à peine on put les entendre; et après les avoir lus il s'arrêta, passant sa main sur ses yeux comme pour en chasser un nuage. 

Mais, après un instant de silence, il continua: 


\begin{mail}{}{}
Le président remonta l'escalier, après avoir repoussé son épée dans sa canne; une trace de sang marquait son chemin dans la neige. Il n'était pas encore en haut de l'escalier, qu'il entendit un clapotement sourd dans l'eau: c'était le corps du général que les témoins venaient de précipiter dans la rivière après avoir constaté la mort. 

Le général a donc succombé dans un duel loyal, et non dans un guet-apens, comme on pourrait le dire. 

En foi de quoi nous avons signé le présent pour établir la vérité des faits, de peur qu'un moment n'arrive où quelqu'un des acteurs de cette scène terrible ne se trouve accusé de meurtre avec préméditation ou de forfaiture aux lois de l'honneur. 

\closeletter[Signé]{Beauregard, Duchampy et Lecharpel.}
\end{mail}

Quand Franz eut terminé cette lecture si terrible pour un fils, quand Valentine, pâle d'émotion, eut essuyé une larme, quand Villefort, tremblant et blotti dans un coin, eut essayé de conjurer l'orage par des regards suppliants adressés au vieillard implacable: 

«Monsieur, dit d'Épinay à Noirtier, puisque vous connaissez cette terrible histoire dans tous ses détails, puisque vous l'avez fait attester par des signatures honorables, puisque enfin vous semblez vous intéresser à moi, quoique votre intérêt ne se soit encore révélé que par la douleur, ne me refusez pas une dernière satisfaction, dites-moi le nom du président du club, que je connaisse enfin celui qui a tué mon pauvre père.» 

Villefort chercha, comme égaré, le bouton de la porte. Valentine, qui avait compris avant tout le monde la réponse du vieillard, et qui souvent avait remarqué sur son avant-bras la trace de deux coups d'épée, recula d'un pas en arrière. 

«Au nom du Ciel! mademoiselle, dit Franz, s'adressant à sa fiancée, joignez-vous à moi, que je sache le nom de cet homme qui m'a fait orphelin à deux ans.» 

Valentine resta immobile et muette. 

«Tenez, monsieur, dit Villefort, croyez-moi, ne prolongez pas cette horrible scène; les noms d'ailleurs ont été cachés à dessein. Mon père lui-même ne connaît pas ce président, et, s'il le connaît, il ne saurait le dire: les noms propres ne se trouvent pas dans le dictionnaire. 

—Oh! malheur! s'écria Franz, le seul espoir qui m'a soutenu pendant toute cette lecture et qui m'a donné la force d'aller jusqu'au bout, c'était de connaître au moins le nom de celui qui a tué mon père! Monsieur! monsieur! s'écria-t-il en se retournant vers Noirtier, au nom du Ciel! faites ce que vous pourrez\dots arrivez, je vous en supplie, à m'indiquer, à me faire comprendre\dots. 

—Oui, répondit Noirtier. 

—Ô mademoiselle, mademoiselle! s'écria Franz, votre grand-père a fait signe qu'il pouvait m'indiquer\dots cet homme\dots. Aidez-moi\dots vous le comprenez\dots prêtez-moi votre concours.» 

Noirtier regarda le dictionnaire. 

Franz le prit avec un tremblement nerveux, et prononça successivement les lettres de l'alphabet jusqu'à l'M. 

À cette lettre, le vieillard fit signe que oui. 

«M!» répéta Franz. 

Le doigt du jeune homme glissa sur les mots; mais, à tous les mots, Noirtier répondait par un signe négatif. Valentine cachait sa tête entre ses mains. Enfin Franz arriva au mot MOI. 

«Oui, fit le vieillard. 

—Vous! s'écria Franz, dont les cheveux se dressèrent sur sa tête; vous, monsieur Noirtier! c'est vous qui avez tué mon père? 

—Oui», répondit Noirtier, en fixant sur le jeune homme un majestueux regard. 

Franz tomba sans force sur un fauteuil. 

Villefort ouvrit la porte et s'enfuit, car l'idée lui venait d'étouffer ce peu d'existence qui restait encore dans le cœur terrible du vieillard. 