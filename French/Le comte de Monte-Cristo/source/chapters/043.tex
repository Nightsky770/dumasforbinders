\chapter{La maison d'Auteuil}

\lettrine{M}{onte-Cristo} avait remarqué qu'en descendant le perron, Bertuccio s'était signé à la manière des Corses, c'est-à-dire en coupant l'air en croix avec le pouce, et qu'en prenant sa place dans la voiture il avait marmotté tout bas une courte prière. Tout autre qu'un homme curieux eût eu pitié de la singulière répugnance manifestée par le digne intendant pour la promenade méditée \textit{extra muros} par le comte; mais, à ce qu'il paraît, celui-ci était trop curieux pour dispenser Bertuccio de ce petit voyage. 

En vingt minutes on fut à Auteuil. L'émotion de l'intendant avait été toujours croissant. En entrant dans le village, Bertuccio, rencogné dans l'angle de la voiture, commença à examiner avec une émotion fiévreuse chacune des maisons devant lesquelles on passait. 

«Vous ferez arrêter rue de la Fontaine, au n° 28», dit le comte en fixant impitoyablement son regard sur l'intendant, auquel il donnait cet ordre. 

La sueur monta au visage de Bertuccio; cependant il obéit, et, se penchant en dehors de la voiture, il cria au cocher:  

«Rue de la Fontaine, n° 28.» 

Ce n° 28 était situé à l'extrémité du village. Pendant le voyage, la nuit était venue, ou plutôt un nuage noir tout chargé d'électricité donnait à ces ténèbres prématurées l'apparence et la solennité d'un épisode dramatique. 

La voiture s'arrêta et le valet de pied se précipita à la portière, qu'il ouvrit. 

«Eh bien, dit le comte, vous ne descendez pas, monsieur Bertuccio? vous restez donc dans la voiture alors? Mais à quoi diable songez-vous donc ce soir?» 

Bertuccio se précipita par la portière et présenta son épaule au comte qui, cette fois, s'appuya dessus et descendit un à un les trois degrés du marchepied. 

«Frappez, dit le comte, et annoncez-moi.» 

Bertuccio frappa, la porte s'ouvrit et le concierge parut. 

«Qu'est-ce que c'est? demanda-t-il. 

—C'est votre nouveau maître, brave homme», dit le valet de pied. 

Et il tendit au concierge le billet de reconnaissance donné par le notaire.  

«La maison est donc vendue? demanda le concierge, et c'est monsieur qui vient l'habiter? 

—Oui, mon ami, dit le comte, et je tâcherai que vous n'ayez pas à regretter votre ancien maître. 

—Oh! monsieur, dit le concierge, je n'aurai pas à le regretter beaucoup, car nous le voyons bien rarement; il y a plus de cinq ans qu'il n'est venu, et il a, ma foi! bien fait de vendre une maison qui ne lui rapportait absolument rien. 

—Et comment se nommait votre ancien maître? demanda Monte-Cristo. 

—M. le marquis de Saint-Méran; ah! il n'a pas vendu la maison ce qu'elle lui a coûté, j'en suis sûr. 

—Le marquis de Saint-Méran! reprit Monte-Cristo; mais il me semble que ce nom ne m'est pas inconnu, dit le comte; le marquis de Saint-Méran\dots. 

Et il parut chercher. 

«Un vieux gentilhomme, continua le concierge, un fidèle serviteur des Bourbons, il avait une fille unique qu'il avait mariée à M. de Villefort, qui a été procureur du roi à Nîmes et ensuite à Versailles.» 

Monte-Cristo jeta un regard qui rencontra Bertuccio plus livide que le mur contre lequel il s'appuyait pour ne pas tomber. 

«Et cette fille n'est-elle pas morte? demanda Monte-Cristo; il me semble que j'ai entendu dire cela. 

—Oui, monsieur, il y a vingt et un ans, et depuis ce temps-là nous n'avons pas revu trois fois le pauvre cher marquis. 

—Merci, merci, dit Monte-Cristo, jugeant à la prostration de l'intendant qu'il ne pouvait tendre davantage cette corde sans risquer de la briser; merci! Donnez-moi de la lumière, brave homme. 

—Accompagnerai-je monsieur?  

—Non, c'est inutile, Bertuccio m'éclairera. 

Et Monte-Cristo accompagna ces paroles du don de deux pièces d'or qui soulevèrent une explosion de bénédictions et de soupirs. 

«Ah! monsieur! dit le concierge après avoir cherché inutilement sur le rebord de la cheminée et sur les planches y attenantes, c'est que je n'ai pas de bougies ici. 

—Prenez une des lanternes de la voiture, Bertuccio, et montrez-moi les appartements», dit le comte. 

L'intendant obéit sans observation, mais il était facile à voir, au tremblement de la main qui tenait la lanterne, ce qu'il lui en coûtait pour obéir. 

On parcourut un rez-de-chaussée assez vaste; un premier étage composé d'un salon, d'une salle de bain et de deux chambres à coucher. Par une de ces chambres à coucher, on arrivait à un escalier tournant dont l'extrémité aboutissait au jardin. 

«Tiens, voilà un escalier de dégagement, dit le comte, c'est assez commode. Éclairez-moi, monsieur Bertuccio; passez devant, et allons où cet escalier nous conduira. 

—Monsieur, dit Bertuccio, il va au jardin. 

—Et comment savez-vous cela, je vous prie?  

—C'est-à-dire qu'il doit y aller. 

—Eh bien, assurons-nous-en.» 

Bertuccio poussa un soupir et marcha devant. L'escalier aboutissait effectivement au jardin. 

À la porte extérieure l'intendant s'arrêta. 

«Allons donc, monsieur Bertuccio!» dit le comte. 

Mais celui auquel il s'adressait était abasourdi, stupide, anéanti. Ses yeux égarés cherchaient tout autour de lui comme les traces d'un passé terrible, et de ses mains crispées il semblait essayer de repousser des souvenirs affreux. 

«Eh bien? insista le comte. 

—Non! non! s'écria Bertuccio en posant la main à l'angle du mur intérieur; non, monsieur, je n'irai pas plus loin, c'est impossible! 

—Qu'est-ce à dire? articula la voix irrésistible de Monte-Cristo. 

—Mais vous voyez bien, monsieur, s'écria l'intendant, que cela n'est point naturel; qu'ayant une maison à acheter à Paris, vous l'achetiez justement à Auteuil, et que l'achetant à Auteuil, cette maison soit le n° 28 de la rue de la Fontaine! Ah! pourquoi ne vous ai-je pas tout dit là-bas, monseigneur. Vous n'auriez certes pas exigé que je vinsse. J'espérais que la maison de monsieur le comte serait une autre maison que celle-ci. Comme s'il n'y avait d'autre maison à Auteuil que celle de l'assassinat! 

—Oh! oh! fit Monte-Cristo s'arrêtant tout à coup, quel vilain mot venez-vous de prononcer là! Diable d'homme! Corse enraciné! toujours des mystères ou des superstitions! Voyons, prenez cette lanterne et visitons le jardin; avec moi vous n'aurez pas peur, j'espère!» 

Bertuccio ramassa la lanterne et obéit. 

La porte en s'ouvrant, découvrit un ciel blafard dans lequel la lune s'efforçait vainement de lutter contre une mer de nuages qui la couvraient de leurs flots sombres qu'elle illuminait un instant, et qui allaient ensuite se perdre, plus sombres encore, dans les profondeurs de l'infini. 

L'intendant voulut appuyer sur la gauche. 

«Non pas, monsieur, dit Monte-Cristo, à quoi bon suivre les allées? voici une belle pelouse, allons devant nous.» 

Bertuccio essuya la sueur qui coulait de son front, mais obéit; cependant, il continuait de prendre à gauche. Monte-Cristo, au contraire, appuyait à droite. Arrivé près d'un massif d'arbres, il s'arrêta.  

L'intendant n'y put tenir. 

«Éloignez-vous, monsieur! s'écria-t-il, éloignez-vous, je vous en supplie, vous êtes justement à la place! 

—À quelle place? 

—À la place même où il est tombé. 

—Mon cher monsieur Bertuccio, dit Monte-Cristo en riant, revenez à vous, je vous y engage; nous ne sommes pas ici à Sartène ou à Corte. Ceci n'est point un maquis, mais un jardin anglais, mal entretenu, j'en conviens, mais qu'il ne faut pas calomnier pour cela.  

—Monsieur, ne restez pas là! ne restez pas là! je vous en supplie. 

—Je crois que vous devenez fou, maître Bertuccio, dit froidement le comte; si cela est, prévenez-moi car je vous ferai enfermer dans quelque maison de santé avant qu'il arrive un malheur. 

—Hélas! Excellence, dit Bertuccio en secouant la tête et en joignant les mains avec une attitude qui eût fait rire le comte, si des pensées d'un intérêt supérieur ne l'eussent captivé en ce moment et rendu fort attentif aux moindres expansions de cette conscience timorée. Hélas! Excellence, le malheur est arrivé. 

—Monsieur Bertuccio, dit le comte, je suis fort aise de vous dire que, tout en gesticulant, vous vous tordez les bras, et que vous roulez des yeux comme un possédé du corps duquel le diable ne veut pas sortir; or, j'ai presque toujours remarqué que le diable le plus entêté à rester à son poste, c'est un secret. Je vous savais Corse, je vous savais sombre et ruminant toujours quelque vieille histoire de vendetta, et je vous passais cela en Italie, parce qu'en Italie ces sortes de choses sont de mise, mais en France on trouve généralement l'assassinat de fort mauvais goût: il y a des gendarmes qui s'en occupent, des juges qui le condamnent et des échafauds qui le vengent.» 

Bertuccio joignit les mains et, comme en exécutant ces différentes évolutions il ne quittait point sa lanterne, la lumière éclaira son visage bouleversé.  

Monte-Cristo l'examina du même œil qu'à Rome il avait examiné le supplice d'Andrea; puis, d'un ton de voix qui fit courir un nouveau frisson par le corps du pauvre intendant: 

«L'abbé Busoni m'avait donc menti, dit-il, lorsque après son voyage en France, en 1829, il vous envoya vers moi, muni d'une lettre de recommandation dans laquelle il me recommandait vos précieuses qualités. Eh bien, je vais écrire à l'abbé; je le rendrai responsable de son protégé, et je saurai sans doute ce que c'est que toute cette affaire d'assassinat. Seulement, je vous préviens, monsieur Bertuccio, que lorsque je vis dans un pays, j'ai l'habitude de me conformer à ses lois, et que je n'ai pas envie de me brouiller pour vous avec la justice de France.  

—Oh! ne faites pas cela, Excellence, je vous ai servi fidèlement, n'est-ce pas? s'écria Bertuccio au désespoir, j'ai toujours été honnête homme, et j'ai même, le plus que j'ai pu, fait de bonnes actions. 

—Je ne dis pas non, reprit le comte, mais pourquoi diable êtes-vous agité de la sorte? C'est mauvais signe: une conscience pure n'amène pas tant de pâleur sur les joues, tant de fièvre dans les mains d'un homme\dots. 

—Mais, monsieur le comte, reprit en hésitant Bertuccio, ne m'avez-vous pas dit vous-même que M. l'abbé Busoni, qui a entendu ma confession dans les prisons de Nîmes, vous avait prévenu, en m'envoyant chez vous, que j'avais un lourd reproche à me faire?  

—Oui, mais comme il vous adressait à moi en me disant que vous feriez un excellent intendant, j'ai cru que vous aviez volé, voilà tout! 

—Oh! monsieur le comte! fit Bertuccio avec mépris. 

—Ou que, comme vous étiez Corse, vous n'aviez pu résister au désir de faire une peau, comme on dit dans le pays par antiphrase, quand au contraire on en défait une. 

—Eh bien, oui, monseigneur, oui, mon bon seigneur, c'est cela! s'écria Bertuccio en se jetant aux genoux du comte; oui, c'est une vengeance, je le jure, une simple vengeance.  

—Je comprends, mais ce que je ne comprends pas, c'est que ce soit cette maison justement qui vous galvanise à ce point. 

—Mais, monseigneur, n'est-ce pas bien naturel, reprit Bertuccio, puisque c'est dans cette maison que la vengeance s'est accomplie? 

—Quoi! ma maison! 

—Oh! monseigneur, elle n'était pas encore à vous, répondit naïvement Bertuccio. 

—Mais à qui donc était-elle? à M. le marquis de Saint-Méran, nous a dit, je crois, le concierge. Que diable aviez-vous donc à vous venger du marquis de Saint-Méran?  

—Oh! ce n'était pas de lui, monseigneur, c'était d'un autre. 

—Voilà une étrange rencontre, dit Monte-Cristo paraissant céder à ses réflexions, que vous vous trouviez comme cela par hasard, sans préparation aucune, dans une maison où s'est passée une scène qui vous donne de si affreux remords. 

—Monseigneur, dit l'intendant, c'est la fatalité qui amène tout cela, j'en suis bien sûr: d'abord, vous achetez une maison juste à Auteuil, cette maison est celle où j'ai commis un assassinat; vous descendez au jardin juste par l'escalier où il est descendu; vous vous arrêtez juste à l'endroit où il reçut le coup; à deux pas, sous ce platane, était la fosse où il venait d'enterrer l'enfant: tout cela n'est pas du hasard, non, car en ce cas le hasard ressemblerait trop à la Providence. 

—Eh bien, voyons, monsieur le Corse, supposons que ce soit la Providence; je suppose toujours tout ce qu'on veut, moi; d'ailleurs aux esprits malades il faut faire des concessions. Voyons, rappelez vos esprits et racontez-moi cela. 

—Je ne l'ai jamais raconté qu'une fois, et c'était à l'abbé Busoni. De pareilles choses, ajouta Bertuccio en secouant la tête, ne se disent que sous le sceau de la confession. 

—Alors, mon cher Bertuccio, dit le comte, vous trouverez bon que je vous renvoie à votre confesseur; vous vous ferez avec lui chartreux ou bernardin, et vous causerez de vos secrets. Mais, moi, j'ai peur d'un hôte effrayé par de pareils fantômes; je n'aime point que mes gens n'osent point se promener le soir dans mon jardin. Puis, je l'avoue, je serais peu curieux de quelque visite de commissaire de police; car, apprenez ceci, maître Bertuccio: en Italie, on ne paie la justice que si elle se tait, mais en France on ne la paie au contraire que quand elle parle. Peste! je vous croyais bien un peu Corse, beaucoup contrebandier, fort habile intendant, mais je vois que vous avez encore d'autres cordes à votre arc. Vous n'êtes plus à moi, monsieur Bertuccio. 

—Oh! monseigneur! monseigneur! s'écria l'intendant frappé de terreur à cette menace; oh! s'il ne tient qu'à cela que je demeure à votre service, je parlerai, je dirai tout; et si je vous quitte, eh bien, alors ce sera pour marcher à l'échafaud.  

—C'est différent alors, dit Monte-Cristo; mais si vous voulez mentir, réfléchissez-y: mieux vaut que vous ne parliez pas du tout. 

—Non, monsieur, je vous le jure sur le salut de mon âme, je vous dirai tout! car l'abbé Busoni lui-même n'a su qu'une partie de mon secret. Mais d'abord, je vous en supplie, éloignez-vous de ce platane; tenez, la lune va blanchir ce nuage, et là, placé comme vous l'êtes, enveloppé de ce manteau qui me cache votre taille et qui ressemble à celui de M. de Villefort!\dots 

—Comment! s'écria Monte-Cristo, c'est M. de Villefort\dots. 

—Votre excellence le connaît?  

—L'ancien procureur du roi de Nîmes? 

—Oui. 

—Qui avait épousé la fille du marquis de Saint-Méran? 

—Oui. 

—Et qui avait dans le barreau la réputation du plus honnête, du plus sévère, du plus rigide magistrat. 

—Eh bien, monsieur, s'écria Bertuccio, cet homme à la réputation irréprochable\dots. 

—Oui. 

—C'était un infâme. 

—Bah! dit Monte-Cristo, impossible. 

—Cela est pourtant comme je vous le dis. 

—Ah! vraiment! dit Monte-Cristo, et vous en avez la preuve? 

—Je l'avais du moins. 

—Et vous l'avez perdue, maladroit? 

—Oui; mais en cherchant bien on peut la retrouver. 

—En vérité! dit le comte, contez-moi cela, monsieur Bertuccio, car cela commence véritablement à m'intéresser.» 

Et le comte, en chantonnant un petit air de la \textit{Lucia}, alla s'asseoir sur un banc, tandis que Bertuccio le suivait en rappelant ses souvenirs. 

Bertuccio resta debout devant lui. 