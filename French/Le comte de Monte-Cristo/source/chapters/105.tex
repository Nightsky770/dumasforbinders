\chapter{Le cimetière du Père-Lachaise} 

\lettrine{M}{.} de Boville avait, en effet, rencontré le convoi funèbre qui conduisait Valentine à sa dernière demeure. 

\zz
Le temps était sombre et nuageux; un vent tiède encore, mais déjà mortel pour les feuilles jaunies, les arrachait aux branches peu à peu dépouillées et les faisait tourbillonner sur la foule immense qui encombrait les boulevards. 

M. de Villefort, parisien pur, regardait le cimetière du Père-Lachaise comme le seul digne de recevoir la dépouille mortelle d'une famille parisienne; les autres lui paraissaient des cimetières de campagne, des hôtels garnis de la mort. Au Père-Lachaise seulement un trépassé de bonne compagnie pouvait être logé chez lui. 

Il avait acheté là, comme nous l'avons vu, la concession à perpétuité sur laquelle s'élevait le monument peuplé si promptement par tous les membres de sa première famille. 

On lisait sur le fronton du mausolée: FAMILLE SAINT-MÉRAN ET VILLEFORT; car tel avait été le dernier vœu de la pauvre Renée, mère de Valentine. 

C'était donc vers le Père-Lachaise que s'acheminait le pompeux cortège parti du faubourg Saint-Honoré. On traversa tout Paris, on prit le faubourg du Temple, puis les boulevards extérieurs jusqu'au cimetière. Plus de cinquante voitures de maîtres suivaient vingt voitures de deuil, et, derrière ces cinquante voitures, plus de cinq cents personnes encore marchaient à pied. 

C'étaient presque tous des jeunes gens que la mort de Valentine avait frappés d'un coup de foudre, et qui, malgré la vapeur glaciale du siècle et le prosaïsme de l'époque, subissaient l'influence poétique de cette belle, de cette chaste, de cette adorable jeune fille enlevée en sa fleur. 

À la sortie de Paris, on vit arriver un rapide attelage de quatre chevaux qui s'arrêtèrent soudain en raidissant leurs jarrets nerveux comme des ressorts d'acier: c'était M. de Monte-Cristo. 

Le comte descendit de sa calèche, et vint se mêler à la foule qui suivait à pied le char funéraire. 

Château-Renaud l'aperçut; il descendit aussitôt de son coupé et vint se joindre à lui. Beauchamp quitta de même le cabriolet de remise dans lequel il se trouvait. 

Le comte regardait attentivement par tous les interstices que laissait la foule; il cherchait visiblement quelqu'un. Enfin, il n'y tint pas. 

«Où est Morrel? demanda-t-il. Quelqu'un de vous, messieurs, sait-il où il est? 

—Nous nous sommes déjà fait cette question à la maison mortuaire, dit Château-Renaud; car personne de nous ne l'a aperçu.» 

Le comte se tut, mais continua à regarder autour de lui. 

Enfin on arriva au cimetière. L'œil perçant de Monte-Cristo sonda tout d'un coup les bosquets d'ifs et de pins, et bientôt il perdit toute inquiétude: une ombre avait glissé sous les noires charmilles, et Monte-Cristo venait sans doute de reconnaître ce qu'il cherchait. 

On sait ce que c'est qu'un enterrement dans cette magnifique nécropole: des groupes noirs disséminés dans les blanches allées, le silence du ciel et de la terre, troublé par l'éclat de quelques branches rompues, de quelque haie enfoncée autour d'une tombe; puis le chant mélancolique des prêtres auquel se mêle çà et là un sanglot échappé d'une touffe de fleurs, sous laquelle on voit quelque femme, abîmée et les mains jointes. 

L'ombre qu'avait remarquée Monte-Cristo traversa rapidement le quinconce jeté derrière la tombe d'Héloïse et d'Abélard, vint se placer, avec les valets de la mort, à la tête des chevaux qui traînaient le corps, et du même pas parvint à l'endroit choisi pour la sépulture. 

Chacun regardait quelque chose. 

Monte-Cristo ne regardait que cette ombre à peine remarquée de ceux qui l'avoisinaient. 

Deux fois le comte sortit des rangs pour voir si les mains de cet homme ne cherchaient pas quelque arme cachée sous ses habits. 

Cette ombre, quand le cortège s'arrêta, fut reconnue pour être Morrel, qui, avec sa redingote noire boutonnée jusqu'en haut, son front livide, ses joues creusées, son chapeau froissé par ses mains convulsives, s'était adossé à un arbre situé sur un tertre dominant le mausolée, de manière à ne perdre aucun des détails de la funèbre cérémonie qui allait s'accomplir. 

Tout se passa selon l'usage. Quelques hommes, et comme toujours, c'étaient les moins impressionnés, quelques hommes prononcèrent des discours. Les uns plaignaient cette mort prématurée; les autres s'étendaient sur la douleur de son père; il y en eut d'assez ingénieux pour trouver que cette jeune fille avait plus d'une fois sollicité M. de Villefort pour les coupables sur la tête desquels il tenait suspendu le glaive de la justice; enfin, on épuisa les métaphores fleuries et les périodes douloureuses, en commentant de toute façon les stances de Malherbe à Dupérier. 

Monte-Cristo n'écoutait rien, ne voyait rien, ou plutôt il ne voyait que Morrel, dont le calme et l'immobilité formaient un spectacle effrayant pour celui qui seul pouvait lire ce qui se passait au fond du cœur du jeune officier. 

«Tiens, dit tout à coup Beauchamp à Debray, voilà Morrel! Où diable s'est-il fourré là?» 

Et ils le firent remarquer à Château-Renaud. 

«Comme il est pâle, dit celui-ci en tressaillant. 

—Il a froid, répliqua Debray. 

—Non pas, dit lentement Château-Renaud; je crois, moi, qu'il est ému. C'est un homme très impressionnable que Maximilien. 

—Bah! dit Debray, à peine s'il connaissait Mlle de Villefort. Vous l'avez dit vous-même. 

—C'est vrai. Cependant je me rappelle qu'à ce bal chez Mme de Morcerf il a dansé trois fois avec elle; vous savez, comte, à ce bal où vous produisîtes tant d'effet. 

—Non, je ne sais pas», répondit Monte-Cristo, sans savoir à quoi ni à qui il répondait, occupé qu'il était de surveiller Morrel dont les joues s'animaient, comme il arrive à ceux qui compriment ou retiennent leur respiration. 

«Les discours sont finis: adieu, messieurs», dit brusquement le comte. 

Et il donna le signal du départ en disparaissant, sans que l'on sût par où il était passé. 

La fête mortuaire était terminée, les assistants reprirent le chemin de Paris. 

Château-Renaud seul chercha un instant Morrel des yeux; mais, tandis qu'il avait suivi du regard le comte qui s'éloignait, Morrel avait quitté sa place, et Château-Renaud, après l'avoir cherché vainement, avait suivi Debray et Beauchamp. 

Monte-Cristo s'était jeté dans un taillis, et, caché derrière une large tombe, il guettait jusqu'au moindre mouvement de Morrel, qui peu à peu s'était approché du mausolée abandonné des curieux, puis des ouvriers. 

Morrel regarda autour de lui lentement et vaguement; mais au moment où son regard embrassait la portion du cercle opposée à la sienne, Monte-Cristo se rapprocha encore d'une dizaine de pas sans avoir été vu. 

Le jeune homme s'agenouilla. 

Le comte, le cou tendu, l'œil fixe et dilaté, les jarrets pliés comme pour s'élancer au premier signal, continuait à se rapprocher de Morrel. 

Morrel courba son front jusque sur la pierre, embrassa la grille de ses deux mains, et murmura: 

«Ô Valentine!» 

Le cœur du comte fut brisé par l'explosion de ces deux mots; il fit un pas encore, et frappant sur l'épaule de Morrel: 

«C'est vous, cher ami! dit-il, je vous cherchais.» 

Monte-Cristo s'attendait à un éclat, à des reproches, à des récriminations: il se trompait. 

Morrel se tourna de son côté, et avec l'apparence du calme: 

«Vous voyez, dit-il, je priais!» 

Et son regard scrutateur parcourut le jeune homme des pieds à la tête. 

Après cet examen il parut plus tranquille. 

«Voulez-vous que je vous ramène à Paris? dit-il. 

—Non, merci. 

—Enfin désirez-vous quelque chose? 

—Laissez-moi prier. 

Le comte s'éloigna sans faire une seule objection, mais ce fut pour prendre un nouveau poste, d'où il ne perdait pas un seul geste de Morrel, qui enfin se releva, essuya ses genoux blanchis par la pierre, et reprit le chemin de Paris sans tourner une seule fois la tête. 

Il descendit lentement la rue de la Roquette. 

Le comte, renvoyant sa voiture qui stationnait au Père-Lachaise, le suivit à cent pas. Maximilien traversa le canal, et rentra rue Meslay par les boulevards. 

Cinq minutes après que la porte se fut refermée pour Morrel, elle se rouvrit pour Monte-Cristo. 

Julie était à l'entrée du jardin, où elle regardait, avec la plus profonde attention, maître Peneton, qui, prenant sa profession de jardinier au sérieux, faisait des boutures de rosier du Bengale. 

«Ah! monsieur le comte de Monte-Cristo! s'écria-t-elle avec cette joie que manifestait d'ordinaire chaque membre de la famille, quand Monte-Cristo faisait sa visite dans la rue Meslay. 

—Maximilien vient de rentrer, n'est-ce pas madame? demanda le comte. 

—Je crois l'avoir vu passer, oui, reprit la jeune femme; mais, je vous en prie, appelez Emmanuel. 

—Pardon, madame; mais il faut que je monte à l'instant même chez Maximilien, répliqua Monte-Cristo, j'ai à lui dire quelque chose de la plus haute importance. 

—Allez donc, fit-elle, en l'accompagnant de son charmant sourire jusqu'à ce qu'il eût disparu dans l'escalier. 

Monte-Cristo eut bientôt franchi les deux étages qui séparaient le rez-de-chaussée de l'appartement de Maximilien; parvenu sur le palier, il écouta: nul bruit ne se faisait entendre. 

Comme dans la plupart des anciennes maisons habitées par un seul maître, le palier n'était fermé que par une porte vitrée. 

Seulement, à cette porte vitrée il n'y avait point de clef. Maximilien s'était enfermé en dedans; mais il était impossible de voir au-delà de la porte, un rideau de soie rouge doublant les vitres. 

L'anxiété du comte se traduisit par une vive rougeur, symptôme d'émotion peu ordinaire chez cet homme impassible. 

«Que faire?» murmura-t-il. 

Et il réfléchit un instant. 

«Sonner? reprit-il, oh! non! souvent le bruit d'une sonnette, c'est-à-dire d'une visite, accélère la résolution de ceux qui se trouvent dans la situation où Maximilien doit être en ce moment, et alors au bruit de la sonnette répond un autre bruit.» 

Monte-Cristo frissonna des pieds à la tête, et, comme chez lui la décision avait la rapidité de l'éclair, il frappa un coup de coude dans un des carreaux de la porte vitrée qui vola en éclats; puis il souleva le rideau et vit Morrel qui, devant son bureau, une plume à la main, venait de bondir sur sa chaise, au fracas de la vitre brisée. 

«Ce n'est rien, dit le comte, mille pardons, mon cher ami! j'ai glissé, et en glissant j'ai donné du coude dans votre carreau; puisqu'il est cassé, je vais en profiter pour entrer chez vous; ne vous dérangez pas, ne vous dérangez pas.» 

Et, passant le bras par la vitre brisée, le comte ouvrit la porte. 

Morrel se leva, évidemment contrarié, et vint au-devant de Monte-Cristo, moins pour le recevoir que pour lui barrer le passage. 

«Ma foi, c'est la faute de vos domestiques, dit Monte-Cristo en se frottant le coude, vos parquets sont reluisants comme des miroirs. 

—Vous êtes-vous blessé, monsieur? demanda froidement Morrel. 

—Je ne sais. Mais que faisiez-vous donc là? Vous écriviez? 

—Moi? 

—Vous avez les doigts tachés d'encre. 

—C'est vrai, répondit Morrel, j'écrivais; cela m'arrive quelquefois, tout militaire que je suis.» 

Monte-Cristo fit quelques pas dans l'appartement. Force fut à Maximilien de le laisser passer; mais il le suivit. 

«Vous écriviez? reprit Monte-Cristo avec un regard fatigant de fixité. 

—J'ai déjà eu l'honneur de vous dire que oui», fit Morrel. 

Le comte jeta un regard autour de lui. 

«Vos pistolets à côté de l'écritoire! dit-il en montrant du doigt à Morrel les armes posées sur son bureau. 

—Je pars pour un voyage, répondit Maximilien. 

—Mon ami! dit Monte-Cristo avec une voix d'une douceur infinie. 

—Monsieur! 

—Mon ami, mon cher Maximilien, pas de résolutions extrêmes, je vous en supplie! 

—Moi, des résolutions extrêmes, dit Morrel en haussant les épaules; et en quoi, je vous prie, un voyage est-il une résolution extrême? 

—Maximilien, dit Monte-Cristo, posons chacun de notre côté le masque que nous portons. 

«Maximilien, vous ne m'abusez pas avec ce calme de commande plus que je ne vous abuse, moi, avec ma frivole sollicitude. 

«Vous comprenez bien, n'est-ce pas? que pour avoir fait ce que j'ai fait, pour avoir enfoncé des vitres, violé le secret de la chambre d'un ami, vous comprenez, dis-je, que, pour avoir fait tout cela, il fallait que j'eusse une inquiétude réelle, ou plutôt une conviction terrible. 

«Morrel, vous voulez vous tuer! 

—Bon! dit Morrel tressaillant, où prenez-vous de ces idées-là, monsieur le comte? 

—Je vous dis que vous voulez vous tuer! continua le comte du même son de voix, et en voici la preuve.» 

Et, s'approchant du bureau, il souleva la feuille blanche que le jeune homme avait jetée sur une lettre commencée, et prit la lettre. 

Morrel s'élança pour la lui arracher des mains. Mais Monte-Cristo prévoyait ce mouvement et le prévint en saisissant Maximilien par le poignet et en l'arrêtant comme la chaîne d'acier arrête le ressort au milieu de son évolution. 

«Vous voyez bien que vous vouliez vous tuer! Morrel, dit le comte, c'est écrit! 

—Eh bien, s'écria Morrel, passant sans transition de l'apparence du calme à l'expression de la violence; eh bien, quand cela serait, quand j'aurais décidé de tourner sur moi le canon de ce pistolet, qui m'en empêcherait? 

«Qui aurait le courage de m'en empêcher? 

«Quand je dirai: 

«Toutes mes espérances sont ruinées, mon cœur est brisé, ma vie est éteinte, il n'y a plus que deuil et dégoût autour de moi; la terre est devenue de la cendre; toute voix humaine me déchire; 

«Quand je dirai: 

«C'est pitié que de me laisser mourir, car si vous ne me laissez mourir je perdrai la raison, je deviendrai fou; 

«Voyons, dites, monsieur, quand je dirai cela, quand on verra que je le dis avec les angoisses et les larmes de mon cœur, me répondra-t-on: 

—Vous avez tort?» 

«M'empêchera-t-on de n'être pas le plus malheureux? 

«Dites, monsieur, dites, est-ce vous qui aurez ce courage? 

—Oui, Morrel, dit Monte-Cristo, d'une voix dont le calme contrastait étrangement avec l'exaltation du jeune homme; oui, ce sera moi. 

—Vous! s'écria Morrel avec une expression croissante de colère et de reproche; vous qui m'avez leurré d'un espoir absurde; vous qui m'avez retenu, bercé, endormi par de vaines promesses, lorsque j'eusse pu, par quelque coup d'éclat, par quelque résolution extrême, la sauver, ou du moins la voir mourir dans mes bras; vous qui affectez toutes les ressources de l'intelligence, toutes les puissances de la matière; vous qui jouez ou plutôt qui faites semblant de jouer le rôle de la Providence, et qui n'avez pas même eu le pouvoir de donner du contrepoison à une jeune fille empoisonnée! Ah! en vérité, monsieur, vous me feriez pitié si vous ne me faisiez horreur! 

—Morrel\dots 

—Oui, vous m'avez dit de poser le masque; eh bien, soyez satisfait, je le pose. 

«Oui, quand vous m'avez suivi au cimetière, je vous ai encore répondu, car mon cœur est bon; quand vous êtes entré, je vous ai laissé venir jusqu'ici\dots Mais puisque vous abusez, puisque vous venez me braver jusque dans cette chambre où je m'étais retiré comme dans ma tombe; puisque vous m'apportez une nouvelle torture, à moi qui croyais les avoir épuisées toutes, comte de Monte-Cristo, mon prétendu bienfaiteur, comte de Monte-Cristo, le sauveur universel, soyez satisfait, vous allez voir mourir votre ami!\dots» 

Et Morrel, le rire de la folie sur les lèvres, s'élança une seconde fois vers les pistolets. 

Monte-Cristo, pâle comme un spectre, mais l'œil éblouissant d'éclairs, étendit la main sur les armes, et dit à l'insensé: 

«Et, je vous répète que vous ne vous tuerez pas! 

—Empêchez-m'en donc! répliqua Morrel avec un dernier élan qui, comme le premier, vint se briser contre le bras d'acier du comte. 

—Je vous en empêcherai! 

—Mais qui êtes-vous donc, à la fin, pour vous arroger ce droit tyrannique sur des créatures libres et pensantes! s'écria Maximilien. 

—Qui je suis? répéta Monte-Cristo. 

«Écoutez: 

«Je suis, poursuivit Monte-Cristo, le seul homme au monde qui ait le droit de vous dire: Morrel je ne veux pas que le fils de ton père meure aujourd'hui!» 

Et Monte-Cristo, majestueux, transfiguré, sublime, s'avança les deux bras croisés vers le jeune homme palpitant, qui, vaincu malgré lui par la presque divinité de cet homme, recula d'un pas. 

«Pourquoi parlez-vous de mon père? balbutia-t-il; pourquoi mêler le souvenir de mon père à ce qui m'arrive aujourd'hui? 

—Parce que je suis celui qui a déjà sauvé la vie à ton père, un jour qu'il voulait se tuer comme tu veux te tuer aujourd'hui; parce que je suis l'homme qui a envoyé la bourse à ta jeune sœur et \textit{Le Pharaon} au vieux Morrel; parce que je suis Edmond Dantès, qui te fit jouer, enfant, sur ses genoux!» 

Morrel fit encore un pas en arrière, chancelant, suffoqué, haletant, écrasé; puis ses forces l'abandonnèrent, et avec un grand cri il tomba prosterné aux pieds de Monte-Cristo. 

Puis tout à coup, dans cette admirable nature, il se fit un mouvement de régénération soudaine et complète: il se releva, bondit hors de la chambre, et se précipita dans l'escalier en criant de toute la puissance de sa voix: 

«Julie! Julie! Emmanuel! Emmanuel!» 

Monte-Cristo voulut s'élancer à son tour, mais Maximilien se fût fait tuer plutôt que de quitter les gonds de la porte qu'il repoussait sur le comte. 

Aux cris de Maximilien, Julie, Emmanuel, Peneton et quelques domestiques accoururent épouvantés. 

Morrel les prit par les mains, et rouvrant la porte: 

«À genoux! s'écria-t-il d'une voix étranglée par les sanglots; à genoux! c'est le bienfaiteur, c'est le sauveur de notre père! c'est\dots» 

Il allait dire: 

«C'est Edmond Dantès!» 

Le comte l'arrêta en lui saisissant le bras. 

Julie s'élança sur la main du comte; Emmanuel l'embrassa comme un dieu tutélaire; Morrel tomba pour la seconde fois à genoux, et frappa le parquet de son front. 

Alors l'homme de bronze sentit son cœur se dilater dans sa poitrine, un jet de flamme dévorante jaillit de sa gorge à ses yeux, il inclina la tête et pleura! 

Ce fut dans cette chambre, pendant quelques instants, un concert de larmes et de gémissements sublimes qui dut paraître harmonieux aux anges mêmes les plus chéris du Seigneur! 

Julie fut à peine revenue de l'émotion si profonde qu'elle venait d'éprouver, qu'elle s'élança hors de la chambre, descendit un étage, courut au salon avec une joie enfantine, et souleva le globe de cristal qui protégeait la bourse donnée par l'inconnu des Allées de Meilhan. 

Pendant ce temps, Emmanuel d'une voix entrecoupée disait au comte: 

«Oh! monsieur le comte, comment, nous voyant parler si souvent de notre bienfaiteur inconnu, comment, nous voyant entourer un souvenir de tant de reconnaissance et d'adoration, comment avez-vous attendu jusqu'aujourd'hui pour vous faire connaître? Oh! c'est de la cruauté envers nous, et, j'oserai presque le dire, monsieur le comte, envers vous-même. 

—Écoutez, mon ami, dit le comte, et je puis vous appeler ainsi, car, sans vous en douter, vous êtes mon ami depuis onze ans; la découverte de ce secret a été amenée par un grand événement que vous devez ignorer. 

«Dieu m'est témoin que je désirais l'enfouir pendant toute ma vie au fond de mon âme; votre frère Maximilien me l'a arraché par des violences dont il se repent, j'en suis sûr.» 

Puis, voyant que Maximilien s'était rejeté de côté sur un fauteuil, tout en demeurant néanmoins à genoux: 

«Veillez sur lui, ajouta tout bas Monte-Cristo en pressant d'une façon significative la main d'Emmanuel. 

—Pourquoi cela? demanda le jeune homme étonné. 

—Je ne puis vous le dire; mais veillez sur lui.» 

Emmanuel embrassa la chambre d'un regard circulaire et aperçut les pistolets de Morrel. 

Ses yeux se fixèrent effrayés sur les armes, qu'il désigna à Monte-Cristo en levant lentement le doigt à leur hauteur. 

Monte-Cristo inclina la tête. 

Emmanuel fit un mouvement vers les pistolets. 

«Laissez», dit le comte. 

Puis allant à Morrel il lui prit la main; les mouvements tumultueux qui avaient un instant secoué le cœur du jeune homme avaient fait place à une stupeur profonde. 

Julie remonta, elle tenait à la main la bourse de soie, et deux larmes brillantes et joyeuses roulaient sur ses joues comme deux gouttes de matinale rosée. 

«Voici la réplique, dit-elle; ne croyez pas qu'elle me soit moins chère depuis que le sauveur nous a été révélé. 

—Mon enfant, répondit Monte-Cristo en rougissant, permettez-moi de reprendre cette bourse; depuis que vous connaissez les traits de mon visage, je ne veux être rappelé à votre souvenir que par l'affection que je vous prie de m'accorder. 

—Oh! dit Julie en pressant la bourse sur son cœur, non, non, je vous en supplie, car un jour vous pourriez nous quitter; car un jour malheureusement vous nous quitterez, n'est-ce pas? 

—Vous avez deviné juste, madame, répondit Monte-Cristo en souriant; dans huit jours, j'aurai quitté ce pays, où tant de gens qui avaient mérité la vengeance du Ciel vivaient heureux, tandis que mon père expirait de faim et de douleur.» 

En annonçant son prochain départ, Monte-Cristo tenait ses yeux fixés sur Morrel, et il remarqua que ces mots \textit{j'aurai quitté ce pays} avaient passé sans tirer Morrel de sa léthargie; il comprit que c'était une dernière lutte qu'il lui fallait soutenir avec la douleur de son ami, et prenant les mains de Julie et d'Emmanuel qu'il réunit en les pressant dans les siennes, il leur dit, avec la douce autorité d'un père: 

«Mes bons amis, laissez-moi seul, je vous prie, avec Maximilien.» 

C'était un moyen pour Julie d'emporter cette relique précieuse dont oubliait de reparler Monte-Cristo. Elle entraîna vivement son mari. 

«Laissons-les», dit-elle. 

Le comte resta avec Morrel, qui demeurait immobile comme une statue. 

«Voyons, dit le comte en lui touchant l'épaule avec son doigt de flamme; redeviens-tu enfin un homme, Maximilien? 

—Oui, car je recommence à souffrir.» 

Le front du comte se plissa, livré qu'il paraissait être à une sombre hésitation. 

«Maximilien! Maximilien! dit-il, ces idées où tu plonges sont indignes d'un chrétien. 

—Oh! tranquillisez-vous, ami, dit Morrel en relevant la tête et en montrant au comte un sourire empreint d'une ineffable tristesse, ce n'est plus moi qui chercherai la mort. 

—Ainsi, dit Monte-Cristo, plus d'armes, plus de désespoir. 

—Non, car j'ai mieux, pour me guérir de ma douleur, que le canon d'un pistolet ou la pointe d'un couteau. 

—Pauvre fou\dots! qu'avez-vous donc? 

—J'ai ma douleur elle-même qui me tuera. 

—Ami, dit Monte-Cristo avec une mélancolie égale à la sienne, écoutez-moi: 

«Un jour, dans un moment de désespoir égal au tien, puisqu'il amenait une résolution semblable, j'ai comme toi voulu me tuer; un jour ton père, également désespéré, a voulu se tuer aussi. 

«Si l'on avait dit à ton père, au moment où il dirigeait le canon du pistolet vers son front, si l'on m'avait dit à moi, au moment où j'écartais de mon lit le pain du prisonnier auquel je n'avais pas touché depuis trois jours, si l'on nous avait dit enfin à tous deux, en ce moment suprême: 

«Vivez! un jour viendra où vous serez heureux et où vous bénirez la vie, de quelque part que vînt la voix, nous l'eussions accueillie avec le sourire du doute ou avec l'angoisse de l'incrédulité, et cependant combien de fois, en t'embrassant, ton père a-t-il béni la vie, combien de fois moi-même\dots 

—Ah! s'écria Morrel, interrompant le comte, vous n'aviez perdu que votre liberté, vous; mon père n'avait perdu que sa fortune, lui; et moi, j'ai perdu Valentine. 

—Regarde-moi, Morrel, dit Monte-Cristo avec cette solennité qui, dans certaines occasions, le faisait si grand et si persuasif; regarde-moi, je n'ai ni larmes dans les yeux, ni fièvre dans les veines, ni battements funèbres dans le cœur, cependant je te vois souffrir, toi, Maximilien, toi que j'aime comme j'aimerais mon fils: eh bien, cela ne te dit-il pas, Morrel, que la douleur est comme la vie, et qu'il y a toujours quelque chose d'inconnu au-delà? Or, si je te prie, si je t'ordonne de vivre, Morrel, c'est dans la conviction qu'un jour tu me remercieras de t'avoir conservé la vie. 

—Mon Dieu! s'écria le jeune homme, mon Dieu! que me dites-vous là, comte? Prenez-y garde! peut-être n'avez-vous jamais aimé, vous? 

—Enfant! répondit le comte. 

—D'amour, reprit Morrel, je m'entends. 

«Moi, voyez-vous, je suis un soldat depuis que je suis un homme; je suis arrivé jusqu'à vingt-neuf ans sans aimer, car aucun des sentiments que j'ai éprouvés jusque-là ne mérite le nom d'amour: eh bien, à vingt-neuf ans j'ai vu Valentine: donc depuis près de deux ans je l'aime, depuis près de deux ans j'ai pu lire les vertus de la fille et de la femme écrites par la main même du Seigneur dans ce cœur ouvert pour moi comme un livre. 

«Comte, il y avait pour moi, avec Valentine, un bonheur infini, immense, inconnu, un bonheur trop grand, trop complet, trop divin, pour ce monde; puisque ce monde ne me l'a pas donné, comte, c'est vous dire que sans Valentine il n'y a pour moi sur la terre que désespoir et désolation. 

—Je vous ai dit d'espérer, Morrel, répéta le comte. 

—Prenez garde alors, répéterai-je aussi, dit Morrel, car vous cherchez à me persuader, et si vous me persuadez, vous me ferez perdre la raison, car vous me ferez croire que je puis revoir Valentine.» 

Le comte sourit. 

«Mon ami, mon père! s'écria Morrel exalté, prenez garde, vous redirai-je pour la troisième fois, car l'ascendant que vous prenez sur moi m'épouvante; prenez garde au sens de vos paroles, car voilà mes yeux qui se raniment, voilà mon cœur qui se rallume et qui renaît; prenez garde, car vous me feriez croire à des choses surnaturelles. 

«J'obéirais si vous me commandiez de lever la pierre du sépulcre qui recouvre la fille de Jaïre, je marcherais sur les flots, comme l'apôtre, si vous me faisiez de la main signe de marcher sur les flots; prenez garde, j'obéirais. 

—Espère, mon ami, répéta le comte. 

—Ah! dit Morrel en retombant de toute la hauteur de son exaltation dans l'abîme de sa tristesse, ah! vous vous jouez de moi: vous faites comme ces bonnes mères, ou plutôt comme ces mères égoïstes qui calment avec des paroles mielleuses la douleur de l'enfant, parce que ses cris les fatiguent. 

«Non, mon ami, j'avais tort de vous dire de prendre garde; non, ne craignez rien, j'enterrerai ma douleur avec tant de soin dans le plus profond de ma poitrine, je la rendrai si obscure, si secrète, que vous n'aurez plus même le souci d'y compatir. 

«Adieu! mon ami; adieu! 

—Au contraire, dit le comte; à partir de cette heure, Maximilien, tu vivras près de moi et avec moi, tu ne me quitteras plus, et dans huit jours nous aurons laissé derrière nous la France. 

—Et vous me dites toujours d'espérer? 

—Je te dis d'espérer, parce que je sais un moyen de te guérir. 

—Comte, vous m'attristez davantage encore s'il est possible. Vous ne voyez, comme résultat du coup qui me frappe, qu'une douleur banale, et vous croyez me consoler par un moyen banal, le voyage.» 

Et Morrel secoua la tête avec une dédaigneuse incrédulité. 

«Que veux-tu que je te dise? reprit Monte-Cristo. 

«J'ai foi dans mes promesses, laisse-moi faire l'expérience. 

—Comte, vous prolongez mon agonie, voilà tout. 

—Ainsi, dit le comte, faible cœur que tu es, tu n'as pas la force de donner à ton ami quelques jours pour l'épreuve qu'il tente! 

«Voyons, sais-tu de quoi le comte de Monte-Cristo est capable? 

«Sais-tu qu'il commande à bien des puissances terrestres? 

«Sais-tu qu'il a assez de foi en Dieu pour obtenir des miracles de celui qui a dit qu'avec la foi l'homme pouvait soulever une montagne? 

«Eh bien, ce miracle que j'espère, attends-le, ou bien\dots 

—Ou bien\dots répéta Morrel. 

—Ou bien, prends-y garde, Morrel, je t'appellerai ingrat. 

—Ayez pitié de moi, comte. 

—J'ai tellement pitié de toi, Maximilien, écoute-moi, tellement pitié, que si je ne te guéris pas dans un mois, jour pour jour, heure pour heure, retiens bien mes paroles, Morrel, je te placerai moi-même en face de ces pistolets tout chargés et d'une coupe du plus sûr poison d'Italie, d'un poison plus sûr et plus prompt, crois-moi, que celui qui a tué Valentine. 

—Vous me le promettez? 

—Oui, car je suis homme, car, moi aussi, comme je te l'ai dit, j'ai voulu mourir, et souvent même, depuis que le malheur s'est éloigné de moi, j'ai rêvé les délices de l'éternel sommeil. 

—Oh! bien sûr, vous me promettez cela, comte? s'écria Maximilien enivré. 

—Je ne te le promets pas, je te le jure, dit Monte-Cristo en étendant la main. 

—Dans un mois, sur votre honneur, si je ne suis pas consolé, vous me laissez libre de ma vie, et, quelque chose que j'en fasse, vous ne m'appellerez pas ingrat? 

—Dans un mois jour pour jour, Maximilien; dans un mois, heure pour heure, et la date est sacrée, Maximilien; je ne sais pas si tu y as songé, nous sommes aujourd'hui le 5 septembre. 

«Il y a aujourd'hui dix ans que j'ai sauvé ton père, qui voulait mourir.» 

Morrel saisit les mains du comte et les baisa; le comte le laissa faire, comme s'il comprenait que cette adoration lui était due. 

«Dans un mois, continua Monte-Cristo, tu auras, sur la table devant laquelle nous serons assis l'un et l'autre, de bonnes armes et une douce mort; mais, en revanche, tu me promets d'attendre jusque-là et de vivre? 

—Oh! à mon tour, s'écria Morrel, je vous le jure!» 

Monte-Cristo attira le jeune homme sur son cœur, et l'y retint longtemps. 

«Et maintenant, lui dit-il, à partir d'aujourd'hui, tu vas venir demeurer chez moi; tu prendras l'appartement d'Haydée, et ma fille au moins sera remplacée par mon fils. 

—Haydée! dit Morrel; qu'est devenue Haydée? 

—Elle est partie cette nuit. 

—Pour vous quitter? 

—Pour m'attendre\dots 

«Tiens-toi donc prêt à venir me rejoindre rue des Champs-Élysées, et fais-moi sortir d'ici sans qu'on me voie.» 

Maximilien baissa la tête, et obéit comme un enfant ou comme un apôtre. 