\chapter{Monsieur Bertuccio}

\lettrine{P}{endant} ce temps le comte était arrivé chez lui; il avait mis six minutes pour faire le chemin. Ces six minutes avaient suffi pour qu'il fût vu de vingt jeunes gens qui, connaissant le prix de l'attelage qu'ils n'avaient pu acheter eux-mêmes, avaient mis leur monture au galop pour entrevoir le splendide seigneur qui se donnait des chevaux de dix mille francs la pièce. 

La maison choisie par Ali, et qui devait servir de résidence de ville à Monte-Cristo, était située à droite en montant les Champs-Élysées, placée entre cour et jardin; un massif fort touffu, qui s'élevait au milieu de la cour, masquait une partie de la façade, autour de ce massif s'avançaient, pareilles à deux bras, deux allées qui, s'étendant à droite et à gauche, amenaient à partir de la grille, les voitures à un double perron supportant à chaque marche un vase de porcelaine plein de fleurs. Cette maison, isolée au milieu d'un large espace, avait, outre l'entrée principale, une autre entrée donnant sur la rue de Ponthieu. 

Avant même que le cocher eût hélé le concierge, la grille massive roula sur ses gonds; on avait vu venir le comte, et à Paris comme à Rome, comme partout, il était servi avec la rapidité de l'éclair. Le cocher entra donc, décrivit le demi-cercle sans avoir ralenti son allure, et la grille était refermée déjà que les roues criaient encore sur le sable de l'allée. 

Au côté gauche du perron la voiture s'arrêta; deux hommes parurent à la portière: l'un était Ali, qui sourit à son maître avec une incroyable franchise de joie, et qui se trouva payé par un simple regard de Monte-Cristo. 

L'autre salua humblement et présenta son bras au comte pour l'aider à descendre de la voiture. 

«Merci, monsieur Bertuccio, dit le comte en sautant légèrement les trois degrés du marchepied; et le notaire? 

—Il est dans le petit salon, Excellence, répondit Bertuccio. 

—Et les cartes de visite que je vous ai dit de faire graver dès que vous auriez le numéro de la maison?  

—Monsieur le comte, c'est déjà fait; j'ai été chez le meilleur graveur du Palais-Royal, qui a exécuté la planche devant moi; la première carte tirée a été portée à l'instant même, selon votre ordre, à M. le baron Danglars, député, rue de la Chaussée-d'Antin, n° 7; les autres sont sur la cheminée de la chambre à coucher de Votre Excellence. 

—Bien. Quelle heure est-il? 

—Quatre heures.» 

Monte-Cristo donna ses gants, son chapeau et sa canne à ce même laquais français qui s'était élancé hors de l'antichambre du comte de Morcerf pour appeler la voiture, puis il passa dans le petit salon conduit par Bertuccio, qui lui montra le chemin. 

«Voilà de pauvres marbres dans cette antichambre, dit Monte-Cristo, j'espère bien qu'on m'enlèvera tout cela.» 

Bertuccio s'inclina. 

Comme l'avait dit l'intendant, le notaire attendait dans le petit salon. 

C'était une honnête figure de deuxième clerc de Paris, élevé à la dignité infranchissable de tabellion de la banlieue.  

«Monsieur est le notaire chargé de vendre la maison de campagne que je veux acheter? demanda Monte-Cristo. 

—Oui, monsieur le comte, répliqua le notaire. 

—L'acte de vente est-il prêt? 

—Oui, monsieur le comte. 

—L'avez-vous apporté? 

—Le voici. 

—Parfaitement. Et où est cette maison que j'achète», demanda négligemment Monte-Cristo, s'adressant moitié à Bertuccio, moitié au notaire. 

L'intendant fit un geste qui signifiait: Je ne sais pas. 

Le notaire regarda Monte-Cristo avec étonnement. 

«Comment, dit-il, monsieur le comte ne sait pas où est la maison qu'il achète? 

—Non, ma foi, dit le comte. 

—Monsieur le comte ne la connaît pas? 

—Et comment diable la connaîtrais-je? j'arrive de Cadix ce matin, je ne suis jamais venu à Paris, c'est même la première fois que je mets le pied en France. 

—Alors c'est autre chose, répondit le notaire; la maison que monsieur le comte achète est située à Auteuil.» 

À ces mots, Bertuccio pâlit visiblement. 

«Et où prenez-vous Auteuil? demanda Monte-Cristo. 

—À deux pas d'ici, monsieur le comte, dit le notaire, un peu après Passy, dans une situation charmante, au milieu du bois de Boulogne.  

—Si près que cela! dit Monte-Cristo, mais ce n'est pas la campagne. Comment diable m'avez-vous été choisir une maison à la porte de Paris, monsieur Bertuccio? 

—Moi! s'écria l'intendant avec un étrange empressement; non, certes, ce n'est pas moi que monsieur le comte a chargé de choisir cette maison; que monsieur le comte veuille bien se rappeler, chercher dans sa mémoire, interroger ses souvenirs. 

—Ah! c'est juste, dit Monte-Cristo; je me rappelle maintenant! j'ai lu cette annonce dans un Journal, et je me suis laissé séduire par ce titre menteur: \textit{Maison de campagne}. 

—Il est encore temps, dit vivement Bertuccio, et si Votre Excellence veut me charger de chercher partout ailleurs, je lui trouverai ce qu'il y aura de mieux, soit à Enghien, soit à Fontenay-aux-Roses, soit à Bellevue. 

—Non, ma foi, dit insoucieusement Monte-Cristo; puisque j'ai celle-là, je la garderai. 

—Et monsieur a raison, dit vivement le notaire, qui craignait de perdre ses honoraires. C'est une charmante propriété: eaux vives, bois touffus, habitation confortable, quoique abandonnée depuis longtemps; sans compter le mobilier, qui, si vieux qu'il soit, a de la valeur, surtout aujourd'hui que l'on recherche les antiquailles. Pardon, mais je crois que monsieur le comte a le goût de son époque.  

—Dites toujours, fit Monte-Cristo; c'est convenable, alors. 

—Ah! monsieur, c'est mieux que cela, c'est magnifique! 

—Peste! ne manquons pas une pareille occasion, dit Monte-Cristo; le contrat, s'il vous plaît, monsieur le notaire?» 

Et il signa rapidement, après avoir jeté un regard à l'endroit de l'acte où étaient désignés la situation de la maison et les noms des propriétaires. 

«Bertuccio, dit-il, donnez cinquante-cinq mille francs à monsieur.» 

L'intendant sortit d'un pas mal assuré, et revint avec une liasse de billets de banque que le notaire compta en homme qui a l'habitude de ne recevoir son argent qu'après la purge légale. 

«Et maintenant, demanda le comte, toutes les formalités sont-elles remplies? 

—Toutes, monsieur le comte. 

—Avez-vous les clefs? 

—Elles sont aux mains du concierge qui garde la maison; mais voici l'ordre que je lui ai donné d'installer monsieur dans sa propriété.  

—Fort bien.» 

Et Monte-Cristo fit au notaire un signe de tête qui voulait dire: 

«Je n'ai plus besoin de vous, allez-vous-en.» 

«Mais, hasarda l'honnête tabellion, monsieur le comte s'est trompé, il me semble; ce n'est que cinquante mille francs, tout compris. 

—Et vos honoraires? 

—Se trouvent payés moyennant cette somme, monsieur le comte. 

—Mais n'êtes-vous pas venu d'Auteuil ici? 

—Oui, sans doute. 

—Eh bien, il faut bien vous payer votre dérangement», dit le comte. 

Et il le congédia du geste. 

Le notaire sortit à reculons et en saluant jusqu'à terre; c'était la première fois, depuis le jour où il avait pris ses inscriptions, qu'il rencontrait un pareil client. 

«Conduisez monsieur», dit le comte à Bertuccio. 

Et l'intendant sortit derrière le notaire. 

À peine le comte fut-il seul qu'il sortit de sa poche un portefeuille à serrure, qu'il ouvrit avec une petite clef attachée à son cou et qui ne le quittait jamais. 

Après avoir cherché un instant, il s'arrêta à un feuillet qui portait quelques notes, confronta ces notes avec l'acte de vente déposé sur la table, et, recueillant ses souvenirs: 

«Auteuil, rue de la Fontaine, n° 28; c'est bien cela, dit-il; maintenant dois-je m'en rapporter à un aveu arraché par la terreur religieuse ou par la terreur physique? Au reste, dans une heure je saurai tout. Bertuccio! cria-t-il en frappant avec une espèce de petit marteau à manche pliant sur un timbre qui rendit un son aigu et prolongé pareil à celui d'un tam-tam, Bertuccio!» 

L'intendant parut sur le seuil. 

«Monsieur Bertuccio, dit le comte, ne m'avez-vous pas dit autrefois que vous aviez voyagé en France? 

—Dans certaines parties de la France, oui, Excellence. 

—Vous connaissez les environs de Paris, sans doute? 

—Non, Excellence, non, répondit l'intendant avec une sorte de tremblement nerveux que Monte-Cristo, connaisseur en fait d'émotions, attribua avec raison à une vive inquiétude.  

—C'est fâcheux, dit-il, que vous n'ayez jamais visité les environs de Paris, car je veux aller ce soir même voir ma nouvelle propriété, et en venant avec moi vous m'eussiez donné sans doute d'utiles renseignements. 

—À Auteuil? s'écria Bertuccio dont le teint cuivré devint presque livide. Moi, aller à Auteuil! 

—Eh bien, qu'y a-t-il d'étonnant que vous veniez à Auteuil, je vous le demande? Quand je demeurerai à Auteuil, il faudra bien que vous y veniez, puisque vous faites partie de la maison.» 

Bertuccio baissa la tête devant le regard impérieux du maître, et il demeura immobile et sans réponse. 

«Ah çà! mais, que vous arrive-t-il. Vous allez donc me faire sonner une seconde fois pour la voiture?» dit Monte-Cristo du ton que Louis XIV mit à prononcer le fameux: «J'ai failli attendre!» 

Bertuccio ne fit qu'un bond du petit salon à l'antichambre, et cria d'une voix rauque: 

«Les chevaux de son Excellence!» 

Monte-Cristo écrivit deux ou trois lettres; comme il cachetait la dernière, l'intendant reparut. 

«La voiture de son Excellence est à la porte, dit-il.  

—Eh bien, prenez vos gants et votre chapeau, dit Monte-Cristo. 

—Est-ce que je vais avec monsieur le comte? s'écria Bertuccio. 

—Sans doute, il faut bien que vous donniez vos ordres, puisque je compte habiter cette maison.» 

Il était sans exemple que l'on eût répliqué à une injonction du comte; aussi l'intendant, sans faire aucune objection, suivit-il son maître, qui monta dans la voiture et lui fit signe de le suivre. L'intendant s'assit respectueusement sur la banquette du devant. 