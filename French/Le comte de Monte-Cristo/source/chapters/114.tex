\chapter{Peppino}

\lettrine{A}{u} moment même où le bateau à vapeur du comte disparaissait derrière le cap Morgiou, un homme, courant la poste sur la route de Florence à Rome, venait de dépasser la petite ville d'Aquapendente. Il marchait assez pour faire beaucoup de chemin, sans toutefois devenir suspect. 

Vêtu d'une redingote ou plutôt d'un surtout que le voyage avait infiniment fatigué, mais qui laissait voir brillant et frais encore un ruban de la Légion d'honneur répété à son habit, cet homme, non seulement à ce double signe, mais encore à l'accent avec lequel il parlait au postillon, devait être reconnu pour Français. Une preuve encore qu'il était né dans le pays de la langue universelle, c'est qu'il ne savait d'autres mots italiens que ces mots de musique qui peuvent, comme le \textit{goddam} de Figaro, remplacer toutes les finesses d'une langue particulière. 

«\textit{Allegro}!» disait-il aux postillons à chaque montée. 

«\textit{Moderato}!» faisait-il à chaque descente. 

Et Dieu sait s'il y a des montées et des descentes en allant de Florence à Rome par la route d'Aquapendente! 

Ces deux mots, au reste, faisaient beaucoup rire les braves gens auxquels ils étaient adressés. 

En présence de la ville éternelle, c'est-à-dire en arrivant à la Storta, point d'où l'on aperçoit Rome, le voyageur n'éprouva point ce sentiment de curiosité enthousiaste qui pousse chaque étranger à s'élever du fond de sa chaise pour tâcher d'apercevoir le fameux dôme de Saint-Pierre, qu'on aperçoit déjà bien avant de distinguer autre chose. Non, il tira seulement un portefeuille de sa poche, et de son portefeuille un papier plié en quatre, qu'il déplia et replia avec une attention qui ressemblait à du respect, et il se contenta de dire: 

«Bon, je l'ai toujours.» 

La voiture franchit la porte del Popolo, prit à gauche et s'arrêta à l'hôtel d'Espagne. 

Maître Pastrini, notre ancienne connaissance, reçut le voyageur sur le seuil de la porte et le chapeau à la main. 

Le voyageur descendit, commanda un bon dîner, et s'informa de l'adresse de la maison Thomson et French, qui lui fut indiquée à l'instant même, cette maison étant une des plus connues de Rome. 

Elle était située via dei Banchi, près de Saint-Pierre. 

À Rome, comme partout, l'arrivée d'une chaise de poste est un événement. Dix jeunes descendants de Marius et des Gracques, pieds nus, les coudes percés, mais le poing sur la hanche et le bras pittoresquement recourbé au-dessus de la tête, regardaient le voyageur, la chaise de poste et les chevaux; à ces gamins de la ville par excellence s'étaient joints une cinquantaine de badauds des États de Sa Sainteté, de ceux-là qui font des ronds en crachant dans le Tibre du haut du pont Saint-Ange, quand le Tibre a de l'eau. 

Or, comme les gamins et les badauds de Rome, plus heureux que ceux de Paris, comprennent toutes les langues, et surtout la langue française, ils entendirent le voyageur demander un appartement, demander à dîner, et demander enfin l'adresse de la maison Thomson et French. 

Il en résulta que, lorsque le nouvel arrivant sortit de l'hôtel avec le cicérone de rigueur, un homme se détacha du groupe des curieux, et sans être remarqué du voyageur, sans paraître être remarqué de son guide, marcha à peu de distance de l'étranger, le suivant avec autant d'adresse qu'aurait pu le faire un agent de la police parisienne. 

Le Français était si pressé de faire sa visite à la maison Thomson et French qu'il n'avait pas pris le temps d'attendre que les chevaux fussent attelés; la voiture devait le rejoindre en route ou l'attendre à la porte du banquier. 

On arriva sans que la voiture eût rejoint. 

Le Français entra, laissant dans l'antichambre son guide, qui aussitôt entra en conversation avec deux ou trois de ces industriels sans industrie, ou plutôt aux mille industries, qui se tiennent à Rome à la porte des banquiers, des églises, des ruines, des musées ou des théâtres. 

En même temps que le Français, l'homme qui s'était détaché du groupe des curieux entra aussi; le Français sonna au guichet des bureaux et pénétra dans la première pièce; son ombre en fit autant. 

«MM. Thomson et French?» demanda l'étranger. 

Une espèce de laquais se leva sur le signe d'un commis de confiance, gardien solennel du premier bureau. 

«Qui annoncerai-je? demanda le laquais, se préparant à marcher devant l'étranger. 

—M. le baron Danglars, répondit le voyageur. 

—Venez», dit le laquais. 

Une porte s'ouvrit, le laquais et le baron disparurent par cette porte. L'homme qui était entré derrière Danglars s'assit sur un banc d'attente. 

Le commis continua d'écrire pendant cinq minutes à peu après; pendant ces cinq minutes, l'homme assis garda le plus profond silence et la plus stricte immobilité. 

Puis la plume du commis cessa de crier sur le papier; il leva la tête, regarda attentivement autour de lui, et après s'être assuré du tête-à-tête: 

«Ah! ah! dit-il, te voilà Peppino? 

—Oui, répondit laconiquement celui-ci. 

—Tu as flairé quelque chose de bon chez ce gros homme? 

—Il n'y a pas grand mérite pour celui-ci, nous sommes prévenus. 

—Tu sais donc ce qu'il vient faire ici, curieux. 

—Pardieu, il vient toucher; seulement, reste à savoir quelle somme. 

—On va te dire cela tout à l'heure, l'ami. 

—Fort bien; mais ne va pas, comme l'autre jour, me donner un faux renseignement. 

—Qu'est-ce à dire, et de qui veux-tu parler? Serait-ce de cet Anglais qui a emporté d'ici trois mille écus l'autre jour? 

—Non, celui-là avait en effet les trois mille écus, et nous les avons trouvés. Je veux parler de ce prince russe. 

—Eh bien? 

—Eh bien, tu nous avais accusé trente mille livres, et nous n'en avons trouvé que vingt-deux. 

—Vous aurez mal cherché. 

—C'est Luigi Vampa qui a fait la perquisition en personne. 

—En ce cas, il avait ou payé ses dettes\dots 

—Un Russe? 

—Ou dépensé son argent. 

—C'est possible, après tout. 

—C'est sûr; mais laisse-moi aller à mon observatoire, le Français ferait son affaire sans que je pusse savoir le chiffre positif.» 

Peppino fit un signe affirmatif, et, tirant un chapelet de sa poche, se mit à marmotter quelque prière, tandis que le commis disparaissait par la même porte qui avait donné passage au laquais et au baron. 

Au bout de dix minutes environ, le commis reparut radieux. 

«Eh bien? demanda Peppino à son ami. 

—Alerte, alerte! dit le commis, la somme est ronde. 

—Cinq à six millions, n'est-ce pas? 

—Oui; tu sais le chiffre? 

—Sur un reçu de Son Excellence le comte de Monte-Cristo. 

—Tu connais le comte? 

—Et dont on l'a crédité sur Rome, Venise et Vienne. 

—C'est cela! s'écria le commis; comment es-tu si bien informé? 

—Je t'ai dit que nous avions été prévenus à l'avance. 

—Alors, pourquoi t'adresses-tu à moi? 

—Pour être sûr que c'est bien l'homme à qui nous avons affaire. 

—C'est bien lui\dots Cinq millions. Une jolie somme hein, Peppino? 

—Oui. 

—Nous n'en aurons jamais autant. 

—Au moins, répondit philosophiquement Peppino, en aurons-nous quelques bribes. 

—Chut! Voici notre homme.» 

Le commis reprit sa plume, et Peppino son chapelet; l'un écrivait, l'autre priait quand la porte se rouvrit. Danglars apparut radieux, accompagné par le banquier, qui le reconduisit jusqu'à la porte. 

Derrière Danglars descendit Peppino. 

Selon les conventions, la voiture qui devait rejoindre Danglars attendait devant la maison Thomson et French. Le cicérone en tenait la portière ouverte: le cicérone est un être très complaisant et qu'on peut employer à toute chose. 

Danglars sauta dans la voiture, léger comme un jeune homme de vingt ans. Le cicérone referma la portière et monta près du cocher. Peppino monta sur le siège de derrière. 

«Son Excellence veut-elle voir Saint-Pierre? demanda le cicérone. 

—Pour quoi faire? répondit le baron. 

—Dame! pour voir. 

—Je ne suis pas venu à Rome pour voir», dit tout haut Danglars; puis il ajouta tout bas avec son sourire cupide: «Je suis venu pour toucher.» 

Et il toucha en effet son portefeuille, dans lequel il venait d'enfermer une lettre. 

«Alors Son Excellence va\dots 

—À l'hôtel. 

—Casa Pastrini», dit le cicérone au cocher. 

Et la voiture partit rapide comme une voiture de maître. 

Dix minutes après, le baron était rentré dans son appartement, et Peppino s'installait sur le banc accolé à la devanture de l'hôtel, après avoir dit quelques mots à l'oreille d'un de ces descendants de Marius et des Gracques que nous avons signalés au commencement de ce chapitre, lequel descendant prit le chemin du Capitole de toute la vitesse de ses jambes. 

Danglars était las, satisfait, et avait sommeil. Il se coucha, mit son portefeuille sous son traversin et s'endormit. 

Peppino avait du temps de reste; il joua à la \textit{morra} avec des facchino, perdit trois écus, et pour se consoler but un flacon de vin d'Orvietto. 

Le lendemain, Danglars s'éveilla tard, quoiqu'il se fût couché de bonne heure; il y avait cinq ou six nuits qu'il dormait fort mal, quand toutefois il dormait. 

Il déjeuna copieusement, et peu soucieux, comme il l'avait dit, de voir les beautés de la Ville éternelle, il demanda ses chevaux de poste pour midi. 

Mais Danglars avait compté sans les formalités de la police et sans la paresse du maître de poste. 

Les chevaux arrivèrent à deux heures seulement, et le cicérone ne rapporta le passeport visé qu'à trois. 

Tous ces préparatifs avaient amené devant la porte de maître Pastrini bon nombre de badauds. 

Les descendants des Gracques et de Marius ne manquaient pas non plus. 

Le baron traversa triomphalement ces groupes, qui l'appelaient Excellence pour avoir un bajocco. 

Comme Danglars, homme très populaire, comme on sait, s'était contenté de se faire appeler baron jusque-là et n'avait pas encore été traité d'Excellence, ce titre le flatta, et il distribua une douzaine de pauls à toute cette canaille, toute prête, pour douze autres pauls, à le traiter d'Altesse. 

«Quelle route? demanda le postillon en italien. 

—Route d'Ancône», répondit le baron. 

Maître Pastrini traduisit la demande et la réponse, et la voiture partit au galop. 

Danglars voulait effectivement passer à Venise et y prendre une partie de sa fortune, puis de Venise aller à Vienne, où il réaliserait le reste. 

Son intention était de se fixer dans cette dernière ville, qu'on lui avait assuré être une ville de plaisirs. 

À peine eut-il fait trois lieues dans la campagne de Rome, que la nuit commença de tomber; Danglars n'avait pas cru partir si tard, sinon il serait resté; il demanda au postillon combien il y avait avant d'arriver à la prochaine ville. 

«\textit{Non capisco}», répondit le postillon. 

Danglars fit un mouvement de la tête qui voulait dire: 

«Très bien!» 

La voiture continua sa route. 

«À la première poste, se dit Danglars, j'arrêterai.» 

Danglars éprouvait encore un reste du bien-être qu'il avait ressenti la veille, et qui lui avait procuré une si bonne nuit. Il était mollement étendu dans une bonne calèche anglaise à doubles ressorts; il se sentait entraîné par le galop de deux bons chevaux; le relais était de sept lieues, il le savait. Que faire quand on est banquier et qu'on a heureusement fait banqueroute? 

Danglars songea dix minutes à sa femme restée à Paris, dix autres minutes à sa fille courant le monde avec Mlle d'Armilly, il donna dix autres minutes à ses créanciers et à la manière dont il emploierait leur argent; puis, n'ayant plus rien à quoi penser, il ferma les yeux et s'endormit. 

Parfois cependant, secoué par un cahot plus fort que les autres, Danglars rouvrait un moment les yeux; alors il se sentait toujours emporté avec la même vitesse à travers cette même campagne de Rome toute parsemée d'aqueducs brisés, qui semblent des géants de granit pétrifiés au milieu de leur course. Mais la nuit était froide, sombre, pluvieuse, et il faisait bien meilleur pour un homme à moitié assoupi de demeurer au fond de sa chaise les yeux fermés, que de mettre la tête à la portière pour demander où il était à un postillon qui ne savait répondre autre chose que: \textit{Non capisco.} 

Danglars continua donc de dormir, en se disant qu'il serait toujours temps de se réveiller au relais. 

La voiture s'arrêta; Danglars pensa qu'il touchait enfin au but tant désiré. 

Il rouvrit les yeux, regarda à travers la vitre, s'attendant à se trouver au milieu de quelque ville, ou tout au moins de quelque village; mais il ne vit rien qu'une espèce de masure isolée, et trois ou quatre hommes qui allaient et venaient comme des ombres. 

Danglars attendit un instant que le postillon qui avait achevé son relais vînt lui réclamer l'argent de la poste; il comptait profiter de l'occasion pour demander quelques renseignements à son nouveau conducteur, mais les chevaux furent dételés et remplacés sans que personne vînt demander d'argent au voyageur. Danglars, étonné, ouvrit la portière; mais une main vigoureuse la repoussa aussitôt, et la chaise roula. 

Le baron, stupéfait, se réveilla entièrement. 

«Eh! dit-il au postillon, eh! \textit{mio caro}!» 

C'était encore de l'italien de romance que Danglars avait retenu lorsque sa fille chantait des duos avec le prince Cavalcanti. 

Mais \textit{mio caro} ne répondit point. 

Danglars se contenta alors d'ouvrir la vitre. 

«Hé, l'ami! où allons-nous donc? dit-il en passant sa tête par l'ouverture. 

—\textit{Dentro la testa}! cria une voix grave et impérieuse, accompagnée d'un geste de menace. 

Danglars comprit que \textit{dentro la testa} voulait dire: Rentrez la tête. Il faisait, comme on voit, de rapides progrès dans l'italien. 

Il obéit, non sans inquiétude; et comme cette inquiétude augmentait de minute en minute, au bout de quelques instants son esprit, au lieu du vide que nous avons signalé au moment où il se mettait en route, et qui avait amené le sommeil, son esprit, disons-nous, se trouva rempli de quantité de pensées plus propres les unes que les autres à tenir éveillé l'intérêt d'un voyageur, et surtout d'un voyageur dans la situation de Danglars. 

Ses yeux prirent dans les ténèbres ce degré de finesse que communiquent dans le premier moment les émotions fortes, et qui s'émousse plus tard pour avoir été trop exercé. Avant d'avoir peur, on voit juste; pendant qu'on a peur, on voit double, et après qu'on a eu peur, on voit trouble. 

Danglars vit un homme enveloppé d'un manteau, qui galopait à la portière de droite. 

«Quelque gendarme, dit-il. Aurais-je été signalé par les télégraphes français aux autorités pontificales?» 

Il résolut de sortir de cette anxiété. 

«Où me menez-vous? demanda-t-il. 

—\textit{Dentro la testa}!» répéta la même voix, avec le même accent de menace. 

Danglars se retourna vers la portière de gauche. 

Un autre homme à cheval galopait à la portière de gauche. 

«Décidément, se dit Danglars la sueur au front, décidément je suis pris.» 

Et il se rejeta au fond de sa calèche, cette fois non pas pour dormir, mais pour songer. 

Un instant après, la lune se leva. 

Du fond de la calèche, il plongea son regard dans la campagne; il revit alors ces grands aqueducs, fantômes de pierre, qu'il avait remarqués en passant; seulement, au lieu de les avoir à droite, il les avait maintenant à gauche. 

Il comprit qu'on avait fait faire demi-tour à la voiture, et qu'on le ramenait à Rome. 

«Oh! malheureux, murmura-t-il, on aura obtenu l'extradition!» 

La voiture continuait de courir avec une effrayante vélocité. Une heure passa terrible, car à chaque nouvel indice jeté sur son passage le fugitif reconnaissait, à n'en point douter, qu'on le ramenait sur ses pas. Enfin, il revit une masse sombre contre laquelle il lui sembla que la voiture allait se heurter. Mais la voiture se détourna, longeant cette masse sombre, qui n'était autre que la ceinture de remparts qui enveloppe Rome. 

«Oh! oh! murmura Danglars, nous ne rentrons pas dans la ville, donc ce n'est pas la justice qui m'arrête. Bon Dieu! autre idée, serait-ce\dots» 

Ses cheveux se hérissèrent. 

Il se rappela ces intéressantes histoires de bandits romains, si peu crues à Paris, et qu'Albert de Morcerf avait racontées à Mme Danglars et à Eugénie lorsqu'il était question, pour le jeune vicomte, de devenir le fils de l'une et le mari de l'autre. 

«Des voleurs, peut-être!» murmura-t-il. 

Tout à coup la voiture roula sur quelque chose de plus dur que le sol d'un chemin sablé. Danglars hasarda un regard aux deux côtés de la route; il aperçut des monuments de forme étrange, et sa pensée préoccupée du récit de Morcerf, qui maintenant se présentait à lui dans tous ses détails, sa pensée lui dit qu'il devait être sur la voie Appienne. 

À gauche de la voiture, dans une espèce de vallée, on voyait une excavation circulaire. 

C'était le cirque de Caracalla. 

Sur un mot de l'homme qui galopait à la portière de droite, la voiture s'arrêta. 

En même temps, la portière de gauche s'ouvrit. 

«\textit{Scendi}!» commanda une voix. 

Danglars descendit à l'instant même; il ne parlait pas encore l'italien, mais il l'entendait déjà. 

Plus mort que vif, le baron regarda autour de lui. 

Quatre hommes l'entouraient, sans compter le postillon. 

«\textit{Di quà}», dit un des quatre hommes en descendant un petit sentier qui conduisait de la voie Appienne au milieu de ces inégales hachures de la campagne de Rome. 

Danglars suivit son guide sans discussion, et n'eut pas besoin de se retourner pour savoir qu'il était suivi des trois autres hommes. 

Cependant il lui sembla que ces hommes s'arrêtaient comme des sentinelles à des distances à peu près égales. 

Après dix minutes de marche à peu près, pendant lesquelles Danglars n'échangea point une seule parole avec son guide, il se trouva entre un tertre et un buisson de hautes herbes; trois hommes debout et muets formaient un triangle dont il était le centre. 

Il voulut parler; sa langue s'embarrassa. 

«\textit{Avanti}», dit la même voix à l'accent bref et impératif. 

Cette fois Danglars comprit doublement: il comprit par la parole et par le geste, car l'homme qui marchait derrière lui le poussa si rudement en avant qu'il alla heurter son guide. 

Ce guide était notre ami Peppino, qui s'enfonça dans les hautes herbes par une sinuosité que les fouines et les lézards pouvaient seuls reconnaître pour un chemin frayé. 

Peppino s'arrêta devant une roche surmontée d'un épais buisson; cette roche entrouverte comme une paupière, livra passage au jeune homme, qui y disparut comme disparaissent dans leurs trappes les diables de nos féeries. 

La voix et le geste de celui qui suivait Danglars engagèrent le banquier à en faire autant. Il n'y avait plus à en douter, le banqueroutier français avait affaire à des bandits romains. 

Danglars s'exécuta comme un homme placé entre deux dangers terribles, et que la peur rend brave. Malgré son ventre assez mal disposé pour pénétrer dans les crevasses de la campagne de Rome, il s'infiltra derrière Peppino, et, se laissant glisser en fermant les yeux, il tomba sur ses pieds. 

En touchant la terre, il rouvrit les yeux. 

Le chemin était large, mais noir. Peppino, peu soucieux de se cacher, maintenant qu'il était chez lui, battit le briquet, et alluma une torche. 

Deux autres hommes descendirent derrière Danglars, formant l'arrière-garde, et, poussant Danglars lorsque par hasard il s'arrêtait, le firent arriver par une pente douce au centre d'un carrefour de sinistre apparence. 

En effet, les parois des murailles, creusées en cercueils superposés les uns aux autres, semblaient, au milieu des pierres blanches, ouvrir ces yeux noirs et profonds qu'on remarque dans les têtes de mort. 

Une sentinelle fit battre contre sa main gauche les capucines de sa carabine. 

«Qui vive? fit la sentinelle. 

—Ami, ami! dit Peppino. Où est le capitaine? 

—Là, dit la sentinelle, en montrant par-dessus son épaule une espèce de grande salle creusée dans le roc et dont la lumière se reflétait dans le corridor par de grandes ouvertures cintrées. 

—Bonne proie, capitaine, bonne proie», dit Peppino en italien. 

Et prenant Danglars par le collet de sa redingote, il le conduisit vers une ouverture ressemblant à une porte, et par laquelle on pénétrait dans la salle dont le capitaine paraissait avoir fait son logement. 

«Est-ce l'homme? demanda celui-ci, qui lisait fort attentivement la \textit{Vie d'Alexandre} dans Plutarque. 

—Lui-même, capitaine, lui-même. 

—Très bien, montrez-le-moi.» 

Sur cet ordre assez impertinent, Peppino approcha si brusquement sa torche du visage de Danglars, que celui-ci se recula vivement pour ne point avoir les sourcils brûlés. Ce visage bouleversé offrait tous les symptômes d'une pâle et hideuse terreur. 

«Cet homme est fatigué, dit le capitaine, qu'on le conduise à son lit. 

—Oh! murmura Danglars, ce lit, c'est probablement un des cercueils qui creusent la muraille; ce sommeil, c'est la mort qu'un des poignards que je vois étinceler dans l'ombre va me procurer.» 

En effet, dans les profondeurs sombres de l'immense salle, on voyait se soulever, sur leurs couches d'herbes sèches ou de peaux de loup, les compagnons de cet homme qu'Albert de Morcerf avait trouvé lisant les \textit{Commentaires de César}, et que Danglars retrouvait lisant la \textit{Vie d'Alexandre}. 

Le banquier poussa un sourd gémissement et suivit son guide: il n'essaya ni de prier ni de crier. Il n'avait plus ni force, ni volonté, ni puissance, ni sentiment; il allait parce qu'on l'entraînait. 

Il heurta une marche, et, comprenant qu'il avait un escalier devant lui, il se baissa instinctivement pour ne pas se briser le front, et se trouva dans une cellule taillée en plein roc. 

Cette cellule était propre, bien que nue, sèche, quoique située sous la terre à une profondeur incommensurable. 

Un lit fait d'herbes sèches, recouvert de peaux de chèvre, était, non pas dressé, mais étendu dans un coin de cette cellule. Danglars, en l'apercevant, crut voir le symbole radieux de son salut. 

«Oh! Dieu soit loué! murmura-t-il, c'est un vrai lit!» 

C'était la seconde fois, depuis une heure, qu'il invoquait le nom de Dieu; cela ne lui était pas arrivé depuis dix ans. 

«\textit{Ecco}», dit le guide. 

Et poussant Danglars dans la cellule, il referma la porte sur lui. 

Un verrou grinça; Danglars était prisonnier. 

D'ailleurs, n'y eût-il pas eu de verrou, il eût fallu être saint Pierre et avoir pour guide un ange du ciel, pour passer au milieu de la garnison qui tenait les catacombes de Saint-Sébastien, et qui campait autour de son chef, dans lequel nos lecteurs ont certainement reconnu le fameux Luigi Vampa. 

Danglars aussi avait reconnu ce bandit, à l'existence duquel il n'avait pas voulu croire quand Morcerf essayait de le naturaliser en France. Non seulement il l'avait reconnu, mais aussi la cellule dans laquelle Morcerf avait été enfermé, et qui, selon toute probabilité, était le logement des étrangers. 

Ces souvenirs, sur lesquels au reste Danglars s'étendait avec une certaine joie, lui rendaient la tranquillité. Du moment où ils ne l'avaient pas tué tout de suite, les bandits n'avaient pas l'intention de le tuer du tout. 

On l'avait arrêté pour le voler, et comme il n'avait sur lui que quelques louis, on le rançonnerait. 

Il se rappela que Morcerf avait été taxé à quelque chose comme quatre mille écus; comme il s'accordait une apparence beaucoup plus importante que Morcerf, il fixa lui-même dans son esprit sa rançon à huit mille écus. 

Huit mille écus faisaient quarante-huit mille livres. 

Il lui restait encore quelque chose comme cinq millions cinquante mille francs. 

Avec cela on se tire d'affaire partout. 

Donc, à peu près certain de se tirer d'affaire, attendu qu'il n'y a pas d'exemple qu'on ait jamais taxé un homme à cinq millions cinquante mille livres, Danglars s'étendit sur son lit, où, après s'être retourné deux ou trois fois, il s'endormit avec la tranquillité du héros dont Luigi Vampa étudiait l'histoire. 