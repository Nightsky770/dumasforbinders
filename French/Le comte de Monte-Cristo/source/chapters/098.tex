\chapter{L'auberge de la Cloche et de la Bouteille}

\lettrine{E}{t} maintenant, laissons Mlle Danglars et son amie rouler sur la route de Bruxelles, et revenons au pauvre Andrea Cavalcanti, si malencontreusement arrêté dans l'essor de sa fortune. 

\zz
C'était, malgré son âge encore peu avancé, un garçon fort adroit et fort intelligent que M. Andrea Cavalcanti. 

Aussi, aux premières rumeurs qui pénétrèrent dans le salon, l'avons-nous vu par degrés se rapprocher de la porte, traverser une ou deux chambres, et enfin disparaître. 

Une circonstance que nous avons oublié de mentionner, et qui cependant ne doit pas être omise, c'est que dans l'une de ces deux chambres que traversa Cavalcanti était exposé le trousseau de la mariée, écrins de diamants, châles de cachemire, dentelles de Valenciennes, voiles d'Angleterre, tout ce qui compose enfin ce monde d'objets tentateurs dont le nom seul fait bondir de joie le cœur des jeunes filles et que l'on appelle la corbeille. 

Or, en passant par cette chambre, ce qui prouve que non seulement Andrea était un garçon fort intelligent et fort adroit, mais encore prévoyant, c'est qu'il se saisit de la plus riche de toutes les parures exposées. 

Muni de ce viatique, Andrea s'était senti de moitié plus léger pour sauter par la fenêtre et glisser entre les mains des gendarmes. 

Grand et découplé comme le lutteur antique, musculeux comme un Spartiate, Andrea avait fourni une course d'un quart d'heure, sans savoir où il allait, et dans le but seul de s'éloigner du lieu où il avait failli être pris. 

Parti de la rue du Mont-Blanc, il s'était retrouvé, avec cet instinct des barrières que les voleurs possèdent, comme le lièvre celui du gîte, au bout de la rue Lafayette. 

Là, suffoqué, haletant, il s'arrêta. 

Il était parfaitement seul, et avait à gauche le clos Saint-Lazare, vaste désert, et, à sa droite, Paris dans toute sa profondeur. 

«Suis-je perdu? se demanda-t-il. Non, si je puis fournir une somme d'activité supérieure à celle de mes ennemis. Mon salut est donc devenu tout simplement une question de myriamètres.» 

En ce moment il aperçut, montant du haut du faubourg Poissonnière, un cabriolet de régie dont le cocher, morne et fumant sa pipe, semblait vouloir regagner les extrémités du faubourg Saint-Denis où, sans doute, il faisait son séjour ordinaire. 

«Hé! l'ami! dit Benedetto. 

—Qu'y a-t-il, notre bourgeois? demanda le cocher. 

—Votre cheval est-il fatigué? 

—Fatigué! ah! bien oui! il n'a rien fait de toute la sainte journée. Quatre méchantes courses et vingt sous de pourboire, sept francs en tout, je dois en rendre dix au patron! 

—Voulez-vous à ces sept francs en ajouter vingt que voici, hein? 

—Avec plaisir, bourgeois; ce n'est pas à mépriser, vingt francs. Que faut-il faire pour cela? voyons. 

—Une chose bien facile, si votre cheval n'est pas fatigué toutefois. 

—Je vous dis qu'il ira comme un zéphir; le tout est de dire de quel côté il faut qu'il aille. 

—Du côté de Louvres. 

—Ah! ah! connu: pays du ratafia? 

—Justement. Il s'agit tout simplement de rattraper un de mes amis avec lequel je dois chasser demain à la Chapelle-en-Serval. Il devait m'attendre ici avec son cabriolet jusqu'à onze heures et demie: il est minuit; il se sera fatigué de m'attendre et sera parti tout seul. 

—C'est probable. 

—Eh bien, voulez-vous essayer de le rattraper? 

—Je ne demande pas mieux. 

—Mais si nous ne le rattrapons pas d'ici au Bourget vous aurez vingt francs; si nous ne le rattrapons pas d'ici à Louvres, trente. 

—Et si nous le rattrapons? 

—Quarante! dit Andrea qui avait eu un moment d'hésitation, mais qui avait réfléchi qu'il ne risquait rien de promettre. 

—Ça va! dit le cocher. Montez, et en route. Prrroum!\dots» 

Andrea monta dans le cabriolet qui, d'une course rapide, traversa le faubourg Saint-Denis, longea le faubourg Saint-Martin, traversa la barrière, et enfila l'interminable Villette. 

On n'avait garde de rejoindre cet ami chimérique; cependant de temps en temps, aux passants attardés ou aux cabarets qui veillaient encore, Cavalcanti s'informait d'un cabriolet vert attelé d'un cheval bai-brun; et, comme sur la route des Pays-Bas il circule bon nombre de cabriolets, que les neuf dixièmes des cabriolets sont verts, les renseignements pleuvaient à chaque pas. 

On venait toujours de le voir passer; il n'avait pas plus de cinq cents, de deux cents, de cent pas d'avance; enfin, on le dépassait, ce n'était pas lui. 

Une fois le cabriolet fut dépassé à son tour; c'était par une calèche rapidement emportée au galop de deux chevaux de poste. 

«Ah! se dit Cavalcanti, si j'avais cette calèche, ces deux bons chevaux, et surtout le passeport qu'il a fallu pour les prendre!» 

Et il soupira profondément. 

Cette calèche était celle qui emportait Mlle Danglars et Mlle d'Armilly. 

«En route! en route! dit Andrea, nous ne pouvons pas tarder à le rejoindre.» 

Et le pauvre cheval reprit le trot enragé qu'il avait suivi depuis la barrière, et arriva tout fumant à Louvres. 

«Décidément, dit Andrea, je vois bien que je ne rejoindrai pas mon ami et que je tuerai votre cheval. Ainsi donc, mieux vaut que je m'arrête. Voilà vos trente francs, je m'en vais coucher au Cheval-Rouge, et la première voiture dans laquelle je trouverai une place, je la prendrai. Bonsoir, mon ami.» 

Et Andrea, après avoir mis six pièces de cinq francs dans la main du cocher, sauta lestement sur le pavé de la route. 

Le cocher empocha joyeusement la somme et reprit au pas le chemin de Paris; Andrea feignit de gagner l'hôtel du Cheval-Rouge; mais après s'être arrêté un instant contre la porte, entendant le bruit du cabriolet qui allait se perdant à l'horizon, il reprit sa course, et d'un pas gymnastique fort relevé, il fournit une course de deux lieues. 

Là, il se reposa, il devait être tout près de la Chapelle-en-Serval, où il avait dit qu'il allait. 

Ce n'était pas la fatigue qui arrêtait Andrea Cavalcanti: c'était le besoin de prendre une résolution, c'était la nécessité d'adopter un plan. 

Monter en diligence, c'était impossible; prendre la poste, c'était également impossible. Pour voyager de l'une ou de l'autre façon un passeport est de toute nécessité. 

Demeurer dans le département de l'Oise, c'est-à-dire dans un des départements les plus découverts et les plus surveillés de France, c'était chose impossible encore, impossible surtout pour un homme expert comme Andrea en matière criminelle. 

Andrea s'assit sur les revers du fossé, laissa tomber sa tête entre ses deux mains et réfléchit. 

Dix minutes après, il releva la tête; sa résolution était arrêtée. 

Il couvrit de poussière tout un côté du paletot qu'il avait eu le temps de décrocher dans l'antichambre et de boutonner par-dessus sa toilette de bal, et, gagnant la Chapelle-en-Serval, il alla frapper hardiment à la porte de la seule auberge du pays. 

L'hôte vint ouvrir. 

«Mon ami, dit Andrea, j'allais de Mortefontaine à Senlis quand mon cheval, qui est un animal difficile, a fait un écart et m'a envoyé à dix pas. Il faut que j'arrive cette nuit à Compiègne sous peine de causer les plus graves inquiétudes à ma famille; avez-vous un cheval à louer?» 

Bon ou mauvais, un aubergiste a toujours un cheval. 

L'aubergiste de la Chapelle-en-Serval appela le garçon d'écurie, lui ordonna de seller \textit{le Blanc}, et réveilla son fils, enfant de sept ans, lequel devait monter en croupe du monsieur et ramener le quadrupède. 

Andrea donna vingt francs à l'aubergiste, et, en les tirant de sa poche, laissa tomber une carte de visite. 

Cette carte de visite était celle d'un de ses amis du Café de Paris; de sorte que l'aubergiste, lorsque Andrea fut parti et qu'il eut ramassé la carte tombée de sa poche, fut convaincu qu'il avait loué son cheval à M. le comte de Mauléon, rue Saint-Dominique, 25: c'était le nom et l'adresse qui se trouvaient sur la carte. 

\textit{Le Blanc} n'allait pas vite, mais il allait d'un pas égal et assidu: en trois heures et demie Andrea fit les neuf lieues qui le séparaient de Compiègne; quatre heures sonnaient à l'horloge de l'hôtel de ville lorsqu'il arriva sur la place où s'arrêtent les diligences. 

Il y a à Compiègne un excellent hôtel, dont se souviennent ceux-là mêmes qui n'y ont logé qu'une fois. 

Andréa, qui y avait fait une halte dans une de ses courses aux environs de Paris, se souvint de l'hôtel de la Cloche et de la Bouteille: il s'orienta, vit à la lueur d'un réverbère l'enseigne indicatrice, et, ayant congédié l'enfant, auquel il donna tout ce qu'il avait sur lui de petite monnaie, il alla frapper à la porte, réfléchissant avec beaucoup de justesse qu'il avait trois ou quatre heures devant lui, et que le mieux était de se prémunir, par un bon somme et un bon souper, contre les fatigues à venir. 

Ce fut un garçon qui vint ouvrir. 

«Mon ami, dit Andrea, je viens de Saint-Jean-au-Bois, où j'ai dîné; je comptais prendre la voiture qui passe à minuit; mais je me suis perdu comme un sot, et voilà quatre heures que je me promène dans la forêt. Donnez-moi donc une de ces jolies petites chambres qui donnent sur la cour, et faites-moi monter un poulet froid et une bouteille de vin de Bordeaux.» 

Le garçon n'eut aucun soupçon: Andrea parlait avec la plus parfaite tranquillité, il avait le cigare à la bouche et les mains dans les poches de son paletot; ses habits étaient élégants, sa barbe fraîche, ses bottes irréprochables; il avait l'air d'un voisin attardé, voilà tout. 

Pendant que le garçon préparait sa chambre, l'hôtesse se leva: Andrea l'accueillit avec son plus charmant sourire, et lui demanda s'il ne pourrait pas avoir le numéro 3, qu'il avait déjà eu à son dernier passage à Compiègne; malheureusement le numéro 3 était pris par un jeune homme qui voyageait avec sa sœur. 

Andrea parut désespéré; il ne se consola que lorsque l'hôtesse lui eut assuré que le numéro 7, qu'on lui préparait, avait absolument la même disposition que le numéro 3; et, tout en se chauffant les pieds et en causant des dernières courses de Chantilly, il attendit qu'on vînt lui annoncer que sa chambre était prête. 

Ce n'était pas sans raison qu'Andrea avait parlé de ces jolis appartements donnant sur la cour; la cour de l'hôtel de la Cloche, avec son triple rang de galeries qui lui donnait l'air d'une salle de spectacle, avec ses jasmins et ses clématites qui montent le long de ses colonnades, légères comme une décoration naturelle, est une des plus charmantes entrées d'auberge qui existent au monde. 

Le poulet était frais, le vin était vieux, le feu clair et pétillant: Andrea se surprit soupant d'aussi bon appétit que s'il ne lui était rien arrivé. 

Puis il se coucha, et s'endormit presque aussitôt de ce sommeil implacable que l'homme trouve toujours à vingt ans, même lorsqu'il a des remords. 

Or, nous sommes forcés d'avouer qu'Andrea aurait pu avoir des remords, mais qu'il n'en avait pas. 

Voici quel était le plan d'Andrea, plan qui lui avait donné la meilleure partie de sa sécurité. 

Avec le jour il se levait, sortait de l'hôtel après avoir rigoureusement payé ses comptes; gagnait la forêt, achetait, sous prétexte de faire des études de peinture, l'hospitalité d'un paysan; se procurait un costume de bûcheron et une cognée, dépouillait l'enveloppe du lion pour prendre celle de l'ouvrier; puis, les mains terreuses, les cheveux brunis par un peigne de plomb, le teint hâlé par une préparation dont ses anciens camarades lui avaient donné la recette, il gagnait, de forêt en forêt, la frontière la plus prochaine, marchant la nuit, dormant le jour dans les forêts ou dans les carrières, et ne s'approchant des endroits habités que pour acheter de temps en temps un pain. 

Une fois la frontière dépassée, Andrea faisait argent de ses diamants, réunissait le prix qu'il en tirait à une dizaine de billets de banque qu'il portait toujours sur lui en cas d'accident, et il se retrouvait encore à la tête d'une cinquantaine de mille livres, ce qui ne semblait pas à sa philosophie un pis-aller par trop rigoureux. 

D'ailleurs, il comptait beaucoup sur l'intérêt que les Danglars avaient à éteindre le bruit de leur mésaventure. 

Voilà pourquoi, outre la fatigue, Andrea dormit si vite et si bien. 

D'ailleurs, pour être réveillé plus matin, Andrea n'avait point fermé ses volets et s'était seulement contenté de pousser les verrous de sa porte et de tenir tout ouvert, sur sa table de nuit, certain couteau fort pointu dont il connaissait la trempe excellente et qui ne le quittait jamais. 

À sept heures du matin environ, Andrea fut éveillé par un rayon de soleil qui venait, tiède et brillant, se jouer sur son visage. 

Dans tout cerveau bien organisé, l'idée dominante et il y en a toujours une, l'idée dominante, disons-nous, est celle qui, après s'être endormie la dernière illumine la première encore le réveil de la pensée. 

Andrea n'avait pas entièrement ouvert les yeux que la pensée dominante le tenait déjà et lui soufflait à l'oreille qu'il avait dormi trop longtemps. 

Il sauta en bas de son lit et courut à sa fenêtre. 

Un gendarme traversait la cour. 

Le gendarme est un des objets les plus frappants qui existent au monde, même pour l'œil d'un homme sans inquiétude: mais pour une conscience timorée et qui a quelque motif de l'être, le jaune, le bleu et le blanc dont se compose son uniforme prennent des teintes effrayantes. 

«Pourquoi un gendarme?» se demanda Andrea. 

Tout à coup il se répondit à lui-même, avec cette logique que le lecteur a déjà dû remarquer en lui: 

«Un gendarme n'a rien qui doive étonner dans une hôtellerie; mais habillons-nous.» 

Et le jeune homme s'habilla avec une rapidité que n'avait pu lui faire perdre son valet de chambre pendant les quelques mois de la vie fashionable qu'il avait menée à Paris. 

«Bon, dit Andrea tout en s'habillant, j'attendrai qu'il soit parti, et quand il sera parti je m'esquiverai.» 

Et tout en disant ces mots, Andrea, rebotté et recravaté, gagna doucement sa fenêtre et souleva une seconde fois le rideau de mousseline. 

Non seulement le premier gendarme n'était point parti, mais encore le jeune homme aperçut un second uniforme bleu, jaune et blanc, au bas de l'escalier, le seul par lequel il pût descendre, tandis qu'un troisième, à cheval et le mousqueton au poing, se tenait en sentinelle à la grande porte de la rue, la seule par laquelle il pût sortir. 

Ce troisième gendarme était significatif au dernier point, car au-devant de lui s'étendait un demi-cercle de curieux qui bloquaient hermétiquement la porte de l'hôtel. 

«On me cherche! fut la première pensée d'Andrea. Diable!» 

La pâleur envahit le front du jeune homme; il regarda autour de lui avec anxiété. 

Sa chambre, comme toutes celles de cet étage, n'avait d'issue que sur la galerie extérieure, ouverte à tous les regards. 

«Je suis perdu!» fut sa seconde pensée. 

En effet, pour un homme dans la situation d'Andrea, l'arrestation signifiait: les assises, le jugement, la mort, la mort sans miséricorde et sans délai. 

Un instant il comprima convulsivement sa tête entre ses deux mains. 

Pendant cet instant il faillit devenir fou de peur. 

Mais bientôt, de ce monde de pensées s'entrechoquant dans sa tête, une pensée d'espérance jaillit; un pâle sourire se dessina sur ses lèvres blêmies et sur ses joues contractées. 

Il regarda autour de lui; les objets qu'il cherchait se trouvaient réunis sur le marbre d'un secrétaire: c'étaient une plume, de l'encre et du papier. 

Il trempa la plume dans l'encre et écrivit d'une main à laquelle il commanda d'être ferme les lignes suivantes, sur la première feuille du cahier: 

«Je n'ai point d'argent pour payer, mais je ne suis pas un malhonnête homme; je laisse en nantissement cette épingle qui vaut dix fois la dépense que j'ai faite. On me pardonnera de m'être échappé au point du jour, j'étais honteux!» 

Il tira son épingle de sa cravate et la posa sur le papier. 

Cela fait, au lieu de laisser ses verrous poussés, il les tira, entrebâilla même sa porte, comme s'il fût sorti de sa chambre en oubliant de la refermer, et se glissant dans la cheminée en homme accoutumé à ces sortes de gymnastiques, il attira à lui la devanture de papier représentant Achille chez Déidamie, effaça avec ses pieds même la trace de ses pas dans les cendres, et commença d'escalader le tuyau cambré qui lui offrait la seule voie de salut dans laquelle il espérât encore. 

En ce moment même, le premier gendarme qui avait frappé la vue d'Andrea montait l'escalier, précédé du commissaire de police, et soutenu par le second gendarme qui gardait le bas de l'escalier, lequel pouvait attendre lui-même du renfort de celui qui stationnait à la porte. 

Voici à quelle circonstance Andrea devait cette visite, qu'avec tant de peine il se disposait à recevoir. 

Au point du jour, les télégraphes avaient joué dans toutes les directions, et chaque localité, prévenue presque immédiatement, avait réveillé les autorités et lancé la force publique à la recherche du meurtrier de Caderousse. 

Compiègne, résidence royale; Compiègne, ville de chasse; Compiègne, ville de garnison, est abondamment pourvue d'autorités, de gendarmes et de commissaires de police; les visites avaient donc commencé aussitôt l'arrivée de l'ordre télégraphique, et l'hôtel de la Cloche et de la Bouteille étant le premier hôtel de la ville, on avait tout naturellement commencé par lui. 

D'ailleurs, d'après le rapport des sentinelles qui avaient pendant cette nuit été de garde à l'hôtel de ville (l'hôtel de ville est attenant à l'auberge de la Cloche), d'après le rapport des sentinelles, disons-nous, il avait été constaté que plusieurs voyageurs étaient descendus pendant la nuit à l'hôtel. 

La sentinelle qu'on avait relevée à six heures du matin se rappelait même, au moment où elle venait d'être placée, c'est-à-dire à quatre heures et quelques minutes, avoir vu un jeune homme monté sur un cheval blanc ayant un petit paysan en croupe, lequel jeune homme était descendu sur la place, avait congédié paysan et cheval, et était allé frapper à l'hôtel de la Cloche, qui s'était ouvert devant lui et s'était refermé sur lui. 

C'était sur ce jeune homme si singulièrement attardé que s'étaient arrêtés les soupçons. 

Or, ce jeune homme n'était autre qu'Andrea. 

C'était forts de ces données, que le commissaire de police et le gendarme, qui était un brigadier, s'acheminaient vers la porte d'Andrea; cette porte était entrebâillée. 

«Oh! oh! dit le brigadier, vieux renard nourri dans les ruses de l'état, mauvais indice qu'une porte ouverte! je l'aimerais mieux verrouillée à triple verrou!» 

En effet, la petite lettre et l'épingle laissées par Andrea sur la table confirmèrent ou plutôt appuyèrent la triste vérité. Andrea s'était enfui. 

Nous disons appuyèrent, parce que le brigadier n'était pas homme à se rendre sur une seule preuve. 

Il regarda autour de lui, plongea son œil sous le lit, dédoubla les rideaux, ouvrit les armoires, et enfin s'arrêta à la cheminée. 

Grâce aux précautions d'Andrea, aucune trace de son passage n'était demeurée dans les cendres. 

Cependant c'était une issue, et dans les circonstances où l'on se trouvait, toute issue devait être l'objet d'une sérieuse investigation. 

Le brigadier se fit donc apporter un fagot et de la paille, bourra la cheminée comme il eût fait d'un mortier, et y mit le feu. 

Le feu fit craquer les parois de brique; une colonne opaque de fumée s'élança par les conduits et monta vers le ciel comme le sombre jet d'un volcan, mais il ne vit point tomber le prisonnier, comme il s'y attendait. 

C'est qu'Andrea, dès sa jeunesse en lutte avec la société, valait bien un gendarme, ce gendarme fût-il élevé au grade respectable de brigadier; prévoyant donc l'incendie, il avait gagné le toit et se tenait blotti contre le tuyau. 

Un instant il eut quelque espoir d'être sauvé, car il entendit le brigadier appelant les deux gendarmes et leur criant tout haut: 

«Il n'y est plus.» 

Mais en allongeant doucement le cou, il vit que les deux gendarmes, au lieu de se retirer, comme la chose naturelle, sur une première annonce, il vit, disons-nous, qu'au contraire les deux gendarmes redoublaient d'attention. 

À son tour il regarda autour de lui: l'hôtel de ville, colossale bâtisse du seizième siècle, s'élevait comme un rempart sombre, à sa droite, et par les ouvertures du monument, on pouvait plonger dans tous les coins et recoins du toit, comme du haut d'une montagne on plonge dans la vallée. 

Andrea comprit qu'il allait incessamment voir paraître la tête du brigadier de gendarmerie à quelqu'une de ces ouvertures. 

Découvert, il était perdu; une chasse sur les toits ne lui présentait aucune chance de succès. 

Il résolut donc de redescendre, non point par le même chemin qu'il était venu, mais par un chemin analogue. 

Il chercha des yeux celle des cheminées de laquelle il ne voyait sortir aucune fumée, l'atteignit en rampant sur le toit, et disparut par son orifice sans avoir été vu de personne. 

Au même instant, une petite fenêtre de l'hôtel de ville s'ouvrait et donnait passage à la tête du brigadier de gendarmerie. 

Un instant cette tête demeura immobile comme un de ces reliefs de pierre qui décorent le bâtiment; puis avec un long soupir de désappointement la tête disparut. 

Le brigadier, calme et digne comme la loi dont il était le représentant, passa sans répondre à ces mille questions de la foule amassée sur la place, et rentra dans l'hôtel. 

«Eh bien? demandèrent à leur tour les deux gendarmes. 

—Eh bien, mes fils, répondit le brigadier, il faut que le brigand se soit véritablement distancé de nous ce matin à la bonne heure; mais nous allons envoyer sur la route de Villers-Cotterêts et de Noyon et fouiller la forêt, où nous le rattraperons indubitablement.» 

L'honorable fonctionnaire venait à peine, avec l'intonation qui est particulière aux brigadiers de gendarmerie, de donner le jour à cet adverbe sonore, lorsqu'un long cri d'effroi, accompagné de tintement redoublé d'une sonnette, retentit dans la cour de l'hôtel. 

«Oh! oh! qu'est-ce que cela? s'écria le brigadier. 

—Voilà un voyageur qui semble bien pressé, dit l'hôte. À quel numéro sonne-t-on? 

—Au numéro 3. 

—Courez-y, garçon!» 

En ce moment, les cris et le bruit de la sonnette redoublèrent. 

Le garçon prit sa course. 

«Non pas, dit le brigadier en arrêtant le domestique; celui qui sonne m'a l'air de demander autre chose que le garçon, et nous allons lui servir un gendarme. Qui loge au numéro 3? 

—Le petit jeune homme arrivé avec sa sœur cette nuit en chaise de poste, et qui a demandé une chambre à deux lits.» 

La sonnette retentit une troisième fois avec une intonation pleine d'angoisse. 

«À moi! monsieur le commissaire! cria le brigadier, suivez-moi et emboîtez le pas. 

—Un instant, dit l'hôte, à la chambre numéro 3, il y a deux escaliers: un extérieur, un intérieur. 

—Bon! dit le brigadier, je prendrai l'intérieur, c'est mon département. Les carabines sont-elles chargées? 

—Oui, brigadier. 

—Eh bien, veillez à l'extérieur, vous autres, et s'il veut fuir, feu dessus; c'est un grand criminel, à ce que dit le télégraphe.» 

Le brigadier, suivi du commissaire, disparut aussitôt dans l'escalier intérieur, accompagné de la rumeur que ses révélations sur Andrea venaient de faire naître dans la foule. 

Voilà ce qui était arrivé: 

Andrea était fort adroitement descendu jusqu'aux deux tiers de la cheminée, mais, arrivé là, le pied lui avait manqué, et, malgré l'appui de ses mains, il était descendu avec plus de vitesse et surtout plus de bruit qu'il n'aurait voulu. Ce n'eût été rien si la chambre eût été solitaire; mais par malheur elle était habitée. 

Deux femmes dormaient dans un lit, ce bruit les avait réveillées. 

Leurs regards s'étaient fixés vers le point d'où venait le bruit, et par l'ouverture de la cheminée elles avaient vu paraître un homme. 

C'était l'une de ces deux femmes, la femme blonde qui avait poussé ce terrible cri dont toute la maison avait retenti, tandis que l'autre qui était brune, s'élançant au cordon de la sonnette, avait donné l'alarme, en l'agitant de toutes ses forces. 

Andrea jouait, comme on le voit, de malheur. 

«Par pitié! cria-t-il, pâle, égaré, sans voir les personnes auxquelles il s'adressait, par pitié! n'appelez pas, sauvez-moi! je ne veux pas vous faire de mal. 

—Andrea l'assassin! cria l'une des deux jeunes femmes. 

—Eugénie! mademoiselle Danglars! murmura Cavalcanti, passant de l'effroi à la stupeur. 

—Au secours! au secours! cria Mlle d'Armilly reprenant la sonnette aux mains inertes d'Eugénie, et sonnant avec plus de force encore que sa compagne. 

—Sauvez-moi, on me poursuit! dit Andrea en joignant les mains; par pitié, par grâce, ne me livrez pas! 

—Il est trop tard, on monte, répondit Eugénie. 

—Eh bien, cachez-moi quelque part, vous direz que vous avez eu peur sans motif d'avoir peur; vous détournerez les soupçons, et vous m'aurez sauvé la vie.» 

Les deux femmes, serrées l'une contre l'autre s'enveloppant dans leurs couvertures, restèrent muettes à cette voix suppliante; toutes les appréhensions, toutes les répugnances se heurtaient dans leur esprit. 

«Eh bien, soit! dit Eugénie, reprenez le chemin par lequel vous êtes venu, malheureux; partez, et nous ne dirons rien. 

—Le voici! le voici! cria une voix sur le palier, le voici, je le vois!» 

En effet, le brigadier avait collé son œil à la serrure, et avait aperçu Andrea debout et suppliant. 

Un violent coup de crosse fit sauter la serrure, deux autres firent sauter les verrous; la porte brisée tomba en dedans. 

Andrea courut à l'autre porte, donnant sur la galerie de la cour, et l'ouvrit, prêt à se précipiter. 

Les deux gendarmes étaient là avec leurs carabines et le couchèrent en joue. 

Andrea s'était arrêté court; debout, pâle, le corps un peu renversé en arrière, il tenait son couteau inutile dans sa main crispée. 

«Fuyez donc! cria Mlle d'Armilly, dans le cœur de laquelle rentrait la pitié à mesure que l'effroi en sortait, fuyez donc! 

—Ou tuez-vous!» dit Eugénie du ton et avec la pose d'une de ces vestales qui, dans le cirque, ordonnaient avec le pouce, au gladiateur victorieux, d'achever son adversaire terrassé. 

Andrea frémit et regarda la jeune fille avec un sourire de mépris qui prouva que sa corruption ne comprenait point cette sublime férocité de l'honneur. 

«Me tuer! dit-il en jetant son couteau, pour quoi faire? 

—Mais, vous l'avez dit! s'écria Mlle Danglars, on vous condamnera à mort, on vous exécutera comme le dernier des criminels! 

—Bah! répliqua Cavalcanti en se croisant les bras, on a des amis.» 

Le brigadier s'avança vers lui le sabre au poing. 

«Allons, allons, dit Cavalcanti, rengainez, mon brave homme, ce n'est point la peine de faire tant d'esbroufe, puisque je me rends.» 

Et il tendit ses mains aux menottes. 

Les deux jeunes filles regardaient avec terreur cette hideuse métamorphose qui s'opérait sous leurs yeux, l'homme du monde dépouillant son enveloppe et redevenant l'homme du bagne. 

Andrea se retourna vers elles, et avec le sourire de l'impudence: 

«Avez-vous quelque commission pour monsieur votre père, mademoiselle Eugénie? dit-il, car, selon toute probabilité, je retourne à Paris.» 

Eugénie cacha sa tête dans ses deux mains. 

«Oh! oh! dit Andrea, il n'y a pas de quoi être honteuse, et je ne vous en veux pas d'avoir pris la poste pour courir après moi\dots N'étais-je pas presque votre mari?» 

Et sur cette raillerie Andrea sortit, laissant les deux fugitives en proie aux souffrances de la honte et aux commentaires de l'assemblée. 

Une heure après, vêtues toutes deux de leurs habits de femmes, elles montaient dans leur calèche de voyage. 

On avait fermé la porte de l'hôtel pour les soustraire aux premiers regards; mais il n'en fallut pas moins, quand cette porte fut ouverte, passer au milieu d'une double haie de curieux, aux yeux flamboyants, aux lèvres murmurantes. 

Eugénie baissa les stores; mais si elle ne voyait plus, elle entendait encore, et le bruit des ricanements arrivait jusqu'à elle. 

«Oh! pourquoi le monde n'est-il pas un désert?» s'écria-t-elle en se jetant dans les bras de Mlle d'Armilly, les yeux étincelants de cette rage qui faisait désirer à Néron que le monde romain n'eût qu'une seule tête, afin de la trancher d'un seul coup. 

Le lendemain, elles descendaient à l'hôtel de Flandre, à Bruxelles. 

Depuis la veille, Andrea était écroué à la Conciergerie. 