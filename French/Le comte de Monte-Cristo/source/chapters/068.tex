\chapter{Un bal d'été}

\lettrine{L}{e} même jour, vers l'heure où Mme Danglars faisait la séance que nous avons dite dans le cabinet de M. le procureur du roi, une calèche de voyage, entrant dans la rue du Helder, franchissait la porte du n°27 et s'arrêtait dans la cour. 

Au bout d'un instant la portière s'ouvrait, et Mme de Morcerf en descendait appuyée au bras de son fils. 

À peine Albert eut-il reconduit sa mère chez elle que, commandant un bain et ses chevaux, après s'être mis aux mains de son valet de chambre, il se fit conduire aux Champs-Élysées, chez le comte de Monte-Cristo. 

Le comte le reçut avec son sourire habituel. C'était une étrange chose: jamais on ne paraissait faire un pas en avant dans le cœur ou dans l'esprit de cet homme. Ceux qui voulaient, si l'on peut dire cela, forcer le passage de son intimité trouvaient un mur. 

Morcerf, qui accourait à lui les bras ouverts, laissa, en le voyant et malgré son sourire amical, tomber ses bras, et osa tout au plus lui tendre la main.  

De son côté, Monte-Cristo la lui toucha, comme il faisait toujours, mais sans la lui serrer. 

«Eh bien, me voilà, dit-il, cher comte. 

—Soyez le bienvenu. 

—Je suis arrivé depuis une heure. 

—De Dieppe? 

—Du Tréport. 

—Ah! c'est vrai. 

—Et ma première visite est pour vous. 

—C'est charmant de votre part, dit Monte-Cristo comme il eût dit toute autre chose. 

—Eh bien, voyons, quelles nouvelles? 

—Des nouvelles! vous demandez cela à moi, à un étranger!» 

—Je m'entends: quand je demande quelles nouvelles, je demande si vous avez fait quelque chose pour moi? 

—M'aviez-vous donc chargé de quelque commission? dit Monte-Cristo en jouant l'inquiétude. 

—Allons, allons, dit Albert, ne simulez pas l'indifférence. On dit qu'il y a des avertissements sympathiques qui traversent la distance: eh bien! au Tréport, j'ai reçu mon coup électrique; vous avez, sinon travaillé pour moi, du moins pensé à moi. 

—Cela est possible, dit Monte-Cristo. J'ai en effet pensé à vous; mais le courant magnétique dont j'étais le conducteur agissait, je l'avoue, indépendamment de ma volonté. 

—Vraiment! Contez-moi cela, je vous prie. 

—C'est facile, M. Danglars a dîné chez moi. 

—Je le sais bien, puisque c'est pour fuir sa présence que nous sommes partis, ma mère et moi. 

—Mais il a dîné avec M. Andrea Cavalcanti. 

—Votre prince italien? 

—N'exagérons pas. M. Andrea se donne seulement le titre de vicomte. 

—Se donne, dites-vous? 

—Je dis: se donne. 

—Il ne l'est donc pas? 

—Eh! le sais-je, moi? Il se le donne, je le lui donne, on le lui donne; n'est-ce pas comme s'il l'avait? 

—Homme étrange que vous faites, allez! Eh bien?  

—Eh bien, quoi? 

—M. Danglars a donc dîné ici? 

—Oui. 

—Avec votre vicomte Andrea Cavalcanti? 

—Avec le vicomte Andrea Cavalcanti, le marquis son père, Mme Danglars, M. et Mme de Villefort, des gens charmants, M. Debray, Maximilien Morrel, et puis qui encore\dots attendez donc\dots ah! M. de Château-Renaud. 

—On a parlé de moi? 

—On n'en a pas dit un mot. 

—Tant pis. 

—Pourquoi cela? Il me semble que, si l'on vous a oublié, on n'a fait, en agissant ainsi, que ce que vous désiriez! 

—Mon cher comte, si l'on n'a point parlé de moi, c'est qu'on y pensait beaucoup, et alors je suis désespéré. 

—Que vous importe, puisque Mlle Danglars n'était point au nombre de ceux qui y pensaient ici! Ah! il est vrai qu'elle pouvait y penser chez elle. 

—Oh! quant à cela, non, j'en suis sûr: ou si elle y pensait, c'est certainement de la même façon que je pense à elle.  

—Touchante sympathie! dit le comte. Alors vous vous détestez? 

—Écoutez, dit Morcerf, si Mlle Danglars était femme à prendre en pitié le martyre que je ne souffre pas pour elle et m'en récompenser en dehors des convenances matrimoniales arrêtées entre nos deux familles, cela m'irait à merveille. Bref, je crois que Mlle Danglars serait une maîtresse charmante, mais comme femme, diable\dots. 

—Ainsi, dit Monte-Cristo en riant, voilà votre façon de penser sur votre future? 

—Oh! mon Dieu! oui, un peu brutale, c'est vrai mais exacte du moins. Or, puisqu'on ne peut faire de ce rêve une réalité; comme pour arriver à un certain but il faut que Mlle Danglars devienne ma femme c'est-à-dire qu'elle vive avec moi, qu'elle pense près de moi, qu'elle chante près de moi, qu'elle fasse des vers et de la musique à dix pas de moi, et cela pendant tout le temps de ma vie, alors je m'épouvante. Une maîtresse, mon cher comte, cela se quitte, mais une femme, peste! c'est autre chose, cela se garde éternellement, de près ou de loin c'est-à-dire. Or, c'est effrayant de garder toujours Mlle Danglars, fût-ce même de loin. 

—Vous êtes difficile, vicomte. 

—Oui, car souvent je pense à une chose impossible. 

—À laquelle? 

—À trouver pour moi une femme comme mon père en a trouvé une pour lui.» 

Monte-Cristo pâlit et regarda Albert en jouant avec des pistolets magnifiques dont il faisait rapidement crier les ressorts. 

«Ainsi, votre père a été bien heureux, dit-il. 

—Vous savez mon opinion sur ma mère, monsieur le comte: un ange du ciel; voyez-la encore belle, spirituelle toujours, meilleure que jamais. J'arrive du Tréport; pour tout autre fils, eh! mon Dieu! accompagner sa mère serait une complaisance ou une corvée mais, moi, j'ai passé quatre jours en tête-à-tête avec elle, plus satisfait, plus reposé, plus poétique, vous le dirais-je, que si j'eusse emmené au Tréport la reine Mab ou Titania. 

—C'est une perfection désespérante, et vous donnez à tous ceux qui vous entendent de graves envies de rester célibataires. 

—Voilà justement, reprit Morcerf, pourquoi, sachant qu'il existe au monde une femme accomplie, je ne me soucie pas d'épouser Mlle Danglars. Avez-vous quelquefois remarqué comme notre égoïsme revêt de couleurs brillantes tout ce qui nous appartient? Le diamant qui chatoyait à la vitre de Marlé ou de Fossin devient bien plus beau depuis qu'il est notre diamant; mais si l'évidence vous force à reconnaître qu'il en est d'une eau plus pure, et que vous soyez condamné à porter éternellement ce diamant inférieur à un autre, comprenez-vous la souffrance? 

—Mondain! murmura le comte. 

—Voilà pourquoi je sauterai de joie le jour où Mlle Eugénie s'apercevra que je ne suis qu'un chétif atome et que j'ai à peine autant de cent mille francs qu'elle a de millions.»  

Monte-Cristo sourit. 

«J'avais bien pensé à autre chose, continua Albert; Franz aime les choses excentriques, j'ai voulu le rendre malgré lui amoureux de Mlle Danglars; mais à quatre lettres que je lui ai écrites dans le plus affriandant des styles, Franz m'a imperturbablement répondu: «Je suis excentrique, c'est vrai, mais mon excentricité ne va pas jusqu'à reprendre ma parole quand je l'ai donnée.» 

—Voilà ce que j'appelle le dévouement de l'amitié: donner à un autre la femme dont on ne voudrait soi-même qu'à titre de maîtresse.» 

Albert sourit. 

«À propos, continua-t-il, il arrive, ce cher Franz; mais peu vous importe, vous ne l'aimez pas, je crois? 

—Moi! dit Monte-Cristo; eh! mon cher vicomte, où donc avez-vous vu que je n'aimais pas M. Franz? J'aime tout le monde. 

—Et je suis compris dans tout le monde\dots merci. 

—Oh! ne confondons pas, dit Monte-Cristo: j'aime tout le monde à la manière dont Dieu nous ordonne d'aimer notre prochain, chrétiennement; mais je ne hais bien que de certaines personnes. Revenons à M. Franz d'Épinay. Vous dites donc qu'il arrive. 

—Oui, mandé par M. de Villefort, aussi enragé, à ce qu'il paraît, de marier Mlle Valentine que M. Danglars est enragé de marier Mlle Eugénie. Décidément, il paraît que c'est un état des plus fatigants que celui de père de grandes filles; il me semble que cela leur donne la fièvre, et que leur pouls bat quatre-vingt-dix fois à la minute, jusqu'à ce qu'ils en soient débarrassés. 

—Mais M. d'Épinay ne vous ressemble pas, lui; il prend son mal en patience. 

—Mieux que cela, il le prend au sérieux; il met des cravates blanches et parle déjà de sa famille. Il a au reste pour les Villefort une grande considération. 

—Méritée, n'est-ce pas? 

—Je le crois. M. de Villefort a toujours passé pour un homme sévère, mais juste. 

—À la bonne heure, dit Monte-Cristo, en voilà un au moins que vous ne traitez pas comme ce pauvre M. Danglars. 

—Cela tient peut-être à ce que je ne suis pas forcé d'épouser sa fille, répondit Albert en riant. 

—En vérité, mon cher monsieur, dit Monte-Cristo, vous êtes d'une fatuité révoltante. 

—Moi? 

—Oui, vous. Prenez donc un cigare. 

—Bien volontiers. Et pourquoi suis-je fat? 

—Mais parce que vous êtes là à vous défendre, à vous débattre d'épouser Mlle Danglars. Eh! mon Dieu! laissez aller les choses, et ce n'est peut-être pas vous qui retirerez votre parole le premier. 

—Bah! fit Albert avec de grands yeux. 

—Eh! sans doute, monsieur le vicomte, on ne vous mettra pas de force le cou dans les portes, que diable! Voyons, sérieusement, reprit Monte-Cristo en changeant d'intonation, avez-vous envie de rompre? 

—Je donnerais cent mille francs pour cela. 

—Eh bien, soyez heureux: M. Danglars est prêt à en donner le double pour atteindre au même but. 

—Est-ce bien vrai, ce bonheur-là? dit Albert, qui cependant en disant cela ne put empêcher qu'un imperceptible nuage passât sur son front. Mais, mon cher comte, M. Danglars a donc des raisons? 

—Ah! te voilà bien, nature orgueilleuse et égoïste! À la bonne heure, je retrouve l'homme qui veut trouer l'amour-propre d'autrui à coups de hache, et qui crie quand on troue le sien avec une aiguille. 

—Non! mais c'est qu'il me semble que M. Danglars\dots. 

—Devait être enchanté de vous n'est-ce pas? Eh bien, M. Danglars est un homme de mauvais goût, c'est convenu, et il est encore plus enchanté d'un autre\dots. 

—De qui donc?  

—Je ne sais pas, moi; étudiez, regardez, saisissez les allusions à leur passage, et faites-en votre profit. 

—Bon, je comprends; écoutez, ma mère\dots non! pas ma mère, je me trompe, mon père a eu l'idée de donner un bal. 

—Un bal dans ce moment-ci de l'année? 

—Les bals d'été sont à la mode. 

—Ils n'y seraient pas, que la comtesse n'aurait qu'à vouloir, et elle les y mettrait. 

—Pas mal; vous comprenez, ce sont des bals pur sang; ceux qui restent à Paris dans le mois de juillet sont de vrais Parisiens. Voulez-vous vous charger d'une invitation pour MM. Cavalcanti? 

—Dans combien de jours a lieu votre bal? 

—Samedi. 

—M. Cavalcanti père sera parti. 

—Mais M. Cavalcanti fils demeure. Voulez-vous vous charger d'amener M. Cavalcanti fils? 

—Écoutez, vicomte, je ne le connais pas. 

—Vous ne le connaissez pas? 

—Non; je l'ai vu pour la première fois il y a trois ou quatre jours, et je n'en réponds en rien. 

—Mais vous le recevez bien, vous! 

—Moi, c'est autre chose; il m'a été recommandé par un brave abbé qui peut lui-même avoir été trompé. Invitez-le directement, à merveille, mais ne me dites pas de vous le présenter; s'il allait plus tard épouser Mlle Danglars, vous m'accuseriez de manège, et vous voudriez vous couper la gorge avec moi; d'ailleurs, je ne sais pas si j'irai moi-même. 

—Où? 

—À votre bal. 

—Pourquoi n'y viendrez-vous point? 

—D'abord parce que vous ne m'avez pas encore invité. 

—Je viens exprès pour vous apporter votre invitation moi-même. 

—Oh! c'est trop charmant; mais je puis en être empêché. 

—Quand je vous aurai dit une chose, vous serez assez aimable pour nous sacrifier tous les empêchements. 

—Dites. 

—Ma mère vous en prie. 

—Mme la comtesse de Morcerf? reprit Monte-Cristo en tressaillant.  

—Ah! comte, dit Albert, je vous préviens que Mme de Morcerf cause librement avec moi; et si vous n'avez pas senti craquer en vous ces fibres sympathiques dont je vous parlais tout à l'heure, c'est que ces fibres-là vous manquent complètement, car pendant quatre jours nous n'avons parlé que de vous. 

—De moi? En vérité vous me comblez! 

—Écoutez, c'est le privilège de votre emploi: quand on est un problème vivant. 

—Ah! je suis donc aussi un problème pour votre mère? En vérité, je l'aurais crue trop raisonnable pour se livrer à de pareils écarts d'imagination! 

—Problème, mon cher comte, problème pour tous, pour ma mère comme pour les autres; problème accepté, mais non deviné, vous demeurez toujours à l'état d'énigme: rassurez-vous. Ma mère seulement demande toujours comment il se fait que vous soyez si jeune. Je crois qu'au fond, tandis que la comtesse G\dots vous prend pour Lord Ruthwen, ma mère vous prend pour Cagliostro ou le comte de Saint-Germain. La première fois que vous viendrez voir Mme de Morcerf, confirmez-la dans cette opinion. Cela ne vous sera pas difficile, vous avez la pierre philosophale de l'un et l'esprit de l'autre. 

—Je vous remercie de m'avoir prévenu, dit le comte en souriant, je tâcherai de me mettre en mesure de faire face à toutes les suppositions. 

—Ainsi vous viendrez samedi? 

—Puisque Mme de Morcerf m'en prie. 

—Vous êtes charmant. 

—Et M. Danglars? 

—Oh! il a déjà reçu la triple invitation; mon père s'en est chargé. Nous tâcherons aussi d'avoir le grand d'Aguesseau, M. de Villefort; mais on en désespère. 

—Il ne faut jamais désespérer de rien, dit le proverbe. 

—Dansez-vous, cher comte? 

—Moi? 

—Oui, vous. Qu'y aurait-il d'étonnant à ce que vous dansassiez? 

—Ah! en effet, tant qu'on n'a pas franchi la quarantaine\dots. Non, je ne danse pas; mais j'aime à voir danser. Et Mme de Morcerf, danse-t-elle? 

—Jamais, non plus; vous causerez, elle a tant envie de causer avec vous! 

—Vraiment? 

—Parole d'honneur! et je vous déclare que vous êtes le premier homme pour lequel ma mère ait manifesté cette curiosité.» 

Albert prit son chapeau et se leva; le comte le reconduisit jusqu'à la porte. 

«Je me fais un reproche, dit-il en l'arrêtant au haut du perron. 

—Lequel? 

—J'ai été indiscret, je ne devais pas vous parler de M. Danglars. 

—Au contraire, parlez-m'en encore, parlez-m'en souvent, parlez-m'en toujours; mais de la même façon. 

—Bien! vous me rassurez. À propos, quand arrive M. d'Épinay? 

—Mais dans cinq ou six jours au plus tard. 

—Et quand se marie-t-il? 

—Aussitôt l'arrivée de M. et de Mme de Saint-Méran. 

—Amenez-le-moi donc quand il sera à Paris. Quoique vous prétendiez que je ne l'aime pas, je vous déclare que je serai heureux de le voir. 

—Bien, vos ordres seront exécutés, seigneur. 

—Au revoir! 

—À samedi, en tout cas, bien sûr, n'est-ce pas? 

—Comment donc! c'est parole donnée.» 

Le comte suivit des yeux Albert en le saluant de la main. Puis, quand il fut remonté dans son phaéton, il se retourna, et trouvant Bertuccio derrière lui: 

«Eh bien? demanda-t-il. 

—Elle est allée au Palais, répondit l'intendant. 

—Elle y est restée longtemps? 

—Une heure et demie. 

—Et elle est rentrée chez elle? 

—Directement. 

—Eh bien, mon cher monsieur Bertuccio, dit le comte, si j'ai maintenant un conseil à vous donner, c'est d'aller voir en Normandie si vous ne trouverez pas cette petite terre dont je vous ai parlée.» 

Bertuccio salua, et, comme ses désirs étaient en parfaite harmonie avec l'ordre qu'il avait reçu, il partit le soir même. 