\chapter{La vendetta} 

\lettrine[ante=«]{D}{'où} monsieur le comte désire-t-il que je reprenne les choses? demanda Bertuccio. 

\zz
—Mais d'où vous voudrez, dit Monte-Cristo, puisque je ne sais absolument rien. 

—Je croyais cependant que M. l'abbé Busoni avait dit à Votre Excellence\dots. 

—Oui, quelques détails sans doute, mais sept ou huit ans ont passé là-dessus, et j'ai oublié tout cela. 

—Alors je puis donc, sans crainte d'ennuyer Votre Excellence\dots. 

—Allez, monsieur Bertuccio, allez, vous me tiendrez lieu de journal du soir. 

—Les choses remontent à 1815. 

—Ah! ah! fit Monte-Cristo, ce n'est pas hier, 1815. 

—Non, monsieur, et cependant les moindres détails me sont aussi présents à la mémoire que si nous étions seulement au lendemain. J'avais un frère, un frère aîné, qui était au service de l'empereur. Il était devenu lieutenant dans un régiment composé entièrement de Corses. Ce frère était mon unique ami; nous étions restés orphelins, moi à cinq ans, lui à dix-huit, il m'avait élevé comme si j'eusse été son fils. En 1814, sous les Bourbons, il s'était marié; l'Empereur revint de l'île d'Elbe, mon frère reprit aussitôt du service, et, blessé légèrement à Waterloo, il se retira avec l'armée derrière la Loire. 

—Mais c'est l'histoire des Cent-Jours que vous me faites là, monsieur Bertuccio, dit le comte, et elle est déjà faite, si je ne me trompe. 

—Excusez-moi, Excellence, mais ces premiers détails sont nécessaires, et vous m'avez promis d'être patient. 

—Allez! allez! je n'ai qu'une parole. 

—Un jour, nous reçûmes une lettre, il faut vous dire que nous habitions le petit village de Rogliano, à l'extrémité du cap Corse: cette lettre était de mon frère; il nous disait que l'armée était licenciée et qu'il revenait par Châteauroux, Clermont-Ferrand, le Puy et Nîmes; si j'avais quelque argent, il me priait de le lui faire tenir à Nîmes, chez un aubergiste de notre connaissance, avec lequel j'avais quelques relations. 

—De contrebande, reprit Monte-Cristo. 

—Eh! mon Dieu! monsieur le comte, il faut bien. 

—Certainement, continuez donc.  

—J'aimais tendrement mon frère, je vous l'ai dit, Excellence; aussi je résolus non pas de lui envoyer l'argent, mais de le lui porter moi-même. Je possédais un millier de francs, j'en laissai cinq cents à Assunta, c'était ma belle-sœur; je pris les cinq cents autres, et je me mis en route pour Nîmes. C'était chose facile, j'avais ma barque, un chargement à faire en mer; tout secondait mon projet. Mais le chargement fait, le vent devint contraire, de sorte que nous fûmes quatre ou cinq jours sans pouvoir entrer dans le Rhône. Enfin nous y parvînmes; nous remontâmes jusqu'à Arles; je laissai la barque entre Bellegarde et Beaucaire, et je pris le chemin de Nîmes. 

—Nous arrivons, n'est-ce pas?  

—Oui, monsieur: excusez-moi, mais, comme Votre Excellence le verra, je ne lui dis que les choses absolument nécessaires. Or, c'était le moment où avaient lieu les fameux massacres du Midi. Il y avait là deux ou trois brigands que l'on appelait Trestaillon, Truphemy et Graffan, qui égorgeaient dans les rues tous ceux qu'on soupçonnait de bonapartisme. Sans doute, monsieur le comte a entendu parler de ces assassinats? 

—Vaguement, j'étais fort loin de la France à cette époque. Continuez. 

—En entrant à Nîmes, on marchait littéralement dans le sang; à chaque pas on rencontrait des cadavres: les assassins, organisés par bandes, tuaient, pillaient et brûlaient.  

«À la vue de ce carnage, un frisson me prit, non pas pour moi; moi, simple pêcheur corse, je n'avais pas grand-chose à craindre; au contraire, ce temps-là, c'était notre bon temps, à nous autres contrebandiers, mais pour mon frère, pour mon frère soldat de l'Empire, revenant de l'armée de la Loire avec son uniforme et ses épaulettes, et qui par conséquent, avait tout à craindre. 

«Je courus chez notre aubergiste. Mes pressentiments ne m'avaient pas trompé: mon frère était arrivé la veille à Nîmes, et à la porte même de celui à qui il venait demander l'hospitalité, il avait été assassiné. 

«Je fis tout au monde pour connaître les meurtriers; mais personne n'osa me dire leurs noms, tant ils étaient redoutés. Je songeai alors à cette justice française, dont on m'avait tant parlé, qui ne redoute rien, elle, et je me présentai chez le procureur du roi. 

—Et ce procureur du roi se nommait Villefort? demanda négligemment Monte-Cristo. 

—Oui, Excellence: il venait de Marseille, où il avait été substitut. Son zèle lui avait valu de l'avancement. Il était un des premiers, disait-on, qui eussent annoncé au gouvernement le débarquement de l'île d'Elbe. 

—Donc, reprit Monte-Cristo, vous vous présentâtes chez lui. 

«—Monsieur, lui dis-je, mon frère a été assassiné hier dans les rues de Nîmes, je ne sais point par qui, mais c'est votre mission de le savoir. Vous êtes ici chef de la justice, et c'est à la justice de venger ceux qu'elle n'a pas su défendre. 

«—Et qu'était votre frère? demanda le procureur du roi\dots. 

«—Lieutenant au bataillon corse. 

«—Un soldat de l'usurpateur, alors? 

«—Un soldat des armées françaises. 

«—Eh bien, répliqua-t-il, il s'est servi et il a péri par l'épée. 

«—Vous vous trompez, monsieur; il a péri par le poignard. 

«—Que voulez-vous que j'y fasse? répondit le magistrat. 

«—Mais je vous l'ai dit: je veux que vous le vengiez. 

«—Et de qui? 

«—De ses assassins. 

«—Est-ce que je les connais, moi? 

«—Faites-les chercher.  

«—Pour quoi faire? Votre frère aura eu quelque querelle et se sera battu en duel. Tous ces anciens soldats se portent à des excès qui leur réussissaient sous l'Empire, mais qui tournent mal pour eux maintenant; or, nos gens du Midi n'aiment ni les soldats, ni les excès. 

«—Monsieur, repris-je, ce n'est pas pour moi que je vous prie. Moi, je pleurerai ou je me vengerai voilà tout; mais mon pauvre frère avait une femme. S'il m'arrivait malheur à mon tour, cette pauvre créature mourrait de faim, car le travail seul de mon frère la faisait vivre. Obtenez pour elle une petite pension du gouvernement. 

«—Chaque révolution a ses catastrophes, répondit M. de Villefort; votre frère a été victime de celle-ci, c'est un malheur, et le gouvernement ne doit rien à votre famille pour cela. Si nous avions à juger toutes les vengeances que les partisans de l'usurpateur ont exercées sur les partisans du roi quand à leur tour ils disposaient du pouvoir, votre frère serait peut-être aujourd'hui condamné à mort. Ce qui s'accomplit est chose toute naturelle, car c'est la loi des représailles. 

«—Eh quoi! monsieur, m'écriai-je, il est possible que vous me parliez ainsi, vous, un magistrat!\dots 

«—Tous ces Corses sont fous, ma parole d'honneur! répondit M. de Villefort, et ils croient encore que leur compatriote est empereur. Vous vous trompez de temps, mon cher; il fallait venir me dire cela il y a deux mois. Aujourd'hui il est trop tard; allez-vous-en donc, et si vous ne vous en allez pas, moi, je vais vous faire reconduire. 

«Je le regardai un instant pour voir si par une nouvelle supplication il y avait quelque chose à espérer. Cet homme était de pierre. Je m'approchai de lui: 

«—Eh bien, lui dis-je à demi-voix, puisque vous connaissez les Corses, vous devez savoir comment ils tiennent leur parole. Vous trouvez qu'on a bien fait de tuer mon frère qui était bonapartiste, parce que vous êtes royaliste, vous; eh bien, moi, qui suis bonapartiste aussi, je vous déclare une chose: c'est que je vous tuerai, vous. À partir de ce moment je vous déclare la vendetta; ainsi, tenez-vous bien, et gardez-vous de votre mieux, car la première fois que nous nous trouverons face à face, c'est que votre dernière heure sera venue. 

«Et là-dessus, avant qu'il fût revenu de sa surprise, j'ouvris la porte et je m'enfuis. 

—Ah! ah! dit Monte-Cristo, avec votre honnête figure, vous faites de ces choses-là, monsieur Bertuccio, et à un procureur du roi, encore! Fi donc! et savait-il au moins ce que cela voulait dire ce mot \textit{vendetta}? 

—Il le savait si bien qu'à partir de ce moment il ne sortit plus seul et se calfeutra chez lui, me faisant chercher partout. Heureusement j'étais si bien caché qu'il ne put me trouver. Alors la peur le prit, il trembla de rester plus longtemps à Nîmes; il sollicita son changement de résidence, et, comme c'était en effet un homme influent, il fut nommé à Versailles; mais, vous le savez, il n'y a pas de distance pour un Corse qui a juré de se venger de son ennemi, et sa voiture, si bien menée qu'elle fût, n'a jamais eu plus d'une demi-journée d'avance sur moi, qui cependant la suivis à pied. 

«L'important n'était pas de le tuer, cent fois j'en avais trouvé l'occasion; mais il fallait le tuer sans être découvert et surtout sans être arrêté. Désormais je ne m'appartenais plus: j'avais à protéger et à nourrir ma belle-sœur. Pendant trois mois je guettai M. de Villefort; pendant trois mois il ne fit pas un pas, une démarche, une promenade, que mon regard ne le suivît là où il allait. Enfin, je découvris qu'il venait mystérieusement à Auteuil: je le suivis encore et je le vis entrer dans cette maison où nous sommes, seulement, au lieu d'entrer comme tout le monde par la grande porte de la rue, il venait soit à cheval, soit en voiture, laissait voiture ou cheval à l'auberge, et entrait par cette petite porte que vous voyez là.» 

Monte-Cristo fit de la tête un signe qui prouvait qu'au milieu de l'obscurité il distinguait en effet l'entrée indiquée par Bertuccio. 

«Je n'avais plus besoin de rester à Versailles, je me fixai à Auteuil et je m'informai. Si je voulais le prendre, c'était évidemment là qu'il me fallait tendre mon piège. 

«La maison appartenait, comme le concierge l'a dit à Votre Excellence, à M. de Saint-Méran, beau-père de Villefort. M. de Saint-Méran habitait Marseille; par conséquent, cette campagne lui était inutile; aussi disait-on qu'il venait de la louer à une jeune veuve que l'on ne connaissait que sous le nom de la baronne. 

«En effet, un soir, en regardant par-dessus le mur, je vis une femme jeune et belle qui se promenait seule dans ce jardin, que nulle fenêtre étrangère ne dominait; elle regardait fréquemment du côté de la petite porte, et je compris que ce soir-là elle attendait M. de Villefort. Lorsqu'elle fut assez près de moi pour que malgré l'obscurité je pusse distinguer ses traits, je vis une belle jeune femme de dix-huit à dix-neuf ans, grande et blonde. Comme elle était en simple peignoir et que rien ne gênait sa taille, je pus remarquer qu'elle était enceinte et que sa grossesse même paraissait avancée. 

«Quelques moments après, on ouvrit la petite porte; un homme entra; la jeune femme courut le plus vite qu'elle put à sa rencontre, ils se jetèrent dans les bras l'un de l'autre, s'embrassèrent tendrement et regagnèrent ensemble la maison. 

«Cet homme, c'était M. de Villefort. Je jugeai qu'en sortant, surtout s'il sortait la nuit, il devait traverser seul le jardin dans toute sa longueur. 

—Et, demanda le comte, avez-vous su depuis le nom de cette femme? 

—Non, Excellence, répondit Bertuccio; vous allez voir que je n'eus pas le temps de l'apprendre. 

—Continuez. 

—Ce soir-là, reprit Bertuccio, j'aurais pu tuer peut-être le procureur du roi; mais je ne connaissais pas encore assez le jardin dans tous ses détails. Je craignis de ne pas le tuer raide, et, si quelqu'un accourait à ses cris, de ne pouvoir fuir. Je remis la partie au prochain rendez-vous, et, pour que rien ne m'échappât, je pris une petite chambre donnant sur la rue que longeait le mur du jardin. 

«Trois jours après, vers sept heures du soir, je vis sortir de la maison un domestique à cheval qui prit au galop le chemin qui conduisait à la route de Sèvres; je présumai qu'il allait à Versailles. Je ne me trompais pas. Trois heures après, l'homme revint tout couvert de poussière; son message était terminé. 

«Dix minutes après, un autre homme à pied, enveloppé d'un manteau, ouvrit la petite porte du jardin, qui se referma sur lui. 

«Je descendis rapidement. Quoique je n'eusse pas vu le visage de Villefort, je le reconnus au battement de mon cœur: je traversai la rue, je gagnai une borne placée à l'angle du mur et à l'aide de laquelle j'avais regardé une première fois dans le jardin. 

«Cette fois je ne me contentai pas de regarder, je tirai mon couteau de ma poche, je m'assurai que la pointe était bien affilée, et je sautai par-dessus le mur. 

«Mon premier soin fut de courir à la porte; il avait laissé la clef en dedans, en prenant la simple précaution de donner un double tour à la serrure. 

Rien n'entravait donc ma fuite de ce côté-là. Je me mis à étudier les localités. Le jardin formait un carré long, une pelouse de fin gazon anglais s'étendait au milieu, aux angles de cette pelouse étaient des massifs d'arbres au feuillage touffu et tout entremêlé de fleurs d'automne. 

«Pour se rendre de la maison à la petite porte, ou de la petite porte à la maison, soit qu'il entrât, soit qu'il sortît, M. de Villefort était obligé de passer près d'un de ces massifs.  

«On était à la fin de septembre; le vent soufflait avec force; un peu de lune pâle, et voilée à chaque instant par de gros nuages qui glissaient rapidement au ciel, blanchissait le sable des allées qui conduisaient à la maison, mais ne pouvait percer l'obscurité de ces massifs touffus dans lesquels un homme pouvait demeurer caché sans qu'il y eût crainte qu'on ne l'aperçût. 

«Je me cachai dans celui le plus près duquel devait passer Villefort; à peine y étais-je, qu'au milieu des bouffées de vent qui courbaient les arbres au-dessus de mon front, je crus distinguer comme des gémissements. Mais vous savez, ou plutôt vous ne savez pas, monsieur le comte, que celui qui attend le moment de commettre un assassinat croit toujours entendre pousser des cris sourds dans l'air. Deux heures s'écoulèrent pendant lesquelles, à plusieurs reprises, je crus entendre les mêmes gémissements. Minuit sonna. 

«Comme le dernier son vibrait encore lugubre et retentissant, j'aperçus une lueur illuminant les fenêtres de l'escalier dérobé par lequel nous sommes descendus tout à l'heure. 

«La porte s'ouvrit, et l'homme au manteau reparut. C'était le moment terrible; mais depuis si longtemps je m'étais préparé à ce moment, que rien en moi ne faiblit: je tirai mon couteau, je l'ouvris et je me tins prêt. 

«L'homme au manteau vint droit à moi, mais à mesure qu'il avançait dans l'espace découvert, je croyais remarquer qu'il tenait une arme de la main droite: j'eus peur, non pas d'une lutte, mais d'un insuccès. Lorsqu'il fut à quelques pas de moi seulement, je reconnus que ce que j'avais pris pour une arme n'était rien autre chose qu'une bêche. 

«Je n'avais pas encore pu deviner dans quel but M. de Villefort tenait une bêche à la main, lorsqu'il s'arrêta sur la lisière du massif, jeta un regard autour de lui, et se mit à creuser un trou dans la terre. Ce fut alors que je m'aperçus qu'il y avait quelque chose dans son manteau, qu'il venait de déposer sur la pelouse pour être plus libre de ses mouvements. 

«Alors, je l'avoue, un peu de curiosité se glissa dans ma haine: je voulus voir ce que venait faire là Villefort; je restai immobile, sans haleine, j'attendis. 

«Puis une idée m'était venue, qui se confirma en voyant le procureur du roi tirer de son manteau un petit coffre long de deux pieds et large de six à huit pouces. 

«Je le laissai déposer le coffre dans le trou, sur lequel il repoussa la terre; puis, sur cette terre fraîche, il appuya ses pieds pour faire disparaître la trace de l'œuvre nocturne. Je m'élançai alors sur lui et je lui enfonçai mon couteau dans la poitrine en lui disant: 

«—Je suis Giovanni Bertuccio! ta mort pour mon frère, ton trésor pour sa veuve: tu vois bien que ma vengeance est plus complète que je ne l'espérais.  

«Je ne sais s'il entendit ces paroles; je ne le crois pas, car il tomba sans pousser un cri; je sentis les flots de son sang rejaillir brûlants sur mes mains et sur mon visage; mais j'étais ivre, j'étais en délire; ce sang me rafraîchissait au lieu de me brûler. En une seconde, j'eus déterré le coffret à l'aide de la bêche; puis, pour qu'on ne vît pas que je l'avais enlevé, je comblai à mon tour le trou, je jetai la bêche par-dessus le mur, je m'élançai par la porte, que je fermai à double tour en dehors et dont j'emportai la clef. 

—Bon! dit Monte-Cristo, c'était, à ce que je vois, un petit assassinat doublé de vol. 

—Non, Excellence, répondit Bertuccio, c'était une vendetta suivie de restitution. 

—Et la somme était ronde, au moins? 

—Ce n'était pas de l'argent. 

—Ah! oui, je me rappelle, dit Monte-Cristo n'avez-vous pas parlé d'un enfant? 

—Justement, Excellence. Je courus jusqu'à la rivière, je m'assis sur le talus, et, pressé de savoir ce que contenait le coffre, je fis sauter la serrure avec mon couteau. 

«Dans un lange de fine batiste était enveloppé un enfant qui venait de naître; son visage empourpré, ses mains violettes annonçaient qu'il avait dû succomber à une asphyxie causée par des ligaments naturels roulés autour de son cou; cependant, comme il n'était pas froid encore, j'hésitai à le jeter dans cette eau qui coulait à mes pieds. En effet, au bout d'un instant je crus sentir un léger battement vers la région du cœur; je dégageai son cou du cordon qui l'enveloppait, et, comme j'avais été infirmier à l'hôpital de Bastia, je fis ce qu'aurait pu faire un médecin en pareille circonstance, c'est-à-dire que je lui insufflai courageusement de l'air dans les poumons, qu'après un quart d'heure d'efforts inouïs je le vis respirer, et j'entendis un cri s'échapper de sa poitrine. 

«À mon tour, je jetai un cri, mais un cri de joie. Dieu ne me maudit donc pas, me dis-je, puisqu'il permet que je rende la vie à une créature humaine en échange de la vie que j'ai ôtée à une autre! 

—Et que fîtes-vous donc de cet enfant? demanda Monte-Cristo; c'était un bagage assez embarrassant pour un homme qui avait besoin de fuir. 

—Aussi n'eus-je point un instant l'idée de le garder. Mais je savais qu'il existait à Paris un hospice où on reçoit ces pauvres créatures. En passant à la barrière, je déclarai avoir trouvé cet enfant sur la route et je m'informai. Le coffre était là qui faisait foi; les langes de batiste indiquaient que l'enfant appartenait à des parents riches; le sang dont j'étais couvert pouvait aussi bien appartenir à l'enfant qu'à tout autre individu. On ne me fit aucune objection; on m'indiqua l'hospice, qui était situé tout au bout de la rue d'Enfer, et, après avoir pris la précaution de couper le lange en deux, de manière qu'une des deux lettres qui le marquaient continuât d'envelopper le corps de l'enfant, je déposai mon fardeau dans le tour, je sonnai et je m'enfuis à toutes jambes. Quinze jours après, j'étais de retour à Rogliano, et je disais à Assunta: 

«—Console-toi, ma sœur; Israël est mort, mais je l'ai vengé. 

«Alors elle me demanda l'explication de ces paroles, et je lui racontai tout ce qui s'était passé. 

«—Giovanni, me dit Assunta, tu aurais dû rapporter cet enfant, nous lui eussions tenu lieu des parents qu'il a perdus, nous l'eussions appelé Benedetto, et en faveur de cette bonne action Dieu nous eût bénis effectivement. 

«Pour toute réponse je lui donnai la moitié de lange que j'avais conservée, afin de faire réclamer l'enfant si nous étions plus riches. 

—Et de quelles lettres était marqué ce lange? demanda Monte-Cristo. 

—D'un H et d'un N surmontés d'un tortil de baron. 

—Je crois, Dieu me pardonne! que vous vous servez de termes de blason, monsieur Bertuccio! Où diable avez-vous fait vos études héraldiques? 

—À votre service, monsieur le comte, où l'on apprend toutes choses. 

—Continuez, je suis curieux de savoir deux choses. 

—Lesquelles, monseigneur? 

—Ce que devint ce petit garçon; ne m'avez-vous pas dit que c'était un petit garçon, monsieur Bertuccio? 

—Non, Excellence; je ne me rappelle pas avoir parlé de cela. 

—Ah! je croyais avoir entendu, je me serai trompé.  

—Non, vous ne vous êtes pas trompé, car c'était effectivement un petit garçon; mais Votre Excellence désirait, disait-elle, savoir deux choses: quelle est la seconde? 

—La seconde était le crime dont vous étiez accusé quand vous demandâtes un confesseur, et que l'abbé Busoni alla vous trouver sur cette demande dans la prison de Nîmes. 

—Peut-être ce récit sera-t-il bien long, Excellence. 

—Qu'importe? il est dix heures à peine, vous savez que je ne dors pas, et je suppose que de votre côté vous n'avez pas grande envie de dormir.» 

Bertuccio s'inclina et reprit sa narration. 

«Moitié pour chasser les souvenirs qui m'assiégeaient, moitié pour subvenir aux besoins de la pauvre veuve, je me remis avec ardeur à ce métier de contrebandier, devenu plus facile par le relâchement des lois qui suit toujours les révolutions. Les côtes du Midi, surtout, étaient mal gardées, à cause des émeutes éternelles qui avaient lieu, tantôt à Avignon, tantôt à Nîmes, tantôt à Uzès. Nous profitâmes de cette espèce de trêve qui nous était accordée par le gouvernement pour lier des relations avec tout le littoral. Depuis l'assassinat de mon frère dans les rues de Nîmes, je n'avais pas voulu rentrer dans cette ville. Il en résulta que l'aubergiste avec lequel nous faisions des affaires, voyant que nous ne voulions plus venir à lui, était venu à nous et avait fondé une succursale de son auberge sur la route de Bellegarde à Beaucaire, à l'enseigne du \textit{Pont du Gard}. Nous avions ainsi, soit du côté d'Aigues-Mortes, soit aux Martigues, soit à Bouc, une douzaine d'entrepôts où nous déposions nos marchandises et où, au besoin, nous trouvions un refuge contre les douaniers et les gendarmes. C'est un métier qui rapporte beaucoup que celui de contrebandier, lorsqu'on y applique une certaine intelligence secondée par quelque vigueur; quant à moi, je vivais dans les montagnes ayant maintenant une double raison de craindre gendarmes et douaniers, attendu que toute comparution devant les juges pouvait amener une enquête, que cette enquête est toujours une excursion dans le passé, et que dans mon passé, à moi, on pouvait rencontrer maintenant quelque chose plus grave que des cigares entrés en contrebande ou des barils d'eau-de-vie circulant sans laissez-passer. Aussi, préférant mille fois la mort à une arrestation, j'accomplissais des choses étonnantes, et qui, plus d'une fois, me donnèrent cette preuve, que le trop grand soin que nous prenons de notre corps est à peu près le seul obstacle à la réussite de ceux de nos projets qui ont besoin d'une décision rapide et d'une exécution vigoureuse et déterminée. En effet une fois qu'on a fait le sacrifice de sa vie, on n'est plus l'égal des autres hommes, ou plutôt les autres hommes ne sont plus vos égaux, et quiconque a pris cette résolution sent, à l'instant même, décupler ses forces et s'agrandir son horizon. 

—De la philosophie, monsieur Bertuccio! interrompit le comte; mais vous avez donc fait un peu de tout dans votre vie? 

—Oh! pardon, Excellence!  

—Non! non! c'est que la philosophie à dix heures et demie du soir, c'est un peu tard. Mais je n'ai pas d'autre observation à faire, attendu que je la trouve exacte, ce qu'on ne peut pas dire de toutes les philosophies. 

—Mes courses devinrent donc de plus en plus étendues, de plus en plus fructueuses. Assunta était ménagère, et notre petite fortune s'arrondissait. Un jour que je partais pour une course: 

«—Va, dit-elle, et à ton retour je te ménage une surprise. 

«Je l'interrogeais inutilement: elle ne voulut rien me dire et je partis.  

«La course dura près de six semaines; nous avions été à Lucques charger de l'huile, et à Livourne prendre des cotons anglais; notre débarquement se fit sans événement contraire, nous réalisâmes nos bénéfices et nous revînmes tout joyeux. 

«En rentrant dans la maison, la première chose que je vis à l'endroit le plus apparent de la chambre d'Assunta dans un berceau somptueux relativement au reste de l'appartement, fut un enfant de sept à huit mois. Je jetai un cri de joie. Les seuls moments de tristesse que j'eusse éprouvés depuis l'assassinat du procureur du roi m'avaient été causés par l'abandon de cet enfant. Il va sans dire que de remords de l'assassinat lui-même je n'en avais point eu. 

«La pauvre Assunta avait tout deviné: elle avait profité de mon absence, et, munie de la moitié du lange, ayant inscrit, pour ne point l'oublier, le jour et l'heure précis où l'enfant avait été déposé à l'hospice, elle était partie pour Paris et avait été elle-même le réclamer. Aucune objection ne lui avait été faite, et l'enfant lui avait été remis. 

«Ah! j'avoue, monsieur le comte, qu'en voyant cette pauvre créature dormant dans son berceau, ma poitrine se gonfla, et que des larmes sortirent de mes yeux. 

«—En vérité, Assunta, m'écriai-je, tu es une digne femme, et la Providence te bénira. 

—Ceci, dit Monte-Cristo, est moins exact que votre philosophie; il est vrai que ce n'est que la foi. 

—Hélas! Excellence, reprit Bertuccio, vous avez bien raison, et ce fut cet enfant lui-même que Dieu chargea de ma punition. Jamais nature plus perverse ne se déclara plus prématurément, et cependant on ne dira pas qu'il fut mal élevé, car ma sœur le traitait comme le fils d'un prince; c'était un garçon d'une figure charmante, avec des yeux d'un bleu clair comme ces tons de faïences chinoises qui s'harmonisent si bien avec le blanc laiteux du ton général; seulement ses cheveux d'un blond trop vif donnaient à sa figure un caractère étrange, qui doublait la vivacité de son regard et la malice de son sourire. Malheureusement il y a un proverbe qui dit que le roux est tout bon ou tout mauvais; le proverbe ne mentit pas pour Benedetto, et dès sa jeunesse il se montra tout mauvais. Il est vrai aussi que la douceur de sa mère encouragea ses premiers penchants; l'enfant, pour qui ma pauvre sœur allait au marché de la ville, située à quatre ou cinq lieues de là, acheter les premiers fruits et les sucreries les plus délicates, préférait aux oranges de Palma et aux conserves de Gênes les châtaignes volées au voisin en franchissant les haies, ou les pommes séchées dans son grenier, tandis qu'il avait à sa disposition les châtaignes et les pommes de notre verger. 

«Un jour, Benedetto pouvait avoir cinq ou six ans, le voisin Wasilio, qui, selon les habitudes de notre pays, n'enfermait ni sa bourse ni ses bijoux, car, monsieur le comte le sait aussi bien que personne, en Corse il n'y a pas de voleurs, le voisin Wasilio se plaignit à nous qu'un louis avait disparu de sa bourse; on crut qu'il avait mal compté, mais lui prétendait être sûr de son fait. Ce jour-là Benedetto avait quitté la maison dès le matin, et c'était une grande inquiétude chez nous, lorsque le soir nous le vîmes revenir traînant un singe qu'il avait trouvé, disait-il, tout enchaîné au pied d'un arbre. 

«Depuis un mois la passion du méchant enfant, qui ne savait quelle chose s'imaginer, était d'avoir un singe. Un bateleur qui était passé à Rogliano, et qui avait plusieurs de ces animaux dont les exercices l'avaient fort réjoui, lui avait inspiré sans doute cette malheureuse fantaisie. 

«—On ne trouve pas de singe dans nos bois, lui dis-je, et surtout de singe enchaîné; avoue-moi donc comment tu t'es procuré celui-ci. 

«Benedetto soutint son mensonge, et l'accompagna de détails qui faisaient plus d'honneur à son imagination qu'à sa véracité; je m'irritai, il se mit à rire; je le menaçai, il fit deux pas en arrière. 

«—Tu ne peux pas me battre, dit-il, tu n'en as pas le droit, tu n'es pas mon père. 

«Nous ignorâmes toujours qui lui avait révélé ce fatal secret, que nous lui avions caché cependant avec tant de soin; quoi qu'il en soit, cette réponse, dans laquelle l'enfant se révéla tout entier, m'épouvanta presque, mon bras levé retomba effectivement sans toucher le coupable; l'enfant triompha, et cette victoire lui donna une telle audace qu'à partir de ce moment tout l'argent d'Assunta, dont l'amour semblait augmenter pour lui à mesure qu'il en était moins digne, passa en caprices qu'elle ne savait pas combattre, en folies qu'elle n'avait pas le courage d'empêcher. Quand j'étais à Rogliano, les choses marchaient encore assez convenablement; mais dès que j'étais parti, c'était Benedetto qui était devenu le maître de la maison, et tout tournait à mal. Âgé de onze ans à peine, tous ses camarades étaient choisis parmi des jeunes gens de dix-huit ou vingt ans, les plus mauvais sujets de Bastia et de Corte, et déjà, pour quelques espiègleries qui méritaient un nom plus sérieux, la justice nous avait donné des avertissements. 

«Je fus effrayé; toute information pouvait avoir des suites funestes: j'allais justement être forcé de m'éloigner de la Corse pour une expédition importante. Je réfléchis longtemps, et, dans le pressentiment d'éviter quelque malheur, je me décidai à emmener Benedetto avec moi. J'espérais que la vie active et rude de contrebandier, la discipline sévère du bord, changeraient ce caractère prêt à se corrompre, s'il n'était pas déjà affreusement corrompu. 

«Je tirai donc Benedetto à part et lui fis la proposition de me suivre, en entourant cette proposition de toutes les promesses qui peuvent séduire un enfant de douze ans. 

«Il me laissa aller jusqu'au bout, et lorsque j'eus finis, éclatant de rire:  

«—Êtes-vous fou, mon oncle? dit-il (il m'appelait ainsi quand il était de belle humeur); moi changer la vie que je mène contre celle que vous menez, ma bonne et excellente paresse contre l'horrible travail que vous vous êtes imposé! passer la nuit au froid, le jour au chaud; se cacher sans cesse; quand on se montre recevoir des coups de fusil, et tout cela pour gagner un peu d'argent! L'argent, j'en ai tant que j'en veux! mère Assunta m'en donne quand je lui en demande. Vous voyez donc bien que je serais un imbécile si j'acceptais ce que vous me proposez. 

«J'étais stupéfait de cette audace et de ce raisonnement. Benedetto retourna jouer avec ses camarades, et je le vis de loin me montrant à eux comme un idiot. 

—Charmant enfant! murmura Monte-Cristo. 

—Oh! s'il eût été à moi, répondit Bertuccio, s'il eût été mon fils, ou tout au moins mon neveu, je l'eusse bien ramené au droit sentier, car la conscience donne la force. Mais l'idée que j'allais battre un enfant dont j'avais tué le père me rendait toute correction impossible. Je donnai de bons conseils à ma sœur, qui, dans nos discussions, prenait sans cesse la défense du petit malheureux, et comme elle m'avoua que plusieurs fois des sommes assez considérables lui avaient manqué, je lui indiquai un endroit où elle pouvait cacher notre petit trésor. Quant à moi, ma résolution était prise. Benedetto savait parfaitement lire, écrire et compter, car lorsqu'il voulait s'adonner par hasard au travail, il apprenait en un jour ce que les autres apprenaient en une semaine. Ma résolution, dis-je, était prise; je devais l'engager comme secrétaire sur quelque navire au long cours, et, sans le prévenir de rien, le faire prendre un beau matin et le faire transporter à bord; de cette façon, et en le recommandant au capitaine, tout son avenir dépendait de lui. Ce plan arrêté, je partis pour la France. 

«Toutes nos opérations devaient cette fois s'exécuter dans le golfe du Lion, et ces opérations devenaient de plus en plus difficiles, car nous étions en 1829. La tranquillité était parfaitement rétablie, et par conséquent le service des côtes était redevenu plus régulier et plus sévère que jamais. Cette surveillance était encore augmentée momentanément par la foire de Beaucaire, qui venait de s'ouvrir. 

«Les commencements de notre expédition s'exécutèrent sans encombre. Nous amarrâmes notre barque, qui avait un double fond dans lequel nous cachions nos marchandises de contrebande, au milieu d'une quantité de bateaux qui bordaient les deux rives du Rhône, depuis Beaucaire jusqu'à Arles. Arrivés là, nous commençâmes à décharger nuitamment nos marchandises prohibées, et à les faire passer dans la ville par l'intermédiaire des gens qui étaient en relations avec nous, ou des aubergistes chez lesquels nous faisions des dépôts. Soit que la réussite nous eût rendus imprudents, soit que nous ayons été trahis, un soir, vers les cinq heures de l'après-midi, comme nous allions nous mettre à goûter, notre petit mousse accourut tout effaré en disant qu'il avait vu une escouade de douaniers se diriger de notre côté. Ce n'était pas précisément l'escouade qui nous effrayait: à chaque instant, surtout dans ce moment-là, des compagnies entières rôdaient sur les bords du Rhône; mais c'étaient les précautions qu'au dire de l'enfant cette escouade prenait pour ne pas être vue. En un instant nous fûmes sur pied, mais il était déjà trop tard; notre barque, évidemment l'objet des recherches, était entourée. Parmi les douaniers, je remarquai quelques gendarmes; et, aussi timide à la vue de ceux-ci que j'étais brave ordinairement à la vue de tout autre corps militaire, je descendis dans la cale, et, me glissant par un sabord, je me laissai couler dans le fleuve, puis je nageai entre deux eaux, ne respirant qu'à de longs intervalles, si bien que je gagnai sans être vu une tranchée que l'on venait de faire, et qui communiquait du Rhône au canal qui se rend de Beaucaire à Aigues-Mortes. Une fois arrivé là, j'étais sauvé, car je pouvais suivre sans être vu cette tranchée. Je gagnai donc le canal sans accident. Ce n'était pas par hasard et sans préméditation que j'avais suivi ce chemin; j'ai déjà parlé à Votre Excellence d'un aubergiste de Nîmes qui avait établi sur la route de Bellegarde à Beaucaire une petite hôtellerie. 

—Oui, dit Monte-Cristo, je me souviens parfaitement. Ce digne homme, si je ne me trompe, était même votre associé. 

—C'est cela, répondit Bertuccio; mais depuis sept ou huit ans, il avait cédé son établissement à un ancien tailleur de Marseille qui, après s'être ruiné dans son état, avait voulu essayer de faire sa fortune dans un autre. Il va sans dire que les petits arrangements que nous avions faits avec le premier propriétaire furent maintenus avec le second; c'était donc à cet homme que je comptais demander asile. 

—Et comment se nommait cet homme? demanda le comte, qui paraissait commencer à reprendre quelque intérêt au récit de Bertuccio. 

—Il s'appelait Gaspard Caderousse, il était marié à une femme du village de la Carconte, et que nous ne connaissions pas sous un autre nom que celui de son village; c'était une pauvre femme atteinte de la fièvre des marais, qui s'en allait mourant de langueur. Quant à l'homme, c'était un robuste gaillard de quarante à quarante-cinq ans, qui plus d'une fois nous avait, dans des circonstances difficiles, donné des preuves de sa présence d'esprit et de son courage. 

—Et vous dites, demanda Monte-Cristo, que ces choses se passaient vers l'année\dots. 

—1829, monsieur le comte. 

—En quel mois? 

—Au mois de juin. 

—Au commencement ou à la fin. 

—C'était le 3 au soir. 

—Ah! fit Monte-Cristo, le 3 juin 1829\dots Bien, continuez. 

—C'était donc à Caderousse que je comptais demander asile; mais, comme d'habitude, et même dans les circonstances ordinaires, nous n'entrions pas chez lui par la porte qui donnait sur la route, je résolus de ne pas déroger à cette coutume, j'enjambai la haie du jardin, je me glissai en rampant à travers les oliviers rabougris et les figuiers sauvages, et je gagnai, dans la crainte que Caderousse n'eût quelque voyageur dans son auberge, une espèce de soupente dans laquelle plus d'une fois j'avais passé la nuit aussi bien que dans le meilleur lit. Cette soupente n'était séparée de la salle commune du rez-de-chaussée de l'auberge que par une cloison en planches dans laquelle des jours avaient été ménagés à notre intention, afin que de là nous pussions guetter le moment opportun de faire reconnaître que nous étions dans le voisinage. Je comptais, si Caderousse était seul, le prévenir de mon arrivée, achever chez lui le repas interrompu par l'apparition des douaniers, et profiter de l'orage qui se préparait pour regagner les bords du Rhône et m'assurer de ce qu'étaient devenus la barque et ceux qui la montaient. Je me glissai donc dans la soupente et bien m'en prit, car à ce moment même Caderousse rentrait chez lui avec un inconnu. 

«Je me tins coi et j'attendis, non point dans l'intention de surprendre les secrets de mon hôte, mais parce que je ne pouvais faire autrement; d'ailleurs, dix fois même chose était déjà arrivée. 

«L'homme qui accompagnait Caderousse était évidemment étranger au Midi de la France: c'était un de ces négociants forains qui viennent vendre des bijoux à la foire de Beaucaire et qui, pendant un mois que dure cette foire, où affluent des marchands et des acquéreurs de toutes les parties de l'Europe, font quelquefois pour cent ou cent cinquante mille francs d'affaires. 

«Caderousse entra vivement et le premier. Puis voyant la salle d'en bas vide comme d'habitude et simplement gardée par son chien, il appela sa femme. 

«—Hé! la Carconte, dit-il, ce digne homme de prêtre ne nous avait pas trompés; le diamant était bon. 

«Une exclamation joyeuse se fit entendre, et presque aussitôt l'escalier craqua sous un pas alourdi par la faiblesse et la maladie. 

«—Qu'est-ce que tu dis? demanda la femme plus pâle qu'une morte. 

«—Je dis que le diamant était bon, que voilà monsieur, un des premiers bijoutiers de Paris, qui est prêt à nous en donner cinquante mille francs. Seulement, pour être sûr que le diamant est bien à nous, il demande que tu lui racontes, comme je l'ai déjà fait, de quelle façon miraculeuse le diamant est tombé entre nos mains. En attendant, monsieur, asseyez-vous, s'il vous plaît, et comme le temps est lourd, je vais aller chercher de quoi vous rafraîchir. 

«Le bijoutier examinait avec attention l'intérieur de l'auberge et la pauvreté bien visible de ceux qui allaient lui vendre un diamant qui semblait sortir de l'écrin d'un prince. 

«—Racontez, madame, dit-il, voulant sans doute profiter de l'absence du mari pour qu'aucun signe de la part de celui-ci n'influençât la femme, et pour voir si les deux récits cadreraient bien l'un avec l'autre. 

«—Eh! mon Dieu! dit la femme avec volubilité, c'est une bénédiction du ciel à laquelle nous étions loin de nous attendre. Imaginez-vous, mon cher monsieur, que mon mari a été lié en 1814 ou 1815 avec un marin nommé Edmond Dantès: ce pauvre garçon, que Caderousse avait complètement oublié ne l'a pas oublié, lui, et lui a laissé en mourant le diamant que vous venez de voir. 

«—Mais comment était-il devenu possesseur de ce diamant? demanda le bijoutier. Il l'avait donc avant d'entrer en prison?  

«—Non, monsieur, répondit la femme, mais en prison il a fait, à ce qu'il paraît, la connaissance d'un Anglais très riche; et comme en prison son compagnon de chambre est tombé malade, et que Dantès en prit les mêmes soins que si c'était son frère, l'Anglais, en sortant de captivité, laissa au pauvre Dantès, qui, moins heureux que lui, est mort en prison, ce diamant qu'il nous a légué à son tour en mourant, et qu'il a chargé le digne abbé qui est venu ce matin de nous remettre. 

«—C'est bien la même chose, murmura le bijoutier, et, au bout du compte l'histoire peut être vraie, tout invraisemblable qu'elle paraisse au premier abord. Il n'y a donc que le prix sur lequel nous ne sommes pas d'accord. 

«—Comment! pas d'accord, dit Caderousse; je croyais que vous aviez consenti au prix que j'en demandais. 

«—C'est-à-dire, reprit le bijoutier, que j'en ai offert quarante mille francs. 

«—Quarante mille! s'écria la Carconte; nous ne le donnerons certainement pas pour ce prix-là. L'abbé nous a dit qu'il valait cinquante mille francs, et sans la monture encore. 

«—Et comment se nommait cet abbé? demanda l'infatigable questionneur. 

«—L'abbé Busoni, répondit la femme.  

«—C'était donc un étranger? 

«—C'était un Italien des environs de Mantoue, je crois. 

«—Montrez-moi ce diamant, reprit le bijoutier, que je le revoie une seconde fois; souvent on juge mal les pierres à une première vue.» 

«Caderousse tira de sa poche un petit étui de chagrin noir, l'ouvrit et le passa au bijoutier. À la vue du diamant, qui était gros comme une petite noisette, je me le rappelle comme si je le voyais encore, les yeux de la Carconte étincelèrent de cupidité. 

—Et que pensiez-vous de tout cela, monsieur l'écouteur aux portes? demanda Monte-Cristo; ajoutiez-vous foi à cette belle fable? 

—Oui, Excellence; je ne regardais pas Caderousse comme un méchant homme, et je le croyais incapable d'avoir commis un crime ou même un vol. 

—Cela fait plus honneur à votre cœur qu'à votre expérience, monsieur Bertuccio. Aviez-vous connu cet Edmond Dantès dont il était question? 

—Non, Excellence, je n'en avais jamais entendu parler jusqu'alors, et je n'en ai jamais entendu reparler depuis qu'une seule fois par l'abbé Busoni lui-même, quand je le vis dans les prisons de Nîmes.  

—Bien! continuez. 

—Le bijoutier prit la bague des mains de Caderousse, et tira de sa poche une petite pince en acier et une petite paire de balances de cuivre; puis, écartant les crampons d'or qui retenaient la pierre dans la bague, il fit sortir le diamant de son alvéole, et le pesa minutieusement dans les balances. 

«—J'irai jusqu'à quarante-cinq mille francs, dit-il, mais je ne donnerai pas un sou avec; d'ailleurs, comme c'était ce que valait le diamant, j'ai pris juste cette somme sur moi. 

«—Oh! qu'à cela ne tienne, dit Caderousse, je retournerai avec vous à Beaucaire pour chercher les cinq autres mille francs. 

«—Non, dit le bijoutier en rendant l'anneau et le diamant à Caderousse; non, cela ne vaut pas davantage, et encore je suis fâché d'avoir offert cette somme, attendu qu'il y a dans la pierre un défaut que je n'avais pas vu d'abord; mais n'importe, je n'ai qu'une parole, j'ai dit quarante-cinq mille francs, je ne m'en dédis pas. 

«—Au moins remettez le diamant dans la bague», dit aigrement la Carconte. 

«—C'est juste, dit le bijoutier. 

«Et il replaça la pierre dans le chaton.  

«—Bon, bon, bon, dit Caderousse remettant l'étui dans sa poche, on le vendra à un autre. 

«—Oui, reprit le bijoutier, mais un autre ne sera pas si facile que moi; un autre ne se contentera pas des renseignements que vous m'avez donnés; il n'est pas naturel qu'un homme comme vous possède un diamant de cinquante mille francs; il ira prévenir les magistrats, il faudra retrouver l'abbé Busoni, et les abbés qui donnent des diamants de deux mille louis sont rares; la justice commencera par mettre la main dessus, on vous enverra en prison, et si vous êtes reconnu innocent, qu'on vous mette dehors après trois ou quatre mois de captivité, la bague se sera égarée au greffe, ou l'on vous donnera une pierre fausse qui vaudra trois francs au lieu d'un diamant qui en vaut cinquante mille, cinquante-cinq mille peut-être, mais que, vous en conviendrez, mon brave homme, on court certains risques à acheter.» 

«Caderousse et sa femme s'interrogèrent du regard. 

«—Non, dit Caderousse, nous ne sommes pas assez riches pour perdre cinq mille francs. 

«—Comme vous voudrez, mon cher ami, dit le bijoutier; j'avais cependant, comme vous le voyez, apporté de la belle monnaie. 

«Et il tira d'une de ses poches une poignée d'or qu'il fit briller aux yeux éblouis de l'aubergiste, et, de l'autre, un paquet de billets de banque.  

«Un rude combat se livrait visiblement dans l'esprit de Caderousse: il était évident que ce petit étui de chagrin qu'il tournait et retournait dans sa main ne lui paraissait pas correspondre comme valeur à la somme énorme qui fascinait ses yeux. Il se retourna vers sa femme. 

«—Qu'en dis-tu? lui demanda-t-il tout bas. 

«—Donne, donne, dit-elle; s'il retourne à Beaucaire sans le diamant, il nous dénoncera! et, comme il le dit, qui sait si nous pourrons jamais remettre la main sur l'abbé Busoni. 

«—Eh bien, soit, dit Caderousse, prenez donc le diamant pour quarante-cinq mille francs; mais ma femme veut une chaîne d'or, et moi une paire de boucles d'argent. 

«Le bijoutier tira de sa poche une boîte longue et plate qui contenait plusieurs échantillons des objets demandés. 

«—Tenez, dit-il, je suis rond en affaires; choisissez. 

«La femme choisit une chaîne d'or qui pouvait valoir cinq louis, et le mari une paire de boucles qui pouvait valoir quinze francs. 

«—J'espère que vous ne vous plaindrez pas, dit le bijoutier. 

«—L'abbé avait dit qu'il valait cinquante mille francs, murmura Caderousse.  

«—Allons, allons, donnez donc! Quel homme terrible! reprit le bijoutier en lui tirant des mains le diamant, je lui compte quarante-cinq mille francs, deux mille cinq cents livres de rente, c'est-à-dire une fortune comme je voudrais bien en avoir une, moi, et il n'est pas encore content. 

«—Et les quarante-cinq mille francs, demanda Caderousse d'une voix rauque; voyons, où sont-ils? 

«—Les voilà, dit le bijoutier. 

«Et il compta sur la table quinze mille francs en or et trente mille francs en billets de banque.  

«—Attendez que j'allume la lampe, dit la Carconte, il n'y fait plus clair, et on pourrait se tromper. 

«En effet, la nuit était venue pendant cette discussion, et, avec la nuit, l'orage qui menaçait depuis une demi-heure. On entendait gronder sourdement le tonnerre dans le lointain; mais ni le bijoutier, ni Caderousse, ni la Carconte, ne paraissaient s'en occuper, possédés qu'ils étaient tous les trois du démon du gain. Moi-même, j'éprouvais une étrange fascination à la vue de tout cet or et de tous ces billets. Il me semblait que je faisais un rêve, et, comme il arrive dans un rêve, je me sentais enchaîné à ma place. 

«Caderousse compta et recompta l'or et les billets, puis il les passa à sa femme, qui les compta et recompta à son tour. 

«Pendant ce temps, le bijoutier faisait miroiter le diamant sous les rayons de la lampe, et le diamant jetait des éclairs qui lui faisaient oublier ceux qui, précurseurs de l'orage, commençaient à enflammer les fenêtres. 

«—Eh bien, le compte y est-il? demanda le bijoutier. 

«—Oui, dit Caderousse; donne le portefeuille et cherche un sac, Carconte. 

«La Carconte alla à une armoire et revint apportant un vieux portefeuille de cuir, duquel on tira quelques lettres graisseuses à la place desquelles on remit les billets, et un sac dans lequel étaient enfermés deux ou trois écus de six livres, qui composaient probablement toute la fortune du misérable ménage. 

«—Là, dit Caderousse, quoique vous nous ayez soulevé une dizaine de mille francs peut-être, voulez-vous souper avec nous? c'est de bon cœur. 

«—Merci, dit le bijoutier, il doit se faire tard, et il faut que je retourne à Beaucaire; ma femme serait inquiète»; il tira sa montre. «Morbleu! s'écria-t-il, neuf heures bientôt, je ne serai pas à Beaucaire avant minuit. Adieu, mes petits enfants; s'il vous revient par hasard des abbés Busoni, pensez à moi. 

«—Dans huit jours, vous ne serez plus à Beaucaire, dit Caderousse, puisque la foire finit la semaine prochaine. 

«—Non, mais cela ne fait rien; écrivez-moi à Paris, à M. Joannès, au Palais-Royal, galerie de Pierre, n° 45, je ferai le voyage exprès si cela en vaut la peine. 

«Un coup de tonnerre retentit, accompagné d'un éclair si violent qu'il effaça presque la clarté de la lampe. 

«—Oh! oh! dit Caderousse, vous allez partir par ce temps-là? 

«—Oh! je n'ai pas peur du tonnerre, dit le bijoutier. 

«—Et des voleurs? demanda la Carconte. La route n'est jamais bien sûre pendant la foire. 

«—Oh! quant aux voleurs, dit Joannès, voilà pour eux. 

«Et il tira de sa poche une paire de petits pistolets chargés jusqu'à la gueule. 

«—Voilà, dit-il, des chiens qui aboient et mordent en même temps: c'est pour les deux premiers qui auraient envie de votre diamant, père Caderousse. 

«Caderousse et sa femme échangèrent un regard sombre. Il paraît qu'ils avaient en même temps quelque terrible pensée. 

«—Alors, bon voyage! dit Caderousse.  

«—Merci!» dit le bijoutier. 

«Il prit sa canne qu'il avait posée contre un vieux bahut, et sortit. Au moment où il ouvrit la porte, une telle bouffée de vent entra qu'elle faillit éteindre la lampe. 

«—Oh! dit-il, il va faire un joli temps, et deux lieues de pays à faire avec ce temps-là! 

«—Restez, dit Caderousse, vous coucherez ici. 

«—Oui, restez, dit la Carconte d'une voix tremblante, nous aurons bien soin de vous.  

«—Non pas, il faut que j'aille coucher à Beaucaire. Adieu.» 

«Caderousse alla lentement jusqu'au seuil. 

«—Il ne fait ni ciel ni terre, dit le bijoutier déjà hors de la maison. Faut-il prendre à droite ou à gauche? 

«—À droite, dit Caderousse; il n'y a pas à s'y tromper, la route est bordée d'arbres de chaque côté. 

«—Bon, j'y suis, dit la voix presque perdue dans le lointain. 

«—Ferme donc la porte, dit la Carconte, je n'aime pas les portes ouvertes quand il tonne.  

«—Et quand il y a de l'argent dans la maison, n'est-ce pas?» dit Caderousse en donnant un double tour à la serrure. 

«Il rentra, alla à l'armoire, retira le sac et le portefeuille, et tous deux se mirent à recompter pour la troisième fois leur or et leurs billets. Je n'ai jamais vu expression pareille à ces deux visages dont cette maigre lampe éclairait la cupidité. La femme surtout était hideuse; le tremblement fiévreux qui l'animait habituellement avait redoublé. Son visage de pâle était devenu livide; ses yeux caves flamboyaient. 

«—Pourquoi donc, demanda-t-elle d'une voix sourde, lui avais-tu offert de coucher ici? 

«—Mais, répondit Caderousse en tressaillant, pour\dots pour qu'il n'eût pas la peine de retourner à Beaucaire. 

«—Ah! dit la femme avec une expression impossible à rendre, je croyais que c'était pour autre chose, moi. 

«—Femme! femme! s'écria Caderousse, pourquoi as-tu de pareilles idées, et pourquoi les ayant ne les gardes-tu pas pour toi? 

«—C'est égal, dit la Carconte après un instant de silence, tu n'es pas un homme. 

«—Comment cela? fit Caderousse. 

«—Si tu avais été un homme, il ne serait pas sorti. 

«—Femme! 

«—Ou bien il n'arriverait pas à Beaucaire. 

«—Femme! 

«—La route fait un coude et il est obligé de suivre la route, tandis qu'il y a le long du canal un chemin qui raccourcit. 

«—Femme, tu offenses le Bon Dieu. Tiens, écoute\dots. 

«En effet, on entendit un effroyable coup de tonnerre en même temps qu'un éclair bleuâtre enflammait toute la salle, et la foudre, décroissant lentement, sembla s'éloigner comme à regret de la maison maudite. 

«—Jésus! dit la Carconte en se signant. 

«Au même instant, et au milieu de ce silence de terreur qui suit ordinairement les coups de tonnerre, on entendit frapper à la porte. 

«Caderousse et sa femme tressaillirent et se regardèrent épouvantés. 

«—Qui va là? s'écria Caderousse en se levant et en réunissant en un seul tas l'or et les billets épars sur la table et qu'il couvrit de ses deux mains. 

«—Moi! dit une voix. 

«—Qui, vous? 

«—Et pardieu! Joannès le bijoutier. 

«—Eh bien, que disais-tu donc, reprit la Carconte avec un effroyable sourire, que j'offensais le Bon Dieu!\dots Voilà le Bon Dieu qui nous le renvoie. 

«Caderousse retomba pâle et haletant sur sa chaise. La Carconte, au contraire, se leva, et alla d'un pas ferme à la porte qu'elle ouvrit.  

«—Entrez donc, cher monsieur Joannès, dit-elle. 

«—Ma foi, dit le bijoutier ruisselant de pluie, il paraît que le diable ne veut pas que je retourne à Beaucaire ce soir. Les plus courtes folies sont les meilleures, mon cher monsieur Caderousse; vous m'avez offert l'hospitalité, je l'accepte et je reviens coucher chez vous.» 

Caderousse balbutia quelques mots en essuyant la sueur qui coulait sur son front. La Carconte referma la porte à double tour derrière le bijoutier. 