\chapter{Le major Cavalcanti}

\lettrine{N}{i} le comte ni Baptistin n'avaient menti en annonçant à Morcerf cette visite du major Lucquois, qui servait à Monte-Cristo de prétexte pour refuser le dîner qui lui était offert. 

\zz
Sept heures venaient de sonner, et M. Bertuccio, selon l'ordre qu'il en avait reçu, était parti depuis deux heures pour Auteuil, lorsqu'un fiacre s'arrêta à la porte de l'hôtel, et sembla s'enfuir tout honteux aussitôt qu'il eut déposé près de la grille un homme de cinquante-deux ans environ, vêtu d'une de ces redingotes vertes à brandebourgs noirs dont l'espèce est impérissable, à ce qu'il paraît, en Europe. Un large pantalon de drap bleu, une botte encore assez propre, quoique d'un vernis incertain et un peu trop épaisse de semelle, des gants de daim, un chapeau se rapprochant pour la forme d'un chapeau de gendarme, un col noir, brodé d'un liséré blanc, qui, si son propriétaire ne l'eût porté de sa pleine et entière volonté, eût pu passer pour un carcan: tel était le costume pittoresque sous lequel se présenta le personnage qui sonna à la grille en demandant si ce n'était point au n° 30 de l'avenue des Champs-Élysées que demeurait M. le comte de Monte-Cristo, et qui, sur la réponse affirmative du concierge, entra, ferma la porte derrière lui et se dirigea vers le perron. 

La tête petite et anguleuse de cet homme, ses cheveux blanchissants, sa moustache épaisse et grise le firent reconnaître par Baptistin, qui avait l'exact signalement du visiteur et qui l'attendait au bas du vestibule. Aussi, à peine eut-il prononcé son nom devant le serviteur intelligent, que Monte-Cristo était prévenu de son arrivée. 

On introduisit l'étranger dans le salon le plus simple. Le comte l'y attendait et alla au-devant de lui d'un air riant. 

«Ah! cher monsieur, dit-il, soyez le bienvenu. Je vous attendais. 

—Vraiment, dit le Lucquois, Votre Excellence m'attendait. 

—Oui, j'avais été prévenu de votre arrivée pour aujourd'hui à sept heures. 

—De mon arrivée? Ainsi vous étiez prévenu? 

—Parfaitement. 

—Ah! tant mieux! Je craignais, je l'avoue, que l'on n'eût oublié cette petite précaution. 

—Laquelle? 

—De vous prévenir. 

—Oh! non pas! 

—Mais vous êtes sûr de ne pas vous tromper? 

—J'en suis sûr. 

—C'est bien moi que Votre Excellence attendait aujourd'hui à sept heures? 

—C'est bien vous. D'ailleurs, vérifions. 

—Oh! si vous m'attendiez, dit le Lucquois, ce n'est pas la peine. 

—Si fait! si fait!» dit Monte-Cristo. 

Le Lucquois parut légèrement inquiet. 

«Voyons, dit Monte-Cristo, n'êtes-vous pas monsieur le marquis Bartolomeo Cavalcanti? 

—Bartolomeo Cavalcanti, répéta le Lucquois joyeux, c'est bien cela. 

—Ex-major au service d'Autriche? 

—Était-ce major que j'étais? demanda timidement le vieux militaire. 

—Oui, dit Monte-Cristo, c'était major. C'est le nom que l'on donne en France au grade que vous occupiez en Italie. 

—Bon, dit le Lucquois, je ne demande pas mieux, moi, vous comprenez\dots. 

—D'ailleurs, vous ne venez pas ici de votre propre mouvement, reprit Monte-Cristo. 

—Oh! bien certainement. 

—Vous m'êtes adressé par quelqu'un. 

—Oui. 

—Par cet excellent abbé Busoni? 

—C'est cela! s'écria le major joyeux. 

—Et vous avez une lettre? 

—La voilà. 

—Eh pardieu! vous voyez bien. Donnez donc.» 

Et Monte-Cristo prit la lettre qu'il ouvrit et qu'il lut. 

Le major regardait le comte avec de gros yeux étonnés qui se portaient curieusement sur chaque partie de l'appartement, mais qui revenaient invariablement à son propriétaire. 

«C'est bien cela\dots ce cher abbé, «le major Cavalcanti, un digne praticien de Lucques, descendant des Cavalcanti de Florence, continua Monte-Cristo tout en lisant, jouissant d'une fortune d'un demi-million de revenu.»  

Monte-Cristo leva les yeux de dessus le papier et salua. 

«D'un demi-million, dit-il; peste! mon cher monsieur Cavalcanti. 

—Y a-t-il un demi-million? demanda le Lucquois. 

—En toutes lettres; et cela doit être, l'abbé Busoni est l'homme qui connaît le mieux toutes les grandes fortunes de l'Europe. 

—Va pour un demi-million, dit le Lucquois; mais, ma parole d'honneur, je ne croyais pas que cela montât si haut. 

—Parce que vous avez un intendant qui vous vole; que voulez-vous, cher monsieur Cavalcanti, il faut bien passer par là! 

—Vous venez de m'éclairer, dit gravement le Lucquois, je mettrai le drôle à la porte.» 

Monte-Cristo continua: 

—«Et auquel il ne manquerait qu'une chose pour être heureux». 

—Oh! mon Dieu, oui! une seule, dit le Lucquois avec un soupir. 

—«De retrouver un fils adoré.» 

—Un fils adoré! 

—«Enlevé dans sa jeunesse, soit par un ennemi de sa noble famille, soit par des Bohémiens.» 

—À l'âge de cinq ans, monsieur, dit le Lucquois avec un profond soupir et en levant les yeux au ciel. 

—Pauvre père!» dit Monte-Cristo. 

Le comte continua: 

—«Je lui rends l'espoir, je lui rends la vie, monsieur le comte, en lui annonçant que ce fils, que depuis quinze ans il cherche vainement, vous pouvez le lui faire retrouver.» 

Le Lucquois regarda Monte-Cristo avec une indéfinissable expression d'inquiétude. 

«Je le puis», répondit Monte-Cristo. 

Le major se redressa. 

«Ah! ah! dit-il, la lettre était donc vraie jusqu'au bout?  

—En aviez-vous douté, cher monsieur Bartolomeo? 

—Non pas, jamais! Comment donc! un homme grave, un homme revêtu d'un caractère religieux comme l'abbé Busoni, ne se serait pas permis une plaisanterie pareille; mais vous n'avez pas tout lu, Excellence. 

—Ah! c'est vrai, dit Monte-Cristo, il y a un \textit{post-scriptum}. 

—Oui, répéta le Lucquois\dots il\dots y\dots a\dots un\dots \textit{post-scriptum}. 

—«Pour ne point causer au major Cavalcanti l'embarras de déplacer des fonds chez son banquier, je lui envoie une traite de deux mille francs pour ses frais de voyage, et le crédit sur vous de la somme de quarante-huit mille francs que vous restez me redevoir.» 

Le major suivit des yeux ce \textit{post-scriptum} avec une visible anxiété. 

«Bon! se contenta de dire le comte. 

—Il a dit bon, murmura le Lucquois. Ainsi\dots monsieur\dots reprit-il. 

—Ainsi?\dots demanda Monte-Cristo. 

—Ainsi, le \textit{post-scriptum}\dots 

—Eh bien, le \textit{post-scriptum}?\dots 

—Est accueilli par vous aussi favorablement que le reste de la lettre? 

—Certainement. Nous sommes en compte, l'abbé Busoni et moi; je ne sais pas si c'est quarante-huit mille livres précisément que je reste lui redevoir, nous n'en sommes pas entre nous à quelques billets de banque. Ah çà! vous attachiez donc une si grande importance à ce post-scriptum, cher monsieur Cavalcanti? 

—Je vous avouerai, répondit le Lucquois, que plein de confiance dans la signature de l'abbé Busoni, je ne m'étais pas muni d'autres fonds; de sorte que si cette ressource m'eût manqué, je me serais trouvé fort embarrassé à Paris. 

—Est-ce qu'un homme comme vous est embarrassé quelque part? dit Monte-Cristo; allons donc! 

—Dame! ne connaissant personne, fit le Lucquois. 

—Mais on vous connaît, vous. 

—Oui, l'on me connaît, de sorte que\dots. 

—Achevez, cher monsieur Cavalcanti! 

—De sorte que vous me remettrez ces quarante-huit mille livres? 

—À votre première réquisition.» 

Le major roulait de gros yeux ébahis. 

«Mais asseyez-vous donc, dit Monte-Cristo: en vérité, je ne sais ce que je fais\dots je vous tiens debout depuis un quart d'heure. 

—Ne faites pas attention.» 

Le major tira un fauteuil et s'assit. 

«Maintenant, dit le comte, voulez-vous prendre quelque chose; un verre de xérès, de porto, d'alicante? 

—D'alicante, puisque vous le voulez bien, c'est mon vin de prédilection. 

—J'en ai d'excellent. Avec un biscuit, n'est-ce pas? 

—Avec un biscuit, puisque vous m'y forcez.» 

Monte-Cristo sonna; Baptistin parut. 

Le comte s'avança vers lui. 

«Eh bien?\dots demanda-t-il tout bas. 

—Le jeune homme est là, répondit le valet de chambre sur le même ton. 

—Bien; où l'avez-vous fait entrer? 

—Dans le salon bleu, comme l'avait ordonné Son Excellence. 

—À merveille. Apportez du vin d'Alicante et des biscuits.» 

Baptistin sortit. 

«En vérité, dit le Lucquois, je vous donne une peine qui me remplit de confusion.  

—Allons donc!» dit Monte-Cristo. 

Baptistin rentra avec les verres, le vin et les biscuits. 

Le comte emplit un verre et versa dans le second quelques gouttes seulement du rubis liquide que contenait la bouteille, toute couverte de toiles d'araignée et de tous les autres signes qui indiquent la vieillesse du vin bien plus sûrement que ne le font les rides pour l'homme. 

Le major ne se trompa point au partage, il prit le verre plein et un biscuit. Le comte ordonna à Baptistin de poser le plateau à la portée de la main de son hôte, qui commença par goûter l'alicante du bout de ses lèvres, fit une grimace de satisfaction, et introduisit délicatement le biscuit dans le verre. 

«Ainsi, monsieur, dit Monte-Cristo, vous habitiez Lucques, vous étiez riche, vous êtes noble, vous jouissiez de la considération générale, vous aviez tout ce qui peut rendre un homme heureux. 

—Tout, Excellence, dit le major en engloutissant son biscuit, tout absolument. 

—Et il ne manquait qu'une chose à votre bonheur? 

—Qu'une seule, dit le Lucquois. 

—C'était de retrouver votre enfant? 

—Ah! fit le major en prenant un second biscuit; mais aussi cela me manquait bien.» 

Le digne Lucquois leva les yeux et tenta un effort pour soupirer. 

«Maintenant, voyons, cher monsieur Cavalcanti, dit Monte-Cristo, qu'était-ce que ce fils tant regretté? car on m'avait dit, à moi, que vous étiez resté célibataire. 

—On le croyait, monsieur, dit le major, et moi-même\dots. 

—Oui, reprit Monte-Cristo, et vous-même aviez accrédité ce bruit. Un péché de jeunesse que vous vouliez cacher à tous les yeux.» 

Le Lucquois se redressa, prit son air le plus calme et le plus digne, en même temps qu'il baissait modestement les yeux, soit pour assurer sa contenance, soit pour aider à son imagination, tout en regardant en dessous le comte, dont le sourire stéréotypé sur les lèvres annonçait toujours la même bienveillante curiosité. 

«Oui, monsieur, dit-il, je voulais cacher cette faute à tous les yeux.» 

—Pas pour vous, dit Monte-Cristo, car un homme est au-dessus de ces choses-là. 

—Oh! non, pas pour moi certainement, dit le major avec un sourire et en hochant la tête. 

—Mais pour sa mère, dit le comte. 

—Pour sa mère! s'écria le Lucquois en prenant un troisième biscuit, pour sa pauvre mère! 

—Buvez donc, cher monsieur Cavalcanti, dit Monte-Cristo en versant au Lucquois un second verre d'alicante; l'émotion vous étouffe. 

—Pour sa pauvre mère! murmura le Lucquois en essayant si la puissance de la volonté ne pourrait pas en agissant sur la glande lacrymale, mouiller le coin de son œil d'une fausse larme. 

—Qui appartenait à l'une des premières familles d'Italie, je crois? 

—Patricienne de Fiesole, monsieur le comte, patricienne de Fiesole! 

—Et se nommant? 

—Vous désirez savoir son nom? 

—Oh! mon Dieu! dit Monte-Cristo, c'est inutile que vous me le disiez, je le connais. 

—Monsieur le comte sait tout, dit le Lucquois en s'inclinant. 

—Olivia Corsinari, n'est-ce pas? 

—Olivia Corsinari. 

—Marquise? 

—Marquise. 

—Et vous avez fini par l'épouser cependant, malgré les oppositions de la famille? 

—Mon Dieu! oui, j'ai fini par là. 

—Et, reprit Monte-Cristo, vous apportez vos papiers bien en règle? 

—Quels papiers? demanda le Lucquois. 

—Mais votre acte de mariage avec Olivia Corsinari, et l'acte de naissance de l'enfant. 

—L'acte de naissance de l'enfant? 

—L'acte de naissance d'Andrea Cavalcanti, de votre fils; ne s'appelle-t-il pas Andrea? 

—Je crois que oui, dit le Lucquois. 

—Comment! vous le croyez? 

—Dame! je n'ose pas affirmer, il y a si longtemps qu'il est perdu. 

—C'est juste, dit Monte-Cristo. Enfin vous avez tous ces papiers? 

—Monsieur le comte, c'est avec regret que je vous annonce que, n'étant pas prévenu de me munir de ces pièces, j'ai négligé de les prendre avec moi. 

—Ah! diable, fit Monte-Cristo. 

—Étaient-elles donc tout à fait nécessaires? 

—Indispensables!» 

Lucquois se gratta le front. 

«Ah! \textit{per Bacco}! dit-il, indispensables! 

—Sans doute; si l'on allait élever ici quelque doute sur la validité de votre mariage, sur la légitimité de votre enfant! 

—C'est juste, dit le Lucquois, on pourrait élever des doutes. 

—Ce serait fâcheux pour ce jeune homme. 

—Ce serait fatal. 

—Cela pourrait lui faire manquer quelque magnifique mariage. 

—\textit{O peccato}! 

—En France, vous comprenez, on est sévère; il ne suffit pas, comme en Italie, d'aller trouver un prêtre et de lui dire: «Nous nous aimons, unissez-nous.» Il y a mariage civil en France, et, pour se marier civilement, il faut des pièces qui constatent l'identité. 

—Voilà le malheur: ces papiers, je ne les ai pas. 

—Heureusement que je les ai, moi, dit Monte-Cristo. 

—Vous? 

—Oui? 

—Vous les avez? 

—Je les ai. 

—Ah! par exemple, dit le Lucquois, qui, voyant le but de son voyage manqué par l'absence de ses papiers, craignait que cet oubli n'amenât quelque difficulté au sujet des quarante-huit mille livres; ah! par exemple, voilà un bonheur! Oui, reprit-il, voilà un bonheur, car je n'y eusse pas songé, moi. 

—Pardieu! je crois bien, on ne songe pas à tout. Mais heureusement l'abbé Busoni y a songé pour vous. 

—Voyez-vous, ce cher abbé! 

—C'est un homme de précaution. 

—C'est un homme admirable, dit le Lucquois; et il vous les a envoyés? 

—Les voici.» 

Le Lucquois joignit les mains en signe d'admiration. 

«Vous avez épousé Olivia Corsinari dans l'église de Sainte-Paule de Monte-Catini; voici le certificat du prêtre. 

—Oui, ma foi! le voilà, dit le major en le regardant avec étonnement. 

—Et voici l'acte de baptême d'Andrea Cavalcanti, délivré par le curé de Saravezza. 

—Tout est en règle, dit le major. 

—Alors prenez ces papiers, dont je n'ai que faire, vous les donnerez à votre fils qui les gardera soigneusement. 

—Je le crois bien!\dots S'il les perdait\dots. 

—Eh bien, s'il les perdait? demanda Monte-Cristo. 

—Eh bien, reprit le Lucquois, on serait obligé d'écrire là-bas, et ce serait fort long de s'en procurer d'autres. 

—En effet, ce serait difficile, dit Monte-Cristo. 

—Presque impossible, répondit le Lucquois. 

—Je suis bien aise que vous compreniez la valeur de ces papiers. 

—C'est-à-dire que je les regarde comme impayables. 

—Maintenant, dit Monte-Cristo, quant à la mère du jeune homme?\dots 

—Quant à la mère du jeune homme\dots répéta le major avec inquiétude. 

—Quant à la marquise Corsinari? 

—Mon Dieu! dit le Lucquois, sous les pas duquel les difficultés semblaient naître, est-ce qu'on aurait besoin d'elle? 

—Non, monsieur, reprit Monte-Cristo; d'ailleurs, n'a-t-elle point?\dots 

—Si fait, si fait, dit le major, elle a\dots. 

—Payé son tribut à la nature?\dots 

—Hélas! oui, dit vivement le Lucquois. 

—J'ai su cela, reprit Monte-Cristo; elle est morte il y a dix ans. 

—Et je pleure encore sa mort, monsieur, dit le major en tirant de sa poche un mouchoir à carreaux et en s'essuyant alternativement d'abord l'œil gauche et ensuite l'œil droit. 

—Que voulez-vous, dit Monte-Cristo, nous sommes tous mortels. Maintenant vous comprenez, cher monsieur Cavalcanti, vous comprenez qu'il est inutile qu'on sache en France que vous êtes séparé de votre fils depuis quinze ans. Toutes ces histoires de Bohémiens qui enlèvent les enfants n'ont pas de vogue chez nous. Vous l'avez envoyé faire son éducation dans un collège de province, et vous voulez qu'il achève cette éducation dans le monde parisien. Voilà pourquoi vous avez quitté Via-Reggio, que vous habitiez depuis la mort de votre femme. Cela suffira. 

—Vous croyez? 

—Certainement. 

—Très bien, alors. 

—Si l'on apprenait quelque chose de cette séparation\dots. 

—Ah! oui. Que dirais-je? 

—Qu'un précepteur infidèle, vendu aux ennemis de votre famille\dots. 

—Aux Corsinari? 

—Certainement\dots avait enlevé cet enfant pour que votre nom s'éteignît. 

—C'est juste, puisqu'il est fils unique. 

—Eh bien, maintenant que tout est arrêté, que vos souvenirs, remis à neuf, ne vous trahiront pas, vous avez deviné sans doute que je vous ai ménagé une surprise? 

—Agréable? demanda le Lucquois. 

—Ah! dit Monte-Cristo, je vois bien qu'on ne trompe pas plus l'œil que le cœur d'un père. 

—Hum! fit le major. 

—On vous a fait quelque révélation indiscrète, ou plutôt vous avez deviné qu'il était là. 

—Qui, là? 

—Votre enfant, votre fils, votre Andrea. 

—Je l'ai deviné, répondit le Lucquois avec le plus grand flegme du monde: ainsi il est ici? 

—Ici même, dit Monte-Cristo; en entrant tout à l'heure, le valet de chambre m'a prévenu de son arrivée. 

—Ah! fort bien! ah! fort bien! dit le major en resserrant à chaque exclamation les brandebourgs de sa polonaise. 

—Mon cher monsieur, dit Monte-Cristo, je comprends toute votre émotion, il faut vous donner le temps de vous remettre; je veux aussi préparer le jeune homme à cette entrevue tant désirée, car je présume qu'il n'est pas moins impatient que vous. 

—Je le crois, dit Cavalcanti. 

—Eh bien, dans un petit quart d'heure nous sommes à vous. 

—Vous me l'amenez donc? vous poussez donc la bonté jusqu'à me le présenter vous-même? 

—Non, je ne veux point me placer entre un père et son fils, vous serez seuls, monsieur le major; mais soyez tranquille, au cas même où la voix du sang resterait muette, il n'y aurait pas à vous tromper: il entrera par cette porte. C'est un beau jeune homme blond, un peu trop blond peut-être, de manières toutes prévenantes; vous verrez. 

—À propos, dit le major, vous savez que je n'ai emporté avec moi que les deux mille francs que ce bon abbé Busoni m'avait fait passer. Là-dessus j'ai fait le voyage, et\dots. 

—Et vous avez besoin d'argent\dots c'est trop juste, cher monsieur Cavalcanti. Tenez, voici pour faire un compte, huit billets de mille francs.» 

Les yeux du major brillèrent comme des escarboucles. 

«C'est quarante mille francs que je vous redois, dit Monte-Cristo. 

—Votre Excellence veut-elle un reçu? dit le major en glissant les billets dans la poche intérieure de sa polonaise. 

—À quoi bon? dit le comte. 

—Mais pour vous décharger vis-à-vis de l'abbé Busoni. 

—Eh bien, vous me donnerez un reçu général en touchant les quarante derniers mille francs. Entre honnêtes gens, de pareilles précautions sont inutiles. 

—Ah! oui, c'est vrai, dit le major, entre honnêtes gens. 

—Maintenant, un dernier mot, marquis. 

—Dites. 

—Vous permettez une petite recommandation, n'est-ce pas? 

—Comment donc! Je la demande. 

—Il n'y aurait pas de mal que vous quittassiez cette polonaise. 

—Vraiment! dit le major en regardant le vêtement avec une certaine complaisance. 

—Oui, cela se porte encore à Via-Reggio, mais à Paris il y a déjà longtemps que ce costume, quelque élégant qu'il soit, a passé de mode. 

—C'est fâcheux, dit le Lucquois. 

—Oh! si vous y tenez, vous le reprendrez en vous en allant. 

—Mais que mettrai-je? 

—Ce que vous trouverez dans vos malles. 

—Comment, dans mes malles! je n'ai qu'un portemanteau. 

—Avec vous sans doute. À quoi bon s'embarrasser? D'ailleurs, un vieux soldat aime à marcher en leste équipage. 

—Voilà justement pourquoi\dots. 

—Mais vous êtes homme de précaution, et vous avez envoyé vos malles en avant. Elles sont arrivées hier à l'hôtel des Princes, rue Richelieu. C'est là que vous avez retenu votre logement. 

—Alors dans ces malles? 

—Je présume que vous avez eu la précaution de faire enfermer par votre valet de chambre tout ce qu'il vous faut: habits de ville, habits d'uniforme. Dans les grandes circonstances, vous mettrez l'habit d'uniforme, cela fait bien. N'oubliez pas votre croix. On s'en moque encore en France, mais on en porte toujours. 

—Très bien, très bien, très bien! dit le major qui marchait d'éblouissements en éblouissements. 

—Et maintenant, dit Monte-Cristo, que votre cœur est affermi contre les émotions trop vives, préparez-vous, cher monsieur Cavalcanti, à revoir votre fils Andrea.» 

Et faisant un charmant salut au Lucquois, ravi, en extase, Monte-Cristo disparut derrière la tapisserie. 