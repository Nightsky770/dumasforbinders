\chapter{Locuste}

\lettrine{V}{alentine} resta seule; deux autres pendules, en retard sur celle de Saint-Philippe-du-Roule, sonnèrent encore minuit à des distances différentes. 

\zz
Puis, à part le bruissement de quelques voitures lointaines, tout retomba dans le silence. 

Alors toute l'attention de Valentine se concentra sur la pendule de sa chambre, dont le balancier marquait les secondes. 

Elle se mit à compter ces secondes et remarqua qu'elles étaient du double plus lentes que les battements de son cœur. Et cependant elle doutait encore; l'inoffensive Valentine ne pouvait se figurer que quelqu'un désirât sa mort; pourquoi? dans quel but? quel mal avait-elle fait qui pût lui susciter un ennemi? 

Il n'y avait pas de crainte qu'elle s'endormît. 

Une seule idée, une idée terrible tenait son esprit tendu: c'est qu'il existait une personne au monde qui avait tenté de l'assassiner et qui allait le tenter encore. 

Si cette fois cette personne, lassée de voir l'inefficacité du poison, allait, comme l'avait dit Monte-Cristo, avoir recours au fer! si le comte n'allait pas avoir le temps d'accourir! si elle touchait à son dernier moment! si elle ne devait plus revoir Morrel! 

À cette pensée qui la couvrait à la fois d'une pâleur livide et d'une sueur glacée, Valentine était prête à saisir le cordon de la sonnette et à appeler au secours. 

Mais il lui semblait, à travers la porte de la bibliothèque, voir étinceler l'œil du comte, cet œil qui pesait sur son souvenir, et qui, lorsqu'elle y songeait, l'écrasait d'une telle honte, qu'elle se demandait si jamais la reconnaissance parviendrait à effacer ce pénible effet de l'indiscrète amitié du comte. 

Vingt minutes, vingt éternités s'écoulèrent ainsi, puis dix autres minutes encore; enfin la pendule, criant une seconde à l'avance, finit par frapper un coup sur le timbre sonore. 

En ce moment même, un grattement imperceptible de l'ongle sur le bois de la bibliothèque apprit à Valentine que le comte veillait et lui recommandait de veiller. 

En effet, du côté opposé, c'est-à-dire vers la chambre d'Édouard, il sembla à Valentine qu'elle entendait crier le parquet; elle prêta l'oreille, retenant sa respiration presque étouffée; le bouton de la serrure grinça et la porte tourna sur ses gonds. 

Valentine s'était soulevée sur son coude, elle n'eut que le temps de se laisser retomber sur son lit et de cacher ses yeux sous son bras. 

Puis, tremblante, agitée, le cœur serré d'un indicible effroi, elle attendit. 

Quelqu'un s'approcha du lit et effleura les rideaux. 

Valentine rassembla toutes ses forces et laissa entendre ce murmure régulier de la respiration qui annonce un sommeil tranquille. 

«Valentine!» dit tout bas une voix. 

La jeune fille frissonna jusqu'au fond du cœur, mais ne répondit point. 

«Valentine!» répéta la même voix. 

Même silence: Valentine avait promis de ne point se réveiller. 

Puis tout demeura immobile. 

Seulement Valentine entendit le bruit presque insensible d'une liqueur tombant dans le verre qu'elle venait de vider. 

Alors elle osa, sous le rempart de son bras étendu, entrouvrir sa paupière. 

Elle vit alors une femme en peignoir blanc, qui vidait dans son verre une liqueur préparée d'avance dans une fiole. 

Pendant ce court instant, Valentine retint peut-être sa respiration ou fit sans doute quelque mouvement, car la femme, inquiète, s'arrêta et se pencha sur son lit pour mieux voir si elle dormait réellement: c'était Mme de Villefort. 

Valentine, en reconnaissant sa belle-mère, fut saisie d'un frisson aigu qui imprima un mouvement à son lit. 

Madame de Villefort s'effaça aussitôt le long du mur, et là, abritée derrière le rideau du lit, muette, attentive, elle épia jusqu'au moindre mouvement de Valentine. 

Celle-ci se rappela les terribles paroles de Monte-Cristo; il lui avait semblé, dans la main qui ne tenait pas la fiole, voir briller une espèce de couteau long et affilé. Alors Valentine, appelant toute la puissance de sa volonté à son secours, s'efforça de fermer les yeux; mais, cette fonction du plus craintif de nos sens, cette fonction, si simple d'ordinaire, devenait en ce moment presque impossible à accomplir, tant l'avide curiosité faisait d'efforts pour repousser cette paupière et attirer la vérité. 

Cependant, assurée, par le silence dans lequel avait recommencé à se faire entendre le bruit égal de la respiration de Valentine, que celle-ci dormait, Mme de Villefort étendit de nouveau le bras, et en demeurant à demi dissimulée par les rideaux rassemblés au chevet du lit, elle acheva de vider dans le verre de Valentine le contenu de sa fiole. 

Puis elle se retira, sans que le moindre bruit avertît Valentine qu'elle était partie. 

Elle avait vu disparaître le bras, voilà tout; ce bras frais et arrondi d'une femme de vingt-cinq ans, jeune et belle, et qui versait la mort. 

Il est impossible d'exprimer ce que Valentine avait éprouvé pendant cette minute et demie que Mme de Villefort était restée dans sa chambre. 

Le grattement de l'ongle sur la bibliothèque tira la jeune fille de cet état de torpeur dans lequel elle était ensevelie, et qui ressemblait à de l'engourdissement. 

Elle souleva la tête avec effort. 

La porte, toujours silencieuse, roula une seconde fois sur ses gonds, et le comte de Monte-Cristo reparut. 

«Eh bien, demanda le comte, doutez-vous encore? 

—Ô mon Dieu! murmura la jeune fille. 

—Vous avez vu? 

—Hélas! 

—Vous avez reconnu?» 

Valentine poussa un gémissement. 

«Oui, dit-elle, mais je n'y puis croire. 

—Vous aimez mieux mourir alors, et faire mourir Maximilien!\dots 

—Mon Dieu, mon Dieu! répéta la jeune fille presque égarée; mais ne puis-je donc pas quitter la maison, me sauver?\dots 

—Valentine, la main qui vous poursuit vous atteindra partout: à force d'or, on séduira vos domestiques, et la mort s'offrira à vous, déguisée sous tous les aspects, dans l'eau que vous boirez à la source, dans le fruit que vous cueillerez à l'arbre. 

—Mais n'avez-vous donc pas dit que la précaution de bon papa m'avait prémunie contre le poison? 

—Contre un poison, et encore non pas employé à forte dose; on changera de poison ou l'on augmentera la dose.» 

Il prit le verre et y trempa ses lèvres. 

«Et tenez, dit-il, c'est déjà fait. Ce n'est plus avec de la brucine qu'on vous empoisonne, c'est avec un simple narcotique. Je reconnais le goût de l'alcool dans lequel on l'a fait dissoudre. Si vous aviez bu ce que Mme de Villefort vient de verser dans ce verre, Valentine, vous étiez perdue. 

—Mais, mon Dieu! s'écria la jeune fille, pourquoi donc me poursuit-elle ainsi? 

—Comment! vous êtes si douce, si bonne, si peu croyante au mal que vous n'avez pas compris, Valentine? 

—Non, dit la jeune fille; je ne lui ai jamais fait de mal. 

—Mais vous êtes riche, Valentine; mais vous avez deux cent mille livres de rente, et ces deux cent mille francs de rente, vous les enlevez à son fils. 

—Comment cela? Ma fortune n'est point la sienne et me vient de mes parents. 

—Sans doute, et voilà pourquoi M. et Mme de Saint-Méran sont morts: c'était pour que vous héritassiez de vos parents; voilà pourquoi du jour où il vous a fait son héritière, M. Noirtier avait été condamné; voilà pourquoi, à votre tour, vous devez mourir, Valentine, c'est afin que votre père hérite de vous, et que votre frère, devenu fils unique, hérite de votre père. 

—Édouard! pauvre enfant, et c'est pour lui qu'on commet tous ces crimes? 

—Ah! vous comprenez, enfin. 

—Ah! mon Dieu! pourvu que tout cela ne retombe pas sur lui! 

—Vous êtes un ange, Valentine. 

—Mais mon grand-père, on a donc renoncé à le tuer, lui? 

—On a réfléchi que vous morte, à moins d'exhérédation, la fortune revenait naturellement à votre frère, et l'on a pensé que le crime, au bout du compte, étant inutile, il était doublement dangereux de le commettre. 

—Et c'est dans l'esprit d'une femme qu'une pareille combinaison a pris naissance! Ô mon Dieu! mon Dieu! 

—Rappelez-vous Pérouse, la treille de l'auberge de la Poste, l'homme au manteau brun, que votre belle-mère interrogeait sur l'aqua-tofana; eh bien, dès cette époque, tout cet infernal projet mûrissait dans son cerveau. 

—Oh! monsieur, s'écria la douce jeune fille en fondant en larmes, je vois bien, s'il en est ainsi, que je suis condamnée à mourir. 

—Non, Valentine, non, car j'ai prévu tous les complots; non, car notre ennemie est vaincue, puisqu'elle est devinée; non, vous vivrez, Valentine, vous vivrez pour aimer et être aimée, vous vivrez pour être heureuse et rendre un noble cœur heureux; mais pour vivre, Valentine, il faut avoir bien confiance en moi. 

—Ordonnez, monsieur, que faut-il faire? 

—Il faut prendre aveuglément ce que je vous donnerai. 

—Oh! Dieu m'est témoin, s'écria Valentine, que si j'étais seule, j'aimerais mieux me laisser mourir! 

—Vous ne vous confierez à personne, pas même à votre père. 

—Mon père n'est pas de cet affreux complot, n'est-ce pas, monsieur? dit Valentine en joignant les mains. 

—Non, et cependant votre père, l'homme habitué aux accusations juridiques, votre père doit se douter que toutes ces morts qui s'abattent sur sa maison ne sont point naturelles. Votre père, c'est lui qui aurait dû veiller sur vous, c'est lui qui devrait être à cette heure à la place que j'occupe; c'est lui qui devrait déjà avoir vidé ce verre; c'est lui qui devrait déjà s'être dressé contre l'assassin. Spectre contre spectre, murmura-t-il, en achevant tout haut sa phrase. 

—Monsieur, dit Valentine, je ferai tout pour vivre, car il existe deux êtres au monde qui m'aiment à en mourir si je mourais: mon grand-père et Maximilien. 

—Je veillerai sur eux comme j'ai veillé sur vous. 

—Eh bien, monsieur, disposez de moi, dit Valentine. Puis à voix basse: mon Dieu! mon Dieu! dit-elle, que va-t-il m'arriver? 

—Quelque chose qui vous arrive, Valentine, ne vous épouvantez point; si vous souffrez, si vous perdez la vue, l'ouïe, le tact, ne craignez rien; si vous vous réveillez sans savoir où vous êtes, n'ayez pas peur, dussiez-vous, en vous éveillant, vous trouver dans quelque caveau sépulcral ou clouée dans quelque bière; rappelez soudain votre esprit, et dites-vous: En ce moment, un ami, un père, un homme qui veut mon bonheur et celui de Maximilien, cet homme veille sur moi. 

—Hélas! hélas! quelle terrible extrémité! 

—Valentine, aimez-vous mieux dénoncer votre belle-mère? 

—J'aimerais mieux mourir cent fois! oh! oui, mourir! 

—Non, vous ne mourrez pas, et quelque chose qui vous arrive, vous me le promettez, vous ne vous plaindrez pas, vous espérerez? 

—Je penserai à Maximilien. 

—Vous êtes ma fille bien-aimée, Valentine; seul, je puis vous sauver, et je vous sauverai.» 

Valentine, au comble de la terreur, joignit les mains (car elle sentait que le moment était venu de demander à Dieu du courage) et se dressa pour prier, murmurant des mots sans suite, et oubliant que ses blanches épaules n'avaient d'autre voile que sa longue chevelure et que l'on voyait battre son cœur sous la fine dentelle de peignoir de nuit. 

Le comte appuya doucement la main sur le bras de la jeune fille, ramena jusque sur son cou la courtepointe de velours, et, avec un sourire paternel: 

«Ma fille, dit-il, croyez en mon dévouement, comme vous croyez en la bonté de Dieu et dans l'amour de Maximilien.» 

Valentine attacha sur lui un regard plein de reconnaissance, et demeura docile comme un enfant sous ses voiles. 

Alors le comte tira de la poche de son gilet le drageoir en émeraude, souleva son couvercle d'or, et versa dans la main droite de Valentine une petite pastille ronde de la grosseur d'un pois. 

Valentine la prit avec l'autre main, et regarda le comte attentivement: il y avait sur les traits de cet intrépide protecteur un reflet de la majesté et de la puissance divines. Il était évident que Valentine l'interrogeait du regard. 

«Oui», répondit celui-ci. 

Valentine porta la pastille à sa bouche et l'avala. 

«Et maintenant, au revoir, mon enfant, dit-il, je vais essayer de dormir car vous êtes sauvée. 

—Allez, dit Valentine, quelque chose qui m'arrive, je vous promets de n'avoir pas peur.» 

Monte-Cristo tint longtemps ses yeux fixés sur la jeune fille, qui s'endormit peu à peu, vaincue par la puissance du narcotique que le comte venait de lui donner. 

Alors il prit le verre, le vida aux trois quarts dans la cheminée, pour que l'on pût croire que Valentine avait bu ce qu'il en manquait, le reposa sur la table de nuit; puis, regagnant la porte de la bibliothèque, il disparut après avoir jeté un dernier regard vers Valentine, qui s'endormait avec la confiance et la candeur d'un ange couché aux pieds du Seigneur. 