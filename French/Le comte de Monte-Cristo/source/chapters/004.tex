\chapter{Complot}

\lettrine{D}{anglars} suivit Edmond et Mercédès des yeux jusqu'à ce que les deux amants eussent disparu à l'un des angles du fort Saint-Nicolas; puis, se retournant alors, il aperçut Fernand, qui était retombé pâle et frémissant sur sa chaise, tandis que Caderousse balbutiait les paroles d'une chanson à boire.

«Ah çà! mon cher monsieur, dit Danglars à Fernand, voilà un mariage qui ne me paraît pas faire le bonheur de tout le monde!

—Il me désespère, dit Fernand.

—Vous aimiez donc Mercédès?

—Je l'adorais!

—Depuis longtemps?

—Depuis que nous nous connaissons, je l'ai toujours aimée.

—Et vous êtes là à vous arracher les cheveux, au lieu de chercher remède à la chose! Que diable! je ne croyais pas que ce fût ainsi qu'agissaient les gens de votre nation.

—Que voulez-vous que je fasse? demanda Fernand.

—Et que sais-je, moi? Est-ce que cela me regarde? Ce n'est pas moi, ce me semble, qui suis amoureux de Mlle Mercédès, mais vous. Cherchez, dit l'Évangile, et vous trouverez.

—J'avais trouvé déjà.

—Quoi?

—Je voulais poignarder \textit{l'homme}, mais la femme m'a dit que s'il arrivait malheur à son fiancé, elle se tuerait.

—Bah! on dit ces choses-là, mais on ne les fait point.

—Vous ne connaissez point Mercédès, monsieur: du moment où elle a menacé, elle exécuterait.

—Imbécile! murmura Danglars: qu'elle se tue ou non, que m'importe, pourvu que Dantès ne soit point capitaine.

—Et avant que Mercédès meure, reprit Fernand avec l'accent d'une immuable résolution, je mourrais moi-même.

—En voilà de l'amour! dit Caderousse d'une voix de plus en plus avinée; en voilà, ou je ne m'y connais plus!

—Voyons, dit Danglars, vous me paraissez un gentil garçon, et je voudrais, le diable m'emporte! vous tirer de peine; mais\dots.

—Oui, dit Caderousse, voyons.

—Mon cher, reprit Danglars, tu es aux trois quarts ivres: achève la bouteille, et tu le seras tout à fait. Bois, et ne te mêle pas de ce que nous faisons: pour ce que nous faisons il faut avoir toute sa tête.

—Moi ivre? dit Caderousse, allons donc! J'en boirais encore quatre, de tes bouteilles, qui ne sont pas plus grandes que des bouteilles d'eau de Cologne! Père Pamphile, du vin!»

Et pour joindre la preuve à la proposition, Caderousse frappa avec son verre sur la table.

«Vous disiez donc, monsieur? reprit Fernand, attendant avec avidité la suite de la phrase interrompue.

—Que disais-je? Je ne me le rappelle plus. Cet ivrogne de Caderousse m'a fait perdre le fil de mes pensées.

—Ivrogne tant que tu le voudras; tant pis pour ceux qui craignent le vin, c'est qu'ils ont quelque mauvaise pensée qu'ils craignent que le vin ne leur tire du cœur.»

Et Caderousse se mit à chanter les deux derniers vers d'une chanson fort en vogue à cette époque:

\textit{Tous les méchants sont buveurs d'eau. C'est bien prouvé par le déluge.}

«Vous disiez, monsieur, reprit Fernand, que vous voudriez me tirer de peine; mais, ajoutiez-vous\dots.

—Oui, mais, ajoutais-je\dots pour vous tirer de peine il suffit que Dantès n'épouse pas celle que vous aimez et le mariage peut très bien manquer, ce me semble, sans que Dantès meure.

—La mort seule les séparera, dit Fernand.

—Vous raisonnez comme un coquillage, mon ami, dit Caderousse, et voilà Danglars, qui est un finaud, un malin, un grec, qui va vous prouver que vous avez tort. Prouve, Danglars. J'ai répondu de toi. Dis-lui qu'il n'est pas besoin que Dantès meure; d'ailleurs ce serait fâcheux qu'il mourût, Dantès. C'est un bon garçon, je l'aime, moi, Dantès. À ta santé, Dantès.»

Fernand se leva avec impatience.

«Laissez-le dire, reprit Danglars en retenant le jeune homme, et d'ailleurs, tout ivre qu'il est, il ne fait point si grande erreur. L'absence disjoint tout aussi bien que la mort; et supposez qu'il y ait entre Edmond et Mercédès les murailles d'une prison, ils seront séparés ni plus ni moins que s'il y avait là la pierre d'une tombe.

—Oui, mais on sort de prison, dit Caderousse, qui avec les restes de son intelligence se cramponnait à la conversation, et quand on est sorti de prison et qu'on s'appelle Edmond Dantès, on se venge.

—Qu'importe! murmura Fernand.

—D'ailleurs, reprit Caderousse, pourquoi mettrait-on Dantès en prison? Il n'a ni volé, ni tué, ni assassiné.

—Tais-toi, dit Danglars.

—Je ne veux pas me taire, moi, dit Caderousse. Je veux qu'on me dise pourquoi on mettrait Dantès en prison. Moi, j'aime Dantès. À ta santé, Dantès!»

Et il avala un nouveau verre de vin. Danglars suivit dans les yeux atones du tailleur les progrès de l'ivresse, et se tournant vers Fernand:

«Eh bien, comprenez-vous, dit-il, qu'il n'y a pas besoin de le tuer?

—Non, certes, si, comme vous le disiez tout à l'heure, on avait le moyen de faire arrêter Dantès. Mais ce moyen, l'avez-vous?

—En cherchant bien, dit Danglars, on pourrait le trouver. Mais continua-t-il, de quoi diable! vais-je me mêler là; est-ce que cela me regarde?

—Je ne sais pas si cela vous regarde, dit Fernand en lui saisissant le bras; mais ce que je sais, c'est que vous avez quelque motif de haine particulière contre Dantès: celui qui hait lui-même ne se trompe pas aux sentiments des autres.

—Moi, des motifs de haine contre Dantès? Aucun, sur ma parole. Je vous ai vu malheureux et votre malheur m'a intéressé, voilà tout; mais du moment où vous croyez que j'agis pour mon propre compte, adieu, mon cher ami, tirez-vous d'affaire comme vous pourrez.»

Et Danglars fit semblant de se lever à son tour.

«Non pas, dit Fernand en le retenant, restez! Peu m'importe, au bout du compte, que vous en vouliez à Dantès, ou que vous ne lui en vouliez pas: je lui en veux, moi; je l'avoue hautement. Trouvez le moyen et je l'exécute, pourvu qu'il n'y ait pas mort d'homme, car Mercédès a dit qu'elle se tuerait si l'on tuait Dantès.»

Caderousse, qui avait laissé tomber sa tête sur la table releva le front, et regardant Fernand et Danglars avec des yeux lourds et hébétés:

«Tuer Dantès! dit-il, qui parle ici de tuer Dantès? je ne veux pas qu'on le tue, moi: c'est mon ami; il a offert ce matin de partager son argent avec moi, comme j'ai partagé le mien avec lui: je ne veux pas qu'on tue Dantès.

—Et qui te parle de le tuer, imbécile! reprit Danglars; il s'agit d'une simple plaisanterie; bois à sa santé, ajouta-t-il en remplissant le verre de Caderousse, et laisse-nous tranquilles.

—Oui, oui, à la santé de Dantès! dit Caderousse en vidant son verre, à sa santé!\dots à sa santé!\dots là!

—Mais le moyen, le moyen? dit Fernand.

—Vous ne l'avez donc pas trouvé encore, vous?

—Non, vous vous en êtes chargé.

—C'est vrai, reprit Danglars, les Français ont cette supériorité sur les Espagnols, que les Espagnols ruminent et que les Français inventent.

—Inventez donc alors, dit Fernand avec impatience.

—Garçon, dit Danglars, une plume, de l'encre et du papier!

—Une plume, de l'encre et du papier! murmura Fernand.

—Oui, je suis agent comptable: la plume, l'encre et le papier sont mes instruments; et sans mes instruments je ne sais rien faire.

—Une plume, de l'encre et du papier! cria à son tour Fernand.

—Il y a ce que vous désirez là sur cette table, dit le garçon en montrant les objets demandés.

—Donnez-les-nous alors.»

Le garçon prit le papier, l'encre et la plume, et les déposa sur la table du berceau.

«Quand on pense, dit Caderousse en laissant tomber sa main sur le papier, qu'il y a là de quoi tuer un homme plus sûrement que si on l'attendait au coin d'un bois pour l'assassiner! J'ai toujours eu plus peur d'une plume, d'une bouteille d'encre et d'une feuille de papier que d'une épée ou d'un pistolet.

—Le drôle n'est pas encore si ivre qu'il en a l'air, dit Danglars; versez-lui donc à boire, Fernand.»

Fernand remplit le verre de Caderousse, et celui-ci en véritable buveur qu'il était, leva la main de dessus le papier et la porta à son verre.

Le Catalan suivit le mouvement jusqu'à ce que Caderousse, presque vaincu par cette nouvelle attaque, reposât ou plutôt laissât retomber son verre sur la table.

«Eh bien? reprit le Catalan en voyant que le reste de la raison de Caderousse commençait à disparaître sous ce dernier verre de vin.

—Eh bien, je disais donc, par exemple, reprit Danglars, que si, après un voyage comme celui que vient de faire Dantès, et dans lequel il a touché à Naples et à l'île d'Elbe, quelqu'un le dénonçait au procureur du roi comme agent bonapartiste\dots.

—Je le dénoncerai, moi! dit vivement le jeune homme.

—Oui; mais alors on vous fait signer votre déclaration, on vous confronte avec celui que vous avez dénoncé: je vous fournis de quoi soutenir votre accusation, je le sais bien; mais Dantès ne peut rester éternellement en prison, un jour ou l'autre il en sort, et, ce jour où il en sort, malheur à celui qui l'y a fait entrer!

—Oh! je ne demande qu'une chose, dit Fernand, c'est qu'il vienne me chercher une querelle!

—Oui, et Mercédès! Mercédès, qui vous prend en haine si vous avez seulement le malheur d'écorcher l'épiderme à son bien-aimé Edmond!

—C'est juste, dit Fernand.

—Non, non, reprit Danglars, si on se décidait à une pareille chose, voyez-vous, il vaudrait bien mieux prendre tout bonnement comme je le fais, cette plume, la tremper dans l'encre, et écrire de la main gauche, pour que l'écriture ne fût pas reconnue, une petite dénonciation ainsi conçue.»

Et Danglars, joignant l'exemple au précepte, écrivit de la main gauche et d'une écriture renversée, qui n'avait aucune analogie avec son écriture habituelle, les lignes suivantes qu'il passa à Fernand, et que Fernand lut à demi-voix:

\begin{mail}{}{}
Monsieur le procureur du roi est prévenu, par un ami du trône et de la religion, que le nommé Edmond Dantès, second du navire le \textit{Pharaon}, arrivé ce matin de Smyrne, après avoir touché à Naples et à Porto-Ferrajo, a été chargé, par Murat, d'une lettre pour l'usurpateur, et, par l'usurpateur, d'une lettre pour le comité bonapartiste de Paris.

On aura la preuve de son crime en l'arrêtant, car on trouvera cette lettre ou sur lui, ou chez son père, ou dans sa cabine à bord du \textit{Pharaon}.
\end{mail}

«À la bonne heure, continua Danglars; ainsi votre vengeance aurait le sens commun, car d'aucune façon alors elle ne pourrait retomber sur vous, et la chose irait toute seule; il n'y aurait plus qu'à plier cette lettre, comme je le fais, et à écrire dessus: «À Monsieur le Procureur royal.» Tout serait dit.»

Et Danglars écrivit l'adresse en se jouant.

«Oui, tout serait dit», s'écria Caderousse, qui par un dernier effort d'intelligence avait suivi la lecture, et qui comprenait d'instinct tout ce qu'une pareille dénonciation pourrait entraîner de malheur; «oui, tout serait dit: seulement, ce serait une infamie.»

Et il allongea le bras pour prendre la lettre.

«Aussi, dit Danglars en la poussant hors de la portée de sa main, aussi, ce que je dis et ce que je dis et ce que je fais, c'est en plaisantant; et, le premier, je serais bien fâché qu'il arrivât quelque chose à Dantès, ce bon Dantès! Aussi, tiens\dots»

Il prit la lettre, la froissa dans ses mains et la jeta dans un coin de la tonnelle.

«À la bonne heure, dit Caderousse, Dantès est mon ami, et je ne veux pas qu'on lui fasse de mal.

—Eh! qui diable y songe à lui faire du mal! ce n'est ni moi ni Fernand! dit Danglars en se levant et en regardant le jeune homme qui était demeuré assis, mais dont l'œil oblique couvait le papier dénonciateur jeté dans un coin.

—En ce cas, reprit Caderousse, qu'on nous donne du vin: je veux boire à la santé d'Edmond et de la belle Mercédès.

—Tu n'as déjà que trop bu, ivrogne, dit Danglars, et si tu continues tu seras obligé de coucher ici, attendu que tu ne pourras plus te tenir sur tes jambes.

—Moi, dit Caderousse en se levant avec la fatuité de l'homme ivre; moi, ne pas pouvoir me tenir sur mes jambes! Je parie que je monte au clocher des Accoules, et sans balancer encore!

—Eh bien, soit, dit Danglars, je parie, mais pour demain: aujourd'hui il est temps de rentrer; donne-moi donc le bras et rentrons.

—Rentrons, dit Caderousse, mais je n'ai pas besoin de ton bras pour cela. Viens-tu, Fernand? rentres-tu avec nous à Marseille?

—Non, dit Fernand, je retourne aux Catalans, moi.

—Tu as tort, viens avec nous à Marseille, viens.

—Je n'ai point besoin à Marseille, et je n'y veux point aller.

—Comment as-tu dit cela? Tu ne veux pas, mon bonhomme! eh bien, à ton aise! liberté pour tout le monde! Viens, Danglars, et laissons monsieur rentrer aux Catalans, puisqu'il le veut.»

Danglars profita de ce moment de bonne volonté de Caderousse pour l'entraîner du côté de Marseille; seulement, pour ouvrir un chemin plus court et plus facile à Fernand, au lieu de revenir par le quai de la Rive-Neuve, il revint par la porte Saint-Victor.

Caderousse le suivait, tout chancelant, accroché à son bras.

Lorsqu'il eut fait une vingtaine de pas, Danglars se retourna et vit Fernand se précipiter sur le papier, qu'il mit dans sa poche; puis aussitôt, s'élançant hors de la tonnelle, le jeune homme tourna du côté du Pillon.

«Eh bien, que fait-il donc? dit Caderousse, il nous a menti: il a dit qu'il allait aux Catalans, et il va à la ville! Holà! Fernand! tu te trompes, mon garçon!

—C'est toi qui vois trouble, dit Danglars, il suit tout droit le chemin des Vieilles-Infirmeries.

—En vérité! dit Caderousse, eh bien, j'aurais juré qu'il tournait à droite; décidément le vin est un traître.

—Allons, allons, murmura Danglars, je crois que maintenant la chose est bien lancée, et qu'il n'y a plus qu'à la laisser marcher toute seule.»



