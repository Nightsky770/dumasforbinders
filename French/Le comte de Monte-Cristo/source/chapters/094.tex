\chapter{L'aveu}

\lettrine{A}{u} même instant, on entendit la voix de M. de Villefort, qui de son cabinet criait: 

\zz
«Qu'y a-t-il?» 

\zz
Morrel consulta du regard Noirtier, qui venait de reprendre tout son sang-froid, et qui d'un coup d'œil lui indiqua le cabinet où déjà une fois, dans une circonstance à peu près pareille, il s'était réfugié. 

Il n'eut que le temps de prendre son chapeau et de s'y jeter tout haletant. On entendait les pas du procureur du roi dans le corridor. 

Villefort se précipita dans la chambre, courut à Valentine et la prit entre ses bras. 

«Un médecin! un médecin!\dots M. d'Avrigny! cria Villefort, ou plutôt j'y vais moi-même.» 

Et il s'élança hors de l'appartement. 

Par l'autre porte s'élançait Morrel. 

Il venait d'être frappé au cœur par un épouvantable souvenir: cette conversation entre Villefort et le docteur, qu'il avait entendue la nuit où mourut Mme de Saint-Méran, lui revenait à la mémoire; ces symptômes, portés à un degré moins effrayant, étaient les mêmes qui avaient précédé la mort de Barrois. 

En même temps il lui avait semblé entendre bruire à son oreille cette voix de Monte-Cristo, qui lui avait dit, il y avait deux heures à peine: 

«De quelque chose que vous ayez besoin, Morrel, venez à moi, je peux beaucoup.» 

Plus rapide que la pensée, il s'élança donc du faubourg Saint-Honoré dans la rue Matignon, et de la rue Matignon dans l'avenue des Champs-Élysées. 

Pendant ce temps, M. de Villefort arrivait, dans un cabriolet de place, à la porte de M. d'Avrigny; il sonna avec tant de violence, que le concierge vint ouvrir d'un air effrayé. Villefort s'élança dans l'escalier sans avoir la force de rien dire. Le concierge le connaissait et le laissa en criant seulement: 

«Dans son cabinet, M. le procureur du roi, dans son cabinet!» 

Villefort en poussait déjà ou plutôt en enfonçait la porte. 

«Ah! dit le docteur, c'est vous! 

—Oui, dit Villefort en refermant la porte derrière lui; oui, docteur, c'est moi qui viens vous demander à mon tour si nous sommes bien seuls. Docteur, ma maison est une maison maudite! 

—Quoi! dit celui-ci froidement en apparence, mais avec une profonde émotion intérieure, avez-vous encore quelque malade? 

—Oui, docteur! s'écria Villefort en saisissant d'une main convulsive une poignée de cheveux, oui!» 

Le regard de d'Avrigny signifia: «Je vous l'avais prédit.» 

Puis ses lèvres accentuèrent lentement ces mots: 

«Qui va donc mourir chez vous et quelle nouvelle victime va nous accuser de faiblesse devant Dieu?» 

Un sanglot douloureux jaillit du cœur de Villefort; il s'approcha du médecin, et lui saisissant le bras: 

«Valentine! dit-il, c'est le tour de Valentine! 

—Votre fille! s'écria d'Avrigny, saisi de douleur et de surprise. 

—Vous voyez que vous vous trompiez, murmura le magistrat; venez la voir, et sur son lit de douleur, demandez-lui pardon de l'avoir soupçonnée. 

—Chaque fois que vous m'avez prévenu, dit M. d'Avrigny, il était trop tard: n'importe, j'y vais; mais hâtons-nous, monsieur, avec les ennemis qui frappent chez vous, il n'y a pas de temps à perdre. 

—Oh! cette fois, docteur, vous ne me reprocherez plus ma faiblesse. Cette fois, je connaîtrai l'assassin et je frapperai. 

—Essayons de sauver la victime avant de penser à la venger, dit d'Avrigny. Venez.» 

Et le cabriolet qui avait amené Villefort le ramena au grand trot, accompagné de d'Avrigny, au moment même où, de son côté, Morrel frappait à la porte de Monte-Cristo. 

Le comte était dans son cabinet, et, fort soucieux, lisait un mot que Bertuccio venait de lui envoyer à la hâte. 

En entendant annoncer Morrel, qui le quittait il y avait deux heures à peine, le comte releva la tête. 

Pour lui, comme pour le comte, il s'était sans doute passé bien des choses pendant ces deux heures, car le jeune homme, qui l'avait quitté le sourire sur les lèvres revenait le visage bouleversé. 

Il se leva et s'élança au-devant de Morrel. 

«Qu'y a-t-il donc, Maximilien? Lui demanda-t-il; vous êtes pâle, et votre front ruisselle de sueur.» 

Morrel tomba sur un fauteuil plutôt qu'il ne s'assit. 

«Oui, dit-il, je suis venu vite, j'avais besoin de vous parler. 

—Tout le monde se porte bien dans votre famille? demanda le comte avec un ton de bienveillance affectueuse à la sincérité de laquelle personne ne se fût trompé. 

—Merci, comte, merci, dit le jeune homme visiblement embarrassé pour commencer l'entretien; oui, dans ma famille tout le monde se porte bien. 

—Tant mieux; cependant vous avez quelque chose à me dire? reprit le comte, de plus en plus inquiet. 

—Oui, dit Morrel, c'est vrai je viens de sortir d'une maison où la mort venait d'entrer, pour accourir à vous. 

—Sortez-vous donc de chez M. de Morcerf? demanda Monte-Cristo. 

—Non, dit Morrel; quelqu'un est-il mort chez M. de Morcerf? 

—Le général vient de se brûler la cervelle, répondit Monte-Cristo. 

—Oh! l'affreux malheur! s'écria Maximilien. 

—Pas pour la comtesse, pas pour Albert, dit Monte-Cristo; mieux vaut un père et un époux mort qu'un père et un époux déshonoré; le sang lavera la honte. 

—Pauvre comtesse! dit Maximilien, c'est elle que je plains surtout, une si noble femme! 

—Plaignez aussi Albert, Maximilien; car, croyez-le, c'est le digne fils de la comtesse. Mais revenons à vous: vous accouriez vers moi, m'avez-vous dit; aurais-je le bonheur que vous eussiez besoin de moi? 

—Oui, j'ai besoin de vous, c'est-à-dire que j'ai cru comme un insensé que vous pouviez me porter secours dans une circonstance où Dieu seul peut me secourir. 

—Dites toujours, répondit Monte-Cristo. 

—Oh! dit Morrel, je ne sais en vérité s'il m'est permis de révéler un pareil secret à des oreilles humaines; mais la fatalité m'y pousse, la nécessité m'y contraint, comte.» 

Morrel s'arrêta hésitant. 

«Croyez-vous que je vous aime? dit Monte-Cristo, prenant affectueusement la main du jeune homme entre les siennes. 

—Oh! tenez, vous m'encouragez, et puis quelque chose me dit là (Morrel posa la main sur son cœur) que je ne dois pas avoir de secret pour vous. 

—Vous avez raison, Morrel, c'est Dieu qui parle à votre cœur, et c'est votre cœur qui vous parle. Redites-moi ce que vous dit votre cœur. 

—Comte, voulez-vous me permettre d'envoyer Baptistin demander de votre part des nouvelles de quelqu'un que vous connaissez? 

—Je me suis mis à votre disposition, à plus forte raison j'y mets mes domestiques. 

—Oh! c'est que je ne vivrai pas, tant que je n'aurai pas la certitude qu'elle va mieux. 

—Voulez-vous que je sonne Baptistin? 

—Non, je vais lui parler moi-même.» 

Morrel sortit, appela Baptistin et lui dit quelques mots tout bas. Le valet de chambre partit tout courant. 

«Eh bien, est-ce fait? demanda Monte-Cristo en voyant reparaître Morrel. 

—Oui, et je vais être un peu plus tranquille. 

—Vous savez que j'attends, dit Monte-Cristo souriant. 

—Oui, et, moi, je parle. Écoutez, un soir je me trouvais dans un jardin; j'étais caché par un massif d'arbres, nul ne se doutait que je pouvais être là. Deux personnes passèrent près de moi; permettez que je taise provisoirement leurs noms, elles causaient à voix basse, et cependant j'avais un tel intérêt à entendre leurs paroles que je ne perdais pas un mot de ce qu'elles disaient. 

—Cela s'annonce lugubrement, si j'en crois votre pâleur et votre frisson, Morrel. 

—Oh oui! bien lugubrement, mon ami! Il venait de mourir quelqu'un chez le maître du jardin où je me trouvais; l'une des deux personnes dont j'entendais la conversation était le maître de ce jardin, et l'autre était le médecin. Or, le premier confiait au second ses craintes et ses douleurs; car c'était la seconde fois depuis un mois que la mort s'abattait, rapide et imprévue, sur cette maison, qu'on croirait désignée par quelque ange exterminateur à la colère de Dieu. 

—Ah! ah!» dit Monte-Cristo en regardant fixement le jeune homme, et en tournant son fauteuil par un mouvement imperceptible de manière à se placer dans l'ombre, tandis que le jour frappait le visage de Maximilien. 

«Oui, continua celui-ci, la mort était entrée deux fois dans cette maison en un mois. 

—Et que répondait le docteur? demanda Monte-Cristo. 

—Il répondait\dots il répondait que cette mort n'était point naturelle, et qu'il fallait l'attribuer\dots 

—À quoi? 

—Au poison! 

—Vraiment! dit Monte-Cristo avec cette toux légère qui, dans les moments de suprême émotion, lui servait à déguiser soit sa rougeur, soit sa pâleur, soit l'attention même avec laquelle il écoutait; vraiment, Maximilien, vous avez entendu de ces choses-là? 

—Oui, cher comte, je les ai entendues, et le docteur a ajouté que, si pareil événement se renouvelait, il se croirait obligé d'en appeler à la justice.» 

Monte-Cristo écoutait ou paraissait écouter avec le plus grand calme. 

«Eh bien, dit Maximilien, la mort a frappé une troisième fois, et ni le maître de la maison ni le docteur n'ont rien dit; la mort va frapper une quatrième fois, peut-être. Comte, à quoi croyez-vous que la connaissance de ce secret m'engage? 

—Mon cher ami, dit Monte-Cristo, vous me paraissez conter là une aventure que chacun de nous sait par cœur. La maison où vous avez entendu cela, je la connais, ou tout au moins j'en connais une pareille; une maison où il y a un jardin, un père de famille, un docteur, une maison où il y a eu trois morts étranges et inattendues. Eh bien! regardez-moi, moi qui n'ai point intercepté de confidence et qui cependant sait tout cela aussi bien que vous, est-ce que j'ai des scrupules de conscience? Non, cela ne me regarde pas, moi. Vous dites qu'un ange exterminateur semble désigner cette maison à la colère du Seigneur; eh bien, qui vous dit que votre supposition n'est pas une réalité? Ne voyez pas les choses que ne veulent pas voir ceux qui ont intérêt à les voir. Si c'est la justice et non la colère de Dieu qui se promène dans cette maison, Maximilien, détournez la tête et laissez passer la justice de Dieu.» 

Morrel frissonna. Il y avait quelque chose à la fois de lugubre, de solennel et de terrible dans l'accent du comte. 

«D'ailleurs, continua-t-il avec un changement de voix si marqué qu'on eût dit que ces dernières paroles ne sortaient pas de la bouche du même homme; d'ailleurs, qui vous dit que cela recommencera? 

—Cela recommence, comte! s'écria Morrel, et voilà pourquoi j'accours chez vous. 

—Eh bien, que voulez-vous que j'y fasse, Morrel? Voudriez-vous, par hasard, que je prévinsse M. le procureur du roi?» 

Monte-Cristo articula ces dernières paroles avec tant de clarté et avec une accentuation si vibrante, que Morrel, se levant tout à coup, s'écria: 

«Comte! Comte! Vous savez de qui je veux parler, n'est-ce pas? 

—Eh! Parfaitement, mon bon ami, et je vais vous le prouver en mettant les points sur les \textit{i}, ou plutôt les noms sur les hommes. Vous vous êtes promené un soir dans le jardin de M. de Villefort; d'après ce que vous m'avez dit, je présume que c'est le soir de la mort de Mme de Saint-Méran. Vous avez entendu M. de Villefort causer avec M. d'Avrigny de la mort de M. de Saint-Méran et de celle non moins étonnante de la marquise. M. d'Avrigny disait qu'il croyait à un empoisonnement et même à deux empoisonnements; et vous voilà, vous honnête homme par excellence, vous voilà depuis ce moment occupé à palper votre cœur, à jeter la sonde dans votre conscience pour savoir s'il faut révéler ce secret ou le taire. Nous ne sommes plus au Moyen Âge, cher ami, et il n'y a plus de Sainte-Vehme, il n'y a plus de francs juges; que diable allez-vous demander à ces gens-là? Conscience, que me veux-tu? comme dit Sterne. Eh! Mon cher, laissez-les dormir s'ils dorment, laissez-les pâlir dans leurs insomnies, et, pour l'amour de Dieu, dormez, vous qui n'avez pas de remords qui vous empêchent de dormir.» 

Une effroyable douleur se peignit sur les traits de Morrel; il saisit la main de Monte-Cristo. 

«Mais cela recommence! vous dis-je. 

—Eh bien, dit le comte, étonné de cette insistance à laquelle il ne comprenait rien, et regardant Maximilien attentivement, laissez recommencer: c'est une famille d'Atrides; Dieu les a condamnés, et ils subiront la sentence; ils vont tous disparaître comme ces moines que les enfants fabriquent avec des cartes pliées, et qui tombent les uns après les autres sous le souffle de leur créateur, y en eût-il deux cents. C'était M. de Saint-Méran il y a trois mois, c'était Mme de Saint-Méran il y a deux mois; c'était Barrois l'autre jour; aujourd'hui c'est le vieux Noirtier ou la jeune Valentine. 

—Vous le saviez? s'écria Morrel dans un tel paroxysme de terreur, que Monte-Cristo tressaillit, lui que la chute du ciel eût trouvé impassible; vous le saviez et vous ne disiez rien! 

—Eh! que m'importe? reprit Monte-Cristo en haussant les épaules, est-ce que je connais ces gens-là, moi, et faut-il que je perde l'un pour sauver l'autre? Ma foi, non, car, entre le coupable et la victime, je n'ai pas de préférence. 

—Mais moi, moi! s'écria Morrel en hurlant de douleur, moi, je l'aime! 

—Vous aimez qui? s'écria Monte-Cristo en bondissant sur ses pieds et en saisissant les deux mains que Morrel élevait, en les tordant, vers le ciel. 

—J'aime éperdument, j'aime en insensé, j'aime en homme qui donnerait tout son sang pour lui épargner une larme; j'aime Valentine de Villefort, qu'on assassine en ce moment, entendez-vous bien! je l'aime, et je demande à Dieu et à vous comment je puis la sauver!» 

Monte-Cristo poussa un cri sauvage dont peuvent seuls se faire une idée ceux qui ont entendu le rugissement du lion blessé. 

«Malheureux! s'écria-t-il en se tordant les mains à son tour, malheureux! tu aimes Valentine! tu aimes cette fille d'une race maudite!» 

Jamais Morrel n'avait vu semblable expression; jamais œil si terrible n'avait flamboyé devant son visage, jamais le génie de la terreur, qu'il avait vu tant de fois apparaître, soit sur les champs de bataille, soit dans les nuits homicides de l'Algérie, n'avait secoué autour de lui de feux plus sinistres. 

Il recula épouvanté. 

Quant à Monte-Cristo, après cet éclat et ce bruit, il ferma un moment les yeux, comme ébloui par des éclairs intérieurs: pendant ce moment, il se recueillit avec tant de puissance, que l'on voyait peu à peu s'apaiser le mouvement onduleux de sa poitrine gonflée de tempêtes, comme on voit après la nuée se fondre sous le soleil les vagues turbulentes et écumeuses. 

Ce silence, ce recueillement, cette lutte, durèrent vingt secondes à peu près. 

Puis le comte releva son front pâli. 

«Voyez, dit-il d'une voix altérée, voyez, cher ami, comme Dieu sait punir de leur indifférence les hommes les plus fanfarons et les plus froids devant les terribles spectacles qu'il leur donne. Moi qui regardais, assistant impassible et curieux, moi qui regardais le développement de cette lugubre tragédie, moi qui, pareil au mauvais ange, riais du mal que font les hommes, à l'abri derrière le secret (et le secret est facile à garder pour les riches et les puissants), voilà qu'à mon tour je me sens mordu par ce serpent dont je regardais la marche tortueuse, et mordu au cœur!» 

Morrel poussa un sourd gémissement. 

«Allons, allons, continua le comte, assez de plaintes comme cela, soyez homme, soyez fort, soyez plein d'espoir, car je suis là, car je veille sur vous.» 

Morrel secoua tristement la tête. 

«Je vous dis d'espérer! me comprenez-vous? s'écria Monte-Cristo. Sachez bien que jamais je ne mens, que jamais je ne me trompe. Il est midi, Maximilien, rendez grâce au ciel de ce que vous êtes venu à midi au lieu de venir ce soir, au lieu de venir demain matin. Écoutez donc ce que je vais vous dire, Morrel: il est midi; si Valentine n'est pas morte à cette heure, elle ne mourra pas. 

—Oh! mon Dieu! mon Dieu! s'écria Morrel, moi qui l'ai laissée mourante!» 

Monte-Cristo appuya une main sur son front. 

Que se passa-t-il dans cette tête si lourde d'effrayants secrets? 

Que dit à cet esprit, implacable et humain à la fois, l'ange lumineux ou l'ange des ténèbres? 

Dieu seul le sait! 

Monte-Cristo releva le front encore une fois, et cette fois il était calme comme l'enfant qui se réveille. 

«Maximilien, dit-il, retournez tranquillement chez vous; je vous commande de ne pas faire un pas, de ne pas tenter une démarche, de ne pas laisser flotter sur votre visage l'ombre d'une préoccupation; je vous donnerai des nouvelles; allez. 

—Mon Dieu! mon Dieu! dit Morrel, vous m'épouvantez, comte, avec ce sang-froid. Pouvez-vous donc quelque chose contre la mort? Êtes-vous plus qu'un homme? Êtes-vous un ange? Êtes-vous un Dieu?» 

Et le jeune homme, qu'aucun danger n'avait fait reculer d'un pas, reculait devant Monte-Cristo, saisi d'une indicible terreur. 

Mais Monte-Cristo le regarda avec un sourire à la fois si mélancolique et si doux, que Maximilien sentit les larmes poindre dans ses yeux. 

«Je peux beaucoup, mon ami, répondit le comte. Allez, j'ai besoin d'être seul.» 

Morrel, subjugué par ce prodigieux ascendant qu'exerçait Monte-Cristo sur tout ce qui l'entourait, n'essaya pas même de s'y soustraire. Il serra la main du comte et sortit. 

Seulement, à la porte, il s'arrêta pour attendre Baptistin, qu'il venait de voir apparaître au coin de la rue Matignon, et qui revenait tout courant. 

Cependant, Villefort et d'Avrigny avaient fait diligence. À leur retour, Valentine était encore évanouie, et le médecin avait examiné la malade avec le soin que commandait la circonstance et avec une profondeur que doublait la connaissance du secret. 

Villefort suspendu à son regard et à ses lèvres, attendait le résultat de l'examen. Noirtier, plus pâle que la jeune fille, plus avide d'une solution que Villefort lui-même, attendait aussi, et tout en lui se faisait intelligence et sensibilité. 

Enfin, d'Avrigny laissa échapper lentement: 

«Elle vit encore. 

—Encore! s'écria Villefort, oh! docteur, quel terrible mot vous avez prononcé là! 

—Oui, dit le médecin, je répète ma phrase: elle vit encore, et j'en suis bien surpris. 

—Mais elle est sauvée? demanda le père. 

—Oui, puisqu'elle vit.» 

En ce moment le regard de d'Avrigny rencontra l'œil de Noirtier, il étincelait d'une joie si extraordinaire d'une pensée tellement riche et féconde, que le médecin en fut frappé. 

Il laissa retomber sur le fauteuil la jeune fille, dont les lèvres se dessinaient à peine, tant pâles et blanches elles étaient, à l'unisson du reste du visage, et demeura immobile et regardant Noirtier, par qui tout mouvement du docteur était attendu et commenté. 

«Monsieur, dit alors d'Avrigny à Villefort, appelez la femme de chambre de Mlle Valentine, s'il vous plaît.» 

Villefort quitta la tête de sa fille qu'il soutenait et courut lui-même appeler la femme de chambre. 

Aussitôt que Villefort eut refermé la porte, d'Avrigny s'approcha de Noirtier. 

«Vous avez quelque chose à me dire?» demanda-t-il. 

Le vieillard cligna expressivement des yeux; c'était, on se le rappelle, le seul signe affirmatif qui fût à sa disposition. 

«À moi seul? 

—Oui, fit Noirtier. 

—Bien, je demeurerai avec vous.» 

En ce moment Villefort rentra, suivi de la femme de chambre; derrière la femme de chambre marchait Mme de Villefort. 

«Mais qu'a donc fait cette chère enfant? s'écria-t-elle, elle sort de chez moi et elle s'est bien plainte d'être indisposée, mais je n'avais pas cru que c'était sérieux.» 

Et la jeune femme, les larmes aux yeux, et avec toutes les marques d'affection d'une véritable mère s'approcha de Valentine, dont elle prit la main. 

D'Avrigny continua de regarder Noirtier, il vit les yeux du vieillard se dilater et s'arrondir, ses joues blêmir et trembler; la sueur perla sur son front. 

«Ah!» fit-il involontairement, en suivant la direction du regard de Noirtier, c'est-à-dire en fixant ses yeux sur Mme de Villefort, qui répétait: 

«Cette pauvre enfant sera mieux dans son lit. Venez, Fanny, nous la coucherons.» 

M. d'Avrigny, qui voyait dans cette proposition un moyen de rester seul avec Noirtier, fit signe de la tête que c'était effectivement ce qu'il y avait de mieux à faire, mais il défendit qu'elle prit rien au monde que ce qu'il ordonnerait. 

On emporta Valentine, qui était revenue à la connaissance, mais qui était incapable d'agir et presque de parler, tant ses membres étaient brisés par la secousse qu'elle venait d'éprouver. Cependant elle eut la force de saluer d'un coup d'œil son grand-père, dont il semblait qu'on arrachât l'âme en l'emportant. 

D'Avrigny suivit la malade, termina ses prescriptions, ordonna à Villefort de prendre un cabriolet, d'aller en personne chez le pharmacien faire préparer devant lui les potions ordonnées, de les rapporter lui-même et de l'attendre dans la chambre de sa fille. 

Puis, après avoir renouvelé l'injonction de ne rien laisser prendre à Valentine, il redescendit chez Noirtier, ferma soigneusement les portes, et après s'être assuré que personne n'écoutait: 

«Voyons, dit-il, vous savez quelque chose sur cette maladie de votre petite-fille? 

—Oui, fit le vieillard. 

—Écoutez, nous n'avons pas de temps à perdre, je vais vous interroger et vous me répondrez.» 

Noirtier fit signe qu'il était prêt à répondre. 

«Avez-vous prévu l'accident qui est arrivé aujourd'hui à Valentine? 

—Oui.» 

D'Avrigny réfléchit un instant puis se rapprochant de Noirtier: 

«Pardonnez-moi ce que je vais vous dire, ajouta-t-il, mais nul indice ne doit être négligé dans la situation terrible où nous sommes. Vous avez vu mourir le pauvre Barrois?» 

Noirtier leva les yeux au ciel. 

«Savez-vous de quoi il est mort? demanda d'Avrigny en posant sa main sur l'épaule de Noirtier. 

—Oui, répondit le vieillard. 

—Pensez-vous que sa mort ait été naturelle?» 

Quelque chose comme un sourire s'esquissa sur les lèvres inertes de Noirtier. 

«Alors l'idée que Barrois avait été empoisonné vous est venue? 

—Oui. 

—Croyez-vous que ce poison dont il a été victime lui ait été destiné? 

—Non. 

—Maintenant pensez-vous que ce soit la même main qui a frappé Barrois, en voulant frapper un autre, qui frappe aujourd'hui Valentine? 

—Oui. 

—Elle va donc succomber aussi?» demanda d'Avrigny en fixant son regard profond sur Noirtier. 

Et il attendit l'effet de cette phrase sur le vieillard. 

«Non, répondit-il avec un air de triomphe qui eût pu dérouter toutes les conjectures du plus habile devin. 

—Alors vous espérez? dit d'Avrigny avec surprise. 

—Oui. 

—Qu'espérez-vous? 

Le vieillard fit comprendre des yeux qu'il ne pouvait répondre. 

«Ah! oui, c'est vrai», murmura d'Avrigny. 

Puis revenant à Noirtier: 

«Vous espérez, dit-il, que l'assassin se lassera? 

—Non. 

—Alors, vous espérez que le poison sera sans effet sur Valentine? 

—Oui. 

—Car je ne vous apprends rien, n'est-ce pas, ajouta d'Avrigny, en vous disant qu'on vient d'essayer de l'empoisonner?» 

Le vieillard fit signe des yeux qu'il ne conservait aucun doute à ce sujet. 

«Alors, comment espérez-vous que Valentine échappera?» 

Noirtier tint avec obstination ses yeux fixés du même côté, d'Avrigny suivit la direction de ses yeux et vit qu'ils étaient attachés sur une bouteille contenant la potion qu'on lui apportait tous les matins. 

«Ah! ah! dit d'Avrigny, frappé d'une idée subite, auriez-vous eu l'idée\dots» 

Noirtier ne le laissa point achever. 

«Oui, fit-il. 

—De la prémunir contre le poison\dots 

—Oui. 

—En l'habituant peu à peu\dots 

—Oui, oui, oui, fit Noirtier, enchanté d'être compris. 

—En effet, vous m'avez entendu dire qu'il entrait de la brucine dans les potions que je vous donne? 

—Oui. 

—Et en l'accoutumant à ce poison, vous avez voulu neutraliser les effets d'un poison?» 

Même joie triomphante de Noirtier. 

«Et vous y êtes parvenu en effet! s'écria d'Avrigny. Sans cette précaution, Valentine était tuée aujourd'hui, tuée sans secours possible, tuée sans miséricorde, la secousse a été violente, mais elle n'a été qu'ébranlée, et cette fois du moins Valentine ne mourra pas.» 

Une joie surhumaine épanouissait les yeux du vieillard, levés au ciel avec une expression de reconnaissance infinie. 

En ce moment Villefort rentra. 

«Tenez, docteur, dit-il, voici ce que vous avez demandé. 

—Cette potion a été préparée devant vous? 

—Oui, répondit le procureur du roi. 

—Elle n'est pas sortie de vos mains? 

—Non.» 

D'Avrigny prit la bouteille, versa quelques gouttes du breuvage qu'elle contenait dans le creux de sa main et les avala. 

«Bien, dit-il, montons chez Valentine, j'y donnerai mes instructions à tout le monde, et vous veillerez vous-même, monsieur de Villefort, à ce que personne ne s'en écarte.» 

Au moment où d'Avrigny rentrait dans la chambre de Valentine, accompagnée de Villefort, un prêtre italien, à la démarche sévère, aux paroles calmes et décidées, louait pour son usage la maison attenante à l'hôtel habité par M. de Villefort. 

On ne put savoir en vertu de quelle transaction les trois locataires de cette maison déménagèrent deux heures après: mais le bruit qui courut généralement dans le quartier fut que la maison n'était pas solidement assise sur ses fondations et menaçait ruine ce qui n'empêchait point le nouveau locataire de s'y établir avec son modeste mobilier le jour même, vers les cinq heures. 

Ce bail fut fait pour trois, six ou neuf ans par le nouveau locataire, qui, selon l'habitude établie par les propriétaires, paya six mois d'avance; ce nouveau locataire, qui, ainsi que nous l'avons dit, était italien, s'appelait-il signor Giacomo Busoni. 

Des ouvriers furent immédiatement appelés, et la nuit même les rares passants attardés au haut du faubourg voyaient avec surprise les charpentiers et les maçons occupés à reprendre en sous-œuvre la maison chancelante. 