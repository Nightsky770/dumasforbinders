\chapter{Le rendez-vous}

\lettrine{L}{e} lendemain, en se levant, le premier mot d'Albert fut pour proposer à Franz d'aller faire une visite au comte; il l'avait déjà remercié la veille, mais il comprenait qu'un service comme celui qu'il lui avait rendu valait bien deux remerciements. 

Franz, qu'un attrait mêlé de terreur attirait vers le comte de Monte-Cristo, ne voulut pas le laisser aller seul chez cet homme et l'accompagna; tous deux furent introduits dans le salon: cinq minutes après, le comte parut. 

«Monsieur le comte, lui dit Albert en allant à lui, permettez-moi de vous répéter ce matin ce que je vous ai mal dit hier: c'est que je n'oublierai jamais dans quelle circonstance vous m'êtes venu en aide, et que je me souviendrai toujours que je vous dois la vie ou à peu près. 

—Mon cher voisin, répondit le comte en riant, vous vous exagérez vos obligations envers moi. Vous me devez une petite économie d'une vingtaine de mille francs sur votre budget de voyage et voilà tout; vous voyez bien que ce n'est pas la peine d'en parler. De votre côté, ajouta-t-il, recevez tous mes compliments, vous avez été adorable de sans-gêne et de laisser-aller. 

—Que voulez-vous, comte, dit Albert; je me suis figuré que je m'étais fait une mauvaise querelle et qu'un duel s'en était suivi, et j'ai voulu faire comprendre une chose à ces bandits: c'est qu'on se bat dans tous les pays du monde, mais qu'il n'y a que les Français qui se battent en riant. Néanmoins, comme mon obligation vis-à-vis de vous n'en est pas moins grande, je viens vous demander si, par moi, par mes amis et par mes connaissances, je ne pourrais pas vous être bon à quelque chose. Mon père, le comte de Morcerf, qui est d'origine espagnole, a une haute position en France et en Espagne, je viens me mettre, moi et tous les gens qui m'aiment, à votre disposition. 

—Eh bien, dit le comte, je vous avoue, monsieur de Morcerf, que j'attendais votre offre et que je l'accepte de grand cœur. J'avais déjà jeté mon dévolu sur vous pour vous demander un grand service. 

—Lequel? 

—Je n'ai jamais été à Paris! je ne connais pas Paris\dots. 

—Vraiment! s'écria Albert, vous avez pu vivre jusqu'à présent sans voir Paris? c'est incroyable! 

—C'est ainsi, cependant; mais je sens comme vous qu'une plus longue ignorance de la capitale du monde intelligent est chose impossible. Il y a plus: peut-être même aurais-je fait ce voyage indispensable depuis longtemps, si j'avais connu quelqu'un qui pût m'introduire dans ce monde où je n'avais aucune relation.  

—Oh! un homme comme vous! s'écria Albert. 

—Vous êtes bien bon, mais comme je ne me reconnais à moi-même d'autre mérite que de pouvoir faire concurrence comme millionnaire à M. Aguado ou à M. Rothschild, et que je ne vais pas à Paris pour jouer à la Bourse, cette petite circonstance m'a retenu. Maintenant votre offre me décide. Voyons, vous engagez-vous, mon cher monsieur de Morcerf (le comte accompagna ces mots d'un singulier sourire), vous engagez-vous, lorsque j'irai en France, à m'ouvrir les portes de ce monde où je serai aussi étranger qu'un Huron ou qu'un Cochinchinois? 

—Oh! quant à cela, monsieur le comte, à merveille et de grand cœur! répondit Albert; et d'autant plus volontiers (mon cher Franz, ne vous moquez pas trop de moi!) que je suis rappelé à Paris par une lettre que je reçois ce matin même et où il est question pour moi d'une alliance avec une maison fort agréable et qui a les meilleures relations dans le monde parisien. 

—Alliance par mariage? dit Franz en riant. 

—Oh! mon Dieu, oui! Ainsi, quand vous reviendrez à Paris vous me trouverez homme posé et peut-être père de famille. Cela ira bien à ma gravité naturelle, n'est-ce pas? En tout cas, comte, je vous le répète, moi et les miens sommes à vous corps et âme. 

—J'accepte, dit le comte, car je vous jure qu'il ne me manquait que cette occasion pour réaliser des projets que je rumine depuis longtemps.» 

Franz ne douta point un instant que ces projets ne fussent ceux dont le comte avait laissé échapper un mot dans la grotte de Monte-Cristo, et il regarda le comte pendant qu'il disait ces paroles pour essayer de saisir sur sa physionomie quelque révélation de ces projets qui le conduisaient à Paris; mais il était bien difficile de pénétrer dans l'âme de cet homme, surtout lorsqu'il la voilait avec un sourire. 

«Mais, voyons, comte, reprit Albert enchanté d'avoir à produire un homme comme Monte-Cristo, n'est-ce pas là un de ces projets en l'air, comme on en fait mille en voyage, et qui, bâtis sur du sable, sont emportés au premier souffle du vent?  

—Non, d'honneur, dit le comte; je veux aller à Paris, il faut que j'y aille. 

—Et quand cela? 

—Mais quand y serez-vous vous-même? 

—Moi, dit Albert; oh! mon Dieu! dans quinze jours ou trois semaines au plus tard; le temps de revenir. 

—Eh bien, dit le comte, je vous donne trois mois; vous voyez que je vous fais la mesure large. 

—Et dans trois mois, s'écria Albert avec joie, vous venez frapper à ma porte? 

—Voulez-vous un rendez-vous jour pour jour, heure pour heure? dit le comte, je vous préviens que je suis d'une exactitude désespérante. 

—Jour pour jour, heure pour heure, dit Albert; cela me va à merveille. 

—Eh bien, soit. Il étendit la main vers un calendrier suspendu près de la glace. Nous sommes aujourd'hui, dit-il, le 21 février (il tira sa montre); il est dix heures et demie du matin. Voulez-vous m'attendre le 21 mai prochain, à dix heures et demie du matin?  

—À merveille! dit Albert, le déjeuner sera prêt. 

—Vous demeurez? 

—Rue du Helder, n° 27. 

—Vous êtes chez vous en garçon, je ne vous gênerai pas? 

—J'habite dans l'hôtel de mon père, mais un pavillon au fond de la cour entièrement séparé. 

—Bien.» 

Le comte prit ses tablettes et écrivit: «Rue du Helder, n° 27, 21 mai, à dix heures et demie du matin.» 

«Et maintenant, dit le comte en remettant ses tablettes dans sa poche, soyez tranquille, l'aiguille de votre pendule ne sera pas plus exacte que moi. 

—Je vous reverrai avant mon départ? demanda Albert. 

—C'est selon: quand partez-vous? 

—Je pars demain, à cinq heures du soir. 

—En ce cas, je vous dis adieu. J'ai affaire à Naples et ne serai de retour ici que samedi soir ou dimanche matin. Et vous, demanda le comte à Franz, partez-vous aussi, monsieur le baron?  

—Oui. 

—Pour la France? 

—Non, pour Venise. Je reste encore un an ou deux en Italie. 

—Nous ne nous verrons donc pas à Paris? 

—Je crains de ne pas avoir cet honneur. 

—Allons, messieurs, bon voyage», dit le comte aux deux amis en leur tendant à chacun une main. 

C'était la première fois que Franz touchait la main de cet homme; il tressaillit, car elle était glacée comme celle d'un mort. 

«Une dernière fois, dit Albert, c'est bien arrêté, sur parole d'honneur, n'est-ce pas? rue du Helder, n° 27, le 21 mai, à dix heures et demie du matin? 

—Le 21 mai, à dix heures et demie du matin, rue du Helder, n° 27», reprit le comte. 

Sur quoi les deux jeunes gens saluèrent le comte et sortirent. 

«Qu'avez-vous donc? dit en rentrant chez lui Albert à Franz, vous avez l'air tout soucieux. 

—Oui, dit Franz, je vous l'avoue, le comte est un homme singulier, et je vois avec inquiétude ce rendez-vous qu'il vous a donné à Paris. 

—Ce rendez-vous\dots avec inquiétude! Ah çà! mais êtes-vous fou, mon cher Franz? s'écria Albert. 

—Que voulez-vous, dit Franz, fou ou non, c'est ainsi. 

—Écoutez, reprit Albert, et je suis bien aise que l'occasion se présente de vous dire cela, mais je vous ai toujours trouvé assez froid pour le comte, que, de son côté, j'ai toujours trouvé parfait, au contraire, pour nous. Avez-vous quelque chose de particulier contre lui? 

—Peut-être. 

—L'aviez-vous vu déjà quelque part avant de le rencontrer ici? 

—Justement. 

—Où cela? 

—Me promettez-vous de ne pas dire un mot de ce que je vais vous raconter? 

—Je vous le promets. 

—Parole d'honneur?  

—Parole d'honneur. 

—C'est bien. Écoutez donc. 

Et alors Franz raconta à Albert son excursion à l'île de Monte-Cristo, comment il y avait trouvé un équipage de contrebandiers, et au milieu de cet équipage deux bandits corses. Il s'appesantit sur toutes les circonstances de l'hospitalité féerique que le comte lui avait donnée dans sa grotte des \textit{Mille et une Nuits}; il lui raconta le souper, le haschich, les statues, la réalité et le rêve, et comment à son réveil il ne restait plus comme preuve et comme souvenir de tous ces événements que ce petit yacht, faisant à l'horizon voile pour Porto-Vecchio.  

Puis il passa à Rome, à la nuit du Colisée, à la conversation qu'il avait entendue entre lui et Vampa, conversation relative à Peppino, et dans laquelle le comte avait promis d'obtenir la grâce du bandit, promesse qu'il avait si bien tenue, ainsi que nos lecteurs ont pu en juger. 

Enfin, il en arriva à l'aventure de la nuit précédente, à l'embarras où il s'était trouvé en voyant qu'il lui manquait pour compléter la somme six ou sept cents piastres; enfin à l'idée qu'il avait eue de s'adresser au comte, idée qui avait eu à la fois un résultat si pittoresque et si satisfaisant. 

Albert écoutait Franz de toutes ses oreilles. 

«Eh bien, lui dit-il quand il eut fini, où voyez-vous dans tout cela quelque chose à reprendre? Le comte est voyageur, le comte a un bâtiment à lui, parce qu'il est riche. Allez à Portsmouth ou à Southampton, vous verrez les ports encombrés de yachts appartenant à de riches Anglais qui ont la même fantaisie. Pour savoir où s'arrêter dans ses excursions, pour ne pas manger cette affreuse cuisine qui nous empoisonne, moi depuis quatre mois, vous depuis quatre ans pour ne pas coucher dans ces abominables lits où l'on ne peut dormir, il se fait meubler un pied-à-terre à Monte-Cristo: quand son pied-à-terre est meublé, il craint que le gouvernement toscan ne lui donne congé et que ses dépenses ne soient perdues, alors il achète l'île et en prend le nom. Mon cher, fouillez dans votre souvenir, et dites-moi combien de gens de votre connaissance prennent le nom des propriétés qu'ils n'ont jamais eues.  

—Mais, dit Franz à Albert, les bandits corses qui se trouvent dans son équipage? 

—Eh bien, qu'y a-t-il d'étonnant à cela? Vous savez mieux que personne, n'est-ce pas, que les bandits corses ne sont pas des voleurs, mais purement et simplement des fugitifs que quelque vendetta a exilés de leur ville ou de leur village; on peut donc les voir sans se compromettre: quant à moi, je déclare que si jamais je vais en Corse, avant de me faire présenter au gouverneur et au préfet, je me fais présenter aux bandits de Colomba, si toutefois on peut mettre la main dessus; je les trouve charmants. 

—Mais Vampa et sa troupe, reprit Franz; ceux-là sont des bandits qui arrêtent pour voler; vous ne le niez pas, je l'espère. Que dites-vous de l'influence du comte sur de pareils hommes? 

—Je dirai, mon cher, que, comme selon toute probabilité je dois la vie à cette influence, ce n'est point à moi à la critiquer de trop près. Ainsi donc, au lieu de lui en faire comme vous un crime capital, vous trouverez bon que je l'excuse, sinon de m'avoir sauvé la vie, ce qui est peut-être un peu exagéré mais du moins de m'avoir épargné quatre mille piastres, qui font bel et bien vingt-quatre mille livres de notre monnaie, somme à laquelle on ne m'aurait certes pas estimé en France; ce qui prouve, ajouta Albert en riant, que nul n'est prophète en son pays. 

—Eh bien, voilà justement; de quel pays est le comte? quelle langue parle-t-il? quels sont ses moyens d'existence? d'où lui vient son immense fortune? quelle a été cette première partie de sa vie mystérieuse et inconnue qui a répandu sur la seconde cette teinte sombre et misanthropique? Voilà, à votre place, ce que je voudrais savoir. 

—Mon cher Franz, reprit Albert, quand en recevant ma lettre vous avez vu que nous avions besoin de l'influence du comte, vous avez été lui dire: «Albert de Morcerf, mon ami, court un danger; aidez-moi à le tirer de ce danger!» n'est-ce pas? 

—Oui. 

—Alors, vous a-t-il demandé: «Qu'est-ce que M. Albert de Morcerf? d'où lui vient son nom? d'où lui vient sa fortune? quels sont ses moyens d'existence? quel est son pays? où est-il né?» Vous a-t-il demandé tout cela, dites? 

—Non, je l'avoue. 

—Il est venu, voilà tout. Il m'a tiré des mains de M. Vampa; où, malgré mes apparences pleines de désinvolture, comme vous dites, je faisais fort mauvaise figure, je l'avoue. Eh bien, mon cher, quand en échange d'un pareil service il me demande de faire pour lui ce qu'on fait tous les jours pour le premier prince russe ou italien qui passe par Paris, c'est-à-dire de le présenter dans le monde, vous voulez que je lui refuse cela! Allons donc vous êtes fou.» 

Il faut dire que, contre l'habitude, toutes les bonnes raisons étaient cette fois du côté d'Albert. 

«Enfin, reprit Franz avec un soupir, faites comme vous voudrez, mon cher vicomte; car tout ce que vous me dites là est fort spécieux, je l'avoue; mais il n'en est pas moins vrai que le comte de Monte-Cristo est un homme étrange. 

—Le comte de Monte-Cristo est un philanthrope. Il ne vous a pas dit dans quel but il venait à Paris. Eh bien, il vient pour concourir aux prix Montyon; et s'il ne lui faut que ma voix pour qu'il les obtienne, et l'influence de ce monsieur si laid qui les fait obtenir, eh bien, je lui donnerai l'une et je lui garantirai l'autre. Sur ce, mon cher Franz, ne parlons plus de cela, mettons-nous à table et allons faire une dernière visite à Saint-Pierre.» 

Il fut fait comme disait Albert, et le lendemain, à cinq heures de l'après-midi, les deux jeunes gens se quittaient, Albert de Morcerf pour revenir à Paris, Franz d'Épinay pour aller passer une quinzaine de jours à Venise. 

Mais, avant de monter en voiture, Albert remit encore au garçon de l'hôtel, tant il avait peur que son convive ne manquât au rendez-vous, une carte pour le comte de Monte-Cristo, sur laquelle au-dessous de ces mots: «Vicomte Albert de Morcerf», il y avait écrit au crayon: 

\textit{21 mai, à dix heures et demie du matin, 27, rue du Helder.} 