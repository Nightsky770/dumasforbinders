\chapter{La route de Belgique}

\lettrine{Q}{uelques} instants après la scène de confusion produite dans les salons de M. Danglars par l'apparition inattendue du brigadier de gendarmerie, et par la révélation qui en avait été la suite, le vaste hôtel s'était vidé avec une rapidité pareille à celle qu'eût amenée l'annonce d'un cas de peste ou de choléra-morbus arrivé parmi les conviés: en quelques minutes par toutes les portes, par tous les escaliers, par toutes les sorties, chacun s'était empressé de se retirer, ou plutôt de fuir; car c'était là une de ces circonstances dans lesquelles il ne faut pas même essayer de donner ces banales consolations qui rendent dans les grandes catastrophes les meilleurs amis si importuns. 

Il n'était resté dans l'hôtel du banquier que Danglars, enfermé dans son cabinet, et faisant sa déposition entre les mains de l'officier de gendarmerie; Mme Danglars, terrifiée, dans le boudoir que nous connaissons, et Eugénie qui, l'œil hautain et la lèvre dédaigneuse, s'était retirée dans sa chambre avec son inséparable compagne, Mlle Louise d'Armilly. 

Quant aux nombreux domestiques, plus nombreux encore ce soir-là que de coutume, car on leur avait adjoint, à propos de la fête, les glaciers, les cuisiniers et les maîtres d'hôtel du Café de Paris, tournant contre leurs maîtres la colère de ce qu'ils appelaient leur affront, ils stationnaient par groupes à l'office, aux cuisines, dans leurs chambres, s'inquiétant fort peu du service, qui d'ailleurs se trouvait tout naturellement interrompu. 

Au milieu de ces différents personnages, frémissant d'intérêts divers, deux seulement méritent que nous nous occupions d'eux: c'est Mlle Eugénie Danglars et Mlle Louise d'Armilly. 

La jeune fiancée, nous l'avons dit, s'était retirée l'air hautain, la lèvre dédaigneuse, et avec la démarche d'une reine outragée, suivie de sa compagne, plus pâle et plus émue qu'elle. 

En arrivant dans sa chambre, Eugénie ferma sa porte en dedans, pendant que Louise tombait sur une chaise. 

«Oh! mon Dieu, mon Dieu! l'horrible chose, dit la jeune musicienne; et qui pouvait se douter de cela? M. Andrea Cavalcanti\dots un assassin\dots un échappé du bagne\dots un forçat!» 

Un sourire ironique crispa les lèvres d'Eugénie. 

«En vérité, j'étais prédestinée, dit-elle. Je n'échappe au Morcerf que pour tomber dans le Cavalcanti! 

—Oh! ne confonds pas l'un avec l'autre, Eugénie. 

—Tais-toi, tous les hommes sont des infâmes, et je suis heureuse de pouvoir faire plus que de les détester; maintenant, je les méprise. 

—Qu'allons-nous faire? demanda Louise. 

—Ce que nous allons faire? 

—Oui. 

—Mais ce que nous devions faire dans trois jours\dots partir. 

—Ainsi, quoique tu ne te maries plus, tu veux toujours? 

—Écoute, Louise, j'ai en horreur cette vie du monde ordonnée, compassée, réglée comme notre papier de musique. Ce que j'ai toujours désiré, ambitionné, voulu, c'est la vie d'artiste, la vie libre, indépendante, où l'on ne relève que de soi, où l'on ne doit de compte qu'à soi. Rester, pour quoi faire? pour qu'on essaie, d'ici à un mois, de me marier encore; à qui? à M. Debray, peut-être, comme il en avait été un instant question. Non, Louise; non, l'aventure de ce soir me sera une excuse: je n'en cherchais pas, je n'en demandais pas; Dieu m'envoie celle-ci, elle est la bienvenue. 

—Comme tu es forte et courageuse! dit la blonde et frêle jeune fille à sa brune compagne. 

—Ne me connaissais-tu point encore? Allons, voyons, Louise, causons de toutes nos affaires. La voiture de poste\dots 

—Est achetée heureusement depuis trois jours. 

—L'as-tu fait conduire où nous devions la prendre? 

—Oui. 

—Notre passeport? 

—Le voilà!» 

Et Eugénie, avec son aplomb habituel, déplia un papier et lut: 

«M. Léon d'Armilly, âgé de vingt ans, profession d'artiste, cheveux noirs, yeux noirs, voyageant avec sa sœur.» 

«À merveille! Par qui t'es-tu procuré ce passeport? 

—En allant demander à M. de Monte-Cristo des lettres pour les directeurs des théâtres de Rome et de Naples, je lui ai exprimé mes craintes de voyager en femme; il les a parfaitement comprises, s'est mis à ma disposition pour me procurer un passeport d'homme; et, deux jours après, j'ai reçu celui-ci, auquel j'ai ajouté de ma main: \textit{Voyageant avec sa sœur.} 

—Eh bien, dit gaiement Eugénie, il ne s'agit plus que de faire nos malles: nous partirons le soir de la signature du contrat, au lieu de partir le soir des noces: voilà tout. 

—Réfléchis bien, Eugénie. 

—Oh! toutes mes réflexions sont faites; je suis lasse de n'entendre parler que de reports, de fins de mois, de hausse, de baisse, de fonds espagnols, de papier haïtien. Au lieu de cela, Louise, comprends-tu l'air, la liberté, le chant des oiseaux, les plaines de la Lombardie, les canaux de Venise, les palais de Rome, la plage de Naples. Combien possédons-nous, Louise?» 

La jeune fille qu'on interrogeait tira d'un secrétaire incrusté un petit portefeuille à serrure qu'elle ouvrit, et dans lequel elle compta vingt-trois billets de banque. 

«Vingt-trois mille francs, dit-elle. 

—Et pour autant au moins de perles, de diamants et bijoux, dit Eugénie. Nous sommes riches. Avec quarante-cinq mille francs, nous avons de quoi vivre en princesses pendant deux ans ou convenablement pendant quatre. 

«Mais avant six mois, toi avec ta musique, moi avec ma voix, nous aurons doublé notre capital. Allons, charge-toi de l'argent, moi, je me charge du coffret aux pierreries; de sorte que si l'une de nous avait le malheur de perdre son trésor, l'autre aurait toujours le sien. Maintenant, la valise: hâtons-nous, la valise! 

—Attends, dit Louise, allant écouter à la porte de Mme Danglars. 

—Que crains-tu? 

—Qu'on ne nous surprenne. 

—La porte est fermée. 

—Qu'on ne nous dise d'ouvrir. 

—Qu'on le dise si l'on veut, nous n'ouvrons pas. 

—Tu es une véritable amazone, Eugénie.» 

Et les deux jeunes filles se mirent, avec une prodigieuse activité, à entasser dans une malle tous les objets de voyage dont elles croyaient avoir besoin. 

«Là, maintenant, dit Eugénie, tandis que je vais changer de costume, ferme la valise, toi.» 

Louise appuya de toute la force de ses petites mains blanches sur le couvercle de la malle. 

«Mais je ne puis pas, dit-elle, je ne suis pas assez forte; ferme-la, toi. 

—Ah! c'est juste, dit en riant Eugénie, j'oubliais que je suis Hercule, moi, et que tu n'es, toi, que la pâle Omphale.» 

Et la jeune fille, appuyant le genou sur la malle, raidit ses bras blancs et musculeux jusqu'à ce que les deux compartiments de la valise fussent joints, et que Mlle d'Armilly eût passé le crochet du cadenas entre les deux pitons. 

Cette opération terminée, Eugénie ouvrit une commode dont elle avait la clef sur elle, et en tira une mante de voyage en soie violette ouatée. 

«Tiens, dit-elle, tu vois que j'ai pensé à tout; avec cette mante tu n'auras point froid. 

—Mais toi? 

—Oh! moi, je n'ai jamais froid, tu le sais bien; d'ailleurs avec ces habits d'homme\dots 

—Tu vas t'habiller ici? 

—Sans doute. 

—Mais auras-tu le temps? 

—N'aie donc pas la moindre inquiétude, poltronne; tous nos gens sont occupés de la grande affaire. D'ailleurs, qu'y a-t-il d'étonnant, quand on songe au désespoir dans lequel je dois être, que je me sois enfermée, dis? 

—Non, c'est vrai, tu me rassures. 

—Viens, aide-moi.» 

Et du même tiroir dont elle avait fait sortir la mante qu'elle venait de donner à Mlle d'Armilly, et dont celle-ci avait déjà couvert ses épaules, elle tira un costume d'homme complet, depuis les bottines jusqu'à la redingote, avec une provision de linge où il n'y avait rien de superflu, mais où se trouvait le nécessaire. 

Alors, avec une promptitude qui indiquait que ce n'était pas sans doute la première fois qu'en se jouant elle avait revêtu les habits d'un autre sexe, Eugénie chaussa ses bottines, passa un pantalon, chiffonna sa cravate, boutonna jusqu'à son cou un gilet montant, et endossa une redingote qui dessinait sa taille fine et cambrée. 

«Oh! c'est très bien! en vérité, c'est très bien, dit Louise en la regardant avec admiration; mais ces beaux cheveux noirs, ces nattes magnifiques qui faisaient soupirer d'envie toutes les femmes, tiendront-ils sous un chapeau d'homme comme celui que j'aperçois là? 

—Tu vas voir», dit Eugénie. 

Et saisissant avec sa main gauche la tresse épaisse sur laquelle ses longs doigts ne se refermaient qu'à peine, elle saisit de sa main droite une paire de longs ciseaux, et bientôt l'acier cria au milieu de la riche et splendide chevelure, qui tomba tout entière aux pieds de la jeune fille, renversée en arrière pour l'isoler de sa redingote. 

Puis, la natte supérieure abattue, Eugénie passa à celles de ses tempes, qu'elle abattit successivement, sans laisser échapper le moindre regret: au contraire, ses yeux brillèrent, plus pétillants et plus joyeux encore que de coutume, sous ses sourcils noirs comme l'ébène. 

«Oh! les magnifiques cheveux! dit Louise avec regret. 

—Eh! ne suis-je pas cent fois mieux ainsi? s'écria Eugénie en lissant les boucles éparses de sa coiffure devenue toute masculine, et ne me trouves-tu donc pas plus belle ainsi? 

—Oh! tu es belle, belle toujours! s'écria Louise. Maintenant, où allons-nous? 

—Mais, à Bruxelles, si tu veux; c'est la frontière la plus proche. Nous gagnerons Bruxelles, Liège, Aix-la-Chapelle; nous remonterons le Rhin jusqu'à Strasbourg, nous traverserons la Suisse et nous descendrons en Italie par le Saint-Gothard. Cela te va-t-il? 

—Mais, oui. 

—Que regardes-tu? 

—Je te regarde. En vérité, tu es adorable ainsi; on dirait que tu m'enlèves. 

—Eh pardieu! on aurait raison. 

—Oh! je crois que tu as juré, Eugénie?» 

Et les deux jeunes filles, que chacun eût pu croire plongées dans les larmes, l'une pour son propre compte, l'autre par dévouement à son amie, éclatèrent de rire, tout en faisant disparaître les traces les plus visibles du désordre qui naturellement avait accompagné les apprêts de leur évasion. 

Puis, ayant soufflé leurs lumières, l'œil interrogateur, l'oreille au guet, le cou tendu, les deux fugitives ouvrirent la porte d'un cabinet de toilette qui donnait sur un escalier de service descendant jusqu'à la cour, Eugénie marchant la première, et soutenant d'un bras la valise que, par l'anse opposée, Mlle d'Armilly soulevait à peine de ses deux mains. 

La cour était vide. Minuit sonnait. 

Le concierge veillait encore. 

Eugénie s'approcha tout doucement et vit le digne suisse qui dormait au fond de sa loge, étendu dans son fauteuil. 

Elle retourna vers Louise, reprit la malle qu'elle avait un instant posée à terre, et toutes deux, suivant l'ombre projetée par la muraille, gagnèrent la voûte. 

Eugénie fit cacher Louise dans l'angle de la porte, de manière que le concierge, s'il lui plaisait par hasard de se réveiller, ne vît qu'une personne. 

Puis, s'offrant elle-même au plein rayonnement de la lampe qui éclairait la cour: 

«La porte!» cria-t-elle de sa plus belle voix de contralto, en frappant à la vitre. 

Le concierge se leva comme l'avait prévu Eugénie, et fit même quelques pas pour reconnaître la personne qui sortait; mais voyant un jeune homme qui fouettait impatiemment son pantalon de sa badine, il ouvrit sur-le-champ. 

Aussitôt Louise se glissa comme une couleuvre par la porte entrebâillée, et bondit légèrement dehors. Eugénie, calme en apparence, quoique, selon toute probabilité, son cœur comptât plus de pulsations que dans l'état habituel, sortit à son tour. 

Un commissionnaire passait, on le chargea de la malle, puis les deux jeunes filles lui ayant indiqué comme le but de leur course la rue de la Victoire et le numéro 36 de cette rue, elles marchèrent derrière cet homme, dont la présence rassurait Louise; quant à Eugénie, elle était forte comme une Judith ou une Dalila. 

On arriva au numéro indiqué. Eugénie ordonna au commissionnaire de déposer la malle, lui donna quelques pièces de monnaie, et, après avoir frappé au volet, le renvoya. 

Ce volet auquel avait frappé Eugénie était celui d'une petite lingère prévenue à l'avance: elle n'était point encore couchée, elle ouvrit. 

«Mademoiselle, dit Eugénie, faites tirer par le concierge la calèche de la remise, et envoyez-le chercher des chevaux à l'hôtel des Postes. Voici cinq francs pour la peine que nous lui donnons. 

—En vérité, dit Louise, je t'admire, et je dirai presque que je te respecte.» 

La lingère regardait avec étonnement; mais comme il était convenu qu'il y aurait vingt louis pour elle, elle ne fit pas la moindre observation. 

Un quart d'heure après, le concierge revenait ramenant le postillon et les chevaux, qui, en un tour de main, furent attelés à la voiture, sur laquelle le concierge assura la malle à l'aide d'une corde et d'un tourniquet. 

«Voici le passeport, dit le postillon; quelle route prenons-nous, notre jeune bourgeois? 

—La route de Fontainebleau, répondit Eugénie avec une voix presque masculine. 

—Eh bien, que dis-tu donc? demanda Louise. 

—Je donne le change, dit Eugénie; cette femme à qui nous donnons vingt louis peut nous trahir pour quarante: sur le boulevard nous prendrons une autre direction.» 

Et la jeune fille s'élança dans le briska établi en excellente dormeuse, sans presque toucher le marchepied. 

«Tu as toujours raison, Eugénie», dit la maîtresse de chant en prenant place près de son amie. 

Un quart d'heure après, le postillon, remis dans le droit chemin, franchissait, en faisant claquer son fouet, la grille de la barrière Saint-Martin. 

«Ah! dit Louise en respirant, nous voilà donc sorties de Paris! 

—Oui, ma chère, et le rapt est bel et bien consommé, répondit Eugénie. 

—Oui, mais sans violence, dit Louise. 

—Je ferai valoir cela comme circonstance atténuante», répondit Eugénie. 

Ces paroles se perdirent dans le bruit que faisait la voiture en roulant sur le pavé de la Villette. 

M. Danglars n'avait plus sa fille. 