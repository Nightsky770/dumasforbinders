\chapter{Le pardon}

\lettrine{L}{e} lendemain Danglars eut encore faim, l'air de cette caverne était on ne peut plus apéritif; le prisonnier crut que, pour ce jour-là, il n'aurait aucune dépense à faire: en homme économe il avait caché la moitié de son poulet et un morceau de son pain dans le coin de sa cellule. 

Mais il n'eut pas plus tôt mangé qu'il eut soif: il n'avait pas compté là-dessus. 

Il lutta contre la soif jusqu'au moment où il sentit sa langue desséchée s'attacher à son palais. 

Alors, ne pouvant plus résister au feu qui le dévorait, il appela. 

La sentinelle ouvrit la porte; c'était un nouveau visage. 

Il pensa que mieux valait pour lui avoir affaire à une ancienne connaissance. Il appela Peppino. 

«Me voici, Excellence, dit le bandit en se présentant avec un empressement qui parut de bon augure à Danglars, que désirez-vous? 

—À boire, dit le prisonnier. 

—Excellence, dit Peppino, vous savez que le vin est hors de prix dans les environs de Rome\dots 

—Donnez-moi de l'eau alors, dit Danglars cherchant à parer la botte. 

—Oh! Excellence, l'eau est plus rare que le vin; il fait une si grande sécheresse! 

—Allons, dit Danglars, nous allons recommencer, à ce qu'il paraît!» 

Et, tout en souriant pour avoir l'air de plaisanter, le malheureux sentait la sueur mouiller ses tempes. 

«Voyons, mon ami, dit Danglars, voyant que Peppino demeurait impassible, je vous demande un verre de vin; me le refuserez-vous? 

—Je vous ai déjà dit, Excellence, répondit gravement Peppino, que nous ne vendions pas au détail. 

—Eh bien, voyons alors, donnez-moi une bouteille. 

—Duquel? 

—Du moins cher. 

—Ils sont tous deux du même prix. 

—Et quel prix? 

—Vingt-cinq mille francs la bouteille. 

—Dites, s'écria Danglars avec une amertume qu'Harpargon seul eût pu noter dans le diapason de la voix humaine, dites que vous voulez me dépouiller, ce sera plus tôt fait que de me dévorer ainsi lambeau par lambeau. 

—Il est possible, dit Peppino, que ce soit là le projet du maître. 

—Le maître, qui est-il donc? 

—Celui auquel on vous a conduit avant-hier. 

—Et où est-il? 

—Ici. 

—Faites que je le voie. 

—C'est facile.» 

L'instant d'après, Luigi Vampa était devant Danglars. 

«Vous m'appelez? demanda-t-il au prisonnier. 

—C'est vous, monsieur, qui êtes le chef des personnes qui m'ont amené ici? 

—Oui Excellence. 

—Que désirez-vous de moi pour rançon? Parlez. 

—Mais tout simplement les cinq millions que vous portez sur vous.» 

Danglars sentit un effroyable spasme lui broyer le cœur. 

«Je n'ai que cela au monde, monsieur, et c'est le reste d'une immense fortune: si vous me l'ôtez, ôtez-moi la vie. 

—Il nous est défendu de verser votre sang, Excellence. 

—Et par qui cela vous est-il défendu? 

—Par celui auquel nous obéissons. 

—Vous obéissez donc à quelqu'un? 

—Oui, à un chef. 

—Je croyais que vous-même étiez le chef? 

—Je suis le chef de ces hommes; mais un autre homme est mon chef à moi. 

—Et ce chef obéit-il à quelqu'un? 

—Oui. 

—À qui? 

—À Dieu.» 

Danglars resta un instant pensif. 

«Je ne vous comprends pas, dit-il. 

—C'est possible. 

—Et c'est ce chef qui vous a dit de me traiter ainsi? 

—Oui. 

—Quel est son but? 

—Je n'en sais rien. 

—Mais ma bourse s'épuisera. 

—C'est probable. 

—Voyons, dit Danglars, voulez-vous un million? 

—Non. 

—Deux millions? 

—Non. 

—Trois millions?\dots quatre?\dots Voyons, quatre? je vous les donne à la condition que vous me laisserez aller. 

—Pourquoi nous offrez-vous quatre millions de ce qui en vaut cinq? dit Vampa; c'est de l'usure cela, seigneur banquier, ou je ne m'y connais pas. 

—Prenez tout! prenez tout, vous dis-je! s'écria Danglars, et tuez-moi! 

—Allons, allons, calmez-vous, Excellence, vous allez vous fouetter le sang, ce qui vous donnera un appétit à manger un million par jour; soyez donc plus économe, morbleu! 

—Mais quand je n'aurai plus d'argent pour vous payer! s'écria Danglars exaspéré. 

—Alors vous aurez faim. 

—J'aurai faim? dit Danglars blêmissant. 

—C'est probable, répondit flegmatiquement Vampa. 

—Mais vous dites que vous ne voulez pas me tuer? 

—Non. 

—Et vous voulez me laisser mourir de faim? 

—Ce n'est pas la même chose. 

—Eh bien, misérables! s'écria Danglars, je déjouerai vos infâmes calculs; mourir pour mourir, j'aime autant en finir tout de suite; faites-moi souffrir, torturez-moi, tuez-moi, mais vous n'aurez plus ma signature! 

—Comme il vous plaira, Excellence», dit Vampa. 

Et il sortit de la cellule. 

Danglars se jeta en rugissant sur ses peaux de bouc. 

Quels étaient ces hommes? quel était ce chef invisible? quels projets poursuivaient-ils donc sur lui? et quand tout le monde pouvait se racheter, pourquoi lui seul ne le pouvait-il pas? 

Oh! certes, la mort, une mort prompte et violente, était un bon moyen de tromper ses ennemis acharnés, qui semblaient poursuivre sur lui une incompréhensible vengeance. 

Oui, mais mourir! 

Pour la première fois peut-être de sa carrière si longue, Danglars songeait à la mort avec le désir et la crainte tout à la fois de mourir; mais le moment était venu pour lui d'arrêter sa vue sur le spectre implacable qui vit au-dedans de toute créature, qui, à chaque pulsation du cœur, dit à lui-même: Tu mourras! 

Danglars ressemblait à ces bêtes fauves que la chasse anime, puis qu'elle désespère, et qui, à force de désespoir, réussissent parfois à se sauver. 

Danglars songea à une évasion. 

Mais les murs étaient le roc lui-même; mais à la seule issue qui conduisait hors de la cellule un homme lisait, et derrière cet homme on voyait passer et repasser des ombres armées de fusils. 

Sa résolution de ne pas signer dura deux jours, après quoi il demanda des aliments et offrit un million. 

On lui servit un magnifique souper, et on prit son million. 

Dès lors, la vie du malheureux prisonnier fut une divagation perpétuelle. Il avait tant souffert qu'il ne voulait plus s'exposer à souffrir, et subissait toutes les exigences; au bout de douze jours, un après-midi qu'il avait dîné comme en ses beaux jours de fortune, il fit ses comptes et s'aperçut qu'il avait tant donné de traites au porteur, qu'il ne lui restait plus que cinquante mille francs. 

Alors il se fit en lui une réaction étrange: lui qui venait d'abandonner cinq millions, il essaya de sauver les cinquante mille francs qui lui restaient, plutôt que de donner ces cinquante mille francs, il se résolut de reprendre une vie de privations, il eut des lueurs d'espoir qui touchaient à la folie; lui qui depuis si longtemps avait oublié Dieu, il y songea pour se dire que Dieu parfois avait fait des miracles: que la caverne pouvait s'abîmer; que les carabiniers pontificaux pouvaient découvrir cette retraite maudite et venir à son secours; qu'alors il lui resterait cinquante mille francs; que cinquante mille francs étaient une somme suffisante pour empêcher un homme de mourir de faim; il pria Dieu de lui conserver ces cinquante mille francs, et en priant il pleura. 

Trois jours se passèrent ainsi, pendant lesquels le nom de Dieu fut constamment, sinon dans son cœur du moins sur ses lèvres; par intervalles il avait des instants de délire pendant lesquels il croyait, à travers les fenêtres, voir dans une pauvre chambre un vieillard agonisant sur un grabat. 

Ce vieillard, lui aussi, mourait de faim. 

Le quatrième jour, ce n'était plus un homme, c'était un cadavre vivant; il avait ramassé à terre jusqu'aux dernières miettes de ses anciens repas et commencé à dévorer la natte dont le sol était couvert. 

Alors il supplia Peppino, comme on supplie son ange gardien, de lui donner quelque nourriture, il lui offrit mille francs d'une bouchée de pain. 

Peppino ne répondit pas. 

Le cinquième jour, il se traîna à l'entrée de la cellule. 

«Mais vous n'êtes donc pas un chrétien? dit-il en se redressant sur les genoux; vous voulez assassiner un homme qui est votre frère devant Dieu? 

«Oh! mes amis d'autrefois, mes amis d'autrefois!» murmura-t-il. 

Et il tomba la face contre terre. 

Puis, se relevant avec une espèce de désespoir: 

«Le chef! cria-t-il, le chef! 

—Me voilà! dit Vampa, paraissant tout à coup, que désirez-vous encore? 

—Prenez mon dernier or, balbutia Danglars en tendant son portefeuille, et laissez-moi vivre ici, dans cette caverne; je ne demande plus la liberté, je ne demande qu'à vivre. 

—Vous souffrez donc bien? demanda Vampa. 

—Oh! oui, je souffre, et cruellement! 

—Il y a cependant des hommes qui ont encore plus souffert que vous. 

—Je ne crois pas. 

—Si fait! ceux qui sont morts de faim.» 

Danglars songea à ce vieillard que, pendant ses heures d'hallucination, il voyait, à travers les fenêtres de sa pauvre chambre, gémir sur son lit. 

Il frappa du front la terre en poussant un gémissement. 

«Oui, c'est vrai, il y en a qui ont plus souffert encore que moi, mais au moins, ceux-là, c'étaient des martyrs. 

—Vous repentez-vous, au moins?» dit une voix sombre et solennelle, qui fit dresser les cheveux sur la tête de Danglars. 

Son regard affaibli essaya de distinguer les objets, et il vit derrière le bandit un homme enveloppé d'un manteau et perdu dans l'ombre d'un pilastre de pierre. 

«De quoi faut-il que je me repente? balbutia Danglars. 

—Du mal que vous avez fait, dit la même voix. 

—Oh! oui, je me repens! je me repens!» s'écria Danglars. 

Et il frappa sa poitrine de son poing amaigri. 

«Alors je vous pardonne, dit l'homme en jetant son manteau et en faisant un pas pour se placer dans la lumière. 

—Le comte de Monte-Cristo! dit Danglars, plus pâle de terreur qu'il ne l'était, un instant auparavant, de faim et de misère. 

—Vous vous trompez; je ne suis pas le comte de Monte-Cristo. 

—Et qui êtes-vous donc? 

—Je suis celui que vous avez vendu, livré, déshonoré: je suis celui dont vous avez prostitué la fiancée; je suis celui sur lequel vous avez marché pour vous hausser jusqu'à la fortune; je suis celui dont vous avez fait mourir le père de faim, qui vous avait condamné à mourir de faim, et qui cependant vous pardonne, parce qu'il a besoin lui-même d'être pardonné: je suis Edmond Dantès!» 

Danglars ne poussa qu'un cri, et tomba prosterné. 

«Relevez-vous, dit le comte, vous avez la vie sauve; pareille fortune n'est pas arrivée à vos deux autres complices: l'un est fou, l'autre est mort! Gardez les cinquante mille francs qui vous restent, je vous en fais don; quant à vos cinq millions volés aux hospices, ils leur sont déjà restitués par une main inconnue. 

«Et maintenant, mangez et buvez; ce soir je vous fais mon hôte. 

«Vampa, quand cet homme sera rassasié, il sera libre.» 

Danglars demeura prosterné tandis que le comte s'éloignait; lorsqu'il releva la tête, il ne vit plus qu'une espèce d'ombre qui disparaissait dans le corridor, et devant laquelle s'inclinaient les bandits. 

Comme l'avait ordonné le comte, Danglars fut servi par Vampa, qui lui fit apporter le meilleur vin et les plus beaux fruits de l'Italie, et qui, l'ayant fait monter dans sa chaise de poste, l'abandonna sur la route, adossé à un arbre. 

Il y resta jusqu'au jour, ignorant où il était. 

Au jour il s'aperçut qu'il était près d'un ruisseau: il avait soif, il se traîna jusqu'à lui. 

En se baissant pour y boire, il s'aperçut que ses cheveux étaient devenus blancs. 