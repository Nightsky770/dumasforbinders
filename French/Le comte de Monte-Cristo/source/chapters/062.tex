\chapter{Les fantômes} 

\lettrine{\accentletter[\gravebox]{A}}{} la première vue, et examinée du dehors, la maison d'Auteuil n'avait rien de splendide, rien de ce qu'on pouvait attendre d'une habitation destinée au magnifique comte de Monte-Cristo: mais cette simplicité tenait à la volonté du maître, qui avait positivement ordonné que rien ne fût changé à l'extérieur; il n'était besoin pour s'en convaincre que de considérer l'intérieur. En effet, à peine la porte était-elle ouverte que le spectacle changeait. 

M. Bertuccio s'était surpassé lui-même pour le goût des ameublements et la rapidité de l'exécution: comme autrefois le duc d'Antin avait fait abattre en une nuit une allée d'arbres qui gênait le regard de Louis XIV, de même en trois jours M. Bertuccio avait fait planter une cour entièrement nue, et de beaux peupliers, des sycomores venus avec leurs blocs énormes de racines, ombrageaient la façade principale de la maison, devant laquelle, au lieu de pavés à moitié cachés par l'herbe, s'étendait une pelouse de gazon, dont les plaques avaient été posées le matin même et qui formait un vaste tapis où perlait encore l'eau dont on l'avait arrosé. 

Au reste, les ordres venaient du comte; lui-même avait remis à Bertuccio un plan où étaient indiqués le nombre et la place des arbres qui devaient être plantés, la forme et l'espace de la pelouse qui devait succéder aux pavés. 

Vue ainsi, la maison était devenue méconnaissable, et Bertuccio lui-même protestait qu'il ne la reconnaissait plus, emboîtée qu'elle était dans son cadre de verdure. 

L'intendant n'eût pas été fâché, tandis qu'il y était, de faire subir quelques transformations au jardin; mais le comte avait positivement défendu qu'on y touchât en rien. Bertuccio s'en dédommagea en encombrant de fleurs les antichambres, les escaliers et les cheminées. 

Ce qui annonçait l'extrême habileté de l'intendant et la profonde science du maître, l'un pour servir, l'autre pour se faire servir, c'est que cette maison, déserte depuis vingt années, si sombre et si triste encore la veille, tout imprégnée qu'elle était de cette fade odeur qu'on pourrait appeler l'odeur du temps, avait pris en un jour, avec l'aspect de la vie, les parfums que préférait le maître, et jusqu'au degré de son jour favori; c'est que le comte, en arrivant, avait là, sous sa main, ses livres et ses armes; sous ses yeux ses tableaux préférés; dans les antichambres les chiens dont il aimait les caresses, les oiseaux dont il aimait le chant; c'est que toute cette maison, réveillée de son long sommeil, comme le palais de la Belle au bois dormant, vivait, chantait, s'épanouissait, pareille à ces maisons que nous avons depuis longtemps chéries, et dans lesquelles, lorsque par malheur nous les quittons, nous laissons involontairement une partie de notre âme. 

Des domestiques allaient et venaient joyeux dans cette belle cour: les uns possesseurs des cuisines, et glissant comme s'ils eussent toujours habité cette maison dans des escaliers restaurés de la veille, les autres peuplant les remises, où les équipages, numérotés et casés, semblaient installés depuis cinquante ans; et les écuries, où les chevaux au râtelier répondaient en hennissant aux palefreniers, qui leur parlaient avec infiniment plus de respect que beaucoup de domestiques ne parlent à leurs maîtres. 

La bibliothèque était disposée sur deux corps, aux deux côtés de la muraille, et contenait deux mille volumes à peu près; tout un compartiment était destiné aux romans modernes, et celui qui avait paru la veille était déjà rangé à sa place, se pavanant dans sa reliure rouge et or.  

De l'autre côté de la maison, faisant pendant à la bibliothèque, il y avait la serre, garnie de plantes rares et s'épanouissant dans de larges potiches japonaises, et au milieu de la serre, merveille à la fois des yeux et de l'odorat, un billard que l'on eût dit abandonné depuis une heure au plus par les joueurs, qui avaient laissé mourir les billes sur le tapis. 

Une seule chambre avait été respectée par le magnifique Bertuccio. Devant cette chambre, située à l'angle gauche du premier étage, à laquelle on pouvait monter par le grand escalier, et dont on pouvait sortir par l'escalier dérobé, les domestiques passaient avec curiosité et Bertuccio avec terreur. 

À cinq heures précises, le comte arriva, suivi d'Ali, devant la maison d'Auteuil. Bertuccio attendait cette arrivée avec une impatience mêlée d'inquiétude; il espérait quelques compliments, tout en redoutant un froncement de sourcils. 

Monte-Cristo descendit dans la cour, parcourut toute la maison et fit le tour du jardin, silencieux et sans donner le moindre signe d'approbation ni de mécontentement. 

Seulement, en entrant dans sa chambre à coucher, située du côté opposé à la chambre fermée, il étendit la main vers le tiroir d'un petit meuble en bois de rose, qu'il avait déjà distingué à son premier voyage. 

«Cela ne peut servir qu'à mettre des gants, dit-il. 

—En effet, Excellence, répondit Bertuccio ravi, ouvrez, et vous y trouverez des gants.»  

Dans les autres meubles, le comte trouva encore ce qu'il comptait y trouver, flacons, cigares, bijoux. 

«Bien!» dit-il encore. 

Et M. Bertuccio se retira l'âme ravie, tant était grande, puissante et réelle l'influence de cet homme sur tout ce qui l'entourait. 

À six heures précises, on entendit piétiner un cheval devant la porte d'entrée. C'était notre capitaine des spahis qui arrivait sur \textit{Médéah}. 

Monte-Cristo l'attendait sur le perron, le sourire aux lèvres. 

«Me voilà le premier, j'en suis bien sûr! lui cria Morrel: je l'ai fait exprès pour vous avoir un instant à moi seul avant tout le monde. Julie et Emmanuel vous disent des millions de choses. Ah! mais, savez-vous que c'est magnifique ici! Dites-moi, comte, est-ce que vos gens auront bien soin de mon cheval? 

—Soyez tranquille, mon cher Maximilien, ils s'y connaissent. 

—C'est qu'il a besoin d'être bouchonné. Si vous saviez de quel train il a été! Une véritable trombe! 

—Peste, je le crois bien, un cheval de cinq mille francs! dit Monte-Cristo du ton qu'un père mettrait à parler à son fils. 

—Vous les regrettez? dit Morrel avec son franc sourire. 

—Moi! Dieu m'en préserve! répondit le comte. Non. Je regretterais seulement que le cheval ne fût pas bon. 

—Il est si bon, mon cher comte, que M. de Château-Renaud, l'homme le plus connaisseur de France, et M. Debray, qui monte les arabes du ministère, courent après moi en ce moment, et sont un peu distancés, comme vous voyez, et encore sont-ils talonnés par les chevaux de la baronne Danglars, qui vont d'un trot à faire tout bonnement leurs six lieues à l'heure. 

—Alors, ils vous suivent? demanda Monte-Cristo. 

—Tenez, les voilà.» 

En effet, au moment même, un coupé à l'attelage tout fumant et deux chevaux de selle hors d'haleine arrivèrent devant la grille de la maison, qui s'ouvrit devant eux. Aussitôt le coupé décrivit son cercle, et vint s'arrêter au perron, suivi de deux cavaliers. 

En un instant Debray eut mis pied à terre, et se trouva à la portière. Il offrit sa main à la baronne, qui lui fit en descendant un geste imperceptible pour tout autre que pour Monte-Cristo. Mais le comte ne perdait rien, et dans ce geste il vit reluire un petit billet blanc aussi imperceptible que le geste, et qui passa, avec une aisance qui indiquait l'habitude de cette manœuvre, de la main de Mme Danglars dans celle du secrétaire du ministre. 

Derrière sa femme descendit le banquier, pâle comme s'il fût sorti du sépulcre au lieu de sortir de son coupé. 

Mme Danglars jeta autour d'elle un regard rapide et investigateur que Monte-Cristo seul put comprendre et dans lequel elle embrassa la cour, le péristyle, la façade de la maison; puis, réprimant une légère émotion, qui se fût certes traduite sur son visage, s'il eût été permis à son visage de pâlir, elle monta le perron tout en disant à Morrel: 

«Monsieur, si vous étiez de mes amis, je vous demanderais si votre cheval est à vendre.» 

Morrel fit un sourire qui ressemblait fort à une grimace, et se retourna vers Monte-Cristo, comme pour le prier de le tirer de l'embarras où il se trouvait. 

Le comte le comprit. 

«Ah! madame, répondit-il, pourquoi n'est-ce point à moi que cette demande s'adresse? 

—Avec vous, monsieur, dit la baronne, on n'a le droit de ne rien désirer, car on est trop sûre d'obtenir. Aussi était-ce à M. Morrel. 

—Malheureusement, reprit le comte, je suis témoin que M. Morrel ne peut céder son cheval, son honneur étant engagé à ce qu'il le garde. 

—Comment cela? 

—Il a parié dompter \textit{Médéah} dans l'espace de six mois. Vous comprenez maintenant, baronne, que s'il s'en défaisait avant le terme fixé par le pari, non seulement il le perdrait, mais encore on dirait qu'il a eu peur; et un capitaine de spahis, même pour passer un caprice à une jolie femme, ce qui est, à mon avis, une des choses les plus sacrées de ce monde, ne peut laisser courir un pareil bruit. 

—Vous voyez, madame\dots dit Morrel tout en adressant à Monte-Cristo un sourire reconnaissant. 

—Il me semble d'ailleurs, dit Danglars avec un ton bourru mal déguisé par son sourire épais, que vous en avez assez comme cela de chevaux.» 

Ce n'était pas l'habitude de Mme Danglars de laisser passer de pareilles attaques sans y riposter, et cependant, au grand étonnement des jeunes gens, elle fit semblant de ne pas entendre et ne répondit rien. 

Monte-Cristo souriait à ce silence, qui dénonçait une humilité inaccoutumée, tout en montrant à la baronne deux immenses pots de porcelaine de Chine, sur lesquels serpentaient des végétations marines d'une grosseur et d'un travail tels, que la nature seule peut avoir cette richesse, cette sève et cet esprit. 

La baronne était émerveillée. 

«Eh! mais, on planterait là-dedans un marronnier des Tuileries! dit-elle; comment donc a-t-on jamais pu faire cuire de pareilles énormités? 

—Ah! madame, dit Monte-Cristo, il ne faut pas nous demander cela à nous autre faiseurs de statuettes et de verre mousseline; c'est un travail d'un autre âge, une espèce d'œuvre des génies de la terre et de la mer. 

—Comment cela et de quelle époque cela peut-il être?  

—Je ne sais pas; seulement j'ai ouï dire qu'un empereur de la Chine avait fait construire un four exprès; que dans ce four, les uns après les autres, on avait fait cuire douze pots pareils à ceux-ci. Deux se brisèrent sous l'ardeur du feu; on descendit les dix autres à trois cents brasses au fond de la mer. La mer, qui savait ce que l'on demandait d'elle, jeta sur eux ses lianes, tordit ses coraux, incrusta ses coquilles; le tout fut cimenté par deux cents années sous ses profondeurs inouïes, car une révolution emporta l'empereur qui avait voulu faire cet essai et ne laissa que le procès-verbal qui constatait la cuisson des vases et leur descente au fond de la mer. Au bout de deux cents ans on retrouva le procès-verbal, et l'on songea à retirer les vases. Des plongeurs allèrent, sous des machines faites exprès, à la découverte dans la baie où on les avait jetés; mais sur les dix on n'en retrouva plus que trois, les autres avaient été dispersés et brisés par les flots. J'aime ces vases, au fond desquels, je me figure parfois que des monstres informes, effrayants, mystérieux, et pareils à ceux que voient les seuls plongeurs, ont fixé avec étonnement leur regard terne et froid, et dans lesquels ont dormi des myriades de poissons qui s'y réfugiaient pour fuir la poursuite de leurs ennemis.» 

Pendant ce temps, Danglars, peu amateur de curiosités, arrachait machinalement, et l'une après l'autre, les fleurs d'un magnifique oranger; quand il eut fini avec l'oranger, il s'adressa à un cactus, mais alors le cactus, d'un caractère moins facile que l'oranger, le piqua outrageusement. 

Alors il tressaillit et se frotta les yeux comme s'il sortait d'un songe. 

«Monsieur, lui dit Monte-Cristo en souriant, vous qui êtes amateur de tableaux et qui avez de si magnifiques choses, je ne vous recommande pas les miens. Cependant voici deux Hobbema, un Paul Potter, un Mieris, deux Gérard Dow, un Raphaël, un Van Dyck, un Zurbaran et deux ou trois Murillo, qui sont dignes de vous être présentés. 

—Tiens! dit Debray, voici un Hobbema que je reconnais. 

—Ah! vraiment! 

—Oui, on est venu le proposer au Musée. 

—Qui n'en a pas, je crois? hasarda Monte-Cristo. 

—Non, et qui cependant a refusé de l'acheter. 

—Pourquoi cela? demanda Château-Renaud. 

—Vous êtes charmant, vous; parce que le gouvernement n'est point assez riche. 

—Ah! pardon! dit Château-Renaud. J'entends dire cependant de ces choses-là tous les jours depuis huit ans, et je ne puis pas encore m'y habituer. 

—Cela viendra, dit Debray. 

—Je ne crois pas, répondit Château-Renaud. 

—M. le major Bartolomeo Cavalcanti! M. le vicomte Andrea Cavalcanti!» annonça Baptistin.  

Un col de satin noir sortant des mains du fabricant, une barbe fraîche, des moustaches grises, l'œil assuré, un habit de major orné de trois plaques et de cinq croix, en somme, une tenue irréprochable de vieux soldat, tel apparut le major Bartolomeo Cavalcanti, ce tendre père que nous connaissons. 

Près de lui, couvert d'habits tout flambant neufs, s'avançait, le sourire sur les lèvres, le vicomte Andrea Cavalcanti, ce respectueux fils que nous connaissons encore. 

Les trois jeunes gens causaient ensemble; leurs regards se portaient du père au fils, et s'arrêtèrent tout naturellement plus longtemps sur ce dernier, qu'ils détaillèrent. 

«Cavalcanti! dit Debray. 

—Un beau nom, fit Morrel, peste! 

—Oui, dit Château-Renaud, c'est vrai, ces Italiens se nomment bien, mais ils s'habillent mal. 

—Vous êtes difficile, Château-Renaud, reprit Debray; ces habits sont d'un excellent faiseur, et tout neufs. 

—Voilà justement ce que je leur reproche. Ce monsieur a l'air de s'habiller aujourd'hui pour la première fois. 

—Qu'est-ce que ces messieurs? demanda Danglars au comte de Monte-Cristo. 

—Vous avez entendu, des Cavalcanti.  

—Cela m'apprend leur nom, voilà tout. 

—Ah! c'est vrai, vous n'êtes pas au courant de nos noblesses d'Italie, qui dit Cavalcanti, dit race de princes. 

—Belle fortune? demanda le banquier. 

—Fabuleuse. 

—Que font-ils? 

—Ils essaient de la manger sans pouvoir en venir à bout. Ils ont d'ailleurs des crédits sur vous, à ce qu'ils m'ont dit en me venant voir avant-hier. Je les ai même invités à votre intention. Je vous les présenterai. 

—Mais il me semble qu'ils parlent très purement le français, dit Danglars. 

—Le fils a été élevé dans un collège du Midi, à Marseille ou dans les environs, je crois. Vous le trouverez dans l'enthousiasme. 

—De quoi? demanda la baronne. 

—Des Françaises, madame. Il veut absolument prendre femme à Paris. 

—Une belle idée qu'il a là!» dit Danglars en haussant les épaules.  

Mme Danglars regarda son mari avec une expression qui, dans tout autre moment, eût présagé un orage, mais pour la seconde fois elle se tut. 

«Le baron paraît bien sombre aujourd'hui, dit Monte-Cristo à Mme Danglars; est-ce qu'on voudrait le faire ministre, par hasard? 

—Non, pas encore, que je sache. Je crois plutôt qu'il aura joué à la Bourse, qu'il aura perdu, et qu'il ne sait à qui s'en prendre. 

—M. et Mme de Villefort!» cria Baptistin. 

Les deux personnes annoncées entrèrent. M. de Villefort, malgré sa puissance sur lui-même, était visiblement ému. En touchant sa main, Monte-Cristo sentit qu'elle tremblait. 

«Décidément, il n'y a que les femmes pour savoir dissimuler», se dit Monte-Cristo à lui-même et en regardant Mme Danglars, qui souriait au procureur du roi et qui embrassait sa femme. 

Après les premiers compliments, le comte vit Bertuccio qui, occupé jusque-là du côté de l'office, se glissait dans un petit salon attenant à celui dans lequel on se trouvait. Il alla à lui. 

«Que voulez-vous, monsieur Bertuccio? lui dit-il. 

—Son Excellence ne m'a pas dit le nombre de ses convives. 

—Ah! c'est vrai. 

—Combien de couverts? 

—Comptez vous-même. 

—Tout le monde est-il arrivé, Excellence? 

—Oui.» 

Bertuccio glissa son regard à travers la porte entrebâillée. Monte-Cristo le couvait des yeux. 

«Ah! mon Dieu! s'écria-t-il. 

—Quoi donc? demanda le comte. 

—Cette femme!\dots cette femme!\dots 

—Laquelle? 

—Celle qui a une robe blanche et tant de diamants!\dots la blonde!\dots 

—Mme Danglars? 

—Je ne sais pas comment on la nomme. Mais c'est elle, monsieur, c'est elle! 

—Qui, elle? 

—La femme du jardin! celle qui était enceinte! celle qui se promenait en attendant!\dots en attendant!\dots» 

Bertuccio demeura la bouche ouverte, pâle et les cheveux hérissés. 

«En attendant qui?» 

Bertuccio, sans répondre, montra Villefort du doigt, à peu près du même geste dont Macbeth montra Banco. 

«Oh!\dots oh!\dots murmura-t-il enfin, voyez-vous? 

—Quoi? qui? 

—Lui!\dots M. le procureur du roi de Villefort? Sans doute, que je vois. 

—Mais je ne l'ai donc pas tué? 

—Ah çà! mais je crois que vous devenez fou, mon brave Bertuccio, dit le comte. 

—Mais il n'est donc pas mort? 

—Eh non! il n'est pas mort, vous le voyez bien; au lieu de le frapper entre la sixième et la septième côte gauche, comme c'est la coutume de vos compatriotes, vous aurez frappé plus haut ou plus bas; et ces gens de justice, ça vous a l'âme chevillée dans le corps; ou bien plutôt rien de ce que vous m'avez raconté n'est vrai, c'est un rêve de votre imagination, une hallucination de votre esprit; vous vous serez endormi ayant mal digéré votre vengeance; elle vous aura pesé sur l'estomac; vous aurez eu le cauchemar, voilà tout. Voyons, rappelez votre calme, et comptez: M. et Mme de Villefort, deux; M. et Mme Danglars, quatre; M. de Château-Renaud, M. Debray, M. Morrel, sept; M. le major Bartolomeo Cavalcanti, huit. 

—Huit! répéta Bertuccio. 

—Attendez donc! attendez donc! vous êtes bien pressé de vous en aller, que diable! vous oubliez un de mes convives. Appuyez un peu sur la gauche\dots tenez\dots M. Andrea Cavalcanti, ce jeune homme en habit noir qui regarde la Vierge de Murillo, qui se retourne.» 

Cette fois Bertuccio commença un cri que le regard de Monte-Cristo éteignit sur ses lèvres. 

«Benedetto! murmura-t-il tout bas, fatalité! 

—Voilà six heures et demie qui sonnent, monsieur Bertuccio, dit sévèrement le comte; c'est l'heure où j'ai donné l'ordre qu'on se mît à table; vous savez que je n'aime point à attendre.» 

Et Monte-Cristo entra dans le salon où l'attendaient ses convives, tandis que Bertuccio regagnait la salle à manger en s'appuyant contre les murailles. 

Cinq minutes après, les deux portes du salon s'ouvrirent. Bertuccio parut, et faisant, comme Vatel à Chantilly, un dernier et héroïque effort: 

«Monsieur le comte est servi», dit-il. 

Monte-Cristo offrit le bras à Mme de Villefort. 

«Monsieur de Villefort, dit-il, faites-vous le cavalier de Mme la baronne Danglars, je vous prie.» 

Villefort obéit, et l'on passa dans la salle à manger. 