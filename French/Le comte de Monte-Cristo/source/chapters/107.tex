\chapter{La Fosse-aux-Lions} 

\lettrine{L}{'un} des quartiers de la Force, celui qui renferme les détenus les plus compromis et les plus dangereux, s'appelle la cour Saint-Bernard. 

\zz
Les prisonniers, dans leur langage énergique, l'ont surnommé la Fosse-aux-Lions, probablement parce que les captifs ont des dents qui mordent souvent les barreaux et parfois les gardiens. 

C'est dans la prison une prison; les murs ont une épaisseur double des autres. Chaque jour un guichetier sonde avec soin les grilles massives, et l'on reconnaît à la stature herculéenne, aux regards froids et incisifs de ces gardiens, qu'ils ont été choisis pour régner sur leur peuple par la terreur et l'activité de l'intelligence. 

Le préau de ce quartier est encadré dans des murs énormes sur lesquels glisse obliquement le soleil lorsqu'il se décide à pénétrer dans ce gouffre de laideurs morales et physiques. C'est là, sur le pavé, que depuis l'heure du lever errent soucieux, hagards, pâlissants, comme des ombres, les hommes que la justice tient courbés sous le couperet qu'elle aiguise. 

On les voit se coller, s'accroupir, le long du mur qui absorbe et retient le plus de chaleur. Ils demeurent là, causant deux à deux, plus souvent isolés, l'œil sans cesse attiré vers la porte qui s'ouvre pour appeler quelqu'un des habitants de ce lugubre séjour, ou pour vomir dans le gouffre une nouvelle scorie rejetée du creuset de la société. 

La cour Saint-Bernard a son parloir particulier; c'est un carré long, divisé en deux parties par deux grilles parallèlement plantées à trois pieds l'une de l'autre, de façon que le visiteur ne puisse serrer la main du prisonnier ou lui passer quelque chose. Ce parloir est sombre, humide, et de tout point horrible, surtout lorsqu'on songe aux épouvantables confidences qui ont glissé sur ces grilles et rouillé le fer des barreaux. 

Cependant ce lieu, tout affreux qu'il est, est le paradis où viennent se retremper dans une société espérée, savourée, ces hommes dont les jours sont comptés: il est si rare qu'on sorte de la Fosse-aux-Lions pour aller autre part qu'à la barrière Saint-Jacques, au bagne ou au cabanon cellulaire! 

Dans cette cour que nous venons de décrire, et qui suait d'une froide humidité, se promenait, les mains dans les poches de son habit, un jeune homme considéré avec beaucoup de curiosité par les habitants de la Fosse. 

Il eût passé pour un homme élégant, grâce à la coupe de ses habits, si ces habits n'eussent été en lambeaux; cependant ils n'avaient pas été usés: le drap, fin et soyeux aux endroits intacts, reprenaient facilement son lustre sous la main caressante du prisonnier qui essayait d'en faire un habit neuf. 

Il appliquait le même soin à fermer une chemise de batiste considérablement changée de couleur depuis son entrée en prison, et sur ses bottes vernies passait le coin d'un mouchoir brodé d'initiales surmontées d'une couronne héraldique. 

Quelques pensionnaires de la Fosse-aux-Lions considéraient avec un intérêt marqué les recherches de toilette du prisonnier. 

«Tiens, voilà le prince qui se fait beau, dit un des voleurs. 

—Il est très beau naturellement, dit un autre, et s'il avait seulement un peigne et de la pommade, il éclipserait tous les messieurs à gants blancs. 

—Son habit a dû être bien neuf et ses bottes reluisent joliment. C'est flatteur pour nous qu'il y ait des confrères si comme il faut; et ces brigands de gendarmes sont bien vils. Les envieux! avoir déchiré une toilette comme cela! 

—Il paraît que c'est un fameux, dit un autre; il a tout fait\dots et dans le grand genre\dots Il vient de là-bas si jeune! oh! c'est superbe!» 

Et l'objet de cette admiration hideuse semblait savourer les éloges ou la vapeur des éloges, car il n'entendait pas les paroles. 

Sa toilette terminée, il s'approcha du guichet de la cantine auquel s'adossait un gardien: 

«Voyons, monsieur, lui dit-il, prêtez-moi vingt francs, vous les aurez bientôt; avec moi, pas de risques à courir. Songez donc que je tiens à des parents qui ont plus de millions que vous n'avez de deniers\dots Voyons, vingt francs, et je vous en prie, afin que je prenne une pistole et que j'achète une robe de chambre. Je souffre horriblement d'être toujours en habit et en bottes. Quel habit! monsieur, pour un prince Cavalcanti!» 

Le gardien lui tourna le dos et haussa les épaules. Il ne rit pas même de ces paroles qui eussent déridé tous les fronts car cet homme en avait entendu bien d'autres, ou plutôt il avait toujours entendu la même chose. 

«Allez, dit Andrea, vous êtes un homme sans entrailles, et je vous ferai perdre votre place.» 

Ce mot fit retourner le gardien, qui, cette fois, laissa échapper un bruyant éclat de rire. 

Alors les prisonniers s'approchèrent et firent cercle. 

«Je vous dis, continua Andrea, qu'avec cette misérable somme je pourrai me procurer un habit et une chambre, afin de recevoir d'une façon décente la visite illustre que j'attends d'un jour à l'autre. 

—Il a raison! il a raison! dirent les prisonniers\dots Pardieu! on voit bien que c'est un homme comme il faut. 

—Eh bien, prêtez-lui les vingt francs, dit le gardien en s'appuyant sur son autre colossale épaule; est-ce que vous ne devez pas cela à un camarade? 

—Je ne suis pas le camarade de ces gens, dit fièrement le jeune homme; ne m'insultez pas, vous n'avez pas ce droit-là.» 

Les voleurs se regardèrent avec de sourds murmures, et une tempête soulevée par la provocation du gardien, plus encore que par les paroles d'Andrea, commença de gronder sur le prisonnier aristocrate. 

Le gardien, sûr de faire le \textit{quos ego} quand les flots seraient trop tumultueux, les laissait monter peu à peu pour jouer un tour au solliciteur importun, et se donner une récréation pendant la longue garde de sa journée. 

Déjà les voleurs se rapprochaient d'Andrea; les uns se disaient: 

«La savate! la savate!» 

Cruelle opération qui consiste à rouer de coups, non pas de savate, mais de soulier ferré, un confrère tombé dans la disgrâce de ces messieurs. 

D'autres proposaient l'anguille; autre genre de récréation consistant à emplir de sable, de cailloux, de gros sous, quand ils en ont, un mouchoir tordu, que les bourreaux déchargent comme un fléau sur les épaules et la tête du patient. 

«Fouettons le beau monsieur, dirent quelques-uns, monsieur l'honnête homme!» 

Mais Andrea, se retournant vers eux, cligna de l'œil, enfla sa joue avec sa langue, et fit entendre ce claquement des lèvres qui équivaut à mille signes d'intelligence parmi les bandits réduits à se taire. 

C'était un signe maçonnique que lui avait indiqué Caderousse. 

Ils reconnurent un des leurs. 

Aussitôt les mouchoirs retombèrent; la savate ferrée rentra au pied du principal bourreau. On entendit quelques voix proclamer que monsieur avait raison, que monsieur pouvait être honnête à sa guise, et que les prisonniers voulaient donner l'exemple de la liberté de conscience. 

L'émeute recula. Le gardien en fut tellement stupéfait qu'il prit aussitôt Andrea par les mains et se mit à le fouiller, attribuant à quelques manifestations plus significatives que la fascination, ce changement subit des habitants de la Fosse-aux-Lions. 

Andrea se laissa faire, non sans protester. 

Tout à coup une voix retentit au guichet. 

«Benedetto!» criait un inspecteur. 

Le gardien lâcha sa proie. 

«On m'appelle? dit Andrea. 

—Au parloir! dit la voix. 

—Voyez-vous, on me rend visite. Ah! mon cher monsieur, vous allez voir si l'on peut traiter un Cavalcanti comme un homme ordinaire!» 

Et Andrea, glissant dans la cour comme une ombre noire, se précipita par le guichet entrebâillé, laissant dans l'admiration ses confrères et le gardien lui-même. 

On l'appelait en effet au parloir, et il ne faudrait pas s'en émerveiller moins qu'Andrea lui-même; car le rusé jeune homme, depuis son entrée à la Force, au lieu d'user, comme les gens du commun de ce bénéfice d'écrire pour se faire réclamer, avait gardé le plus stoïque silence. 

«Je suis, disait-il, évidemment protégé par quelqu'un de puissant; tout me le prouve; cette fortune soudaine, cette facilité avec laquelle j'ai aplani tous les obstacles, une famille improvisée, un nom illustre devenu ma propriété, l'or pleuvant chez moi, les alliances les plus magnifiques promises à mon ambition. Un malheureux oubli de ma fortune, une absence de mon protecteur m'a perdu, oui, mais pas absolument, pas à jamais! La main s'est retirée pour un moment, elle doit se tendre vers moi et me ressaisir de nouveau au moment où je me croirai prêt à tomber dans l'abîme. 

«Pourquoi risquerai-je une démarche imprudente? Je m'aliénerais peut-être le protecteur! Il y a deux moyens pour lui de me tirer d'affaire: l'évasion mystérieuse, achetée à prix d'or, et la main forcée aux juges pour obtenir une absolution. Attendons pour parler, pour agir qu'il me soit prouvé qu'on m'a totalement abandonné, et alors\dots» 

Andrea avait bâti un plan qu'on peut croire habile; le misérable était intrépide à l'attaque et rude à la défense. 

La misère de la prison commune, les privations de tout genre, il les avait supportées. Cependant peu à peu le naturel, ou plutôt l'habitude, avait repris le dessus. Andrea souffrait d'être nu, d'être sale, d'être affamé; le temps lui durait. 

C'est à ce moment d'ennui que la voix de l'inspecteur l'appela au parloir. 

Andrea sentit son cœur bondir de joie. Il était trop tôt pour que ce fût la visite du juge d'instruction, et trop tard pour que ce fût un appel du directeur de la prison ou du médecin; c'était donc la visite inattendue. 

Derrière la grille du parloir où Andrea fut introduit, il aperçut, avec ses yeux dilatés par une curiosité avide, la figure sombre et intelligente de M. Bertuccio, qui regardait aussi, lui, avec un étonnement douloureux, les grilles, les portes verrouillées et l'ombre qui s'agitait derrière les barreaux entrecroisés. 

«Ah! fit Andrea, touché au cœur. 

—Bonjour, Benedetto, dit Bertuccio de sa voix creuse et sonore. 

—Vous! vous! dit le jeune homme en regardant avec effroi autour de lui. 

—Tu ne me reconnais pas, dit Bertuccio, malheureux enfant! 

—Silence, mais silence donc! fit Andrea qui connaissait la finesse d'ouïe de ces murailles; mon Dieu, mon Dieu, ne parlez pas si haut! 

—Tu voudrais causer avec moi, n'est-ce pas, dit Bertuccio, seul à seul? 

—Oh! oui, dit Andrea. 

—C'est bien.» 

Et Bertuccio, fouillant dans sa poche, fit signe à un gardien qu'on apercevait derrière la vitre du guichet. 

«Lisez, dit-il. 

—Qu'est-ce que cela? dit Andrea. 

—L'ordre de te conduire dans une chambre, de t'installer et de me laisser communiquer avec toi. 

—Oh!» fit Andrea, bondissant de joie. 

Et tout de suite, se repliant en lui-même, il se dit: 

«Encore le protecteur inconnu! on ne m'oublie pas! On cherche le secret, puisqu'on veut causer dans une chambre isolée. Je les tiens\dots Bertuccio a été envoyé par le protecteur!» 

Le gardien conféra un moment avec un supérieur, puis ouvrit les deux portes grillées et conduisit à une chambre du premier étage ayant vue sur la cour Andrea, qui ne se sentait plus de joie. 

La chambre était blanchie à la chaux, comme c'est l'usage dans les prisons. Elle avait un aspect de gaieté qui parut rayonnant au prisonnier: un poêle, un lit, une chaise, une table en formaient le somptueux ameublement. 

Bertuccio s'assit sur la chaise. Andrea se jeta sur le lit. Le gardien se retira. 

«Voyons, dit l'intendant, qu'as-tu à me dire? 

—Et vous? dit Andrea. 

—Mais parle d'abord\dots 

—Oh! non; c'est vous qui avez beaucoup m'apprendre, puisque vous êtes venu me trouver. 

—Eh bien, soit. Tu as continué le cours de tes scélératesses: tu as volé, tu as assassiné. 

—Bon! si c'est pour me dire ces choses-là que vous me faites passer dans une chambre particulière, autant valait ne pas vous déranger. Je sais toutes ces choses. Il en est d'autres que je ne sais pas, au contraire. Parlons de celles-là, s'il vous plaît. Qui vous a envoyé? 

—Oh! oh! vous allez vite, monsieur Benedetto. 

—N'est-ce pas? et au but. Surtout ménageons les mots inutiles. Qui vous envoie? 

—Personne. 

—Comment savez-vous que je suis en prison? 

—Il y a longtemps que je t'ai reconnu dans le fashionable insolent qui poussait si gracieusement un cheval aux Champs-Élysées. 

—Les Champs-Élysées!\dots Ah! ah! nous brûlons, comme on dit au jeu de la pincette\dots Les Champs-Élysées\dots Ça, parlons un peu de mon père, voulez-vous? 

—Que suis-je donc? 

—Vous, mon brave monsieur, vous êtes mon père adoptif\dots Mais ce n'est pas vous, j'imagine, qui avez disposé en ma faveur d'une centaine de mille francs que j'ai dévorés en quatre ou cinq mois; ce n'est pas vous qui m'avez forgé un père italien et gentilhomme; ce n'est pas vous qui m'avez fait entrer dans le monde et invité à un certain dîner que je crois manger encore, à Auteuil, avec la meilleure compagnie de tout Paris, avec certain procureur du roi dont j'ai eu bien tort de ne pas cultiver la connaissance, qui me serait si utile en ce moment; ce n'est pas vous, enfin, qui me cautionniez pour un ou deux millions quand m'est arrivé l'accident fatal de la découverte du pot aux roses\dots Allons, parlez, estimable Corse, parlez\dots 

—Que veux-tu que je te dise? 

—Je t'aiderai. 

«Tu parlais des Champs-Élysées tout à l'heure, mon digne père nourricier. 

—Eh bien? 

—Eh bien, aux Champs-Élysées demeure un monsieur bien riche, bien riche. 

—Chez qui tu as volé et assassiné, n'est-ce pas? 

—Je crois que oui. 

—M. le comte de Monte-Cristo? 

—C'est vous qui l'avez nommé, comme dit M. Racine. Eh bien, dois-je me jeter entre ses bras, l'étrangler sur mon cœur en criant: «Mon père! mon père!» comme dit M. Pixérécourt? 

—Ne plaisantons pas, répondit gravement Bertuccio, et qu'un pareil nom ne soit pas prononcé ici comme vous osez le prononcer. 

—Bah! fit Andrea un peu étourdi de la solennité du maintien de Bertuccio, pourquoi pas? 

—Parce que celui qui porte ce nom est trop favorisé du ciel pour être le père d'un misérable tel que vous. 

—Oh! de grands mots\dots 

—Et de grands effets si vous n'y prenez garde! 

—Des menaces!\dots Je ne les crains pas\dots Je dirai\dots 

—Croyez-vous avoir affaire à des pygmées de votre espèce? dit Bertuccio d'un ton si calme et avec un regard si assuré qu'Andrea en fut remué jusqu'au fond des entrailles; croyez-vous avoir affaire à vos scélérats routiniers du bagne, ou à vos naïves dupes du monde?\dots Benedetto, vous êtes dans une main terrible, cette main veut bien s'ouvrir pour vous: profitez-en. Ne jouez pas avec la foudre qu'elle dépose pour un instant, mais qu'elle peut reprendre si vous essayez de la déranger dans son libre mouvement. 

—Mon père\dots je veux savoir qui est mon père! dit l'entêté; j'y périrai s'il le faut, mais je le saurai. Que me fait le scandale, à moi? du bien\dots de la réputation\dots des réclames\dots comme dit Beauchamp le journaliste. Mais vous autres, gens du grand monde, vous avez toujours quelque chose à perdre au scandale, malgré vos millions et vos armoiries\dots Çà, qui est mon père? 

—Je suis venu pour te le dire. 

—Ah!» s'écria Benedetto les yeux étincelants de joie. 

À ce moment la porte s'ouvrit, et le guichetier, s'adressant à Bertuccio: 

«Pardon, monsieur, dit-il, mais le juge d'instruction attend le prisonnier. 

—C'est la clôture de mon interrogatoire, dit Andrea au digne intendant\dots Au diable l'importun! 

—Je reviendrai demain, dit Bertuccio. 

—Bon! fit Andrea. Messieurs les gendarmes, je suis tout à vous\dots Ah! cher monsieur, laissez donc une dizaine d'écus au greffe pour qu'on me donne ici ce dont j'ai besoin. 

—Ce sera fait», répliqua Bertuccio. 

Andrea lui tendit la main, Bertuccio garda la sienne dans sa poche, et y fit seulement sonner quelques pièces d'argent. 

«C'est ce que je voulais dire,» fit Andrea grimaçant un sourire, mais tout à fait subjugué par l'étrange tranquillité de Bertuccio. 

«Me serais-je trompé? se dit-il en montant dans la voiture oblongue et grillée qu'on appelle le \textit{panier à salade}. Nous verrons! Ainsi, à demain! ajouta-t-il en se tournant vers Bertuccio. 

—À demain!» répondit l'intendant. 