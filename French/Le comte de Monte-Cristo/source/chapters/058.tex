\chapter{M. Noirtier de Villefort}

\lettrine{V}{oici} ce qui s'était passé dans la maison du procureur du roi après le départ de Mme Danglars et de sa fille, et pendant la conversation que nous venons de rapporter. 

\zz
M. de Villefort était entré chez son père, suivi de Mme de Villefort; quant à Valentine, nous savons où elle était. 

Tous deux, après avoir salué le vieillard, après avoir congédié Barrois, vieux domestique depuis plus de vingt-cinq ans à son service, avaient pris place à ses côtés. 

M. Noirtier, assis dans son grand fauteuil à roulettes, où on le plaçait le matin et d'où on le tirait le soir, assis devant une glace qui réfléchissait tout l'appartement et lui permettait de voir, sans même tenter un mouvement devenu impossible, qui entrait dans sa chambre, qui en sortait, et ce qu'on faisait tout autour de lui; M. Noirtier, immobile comme un cadavre, regardait avec des yeux intelligents et vifs ses enfants, dont la cérémonieuse révérence lui annonçait quelque démarche officielle inattendue. 

La vue et l'ouïe étaient les deux seuls sens qui animassent encore, comme deux étincelles, cette matière humaine déjà aux trois quarts façonnée pour la tombe; encore, de ces deux sens, un seul pouvait-il révéler au-dehors la vie intérieure qui animait la statue; et le regard qui dénonçait cette vie intérieure était semblable à une de ces lumières lointaines qui, durant la nuit, apprennent au voyageur perdu dans un désert qu'il y a encore un être existant qui veille dans ce silence et cette obscurité. 

Aussi, dans cet œil noir du vieux Noirtier, surmonté d'un sourcil noir, tandis que toute la chevelure, qu'il portait longue et pendante sur les épaules, était blanche; dans cet œil, comme cela arrive pour tout organe de l'homme exercé aux dépens des autres organes, s'étaient concentrées toute l'activité, toute l'adresse, toute la force, toute l'intelligence, répandues autrefois dans ce corps et dans cet esprit. Certes, le geste du bras, le son de la voix, l'attitude du corps manquaient, mais cet œil puissant suppléait à tout: il commandait avec les yeux; il remerciait avec les yeux; c'était un cadavre avec des yeux vivants, et rien n'était plus effrayant parfois que ce visage de marbre au haut duquel s'allumait une colère ou luisait une joie. Trois personnes seulement savaient comprendre ce langage du pauvre paralytique: c'était Villefort, Valentine et le vieux domestique dont nous avons déjà parlé. Mais comme Villefort ne voyait que rarement son père, et, pour ainsi dire, quand il ne pouvait faire autrement; comme, lorsqu'il le voyait, il ne cherchait pas à lui plaire en le comprenant, tout le bonheur du vieillard reposait en sa petite-fille, et Valentine était parvenue, à force de dévouement, d'amour et de patience, à comprendre du regard toutes les pensées de Noirtier. À ce langage muet ou inintelligible pour tout autre, elle répondait avec toute sa voix, toute sa physionomie, toute son âme, de sorte qu'il s'établissait des dialogues animés entre cette jeune fille et cette prétendue argile, à peu près redevenue poussière, et qui cependant était encore un homme d'un savoir immense, d'une pénétration inouïe et d'une volonté aussi puissante que peut l'être l'âme enfermée dans une matière par laquelle elle a perdu le pouvoir de se faire obéir. 

Valentine avait donc résolu cet étrange problème de comprendre la pensée du vieillard pour lui faire comprendre sa pensée à elle; et, grâce à cette étude, il était bien rare que, pour les choses ordinaires de la vie, elle ne tombât point avec précision sur le désir de cette âme vivante, ou sur le besoin de ce cadavre à moitié insensible. 

Quant au domestique, comme depuis vingt-cinq ans, ainsi que nous l'avons dit, il servait son maître, il connaissait si bien toutes ses habitudes, qu'il était rare que Noirtier eût besoin de lui demander quelque chose. 

Villefort n'avait en conséquence besoin du secours ni de l'un ni de l'autre pour entamer avec son père l'étrange conversation qu'il venait provoquer. Lui-même, nous l'avons dit, connaissait parfaitement le vocabulaire du vieillard, et s'il ne s'en servait point plus souvent, c'était par ennui et par indifférence. Il laissa donc Valentine descendre au jardin, il éloigna donc Barrois, et après avoir pris sa place à la droite de son père, tandis que Mme de Villefort s'asseyait à sa gauche: 

«Monsieur, dit-il, ne vous étonnez pas que Valentine ne soit pas montée avec nous et que j'aie éloigné Barrois, car la conférence que nous allons avoir ensemble est de celles qui ne peuvent avoir lieu devant une jeune fille ou un domestique; Mme de Villefort et moi avons une communication à vous faire.» 

Le visage de Noirtier resta impassible pendant ce préambule, tandis qu'au contraire l'œil de Villefort semblait vouloir plonger jusqu'au plus profond du cœur du vieillard. 

«Cette communication, continua le procureur du roi avec son ton glacé et qui semblait ne jamais admettre la contestation, nous sommes sûrs, Mme de Villefort et moi, qu'elle vous agréera.» 

L'œil du vieillard continua de demeurer atone; il écoutait: voilà tout. 

«Monsieur, reprit Villefort, nous marions Valentine.» 

Une figure de cire ne fût pas restée plus froide à cette nouvelle que ne resta la figure du vieillard. 

«Le mariage aura lieu avant trois mois», reprit Villefort. 

L'œil du vieillard continua d'être inanimé. 

Mme de Villefort prit la parole à son tour, et se hâta d'ajouter: 

«Nous avons pensé que cette nouvelle aurait de l'intérêt pour vous, monsieur; d'ailleurs Valentine a toujours semblé attirer votre affection; il nous reste donc à vous dire seulement le nom du jeune homme qui lui est destiné. C'est un des plus honorables partis auxquels Valentine puisse prétendre; il y a de la fortune, un beau nom et des garanties parfaites de bonheur dans la conduite et les goûts de celui que nous lui destinons, et dont le nom ne doit pas vous être inconnu. Il s'agit de M. Franz de Quesnel, baron d'Épinay.» 

Villefort, pendant le petit discours de sa femme, attachait sur le vieillard un regard plus attentif que jamais. Lorsque Mme de Villefort prononça le nom de Franz, l'œil de Noirtier, que son fils connaissait si bien, frissonna, et les paupières, se dilatant comme eussent pu faire des lèvres pour laisser passer des paroles, laissèrent, elles, passer un éclair. 

Le procureur du roi, qui savait les anciens rapports d'inimitié publique qui avaient existé entre son père et le père de Franz, comprit ce feu et cette agitation; mais cependant il les laissa passer comme inaperçus, et reprenant la parole où sa femme l'avait laissée: 

«Monsieur, dit-il, il est important, vous le comprenez bien, près comme elle est d'atteindre sa dix-neuvième année, que Valentine soit enfin établie. Néanmoins, nous ne vous avons point oublié dans les conférences, et nous nous sommes assurés d'avance que le mari de Valentine accepterait, sinon de vivre près de nous, qui gênerions peut-être un jeune ménage, du moins que vous, que Valentine chérit particulièrement, et qui, de votre côté, paraissez lui rendre cette affection, vivriez près d'eux, de sorte que vous ne perdrez aucune de vos habitudes, et que vous aurez seulement deux enfants au lieu d'un pour veiller sur vous.» 

L'éclair du regard de Noirtier devint sanglant. 

Assurément il se passait quelque chose d'affreux dans l'âme de ce vieillard; assurément le cri de la douleur et de la colère montait à sa gorge, et, ne pouvant éclater, l'étouffait, car son visage s'empourpra et ses lèvres devinrent bleues. 

Villefort ouvrit tranquillement une fenêtre en disant: 

«Il fait bien chaud ici, et cette chaleur fait mal à M. Noirtier.» 

Puis il revint, mais sans se rasseoir. 

«Ce mariage, ajouta Mme de Villefort, plaît à M. d'Épinay et à sa famille; d'ailleurs sa famille se compose seulement d'un oncle et d'une tante. Sa mère étant morte au moment où elle le mettait au monde, et son père ayant été assassiné en 1815, c'est-à-dire quand l'enfant avait deux ans à peine, il ne relève donc que de sa propre volonté. 

—Assassinat mystérieux, dit Villefort, et dont les auteurs sont restés inconnus, quoique le soupçon ait plané sans s'abattre au-dessus de la tête de beaucoup de gens.» 

Noirtier fit un tel effort que ses lèvres se contractèrent comme pour sourire. 

«Or, continua Villefort, les véritables coupables, ceux-là qui savent qu'ils ont commis le crime, ceux-là sur lesquels peut descendre la justice des hommes pendant leur vie et la justice de Dieu après leur mort, seraient bien heureux d'être à notre place, et d'avoir une fille à offrir à M. Franz d'Épinay pour éteindre jusqu'à l'apparence du soupçon.» 

Noirtier s'était calmé avec une puissance que l'on n'aurait pas dû attendre de cette organisation brisée. 

«Oui, je comprends», répondit-il du regard à Villefort; et ce regard exprimait tout ensemble le dédain profond et la colère intelligente. 

Villefort, de son côté, répondit à ce regard, dans lequel il avait lu ce qu'il contenait, par un léger mouvement d'épaules. 

Puis il fit signe à sa femme de se lever. 

«Maintenant, monsieur, dit Mme de Villefort, agréez tous mes respects. Vous plaît-il qu'Édouard vienne vous présenter ses respects?» 

Il était convenu que le vieillard exprimait son approbation en fermant les yeux, son refus en les clignant à plusieurs reprises, et avait quelque désir à exprimer quand il les levait au ciel. 

S'il demandait Valentine, il fermait l'œil droit seulement.  

S'il demandait Barrois, il fermait l'œil gauche. 

À la proposition de Mme de Villefort, il cligna vivement les yeux. 

Mme de Villefort, accueillie par un refus évident, se pinça les lèvres. 

«Je vous enverrai donc Valentine, alors? dit-elle. 

—Oui», fit le vieillard en fermant les yeux avec vivacité. 

M. et Mme de Villefort saluèrent et sortirent en ordonnant qu'on appelât Valentine, déjà prévenue au reste qu'elle aurait quelque chose à faire dans la journée près de M. Noirtier. 

Derrière eux, Valentine, toute rose encore d'émotion, entra chez le vieillard. Il ne lui fallut qu'un regard pour qu'elle comprît combien souffrait son aïeul et combien de choses il avait à lui dire. 

«Oh! bon papa, s'écria-t-elle, qu'est-il donc arrivé? On t'a fâché, n'est-ce pas, et tu es en colère? 

—Oui, fit-il, en fermant les yeux. 

—Contre qui donc? contre mon père? non; contre Mme de Villefort? non; contre moi?» 

Le vieillard fit signe que oui. 

«Contre moi?» reprit Valentine étonnée. 

Le vieillard renouvela le signe. 

«Et que t'ai-je donc fait, cher bon papa?» s'écria Valentine. 

Pas de réponse, elle continua: 

«Je ne t'ai pas vu de la journée; on t'a donc rapporté quelque chose de moi? 

—Oui, dit le regard du vieillard avec vivacité. 

—Voyons donc que je cherche. Mon Dieu, je te jure, bon père\dots. Ah!\dots M. et Mme de Villefort sortent d'ici, n'est-ce pas? 

—Oui. 

—Et ce sont eux qui t'ont dit ces choses qui te fâchent? Qu'est-ce donc? Veux-tu que j'aille le leur demander pour que je puisse m'excuser près de toi? 

—Non, non, fit le regard. 

—Oh! mais tu m'effraies. Qu'ont-ils pu dire, mon Dieu!» 

Et elle chercha. 

«Oh! j'y suis, dit-elle en baissant la voix et en se rapprochant du vieillard. Ils ont parlé de mon mariage peut-être? 

—Oui, répliqua le regard courroucé. 

—Je comprends; tu m'en veux de mon silence. Oh! vois-tu, c'est qu'ils m'avaient bien recommandé de ne t'en rien dire; c'est qu'ils ne m'en avaient rien dit à moi-même, et que j'avais surpris en quelque sorte ce secret par indiscrétion; voilà pourquoi j'ai été si réservée avec toi. Pardonne-moi, bon papa Noirtier.» 

Redevenu fixe et atone, le regard sembla répondre: «Ce n'est pas seulement ton silence qui m'afflige.» 

«Qu'est-ce donc? demanda la jeune fille: tu crois peut-être que je t'abandonnerais, bon père, et que mon mariage me rendrait oublieuse? 

—Non, dit le vieillard. 

—Ils t'ont dit alors que M. d'Épinay consentait à ce que nous demeurassions ensemble? 

—Oui. 

—Alors pourquoi es-tu fâché?» 

Les yeux du vieillard prirent une expression de douceur infinie. 

«Oui, je comprends, dit Valentine; parce que tu m'aimes?» 

Le vieillard fit signe que oui. 

«Et tu as peur que je ne sois malheureuse? 

—Oui. 

—Tu n'aimes pas M. Franz?» 

Les yeux répétèrent trois ou quatre fois: 

«Non, non, non. 

—Alors tu as bien du chagrin, bon père? 

—Oui. 

—Eh bien, écoute, dit Valentine en se mettant à genoux devant Noirtier et en lui passant ses bras autour du cou, moi aussi, j'ai bien du chagrin, car, moi non plus, je n'aime pas M. Franz d'Épinay.» 

Un éclair de joie passa dans les yeux de l'aïeul. 

«Quand j'ai voulu me retirer au couvent, tu te rappelles bien que tu as été si fort fâché contre moi?» 

Une larme humecta la paupière aride du vieillard. 

«Eh bien, continua Valentine, c'était pour échapper à ce mariage qui fait mon désespoir.» 

La respiration de Noirtier devint haletante. 

«Alors, ce mariage te fait bien du chagrin, bon père? Ô mon Dieu, si tu pouvais m'aider, si nous pouvions à nous deux rompre leur projet! Mais tu es sans force contre eux, toi dont l'esprit cependant est si vif et la volonté si ferme, mais quand il s'agit de lutter tu es aussi faible et même plus faible que moi. Hélas! tu eusses été pour moi un protecteur si puissant aux jours de ta force et de ta santé; mais aujourd'hui tu ne peux plus que me comprendre et te réjouir ou t'affliger avec moi. C'est un dernier bonheur que Dieu a oublié de m'enlever avec les autres.» 

Il y eut à ces paroles, dans les yeux de Noirtier, une telle impression de malice et de profondeur, que la jeune fille crut y lire ces mots: 

«Tu te trompes, je puis encore beaucoup pour toi. 

—Tu peux quelque chose pour moi, cher bon papa? traduisit Valentine. 

—Oui.» 

Noirtier leva les yeux au ciel. C'était le signe convenu entre lui et Valentine lorsqu'il désirait quelque chose. 

«Que veux-tu, cher père? voyons.» 

Valentine chercha un instant dans son esprit, exprima tout haut ses pensées à mesure qu'elles se présentaient à elle, et voyant qu'à tout ce qu'elle pouvait dire le vieillard répondait constamment \textit{non}: 

«Allons, fit-elle, les grands moyens, puisque je suis si sotte!» 

Alors elle récita l'une après l'autre toutes les lettres de l'alphabet, depuis A jusqu'à N, tandis que son sourire interrogeait l'œil du paralytique; à N, Noirtier fit signe que oui. 

«Ah! dit Valentine, la chose que vous désirez commence par la lettre N! c'est à l'N que nous avons affaire? Eh bien, voyons, que lui voulons-nous à l'N? Na, ne, ni, no. 

—Oui, oui, oui, fit le vieillard. 

—Ah! c'est \textit{no}? 

—Oui.» 

Valentine alla chercher un dictionnaire qu'elle posa sur un pupitre devant Noirtier: elle l'ouvrit, et quand elle eut vu l'œil du vieillard fixé sur les feuilles, son doigt courut vivement du haut en bas des colonnes. L'exercice, depuis six ans que Noirtier était tombé dans le fâcheux état où il se trouvait, lui avait rendu les épreuves si faciles, qu'elle devinait aussi vite la pensée du vieillard que si lui-même eût pu chercher dans le dictionnaire. 

Au mot \textit{notaire}, Noirtier fit signe de s'arrêter. 

«\textit{Notaire}, dit-elle; tu veux un notaire, bon papa?» 

Le vieillard fit signe que c'était effectivement un notaire qu'il désirait. 

«Il faut donc envoyer chercher un notaire? demanda Valentine. 

—Oui, fit le paralytique. 

—Mon père doit-il le savoir? 

—Oui. 

—Es-tu pressé d'avoir ton notaire? 

—Oui. 

—Alors on va te l'envoyer chercher tout de suite, cher père. Est-ce tout ce que tu veux? 

—Oui.» 

Valentine courut à la sonnette et appela un domestique pour le prier de faire venir M. ou Mme de Villefort chez le grand-père. 

«Es-tu content? dit Valentine; oui\dots je le crois bien: hein? ce n'était pas facile à trouver, cela?» 

Et la jeune fille sourit à l'aïeul comme elle eût pu faire à un enfant. 

M. de Villefort entra ramené par Barrois. 

«Que voulez-vous, monsieur? demanda-t-il au paralytique. 

—Monsieur, dit Valentine, mon grand-père désire un notaire.» 

À cette demande étrange et surtout inattendue, M. de Villefort échangea un regard avec le paralytique. 

«Oui», fit ce dernier avec une fermeté qui indiquait qu'avec l'aide de Valentine et de son vieux serviteur, qui savait maintenant ce qu'il désirait, il était prêt à soutenir la lutte. 

«Vous demandez le notaire? répéta Villefort. 

—Oui. 

—Pour quoi faire?» 

Noirtier ne répondit pas. 

«Mais qu'avez-vous besoin d'un notaire?» demanda Villefort. 

Le regard du paralytique demeura immobile et par conséquent muet, ce qui voulait dire: Je persiste dans ma volonté. 

«Pour nous faire quelque mauvais tour? dit Villefort; est-ce la peine? 

—Mais enfin, dit Barrois, prêt à insister avec la persévérance habituelle aux vieux domestiques, si monsieur veut un notaire, c'est apparemment qu'il en a besoin. Ainsi je vais chercher un notaire.» 

Barrois ne reconnaissait d'autre maître que Noirtier et n'admettait jamais que ses volontés fussent contestées en rien. 

«Oui, je veux un notaire», fit le vieillard en fermant les yeux d'un air de défi et comme s'il eût dit: Voyons si l'on osera me refuser ce que je veux. 

«On aura un notaire, puisque vous en voulez absolument un, monsieur; mais je m'excuserai près de lui et vous excuserai vous-même, car la scène sera fort ridicule. 

—N'importe, dit Barrois, je vais toujours l'aller chercher.» 

Et le vieux serviteur sortit triomphant. 