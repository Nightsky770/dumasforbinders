\chapter{Haydée}

\lettrine{O}{n} se rappelle quelles étaient les nouvelles ou plutôt les anciennes connaissances du comte de Monte-Cristo qui demeuraient rue Meslay: c'étaient Maximilien, Julie et Emmanuel. 

\zz
L'espoir de cette bonne visite qu'il allait faire, de ces quelques moments heureux qu'il allait passer, de cette lueur du paradis glissant dans l'enfer où il s'était volontairement engagé, avait répandu, à partir du moment où il avait perdu de vue Villefort, la plus charmante sérénité sur le visage du comte, et Ali, qui était accouru au bruit du timbre, en voyant ce visage si rayonnant d'une joie si rare, s'était retiré sur la pointe du pied et la respiration suspendue, comme pour ne pas effaroucher les bonnes pensées qu'il croyait voir voltiger autour de son maître. 

Il était midi: le comte s'était réservé une heure pour monter chez Haydée; on eût dit que la joie ne pouvait rentrer tout à coup dans cette âme si longtemps brisée, et qu'elle avait besoin de se préparer aux émotions douces, comme les autres âmes ont besoin de se préparer aux émotions violentes. 

La jeune Grecque était, comme nous l'avons dit, dans un appartement entièrement séparé de l'appartement du comte. Cet appartement était tout entier meublé à la manière orientale; c'est-à-dire que les parquets étaient couverts d'épais tapis de Turquie, que des étoffes de brocart retombaient le long des murailles, et que dans chaque pièce, un large divan régnait tout autour de la chambre avec des piles de coussins qui se déplaçaient à la volonté de ceux qui en usaient. 

Haydée avait trois femmes françaises et une femme grecque. Les trois femmes françaises se tenaient dans la première pièce, prêtes à accourir au bruit d'une petite sonnette d'or et à obéir aux ordres de l'esclave romaïque, laquelle savait assez de français pour transmettre les volontés de sa maîtresse à ses trois caméristes, auxquelles Monte-Cristo avait recommandé d'avoir pour Haydée les égards que l'on aurait pour une reine. 

La jeune fille était dans la pièce la plus reculée de son appartement, c'est-à-dire dans une espèce de boudoir rond, éclairé seulement par le haut, et dans lequel le jour ne pénétrait qu'à travers des carreaux de verre rose. Elle était couchée à terre sur des coussins de satin bleu brochés d'argent, à demi renversée en arrière sur le divan, encadrant sa tête avec son bras droit mollement arrondi, tandis que, du gauche, elle fixait à travers ses lèvres le tube de corail dans lequel était enchâssé le tuyau flexible d'un narguilé, qui ne laissait arriver la vapeur à sa bouche que parfumée par l'eau de benjoin, à travers laquelle sa douce aspiration la forçait de passer. 

Sa pose, toute naturelle pour une femme d'Orient, eût été pour une Française d'une coquetterie peut-être un peu affectée. 

Quant à sa toilette, c'était celle des femmes épirotes, c'est-à-dire un caleçon de satin blanc broché de fleurs roses, et qui laissait à découvert deux pieds d'enfant qu'on eût crus de marbre de Paros, si on ne les eût vus se jouer avec deux petites sandales à la pointe recourbée, brodée d'or et de perles; une veste à longues raies bleues et blanches, à larges manches fendues pour les bras, avec des boutonnières d'argent et des boutons de perles; enfin une espèce de corset laissant, par sa coupe ouverte en cœur, voir le cou et tout le haut de la poitrine, et se boutonnant au-dessous du sein par trois boutons de diamant. Quant au bas du corset et au haut du caleçon, ils étaient perdus dans une des ceintures aux vives couleurs et aux longues franges soyeuses qui font l'ambition de nos élégantes Parisiennes. 

La tête était coiffée d'une petite calotte d'or brodée de perles, inclinée sur le côté, et au-dessous de la calotte, du côté où elle inclinait, une belle rose naturelle de couleur pourpre ressortait mêlée à des cheveux si noirs qu'ils paraissaient bleus. 

Quant à la beauté de ce visage, c'était la beauté grecque dans toute la perfection de son type, avec ses grands yeux noirs veloutés, son nez droit, ses lèvres de corail et ses dents de perles. 

Puis, sur ce charmant ensemble, la fleur de la jeunesse était répandue avec tout son éclat et tout son parfum; Haydée pouvait avoir dix-neuf ou vingt ans. 

Monte-Cristo appela la suivante grecque, et fit demander à Haydée la permission d'entrer auprès d'elle. 

Pour toute réponse, Haydée fit signe à la suivante de relever la tapisserie qui pendait devant la porte, dont le chambranle carré encadra la jeune fille couchée comme un charmant tableau. Monte-Cristo s'avança. 

Haydée se souleva sur le coude qui tenait le narguilé, et tendant au comte sa main en même temps qu'elle l'accueillait avec un sourire: 

«Pourquoi, dit-elle dans la langue sonore des filles de Sparte et d'Athènes, pourquoi me fais-tu demander la permission d'entrer chez moi? N'es-tu plus mon maître, ne suis-je plus ton esclave?» 

Monte-Cristo sourit à son tour. 

«Haydée, dit-il, vous savez\dots. 

—Pourquoi ne me dis-tu pas tu comme d'habitude? interrompit la jeune Grecque; ai-je donc commis quelque faute? En ce cas il faut me punir, mais non pas me dire vous. 

—Haydée, reprit le comte, tu sais que nous sommes en France, et par conséquent que tu es libre. 

—Libre de quoi faire? demanda la jeune fille. 

—Libre de me quitter.  

—Te quitter!\dots et pourquoi te quitterais-je? 

—Que sais-je, moi? Nous allons voir le monde. 

—Je ne veux voir personne. 

—Et si parmi les beaux jeunes gens que tu rencontreras, tu en trouvais quelqu'un qui te plût, je ne serais pas assez injuste\dots. 

—Je n'ai jamais vu d'hommes plus beaux que toi, et je n'ai jamais aimé que mon père et toi. 

—Pauvre enfant, dit Monte-Cristo, c'est que tu n'as guère parlé qu'à ton père et à moi.  

—Eh bien, qu'ai-je besoin de parler à d'autres? Mon père m'appelait \textit{sa joie}; toi, tu m'appelles \textit{ton amour}, et tous deux vous m'appelez \textit{votre enfant}. 

—Tu te rappelles ton père, Haydée?» 

La jeune fille sourit. 

«Il est là et là, dit-elle en mettant la main sur ses yeux et sur son cœur. 

—Et moi, où suis-je? demanda en souriant Monte-Cristo. 

—Toi, dit-elle, tu es partout.» 

Monte-Cristo prit la main d'Haydée pour la baiser; mais la naïve enfant retira sa main et présenta son front. 

«Maintenant, Haydée, lui dit-il, tu sais que tu es libre, que tu es maîtresse, que tu es reine; tu peux garder ton costume ou le quitter à ta fantaisie; tu resteras ici quand tu voudras rester, tu sortiras quand tu voudras sortir; il y aura toujours une voiture attelée pour toi; Ali et Myrto t'accompagneront partout et seront à tes ordres; seulement, une seule chose, je te prie. 

—Dis. 

—Garde le secret sur ta naissance, ne dis pas un mot de ton passé; ne prononce dans aucune occasion le nom de ton illustre père ni celui de ta pauvre mère. 

—Je te l'ai déjà dit, seigneur, je ne verrai personne. 

—Écoute, Haydée; peut-être cette réclusion tout orientale sera-t-elle impossible à Paris: continue d'apprendre la vie de nos pays du Nord comme tu l'as fait à Rome, à Florence, à Milan et à Madrid; cela te servira toujours, que tu continues à vivre ici ou que tu retournes en Orient.» 

La jeune fille leva sur le comte ses grands yeux humides et répondit: 

«Ou que nous retournions en Orient, veux-tu dire, n'est-ce pas, mon seigneur? 

—Oui, ma fille, dit Monte-Cristo; tu sais bien que ce n'est jamais moi qui te quitterai. Ce n'est point l'arbre qui quitte la fleur, c'est la fleur qui quitte l'arbre. 

—Je ne te quitterai jamais, seigneur, dit Haydée, car je suis sûre que je ne pourrais pas vivre sans toi. 

—Pauvre enfant! dans dix ans je serai vieux, et dans dix ans tu seras jeune encore. 

—Mon père avait une longue barbe blanche, cela ne m'empêchait point de l'aimer; mon père avait soixante ans, et il me paraissait plus beau que tous les jeunes hommes que je voyais. 

—Mais voyons, dis-moi, crois-tu que tu t'habitueras ici? 

—Te verrai-je? 

—Tous les jours. 

—Eh bien, que me demandes-tu donc, seigneur? 

—Je crains que tu ne t'ennuies. 

—Non, seigneur, car le matin je penserai que tu viendras, et le soir je me rappellerai que tu es venu; d'ailleurs, quand je suis seule, j'ai de grands souvenirs, je revois d'immenses tableaux, de grands horizons avec le Pinde et l'Olympe dans le lointain; puis j'ai dans le cœur trois sentiments avec lesquels on ne s'ennuie jamais: de la tristesse, de l'amour et de la reconnaissance. 

—Tu es une digne fille de l'Épire, Haydée, gracieuse et poétique, et l'on voit que tu descends de cette famille de déesses qui est née dans ton pays. Sois donc tranquille, ma fille, je ferai en sorte que ta jeunesse ne soit pas perdue, car si tu m'aimes comme ton père, moi, je t'aime comme mon enfant. 

—Tu te trompes, seigneur; je n'aimais point mon père comme je t'aime; mon amour pour toi est un autre amour: mon père est mort et je ne suis pas morte; tandis que toi, si tu mourais, je mourrais.»  

Le comte tendit la main à la jeune fille avec un sourire de profonde tendresse; elle y imprima ses lèvres comme d'habitude. 

Et le comte, ainsi disposé à l'entrevue qu'il allait avoir avec Morrel et sa famille, partit en murmurant ces vers de Pindare: 

«La jeunesse est une fleur dont l'amour est le fruit\dots. Heureux le vendangeur qui le cueille après l'avoir vu lentement mûrir.» 

Selon ses ordres, la voiture était prête. Il y monta, et la voiture, comme toujours, partit au galop. 