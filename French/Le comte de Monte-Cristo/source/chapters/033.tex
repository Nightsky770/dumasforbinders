\chapter{Bandits romains}

\lettrine{L}{e} lendemain, Franz se réveilla le premier, et aussitôt réveillé, sonna. 

\zz
Le tintement de la clochette vibrait encore, lorsque maître Pastrini entra en personne. 

«Eh bien, dit l'hôte triomphant, et sans même attendre que Franz l'interrogeât, je m'en doutais bien hier, Excellence, quand je ne voulais rien vous promettre; vous vous y êtes pris trop tard, et il n'y a plus une seule calèche à Rome: pour les trois derniers jours, s'entend. 

—Oui, reprit Franz, c'est-à-dire pour ceux où elle est absolument nécessaire. 

—Qu'y a-t-il? demanda Albert en entrant, pas de calèche? 

—Justement, mon cher ami, répondit Franz, et vous avez deviné du premier coup. 

—Eh bien, voilà une jolie ville que votre ville éternelle! 

—C'est-à-dire, Excellence, reprit maître Pastrini, qui désirait maintenir la capitale du monde chrétien dans une certaine dignité à l'égard de ses voyageurs, c'est-à-dire qu'il n'y a plus de calèche à partir de dimanche matin jusqu'à mardi soir, mais d'ici là vous en trouverez cinquante si vous voulez. 

—Ah! c'est déjà quelque chose, dit Albert; nous sommes aujourd'hui jeudi; qui sait, d'ici à dimanche, ce qui peut arriver? 

—Il arrivera dix à douze mille voyageurs, répondit Franz, lesquels rendront la difficulté plus grande encore. 

—Mon ami, dit Morcerf, jouissons du présent et n'assombrissons pas l'avenir. 

—Au moins, demanda Franz, nous pourrons avoir une fenêtre? 

—Sur quoi? 

—Sur la rue du Cours, parbleu! 

—Ah! bien oui, une fenêtre! s'exclama maître Pastrini; impossible; de toute impossibilité! Il en restait une au cinquième étage du palais Doria, et elle a été louée à un prince russe pour vingt sequins par jour.» 

Les deux jeunes gens se regardaient d'un air stupéfait. 

«Eh bien, mon cher, dit Franz à Albert, savez-vous ce qu'il y a de mieux à faire? c'est de nous en aller passer le carnaval à Venise; au moins là, si nous ne trouvons pas de voiture, nous trouverons des gondoles. 

—Ah! ma foi non! s'écria Albert, j'ai décidé que je verrais le carnaval à Rome, et je l'y verrai, fût-ce sur des échasses. 

—Tiens! s'écria Franz, c'est une idée triomphante, surtout pour éteindre les moccoletti, nous nous déguiserons en polichinelles vampires ou en habitants des Landes, et nous aurons un succès fou. 

—Leurs Excellences désirent-elles toujours une voiture jusqu'à dimanche? 

—Parbleu! dit Albert, est-ce que vous croyez que nous allons courir les rues de Rome à pied, comme des clercs d'huissier? 

—Je vais m'empresser d'exécuter les ordres de Leurs Excellences, dit maître Pastrini: seulement je les préviens que la voiture leur coûtera six piastres par jour. 

—Et moi, mon cher monsieur Pastrini, dit Franz, moi qui ne suis pas notre voisin le millionnaire, je vous préviens à mon tour, qu'attendu que c'est la quatrième fois que je viens à Rome, je sais le prix des calèches, jours ordinaires, dimanches et fêtes. Nous vous donnerons douze piastres pour aujourd'hui, demain et après-demain, et vous aurez encore un fort joli bénéfice. 

—Cependant, Excellence!\dots dit maître Pastrini, essayant de se rebeller. 

—Allez, mon cher hôte, allez, dit Franz, ou je vais moi-même faire mon prix avec votre \textit{affettatore}, qui est le mien aussi, c'est un vieil ami à moi, qui m'a déjà pas mal volé d'argent dans sa vie, et qui, dans l'espérance de m'en voler encore, en passera par un prix moindre que celui que je vous offre: vous perdrez donc la différence et ce sera votre faute. 

—Ne prenez pas cette peine, Excellence, dit maître Pastrini, avec ce sourire du spéculateur italien qui s'avoue vaincu, je ferai de mon mieux, et j'espère que vous serez content. 

—À merveille! voilà ce qui s'appelle parler. Quand voulez-vous la voiture? 

—Dans une heure. 

—Dans une heure elle sera à la porte.» 

Une heure après, effectivement, la voiture attendait les deux jeunes gens: c'était un modeste fiacre que, vu la solennité de la circonstance, on avait élevé au rang de calèche; mais, quelque médiocre apparence qu'il eût, les deux jeunes gens se fussent trouvés bien heureux d'avoir un pareil véhicule pour les trois derniers jours. 

«Excellence! cria le cicérone en voyant Franz mettre le nez à la fenêtre, faut-il faire approcher le carrosse du palais?» 

Si habitué que fût Franz à l'emphase italienne, son premier mouvement fut de regarder autour de lui mais c'était bien à lui-même que ces paroles s'adressaient. 

Franz était l'Excellence; le carrosse, c'était le fiacre; le palais, c'était l'hôtel de Londres. 

Tout le génie laudatif de la nation était dans cette seule phrase. 

Franz et Albert descendirent. Le carrosse s'approcha du palais. Leurs Excellences allongèrent leurs jambes sur les banquettes, le cicérone sauta sur le siège de derrière. 

«Où Leurs Excellences veulent-elles qu'on les conduise? 

—Mais, à Saint-Pierre d'abord, et au Colisée ensuite», dit Albert en véritable Parisien. 

Mais Albert ne savait pas une chose: c'est qu'il faut un jour pour voir Saint-Pierre, et un mois pour l'étudier: la journée se passa donc rien qu'à voir Saint-Pierre. 

Tout à coup, les deux amis s'aperçurent que le jour baissait. 

Franz tira sa montre, il était quatre heures et demie. 

On reprit aussitôt le chemin de l'hôtel. À la porte, Franz donna l'ordre au cocher de se tenir prêt à huit heures. Il voulait faire voir à Albert le Colisée au clair de lune, comme il lui avait fait voir Saint-Pierre au grand jour. Lorsqu'on fait voir à un ami une ville qu'on a déjà vue, on y met la même coquetterie qu'à montrer une femme dont on a été l'amant. 

En conséquence, Franz traça au cocher son itinéraire; il devait sortir par la porte del Popolo, longer la muraille extérieure et rentrer par la porte San-Giovanni. Ainsi le Colisée leur apparaissait sans préparation aucune, et sans que le Capitole, le Forum, l'arc de Septime Sévère, le temple d'Antonin et Faustine et la Via Sacra eussent servi de degrés placés sur sa route pour le rapetisser. 

On se mit à table: maître Pastrini avait promis à ses hôtes un festin excellent; il leur donna un dîner passable: il n'y avait rien à dire. 

À la fin du dîner, il entra lui-même: Franz crut d'abord que c'était pour recevoir ses compliments et s'apprêtait à les lui faire, lorsqu'aux premiers mots il l'interrompit: 

«Excellence, dit-il, je suis flatté de votre approbation; mais ce n'était pas pour cela que j'étais monté chez vous\dots. 

—Était-ce pour nous dire que vous aviez trouvé une voiture? demanda Albert en allumant son cigare. 

—Encore moins, et même, Excellence, vous ferez bien de n'y plus penser et d'en prendre votre parti. À Rome, les choses se peuvent ou ne se peuvent pas. Quand on vous a dit qu'elles ne se pouvaient pas, c'est fini. 

—À Paris, c'est bien plus commode: quand cela ne se peut pas, on paie le double et l'on a à l'instant même ce que l'on demande. 

—J'entends dire cela à tous les Français, dit maître Pastrini un peu piqué, ce qui fait que je ne comprends pas comment ils voyagent. 

—Mais aussi, dit Albert en poussant flegmatiquement sa fumée au plafond et en se renversant balancé sur les deux pieds de derrière de son fauteuil, ce sont les fous et les niais comme nous qui voyagent; les gens sensés ne quittent pas leur hôtel de la rue du Helder, le boulevard de Gand et le café de Paris.» 

Il va sans dire qu'Albert demeurait dans la rue susdite, faisait tous les jours sa promenade fashionable, et dînait quotidiennement dans le seul café où l'on dîne, quand toutefois on est en bons termes avec les garçons. 

Maître Pastrini resta un instant silencieux, il était évident qu'il méditait la réponse, qui sans doute ne lui paraissait pas parfaitement claire. 

«Mais enfin, dit Franz à son tour, interrompant les réflexions géographiques de son hôte, vous étiez venu dans un but quelconque; voulez-vous nous exposer l'objet de votre visite? 

—Ah! c'est juste; le voici: vous avez commandé la calèche pour huit heures? 

—Parfaitement. 

—Vous avez l'intention de visiter il Colosseo? 

—C'est-à-dire le Colisée? 

—C'est exactement la même chose. 

—Soit. 

—Vous avez dit à votre cocher de sortir par la porte del Popolo, de faire le tour des murs et de rentrer par la porte San-Giovanni? 

—Ce sont mes propres paroles. 

—Eh bien, cet itinéraire est impossible. 

—Impossible! 

—Ou du moins fort dangereux. 

—Dangereux! et pourquoi? 

—À cause du fameux Luigi Vampa. 

—D'abord, mon cher hôte, qu'est-ce que le fameux Luigi Vampa? demanda Albert; il peut être très fameux à Rome, mais je vous préviens qu'il est ignoré à Paris. 

—Comment! vous ne le connaissez pas? 

—Je n'ai pas cet honneur. 

—Vous n'avez jamais entendu prononcer son nom? 

—Jamais.  

—Eh bien, c'est un bandit auprès duquel les Deseraris et les Gasparone sont des espèces d'enfants de chœur. 

—Attention, Albert! s'écria Franz, voilà donc enfin un bandit! 

—Je vous préviens, mon cher hôte, que je ne croirai pas un mot de ce que vous allez nous dire. Ce point arrêté entre nous, parlez tant que vous voudrez, je vous écoute. «Il y avait une fois\dots» Eh bien, allez donc!» 

Maître Pastrini se retourna du côté de Franz, qui lui paraissait le plus raisonnable des deux jeunes gens. Il faut rendre justice au brave homme: il avait logé bien des Français dans sa vie, mais jamais il n'avait compris certain côté de leur esprit. 

«Excellence, dit-il fort gravement, s'adressant, comme nous l'avons dit, à Franz, si vous me regardez comme un menteur, il est inutile que je vous dise ce que je voulais vous dire; je puis cependant vous affirmer que c'était dans l'intérêt de Vos Excellences. 

—Albert ne vous dit pas que vous êtes un menteur, mon cher monsieur Pastrini, reprit Franz, il vous dit qu'il ne vous croira pas, voilà tout. Mais, moi, je vous croirai, soyez tranquille; parlez donc. 

—Cependant, Excellence, vous comprenez bien que si l'on met en doute ma véracité\dots 

—Mon cher, reprit Franz, vous êtes plus susceptible que Cassandre, qui cependant était prophétesse, et que personne n'écoutait; tandis que vous, au moins, vous êtes sûr de la moitié de votre auditoire. Voyons, asseyez-vous, et dites-nous ce que c'est que M. Vampa. 

—Je vous l'ai dit, Excellence, c'est un bandit, comme nous n'en avons pas encore vu depuis le fameux Mastrilla. 

—Eh bien, quel rapport a ce bandit avec l'ordre que j'ai donné à mon cocher de sortir par la porte del Popolo et de rentrer par la porte San-Giovanni? 

—Il y a, répondit maître Pastrini, que vous pourrez bien sortir par l'une, mais que je doute que vous rentriez par l'autre. 

—Pourquoi cela? demanda Franz. 

—Parce que, la nuit venue, on n'est plus en sûreté à cinquante pas des portes. 

—D'honneur? s'écria Albert. 

—Monsieur le vicomte, dit maître Pastrini, toujours blessé jusqu'au fond du cœur du doute émis par Albert sur sa véracité, ce que je dis n'est pas pour vous, c'est pour votre compagnon de voyage, qui connaît Rome, lui, et qui sait qu'on ne badine pas avec ces choses-là. 

—Mon cher, dit Albert s'adressant à Franz, voici une aventure admirable toute trouvée: nous bourrons notre calèche de pistolets, de tromblons et de fusils à deux coups. Luigi Vampa vient pour nous arrêter, nous l'arrêtons. Nous le ramenons à Rome; nous en faisons hommage à Sa Sainteté, qui nous demande ce qu'elle peut faire pour reconnaître un si grand service. Alors nous réclamons purement et simplement un carrosse et deux chevaux de ses écuries, et nous voyons le carnaval en voiture; sans compter que probablement le peuple romain, reconnaissant, nous couronne au Capitole et nous proclame, comme Curtius et Horatius Coclès, les sauveurs de la patrie.» 

Pendant qu'Albert déduisait cette proposition, maître Pastrini faisait une figure qu'on essayerait vainement de décrire. 

«Et d'abord, demanda Franz à Albert, où prendrez-vous ces pistolets, ces tromblons, ces fusils à deux coups dont vous voulez farcir votre voiture? 

—Le fait est que ce ne sera pas dans mon arsenal, dit-il, car à la Terracine, on m'a pris jusqu'à mon couteau poignard; et à vous? 

—À moi, on m'en a fait autant à Aqua-Pendente. 

—Ah çà! mon cher hôte, dit Albert en allumant son second cigare au reste de son premier, savez-vous que c'est très commode pour les voleurs cette mesure-là, et qu'elle m'a tout l'air d'avoir été prise de compte à demi avec eux?» 

Sans doute maître Pastrini trouva la plaisanterie compromettante, car il n'y répondit qu'à moitié et encore en adressant la parole à Franz, comme au seul être raisonnable avec lequel il pût convenablement s'entendre. 

«Son Excellence sait que ce n'est pas l'habitude de se défendre quand on est attaqué par des bandits. 

—Comment! s'écria Albert, dont le courage se révoltait à l'idée de se laisser dévaliser sans rien dire; comment! ce n'est pas l'habitude? 

—Non, car toute défense serait inutile. Que voulez-vous faire contre une douzaine de bandits qui sortent d'un fossé, d'une masure ou d'un aqueduc, et qui vous couchent en joue tous à la fois? 

—Eh sacrebleu! je veux me faire tuer!» s'écria Albert. 

L'aubergiste se tourna vers Franz d'un air qui voulait dire: Décidément, Excellence, votre camarade est fou. 

«Mon cher Albert, reprit Franz, votre réponse est sublime, et vaut le \textit{Qu'il mourût} du vieux Corneille: seulement, quand Horace répondait cela, il s'agissait du salut de Rome, et la chose en valait la peine. Mais quant à nous, remarquez qu'il s'agit simplement d'un caprice à satisfaire, et qu'il serait ridicule, pour un caprice, de risquer notre vie. 

—Ah! \textit{per Bacco}! s'écria maître Pastrini, à la bonne heure, voilà ce qui s'appelle parler.» 

Albert se versa un verre de \textit{lacryma Christi}, qu'il but à petits coups, en grommelant des paroles inintelligibles. 

«Eh bien, maître Pastrini, reprit Franz, maintenant que voilà mon compagnon calmé, et que vous avez pu apprécier mes dispositions pacifiques, maintenant, voyons qu'est-ce que le seigneur Luigi Vampa? Est-il berger ou patricien? est-il jeune ou vieux? est-il petit ou grand? Dépeignez-nous le, afin que si nous le rencontrions par hasard dans le monde, comme Jean Sbogar ou Lara, nous puissions au moins le reconnaître. 

—Vous ne pouvez pas mieux vous adresser qu'à moi, Excellence, pour avoir des détails exacts, car j'ai connu Luigi Vampa tout enfant; et, un jour que j'étais tombé moi-même entre ses mains, en allant de Ferentino à Alatri, il se souvint, heureusement pour moi, de notre ancienne connaissance; il me laissa aller, non seulement sans me faire payer de rançon, mais encore après m'avoir fait cadeau d'une fort belle montre et m'avoir raconté son histoire. 

—Voyons la montre», dit Albert. 

Maître Pastrini tira de son gousset une magnifique Breguet portant le nom de son auteur, le timbre de Paris et une couronne de comte. 

«Voilà, dit-il. 

—Peste! fit Albert, je vous en fais mon compliment; j'ai la pareille à peu près—il tira sa montre de la poche de son gilet—et elle m'a coûté trois mille francs. 

—Voyons l'histoire, dit Franz à son tour, en tirant un fauteuil et en faisant signe à maître Pastrini de s'asseoir. 

—Leurs Excellences permettent? dit l'hôte. 

—Pardieu! dit Albert, vous n'êtes pas un prédicateur, mon cher, pour parler debout.» 

L'hôtelier s'assit, après avoir fait à chacun de ses futurs auditeurs un salut respectueux, lequel avait pour but d'indiquer qu'il était prêt à leur donner sur Luigi Vampa les renseignements qu'ils demandaient. 

«Ah çà, fit Franz, arrêtant maître Pastrini au moment où il ouvrait la bouche, vous dites que vous avez connu Luigi Vampa tout enfant; c'est donc encore un jeune homme? 

—Comment, un jeune homme! je crois bien; il a vingt-deux ans à peine! Oh! c'est un gaillard qui ira loin, soyez tranquille! 

—Que dites-vous de cela, Albert? c'est beau, à vingt-deux ans, de s'être déjà fait une réputation, dit Franz. 

—Oui, certes, et, à son âge, Alexandre, César et Napoléon, qui depuis ont fait un certain bruit dans le monde, n'étaient pas si avancés que lui. 

—Ainsi, reprit Franz, s'adressant à son hôte, le héros dont nous allons entendre l'histoire n'a que vingt-deux ans.  

—À peine, comme j'ai eu l'honneur de vous le dire. 

—Est-il grand ou petit? 

—De taille moyenne: à peu près comme Son Excellence, dit l'hôte en montrant Albert. 

—Merci de la comparaison, dit celui-ci en s'inclinant. 

—Allez toujours, maître Pastrini, reprit Franz, souriant de la susceptibilité de son ami. Et à quelle classe de la société appartenait-il? 

—C'était un simple petit pâtre attaché à la ferme du comte de San-Felice, située entre Palestrina et le lac de Gabri. Il était né à Pampinara, et était entré à l'âge de cinq ans au service du comte. Son père, berger lui-même à Anagni, avait un petit troupeau à lui; et vivait de la laine de ses moutons et de la récolte faite avec le lait de ses brebis, qu'il venait vendre à Rome. 

«Tout enfant, le petit Vampa avait un caractère étrange. Un jour, à l'âge de sept ans, il était venu trouver le curé de Palestrina, et l'avait prié de lui apprendre à lire. C'était chose difficile; car le jeune pâtre ne pouvait pas quitter son troupeau. Mais le bon curé allait tous les jours dire la messe dans un pauvre petit bourg trop peu considérable pour payer un prêtre, et qui, n'ayant pas même de nom, était connu sous celui dell'Borgo. Il offrit à Luigi de se trouver sur son chemin à l'heure de son retour et de lui donner ainsi sa leçon, le prévenant que cette leçon serait courte et qu'il eût par conséquent à en profiter. 

«L'enfant accepta avec joie. 

«Tous les jours, Luigi menait paître son troupeau sur la route de Palestrina au Borgo; tous les jours, à neuf heures du matin, le curé passait, le prêtre et l'enfant s'asseyaient sur le revers d'un fossé, et le petit pâtre prenait sa leçon dans le bréviaire du curé. 

«Au bout de trois mois, il savait lire. 

«Ce n'était pas tout, il lui fallait maintenant apprendre à écrire. 

«Le prêtre fit faire par un professeur d'écriture de Rome trois alphabets: un en gros, un en moyen, et un en fin, et il lui montra qu'en suivant cet alphabet sur une ardoise il pouvait, à l'aide d'une pointe de fer, apprendre à écrire. 

«Le même soir, lorsque le troupeau fut rentré à la ferme, le petit Vampa courut chez le serrurier de Palestrina, prit un gros clou, le forgea, le martela, l'arrondit, et en fit une espèce de stylet antique. 

«Le lendemain, il avait réuni une provision d'ardoises et se mettait à l'œuvre. 

«Au bout de trois mois, il savait écrire. 

«Le curé, étonné de cette profonde intelligence et touché de cette aptitude, lui fit cadeau de plusieurs cahiers de papier, d'un paquet de plumes et d'un canif. 

«Ce fut une nouvelle étude à faire, mais étude qui n'était rien auprès de la première. Huit jours après, il maniait la plume comme il maniait le stylet. 

«Le curé raconta cette anecdote au comte de San-Felice, qui voulut voir le petit pâtre, le fit lire et écrire devant lui, ordonna à son intendant de le faire manger avec les domestiques, et lui donna deux piastres par mois. 

«Avec cet argent, Luigi acheta des livres et des crayons. 

«En effet, il avait appliqué à tous les objets cette facilité d'imitation qu'il avait, et, comme Giotto enfant, il dessinait sur ses ardoises ses brebis, les arbres, les maisons. 

«Puis, avec la pointe de son canif, il commença à tailler le bois et à lui donner toutes sortes de formes. C'est ainsi que Pinelli, le sculpteur populaire, avait commencé. 

«Une jeune fille de six ou sept ans, c'est-à-dire un peu plus jeune que Vampa, gardait de son côté les brebis dans une ferme voisine de Palestrina; elle était orpheline, née à Valmontone, et s'appelait Teresa. 

«Les deux enfants se rencontraient, s'asseyaient l'un près de l'autre, laissaient leurs troupeaux se mêler et paître ensemble, causaient, riaient et jouaient puis, le soir, on démêlait les moutons du comte de San-Felice d'avec ceux du baron de Cervetri, et les enfants se quittaient pour revenir à leur ferme respective, en se promettant de se retrouver le lendemain matin. 

«Le lendemain ils tenaient parole, et grandissaient ainsi côte à côte. 

«Vampa atteignit douze ans, et la petite Teresa onze. 

«Cependant, leurs instincts naturels se développaient. 

«À côté du goût des arts que Luigi avait poussé aussi loin qu'il le pouvait faire dans l'isolement, il était triste par boutade, ardent par secousse, colère par caprice, railleur toujours. Aucun des jeunes garçons de Pampinara, de Palestrina ou de Valmontone n'avait pu non seulement prendre aucune influence sur lui, mais encore devenir son compagnon. Son tempérament volontaire, toujours disposé à exiger sans jamais vouloir se plier à aucune concession, écartait de lui tout mouvement amical, toute démonstration sympathique. Teresa seule commandait d'un mot, d'un regard, d'un geste à ce caractère entier qui pliait sous la main d'une femme, et qui, sous celle de quelque homme que ce fût, se serait raidi jusqu'à rompre. 

«Teresa était, au contraire, vive, alerte et gaie, mais coquette à l'excès, les deux piastres que donnait à Luigi l'intendant du comte de San-Felice, le prix de tous les petits ouvrages sculptés qu'il vendait aux marchands de joujoux de Rome passaient en boucles d'oreilles de perles, en colliers de verre, en aiguilles d'or. Aussi, grâce à cette prodigalité de son jeune ami, Teresa était-elle la plus belle et la plus élégante paysanne des environs de Rome. 

«Les deux enfants continuèrent à grandir, passant toutes leurs journées ensemble, et se livrant sans combat aux instincts de leur nature primitive. Aussi, dans leurs conversations, dans leurs souhaits, dans leurs rêves, Vampa se voyait toujours capitaine de vaisseau, général d'armée ou gouverneur d'une province; Teresa se voyait riche, vêtue des plus belles robes et suivie de domestiques en livrée, puis, quand ils avaient passé toute la journée à broder leur avenir de ces folles et brillantes arabesques, ils se séparaient pour ramener chacun leurs moutons dans leur étable, et redescendre, de la hauteur de leurs songes, à l'humilité de leur position réelle. 

«Un jour, le jeune berger dit à l'intendant du comte qu'il avait vu un loup sortir des montagnes de la Sabine et rôder autour de son troupeau. L'intendant lui donna un fusil: c'est ce que voulait Vampa. 

«Ce fusil se trouva par hasard être un excellent canon de Brescia, portant la balle comme une carabine anglaise; seulement un jour le comte, en assommant un renard blessé, en avait cassé la crosse et l'on avait jeté le fusil au rebut. 

«Cela n'était pas une difficulté pour un sculpteur comme Vampa. Il examina la couche primitive, calcula ce qu'il fallait y changer pour la mettre à son coup d'œil, et fit une autre crosse chargée d'ornements si merveilleux que, s'il eût voulu aller vendre à la ville le bois seul, il en eût certainement tiré quinze ou vingt piastres. 

«Mais il n'avait garde d'agir ainsi: un fusil avait longtemps été le rêve du jeune homme. Dans tous les pays où l'indépendance est substituée à la liberté, le premier besoin qu'éprouve tout cœur fort, toute organisation puissante, est celui d'une arme qui assure en même temps l'attaque et la défense, et qui faisant celui qui la porte terrible, le fait souvent redouté. 

«À partir de ce moment, Vampa donna tous les instants qui lui restèrent à l'exercice du fusil; il acheta de la poudre et des balles, et tout lui devint un but: le tronc de l'olivier, triste, chétif et gris, qui pousse au versant des montagnes de la Sabine; le renard qui, le soir, sortait de son terrier pour commencer sa chasse nocturne, et l'aigle qui planait dans l'air. Bientôt il devint si adroit, que Teresa surmontait la crainte qu'elle avait éprouvée d'abord en entendant la détonation, et s'amusa à voir son jeune compagnon placer la balle de son fusil où il voulait la mettre, avec autant de justesse que s'il l'eût poussée avec la main. 

«Un soir, un loup sortit effectivement d'un bois de sapins près duquel les deux jeunes gens avaient l'habitude de demeurer: le loup n'avait pas fait dix pas en plaine qu'il était mort. 

«Vampa, tout fier de ce beau coup, le chargea sur ses épaules et le rapporta à la ferme. 

«Tous ces détails donnaient à Luigi une certaine réputation aux alentours de la ferme; l'homme supérieur partout où il se trouve, se crée une clientèle d'admirateurs. On parlait dans les environs de ce jeune pâtre comme du plus adroit, du plus fort et du plus brave contadino qui fût à dix lieues à la ronde; et quoique de son côté Teresa, dans un cercle plus étendu encore, passât pour une des plus jolies filles de la Sabine, personne ne s'avisait de lui dire un mot d'amour, car on la savait aimée par Vampa. 

«Et cependant les deux jeunes gens ne s'étaient jamais dit qu'ils s'aimaient. Ils avaient poussé l'un à côté de l'autre comme deux arbres qui mêlent leurs racines sous le sol, leurs branches dans l'air, leur parfum dans le ciel; seulement leur désir de se voir était le même; ce désir était devenu un besoin, et ils comprenaient plutôt la mort qu'une séparation d'un seul jour. 

«Teresa avait seize ans et Vampa dix-sept. 

«Vers ces temps, on commença de parler beaucoup d'une bande de brigands qui s'organisait dans les monts Lepini. Le brigandage n'a jamais été sérieusement extirpé dans le voisinage de Rome. Il manque de chefs parfois, mais quand un chef se présente, il est rare qu'il lui manque une bande. 

«Le célèbre Cucumetto, traqué dans les Abruzzes chassé du royaume de Naples, où il avait soutenu une véritable guerre, avait traversé Garigliano comme Manfred, et était venu entre Sonnino et Juperno se réfugier sur les bords de l'Amasine. 

«C'était lui qui s'occupait à réorganiser une troupe, et qui marchait sur les traces de Decesaris et de Gasparone, qu'il espérait bientôt surpasser. Plusieurs jeunes gens de Palestrina, de Frascati et de Pampinara disparurent. On s'inquiéta d'eux d'abord puis bientôt on sut qu'ils étaient allés rejoindre la bande de Cucumetto. 

«Au bout de quelque temps, Cucumetto devint l'objet de l'attention générale. On citait de ce chef de bandits des traits d'audace extraordinaires et de brutalité révoltante. 

«Un jour, il enleva une jeune fille: c'était la fille de l'arpenteur de Frosinone. Les lois des bandits sont positives: une jeune fille est à celui qui l'enlève d'abord, puis les autres la tirent au sort, et la malheureuse sert aux plaisirs de toute la troupe jusqu'à ce que les bandits l'abandonnent ou qu'elle meure. 

«Lorsque les parents sont assez riches pour la racheter, on envoie un messager qui traite de la rançon; la tête de la prisonnière répond de la sécurité de l'émissaire. Si la rançon est refusée, la prisonnière est condamnée irrévocablement. 

«La jeune fille avait son amant dans la troupe de Cucumetto: il s'appelait Carlini. 

«En reconnaissant le jeune homme, elle tendit les bras vers lui et se crut sauvée. Mais le pauvre Carlini, en la reconnaissant, lui, sentit son cœur se briser, car il se doutait bien du sort qui attendait sa maîtresse. 

«Cependant, comme il était le favori de Cucumetto, comme il avait partagé ses dangers depuis trois ans, comme il lui avait sauvé la vie en abattant d'un coup de pistolet un carabinier qui avait déjà le sabre levé sur sa tête, il espéra que Cucumetto aurait quelque pitié de lui. 

«Il prit donc le chef à part, tandis que la jeune fille, assise contre le tronc d'un grand pin qui s'élevait au milieu d'une clairière de la forêt, s'était fait un voile de la coiffure pittoresque des paysannes romaines et cachait son visage aux regards luxurieux des bandits. 

«Là, il lui raconta tout, ses amours avec la prisonnière, leurs serments de fidélité, et comment chaque nuit, depuis qu'ils étaient dans les environs, ils se donnaient rendez-vous dans une ruine. 

«Ce soir-là justement, Cucumetto avait envoyé Carlini dans un village voisin, il n'avait pu se trouver au rendez-vous; mais Cucumetto s'y était trouvé par hasard, disait-il, et c'est alors qu'il avait enlevé la jeune fille. 

«Carlini supplia son chef de faire une exception en sa faveur et de respecter Rita, lui disant que le père était riche et qu'il payerait une bonne rançon. 

«Cucumetto parut se rendre aux prières de son ami, et le chargea de trouver un berger qu'on pût envoyer chez le père de Rita à Frosinone. 

«Alors Carlini s'approcha tout joyeux de la jeune fille, lui dit qu'elle était sauvée, et l'invita à écrire à son père une lettre dans laquelle elle racontait ce qui lui était arrivé, et lui annoncerait que sa rançon était fixée à trois cents piastres. 

«On donnait pour tout délai au père douze heures, c'est-à-dire jusqu'au lendemain neuf heures du matin. 

«La lettre écrite, Carlini s'en empara aussitôt et courut dans la plaine pour chercher un messager. 

«Il trouva un jeune pâtre qui parquait son troupeau. Les messagers naturels des bandits sont les bergers, qui vivent entre la ville et la montagne, entre la vie sauvage et la vie civilisée. 

«Le jeune berger partit aussitôt, promettant d'être avant une heure à Frosinone. 

«Carlini revint tout joyeux pour rejoindre sa maîtresse et lui annoncer cette bonne nouvelle. 

«Il trouva la troupe dans la clairière, où elle soupait joyeusement des provisions que les bandits levaient sur les paysans comme un tribut seulement; au milieu de ces gais convives, il chercha vainement Cucumetto et Rita. 

«Il demanda où ils étaient, les bandits répondirent par un grand éclat de rire. Une sueur froide coula sur le front de Carlini, et il sentit l'angoisse qui le prenait aux cheveux. 

«Il renouvela sa question. Un des convives remplit un verre de vin d'Orvieto et le lui tendit en disant:  

«—À la santé du brave Cucumetto et de la belle Rita! 

«En ce moment, Carlini crut entendre un cri de femme. Il devina tout. Il prit le verre, le brisa sur la face de celui qui le lui présentait, et s'élança dans la direction du cri. 

«Au bout de cent pas, au détour d'un buisson, il trouva Rita évanouie entre les bras de Cucumetto. 

«En apercevant Carlini, Cucumetto se releva tenant un pistolet de chaque main. 

«Les deux bandits se regardèrent un instant: l'un le sourire de la luxure sur les lèvres, l'autre la pâleur de la mort sur le front. 

«On eût cru qu'il allait se passer entre ces deux hommes quelque chose de terrible. Mais peu à peu les traits de Carlini se détendirent, sa main, qu'il avait portée à un des pistolets de sa ceinture, retomba près de lui pendante à son côté. 

«Rita était couchée entre eux deux. 

«La lune éclairait cette scène. 

«—Eh bien, lui dit Cucumetto, as-tu fait la commission dont tu t'étais chargé? 

«—Oui, capitaine, répondit Carlini, et demain, avant neuf heures, le père de Rita sera ici avec l'argent. 

«—À merveille. En attendant, nous allons passer une joyeuse nuit. Cette jeune fille est charmante, et tu as, en vérité, bon goût, maître Carlini. Aussi comme je ne suis pas égoïste nous allons retourner auprès des camarades et tirer au sort à qui elle appartiendra maintenant. 

«—Ainsi vous êtes décidé à l'abandonner à la loi commune? demanda Carlini. 

«—Et pourquoi ferait-on exception en sa faveur? 

«—J'avais cru qu'à ma prière\dots. 

«—Et qu'es-tu plus que les autres? 

«—C'est juste. 

«—Mais sois tranquille, reprit Cucumetto en riant, un peu plus tôt, un peu plus tard, ton tour viendra. 

«Les dents de Carlini se serraient à se briser. 

«—Allons, dit Cucumetto en faisant un pas vers les convives, viens-tu? 

«—Je vous suis\dots. 

«Cucumetto s'éloigna sans perdre de vue Carlini, car sans doute il craignait qu'il ne le frappât par derrière. Mais rien dans le bandit ne dénonçait une intention hostile. 

«Il était debout, les bras croisés, près de Rita toujours évanouie. 

«Un instant, l'idée de Cucumetto fut que le jeune homme allait la prendre dans ses bras et fuir avec elle. Mais peu lui importait maintenant, il avait eu de Rita ce qu'il voulait; et quant à l'argent, trois cents piastres réparties à la troupe faisaient une si pauvre somme qu'il s'en souciait médiocrement. 

«Il continua donc sa route vers la clairière; mais, à son grand étonnement, Carlini y arriva presque aussitôt que lui. 

«—Le tirage au sort! le tirage au sort! crièrent tous les bandits en apercevant le chef. 

«Et les yeux de tous ces hommes brillèrent d'ivresse et de lascivité, tandis que la flamme du foyer jetait sur toute leur personne une lueur rougeâtre qui les faisait ressembler à des démons. 

«Ce qu'ils demandaient était juste; aussi le chef fit-il de la tête un signe annonçant qu'il acquiesçait à leur demande. On mit tous les noms dans un chapeau, celui de Carlini comme ceux des autres, et le plus jeune de la bande tira de l'urne improvisée un bulletin. 

«Ce bulletin portait le nom de Diavolaccio. 

«C'était celui-là même qui avait proposé à Carlini la santé du chef, et à qui Carlini avait répondu en lui brisant le verre sur la figure. 

«Une large blessure ouverte de la tempe à la bouche, laissait couler le sang à flots. 

«Diavolaccio, se voyant ainsi favorisé de la fortune, poussa un éclat de rire. 

«—Capitaine, dit-il, tout à l'heure Carlini n'a pas voulu boire à votre santé, proposez-lui de boire à la mienne; il aura peut-être plus de condescendance pour vous que pour moi.» 

«Chacun s'attendait à une explosion de la part de Carlini; mais au grand étonnement de tous, il prit un verre d'une main, un fiasco de l'autre, puis, remplissant le verre: 

«—À ta santé, Diavolaccio, dit-il d'une voix parfaitement calme. 

«Et il avala le contenu du verre sans que sa main tremblât. Puis, s'asseyant près du feu: 

«—Ma part de souper! dit-il; la course que je viens de faire m'a donné de l'appétit. 

«—Vive Carlini! s'écrièrent les brigands. 

«—À la bonne heure, voilà ce qui s'appelle prendre la chose en bon compagnon.  

«Et tous reformèrent le cercle autour du foyer, tandis que Diavolaccio s'éloignait. 

«Carlini mangeait et buvait, comme si rien ne s'était passé. 

«Les bandits le regardaient avec étonnement, ne comprenant rien à cette impassibilité, lorsqu'ils entendirent derrière eux retentir sur le sol un pas alourdi. 

«Ils se retournèrent et aperçurent Diavolaccio tenant la jeune fille entre ses bras. 

«Elle avait la tête renversée, et ses longs cheveux pendaient jusqu'à terre. 

«À mesure qu'ils entraient dans le cercle de la lumière projetée par le foyer, on s'apercevait de la pâleur de la jeune fille et de la pâleur du bandit. 

«Cette apparition avait quelque chose de si étrange et de si solennel, que chacun se leva, excepté Carlini, qui resta assis et continua de boire et de manger, comme si rien ne se passait autour de lui. 

«Diavolaccio continuait de s'avancer au milieu du plus profond silence, et déposa Rita aux pieds du capitaine. 

«Alors tout le monde put reconnaître la cause de cette pâleur de la jeune fille et de cette pâleur du bandit: Rita avait un couteau enfoncé jusqu'au manche au-dessous de la mamelle gauche. 

«Tous les yeux se portèrent sur Carlini: la gaine était vide à sa ceinture. 

«—Ah! ah! dit le chef, je comprends maintenant pourquoi Carlini était resté en arrière. 

«Toute nature sauvage est apte à apprécier une action forte; quoique peut-être aucun des bandits n'eût fait ce que venait de faire Carlini, tous comprirent ce qu'il avait fait. 

«—Eh bien, dit Carlini en se levant à son tour et en s'approchant du cadavre, la main sur la crosse d'un de ses pistolets, y a-t-il encore quelqu'un qui me dispute cette femme? 

«—Non, dit le chef, elle est à toi!» 

«Alors Carlini la prit à son tour dans ses bras, et l'emporta hors du cercle de lumière que projetait la flamme du foyer. 

«Cucumetto disposa les sentinelles comme d'habitude, et les bandits se couchèrent, enveloppés dans leurs manteaux, autour du foyer. 

«À minuit, la sentinelle donna l'éveil, et en un instant le chef et ses compagnons furent sur pied. 

«C'était le père de Rita, qui arrivait lui-même, portant la rançon de sa fille.  

«—Tiens, dit-il à Cucumetto en lui tendant un sac d'argent, voici trois cents pistoles, rends-moi mon enfant. 

«Mais le chef, sans prendre l'argent, lui fit signe de le suivre. Le vieillard obéit; tous deux s'éloignèrent sous les arbres, à travers les branches desquels filtraient les rayons de la lune. Enfin Cucumetto s'arrêta étendant la main et montrant au vieillard deux personnes groupées au pied d'un arbre: 

«—Tiens, lui dit-il, demande ta fille à Carlini, c'est lui qui t'en rendra compte. 

«Et il s'en retourna vers ses compagnons. 

«Le vieillard resta immobile et les yeux fixes. Il sentait que quelque malheur inconnu, immense, inouï, planait sur sa tête. 

«Enfin, il fit quelques pas vers le groupe informe dont il ne pouvait se rendre compte. 

«Au bruit qu'il faisait en s'avançant vers lui, Carlini releva la tête, et les formes des deux personnages commencèrent à apparaître plus distinctes aux yeux du vieillard. 

«Une femme était couchée à terre, la tête posée sur les genoux d'un homme assis et qui se tenait penché vers elle; c'était en se relevant que cet homme avait découvert le visage de la femme qu'il tenait serrée contre sa poitrine. 

«Le vieillard reconnut sa fille, et Carlini reconnut le vieillard. 

«—Je t'attendais, dit le bandit au père de Rita. 

«—Misérable! dit le vieillard, qu'as-tu fait? 

«Et il regardait avec terreur Rita, pâle, immobile, ensanglantée, avec un couteau dans la poitrine. 

«Un rayon de la lune frappait sur elle et l'éclairait de sa lueur blafarde. 

«—Cucumetto avait violé ta fille, dit le bandit, et, comme je l'aimais, je l'ai tuée; car, après lui, elle allait servir de jouet à toute la bande. 

«Le vieillard ne prononça point une parole, seulement il devint pâle comme un spectre. 

«—Maintenant, dit Carlini, si j'ai eu tort, venge-la. 

«Et il arracha le couteau du sein de la jeune fille et, se levant, il l'alla offrir d'une main au vieillard tandis que de l'autre il écartait sa veste et lui présentait sa poitrine nue. 

«—Tu as bien fait, lui dit le vieillard d'une voix sourde. Embrasse-moi, mon fils. 

«Carlini se jeta en sanglotant dans les bras du père de sa maîtresse. C'étaient les premières larmes que versait cet homme de sang. 

«—Maintenant, dit le vieillard à Carlini, aide-moi à enterrer ma fille. 

«Carlini alla chercher deux pioches, et le père et l'amant se mirent à creuser la terre au pied d'un chêne dont les branches touffues devaient recouvrir la tombe de la jeune fille. 

«Quand la tombe fut creusée, le père l'embrassa le premier, l'amant ensuite; puis, l'un la prenant par les pieds, l'autre par-dessous les épaules, ils la descendirent dans la fosse. 

«Puis ils s'agenouillèrent des deux côtés et dirent les prières des morts. 

«Puis, lorsqu'ils eurent fini, ils repoussèrent la terre sur le cadavre jusqu'à ce que la fosse fût comblée. 

«Alors, lui tendant la main: 

«—Je te remercie, mon fils! dit le vieillard à Carlini; maintenant, laisse-moi seul. 

«—Mais cependant\dots dit celui-ci. 

«—Laisse-moi, je te l'ordonne. 

«Carlini obéit, alla rejoindre ses camarades, s'enveloppa dans son manteau, et bientôt parut aussi profondément endormi que les autres. 

«Il avait été décidé la veille que l'on changerait de campement. 

«Une heure avant le jour Cucumetto éveilla ses hommes et l'ordre fut donné de partir. 

«Mais Carlini ne voulut pas quitter la forêt sans savoir ce qu'était devenu le père de Rita. 

«Il se dirigea vers l'endroit où il l'avait laissé. 

«Il trouva le vieillard pendu à une des branches du chêne qui ombrageait la tombe de sa fille. 

«Il fit alors sur le cadavre de l'un et sur la fosse de l'autre le serment de les venger tous deux. 

«Mais il ne put tenir ce serment; car, deux jours après dans une rencontre avec les carabiniers romains, Carlini fut tué. 

«Seulement, on s'étonna que, faisant face à l'ennemi, il eût reçu une balle entre les deux épaules. 

«L'étonnement cessa quand un des bandits eut fait remarquer à ses camarades que Cucumetto était placé dix pas en arrière de Carlini lorsque Carlini était tombé. 

«Le matin du départ de la forêt de Frosinone, il avait suivi Carlini dans l'obscurité, avait entendu le serment qu'il avait fait, et, en homme de précaution, il avait pris l'avance. 

«On racontait encore sur ce terrible chef de bande dix autres histoires non moins curieuses que celle-ci. 

«Ainsi, de Fondi à Pérouse, tout le monde tremblait au seul nom de Cucumetto. 

«Ces histoires avaient souvent été l'objet des conversations de Luigi et de Teresa. 

«La jeune fille tremblait fort à tous ces récits; mais Vampa la rassurait avec un sourire, frappant son bon fusil, qui portait si bien la balle; puis, si elle n'était pas rassurée, il lui montrait à cent pas quelque corbeau perché sur une branche morte, le mettait en joue, lâchait la détente, et l'animal, frappé, tombait au pied de l'arbre. 

«Néanmoins, le temps s'écoulait: les deux jeunes gens avaient arrêté qu'ils se marieraient lorsqu'ils auraient, Vampa vingt ans, et Teresa dix-neuf. 

«Ils étaient orphelins tous deux; ils n'avaient de permission à demander qu'à leur maître; ils l'avaient demandée et obtenue. 

«Un jour qu'ils causaient de leur projet d'avenir, ils entendirent deux ou trois coups de feu; puis tout à coup un homme sortit du bois près duquel les deux jeunes gens avaient l'habitude de faire paître leurs troupeaux, et accourut vers eux. 

«Arrivé à la portée de la voix: 

«—Je suis poursuivi! leur cria-t-il; pouvez-vous me cacher? 

«Les deux jeunes gens reconnurent bien que ce fugitif devait être quelque bandit; mais il y a entre le paysan et le bandit romain une sympathie innée qui fait que le premier est toujours prêt à rendre service au second. 

«Vampa, sans rien dire, courut donc à la pierre qui bouchait l'entrée de leur grotte, démasqua cette entrée en tirant la pierre à lui, fit signe au fugitif de se réfugier dans cet asile inconnu de tous, repoussa la pierre sur lui et revint s'asseoir près de Teresa. 

«Presque aussitôt, quatre carabiniers à cheval apparurent à la lisière du bois; trois paraissaient être à la recherche du fugitif, le quatrième traînait par le cou un bandit prisonnier. 

«Les trois carabiniers explorèrent le pays d'un coup d'œil, aperçurent les deux jeunes gens, accoururent à eux au galop, et les interrogèrent. 

«Ils n'avaient rien vu. 

«—C'est fâcheux, dit le brigadier, car celui que nous cherchons, c'est le chef. 

«—Cucumetto? ne purent s'empêcher de s'écrier ensemble Luigi et Teresa. 

«—Oui, répondit le brigadier; et comme sa tête est mise à prix à mille écus romains, il y en aurait eu cinq cents pour vous si vous nous aviez aidés à le prendre. 

«Les deux jeunes gens échangèrent un regard. Le brigadier eut un instant d'espérance. Cinq cents écus romains font trois mille francs, et trois mille francs sont une fortune pour deux pauvres orphelins qui vont se marier. 

«—Oui, c'est fâcheux, dit Vampa, mais nous ne l'avons pas vu. 

«Alors les carabiniers battirent le pays dans des directions différentes, mais inutilement. 

«Puis, successivement, ils disparurent. 

«Alors Vampa alla tirer la pierre, et Cucumetto sortit. 

«Il avait vu, à travers les jours de la porte de granit, les deux jeunes gens causer avec les carabiniers; il s'était douté du sujet de leur conversation, il avait lu sur le visage de Luigi et de Teresa l'inébranlable résolution de ne point le livrer et tira de sa poche une bourse pleine d'or et la leur offrit. 

«Mais Vampa releva la tête avec fierté; quant à Teresa, ses yeux brillèrent en pensant à tout ce qu'elle pourrait acheter de riches bijoux et beaux habits avec cette bourse pleine d'or. 

«Cucumetto était un Satan fort habile: il avait pris la forme d'un bandit au lieu de celle d'un serpent; il surprit ce regard, reconnut dans Teresa une digne fille d'Ève, et rentra dans la forêt en se retournant plusieurs fois sous prétexte de saluer ses libérateurs. 

«Plusieurs jours s'écoulèrent sans que l'on revit Cucumetto, sans qu'on entendit reparler de lui. 

«Le temps du carnaval approchait. Le comte de San-Felice annonça un grand bal masqué où tout ce que Rome avait de plus élégant fut invité. 

«Teresa avait grande envie de voir ce bal. Luigi demanda à son protecteur l'intendant la permission pour elle et pour lui d'y assister cachés parmi les serviteurs de la maison. Cette permission lui fut accordée. 

«Ce bal était surtout donné par le comte pour faire plaisir à sa fille Carmela, qu'il adorait. 

«Carmela était juste de l'âge et de la taille de Teresa, et Teresa était au moins aussi belle que Carmela. 

«Le soir du bal, Teresa mit sa plus belle toilette, ses plus riches aiguilles, ses plus brillantes verroteries. Elle avait le costume des femmes de Frascati. 

«Luigi avait l'habit si pittoresque du paysan romain les jours de fête. 

«Tous deux se mêlèrent, comme on l'avait permis, aux serviteurs et aux paysans. 

«La fête était magnifique. Non seulement la villa était ardemment illuminée, mais des milliers de lanternes de couleur étaient suspendues aux arbres du jardin. Aussi bientôt le palais eut-il débordé sur les terrasses et les terrasses dans les allées. 

«À chaque carrefour il y avait un orchestre, des buffets et des rafraîchissements; les promeneurs s'arrêtaient, les quadrilles se formaient et l'on dansait là où il plaisait de danser. 

«Carmela était vêtue en femme de Sonino. Elle avait son bonnet tout brodé de perles, les aiguilles de ses cheveux étaient d'or et de diamants, sa ceinture était de soie turque à grandes fleurs brochées, son surtout et son jupon étaient de cachemire, son tablier était de mousseline des Indes; les boutons de son corset étaient autant de pierreries. 

«Deux autres de ses compagnes étaient vêtues, l'une en femme de Nettuno, l'autre en femme de la Riccia. 

«Quatre jeunes gens des plus riches et des plus nobles familles de Rome les accompagnaient avec cette liberté italienne qui a son égale dans aucun autre pays du monde: ils étaient vêtus de leur côté en paysans d'Albano, de Velletri, de Civita-Castellana et de Sora. 

«Il va sans dire que ces costumes de paysans, comme ceux de paysannes, étaient resplendissant d'or et de pierreries. 

«Il vint à Carmela l'idée de faire un quadrille uniforme, seulement il manquait une femme. 

«Carmela regardait tout autour d'elle, pas une de ses invitées n'avait un costume analogue au sien et à ceux de ses compagnes. 

«Le comte San-Felice lui montra, au milieu des paysannes, Teresa appuyée au bras de Luigi. 

«—Est-ce que vous permettez, mon père? dit Carmela. 

«—Sans doute, répondit le comte, ne sommes-nous pas en carnaval!  

«Carmela se pencha vers un jeune homme qui l'accompagnait en causant, et lui dit quelques mots tout en lui montrant du doigt la jeune fille. 

«Le jeune homme suivit des yeux la jolie main qui lui servait de conductrice, fit un geste d'obéissance et vint inviter Teresa à figurer au quadrille dirigé par la fille du comte. 

«Teresa sentit comme une flamme qui lui passait sur le visage. Elle interrogea du regard Luigi: il n'y avait pas moyen de refuser. Luigi laissa lentement glisser le bras de Teresa, qu'il tenait sous le sien, et Teresa, s'éloignant conduite par son élégant cavalier, vint prendre, toute tremblante, sa place au quadrille aristocratique.  

«Certes, aux yeux d'un artiste, l'exact et sévère costume de Teresa eût eu un bien autre caractère que celui de Carmela et des ses compagnes, mais Teresa était une jeune fille frivole et coquette; les broderies de la mousseline, les palmes de la ceinture, l'éclat du cachemire l'éblouissaient, le reflet des saphirs et des diamants la rendaient folle. 

«De son côté Luigi sentait naître en lui un sentiment inconnu: c'était comme une douleur sourde qui le mordait au cœur d'abord, et de là, toute frémissante, courait par ses veines et s'emparait de tout son corps; il suivit des yeux les moindres mouvements de Teresa et de son cavalier; lorsque leurs mains se touchaient il ressentait comme des éblouissements, ses artères battaient avec violence, et l'on eût dit que le son d'une cloche vibrait à ses oreilles. Lorsqu'ils se parlaient, quoique Teresa écoutât, timide et les yeux baissés, les discours de son cavalier, comme Luigi lisait dans les yeux ardents du beau jeune homme que ces discours étaient des louanges, il lui semblait que la terre tournait sous lui et que toutes les voix de l'enfer lui soufflaient des idées de meurtre et d'assassinat. Alors, craignant de se laisser emporter à sa folie, il se cramponnait d'une main à la charmille contre laquelle il était debout, et de l'autre il serrait d'un mouvement convulsif le poignard au manche sculpté qui était passé dans sa ceinture et que, sans s'en apercevoir, il tirait quelquefois presque entier du fourreau. 

«Luigi était jaloux! il sentait qu'emportée par sa nature coquette et orgueilleuse Teresa pouvait lui échapper.  

«Et cependant la jeune paysanne, timide et presque effrayée d'abord, s'était bientôt remise. Nous avons dit que Teresa était belle. Ce n'est pas tout, Teresa était gracieuse, de cette grâce sauvage bien autrement puissante que notre grâce minaudière et affectée. 

«Elle eut presque les honneurs du quadrille, et si elle fut envieuse de la fille du comte de San-Felice, nous n'oserions pas dire que Carmela ne fut pas jalouse d'elle. 

«Aussi fût-ce avec force compliments que son beau cavalier la reconduisit à la place où il l'avait prise, et où l'attendait Luigi.  

«Deux ou trois fois, pendant la contredanse, la jeune fille avait jeté un regard sur lui, et à chaque fois elle l'avait vu pâle et les traits crispés. Une fois même la lame de son couteau, à moitié tirée de sa gaine, avait ébloui ses yeux comme un sinistre éclair. 

«Ce fut donc presque en tremblant qu'elle reprit le bras de son amant. 

«Le quadrille avait eu le plus grand succès, et il était évident qu'il était question d'en faire une seconde édition; Carmela seule s'y opposait; mais le comte de San-Felice pria sa fille si tendrement, qu'elle finit par consentir. 

«Aussitôt un des cavaliers s'avança pour inviter Teresa, sans laquelle il était impossible que la contredanse eût lieu; mais la jeune fille avait déjà disparu. 

«En effet, Luigi ne s'était pas senti la force de supporter une seconde épreuve; et, moitié par persuasion, moitié par force, il avait entraîné Teresa vers un autre point du jardin. Teresa avait cédé bien malgré elle; mais elle avait vu à la figure bouleversée du jeune homme, elle comprenait à son silence entrecoupé de tressaillements nerveux, que quelque chose d'étrange se passait en lui. Elle-même n'était pas exempte d'une agitation intérieure, et sans avoir cependant rien fait de mal, elle comprenait que Luigi était en droit de lui faire des reproches: sur quoi? elle l'ignorait; mais elle ne sentait pas moins que ces reproches seraient mérités.  

«Cependant, au grand étonnement de Teresa, Luigi demeura muet, et pas une parole n'entrouvrit ses lèvres pendant tout le reste de la soirée. Seulement, lorsque le froid de la nuit eut chassé les invités des jardins et que les portes de la villa se furent refermées sur eux pour une fête intérieure, il reconduisit Teresa; puis, comme elle allait rentrer chez elle: 

«—Teresa, dit-il, à quoi pensais-tu lorsque tu dansais en face de la jeune comtesse de San-Felice? 

«—Je pensais, répondit la jeune fille dans toute la franchise de son âme, que je donnerais la moitié de ma vie pour avoir un costume comme celui qu'elle portait. 

«—Et que te disait ton cavalier? 

«—Il me disait qu'il ne tiendrait qu'à moi de l'avoir, et que je n'avais qu'un mot à dire pour cela. 

«—Il avait raison, répondit Luigi. Le désires-tu aussi ardemment que tu le dis? 

«—Oui. 

«—Eh bien tu l'auras! 

«La jeune fille, étonnée, leva la tête pour le questionner; mais son visage était si sombre et si terrible que la parole se glaça sur ses lèvres. 

«D'ailleurs, en disant ces paroles, Luigi s'était éloigné. 

«Teresa le suivit des yeux dans la nuit tant qu'elle put l'apercevoir. Puis, lorsqu'il eut disparu, elle rentra chez elle en soupirant. 

«Cette même nuit, il arriva un grand événement par l'imprudence sans doute de quelque domestique qui avait négligé d'éteindre les lumières; le feu prit à la villa San-Felice, juste dans les dépendances de l'appartement de la belle Carmela. Réveillée au milieu de la nuit par la lueur des flammes, elle avait sauté au bas de son lit, s'était enveloppée de sa robe de chambre, et avait essayé de fuir par la porte; mais le corridor par lequel il fallait passer était déjà la proie de l'incendie. Alors elle était rentrée dans sa chambre, appelant à grands cris du secours, quand tout à coup sa fenêtre, située à vingt pieds du sol, s'était ouverte; un jeune paysan s'était élancé dans l'appartement, l'avait prise dans ses bras, et, avec une force et une adresse surhumaines l'avait transportée sur le gazon de la pelouse, où elle s'était évanouie. Lorsqu'elle avait repris ses sens, son père était devant elle. Tous les serviteurs l'entouraient, lui portant des secours. Une aile tout entière de la villa était brûlée; mais qu'importait, puisque Carmela était saine et sauve. 

«On chercha partout son libérateur, mais son libérateur ne reparut point; on le demanda à tout le monde, mais personne ne l'avait vu. Quant à Carmela, elle était si troublée qu'elle ne l'avait point reconnu. 

«Au reste, comme le comte était immensément riche, à part le danger qu'avait couru Carmela, et qui lui parut, par la manière miraculeuse dont elle y avait échappé, plutôt une nouvelle faveur de la Providence qu'un malheur réel, la perte occasionnée par les flammes fut peu de chose pour lui. 

«Le lendemain, à l'heure habituelle, les deux jeunes gens se retrouvèrent à la lisière de la forêt. Luigi était arrivé le premier. Il vint au-devant de la jeune fille avec une grande gaieté; il semblait avoir complètement oublié la scène de la veille. Teresa était visiblement pensive, mais en voyant Luigi ainsi disposé, elle affecta de son côté l'insouciance rieuse qui était le fond de son caractère quand quelque passion ne le venait pas troubler. 

«Luigi prit le bras de Teresa sous le sien, et la conduisit jusqu'à la porte de la grotte. Là il s'arrêta. La jeune fille, comprenant qu'il y avait quelque chose d'extraordinaire, le regarda fixement. 

«—Teresa, dit Luigi, hier soir tu m'as dit que tu donnerais tout au monde pour avoir un costume pareil à celui de la fille du comte? 

«—Oui, dit Teresa, avec étonnement, mais j'étais folle de faire un pareil souhait. 

«—Et moi, je t'ai répondu: C'est bien, tu l'auras. 

«—Oui, reprit la jeune fille, dont l'étonnement croissait à chaque parole de Luigi; mais tu as répondu cela sans doute pour me faire plaisir. 

«—Je ne t'ai jamais rien promis que je ne te l'aie donné, Teresa, dit orgueilleusement Luigi; entre dans la grotte et habille-toi. 

«À ces mots, il tira la pierre, et montra à Teresa la grotte éclairée par deux bougies qui brûlaient de chaque côté d'un magnifique miroir; sur la table rustique, faite par Luigi, étaient étalés le collier de perles et les épingles de diamants; sur une chaise à côté était déposé le reste du costume.  

«Teresa poussa un cri de joie, et, sans s'informer d'où venait ce costume, sans prendre le temps de remercier Luigi, elle s'élança dans la grotte transformée en cabinet de toilette. 

«Derrière elle Luigi repoussa la pierre, car il venait d'apercevoir, sur la crête d'une petite colline qui empêchait que de la place où il était on ne vît Palestrina, un voyageur à cheval, qui s'arrêta un instant comme incertain de sa route, se dessinant sur l'azur du ciel avec cette netteté de contour particulière aux lointains des pays méridionaux. 

«En apercevant Luigi, le voyageur mit son cheval au galop, et vint à lui. 

«Luigi ne s'était pas trompé; le voyageur, qui allait de Palestrina à Tivoli, était dans le doute de son chemin. 

«Le jeune homme le lui indiqua; mais, comme à un quart de mille de là la route se divisait en trois sentiers, et qu'arrivé à ces trois sentiers le voyageur pouvait de nouveau s'égarer, il pria Luigi de lui servir de guide. 

«Luigi détacha son manteau et le déposa à terre, jeta sur son épaule sa carabine, et, dégagé ainsi du lourd vêtement, marcha devant le voyageur de ce pas rapide du montagnard que le pas d'un cheval a peine à suivre. 

«En dix minutes, Luigi et le voyageur furent à l'espèce de carrefour indiqué par le jeune pâtre. 

«Arrivés là, d'un geste majestueux comme celui d'un empereur, il étendit la main vers celle des trois routes que le voyageur devait suivre: 

«—Voilà votre chemin, dit-il, Excellence, vous n'avez plus à vous tromper maintenant. 

«—Et toi, voici ta récompense, dit le voyageur en offrant au jeune pâtre quelques pièces de menue monnaie. 

«—Merci, dit Luigi en retirant sa main; je rends un service, je ne le vends pas. 

«—Mais», dit le voyageur, qui paraissait du reste habitué à cette différence entre la servilité de l'homme des villes et l'orgueil du campagnard, «si tu refuses un salaire, tu acceptes au moins un cadeau. 

«—Ah! oui, c'est autre chose. 

«—Eh bien, dit le voyageur, prends ces deux sequins de Venise, et donne-les à ta fiancée pour en faire une paire de boucles d'oreilles. 

«—Et vous, alors, prenez ce poignard, dit le jeune pâtre, vous n'en trouveriez pas un dont la poignée fût mieux sculptée d'Albano à Civita-Castellana.  

«—J'accepte, dit le voyageur; mais alors, c'est moi qui suis ton obligé, car ce poignard vaut plus de deux sequins. 

«—Pour un marchand peut-être, mais pour moi, qui l'ai sculpté moi-même, il vaut à peine une piastre. 

«—Comment t'appelles-tu? demanda le voyageur. 

«—Luigi Vampa, répondit le pâtre du même air qu'il eût répondu: Alexandre, roi de Macédoine. Et vous? 

«—Moi, dit le voyageur, je m'appelle Simbad le marin.» 

Franz d'Épinay jeta un cri de surprise.  

«Simbad le marin! dit-il. 

—Oui, reprit le narrateur, c'est le nom que le voyageur donna à Vampa comme étant le sien. 

—Eh bien, mais, qu'avez-vous à dire contre ce nom? interrompit Albert; c'est un fort beau nom, et les aventures du patron de ce monsieur m'ont, je dois l'avouer, fort amusé dans ma jeunesse.» 

Franz n'insista pas davantage. Ce nom de Simbad le marin, comme on le comprend bien, avait réveillé en lui tout un monde de souvenirs, comme avait fait la veille celui du comte de Monte-Cristo.  

«Continuez, dit-il à l'hôte. 

—Vampa mit dédaigneusement les deux sequins dans sa poche, et reprit lentement le chemin par lequel il était venu. Arrivé à deux ou trois cents pas de la grotte, il crut entendre un cri. 

«Il s'arrêta, écoutant de quel côté venait ce cri. 

«Au bout d'une seconde, il entendit son nom prononcé distinctement. 

«L'appel venait du côté de la grotte. 

«Il bondit comme un chamois, armant son fusil tout en courant, et parvint en moins d'une minute au sommet de la colline opposée à celle où il avait aperçu le voyageur. 

«Là, les cris: Au secours! arrivèrent à lui plus distincts. 

«Il jeta les yeux sur l'espace qu'il dominait; un homme enlevait Teresa, comme le centaure Nessus Déjanire. 

«Cet homme, qui se dirigeait vers le bois, était déjà aux trois quarts du chemin de la grotte à la forêt. 

«Vampa mesura l'intervalle; cet homme avait deux cents pas d'avance au moins sur lui, il n'y avait pas de chance de le rejoindre avant qu'il eût gagné le bois.  

«Le jeune pâtre s'arrêta comme si ses pieds eussent pris racine. Il appuya la crosse de son fusil à l'épaule, leva lentement le canon dans la direction du ravisseur, le suivit une seconde dans sa course et fit feu. 

«Le ravisseur s'arrêta court; ses genoux plièrent et il tomba entraînant Teresa dans sa chute. 

«Mais Teresa se releva aussitôt; quant au fugitif, il resta couché, se débattant dans les convulsions de l'agonie. 

«Vampa s'élança aussitôt vers Teresa, car à dix pas du moribond les jambes lui avaient manqué à son tour, et elle était retombée à genoux: le jeune homme avait cette crainte terrible que la balle qui venait d'abattre son ennemi n'eût en même temps blessé sa fiancée. 

«Heureusement il n'en était rien, c'était la terreur seule qui avait paralysé les forces de Teresa. Lorsque Luigi se fut bien assuré qu'elle était saine et sauve, il se retourna vers le blessé. 

«Il venait d'expirer les poings fermés, la bouche contractée par la douleur, et les cheveux hérissés sous la sueur de l'agonie. 

«Ses yeux étaient restés ouverts et menaçants. 

«Vampa s'approcha du cadavre, et reconnut Cucumetto.  

«Depuis le jour où le bandit avait été sauvé par les deux jeunes gens, il était devenu amoureux de Teresa et avait juré que la jeune fille serait à lui. Depuis ce jour il l'avait épiée; et, profitant du moment où son amant l'avait laissée seule pour indiquer le chemin au voyageur, il l'avait enlevée et la croyait déjà à lui, lorsque la balle de Vampa, guidée par le coup d'œil infaillible du jeune pâtre, lui avait traversé le cœur. 

«Vampa le regarda un instant sans que la moindre émotion se trahît sur son visage, tandis qu'au contraire Teresa, toute tremblante encore, n'osait se rapprocher du bandit mort qu'à petits pas, et jetait en hésitant un coup d'œil sur le cadavre par-dessus l'épaule de son amant. 

«Au bout d'un instant, Vampa se retourna vers sa maîtresse: 

«—Ah! ah! dit-il, c'est bien, tu es habillée; à mon tour de faire ma toilette. 

«En effet, Teresa était revêtue de la tête aux pieds du costume de la fille du comte de San-Felice. 

«Vampa prit le corps de Cucumetto entre ses bras, l'emporta dans la grotte, tandis qu'à son tour Teresa restait dehors. 

«Si un second voyageur fût alors passé, il eût vu une chose étrange: c'était une bergère gardant ses brebis avec une robe de cachemire, des boucles d'oreilles et un collier de perles, des épingles de diamants et des boutons de saphirs, d'émeraudes et de rubis. 

«Sans doute, il se fût cru revenu au temps de Florian, et eût affirmé, en revenant à Paris, qu'il avait rencontré la bergère des Alpes assise au pied des monts Sabins. 

«Au bout d'un quart d'heure, Vampa sortit à son tour de la grotte. Son costume n'était pas moins élégant, dans son genre, que celui de Teresa. 

«Il avait une veste de velours grenat à boutons d'or ciselé, un gilet de soie tout couvert de broderies, une écharpe romaine nouée autour du cou, une cartouchière toute piquée d'or et de soie rouge et verte; des culottes de velours bleu de ciel attachées au-dessous du genou par des boucles de diamants, des guêtres de peau de daim bariolées de mille arabesques, et un chapeau où flottaient des rubans de toutes couleurs; deux montres pendaient à sa ceinture, et un magnifique poignard était passé à sa cartouchière. 

«Teresa jeta un cri d'admiration. Vampa, sous cet habit, ressemblait à une peinture de Léopold Robert ou de Schnetz. 

«Il avait revêtu le costume complet de Cucumetto. 

«Le jeune homme s'aperçut de l'effet qu'il produisait sur sa fiancée, et un sourire d'orgueil passa sur sa bouche. 

«—Maintenant, dit-il à Teresa, es-tu prête à partager ma fortune quelle qu'elle soit? 

«—Oh oui! s'écria la jeune fille avec enthousiasme. 

«—À me suivre partout où j'irai? 

«—Au bout du monde. 

«—Alors, prends mon bras et partons, car nous n'avons pas de temps à perdre.» 

«La jeune fille passa son bras sous celui de son amant, sans même lui demander où il la conduisait; car, en ce moment, il lui paraissait beau, fier et puissant comme un dieu. 

«Et tous deux s'avancèrent dans la forêt, dont au bout de quelques minutes, ils eurent franchi la lisière. 

«Il va sans dire que tous les sentiers de la montagne étaient connus de Vampa; il avança donc dans la forêt sans hésiter un seul instant, quoiqu'il n'y eût aucun chemin frayé, mais seulement reconnaissant la route qu'il devait suivre à la seule inspection des arbres et des buissons; ils marchèrent ainsi une heure et demie à peu près. 

«Au bout de ce temps, ils étaient arrivés à l'endroit le plus touffu du bois. Un torrent dont le lit était à sec conduisait dans une gorge profonde. Vampa prit cet étrange chemin, qui, encaissé entre deux rives et rembruni par l'ombre épaisse des pins, semblait, moins la descente facile, ce sentier de l'Averne dont parle Virgile. 

«Teresa, redevenue craintive à l'aspect de ce lieu sauvage et désert, se serrait contre son guide, sans dire une parole; mais comme elle le voyait marcher toujours d'un pas égal, comme un calme profond rayonnait sur son visage, elle avait elle-même la force de dissimuler son émotion. 

«Tout à coup, à dix pas d'eux, un homme sembla se détacher d'un arbre derrière lequel il était caché, et mettait Vampa en joue: 

«—Pas un pas de plus! cria-t-il, ou tu es mort. 

«—Allons donc», dit Vampa en levant la main avec un geste de mépris; tandis que Teresa, ne dissimulant plus sa terreur, se pressait contre lui, «est-ce que les loups se déchirent entre eux! 

«—Qui es-tu? demanda la sentinelle. 

«—Je suis Luigi Vampa, le berger de la ferme de San-Felice. 

«—Que veux-tu? 

«—Je veux parler à tes compagnons qui sont à la clairière de Rocca Bianca. 

«—Alors, suis-moi, dit la sentinelle, ou plutôt, puisque tu sais où cela est, marche devant. 

«Vampa sourit d'un air de mépris à cette précaution du bandit, passa devant avec Teresa et continua son chemin du même pas ferme et tranquille qui l'avait conduit jusque-là. 

«Au bout de cinq minutes, le bandit leur fit signe de s'arrêter. 

«Les deux jeunes gens obéirent. 

«Le bandit imita trois fois le cri du corbeau. 

«Un croassement répondit à ce triple appel. 

«—C'est bien, dit le bandit. Maintenant tu peux continuer ta route.»  

«Luigi et Teresa se remirent en chemin. 

«Mais à mesure qu'ils avançaient, Teresa, tremblante se serrait contre son amant; en effet, à travers les arbres, on voyait apparaître des armes et étinceler des canons de fusil. 

«La clairière de Rocca Bianca était au sommet d'une petite montagne qui autrefois sans doute avait été un volcan, volcan éteint avant que Rémus et Romulus eussent déserté Albe pour venir bâtir Rome. 

«Teresa et Luigi atteignirent le sommet et se trouvèrent au même instant en face d'une vingtaine de bandits. 

«—Voici un jeune homme qui vous cherche et qui désire vous parler, dit la sentinelle. 

«—Et que veut-il nous dire? demanda celui qui, en l'absence du chef, faisait l'intérim du capitaine. 

«—Je veux dire que je m'ennuie de faire le métier de berger, dit Vampa. 

«—Ah! je comprends, dit le lieutenant, et tu viens nous demander à être admis dans nos rangs? 

«—Qu'il soit le bienvenu! crièrent plusieurs bandits de Ferrusino, de Pampinara et d'Anagni, qui avaient reconnu Luigi Vampa. 

«—Oui, seulement je viens vous demander une autre chose que d'être votre compagnon. 

«—Et que viens-tu nous demander? dirent les bandits avec étonnement. 

«—Je viens vous demander à être votre capitaine, dit le jeune homme. 

«Les bandits éclatèrent de rire. 

«—Et qu'as-tu fait pour aspirer à cet honneur? demanda le lieutenant.  

«—J'ai tué votre chef Cucumetto, dont voici la dépouille, dit Luigi, et j'ai mis le feu à la villa de San-Felice pour donner une robe de noce à ma fiancée. 

«Une heure après, Luigi Vampa était élu capitaine en remplacement de Cucumetto. 

—Eh bien, mon cher Albert, dit Franz en se retournant vers son ami, que pensez-vous maintenant du citoyen Luigi Vampa? 

—Je dis que c'est un mythe, répondit Albert, et qu'il n'a jamais existé. 

—Qu'est-ce que c'est qu'un mythe? demanda Pastrini. 

—Ce serait trop long à vous expliquer, mon cher hôte, répondit Franz. Et vous dites donc que maître Vampa exerce en ce moment sa profession aux environs de Rome? 

—Et avec une hardiesse dont jamais bandit avant lui n'avait donné l'exemple. 

—La police a tenté vainement de s'en emparer, alors? 

—Que voulez-vous! il est d'accord à la fois avec les bergers de la plaine, les pêcheurs du Tibre et les contrebandiers de la côte. On le cherche dans la montagne, il est sur le fleuve; on le poursuit sur le fleuve, il gagne la pleine mer; puis tout à coup, quand on le croit réfugié dans l'île del Giglio, del Guanouti ou de Monte-Cristo, on le voit reparaître à Albano, à Tivoli ou à la Riccia. 

—Et quelle est sa manière de procéder à l'égard des voyageurs? 

—Ah! mon Dieu! c'est bien simple. Selon la distance où l'on est de la ville, il leur donne huit heures, douze heures, un jour, pour payer leur rançon; puis, ce temps écoulé, il accorde une heure de grâce. À la soixantième minute de cette heure, s'il n'a pas l'argent, il fait sauter la cervelle du prisonnier d'un coup de pistolet, ou lui plante son poignard dans le cœur, et tout est dit. 

—Eh bien, Albert, demanda Franz à son compagnon, êtes-vous toujours disposé à aller au Colisée par les boulevards extérieurs? 

—Parfaitement, dit Albert, si la route est plus pittoresque.» 

En ce moment, neuf heures sonnèrent, la porte s'ouvrit et notre cocher parut. 

«Excellences, dit-il, la voiture vous attend. 

—Eh bien, dit Franz, en ce cas, au Colisée! 

—Par la porte del Popolo, Excellences, ou par les rues? 

—Par les rues, morbleu! par les rues! s'écria Franz.  

—Ah! mon cher! dit Albert en se levant à son tour et en allumant son troisième cigare, en vérité, je vous croyais plus brave que cela.» 

Sur ce, les deux jeunes gens descendirent l'escalier et montèrent en voiture. 