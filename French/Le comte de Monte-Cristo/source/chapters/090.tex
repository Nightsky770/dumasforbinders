\chapter{La rencontre}

\lettrine{A}{près} le départ de Mercédès, tout retomba dans l'ombre chez Monte-Cristo. Autour de lui et au-dedans de lui sa pensée s'arrêta; son esprit énergique s'endormit comme fait le corps après une suprême fatigue. 

«Quoi! se disait-il, tandis que la lampe et les bougies se consumaient tristement et que les serviteurs attendaient avec impatience dans l'antichambre; quoi! voilà l'édifice si lentement préparé, élevé avec tant de peines et de soucis, écroulé d'un seul coup, avec un seul mot, sous un souffle! Eh quoi! ce moi que je croyais quelque chose, ce moi dont j'étais si fier, ce moi que j'avais vu si petit dans les cachots du château d'If, et que j'avais su rendre si grand, sera demain un peu de poussière! Hélas! ce n'est point la mort du corps que je regrette: cette destruction du principe vital n'est-elle point le repos où tout tend, où tout malheureux aspire, ce calme de la matière après lequel j'ai soupiré si longtemps, au-devant duquel je m'acheminais par la route douloureuse de la faim, quand Faria est apparu dans mon cachot? Qu'est-ce que la mort? Un degré de plus dans le calme et deux peut-être dans le silence. Non, ce n'est donc pas l'existence que je regrette, c'est la ruine de mes projets si lentement élaborés, si laborieusement bâtis. La Providence, que j'avais crue pour eux, était donc contre eux. Dieu ne voulait donc pas qu'ils s'accomplissent! 

«Ce fardeau que j'ai soulevé, presque aussi pesant qu'un monde, et que j'avais cru porter jusqu'au bout, était selon mon désir et non selon ma force; selon ma volonté et non selon mon pouvoir, et il me le faudra déposer à peine à moitié de ma course. Oh! je redeviendrai donc fataliste, moi que quatorze ans de désespoir et dix ans d'espérance avaient rendu providentiel. 

«Et tout cela, mon Dieu! parce que mon cœur, que je croyais mort, n'était qu'engourdi; parce qu'il s'est réveillé, parce qu'il a battu, parce que j'ai cédé à la douleur de ce battement soulevé du fond de ma poitrine par la voix d'une femme! 

«Et cependant, continua le comte, s'abîmant de plus en plus dans les prévisions de ce lendemain terrible qu'avait accepté Mercédès; cependant il est impossible que cette femme, qui est un si noble cœur, ait ainsi, par égoïsme, consenti à me laisser tuer, moi plein de force et d'existence! Il est impossible qu'elle pousse à ce point l'amour, ou plutôt le délire maternel! Il y a des vertus dont l'exagération serait un crime. Non, elle aura imaginé quelque scène pathétique, elle viendra se jeter entre les épées, et ce sera ridicule sur le terrain, de sublime que c'était ici.» 

Et la rougeur de l'orgueil montait au front du comte. 

«Ridicule, répéta-t-il, et le ridicule rejaillira sur moi\dots Moi, ridicule! Allons! j'aime encore mieux mourir.» 

Et à force de s'exagérer ainsi d'avance les mauvaises chances du lendemain, auxquelles il s'était condamné en promettant à Mercédès de laisser vivre son fils, le comte s'en vint à se dire: 

«Sottise, sottise, sottise! que faire ainsi de la générosité en se plaçant comme un but inerte au bout du pistolet de ce jeune homme! Jamais il ne croira que ma mort est un suicide, et cependant il importe pour l'honneur de ma mémoire\dots (ce n'est point de la vanité, n'est-ce pas, mon Dieu? mais bien un juste orgueil, voilà tout), il importe pour l'honneur de ma mémoire que le monde sache que j'ai consenti moi-même, par ma volonté, de mon libre arbitre, à arrêter mon bras déjà levé pour frapper, et que de ce bras, si puissamment armé contre les autres, je me suis frappé moi-même: il le faut, je le ferai.» 

Et saisissant une plume, il tira un papier de l'armoire secrète de son bureau, et traça au bas de ce papier, qui n'était autre chose que son testament fait depuis son arrivée à Paris, une espèce de codicille dans lequel il faisait comprendre sa mort aux gens les moins clairvoyants. 

«Je fais cela, mon Dieu! dit-il les yeux levés au ciel, autant pour votre honneur que pour le mien. Je me suis considéré, depuis dix ans, ô mon Dieu! comme l'envoyé de votre vengeance, et il ne faut pas que d'autres misérables que ce Morcerf, il ne faut pas qu'un Danglars, un Villefort, il ne faut pas enfin que ce Morcerf lui-même se figurent que le hasard les a débarrassés de leur ennemi. Qu'ils sachent, au contraire, que la Providence, qui avait déjà décrété leur punition, a été corrigée par la seule puissance de ma volonté, que le châtiment évité dans ce monde les attend dans l'autre, et qu'ils n'ont échangé le temps que contre l'éternité.» 

Tandis qu'il flottait entre ces sombres incertitudes, mauvais rêve de l'homme éveillé par la douleur, le jour vint blanchir les vitres et éclairer sous ses mains le pâle papier azur sur lequel il venait de tracer cette suprême justification de la Providence. 

Il était cinq heures du matin. 

Tout à coup un léger bruit parvint à son oreille. Monte-Cristo crut avoir entendu quelque chose comme un soupir étouffé; il tourna la tête, regarda autour de lui et ne vit personne. Seulement le bruit se répéta assez distinct pour qu'au doute succédât la certitude. 

Alors le comte se leva, ouvrit doucement la porte du salon, et sur un fauteuil, les bras pendants, sa belle tête pâle inclinée en arrière, il vit Haydée qui s'était placée en travers de la porte, afin qu'il ne pût sortir sans la voir, mais que le sommeil, si puissant contre la jeunesse, avait surprise après la fatigue d'une si longue veille. 

Le bruit que la porte fit en s'ouvrant ne put tirer Haydée de son sommeil. 

Monte-Cristo arrêta sur elle un regard plein de douceur et de regret. 

«Elle s'est souvenue qu'elle avait un fils, dit-il, et moi, j'ai oublié que j'avais une fille! 

Puis, secouant tristement la tête: 

«Pauvre Haydée! dit-elle, elle a voulu me voir, elle a voulu me parler, elle a craint ou deviné quelque chose\dots Oh! je ne puis partir sans lui dire adieu, je ne puis mourir sans la confier à quelqu'un.» 

Et il regagna doucement sa place et écrivit au bas des premières lignes: 

«Je lègue à Maximilien Morrel, capitaine de spahis et fils de mon ancien patron, Pierre Morrel, armateur à Marseille, la somme de vingt millions, dont une partie sera offerte par lui à sa sœur Julie et à son beau-frère Emmanuel, s'il ne croit pas toutefois que ce surplus de fortune doive nuire à leur bonheur. Ces vingt millions sont enfouis dans ma grotte de Monte-Cristo, dont Bertuccio sait le secret. 

«Si son cœur est libre et qu'il veuille épouser Haydée, fille d'Ali, pacha de Janina, que j'ai élevée avec l'amour d'un père et qui a eu pour moi la tendresse d'une fille, il accomplira, je ne dirai point ma dernière volonté, mais mon dernier désir. 

«Le présent testament a déjà fait Haydée héritière du reste de ma fortune, consistant en terres, rentes sur l'Angleterre, l'Autriche et la Hollande, mobilier dans mes différents palais et maisons, et qui, ces vingt millions prélevés, ainsi que les différents legs faits à mes serviteurs, pourront monter encore à soixante millions.» 

Il achevait d'écrire cette dernière ligne, lorsqu'un cri poussé derrière lui, lui fit tomber la plume des mains. 

«Haydée, dit-il, vous avez lu?» 

En effet, la jeune femme, réveillée par le jour qui avait frappé ses paupières, s'était levée et s'était approchée du comte sans que ses pas légers, assourdis par le tapis, eussent été entendus. 

«Oh! mon seigneur, dit-elle en joignant les mains, pourquoi écrivez-vous ainsi à une pareille heure? Pourquoi me léguez-vous toute votre fortune, mon seigneur? Vous me quittez donc? 

—Je vais faire un voyage, cher ange, dit Monte-Cristo avec une expression de mélancolie et de tendresse infinies, et s'il m'arrivait malheur\dots» 

Le comte s'arrêta. 

«Eh bien?\dots demanda la jeune fille avec un accent d'autorité que le comte ne lui connaissait point et qui le fit tressaillir. 

—Eh bien, s'il m'arrive malheur, reprit Monte-Cristo, je veux que ma fille soit heureuse.» 

Haydée sourit tristement en secouant la tête. 

«Vous pensez à mourir, mon seigneur? dit-elle. 

—C'est une pensée salutaire, mon enfant, a dit le sage. 

—Eh bien, si vous mourez, dit-elle, léguez votre fortune à d'autres, car, si vous mourez\dots je n'aurai plus besoin de rien.» 

Et prenant le papier, elle le déchira en quatre morceaux qu'elle jeta au milieu du salon. Puis, cette énergie si peu habituelle à une esclave ayant épuisé ses forces, elle tomba, non plus endormie cette fois, mais évanouie sur le parquet. 

Monte-Cristo se pencha vers elle, la souleva entre ses bras; et, voyant ce beau teint pâli, ces beaux yeux fermés, ce beau corps inanimé et comme abandonné, l'idée lui vint pour la première fois qu'elle l'aimait peut-être autrement que comme une fille aime son père. 

«Hélas! murmura-t-il avec un profond découragement, j'aurais donc encore pu être heureux!» 

Puis il porta Haydée jusqu'à son appartement, la remit, toujours évanouie, aux mains de ses femmes; et, rentrant dans son cabinet, qu'il ferma cette fois vivement sur lui, il recopia le testament détruit. 

Comme il achevait, le bruit d'un cabriolet entrant dans la cour se fit entendre. Monte-Cristo s'approcha de la fenêtre et vit descendre Maximilien et Emmanuel. 

«Bon, dit-il, il était temps!» 

Et il cacheta son testament d'un triple cachet. 

Un instant après il entendit un bruit de pas dans le salon, et alla ouvrir lui-même. Morrel parut sur le seuil. 

Il avait devancé l'heure de près de vingt minutes. 

«Je viens trop tôt peut-être, monsieur le comte dit-il, mais je vous avoue franchement que je n'ai pu dormir une minute, et qu'il en a été de même de toute la maison. J'avais besoin de vous voir fort de votre courageuse assurance pour redevenir moi-même.» 

Monte-Cristo ne put tenir à cette preuve d'affection et ce ne fut point la main qu'il tendit au jeune homme mais ses deux bras qu'il lui ouvrit. 

«Morrel, lui dit-il d'une voix émue, c'est un beau jour pour moi que celui où je me sens aimé d'un homme comme vous. Bonjour, monsieur Emmanuel. Vous venez donc avec moi, Maximilien? 

—Pardieu! dit le jeune capitaine, en aviez-vous douté? 

—Mais cependant si j'avais tort\dots 

—Écoutez, je vous ai regardé hier pendant toute cette scène de provocation, j'ai pensé à votre assurance toute cette nuit, et je me suis dit que la justice devait être pour vous, ou qu'il n'y avait plus aucun fond à faire sur le visage des hommes. 

—Cependant, Morrel, Albert est votre ami. 

—Une simple connaissance, comte. 

—Vous l'avez vu pour la première fois le jour même que vous m'avez vu? 

—Oui, c'est vrai; que voulez-vous? il faut que vous me le rappeliez pour que je m'en souvienne. 

—Merci, Morrel.» 

Puis, frappant un coup sur le timbre: 

«Tiens, dit-il à Ali qui apparut aussitôt, fais porter cela chez mon notaire. C'est mon testament, Morrel. Moi mort, vous irez en prendre connaissance. 

—Comment! s'écria Morrel, vous mort? 

—Eh! ne faut-il pas tout prévoir, cher ami? Mais qu'avez-vous fait hier après m'avoir quitté? 

—J'ai été chez Tortoni, où, comme je m'y attendais, j'ai trouvé Beauchamp et Château-Renaud. Je vous avoue que je les cherchais. 

—Pour quoi faire, puisque tout cela était convenu? 

—Écoutez, comte, l'affaire est grave, inévitable. 

—En doutiez-vous? 

—Non. L'offense a été publique, et chacun en parlait déjà. 

—Eh bien? 

—Eh bien, j'espérais faire changer les armes, substituer l'épée au pistolet. Le pistolet est aveugle. 

—Avez-vous réussi? demanda vivement Monte-Cristo avec une imperceptible lueur d'espoir. 

—Non, car on connaît votre force à l'épée. 

—Bah! qui m'a donc trahi? 

—Les maîtres d'armes que vous avez battus. 

—Et vous avez échoué? 

—Ils ont refusé positivement. 

—Morrel, dit le comte, m'avez-vous jamais vu tirer le pistolet? 

—Jamais. 

—Eh bien, nous avons le temps, regardez.» 

Monte-Cristo prit les pistolets qu'il tenait quand Mercédès était entrée, et collant un as de trèfle contre la plaque, en quatre coups il enleva successivement les quatre branches du trèfle. 

À chaque coup Morrel pâlissait. 

Il examina les balles avec lesquelles Monte-Cristo exécutait ce tour de force, et il vit qu'elles n'étaient pas plus grosses que des chevrotines. 

«C'est effrayant, dit-il; voyez donc, Emmanuel!» 

Puis, se retournant vers Monte-Cristo: 

«Comte, dit-il, au nom du Ciel, ne tuez pas Albert! le malheureux a une mère! 

—C'est juste, dit Monte-Cristo, et, moi, je n'en ai pas.» 

Ces mots furent prononcés avec un ton qui fit frissonner Morrel. 

«Vous êtes l'offensé, comte. 

—Sans doute; qu'est-ce que cela veut dire? 

—Cela veut dire que vous tirez le premier. 

—Je tire le premier? 

—Oh! cela, je l'ai obtenu ou plutôt exigé; nous leur faisons assez de concessions pour qu'ils nous fissent celles-là. 

—Et à combien de pas? 

—À vingt.» 

Un effrayant sourire passa sur les lèvres du comte. 

«Morrel, dit-il, n'oubliez pas ce que vous venez de voir. 

—Aussi, dit le jeune homme, je ne compte que sur votre émotion pour sauver Albert. 

—Moi, ému? dit Monte-Cristo. 

—Ou sur votre générosité, mon ami; sûr de votre coup comme vous l'êtes, je puis vous dire une chose qui serait ridicule si je la disais à un autre. 

—Laquelle? 

—Cassez-lui un bras, blessez-le, mais ne le tuez pas. 

—Morrel, écoutez encore ceci, dit le comte, je n'ai pas besoin d'être encouragé à ménager M. de Morcerf; M. de Morcerf, je vous l'annonce d'avance, sera si bien ménagé qu'il reviendra tranquillement avec ses deux amis tandis que moi\dots 

—Eh bien, vous? 

—Oh! c'est autre chose, on me rapportera, moi. 

—Allons donc! s'écria Maximilien hors de lui. 

—C'est comme je vous l'annonce, mon cher Morrel, M. de Morcerf me tuera.» 

Morrel regarda le comte en homme qui ne comprend plus. 

«Que vous est-il donc arrivé depuis hier soir, comte? 

—Ce qui est arrivé à Brutus la veille de la bataille de Philippes: j'ai vu un fantôme. 

—Et ce fantôme? 

—Ce fantôme, Morrel, m'a dit que j'avais assez vécu.» 

Maximilien et Emmanuel se regardèrent; Monte-Cristo tira sa montre. 

«Partons, dit-il, il est sept heures cinq minutes, et le rendez-vous est pour huit heures juste.» 

Une voiture attendait toute attelée; Monte-Cristo y monta avec ses deux témoins. 

En traversant le corridor, Monte-Cristo s'était arrêté pour écouter devant une porte, et Maximilien et Emmanuel, qui, par discrétion, avaient fait quelques pas en avant, crurent entendre répondre à un sanglot par un soupir. 

À huit heures sonnantes on était au rendez-vous. 

«Nous voici arrivés, dit Morrel en passant la tête par la portière, et nous sommes les premiers. 

—Monsieur m'excusera, dit Baptistin qui avait suivi son maître avec une terreur indicible, mais je crois apercevoir là-bas une voiture sous les arbres. 

—En effet, dit Emmanuel, j'aperçois deux jeunes gens qui se promènent et semblent attendre.» 

Monte-Cristo sauta légèrement en bas de sa calèche et donna la main à Emmanuel et à Maximilien pour les aider à descendre. 

Maximilien retint la main du comte entre les siennes. 

«À la bonne heure, dit-il, voici une main comme j'aime la voir à un homme dont la vie repose dans la bonté de sa cause.» 

Monte-Cristo tira Morrel, non pas à part, mais d'un pas ou deux en arrière de son beau-frère. 

«Maximilien, lui demanda-t-il, avez-vous le cœur libre?» 

Morrel regarda Monte-Cristo avec étonnement. 

«Je ne vous demande pas une confidence, cher ami, je vous adresse une simple question; répondez oui ou non, c'est tout ce que je vous demande. 

—J'aime une jeune fille, comte. 

—Vous l'aimez beaucoup? 

—Plus que ma vie. 

—Allons, dit Monte-Cristo, voilà encore une espérance qui m'échappe.» 

Puis, avec un soupir: 

«Pauvre Haydée! murmura-t-il. 

—En vérité, comte! s'écria Morrel, si je vous connaissais moins, je vous croirais moins brave que vous n'êtes! 

—Parce que je pense à quelqu'un que je vais quitter, et que je soupire! Allons donc, Morrel, est-ce à un soldat de se connaître si mal en courage? est-ce que c'est la vie que je regrette? Qu'est-ce que cela me fait à moi, qui ai passé vingt ans entre la vie et la mort, de vivre ou de mourir? D'ailleurs, soyez tranquille, Morrel, cette faiblesse, si c'en est une, est pour vous seul. Je sais que le monde est un salon dont il faut sortir poliment et honnêtement, c'est-à-dire en saluant et en payant ses dettes de jeu. 

—À la bonne heure, dit Morrel, voilà qui est parler. À propos, avez-vous apporté vos armes? 

—Moi! pour quoi faire? J'espère bien que ces messieurs auront les leurs. 

—Je vais m'en informer, dit Morrel. 

—Oui, mais pas de négociations, vous m'entendez? 

—Oh! soyez tranquille.» 

Morrel s'avança vers Beauchamp et Château-Renaud. Ceux-ci, voyant le mouvement de Maximilien, firent quelques pas au-devant de lui. 

Les trois jeunes gens se saluèrent, sinon avec affabilité, du moins avec courtoisie. 

«Pardon, messieurs, dit Morrel, mais je n'aperçois pas M. de Morcerf! 

—Ce matin, répondit Château-Renaud, il nous a fait prévenir qu'il nous rejoindrait sur le terrain seulement. 

—Ah!» fit Morrel. 

Beauchamp tira sa montre. 

«Huit heures cinq minutes; il n'y a pas de temps de perdu, monsieur Morrel, dit-il. 

—Oh! répondit Maximilien, ce n'est point dans cette intention que je le disais. 

—D'ailleurs, interrompit Château-Renaud, voici une voiture.» 

En effet, une voiture s'avançait au grand trot par une des avenues aboutissant au carrefour où l'on se trouvait. 

«Messieurs, dit Morrel, sans doute que vous vous êtes munis de pistolets. M. de Monte-Cristo déclare renoncer au droit qu'il avait de se servir des siens. 

—Nous avons prévu cette délicatesse de la part du comte, monsieur Morrel, répondit Beauchamp, et j'ai apporté des armes, que j'ai achetées il y a huit ou dix jours, croyant que j'en aurais besoin pour une affaire pareille. Elles sont parfaitement neuves et n'ont encore servi à personne. Voulez-vous les visiter? 

—Oh! monsieur Beauchamp, dit Morrel en s'inclinant, lorsque vous m'assurez que M. de Morcerf ne connaît point ces armes, vous pensez bien, n'est-ce pas, que votre parole me suffit? 

—Messieurs, dit Château-Renaud, ce n'était point Morcerf qui nous arrivait dans cette voiture, c'était, ma foi! c'étaient Franz et Debray.» 

En effet, les deux jeunes gens annoncés s'avancèrent. 

«Vous ici, messieurs! dit Château-Renaud en échangeant avec chacun une poignée de main; et par quel hasard? 

—Parce que, dit Debray, Albert nous a fait prier ce matin, de nous trouver sur le terrain.» 

Beauchamp et Château-Renaud se regardèrent d'un air étonné. 

«Messieurs, dit Morrel, je crois comprendre. 

—Voyons! 

—Hier, dans l'après-midi, j'ai reçu une lettre de M. de Morcerf, qui me priait de me trouver à l'Opéra. 

—Et moi aussi, dit Debray. 

—Et moi aussi, dit Franz. 

—Et nous aussi, dirent Château-Renaud et Beauchamp. 

—Il voulait que vous fussiez présents à la provocation, dit Morrel, il veut que vous soyez présents au combat. 

—Oui, dirent les jeunes gens, c'est cela, monsieur Maximilien; et, selon toute probabilité, vous avez deviné juste. 

—Mais, avec tout cela, murmura Château-Renaud, Albert ne vient pas; il est en retard de dix minutes. 

—Le voilà, dit Beauchamp, il est à cheval; tenez, il vient ventre à terre suivi de son domestique. 

—Quelle imprudence, dit Château-Renaud, de venir à cheval pour se battre au pistolet! Moi qui lui avais si bien fait la leçon! 

—Et puis, voyez, dit Beauchamp, avec un col à sa cravate, avec un habit ouvert, avec un gilet blanc; que ne s'est-il fait tout de suite dessiner une mouche sur l'estomac? ç'eût été plus simple et plus tôt fini!» 

Pendant ce temps, Albert était arrivé à dix pas du groupe que formaient les cinq jeunes gens; il arrêta son cheval, sauta à terre, et jeta la bride au bras de son domestique. 

Albert s'approcha. Il était pâle, ses yeux étaient rougis et gonflés. On voyait qu'il n'avait pas dormi une seconde de toute la nuit. Il y avait, répandue sur toute sa physionomie, une nuance de gravité triste qui ne lui était pas habituelle. 

«Merci, messieurs, dit-il, d'avoir bien voulu vous rendre à mon invitation: croyez que je vous suis on ne peut plus reconnaissant de cette marque d'amitié.» 

Morrel, à l'approche de Morcerf, avait fait une dizaine de pas en arrière et se trouvait à l'écart. 

«Et à vous aussi, monsieur Morrel, dit Albert, mes remerciements vous appartiennent. Approchez donc, vous n'êtes pas de trop. 

—Monsieur, dit Maximilien, vous ignorez peut-être que je suis le témoin de M. de Monte-Cristo? 

—Je n'en étais pas sûr, mais je m'en doutais. Tant mieux, plus il y aura d'hommes d'honneur ici, plus je serai satisfait. 

—Monsieur Morrel, dit Château-Renaud, vous pouvez annoncer à M. le comte de Monte-Cristo que M. de Morcerf est arrivé, et que nous nous tenons à sa disposition.» 

Morrel fit un mouvement pour s'acquitter de sa commission. Beauchamp, en même temps, tirait la boîte de pistolets de la voiture. 

«Attendez, messieurs, dit Albert, j'ai deux mots à dire à M. le comte de Monte-Cristo. 

—En particulier? demanda Morrel. 

—Non, monsieur, devant tout le monde.» 

Les témoins d'Albert se regardèrent tout surpris; Franz et Debray échangèrent quelques paroles à voix basse, et Morrel, joyeux de cet incident inattendu, alla chercher le comte, qui se promenait dans une contre-allée avec Emmanuel. 

«Que me veut-il? demanda Monte-Cristo. 

—Je l'ignore, mais il demande à vous parler. 

—Oh! dit Monte-Cristo, qu'il ne tente pas Dieu par quelque nouvel outrage! 

—Je ne crois pas que ce soit son intention», dit Morrel. 

Le comte s'avança, accompagné de Maximilien et d'Emmanuel: son visage calme et plein de sérénité faisait une étrange opposition avec le visage bouleversé d'Albert, qui s'approchait, de son côté, suivi des quatre jeunes gens. 

À trois pas l'un de l'autre, Albert et le comte s'arrêtèrent. 

«Messieurs, dit Albert, approchez-vous; je désire que pas un mot de ce que je vais avoir l'honneur de dire à M. le comte de Monte-Cristo ne soit perdu; car ce que je vais avoir l'honneur de lui dire doit être répété par vous à qui voudra l'entendre, si étrange que mon discours vous paraisse. 

—J'attends, monsieur, dit le comte. 

—Monsieur, dit Albert d'une voix tremblante d'abord, mais qui s'assura de plus en plus; monsieur, je vous reprochais d'avoir divulgué la conduite de M. de Morcerf en Épire; car, si coupable que fût M. le comte de Morcerf, je ne croyais pas que ce fût vous qui eussiez le droit de le punir. Mais aujourd'hui, monsieur, je sais que ce droit vous est acquis. Ce n'est point la trahison de Fernand Mondego envers Ali-Pacha qui me rend si prompt à vous excuser, c'est la trahison du pécheur Fernand envers vous, ce sont les malheurs inouïs qui ont été la suite de cette trahison. Aussi je le dis, aussi je le proclame tout haut: oui, monsieur, vous avez eu raison de vous venger de mon père, et moi, son fils, je vous remercie de n'avoir pas fait plus!» 

La foudre, tombée au milieu des spectateurs de cette scène inattendue, ne les eût pas plus étonnés que cette déclaration d'Albert. 

Quant à Monte-Cristo, ses yeux s'étaient lentement levés au ciel avec une expression de reconnaissance infinie, et il ne pouvait assez admirer comment cette nature fougueuse d'Albert, dont il avait assez connu le courage au milieu des bandits romains, s'était tout à coup pliée à cette subite humiliation. Aussi reconnut-il l'influence de Mercédès, et comprit-il comment ce noble cœur ne s'était pas opposé au sacrifice qu'elle savait d'avance devoir être inutile. 

«Maintenant, monsieur, dit Albert, si vous trouvez que les excuses que je viens de vous faire sont suffisantes, votre main, je vous prie. Après le mérite si rare de l'infaillibilité qui semble être le vôtre, le premier de tous les mérites, à mon avis, est de savoir avouer ses torts. Mais cet aveu me regarde seul. J'agissais bien selon les hommes, mais vous, vous agissiez bien selon Dieu. Un ange seul pouvait sauver l'un de nous de la mort et l'ange est descendu du ciel, sinon pour faire de nous deux amis, hélas! la fatalité rend la chose impossible, mais tout au moins deux hommes qui s'estiment.» 

Monte-Cristo, l'œil humide, la poitrine haletante, la bouche entrouverte, tendit à Albert une main que celui-ci saisit et pressa avec un sentiment qui ressemblait à un respectueux effroi. 

«Messieurs, dit-il, monsieur de Monte-Cristo veut bien agréer mes excuses. J'avais agi précipitamment envers lui. La précipitation est mauvaise conseillère: j'avais mal agi. Maintenant ma faute est réparée. J'espère bien que le monde ne me tiendra point pour lâche parce que j'ai fait ce que ma conscience m'a ordonné de faire. Mais, en tout cas, si l'on se trompait sur mon compte, ajouta le jeune homme en relevant la tête avec fierté et comme s'il adressait un défi à ses amis et à ses ennemis, je tâcherais de redresser les opinions. 

—Que s'est-il donc passé cette nuit? demanda Beauchamp à Château-Renaud; il me semble que nous jouons ici un triste rôle. 

—En effet, ce qu'Albert vient de faire est bien misérable ou bien beau, répondit le baron. 

—Ah! voyons, demanda Debray à Franz, qu'est-ce que cela veut dire? Comment! le comte de Monte-Cristo déshonore M. de Morcerf, et il a eu raison aux yeux de son fils! Mais, eussé-je dix Janina dans ma famille, je ne me croirais obligé qu'à une chose, ce serait de me battre dix fois.» 

Quant à Monte-Cristo, le front penché, les bras inertes, écrasé sous le poids de vingt-quatre ans de souvenirs, il ne songeait ni à Albert, ni à Beauchamp, ni à Château-Renaud, ni à personne de ceux qui se trouvaient là: il songeait à cette courageuse femme qui était venue lui demander la vie de son fils, à qui il avait offert la sienne et qui venait de la sauver par l'aveu terrible d'un secret de famille, capable de tuer à jamais chez ce jeune homme le sentiment de la piété filiale. 

«Toujours la Providence! murmura-t-il: ah! c'est d'aujourd'hui seulement que je suis bien certain d'être l'envoyé de Dieu!» 