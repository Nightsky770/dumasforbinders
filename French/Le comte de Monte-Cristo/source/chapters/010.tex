\chapter{Le petit cabinet des Tuileries}

\lettrine{A}{bandonnons} Villefort sur la route de Paris, où, grâce aux triples guides qu'il paie, il brûle le chemin et pénétrons à travers les deux ou trois salons qui le précèdent dans ce petit cabinet des Tuileries, à la fenêtre cintrée, si bien connu pour avoir été le cabinet favori de Napoléon et de Louis XVIII, et pour être aujourd'hui celui de Louis-Philippe.

Là, dans ce cabinet, assis devant une table de noyer qu'il avait rapportée d'Hartwell, et que, par une de ces manies familières aux grands personnages, il affectionnait tout particulièrement, le roi Louis XVIII écoutait assez légèrement un homme de cinquante à cinquante-deux ans, à cheveux gris, à la figure aristocratique et à la mise scrupuleuse, tout en notant à la marge un volume d'Horace, édition de Gryphias, assez incorrecte quoique estimée, et qui prêtait beaucoup aux sagaces observations philologiques de Sa Majesté.

«Vous dites donc, monsieur? dit le roi.

—Que je suis on ne peut plus inquiet, Sire.

—Vraiment? auriez-vous vu en songe sept vaches grasses et sept vaches maigres?

—Non, Sire, car cela ne nous annoncerait que sept années de fertilité et sept années de disette, et, avec un roi aussi prévoyant que l'est Votre Majesté, la disette n'est pas à craindre.

—De quel autre fléau est-il donc question, mon cher Blacas?

—Sire, je crois, j'ai tout lieu de croire qu'un orage se forme du côté du Midi.

—Eh bien, mon cher duc, répondit Louis XVIII, je vous crois mal renseigné, et je sais positivement, au contraire, qu'il fait très beau temps de ce côté-là.»

Tout homme d'esprit qu'il était, Louis XVIII aimait la plaisanterie facile.

«Sire, dit M. de Blacas, ne fût-ce que pour rassurer un fidèle serviteur, Votre Majesté ne pourrait-elle pas envoyer dans le Languedoc, dans la Provence et dans le Dauphiné des hommes sûrs qui lui feraient un rapport sur l'esprit de ces trois provinces?

—\textit{Conimus surdis}, répondit le roi, tout en continuant d'annoter son Horace.

—Sire, répondit le courtisan en riant, pour avoir l'air de comprendre l'hémistiche du poète de Vénouse, Votre Majesté peut avoir parfaitement raison en comptant sur le bon esprit de la France; mais je crois ne pas avoir tout à fait tort en craignant quelque tentative désespérée.

—De la part de qui?

—De la part de Bonaparte, ou du moins de son parti.

—Mon cher Blacas, dit le roi, vous m'empêchez de travailler avec vos terreurs.

—Et moi, Sire, vous m'empêchez de dormir avec votre sécurité.

—Attendez, mon cher, attendez, je tiens une note très heureuse sur le \textit{Pastor quum traheret}; attendez et vous continuerez après.»

Il se fit un instant de silence, pendant lequel Louis XVIII inscrivit, d'une écriture qu'il faisait aussi menue que possible, une nouvelle note en marge de son Horace; puis, cette note inscrite:

—Continuez, mon cher duc, dit-il en se relevant de l'air satisfait d'un homme qui croit avoir eu une idée lorsqu'il a commenté l'idée d'un autre. Continuez, je vous écoute.

—Sire, dit Blacas, qui avait eu un instant l'espoir de confisquer Villefort à son profit, je suis forcé de vous dire que ce ne sont point de simples bruits dénués de tout fondement, de simples nouvelles en l'air, qui m'inquiètent. C'est un homme bien-pensant méritant toute ma confiance, et chargé par moi de surveiller le Midi (le duc hésita en prononçant ces mots), qui arrive en poste pour me dire: Un grand péril menace le roi. Alors, je suis accouru, Sire.

—\textit{Mala ducis agi domum}, continua Louis XVIII en annotant.

—Votre Majesté m'ordonne-t-elle de ne plus insister sur ce sujet?

—Non, mon cher duc, mais allongez la main.

—Laquelle?

—Celle que vous voudrez, là-bas, à gauche.

—Ici, Sire?

—Je vous dis à gauche et vous cherchez à droite; c'est à ma gauche que je veux dire: là; vous y êtes; vous devez trouver le rapport du ministre de la police en date d'hier\dots. Mais, tenez voici M. Dandré lui-même\dots n'est-ce pas, vous dites M. Dandré? interrompit Louis XVIII, s'adressant à l'huissier qui venait en effet d'annoncer le ministre de la police.

—Oui, Sire, M. le baron Dandré, reprit l'huissier.

—C'est juste, baron, reprit Louis XVIII avec un imperceptible sourire; entrez, baron, et racontez au duc ce que vous savez de plus récent sur M. de Bonaparte. Ne nous dissimulez rien de la situation, quelque grave qu'elle soit. Voyons, l'île d'Elbe est-elle un volcan, et allons-nous en voir sortir la guerre flamboyante et toute hérissée: \textit{belle, horrida bella}?»

M. Dandré se balança fort gracieusement sur le dos d'un fauteuil auquel il appuyait ses deux mains et dit:

«Votre Majesté a-t-elle bien voulu consulter le rapport d'hier?

—Oui, oui, mais dites au duc lui-même, qui ne peut le trouver, ce que contenait le rapport; détaillez-lui ce que fait l'usurpateur dans son île.

—Monsieur, dit le baron au duc, tous les serviteurs de Sa Majesté doivent s'applaudir des nouvelles récentes qui nous parviennent de l'île d'Elbe. Bonaparte\dots»

M. Dandré regarda Louis XVIII qui, occupé à écrire une note, ne leva pas même la tête.

«Bonaparte, continua le baron, s'ennuie mortellement; il passe des journées entières à regarder travailler ses mineurs de Porto-Longone.

—Et il se gratte pour se distraire, dit le roi.

—Il se gratte? demanda le duc; que veut dire votre Majesté?

—Eh oui, mon cher duc; oubliez-vous donc que ce grand homme, ce héros, ce demi-dieu est atteint d'une maladie de peau qui le dévore, \textit{prurigo}?

—Il y a plus, monsieur le duc, continua le ministre de la police, nous sommes à peu près sûrs que dans peu de temps l'usurpateur sera fou.

—Fou?

—Fou à lier: sa tête s'affaiblit, tantôt il pleure des larmes, tantôt il rit à gorge déployée; d'autres fois, il passe des heures sur le rivage à jeter des cailloux dans l'eau, et lorsque le caillou a fait cinq ou six ricochets, il paraît aussi satisfait que s'il avait gagné un autre Marengo ou un nouvel Austerlitz. Voilà, vous en conviendrez, des signes de folie.

—Ou de sagesse, monsieur le baron, ou de sagesse, dit Louis XVIII en riant: c'était en jetant des cailloux à la mer que se récréaient les grands capitaines de l'Antiquité; voyez Plutarque, à la vie de Scipion l'Africain.»

M. de Blacas demeura rêveur entre ces deux insouciances. Villefort, qui n'avait pas voulu tout lui dire pour qu'un autre ne lui enlevât point le bénéfice tout entier de son secret, lui en avait dit assez, cependant, pour lui donner de graves inquiétudes.

«Allons, allons, Dandré, dit Louis XVIII, Blacas n'est point encore convaincu, passez à la conversion de l'usurpateur.»

Le ministre de la police s'inclina.

«Conversion de l'usurpateur! murmura le duc, regardant le roi et Dandré, qui alternaient comme deux bergers de Virgile. L'usurpateur est-il converti?

—Absolument, mon cher duc.

—Aux bons principes; expliquez cela, baron.

—Voici ce que c'est, monsieur le duc, dit le ministre avec le plus grand sérieux du monde: dernièrement Napoléon a passé une revue, et comme deux ou trois de ses vieux grognards, comme il les appelle, manifestaient le désir de revenir en France il leur a donné leur congé en les exhortant à servir leur bon roi; ce furent ses propres paroles, monsieur le duc, j'en ai la certitude.

—Eh bien, Blacas, qu'en pensez-vous? dit le roi triomphant, en cessant un instant de compulser le scoliaste volumineux ouvert devant lui.

—Je dis, Sire, que M. le ministre de la Police ou moi nous nous trompons; mais comme il est impossible que ce soit le ministre de la Police, puisqu'il a en garde le salut et l'honneur de Votre Majesté, il est probable que c'est moi qui fais erreur. Cependant, Sire, à la place de Votre Majesté, je voudrais interroger la personne dont je lui ai parlé; j'insisterai même pour que Votre Majesté lui fasse cet honneur.

—Volontiers, duc, sous vos auspices je recevrai qui vous voudrez; mais je veux le recevoir les armes en main. Monsieur le ministre, avez-vous un rapport plus récent que celui-ci! car celui-ci a déjà la date du 20 février, et nous sommes au 3 mars!

—Non, Sire, mais j'en attendais un d'heure en heure. Je suis sorti depuis le matin, et peut-être depuis mon absence est-il arrivé.

—Allez à la préfecture, et s'il n'y en a pas, eh bien, eh bien, continua riant Louis XVIII, faites-en un; n'est-ce pas ainsi que cela se pratique?

—Oh! Sire! dit le ministre, Dieu merci, sous ce rapport, il n'est besoin de rien inventer; chaque jour encombre nos bureaux des dénonciations les plus circonstanciées, lesquelles proviennent d'une foule de pauvres hères qui espèrent un peu de reconnaissance pour des services qu'ils ne rendent pas, mais qu'ils voudraient rendre. Ils tablent sur le hasard, et ils espèrent qu'un jour quelque événement inattendu donnera une espèce de réalité à leurs prédictions.

—C'est bien; allez, monsieur, dit Louis XVIII, et songez que je vous attends.

—Je ne fais qu'aller et venir, Sire; dans dix minutes je suis de retour.

—Et moi, Sire, dit M. de Blacas, je vais chercher mon messager.

—Attendez donc, attendez donc, dit Louis XVIII. En vérité, Blacas, il faut que je vous change vos armes; je vous donnerai un aigle aux ailes déployées, tenant entre ses serres une proie qui essaie vainement de lui échapper, avec cette devise: \textit{Tenax}.

—Sire, j'écoute, dit M. de Blacas, se rongeant les poings d'impatience.


—Je voudrais vous consulter sur ce passage: \textit{Molli fugiens anhelitu}; vous savez, il s'agit du cerf qui fuit devant le loup. N'êtes-vous pas chasseur et grand louvetier? Comment trouvez-vous, à ce double titre, le \textit{molli anhelitu}?

—Admirable, Sire; mais mon messager est comme le cerf dont vous parlez, car il vient de faire 220 lieues en poste, et cela en trois jours à peine.

—C'est prendre bien de la fatigue et bien du souci, mon cher duc, quand nous avons le télégraphe qui ne met que trois ou quatre heures, et cela sans que son haleine en souffre le moins du monde.

—Ah! Sire, vous récompensez bien mal ce pauvre jeune homme, qui arrive de si loin et avec tant d'ardeur pour donner à Votre Majesté un avis utile; ne fût-ce que pour M. de Salvieux, qui me le recommande, recevez-le bien, je vous en supplie.

—M. de Salvieux, le chambellan de mon frère?

—Lui-même.

—En effet, il est à Marseille.

—C'est de là qu'il m'écrit.

—Vous parle-t-il donc aussi de cette conspiration?

—Non, mais il me recommande M. de Villefort, et me charge de l'introduire près de Votre Majesté.

—M. de Villefort? s'écria le roi; ce messager s'appelle-t-il donc M. de Villefort?

—Oui, Sire.

—Et c'est lui qui vient de Marseille?

—En personne.

—Que ne me disiez-vous son nom tout de suite! reprit le roi, en laissant percer sur son visage un commencement d'inquiétude.

—Sire, je croyais ce nom inconnu de Votre Majesté.

—Non pas, non pas, Blacas; c'est un esprit sérieux, élevé, ambitieux surtout; et, pardieu, vous connaissez de nom son père.

—Son père?

—Oui, Noirtier.

—Noirtier le girondin? Noirtier le sénateur?

—Oui, justement.

—Et Votre Majesté a employé le fils d'un pareil homme?

—Blacas, mon ami, vous n'y entendez rien, je vous ai dit que Villefort était ambitieux: pour arriver, Villefort sacrifiera tout, même son père.

—Alors, Sire, je dois donc le faire entrer?

—À l'instant même, duc. Où est-il?

—Il doit m'attendre en bas, dans ma voiture.

—Allez me le chercher.

—J'y cours.»

Le duc sortit avec la vivacité d'un jeune homme; l'ardeur de son royalisme sincère lui donnait vingt ans.

Louis XVIII resta seul, reportant les yeux sur son Horace entrouvert et murmurant:

\begin{quote}\textit{Justum et tenacem propositi virum.}\end{quote}

M. de Blacas remonta avec la même rapidité qu'il était descendu; mais dans l'antichambre il fut forcé d'invoquer l'autorité du roi. L'habit poudreux de Villefort, son costume, où rien n'était conforme à la tenue de cour, avait excité la susceptibilité de M. de Brézé, qui fut tout étonné de trouver dans ce jeune homme la prétention de paraître ainsi vêtu devant le roi. Mais le duc leva toutes les difficultés avec un seul mot: Ordre de Sa Majesté; et malgré les observations que continua de faire le maître des cérémonies, pour l'honneur du principe, Villefort fut introduit.

Le roi était assis à la même place où l'avait laissé le duc. En ouvrant la porte, Villefort se trouva juste en face de lui: le premier mouvement du jeune magistrat fut de s'arrêter.

«Entrez, monsieur de Villefort, dit le roi, entrez.»

Villefort salua et fit quelques pas en avant, attendant que le roi l'interrogeât.

«Monsieur de Villefort, continua Louis XVIII, voici le duc de Blacas, qui prétend que vous avez quelque chose d'important à nous dire.

—Sire, M. le duc a raison, et j'espère que Votre Majesté va le reconnaître elle-même.

—D'abord, et avant toutes choses, monsieur, le mal est-il aussi grand, à votre avis, que l'on veut me le faire croire?

—Sire, je le crois pressant; mais, grâce à la diligence que j'ai faite, il n'est pas irréparable, je l'espère.

—Parlez longuement si vous le voulez, monsieur, dit le roi, qui commençait à se laisser aller lui-même à l'émotion qui avait bouleversé le visage de M. de Blacas, et qui altérait la voix de Villefort; parlez, et surtout commencez par le commencement: j'aime l'ordre en toutes choses.

—Sire, dit Villefort, je ferai à Votre Majesté un rapport fidèle, mais je la prierai cependant de m'excuser si le trouble où je suis jette quelque obscurité dans mes paroles.»

Un coup d'œil jeté sur le roi après cet exorde insinuant, assura Villefort de la bienveillance de son auguste auditeur, et il continua:

«Sire, je suis arrivé le plus rapidement possible à Paris pour apprendre à Votre Majesté que j'ai découvert dans le ressort de mes fonctions, non pas un de ces complots vulgaires et sans conséquence, comme il s'en trame tous les jours dans les derniers rangs du peuple et de l'armée, mais une conspiration véritable, une tempête qui ne menace rien de moins que le trône de Votre Majesté. Sire, l'usurpateur arme trois vaisseaux; il médite quelque projet, insensé peut-être, mais peut-être aussi terrible, tout insensé qu'il est. À cette heure, il doit avoir quitté l'île d'Elbe pour aller où? je l'ignore, mais à coup sûr pour tenter une descente soit à Naples, soit sur les côtes de Toscane, soit même en France. Votre Majesté n'ignore pas que le souverain de l'île d'Elbe a conservé des relations avec l'Italie et avec la France.

—Oui, monsieur, je le sais, dit le roi fort ému, et, dernièrement encore, on a eu avis que des réunions bonapartistes avaient lieu rue Saint-Jacques; mais continuez, je vous prie; comment avez-vous eu ces détails?

—Sire, ils résultent d'un interrogatoire que j'ai fait subir à un homme de Marseille que depuis longtemps je surveillais et que j'ai fait arrêter le jour même de mon départ; cet homme, marin turbulent et d'un bonapartisme qui m'était suspect, a été secrètement à l'île d'Elbe; il y a vu le grand maréchal qui l'a chargé d'une mission verbale pour un bonapartiste de Paris, dont je n'ai jamais pu lui faire dire le nom; mais cette mission était de charger ce bonapartiste de préparer les esprits à un retour (remarquez que c'est l'interrogatoire qui parle, Sire), à un retour qui ne peut manquer d'être prochain.

—Et où est cet homme? demanda Louis XVIII.

—En prison, Sire.

—Et la chose vous a paru grave?

—Si grave, Sire, que cet événement m'ayant surpris au milieu d'une fête de famille, le jour même de mes fiançailles, j'ai tout quitté, fiancée et amis, tout remis à un autre temps pour venir déposer aux pieds de Votre Majesté et les craintes dont j'étais atteint et l'assurance de mon dévouement.

—C'est vrai, dit Louis XVIII; n'y avait-il pas un projet d'union entre vous et Mlle de Saint-Méran?

—La fille d'un des plus fidèles serviteurs de Votre Majesté.

—Oui, oui; mais revenons à ce complot, monsieur de Villefort.

—Sire, j'ai peur que ce soit plus qu'un complot, j'ai peur que ce soit une conspiration.

—Une conspiration dans ces temps-ci, dit le roi en souriant, est chose facile à méditer, mais plus difficile à conduire à son but, par cela même que, rétabli d'hier sur le trône de nos ancêtres, nous avons les yeux ouverts à la fois sur le passé, sur le présent et sur l'avenir; depuis dix mois, mes ministres redoublent de surveillance pour que le littoral de la Méditerranée soit bien gardé. Si Bonaparte descendait à Naples, la coalition tout entière serait sur pied, avant seulement qu'il fût à Piombino; s'il descendait en Toscane, il mettrait le pied en pays ennemi; s'il descend en France, ce sera avec une poignée d'hommes, et nous en viendrons facilement à bout, exécré comme il l'est par la population. Rassurez-vous donc, monsieur; mais ne comptez pas moins sur notre reconnaissance royale.

—Ah! voici M. Dandré!» s'écria le duc de Blacas.

En ce moment, parut en effet sur le seuil de la porte M. le ministre de la Police, pâle, tremblant, et dont le regard vacillait, comme s'il eût été frappé d'un éblouissement.

Villefort fit un pas pour se retirer; mais un serrement de main de M. de Blacas le retint.




