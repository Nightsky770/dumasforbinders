\chapter{Le testament}

\lettrine{A}{u} moment où Barrois sortit, Noirtier regarda Valentine avec cet intérêt malicieux qui annonçait tant de choses. La jeune fille comprit ce regard et Villefort aussi, car son front se rembrunit et son sourcil se fronça. 

Il prit un siège, s'installa dans la chambre du paralytique et attendit. 

Noirtier le regardait faire avec une parfaite indifférence; mais, du coin de l'œil, il avait ordonné à Valentine de ne point s'inquiéter et de rester aussi. 

Trois quarts d'heure après, le domestique rentra avec le notaire. 

«Monsieur, dit Villefort après les premières salutations, vous êtes mandé par M. Noirtier de Villefort, que voici; une paralysie générale lui a ôté l'usage des membres et de la voix, et nous seuls, à grand-peine, parvenons à saisir quelques lambeaux de ses pensées.» 

Noirtier fit de l'œil un appel à Valentine, appel si sérieux et si impératif, qu'elle répondit sur-le-champ: 

«Moi, monsieur, je comprends tout ce que veut dire mon grand-père. 

—C'est vrai, ajouta Barrois, tout, absolument tout, comme je le disais à monsieur en venant. 

—Permettez, monsieur, et vous aussi, mademoiselle, dit le notaire en s'adressant à Villefort et à Valentine, c'est là un de ces cas où l'officier public ne peut inconsidérément procéder sans assumer une responsabilité dangereuse. La première nécessité pour qu'un acte soit valable est que le notaire soit bien convaincu qu'il a fidèlement interprété la volonté de celui qui la dicte. Or, je ne puis pas moi-même être sûr de l'approbation ou de l'improbation d'un client qui ne parle pas; et comme l'objet de ses désirs et de ses répugnances, vu son mutisme, ne peut m'être prouvé clairement, mon ministère est plus qu'inutile et serait illégalement exercé.» 

Le notaire fit un pas pour se retirer. Un imperceptible sourire de triomphe se dessina sur les lèvres du procureur du roi. De son côté, Noirtier regarda Valentine avec une telle expression de douleur, qu'elle se plaça sur le chemin du notaire. 

«Monsieur, dit-elle, la langue que je parle avec mon grand-père est une langue qui se peut apprendre facilement, et de même que je la comprends, je puis en quelques minutes vous amener à la comprendre. Que vous faut-il, voyons, monsieur, pour arriver à la parfaite édification de votre conscience? 

—Ce qui est nécessaire pour que nos actes soient valables, mademoiselle, répondit le notaire, c'est-à-dire la certitude de l'approbation ou de l'improbation. On peut tester malade de corps, mais il faut tester sain d'esprit. 

—Eh bien, monsieur, avec deux signes vous acquerrez cette certitude que mon grand-père n'a jamais mieux joui qu'à cette heure de la plénitude de son intelligence. M. Noirtier, privé de sa voix, privé du mouvement, ferme les yeux quand il veut dire oui, et les cligne à plusieurs reprises quand il veut dire non. Vous en savez assez maintenant pour causer avec M. Noirtier, essayez.» 

Le regard que lança le vieillard à Valentine était si humide de tendresse et de reconnaissance, qu'il fut compris du notaire lui-même. 

«Vous avez entendu et compris ce que vient de dire votre petite-fille, monsieur?» demanda le notaire. 

Noirtier ferma doucement les yeux, et les rouvrit après un instant. 

«Et vous approuvez ce qu'elle a dit? c'est-à-dire que les signes indiqués par elle sont bien ceux à l'aide desquels vous faites comprendre votre pensée? 

—Oui, fit encore le vieillard. 

—C'est vous qui m'avez fait demander? 

—Oui.  

—Pour faire votre testament? 

—Oui. 

—Et vous ne voulez pas que je me retire sans avoir fait ce testament?» 

Le paralytique cligna vivement et à plusieurs reprises ses yeux. 

«Eh bien, monsieur, comprenez-vous, maintenant, demanda la jeune fille, et votre conscience sera-t-elle en repos?» 

Mais avant que le notaire eût pu répondre, Villefort le tira à part: 

«Monsieur, dit-il, croyez-vous qu'un homme puisse supporter impunément un choc physique aussi terrible que celui qu'a éprouvé M. Noirtier de Villefort, sans que le moral ait reçu lui-même une grave atteinte? 

—Ce n'est point cela précisément qui m'inquiète, monsieur, répondit le notaire, mais je me demande comment nous arriverons à deviner les pensées, afin de provoquer les réponses. 

—Vous voyez donc que c'est impossible», dit Villefort. 

Valentine et le vieillard entendaient cette conversation. Noirtier arrêta son regard si fixe et si ferme sur Valentine, que ce regard appelait évidemment une riposte. 

«Monsieur, dit-elle, que cela ne vous inquiète point: si difficile qu'il soit, ou plutôt qu'il vous paraisse de découvrir la pensée de mon grand-père, je vous la révélerai, moi, de façon à lever tous les doutes à cet égard. Voilà six ans que je suis près de M. Noirtier, et, qu'il le dise lui-même, si, depuis six ans, un seul de ses désirs est resté enseveli dans son cœur faute de pouvoir me le faire comprendre? 

—Non, fit le vieillard. 

—Essayons donc, dit le notaire; vous acceptez mademoiselle pour votre interprète?» 

Le paralytique fit signe que oui. 

«Bien; voyons, monsieur, que désirez-vous de moi, et quel est l'acte que vous désirez faire?» 

Valentine nomma toutes les lettres de l'alphabet jusqu'à la lettre T. À cette lettre, l'éloquent coup d'œil de Noirtier arrêta. 

«C'est la lettre T que monsieur demande, dit le notaire; la chose est visible. 

—Attendez», dit Valentine; puis, se retournant vers son grand-père: «Ta\dots te\dots.» 

Le vieillard arrêta à la seconde de ces syllabes. 

Alors Valentine prit le dictionnaire, et aux yeux du notaire attentif elle feuilleta les pages. 

«Testament, dit son doigt arrêté par le coup d'œil de Noirtier.  

—Testament! s'écria le notaire, la chose est visible, monsieur veut tester. 

—Oui, fit Noirtier à plusieurs reprises. 

—Voilà qui est merveilleux, monsieur, convenez-en, dit le notaire à Villefort stupéfait. 

—En effet, répliqua-t-il, et plus merveilleux encore serait ce testament; car, enfin, je ne pense pas que les articles se viennent ranger sur le papier, mot par mot, sans l'intelligente inspiration de ma fille. Or, Valentine sera peut-être un peu trop intéressée à ce testament pour être un interprète convenable des obscures volontés de M. Noirtier de Villefort. 

—Non, non! fit le paralytique. 

—Comment! dit M. de Villefort, Valentine n'est point intéressée à votre testament? 

—Non, fit Noirtier. 

—Monsieur, dit le notaire, qui, enchanté de cette épreuve, se promettait de raconter dans le monde les détails de cet épisode pittoresque; monsieur, rien ne me paraît plus facile maintenant que ce que tout à l'heure je regardais comme une chose impossible, et ce testament sera tout simplement un testament mystique, c'est-à-dire prévu et autorisé par la loi pourvu qu'il soit lu en face de sept témoins, approuvé par le testateur devant eux, et fermé par le notaire, toujours devant eux. Quant au temps, il durera à peine plus longtemps qu'un testament ordinaire; il y a d'abord les formules consacrées et qui sont toujours les mêmes, et quant aux détails, la plupart seront fournis par l'état même des affaires du testateur et par vous qui, les ayant gérées, les connaissez. Mais d'ailleurs, pour que cet acte demeure inattaquable, nous allons lui donner l'authenticité la plus complète; l'un de mes confrères me servira d'aide et, contre les habitudes, assistera à la dictée. Êtes-vous satisfait, monsieur? continua le notaire en s'adressant au vieillard. 

—Oui», répondit Noirtier, radieux d'être compris. 

«Que va-t-il faire?» se demanda Villefort à qui sa haute position commandait tant de réserve, et qui d'ailleurs, ne pouvait deviner vers quel but tendait son père. 

Il se retourna donc pour envoyer chercher le deuxième notaire désigné par le premier; mais Barrois, qui avait tout entendu et qui avait deviné le désir de son maître, était déjà parti. 

Alors le procureur du roi fit dire à sa femme de monter. 

Au bout d'un quart d'heure, tout le monde était réuni dans la chambre du paralytique, et le second notaire était arrivé. 

En peu de mots les deux officiers ministériels furent d'accord. On lut à Noirtier une formule de testament vague, banale; puis pour commencer, pour ainsi dire l'investigation de son intelligence, le premier notaire se retournant de son côté, lui dit: 

«Lorsqu'on fait son testament, monsieur, c'est en faveur de quelqu'un. 

—Oui, fit Noirtier.  

—Avez-vous quelque idée du chiffre auquel se monte votre fortune? 

—Oui. 

—Je vais vous nommer plusieurs chiffres qui monteront successivement; vous m'arrêterez quand j'aurai atteint celui que vous croirez être le vôtre. 

—Oui.» 

Il y avait dans cet interrogatoire une espèce de solennité; d'ailleurs jamais la lutte de l'intelligence contre la matière n'avait peut-être été plus visible; et si ce n'était un sublime, comme nous allions le dire, c'était au moins un curieux spectacle. 

On faisait cercle autour de Villefort, le second notaire était assis à une table, tout prêt à écrire; le premier notaire se tenait debout devant lui et interrogeait. 

«Votre fortune dépasse trois cent mille francs n'est-ce pas? demanda-t-il. 

Noirtier fit signe que oui. 

«Possédez-vous quatre cent mille francs?» demanda le notaire. 

Noirtier resta immobile. 

«Cinq cent mille? 

Même immobilité.  

«Six cent mille? sept cent mille? huit cent mille? neuf cent mille?» 

Noirtier fit signe que oui. 

«Vous possédez neuf cent mille francs? 

—Oui. 

—En immeubles?» demanda le notaire. 

Noirtier fit signe que non. 

«En inscriptions de rentes?» 

Noirtier fit signe que oui. 

«Ces inscriptions sont entre vos mains?» 

Un coup d'œil adressé à Barrois fit sortir le vieux serviteur, qui revint un instant après avec une petite cassette. 

«Permettez-vous qu'on ouvre cette cassette? demanda le notaire. 

Noirtier fit signe que oui. 

On ouvrit la cassette et l'on trouva pour neuf cent mille francs d'inscriptions sur le Grand-Livre. 

Le premier notaire passa, les unes après les autres, chaque inscription à son collègue; le compte y était, comme l'avait accusé Noirtier. 

«C'est bien cela, dit-il; il est évident que l'intelligence est dans toute sa force et dans toute son étendue.» 

Puis, se retournant vers le paralytique: 

«Donc, lui dit-il, vous possédez neuf cent mille francs de capital, qui, à la façon dont ils sont placés, doivent vous produire quarante mille livres de rente à peu près? 

—Oui, fit Noirtier. 

—À qui désirez-vous laisser cette fortune? 

—Oh! dit Mme de Villefort, cela n'est point douteux; M. Noirtier aime uniquement sa petite-fille, Mlle Valentine de Villefort: c'est elle qui le soigne depuis six ans; elle a su captiver par ses soins assidus l'affection de son grand-père, et je dirai presque sa reconnaissance; il est donc juste qu'elle recueille le prix de son dévouement.» 

L'œil de Noirtier lança un éclair comme s'il n'était pas dupe de ce faux assentiment donné par Mme de Villefort aux intentions qu'elle lui supposait. 

«Est-ce donc à Mlle Valentine de Villefort que vous laissez ces neuf cent mille francs?» demanda le notaire, qui croyait n'avoir plus qu'à enregistrer cette clause, mais qui tenait à s'assurer cependant de l'assentiment de Noirtier, et voulait faire constater cet assentiment par tous les témoins de cette étrange scène.  

Valentine avait fait un pas en arrière et pleurait, les yeux baissés; le vieillard la regarda un instant avec l'expression d'une profonde tendresse; puis se retournant vers le notaire, il cligna des yeux de la façon la plus significative. 

«Non? dit le notaire; comment ce n'est pas Mlle Valentine de Villefort que vous instituez pour votre légataire universelle?» 

Noirtier fit signe que non. 

«Vous ne vous trompez pas? s'écria le notaire étonné; vous dites bien non? 

—Non! répéta Noirtier, non!» 

Valentine releva la tête; elle était stupéfaite, non pas de son exhérédation, mais d'avoir provoqué le sentiment qui dicte d'ordinaire de pareils actes. 

Mais Noirtier la regarda avec une si profonde expression de tendresse qu'elle s'écria: 

«Oh! mon bon père, je le vois bien, ce n'est que votre fortune que vous m'ôtez, mais vous me laissez toujours votre cœur? 

—Oh! oui, bien certainement, dirent les yeux du paralytique, se fermant avec une expression à laquelle Valentine ne pouvait se tromper. 

—Merci! merci!» murmura la jeune fille. 

Cependant ce refus avait fait naître dans le cœur de Mme de Villefort une espérance inattendue; elle se rapprocha du vieillard. 

«Alors c'est donc à votre petit-fils Édouard de Villefort que vous laissez votre fortune, cher monsieur Noirtier?» demanda la mère. 

Le clignement des yeux fut terrible: il exprimait presque la haine. 

«Non, fit le notaire; alors c'est à monsieur votre fils ici présent? 

—Non», répliqua le vieillard. 

Les deux notaires se regardèrent stupéfaits; Villefort et sa femme se sentaient rougir, l'un de honte, l'autre de colère. 

«Mais, que vous avons-nous donc fait, père, dit Valentine; vous ne nous aimez donc plus?» 

Le regard du vieillard passa rapidement sur son fils, sur sa belle-fille, et s'arrêta sur Valentine avec une expression de profonde tendresse. 

«Eh bien, dit-elle, si tu m'aimes, voyons, bon père, tâche d'allier cet amour avec ce que tu fais en ce moment. Tu me connais, tu sais que je n'ai jamais songé à ta fortune: d'ailleurs, on dit que je suis riche du côté de ma mère, trop riche; explique-toi donc.» 

Noirtier fixa son regard ardent sur la main de Valentine.  

«Ma main? dit-elle. 

—Oui, fit Noirtier. 

—Sa main! répétèrent tous les assistants. 

—Ah! messieurs, vous voyez bien que tout est inutile, et que mon pauvre père est fou, dit Villefort. 

—Oh! s'écria tout à coup Valentine, je comprends! Mon mariage, n'est-ce pas, bon père? 

—Oui, oui, oui, répéta trois fois le paralytique lançant un éclair à chaque fois que se relevait sa paupière. 

—Tu nous en veux pour le mariage, n'est-ce pas? 

—Oui. 

—Mais c'est absurde, dit Villefort. 

—Pardon, monsieur, dit le notaire, tout cela au contraire est très logique et me fait l'effet de s'enchaîner parfaitement. 

—Tu ne veux pas que j'épouse M. Franz d'Épinay? 

—Non, je ne veux pas, exprima l'œil du vieillard. 

—Et vous déshéritez votre petite-fille, s'écria le notaire parce qu'elle fait un mariage contre votre gré? 

—Oui, répondit Noirtier. 

—De sorte que sans ce mariage elle serait votre héritière? 

—Oui.» 

Il se fit alors un profond silence autour du vieillard. 

Les deux notaires se consultaient; Valentine, les mains jointes, regardait son grand-père avec un sourire reconnaissant; Villefort mordait ses lèvres minces; Mme de Villefort ne pouvait réprimer un sentiment joyeux qui, malgré elle, s'épanouissait sur son visage. 

«Mais, dit enfin Villefort, rompant le premier ce silence, il me semble que je suis seul juge des convenances qui plaident en faveur de cette union. Seul maître de la main de ma fille, je veux qu'elle épouse M. Franz d'Épinay, et elle l'épousera.» 

Valentine tomba pleurante sur un fauteuil. 

«Monsieur, dit le notaire, s'adressant au vieillard, que comptez-vous faire de votre fortune au cas où Mlle Valentine épouserait M. Franz? 

Le vieillard resta immobile. 

«Vous comptez en disposer, cependant? 

—Oui, fit Noirtier. 

—En faveur de quelqu'un de votre famille? 

—Non. 

—En faveur des pauvres, alors? 

—Oui. 

—Mais, dit le notaire, vous savez que la loi s'oppose à ce que vous dépouilliez entièrement votre fils? 

—Oui. 

—Vous ne disposerez donc que de la partie que la loi vous autorise à distraire.» 

Noirtier demeura immobile. 

«Vous continuez à vouloir disposer de tout? 

—Oui. 

—Mais après votre mort on attaquera le testament! 

—Non. 

—Mon père me connaît, monsieur, dit M. de Villefort, il sait que sa volonté sera sacrée pour moi; d'ailleurs il comprend que dans ma position je ne puis plaider contre les pauvres.» 

L'œil de Noirtier exprima le triomphe. 

«Que décidez-vous, monsieur? demanda le notaire à Villefort. 

—Rien, monsieur, c'est une résolution prise dans l'esprit de mon père, et je sais que mon père ne change pas de résolution. Je me résigne donc. Ces neuf cent mille francs sortiront de la famille pour aller enrichir les hôpitaux; mais je ne céderai pas à un caprice de vieillard, et je ferai selon ma conscience.» 

Et Villefort se retira avec sa femme, laissant son père libre de tester comme il l'entendrait. 

Le même jour le testament fut fait; on alla chercher les témoins, il fut approuvé par le vieillard, fermé en leur présence et déposé chez M. Deschamps, le notaire de la famille. 