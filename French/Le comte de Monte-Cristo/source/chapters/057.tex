\chapter{L'enclos à la luzerne}

\lettrine{I}{l} faut que nos lecteurs nous permettent de les ramener à cet enclos qui confine à la maison de M. de Villefort, et, derrière la grille envahie par des marronniers, nous retrouverons des personnages de notre connaissance. 

Cette fois Maximilien est arrivé le premier. C'est lui qui a collé son œil contre la cloison, et qui guette dans le jardin profond une ombre entre les arbres et le craquement d'un brodequin de soie sur le sable des allées. 

Enfin, le craquement tant désiré se fit entendre, et au lieu d'une ombre ce furent deux ombres qui s'approchèrent. Le retard de Valentine avait été occasionné par une visite de Mme Danglars et d'Eugénie, visite qui était prolongée au-delà de l'heure où Valentine était attendue. Alors, pour ne pas manquer à son rendez-vous, la jeune fille avait proposé à Mlle Danglars une promenade au jardin, voulant montrer à Maximilien qu'il n'y avait point de sa faute dans le retard dont sans doute il souffrait. 

Le jeune homme comprit tout avec cette rapidité d'intuition particulière aux amants et son cœur fut soulagé. D'ailleurs, sans arriver à la portée de la voix, Valentine dirigea sa promenade de manière que Maximilien pût la voir passer et repasser, et chaque fois qu'elle passait et repassait, un regard inaperçu de sa compagne, mais jeté de l'autre côté de la grille et recueilli par le jeune homme, lui disait: 

«Prenez patience, ami, vous voyez qu'il n'y a point de ma faute.» 

Et Maximilien, en effet, prenait patience tout en admirant ce contraste entre les deux jeunes filles: entre cette blonde aux yeux languissants et à la taille inclinée comme un beau saule, et cette brune aux yeux fiers et à la taille droite comme un peuplier; puis il va sans dire que dans cette comparaison entre deux natures si opposées, tout l'avantage, dans le cœur du jeune homme du moins, était pour Valentine.  

Au bout d'une demi-heure de promenade, les deux jeunes filles s'éloignèrent. Maximilien comprit que le terme de la visite de Mme Danglars était arrivé. 

En effet, un instant après, Valentine reparut seule. De crainte qu'un regard indiscret ne suivît son retour, elle venait lentement; et, au lieu de s'avancer directement vers la grille, elle alla s'asseoir sur un banc, après avoir sans affectation interrogé chaque touffe de feuillage et plongé son regard dans le fond de toutes les allées. 

Ces précautions prises, elle courut à la grille. 

«Bonjour, Valentine, dit une voix. 

—Bonjour, Maximilien; je vous ai fait attendre, mais vous avez vu la cause? 

—Oui, j'ai reconnu Mlle Danglars; je ne vous croyais pas si liée avec cette jeune personne. 

—Qui vous a donc dit que nous étions liées, Maximilien? 

—Personne; mais il m'a semblé que cela ressortait de la façon dont vous vous donnez le bras, de la façon dont vous causiez: on eût dit deux compagnes de pension se faisant des confidences. 

—Nous nous faisions nos confidences, en effet, dit Valentine, elle m'avouait sa répugnance pour un mariage avec M. de Morcerf, et moi, je lui avouais de mon côté que je regardais comme un malheur d'épouser M. d'Épinay. 

—Chère Valentine! 

—Voilà pourquoi, mon ami, continua la jeune fille, vous avez vu cette apparence d'abandon entre moi et Eugénie; c'est que, tout en parlant de l'homme que je ne puis aimer, je pensais à l'homme que j'aime. 

—Que vous êtes bonne en toutes choses, Valentine, et que vous avez en vous une chose que Mlle Danglars n'aura jamais: c'est ce charme indéfini qui est à la femme ce que le parfum est à la fleur, ce que la saveur est au fruit; car ce n'est pas le tout pour une fleur que d'être belle, ce n'est pas le tout pour un fruit que d'être beau. 

—C'est votre amour qui vous fait voir les choses ainsi, Maximilien. 

—Non, Valentine, je vous jure. Tenez, je vous regardais toutes deux tout à l'heure, et, sur mon honneur, tout en rendant justice à la beauté de Mlle Danglars, je ne comprenais pas qu'un homme devînt amoureux d'elle. 

—C'est que, comme vous le disiez, Maximilien, j'étais là, et que ma présence vous rendait injuste. 

—Non\dots mais dites-moi\dots une question de simple curiosité, et qui émane de certaines idées que je me suis faites sur Mlle Danglars. 

—Oh! bien injustes, sans que je sache lesquelles certainement. Quand vous nous jugez, nous autres pauvres femmes, nous ne devons pas nous attendre à l'indulgence. 

—Avec cela qu'entre vous vous êtes bien justes les unes envers les autres! 

—Parce que, presque toujours, il y a de la passion dans nos jugements. Mais revenez à votre question. 

—Est-ce parce que Mlle Danglars aime quelqu'un qu'elle redoute son mariage avec M. de Morcerf? 

—Maximilien, je vous ai dit que je n'étais pas l'amie d'Eugénie. 

—Eh! mon Dieu! dit Morrel, sans être amies, les jeunes filles se font des confidences; convenez que vous lui avez fait quelques questions là-dessus. Ah! je vous vois sourire. 

—S'il en est ainsi, Maximilien, ce n'est pas la peine que nous ayons entre nous cette cloison de planches. 

—Voyons, que vous a-t-elle dit? 

—Elle m'a dit qu'elle n'aimait personne, dit Valentine; qu'elle avait le mariage en horreur; que sa plus grande joie eût été de mener une vie libre et indépendante, et qu'elle désirait presque que son père perdît sa fortune pour se faire artiste comme son amie, Mlle Louise d'Armilly. 

—Ah! vous voyez! 

—Eh bien, qu'est-ce que cela prouve? demanda Valentine. 

—Rien, répondit en souriant Maximilien. 

—Alors, dit Valentine, pourquoi souriez-vous à votre tour? 

—Ah! dit Maximilien, vous voyez bien que, vous aussi, vous regardez, Valentine. 

—Voulez-vous que je m'éloigne? 

—Oh! non! non pas! Mais revenons à vous. 

—Ah! oui, c'est vrai, car à peine avons-nous dix minutes à passer ensemble. 

—Mon Dieu! s'écria Maximilien consterné. 

—Oui, Maximilien, vous avez raison, dit avec mélancolie Valentine, et vous avez là une pauvre amie. Quelle existence je vous fais passer, pauvre Maximilien, vous si bien fait pour être heureux! Je me le reproche amèrement, croyez-moi. 

—Eh bien, que vous importe, Valentine: si je me trouve heureux ainsi; si cette attente éternelle me semble payée, à moi, par cinq minutes de votre vue, par deux mots de votre bouche, et par cette conviction profonde, éternelle, que Dieu n'a pas créé deux cœurs aussi en harmonie que les nôtres, et ne les a pas presque miraculeusement réunis, surtout pour les séparer. 

—Bon, merci, espérez pour nous deux, Maximilien: cela me rend à moitié heureuse. 

—Que vous arrive-t-il donc encore, Valentine, que vous me quittez si vite? 

—Je ne sais; Mme de Villefort m'a fait prier de passer chez elle pour une communication de laquelle dépend, m'a-t-elle fait dire, une portion de ma fortune. Eh! mon Dieu, qu'ils la prennent ma fortune, je suis trop riche, et qu'après me l'avoir prise ils me laissent tranquille et libre; vous m'aimerez tout autant pauvre, n'est-ce pas, Morrel? 

—Oh! je vous aimerai toujours, moi; que m'importe richesse ou pauvreté, si ma Valentine était près de moi et que je fusse sûr que personne ne me la pût ôter! Mais cette communication, Valentine, ne craignez-vous point que ce ne soit quelque nouvelle relative à votre mariage? 

—Je ne le crois pas. 

—Cependant, écoutez-moi, Valentine, et ne vous effrayez pas, car tant que je vivrai je ne serai pas à une autre. 

—Vous croyez me rassurer en me disant cela, Maximilien? 

—Pardon! vous avez raison, je suis un brutal. Eh bien, je voulais donc vous dire que l'autre jour j'ai rencontré M. de Morcerf. 

—Eh bien? 

—M. Franz est son ami, comme vous savez. 

—Oui; eh bien? 

—Eh bien, il a reçu une lettre de Franz, qui lui annonce son prochain retour.» 

Valentine pâlit et appuya sa main contre la grille. 

«Ah! mon Dieu! dit-elle, si c'était cela! Mais non, la communication ne viendrait pas de Mme de Villefort. 

—Pourquoi cela? 

—Pourquoi\dots je n'en sais rien\dots mais il me semble que Mme de Villefort, tout en ne s'y opposant point franchement, n'est pas sympathique à ce mariage. 

—Eh bien, mais, Valentine, il me semble que je vais l'adorer, Mme de Villefort. 

—Oh! ne vous pressez pas, Maximilien, dit Valentine avec un triste sourire. 

—Enfin, si elle est antipathique à ce mariage, ne fût-ce que pour le rompre, peut-être ouvrirait-elle l'oreille à quelque autre proposition. 

—Ne croyez point cela, Maximilien; ce ne sont point les maris que Mme de Villefort repousse, c'est le mariage. 

—Comment? le mariage! Si elle déteste si fort le mariage, pourquoi s'est-elle mariée elle-même? 

—Vous ne me comprenez pas, Maximilien; ainsi, lorsqu'il y a un an j'ai parlé de me retirer dans un couvent, elle avait, malgré les observations qu'elle avait cru devoir faire, adopté ma proposition avec joie; mon père même y avait consenti, à son instigation, j'en suis sûre; il n'y eut que mon pauvre grand-père qui m'a retenue. Vous ne pouvez vous figurer, Maximilien, quelle expression il y a dans les yeux de ce pauvre vieillard, qui n'aime que moi au monde, et qui, Dieu me pardonne si c'est un blasphème, et qui n'est aimé au monde que de moi. Si vous saviez, quand il a appris ma résolution, comme il m'a regardée, ce qu'il y avait de reproche dans ce regard et de désespoir dans ces larmes qui roulaient sans plaintes, sans soupirs, le long de ses joues immobiles! Ah! Maximilien, j'ai éprouvé quelque chose comme un remords, je me suis jetée à ses pieds en lui criant: «Pardon! pardon! mon père! On fera de moi ce qu'on voudra, mais je ne vous quitterai jamais.» Alors il leva les yeux au ciel!\dots Maximilien, je puis souffrir beaucoup, ce regard de mon vieux grand-père m'a payée d'avance pour ce que je souffrirai. 

—Chère Valentine! vous êtes un ange, et je ne sais vraiment pas comment j'ai mérité, en sabrant à droite et à gauche des Bédouins, à moins que Dieu ait considéré que ce sont des infidèles, je ne sais pas comment j'ai mérité que vous vous révéliez à moi. Mais enfin, voyons, Valentine, quel est donc l'intérêt de Mme de Villefort à ce que vous ne vous mariez pas? 

—N'avez-vous pas entendu tout à l'heure que je vous disais que j'étais riche, Maximilien, trop riche? J'ai, du chef de ma mère, près de cinquante mille livres de rente; mon grand-père et ma grand-mère, le marquis et la marquise de Saint-Méran, doivent m'en laisser autant; M. Noirtier a bien visiblement l'intention de me faire sa seule héritière. Il en résulte donc que, comparativement à moi, mon frère Édouard, qui n'attend, du côté de Mme de Villefort, aucune fortune, est pauvre. Or, Mme de Villefort aime cet enfant avec adoration, et si je fusse entrée en religion, toute ma fortune, concentrée sur mon père, qui héritait du marquis, de la marquise et de moi, revenait à son fils. 

—Oh! que c'est étrange cette cupidité dans une jeune et belle femme! 

—Remarquez que ce n'est point pour elle, Maximilien, mais pour son fils, et que ce que vous lui reprochez comme un défaut, au point de vue de l'amour maternel, est presque une vertu. 

—Mais voyons, Valentine, dit Morrel, si vous abandonniez une portion de cette fortune à ce fils. 

—Le moyen de faire une pareille proposition, dit Valentine, et surtout à une femme qui a sans cesse à la bouche le mot de désintéressement? 

—Valentine, mon amour m'est toujours resté sacré, et comme toute chose sacrée, je l'ai couvert du voile de mon respect et enfermé dans mon cœur; personne au monde, pas même ma sœur, ne se doute donc de cet amour que je n'ai confié à qui que ce soit au monde. Valentine, me permettez-vous de parler de cet amour à un ami?» 

Valentine tressaillit. 

«À un ami? dit-elle. Oh! mon Dieu! Maximilien, je frissonne rien qu'à vous entendre parler ainsi! À un ami? et qui donc est cet ami? 

—Écoutez, Valentine: avez-vous jamais senti pour quelqu'un une de ces sympathies irrésistibles qui font que, tout en voyant cette personne pour la première fois, vous croyez la connaître depuis longtemps, et vous vous demandez où et quand vous l'avez vue, si bien que, ne pouvant vous rappeler ni le lieu ni le temps, vous arrivez à croire que c'est dans un monde antérieur au nôtre, et que cette sympathie n'est qu'un souvenir qui se réveille? 

—Oui. 

—Eh bien, voilà ce que j'ai éprouvé la première fois que j'ai vu cet homme extraordinaire. 

—Un homme extraordinaire? 

—Oui. 

—Que vous connaissez depuis longtemps alors? 

—Depuis huit ou dix jours à peine. 

—Et vous appelez votre ami un homme que vous connaissez depuis huit jours? Oh! Maximilien, je vous croyais plus avare de ce beau nom d'ami. 

—Vous avez raison en logique, Valentine; mais dites ce que vous voudrez, rien ne me fera revenir sur ce sentiment instinctif. Je crois que cet homme sera mêlé à tout ce qui m'arrivera de bien dans l'avenir, que parfois son regard profond semble connaître et sa main puissante diriger. 

—C'est donc un devin? dit en souriant Valentine. 

—Ma foi, dit Maximilien, je suis tenté de croire souvent qu'il devine\dots le bien surtout. 

—Oh! dit Valentine tristement, faites-moi connaître cet homme, Maximilien, que je sache de lui si je serai assez aimée pour me dédommager de tout ce que j'ai souffert. 

—Pauvre amie! mais vous le connaissez! 

—Moi? 

—Oui. C'est celui qui a sauvé la vie à votre belle-mère et à son fils. 

—Le comte de Monte-Cristo? 

—Lui-même. 

—Oh! s'écria Valentine, il ne peut jamais être mon ami, il est trop celui de ma belle-mère. 

—Le comte, l'ami de votre belle-mère, Valentine? mon instinct ne faillirait pas à ce point; je suis sûr que vous vous trompez. 

—Oh! si vous saviez, Maximilien! mais ce n'est plus Édouard qui règne à la maison, c'est le comte: recherché de madame de Villefort, qui voit en lui le résumé des connaissances humaines; admiré, entendez-vous, admiré de mon père, qui dit n'avoir jamais entendu formuler avec plus d'éloquence des idées plus élevées; idolâtré d'Édouard, qui, malgré sa peur des grands yeux noirs du comte, court à lui aussitôt qu'il le voit arriver, et lui ouvre la main, où il trouve toujours quelque jouet admirable: M. de Monte-Cristo n'est pas ici chez mon père; M. de Monte-Cristo n'est pas ici chez Mme de Villefort: M. de Monte-Cristo est chez lui. 

—Eh bien, chère Valentine, si les choses sont ainsi que vous dites, vous devez déjà ressentir ou vous ressentirez bientôt les effets de sa présence. Il rencontre Albert de Morcerf en Italie, c'est pour le tirer des mains des brigands; il aperçoit Mme Danglars, c'est pour lui faire un cadeau royal; votre belle-mère et votre frère passent devant sa porte, c'est pour que son Nubien leur sauve la vie. Cet homme a évidemment reçu le pouvoir d'influer sur les choses. Je n'ai jamais vu des goûts plus simples alliés à une haute magnificence. Son sourire est si doux, quand il me l'adresse que j'oublie combien les autres trouvent son sourire amer. Oh! dites-moi, Valentine, vous a-t-il souri ainsi? S'il l'a fait, vous serez heureuse. 

—Moi! dit la jeune fille, oh! mon Dieu! Maximilien, il ne me regarde seulement pas, ou plutôt, si je passe par hasard, il détourne la vue de moi. Oh! il n'est pas généreux, allez! ou il n'a pas ce regard profond qui lit au fond des cœurs, et que vous lui supposez à tort; car s'il eût été généreux, me voyant seule et triste au milieu de toute cette maison, il m'eût protégée de cette influence qu'il exerce; et puisqu'il joue, à ce que vous prétendez, le rôle de soleil, il eût réchauffé mon cœur à l'un de ses rayons. Vous dites qu'il vous aime, Maximilien; eh! mon Dieu, qu'en savez-vous? Les hommes font gracieux visage à un officier de cinq pieds six pouces comme vous, qui a une longue moustache et un grand sabre, mais ils croient pouvoir écraser sans crainte une pauvre fille qui pleure. 

—Oh! Valentine! vous vous trompez, je vous jure. 

—S'il en était autrement, voyons, Maximilien, s'il me traitait diplomatiquement, c'est-à-dire en homme qui, d'une façon ou de l'autre, veut s'impatroniser dans la maison, il m'eût, ne fût-ce qu'une seule fois honorée de ce sourire que vous me vantez si fort, mais non, il m'a vue malheureuse, il comprend que je ne puis lui être bonne à rien, et il ne fait pas même attention à moi. Qui sait même si, pour faire sa cour à mon père, à Mme de Villefort ou à mon frère, il ne me persécutera point aussi en tant qu'il sera en son pouvoir de le faire? Voyons, franchement, je ne suis pas une femme que l'on doive mépriser ainsi sans raison; vous me l'avez dit. Ah! pardonnez-moi, continua la jeune fille en voyant l'impression que ces paroles produisaient sur Maximilien, je suis mauvaise, et je vous dis là sur cet homme des choses que je ne savais pas même avoir dans le cœur. Tenez, je ne nie pas que cette influence dont vous me parlez existe, et qu'il ne l'exerce même sur moi; mais s'il l'exerce, c'est d'une manière nuisible et corruptrice, comme vous le voyez, de bonnes pensées. 

—C'est bien, Valentine, dit Morrel avec un soupir, n'en parlons plus; je ne lui dirai rien. 

—Hélas! mon ami, dit Valentine, je vous afflige, je le vois. Oh! que ne puis-je vous serrer la main pour vous demander pardon! Mais enfin je ne demande pas mieux que d'être convaincue; dites, qu'a donc fait pour vous ce comte de Monte-Cristo? 

—Vous m'embarrassez fort, je l'avoue, Valentine, en me demandant ce que le comte a fait pour moi: rien d'ostensible, je le sais bien. Aussi, comme je vous l'ai déjà dit, mon affection pour lui est-elle tout instinctive et n'a-t-elle rien de raisonné. Est-ce que le soleil m'a fait quelque chose? Non; il me réchauffe, et à sa lumière je vous vois, voilà tout. Est-ce que tel ou tel parfum a fait quelque chose pour moi? Non; son odeur récrée agréablement un de mes sens. Je n'ai pas autre chose à dire quand on me demande pourquoi je vante ce parfum, mon amitié pour lui est étrange comme la sienne pour moi. Une voix secrète m'avertit qu'il y a plus que du hasard dans cette amitié imprévue et réciproque. Je trouve de la corrélation jusque dans ses plus simples actions, jusque dans ses plus secrètes pensées entre mes actions et mes pensées. Vous allez encore rire de moi, Valentine, mais depuis que je connais cet homme, l'idée absurde m'est venue que tout ce qui m'arrive de bien émane de lui. Cependant, j'ai vécu trente ans sans avoir eu besoin de ce protecteur, n'est-ce pas? n'importe, tenez, un exemple: il m'a invité à dîner pour samedi, c'est naturel au point où nous en sommes, n'est-ce pas? Eh bien, qu'ai-je su depuis? Votre père est invité à ce dîner, votre mère y viendra. Je me rencontrerai avec eux, et qui sait ce qui résultera dans l'avenir de cette entrevue? Voilà des circonstances fort simples en apparence; cependant, moi, je vois là-dedans quelque chose qui m'étonne; j'y puise une confiance étrange. Je me dis que le comte, cet homme singulier qui devine tout, a voulu me faire trouver avec M. et Mme de Villefort, et quelquefois je cherche, je vous le jure, à lire dans ses yeux s'il a deviné mon amour. 

—Mon bon ami, dit Valentine, je vous prendrais pour un visionnaire, et j'aurais véritablement peur pour votre bon sens, si je n'écoutais de vous que de semblables raisonnements. Quoi! vous voyez autre chose que du hasard dans cette rencontre? En vérité, réfléchissez donc. Mon père, qui ne sort jamais, a été sur le point dix fois de refuser cette invitation à Mme de Villefort, qui, au contraire, brûle du désir de voir chez lui ce nabab extraordinaire, et c'est à grand-peine qu'elle a obtenu qu'il l'accompagnerait. Non, non, croyez-moi, je n'ai, à part vous, Maximilien, d'autre secours à demander dans ce monde qu'à mon grand-père, un cadavre! d'autre appui à chercher que dans ma pauvre mère, une ombre! 

—Je sens que vous avez raison, Valentine, et que la logique est pour vous, dit Maximilien; mais votre douce voix, toujours si puissante sur moi, aujourd'hui, ne me convainc pas. 

—Ni la vôtre non plus, dit Valentine, et j'avoue que si vous n'avez pas d'autre exemple à me citer\dots. 

—J'en ai un, dit Maximilien en hésitant; mais en vérité, Valentine, je suis forcé de l'avouer moi-même, il est encore plus absurde que le premier. 

—Tant pis, dit en souriant Valentine. 

—Et cependant, continua Morrel, il n'en est pas moins concluant pour moi, homme tout d'inspiration et de sentiment, et qui ai quelquefois, depuis dix ans que je sers, dû la vie à un de ces éclairs intérieurs qui vous dictent un mouvement en avant ou en arrière, pour que la balle qui devait vous tuer passe à côté de vous. 

—Cher Maximilien, pourquoi ne pas faire honneur à mes prières de cette déviation des balles? Quand vous êtes là-bas, ce n'est plus pour moi que je prie Dieu et ma mère, c'est pour vous. 

—Oui, depuis que je vous connais, dit en souriant Morrel; mais avant que je vous connusse, Valentine? 

—Voyons, puisque vous ne voulez rien me devoir, méchant, revenez donc à cet exemple que vous-même avouez être absurde. 

—Eh bien, regardez par les planches, et voyez là-bas, à cet arbre, le cheval nouveau avec lequel je suis venu. 

—Oh! l'admirable bête! s'écria Valentine, pourquoi ne l'avez-vous pas amené près de la grille? je lui eusse parlé et il m'eût entendue. 

—C'est en effet, comme vous le voyez, une bête d'un assez grand prix, dit Maximilien. Eh bien, vous savez que ma fortune est bornée, Valentine, et que je suis ce qu'on appelle un homme raisonnable. Eh bien, j'avais vu chez un marchand de chevaux ce magnifique \textit{Médéah}, je le nomme ainsi. Je demandai quel était son prix: on me répondit quatre mille cinq cents francs; je dus m'abstenir, comme vous le comprenez bien, de le trouver beau plus longtemps, et je partis, je l'avoue, le cœur assez gros, car le cheval m'avait tendrement regardé, m'avait caressé avec sa tête et avait caracolé sous moi de la façon la plus coquette et la plus charmante. Le même soir j'avais quelques amis à la maison: M. de Château-Renaud, M. Debray et cinq ou six autres mauvais sujets que vous avez le bonheur de ne pas connaître, même de nom. On proposa une bouillotte; je ne joue jamais, car je ne suis pas assez riche pour pouvoir perdre, ni assez pauvre pour désirer gagner. Mais j'étais chez moi, vous comprenez, je n'avais autre chose à faire que d'envoyer chercher des cartes, et c'est ce que je fis. 

«Comme on se mettait à table, M. de Monte-Cristo arriva. Il prit sa place, on joua, et, moi, je gagnai; j'ose à peine vous avouer cela, Valentine, je gagnai cinq mille francs. Nous nous quittâmes à minuit. Je n'y pus tenir, je pris un cabriolet et me fis conduire chez mon marchand de chevaux. Tout palpitant, tout fiévreux, je sonnai; celui qui vint m'ouvrir dut me prendre pour un fou. Je m'élançai de l'autre côté de la porte à peine ouverte. J'entrai dans l'écurie, je regardai au râtelier. Oh! bonheur! \textit{Médéah} grignotait son foin. Je saute sur une selle; je la lui applique moi-même sur le dos, je lui passe la bride, \textit{Médéah} se prête de la meilleure grâce du monde à cette opération! Puis, déposant les quatre mille cinq cents francs entre les mains du marchand stupéfait, je reviens ou plutôt je passe la nuit à me promener dans les Champs-Élysées. Eh bien, j'ai vu de la lumière à la fenêtre du comte, il m'a semblé apercevoir son ombre derrière les rideaux. Maintenant, Valentine, je jurerais que le comte a su que je désirais ce cheval, et qu'il a perdu exprès pour me le faire gagner. 

—Mon cher Maximilien, dit Valentine, vous êtes trop fantastique, en vérité\dots vous ne m'aimerez pas longtemps\dots. Un homme qui fait ainsi de la poésie ne saurait s'étioler à plaisir dans une passion monotone comme la nôtre\dots. Mais, grand Dieu! tenez, on m'appelle\dots entendez-vous? 

—Oh! Valentine, dit Maximilien, par le petit jour de la cloison\dots votre doigt le plus petit, que je le baise. 

—Maximilien, nous avions dit que nous serions l'un pour l'autre deux voix, deux ombres! 

—Comme il vous plaira, Valentine. 

—Serez-vous heureux si je fais ce que vous voulez? 

—Oh! oui.» 

Valentine monta sur un banc et passa, non pas son petit doigt à travers l'ouverture, mais sa main tout entière par-dessus la cloison. 

Maximilien poussa un cri, et s'élançant à son tour sur la borne, saisit cette main adorée et y appliqua ses lèvres ardentes; mais aussitôt la petite main glissa entre les siennes, et le jeune homme entendit fuir Valentine, effrayée peut-être de la sensation qu'elle venait d'éprouver! 