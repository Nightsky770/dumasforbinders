\chapter{Les catacombes de Saint-Sébastien}

\lettrine{P}{eut-être,} de sa vie, Franz n'avait-il éprouvé une impression si tranchée, un passage si rapide de la gaieté à la tristesse, que dans ce moment; on eût dit que Rome, sous le souffle magique de quelque démon de la nuit, venait de se changer en un vaste tombeau. Par un hasard qui ajoutait encore à l'intensité des ténèbres, la lune, qui était dans sa décroissance ne devait se lever que vers les onze heures du soir; les rues que le jeune homme traversait étaient donc plongées dans la plus profonde obscurité. Au reste, le trajet était court; au bout de dix minutes, sa voiture ou plutôt celle du comte s'arrêta devant l'hôtel de Londres. 

Le dîner attendait; mais comme Albert avait prévenu qu'il ne comptait pas rentrer de sitôt, Franz se mit à table sans lui. 

Maître Pastrini, qui avait l'habitude de les voir dîner ensemble, s'informa des causes de son absence; mais Franz se contenta de répondre qu'Albert avait reçu la surveille une invitation à laquelle il s'était rendu. L'extinction subite des moccoletti, cette obscurité qui avait remplacé la lumière, ce silence qui avait succédé au bruit, avaient laissé dans l'esprit de Franz une certaine tristesse qui n'était pas exempte d'inquiétude. Il dîna donc fort silencieusement malgré l'officieuse sollicitude de son hôte, qui entra deux ou trois fois pour s'informer s'il n'avait besoin de rien. 

Franz était résolu à attendre Albert aussi tard que possible. Il demanda donc la voiture pour onze heures seulement, en priant maître Pastrini de le faire prévenir à l'instant même si Albert reparaissait à l'hôtel pour quelque chose que ce fût. À onze heures, Albert n'était pas rentré. Franz s'habilla et partit, en prévenant son hôte qu'il passait la nuit chez le duc de Bracciano. 

La maison du duc de Bracciano est une des plus charmantes maisons de Rome; sa femme, une des dernières héritières des Colonna, en fait les honneurs d'une façon parfaite: il en résulte que les fêtes qu'il donne ont une célébrité européenne. Franz et Albert étaient arrivés à Rome avec des lettres de recommandation pour lui; aussi sa première question fut-elle pour demander à Franz ce qu'était devenu son compagnon de voyage. Franz lui répondit qu'il l'avait quitté au moment où on allait éteindre les moccoli, et qu'il l'avait perdu de vue à la via Macello. 

«Alors il n'est pas rentré? demanda le duc. 

—Je l'ai attendu jusqu'à cette heure, répondit Franz. 

—Et savez-vous où il allait? 

—Non, pas précisément; cependant je crois qu'il s'agissait de quelque chose comme un rendez-vous. 

—Diable! dit le duc, c'est un mauvais jour, ou plutôt c'est une mauvaise nuit pour s'attarder, n'est-ce pas, madame la comtesse?» 

Ces derniers mots s'adressaient à la comtesse G\dots qui venait d'arriver, et qui se promenait au bras de M. Torlonia, frère du duc. 

«Je trouve au contraire que c'est une charmante nuit, répondit la comtesse; et ceux qui sont ici ne se plaindront que d'une chose, c'est qu'elle passera trop vite. 

—Aussi, reprit le duc en souriant, je ne parle pas des personnes qui sont ici, elles ne courent d'autres dangers, les hommes que de devenir amoureux de vous, les femmes de tomber malades de jalousie en vous voyant si belle; je parle de ceux qui courent les rues de Rome. 

—Eh! bon Dieu, demanda la comtesse, qui court les rues de Rome à cette heure-ci, à moins que ce ne soit pour aller au bal? 

—Notre ami Albert de Morcerf, madame la comtesse, que j'ai quitté à la poursuite de son inconnue vers les sept heures du soir, dit Franz, et que je n'ai pas revu depuis. 

—Comment! et vous ne savez pas où il est? 

—Pas le moins du monde. 

—Et a-t-il des armes? 

—Il est en paillasse. 

—Vous n'auriez pas dû le laisser aller, dit le duc à Franz, vous qui connaissez Rome mieux que lui. 

—Oh! bien oui, autant aurait valu essayer d'arrêter le numéro trois des barberi qui a gagné aujourd'hui le prix de la course, répondit Franz; et puis, d'ailleurs, que voulez-vous qu'il lui arrive? 

—Qui sait! la nuit est très sombre, et le Tibre est bien près de la via Macello.» 

Franz sentit un frisson qui lui courait dans les veines en voyant l'esprit du duc et de la comtesse si bien d'accord avec ses inquiétudes personnelles. 

«Aussi ai-je prévenu à l'hôtel que j'avais l'honneur de passer la nuit chez vous, monsieur le duc, dit Franz, et on doit venir m'annoncer son retour. 

—Tenez, dit le duc, je crois justement que voilà un de mes domestiques qui vous cherche.» 

Le duc ne se trompait pas; en apercevant Franz, le domestique s'approcha de lui: 

«Excellence, dit-il, le maître de l'hôtel de Londres vous fait prévenir qu'un homme vous attend chez lui avec une lettre du vicomte de Morcerf. 

—Avec une lettre du vicomte! s'écria Franz. 

—Oui. 

—Et quel est cet homme?  

—Je l'ignore. 

—Pourquoi n'est-il point venu me l'apporter ici? 

—Le messager ne m'a donné aucune explication. 

—Et où est le messager? 

—Il est parti aussitôt qu'il m'a vu entrer dans la salle du bal pour vous prévenir. 

—Oh! mon Dieu! dit la comtesse à Franz, allez vite. Pauvre jeune homme, il lui est peut-être arrivé quelque accident.  

—J'y cours, dit Franz. 

—Vous reverrons-nous pour nous donner des nouvelles? demanda la comtesse. 

—Oui, si la chose n'est pas grave; sinon, je ne réponds pas de ce que je vais devenir moi-même. 

—En tout cas, de la prudence, dit la comtesse. 

—Oh! soyez tranquille.» 

Franz prit son chapeau et partit en toute hâte. Il avait renvoyé sa voiture en lui donnant l'ordre pour deux heures; mais, par bonheur, le palais Bracciano, qui donne d'un côté rue du Cours et de l'autre place des Saints-Apôtres, est à dix minutes de chemin à peine de l'hôtel de Londres. En approchant de l'hôtel, Franz vit un homme debout au milieu de la rue, il ne douta pas un seul instant que ce ne fût le messager d'Albert. Cet homme était lui-même enveloppé d'un grand manteau. Il alla à lui; mais au grand étonnement de Franz, ce fut cet homme qui lui adressa la parole le premier. 

«Que me voulez-vous, Excellence? dit-il en faisant un pas en arrière comme un homme qui désire demeurer sur ses gardes. 

—N'est-ce pas vous, demanda Franz, qui m'apportez une lettre du vicomte de Morcerf?  

—C'est Votre Excellence qui loge à l'hôtel de Pastrini? 

—Oui. 

—C'est Votre Excellence qui est le compagnon de voyage du vicomte? 

—Oui. 

—Comment s'appelle Votre Excellence? 

—Le baron Franz d'Épinay. 

—C'est bien à Votre Excellence alors que cette lettre est adressée. 

—Y a-t-il une réponse? demanda Franz en lui prenant la lettre des mains. 

—Oui, du moins votre ami l'espère bien. 

—Montez chez moi, alors, je vous la donnerai. 

—J'aime mieux l'attendre ici, dit en riant le messager. 

—Pourquoi cela? 

—Votre Excellence comprendra la chose quand elle aura lu la lettre. 

—Alors je vous retrouverai ici? 

—Sans aucun doute.» 

Franz rentra; sur l'escalier il rencontra maître Pastrini. 

«Eh bien? lui demanda-t-il. 

—Eh bien quoi? répondit Franz. 

—Vous avez vu l'homme qui désirait vous parler de la part de votre ami? demanda-t-il à Franz. 

—Oui, je l'ai vu, répondit celui-ci, et il m'a remis cette lettre. Faites allumer chez moi, je vous prie.» 

L'aubergiste donna l'ordre à un domestique de précéder Franz avec une bougie. Le jeune homme avait trouvé à maître Pastrini un air effaré, et cet air ne lui avait donné qu'un désir plus grand de lire la lettre d'Albert: il s'approcha de la bougie aussitôt qu'elle fut allumée, et déplia le papier. La lettre était écrite de la main d'Albert et signée par lui. Franz la relut deux fois, tant il était loin de s'attendre à ce qu'elle contenait. 

La voici textuellement reproduite: 

\begin{mail}{}{Cher ami,}
Aussitôt la présente reçue, ayez l'obligeance de prendre dans mon portefeuille, que vous trouverez dans le tiroir carré du secrétaire, la lettre de crédit; joignez-y la vôtre si elle n'est pas suffisante. Courez chez Torlonia, prenez-y à l'instant même quatre mille piastres et remettez-les au porteur. Il est urgent que cette somme me soit adressée sans aucun retard. 

Je n'insiste pas davantage, comptant sur vous comme vous pourriez compter sur moi. 

\addPS{I believe now to italian banditti.} 

\closeletter[Votre ami,]{Albert de Morcerf.}
\end{mail}

Au-dessous de ces lignes étaient écrits d'une main étrangère ces quelques mots italiens: 

\begin{mail}{}{}
\textit{Se alle sei della mattina le quattro mille piastre non sono nelle mie mani, alle sette il comte Alberto avrà cessato di vivere.} [Si, à six heures du matin, les quatre mille piastres ne sont point entre mes mains, à sept heures, le vicomte Albert de Morcerf aura cessé d'exister.] 
\closeletter{Luigi Vampa}
\end{mail}

Cette seconde signature expliqua tout à Franz, qui comprit la répugnance du messager à monter chez lui; la rue lui paraissait plus sûre que la chambre de Franz. Albert était tombé entre les mains du fameux chef de bandits à l'existence duquel il s'était si longtemps refusé de croire. 

Il n'y avait pas de temps à perdre. Il courut au secrétaire, l'ouvrit, dans le tiroir indiqué trouva le portefeuille, et dans le portefeuille la lettre de crédit: elle était en tout de six mille piastres, mais sur ces six mille piastres Albert en avait déjà dépensé trois mille. Quant à Franz, il n'avait aucune lettre de crédit; comme il habitait Florence, et qu'il était venu à Rome pour passer sept à huit jours seulement, il avait pris une centaine de louis, et de ces cent louis il en restait cinquante tout au plus.  

Il s'en fallait donc de sept à huit cents piastres pour qu'à eux deux Franz et Albert pussent réunir la somme demandée. Il est vrai que Franz pouvait compter, dans un cas pareil, sur l'obligeance de MM. Torlonia. 

Il se préparait donc à retourner au palais Bracciano sans perdre un instant, quand tout à coup une idée lumineuse traversa son esprit. 

Il songea au comte de Monte-Cristo. Franz allait donner l'ordre qu'on fît venir maître Pastrini, lorsqu'il le vit apparaître en personne sur le seuil de sa porte. 

«Mon cher monsieur Pastrini, lui dit-il vivement, croyez-vous que le comte soit chez lui? 

—Oui, Excellence, il vient de rentrer. 

—A-t-il eu le temps de se mettre au lit? 

—J'en doute. 

—Alors, sonnez à sa porte, je vous prie, et demandez-lui pour moi la permission de me présenter chez lui.» 

Maître Pastrini s'empressa de suivre les instructions qu'on lui donnait; cinq minutes après il était de retour. 

«Le comte attend Votre Excellence», dit-il. 

Franz traversa le carré, un domestique l'introduisit chez le comte. Il était dans un petit cabinet que Franz n'avait pas encore vu, et qui était entouré de divans. Le comte vint au-devant de lui. 

«Eh! quel bon vent vous amène à cette heure, lui dit-il; viendriez-vous me demander à souper, par hasard? Ce serait pardieu bien aimable à vous. 

—Non, je viens pour vous parler d'une affaire grave. 

—D'une affaire! dit le comte en regardant Franz de ce regard profond qui lui était habituel; et de quelle affaire?  

—Sommes-nous seuls?» 

Le comte alla à la porte et revint. 

«Parfaitement seuls», dit-il. 

Franz lui présenta la lettre d'Albert. 

«Lisez», lui dit-il. 

Le comte lut la lettre. 

«Ah! ah! fit-il.  

—Avez-vous pris connaissance du post-scriptum? 

—Oui, dit-il, je vois bien: 

\begin{mail}{}{}
\textit{Se alle sei della mattina le quattro mille piastre non sono nelle mie mani, alle sette il comte Alberto avrà cessato di vivere.} [Si, à six heures du matin, les quatre mille piastres ne sont point entre mes mains, à sept heures, le vicomte Albert de Morcerf aura cessé d'exister.] 
\closeletter{Luigi Vampa}
\end{mail}

«Que dites-vous de cela? demanda Franz. 

—Avez-vous la somme qu'on vous a demandée? 

—Oui, moins huit cents piastres.» 

Le comte alla à son secrétaire, l'ouvrit, et faisant glisser un tiroir plein d'or: 

«J'espère, dit-il à Franz, que vous ne me ferez pas l'injure de vous adresser à un autre qu'à moi? 

—Vous voyez, au contraire, que je suis venu droit à vous, dit Franz. 

—Et je vous en remercie; prenez.» 

Et il fit signe à Franz de puiser dans le tiroir.  

«Est-il bien nécessaire d'envoyer cette somme à Luigi Vampa? demanda le jeune homme en regardant à son tour fixement le comte. 

—Dame! fit-il, jugez-en vous-même, le post-scriptum est précis. 

—Il me semble que si vous vous donniez la peine de chercher, vous trouveriez quelque moyen qui simplifierait beaucoup la négociation, dit Franz. 

—Et lequel? demanda le comte étonné. 

—Par exemple, si nous allions trouver Luigi Vampa ensemble, je suis sûr qu'il ne vous refuserait pas la liberté d'Albert. 

—À moi? et quelle influence voulez-vous que j'aie sur ce bandit? 

—Ne venez-vous pas de lui rendre un de ces services qui ne s'oublient point? 

—Et lequel? 

—Ne venez-vous pas de sauver la vie à Peppino? 

—Ah! ah! qui vous a dit cela? 

—Que vous importe? Je le sais.» 

Le comte resta un instant muet et les sourcils froncés.  

«Et si j'allais trouver Vampa, vous m'accompagneriez? 

—Si ma compagnie ne vous était pas trop désagréable. 

—Eh bien, soit; le temps est beau, une promenade dans la campagne de Rome ne peut que nous faire du bien. 

—Faut-il prendre des armes? 

—Pour quoi faire? 

—De l'argent? 

—C'est inutile. Où est l'homme qui a apporté ce billet?  

—Dans la rue. 

—Il attend la réponse? 

—Oui. 

—Il faut un peu savoir où nous allons; je vais l'appeler. 

—Inutile, il n'a pas voulu monter. 

—Chez vous, peut-être; mais, chez moi, il ne fera pas de difficultés.» 

Le comte alla à la fenêtre du cabinet qui donnait sur la rue, et siffla d'une certaine façon. L'homme au manteau se détacha de la muraille et s'avança jusqu'au milieu de la rue. 

«\textit{Salite!}» dit le comte, du ton dont il aurait donné un ordre à un domestique. 

Le messager obéit sans retard, sans hésitation, avec empressement même, et, franchissant les quatre marches du perron, entra dans l'hôtel. Cinq secondes après, il était à la porte du cabinet. 

«Ah! c'est toi, Peppino!» dit le comte. 

Mais Peppino, au lieu de répondre, se jeta à genoux, saisit la main du comte et y appliqua ses lèvres à plusieurs reprises.  

«Ah! ah! dit le comte, tu n'as pas encore oublié que je t'ai sauvé la vie! C'est étrange, il y a pourtant, aujourd'hui huit jours de cela. 

—Non, Excellence, et je ne l'oublierai jamais, répondit Peppino avec l'accent d'une profonde reconnaissance. 

—Jamais, c'est bien long! mais enfin c'est déjà beaucoup que tu le croies. Relève-toi et réponds.» 

Peppino jeta un coup d'œil inquiet sur Franz. 

«Oh! tu peux parler devant Son Excellence, dit-il, c'est un de mes amis. 

«Vous permettez que je vous donne ce titre, dit en français le comte en se tournant du côté de Franz; il est nécessaire pour exciter la confiance de cet homme. 

—Vous pouvez parler devant moi, reprit Franz, je suis un ami du comte. 

—À la bonne heure, dit Peppino en se retournant à son tour vers le comte; que Votre Excellence m'interroge, et je répondrai. 

—Comment le vicomte Albert est-il tombé entre les mains de Luigi? 

—Excellence, la calèche du Français a croisé plusieurs fois celle où était Teresa. 

—La maîtresse du chef? 

—Oui. Le Français lui a fait les yeux doux, Teresa s'est amusée à lui répondre; le Français lui a jeté des bouquets, elle lui en a rendu: tout cela, bien entendu, du consentement du chef, qui était dans la même calèche. 

—Comment! s'écria Franz, Luigi Vampa était dans la calèche des paysannes romaines? 

—C'était lui qui conduisait, déguisé en cocher, répondit Peppino.  

—Après? demanda le comte. 

—Eh bien, après, le Français se démasqua; Teresa toujours du consentement du chef, en fit autant; le Français demanda un rendez-vous, Teresa accorda le rendez-vous demandé; seulement, au lieu de Teresa, ce fut Beppo qui se trouva sur les marches de l'église San-Giacomo. 

—Comment! interrompit encore Franz, cette paysanne qui lui a arraché son moccoletto?\dots 

—C'était un jeune garçon de quinze ans, répondit Peppino; mais il n'y a pas de honte pour votre ami à y avoir été pris; Beppo en a attrapé bien d'autres, allez. 

—Et Beppo l'a conduit hors des murs? dit le comte. 

—Justement, une calèche attendait au bout de la via Macello; Beppo est monté dedans en invitant le Français à le suivre; il ne se l'est pas fait dire deux fois. Il a galamment offert la droite à Beppo, et s'est placé près de lui. Beppo lui a annoncé alors qu'il allait le conduire à une villa située à une lieue de Rome. Le Français a assuré Beppo qu'il était prêt à le suivre au bout du monde. Aussitôt le cocher a remonté la rue di Ripetta, a gagné la porte San-Paolo; et à deux cents pas dans la campagne, comme le Français devenait trop entreprenant, ma foi, Beppo lui a mis une paire de pistolets sur la gorge; aussitôt le cocher a arrêté ses chevaux, s'est retourné sur son siège et en a fait autant. En même temps quatre des nôtres, qui étaient cachés sur les bords de l'Almo, se sont élancés aux portières. Le Français avait bonne envie de se détendre, il a même un peu étranglé Beppo, à ce que j'ai entendu dire, mais il n'y avait rien à faire contre cinq hommes armés. Il a bien fallu se rendre; on l'a fait descendre de voiture, on a suivi les bords de la petite rivière, et on l'a conduit à Teresa et à Luigi, qui l'attendaient dans les catacombes de Saint-Sébastien. 

—Eh bien, mais, dit le comte en se tournant du côté de Franz, il me semble qu'elle en vaut bien une autre, cette histoire. Qu'en dites-vous, vous qui êtes connaisseur? 

—Je dis que je la trouverais fort drôle, répondit Franz, si elle était arrivée à un autre qu'à ce pauvre Albert. 

—Le fait est, dit le comte, que si vous ne m'aviez pas trouvé là, c'était une bonne fortune qui coûtait un peu cher à votre ami; mais, rassurez-vous, il en sera quitte pour la peur. 

—Et nous allons toujours le chercher? demanda Franz. 

—Pardieu! d'autant plus qu'il est dans un endroit fort pittoresque. Connaissez-vous les catacombes de Saint-Sébastien? 

—Non, je n'y suis jamais descendu, mais je me promettais d'y descendre un jour. 

—Eh bien, voici l'occasion toute trouvée et il serait difficile d'en rencontrer une autre meilleure. Avez-vous votre voiture? 

—Non. 

—Cela ne fait rien; on a l'habitude de m'en tenir une tout attelée, nuit et jour. 

—Tout attelée? 

—Oui, je suis un être fort capricieux; il faut vous dire que parfois en me levant, à la fin de mon dîner, au milieu de la nuit, il me prend l'envie de partir pour un point du monde quelconque, et je pars.» 

Le comte sonna un coup, son valet de chambre parut. 

«Faites sortir la voiture de la remise, dit-il, et ôtez en les pistolets qui sont dans les poches, il est inutile de réveiller le cocher, Ali conduira.» 

Au bout d'un instant on entendit le bruit de la voiture qui s'arrêtait devant la porte. 

Le comte tira sa montre. 

«Minuit et demi, dit-il, nous aurions pu partir d'ici à cinq heures du matin et arriver encore à temps; mais peut-être ce retard aurait-il fait passer une mauvaise nuit à votre compagnon, il vaut donc mieux aller tout courant le tirer des mains des infidèles. Êtes-vous toujours décidé à m'accompagner? 

—Plus que jamais. 

—Eh bien, venez alors.» 

Franz et le comte sortirent, suivis de Peppino. 

À la porte, ils trouvèrent la voiture. Ali était sur le siège. Franz reconnut l'esclave muet de la grotte de Monte-Cristo. 

Franz et le comte montèrent dans la voiture, qui était un coupé, Peppino se plaça près d'Ali, et l'on partit au galop. Ali avait reçu des ordres d'avance, car il prit la rue du Cours, traversa le Campo Vaccino, remonta la strada San-Gregorio et arriva à la porte Saint-Sébastien; là le concierge voulut faire quelques difficultés, mais le comte de Monte-Cristo présenta une autorisation du gouverneur de Rome d'entrer dans la ville et d'en sortir à toute heure du jour et de la nuit; la herse fut donc levée, le concierge reçut un louis pour sa peine, et l'on passa. 

La route que suivait la voiture était l'ancienne voie Appienne, toute bordée de tombeaux. De temps en temps, au clair de la lune qui commençait à se lever, il semblait à Franz voir comme une sentinelle se détacher d'une ruine, mais aussitôt, à un signe échangé entre Peppino et cette sentinelle, elle rentrait dans l'ombre et disparaissait. 

Un peu avant le cirque de Caracalla, la voiture s'arrêta, Peppino vint ouvrir la portière, et le comte et Franz descendirent. 

«Dans dix minutes, dit le comte à son compagnon, nous serons arrivés.» 

Puis il prit Peppino à part, lui donna un ordre tout bas, et Peppino partit après s'être muni d'une torche que l'on tira du coffre du coupé. 

Cinq minutes s'écoulèrent encore, pendant lesquelles Franz vit le berger s'enfoncer par un petit sentier au milieu des mouvements de terrain qui forment le sol convulsionné de la plaine de Rome, et disparaître dans ces hautes herbes rougeâtres qui semblent la crinière hérissée de quelque lion gigantesque.  

«Maintenant, dit le comte, suivons-le.» 

Franz et le comte s'engagèrent à leur tour dans le même sentier qui, au bout de cent pas, les conduisit par une pente inclinée au fond d'une petite vallée. 

Bientôt on aperçut deux hommes causant dans l'ombre. 

«Devons-nous continuer d'avancer? demanda Franz au comte, ou faut-il attendre? 

—Marchons; Peppino doit avoir prévenu la sentinelle de notre arrivée.» 

En effet, l'un de ces deux hommes était Peppino, l'autre était un bandit placé en vedette. 

Franz et le comte s'approchèrent; le bandit salua. 

«Excellence, dit Peppino en s'adressant au comte, si vous voulez me suivre, l'ouverture des catacombes est à deux pas d'ici. 

—C'est bien, dit le comte, marche devant.» 

En effet, derrière un massif de buissons et au milieu de quelques roches s'offrait une ouverture par laquelle un homme pouvait à peine passer. 

Peppino se glissa le premier par cette gerçure, mais à peine eut-il fait quelques pas que le passage souterrain s'élargit. Alors il s'arrêta, alluma sa torche et se retourna pour voir s'il était suivi. 

Le comte s'était engagé le premier dans une espèce de soupirail, et Franz venait après lui. 

Le terrain s'enfonçait par une pente douce et s'élargissait à mesure que l'on avançait; mais cependant Franz et le comte étaient encore forcés de marcher courbés et eussent eu peine à passer deux de front. Ils firent encore cent cinquante pas ainsi, puis ils furent arrêtés par le cri de: \textit{Qui vive}? 

En même temps ils virent au milieu de l'obscurité briller sur le canon d'une carabine le reflet de leur propre torche. 

«\textit{Ami}!» dit Peppino. 

Et il s'avança seul et dit quelques mots à voix basse à cette seconde sentinelle, qui, comme la première, salua en faisant signe aux visiteurs nocturnes qu'ils pouvaient continuer leur chemin. 

Derrière la sentinelle était un escalier d'une vingtaine de marches; Franz et le comte descendirent les vingt marches, et se trouvèrent dans une espèce de carrefour mortuaire. Cinq routes divergeaient comme les rayons d'une étoile, et les parois des murailles creusées de niches superposées ayant la forme de cercueils, indiquaient que l'on était entré enfin dans les catacombes. 

Dans l'une de ces cavités, dont il était impossible de distinguer l'étendue, on voyait, le jour, quelques reflets de lumière. 

Le comte posa la main sur l'épaule de Franz. 

«Voulez-vous voir un camp de bandits au repos? lui dit-il. 

—Certainement, répondit Franz. 

—Eh bien, venez avec moi\dots. Peppino, éteins la torche.» 

Peppino obéit, et Franz et le comte se trouvèrent dans la plus profonde obscurité; seulement, à cinquante pas à peu près en avant d'eux, continuèrent de danser le long des murailles quelques lueurs rougeâtres devenues encore plus visibles depuis que Peppino avait éteint sa torche. 

Ils avancèrent silencieusement, le comte guidant Franz comme s'il avait eu cette singulière faculté de voir dans les ténèbres. Au reste, Franz lui-même distinguait plus facilement son chemin à mesure qu'il s'approchait de ces reflets qui leur servaient de guides. 

Trois arcades, dont celle du milieu servait de porte, leur donnaient passage. 

Ces arcades s'ouvraient d'un côté sur le corridor où étaient le comte et Franz, et de l'autre sur une grande chambre carrée tout entourée de niches pareilles à celles dont nous avons déjà parlé. Au milieu de cette chambre s'élevaient quatre pierres qui autrefois avaient servi d'autel, comme l'indiquait la croix qui les surmontait encore. 

Une seule lampe, posée sur un fût de colonne, éclairait d'une lumière pâle et vacillante l'étrange scène qui s'offrait aux yeux des deux visiteurs cachés dans l'ombre. 

Un homme était assis, le coude appuyé sur cette colonne, et lisait, tournant le dos aux arcades par l'ouverture desquelles les nouveaux arrivés le regardaient. 

C'était le chef de la bande, Luigi Vampa.  

Tout autour de lui, groupés selon leur caprice, couchés dans leurs manteaux ou adossés à une espèce de banc de pierre qui régnait tout autour du columbarium, on distinguait une vingtaine de brigands; chacun avait sa carabine à portée de la main. 

Au fond, silencieuse, à peine visible et pareille à une ombre, une sentinelle se promenait de long en large devant une espèce d'ouverture qu'on ne distinguait que parce que les ténèbres semblaient plus épaisses en cet endroit. 

Lorsque le comte crut que Franz avait suffisamment réjoui ses regards de ce pittoresque tableau, il porta le doigt à ses lèvres pour lui recommander le silence, et montant les trois marches qui conduisaient du corridor au columbarium, il entra dans la chambre par l'arcade du milieu et s'avança vers Vampa, qui était si profondément plongé dans sa lecture qu'il n'entendit point le bruit de ses pas. 

«Qui vive?» cria la sentinelle moins préoccupée, et qui vit à la lueur de la lampe une espèce d'ombre qui grandissait derrière son chef. 

À ce cri Vampa se leva vivement, tirant du même coup un pistolet de sa ceinture. 

En un instant tous les bandits furent sur pied, et vingt canons de carabine se dirigèrent sur le comte. 

«Eh bien, dit tranquillement celui-ci d'une voix parfaitement calme et sans qu'un seul muscle de son visage bougeât; eh bien, mon cher Vampa, il me semble que voilà bien des frais pour recevoir un ami! 

—Armes bas!» cria le chef en faisant un signe impératif d'une main, tandis que de l'autre il ôtait respectueusement son chapeau. 

Puis se retournant vers le singulier personnage qui dominait toute cette scène: 

«Pardon, monsieur le comte, lui dit-il, mais j'étais si loin de m'attendre à l'honneur de votre visite, que je ne vous ai pas reconnu. 

—Il paraît que vous avez la mémoire courte en toute chose, Vampa, dit le comte, et que non seulement vous oubliez le visage des gens, mais encore les conditions faites avec eux. 

—Et quelles conditions ai-je donc oubliées, monsieur le comte? demanda le bandit en homme qui, s'il a commis une erreur, ne demande pas mieux que de la réparer. 

—N'a-t-il pas été convenu, dit le comte, que non seulement ma personne, mais encore celle de mes amis, vous seraient sacrées? 

—Et en quoi ai-je manqué au traité, Excellence? 

—Vous avez enlevé ce soir et vous avez transporté ici le vicomte Albert de Morcerf; eh bien, continua le comte avec un accent qui fit frissonner Franz, ce jeune homme est \textit{de mes amis}, ce jeune homme loge dans le même hôtel que moi, ce jeune homme a fait Corso pendant huit jours dans ma propre calèche, et cependant, je vous le répète, vous l'avez enlevé, vous l'avez transporté ici, et, ajouta le comte en tirant la lettre de sa poche, vous l'avez mis à rançon comme s'il était le premier venu. 

—Pourquoi ne m'avez-vous pas prévenu de cela, vous autres? dit le chef en se tournant vers ses hommes, qui reculèrent tous devant son regard; pourquoi m'avez-vous exposé ainsi à manquer à ma parole envers un homme comme M. le comte, qui tient notre vie à tous entre ses mains? Par le sang du Christ! si je croyais qu'un de vous eût su que le jeune homme était l'ami de Son Excellence, je lui brûlerais la cervelle de ma propre main.  

—Eh bien, dit le comte en se retournant du côté de Franz, je vous avais bien dit qu'il y avait quelque erreur là-dessous. 

—N'êtes-vous pas seul? demanda Vampa avec inquiétude. 

—Je suis avec la personne à qui cette lettre était adressée, et à qui j'ai voulu prouver que Luigi Vampa est un homme de parole. Venez, Excellence, dit-il à Franz, voilà Luigi Vampa qui va vous dire lui-même qu'il est désespéré de l'erreur qu'il vient de commettre.» 

Franz s'approcha; le chef fit quelques pas au-devant de Franz. 

«Soyez le bienvenu parmi nous, Excellence, lui dit-il; vous avez entendu ce que vient de dire le comte, et ce que je lui ai répondu: j'ajouterai que je ne voudrais pas, pour les quatre mille piastres auxquelles j'avais fixé la rançon de votre ami, que pareille chose fût arrivée. 

—Mais, dit Franz en regardant tout autour de lui avec inquiétude, où donc est le prisonnier? je ne le vois pas. 

—Il ne lui est rien arrivé, j'espère! demanda le comte en fronçant le sourcil. 

—Le prisonnier est là, dit Vampa en montrant de la main l'enfoncement devant lequel se promenait le bandit en faction, et je vais lui annoncer moi-même qu'il est libre.»  

Le chef s'avança vers l'endroit désigné par lui comme servant de prison à Albert, et Franz et le comte le suivirent. 

«Que fait le prisonnier? demanda Vampa à la sentinelle. 

—Ma foi, capitaine, répondit celle-ci, je n'en sais rien; depuis plus d'une heure, je ne l'ai pas entendu remuer. 

—Venez, Excellence!» dit Vampa. 

Le comte et Franz montèrent sept ou huit marches, toujours précédés par le chef, qui tira un verrou et poussa une porte. 

Alors, à la lueur d'une lampe pareille à celle qui éclairait le columbarium, on put voir Albert, enveloppé d'un manteau que lui avait prêté un des bandits, couché dans un coin et dormant du plus profond sommeil. 

«Allons! dit le comte souriant de ce sourire qui lui était particulier, pas mal pour un homme qui devait être fusillé à sept heures du matin.» 

Vampa regardait Albert endormi avec une certaine admiration; on voyait qu'il n'était pas insensible à cette preuve de courage. 

«Vous avez raison, monsieur le comte, dit-il, cet homme doit être de vos amis.» 

Puis s'approchant d'Albert et lui touchant l'épaule: 

«Excellence! dit-il, vous plaît-il de vous éveiller?» 

Albert étendit les bras, se frotta les paupières et ouvrit les yeux. 

«Ah! ah! dit-il, c'est vous, capitaine! pardieu, vous auriez bien dû me laisser dormir; je faisais un rêve charmant: je rêvais que je dansais le galop chez Torlonia avec la comtesse G\dots!» 

Il tira sa montre, qu'il avait gardée pour juger lui-même le temps écoulé. 

«Une heure et demie du matin! dit-il, mais pourquoi diable m'éveillez-vous à cette heure-ci? 

—Pour vous dire que vous êtes libre, Excellence. 

—Mon cher, reprit Albert avec une liberté d'esprit parfaite, retenez bien à l'avenir cette maxime de Napoléon le Grand: «Ne m'éveillez que pour les mauvaises nouvelles.» Si vous m'aviez laissé dormir, j'achevais mon galop, et je vous en aurais été reconnaissant toute ma vie\dots. On a donc payé ma rançon? 

—Non, Excellence. 

—Eh bien, alors, comment suis-je libre? 

—Quelqu'un, à qui je n'ai rien à refuser, est venu vous réclamer. 

—Jusqu'ici? 

—Jusqu'ici. 

—Ah! pardieu, ce quelqu'un-là est bien aimable!» 

Albert regarda tout autour de lui et aperçut Franz. 

«Comment, lui dit-il, c'est vous, mon cher Franz, qui poussez le dévouement jusque-là? 

—Non, pas moi, répondit Franz, mais notre voisin, M. le comte de Monte-Cristo. 

—Ah pardieu! monsieur le comte, dit gaiement Albert en rajustant sa cravate et ses manchettes, vous êtes un homme véritablement précieux, et j'espère que vous me regarderez comme votre éternel obligé, d'abord pour l'affaire de la voiture, ensuite pour celle-ci!» et il tendit la main au comte, qui frissonna au moment de lui donner la sienne, mais qui cependant la lui donna. 

Le bandit regardait toute cette scène d'un air stupéfait; il était évidemment habitué à voir ses prisonniers trembler devant lui, et voilà qu'il y en avait un dont l'humeur railleuse n'avait subi aucune altération: quant à Franz, il était enchanté qu'Albert eût soutenu, même vis-à-vis d'un bandit, l'honneur national.  

«Mon cher Albert, lui dit-il, si vous voulez vous hâter, nous aurons encore le temps d'aller finir la nuit chez Torlonia; vous prendrez votre galop où vous l'avez interrompu, de sorte que vous ne garderez aucune rancune au seigneur Luigi, qui s'est véritablement, dans toute cette affaire, conduit en galant homme. 

—Ah! vraiment, dit-il, vous avez raison, et nous pourrons y être à deux heures. Seigneur Luigi, continua Albert, y a-t-il quelque autre formalité à remplir pour prendre congé de Votre Excellence? 

—Aucune, monsieur, répondit le bandit, et vous êtes libre comme l'air. 

—En ce cas, bonne et joyeuse vie; venez, messieurs, venez! 

Et Albert, suivi de Franz et du comte, descendit l'escalier et traversa la grande salle carrée; tous les bandits étaient debout et le chapeau à la main. 

«Peppino, dit le chef, donne-moi la torche. 

—Eh bien, que faites-vous donc? demanda le comte. 

—Je vous reconduis, dit le capitaine; c'est bien le moindre honneur que je puisse rendre à Votre Excellence.» 

Et prenant la torche allumée des mains du pâtre, il marcha devant ses hôtes, non pas comme un valet qui accomplit une œuvre de servilité, mais comme un roi qui précède des ambassadeurs. 

Arrivé à la porte il s'inclina. 

«Et maintenant, monsieur le comte, dit-il, je vous renouvelle mes excuses, et j'espère que vous ne me gardez aucun ressentiment de ce qui vient d'arriver? 

—Non, mon cher Vampa, dit le comte; d'ailleurs vous rachetez vos erreurs d'une façon si galante, qu'on est presque tenté de vous savoir gré de les avoir commises. 

—Messieurs! reprit le chef en se retournant du côté des jeunes gens, peut-être l'offre ne vous paraîtra-t-elle pas bien attrayante; mais, s'il vous prenait jamais envie de me faire une seconde visite, partout où je serai vous serez les bienvenus.» 

Franz et Albert saluèrent. Le comte sortit le premier, Albert ensuite, Franz restait le dernier. 

«Votre Excellence a quelque chose à me demander? dit Vampa en souriant. 

—Oui, je l'avoue, répondit Franz, je serais curieux de savoir quel était l'ouvrage que vous lisiez avec tant d'attention quand nous sommes arrivés. 

—Les \textit{Commentaires de César}, dit le bandit, c'est mon livre de prédilection.  

—Eh bien, ne venez-vous pas? demanda Albert. 

—Si fait, répondit Franz, me voilà!» 

Et il sortit à son tour du soupirail. 

On fit quelques pas dans la plaine. 

«Ah! pardon! dit Albert en revenant en arrière, voulez-vous permettre, capitaine? 

Et il alluma son cigare à la torche de Vampa. 

«Maintenant, monsieur le comte, dit-il, la plus grande diligence possible! je tiens énormément à aller finir ma nuit chez le duc de Bracciano.» 

On retrouva la voiture où on l'avait laissée; le comte dit un seul mot arabe à Ali, et les chevaux partirent à fond de train. 

Il était deux heures juste à la montre d'Albert quand les deux amis rentrèrent dans la salle de danse. 

Leur retour fit événement; mais, comme ils entraient ensemble, toutes les inquiétudes que l'on avait pu concevoir sur Albert cessèrent à l'instant même. 

«Madame, dit le vicomte de Morcerf en s'avançant vers la comtesse, hier vous avez eu la bonté de me promettre un galop, je viens un peu tard réclamer cette gracieuse promesse; mais voilà mon ami, dont vous connaissez la véracité, qui vous affirmera qu'il n'y a pas de ma faute.» 

Et comme en ce moment la musique donnait le signal de la valse, Albert passa son bras autour de la taille de la comtesse et disparut avec elle dans le tourbillon des danseurs. 

Pendant ce temps Franz songeait au singulier frissonnement qui avait passé par tout le corps du comte de Monte-Cristo au moment où il avait été en quelque sorte forcé de donner la main à Albert. 