\chapter{Haydée} 

\lettrine{\accentletter[\gravebox]{A}}{} peine les chevaux du comte avaient-ils tourné l'angle du boulevard, qu'Albert se retourna vers le comte en éclatant d'un rire trop bruyant pour ne pas être un peu forcé. 

\zz
«Eh bien, lui dit-il, je vous demanderai, comme le roi Charles IX demandait à Catherine de Médicis après la Saint-Barthélemy: Comment trouvez-vous que j'ai joué mon petit rôle?» 

—À quel propos? demanda Monte-Cristo. 

—Mais à propos de l'installation de mon rival chez M. Danglars\dots. 

—Quel rival? 

—Parbleu! quel rival? votre protégé, M. Andrea Cavalcanti! 

—Oh! pas de mauvaises plaisanteries, vicomte; je ne protège nullement M. Andrea, du moins près de M. Danglars. 

—Et c'est le reproche que je vous ferais si le jeune homme avait besoin de protection. Mais, heureusement pour moi, il peut s'en passer. 

—Comment! vous croyez qu'il fait sa cour? 

—Je vous en réponds: il roule des yeux de soupirant et module des sons d'amoureux; il aspire à la main de la fière Eugénie. Tiens, je viens de faire un vers! Parole d'honneur, ce n'est pas de ma faute. N'importe, je le répète: il aspire à la main de la fière Eugénie. 

—Qu'importe, si l'on ne pense qu'à vous? 

—Ne dites pas cela, mon cher comte; on me rudoie des deux côtés. 

—Comment, des deux côtés? 

—Sans doute: Mlle Eugénie m'a répondu à peine, et Mlle d'Armilly, sa confidente, ne m'a pas répondu du tout. 

—Oui, mais le père vous adore, dit Monte-Cristo. 

—Lui? mais au contraire, il m'a enfoncé mille poignards dans le cœur; poignards rentrant dans le manche, il est vrai, poignards de tragédie, mais qu'il croyait bel et bien réels. 

—La jalousie indique l'affection. 

—Oui, mais je ne suis pas jaloux. 

—Il l'est, lui. 

—De qui? de Debray? 

—Non, de vous. 

—De moi? je gage qu'avant huit jours il m'a fermé la porte au nez. 

—Vous vous trompez, mon cher vicomte. 

—Une preuve? 

—La voulez-vous? 

—Oui. 

—Je suis chargé de prier M. le comte de Morcerf de faire une démarche définitive près du baron. 

—Par qui? 

—Par le baron lui-même. 

—Oh! dit Albert avec toute la câlinerie dont il était capable, vous ne ferez pas cela, n'est-ce pas, mon cher comte? 

—Vous vous trompez, Albert, je le ferai, puisque j'ai promis. 

—Allons, dit Albert avec un soupir, il paraît que vous tenez absolument à me marier. 

—Je tiens à être bien avec tout le monde; mais, à propos de Debray, je ne le vois plus chez la baronne. 

—Il y a de la brouille. 

—Avec madame? 

—Non, avec monsieur. 

—Il s'est donc aperçu de quelque chose? 

—Ah! la bonne plaisanterie! 

—Vous croyez qu'il s'en doutait? fit Monte-Cristo avec une naïveté charmante. 

—Ah çà! mais, d'où venez-vous donc, mon cher comte? 

—Du Congo, si vous voulez. 

—Ce n'est pas d'assez loin encore. 

—Est-ce que je connais vos maris parisiens? 

—Eh! mon cher comte, les maris sont les mêmes partout; du moment où vous avez étudié l'individu dans un pays quelconque, vous connaissez la race. 

—Mais alors quelle cause a pu brouiller Danglars et Debray? Ils paraissaient si bien s'entendre, dit Monte-Cristo avec un renouvellement de naïveté. 

—Ah! voilà! nous rentrons dans les mystères d'Isis, et je ne suis pas initié. Quand M. Cavalcanti fils sera de la famille, vous lui demanderez cela. 

La voiture s'arrêta. 

«Nous voilà arrivés, dit Monte-Cristo; il n'est que dix heures et demie, montez donc. 

—Bien volontiers. 

—Ma voiture vous conduira. 

—Non, merci, mon coupé a dû nous suivre. 

—En effet, le voilà», dit Monte-Cristo en sautant à terre. 

Tous deux entrèrent dans la maison; le salon était éclairé, ils y entrèrent. 

«Vous allez nous faire du thé, Baptistin», dit Monte-Cristo. 

Baptistin sortit sans souffler le mot. Deux secondes après, il reparut avec un plateau tout servi, et qui, comme les collations des pièces féeriques, semblait sortir de terre. 

«En vérité, dit Morcerf, ce que j'admire en vous, mon cher comte, ce n'est pas votre richesse, peut-être y a-t-il des gens plus riches que vous; ce n'est pas votre esprit, Beaumarchais n'en avait pas plus, mais il en avait autant; c'est votre manière d'être servi, sans qu'on vous réponde un mot, à la minute, à la seconde, comme si l'on devinait, à la manière dont vous sonnez, ce que vous désirez avoir, et comme si ce que vous désirez avoir était toujours tout prêt. 

—Ce que vous dites est un peu vrai. On sait mes habitudes. Par exemple, vous allez voir: ne désirez-vous pas faire quelque chose en buvant votre thé? 

—Pardieu, je désire fumer.» 

Monte-Cristo s'approcha du timbre et frappa un coup. 

Au bout d'une seconde, une porte particulière s'ouvrit, et Ali parut avec deux chibouques toutes bourrées d'excellent latakié. 

«C'est merveilleux, dit Morcerf. 

—Mais non, c'est tout simple, reprit Monte-Cristo; Ali sait qu'en prenant le thé ou le café je fume ordinairement: il sait que j'ai demandé le thé, il sait que je suis rentré avec vous, il entend que je l'appelle, il se doute de la cause, et comme il est d'un pays où l'hospitalité s'exerce avec la pipe surtout, au lieu d'une chibouque, il en apporte deux. 

—Certainement, c'est une explication comme une autre; mais il n'en est pas moins vrai qu'il n'y a que vous\dots. Oh! mais, qu'est-ce que j'entends?» 

Et Morcerf s'inclina vers la porte par laquelle entraient effectivement des sons correspondant à ceux d'une guitare. 

«Ma foi, mon cher vicomte, vous êtes voué à la musique, ce soir; vous n'échappez au piano de Mlle Danglars que pour tomber dans la guzla d'Haydée. 

—Haydée! quel adorable nom! Il y a donc des femmes qui s'appellent véritablement Haydée autre part que dans les poèmes de Lord Byron? 

—Certainement, Haydée est un nom fort rare en France, mais assez commun en Albanie et en Épire; c'est comme si vous disiez, par exemple, chasteté, pudeur, innocence; c'est une espèce de nom de baptême, comme disent vos Parisiens. 

—Oh! que c'est charmant! dit Albert, comme je voudrais voir nos Françaises s'appeler Mlle Bonté, Mlle Silence, Mlle Charité chrétienne! Dites donc, si Mlle Danglars, au lieu de s'appeler Claire-Marie-Eugénie, comme on la nomme, s'appelait Mlle Chasteté-Pudeur-Innocence Danglars, peste, quel effet cela ferait dans une publication de bans! 

—Fou! dit le comte, ne plaisantez pas si haut, Haydée pourrait vous entendre. 

—Et elle se fâcherait? 

—Non pas, dit le comte avec son air hautain. 

—Elle est bonne personne? demanda Albert. 

—Ce n'est pas bonté, c'est devoir: une esclave ne se lâche pas contre son maître. 

—Allons donc! ne plaisantez pas vous-même. Est-ce qu'il y a encore des esclaves? 

—Sans doute, puisque Haydée est la mienne. 

—En effet, vous ne faites rien et vous n'avez rien comme un autre, vous. Esclave de M. le comte de Monte-Cristo! c'est une position en France. À la façon dont vous remuez l'or, c'est une place qui doit valoir cent mille écus par an. 

—Cent mille écus! la pauvre enfant a possédé plus que cela; elle est venue au monde couchée sur des trésors près desquels ceux des \textit{Mille et une Nuits} sont bien peu de chose. 

—C'est donc vraiment une princesse? 

—Vous l'avez dit, et même une des plus grandes de son pays. 

—Je m'en étais douté. Mais comment une grande princesse est-elle devenue esclave? 

—Comment Denys le Tyran est-il devenu maître d'école? le hasard de la guerre, mon cher vicomte, le caprice de la fortune. 

—Et son nom est un secret? 

—Pour tout le monde, oui; mais pas pour vous, cher vicomte, qui êtes de mes amis, et qui vous tairez, n'est-ce pas, si vous me promettez de vous taire? 

—Oh! parole d'honneur! 

—Vous connaissez l'histoire du pacha de Janina? 

—D'Ali-Tebelin? sans doute, puisque c'est à son service que mon père a fait fortune. 

—C'est vrai, je l'avais oublié. 

—Eh bien, qu'est Haydée à Ali-Tebelin? 

—Sa fille tout simplement. 

—Comment! la fille d'Ali-Pacha?  

—Et de la belle Vasiliki. 

—Et elle est votre esclave? 

—Oh! mon Dieu, oui. 

—Comment cela? 

—Dame! un jour que je passais sur le marché de Constantinople, je l'ai achetée. 

—C'est splendide! Avec vous, mon cher comte, on ne vit pas, on rêve. Maintenant, écoutez, c'est bien indiscret ce que je vais vous demander là. 

—Dites toujours. 

—Mais puisque vous sortez avec elle, puisque vous la conduisez à l'Opéra\dots. 

—Après? 

—Je puis bien me risquer à vous demander cela? 

—Vous pouvez vous risquer à tout me demander. 

—Eh bien, mon cher comte, présentez-moi à votre princesse. 

—Volontiers, mais à deux conditions. 

—Je les accepte d'avance.  

—La première, c'est que vous ne confierez jamais à personne cette présentation. 

—Très bien (Morcerf étendit la main). Je le jure. 

—La seconde, c'est que vous ne lui direz pas que votre père a servi le sien. 

—Je le jure encore. 

—À merveille, vicomte, vous vous rappellerez ces deux serments, n'est-ce pas? 

—Oh! fit Albert. 

—Très bien. Je vous sais homme d'honneur.» 

Le comte frappa de nouveau sur le timbre; Ali reparut. 

«Préviens Haydée, lui dit-il, que je vais aller prendre le café chez elle, et fais-lui comprendre que je demande la permission de lui présenter un de mes amis.» 

Ali s'inclina et sortit. 

«Ainsi, c'est convenu, pas de questions directes, cher vicomte. Si vous désirez savoir quelque chose, demandez-le à moi, et je le demanderai à elle. 

—C'est convenu.»  

Ali reparut pour la troisième fois et tint la portière soulevée, pour indiquer à son maître et à Albert qu'ils pouvaient passer. 

«Entrons», dit Monte-Cristo. 

Albert passa une main dans ses cheveux et frisa sa moustache, le comte reprit son chapeau, mit ses gants et précéda Albert dans l'appartement que gardait, comme une sentinelle avancée, Ali, et que défendaient, comme un poste, les trois femmes de chambre françaises commandées par Myrtho. 

Haydée attendait dans la première pièce, qui était le salon, avec de grands yeux dilatés par la surprise; car c'était la première fois qu'un autre homme que Monte-Cristo pénétrait jusqu'à elle; elle était assise sur un sofa, dans un angle, les jambes croisées sous elle, et s'était fait, pour ainsi dire, un nid, dans les étoffes de soie rayées et brodées les plus riches de l'Orient. Près d'elle était l'instrument dont les sons l'avaient dénoncée; elle était charmante ainsi. 

En apercevant Monte-Cristo, elle se souleva avec ce double sourire de fille et d'amante qui n'appartenait qu'à elle; Monte-Cristo alla à elle et lui tendit sa main sur laquelle, comme d'habitude, elle appuya ses lèvres. 

Albert était resté près de la porte, sous l'empire de cette beauté étrange qu'il voyait pour la première fois, et dont on ne pouvait se faire aucune idée en France. 

«Qui m'amènes-tu? demanda en romaïque la jeune fille à Monte-Cristo; un frère, un ami, une simple connaissance, ou un ennemi? 

—Un ami, dit Monte-Cristo dans la même langue. 

—Son nom? 

—Le comte Albert; c'est le même que j'ai tiré des mains des bandits, à Rome. 

—Dans quelle langue veux-tu que je lui parle?» 

Monte-Cristo se retourna vers Albert: 

«Savez-vous le grec moderne? demanda-t-il au jeune homme. 

—Hélas! dit Albert, pas même le grec ancien, mon cher comte, jamais Homère et Platon n'ont eu de plus pauvre, et j'oserai même dire de plus dédaigneux écolier. 

—Alors, dit Haydée, prouvant par la demande qu'elle faisait elle-même qu'elle venait d'entendre la question de Monte-Cristo et la réponse d'Albert, je parlerai en français ou en italien, si toutefois mon seigneur veut que je parle.» 

Monte-Cristo réfléchit un instant: 

«Tu parleras en italien», dit-il. 

Puis se tournant vers Albert: 

«C'est fâcheux que vous n'entendiez pas le grec moderne ou le grec ancien, qu'Haydée parle tous deux admirablement; la pauvre enfant va être forcée de vous parler italien, ce qui vous donnera peut-être une fausse idée d'elle.» 

Il fit un signe à Haydée. 

«Sois le bienvenu, ami, qui viens avec mon seigneur et maître, dit la jeune fille en excellent toscan, avec ce doux accent romain qui fait la langue de Dante aussi sonore que la langue d'Homère; Ali! du café et des pipes!» 

Et Haydée fit de la main signe à Albert de s'approcher, tandis qu'Ali se retirait pour exécuter les ordres de sa jeune maîtresse. 

Monte-Cristo montra à Albert deux pliants, et chacun alla chercher le sien pour l'approcher d'une espèce de guéridon, dont un narguilé faisait le centre, et que chargeaient des fleurs naturelles, des dessins, des albums de musique. 

Ali rentra, apportant le café et les chibouques; quant à M. Baptistin, cette partie de l'appartement lui était interdite. 

Albert repoussa la pipe que lui présentait le Nubien. 

«Oh! prenez, prenez, dit Monte-Cristo; Haydée est presque aussi civilisée qu'une Parisienne: le havane lui est désagréable, parce qu'elle n'aime pas les mauvaises odeurs; mais le tabac d'Orient est un parfum, vous le savez.» 

Ali sortit. 

Les tasses de café étaient préparées; seulement on avait, pour Albert, ajouté un sucrier. Monte-Cristo et Haydée prenaient la liqueur arabe à la manière des Arabes, c'est-à-dire sans sucre. 

Haydée allongea la main et prit du bout de ses petits doigts roses et effilés la tasse de porcelaine du Japon, qu'elle porta à ses lèvres avec le naïf plaisir d'un enfant qui boit ou mange une chose qu'il aime. 

En même temps deux femmes entrèrent, portant deux autres plateaux chargés de glaces et de sorbets, qu'elles déposèrent sur deux petites tables destinées à cet usage. 

«Mon cher hôte, et vous, signora, dit Albert en italien, excusez ma stupéfaction. Je suis tout étourdi, et c'est assez naturel; voici que je retrouve l'Orient, l'Orient véritable, non point malheureusement tel que je l'ai vu, mais tel que je l'ai rêvé au sein de Paris; tout à l'heure j'entendais rouler des omnibus et tinter les sonnettes des marchands de limonades. Ô signora!\dots que ne sais-je parler le grec, votre conversation jointe à cet entourage féerique, me composerait une soirée dont je me souviendrais toujours. 

—Je parle assez bien l'italien pour parler avec vous, monsieur, dit tranquillement Haydée; et je ferai de mon mieux, si vous aimez l'Orient, pour que vous le retrouviez ici. 

—De quoi puis-je parler? demanda tout bas Albert à Monte-Cristo. 

—Mais de tout ce que vous voudrez: de son pays, de sa jeunesse, de ses souvenirs; puis, si vous l'aimez mieux, de Rome, de Naples ou de Florence. 

—Oh! dit Albert, ce ne serait pas la peine d'avoir une Grecque devant soi pour lui parler de tout ce dont on parlerait à une Parisienne; laissez-moi lui parler de l'Orient. 

—Faites, mon cher Albert, c'est la conversation qui lui est la plus agréable.» 

Albert se retourna vers Haydée. 

«À quel âge la signora a-t-elle quitté la Grèce? demanda-t-il. 

—À cinq ans, répondit Haydée. 

—Et vous vous rappelez votre patrie? demanda Albert. 

—Quand je ferme les yeux, je revois tout ce que j'ai vu. Il y a deux regards: le regard du corps et le regard de l'âme. Le regard du corps peut oublier parfois, mais celui de l'âme se souvient toujours. 

—Et quel est le temps le plus loin dont vous puissiez vous souvenir? 

—Je marchais à peine, ma mère, que l'on appelle Vasiliki (Vasiliki veut dire royale, ajouta la jeune fille en relevant la tête), ma mère me prenait par la main, et, toutes deux couvertes d'un voile, après avoir mis au fond de la bourse tout l'or que nous possédions, nous allions demander l'aumône pour les prisonniers, en disant: 

«Celui qui donne aux pauvres prête à l'Éternel.»\footnote{Proverbe XIX.}  

«Puis, quand notre bourse était pleine, nous rentrions au palais, et, sans rien dire à mon père, nous envoyions tout cet argent qu'on nous avait donné, nous prenant pour de pauvres femmes, à l'égoumenos\footnote{En grec, prêtre, abbé (Note du correcteur.)} du couvent qui le répartissait entre les prisonniers.  

—Et à cette époque, quel âge aviez-vous? 

—Trois ans, dit Haydée. 

—Alors, vous vous souvenez de tout ce qui s'est passé autour de vous depuis l'âge de trois ans? 

—De tout. 

—Comte, dit tout bas Morcerf à Monte-Cristo, vous devriez permettre à la signora de nous raconter quelque chose de son histoire. Vous m'avez défendu de lui parler de mon père, mais peut-être m'en parlera-t-elle, et vous n'avez pas idée combien je serais heureux d'entendre sortir son nom d'une si jolie bouche.» 

Monte-Cristo se tourna vers Haydée, et par un signe de sourcil qui lui indiquait d'accorder la plus grande attention à la recommandation qu'il allait lui faire, il lui dit en grec: Πατροξ μεν ατην, μη δε ονομ προδοτου χαι προδοσιαν, ειπε ημιν.\footnote{Mot à mot: «De ton père le sort, mais pas le nom du traître, ni la trahison, raconte-nous.}

Haydée poussa un long soupir, et un nuage sombre passa sur son front si pur. 

«Que lui dites-vous? demanda tout bas Morcerf. 

—Je lui répète que vous êtes un ami, et qu'elle n'a point à se cacher vis-à-vis de vous. 

—Ainsi, dit Albert, ce vieux pèlerinage pour les prisonniers est votre premier souvenir; quel est l'autre? 

—L'autre? je me vois sous l'ombre des sycomores, près d'un lac dont j'aperçois encore, à travers le feuillage, le miroir tremblant; contre le plus vieux et le plus touffu, mon père était assis sur des coussins, et moi, faible enfant, tandis que ma mère était couchée à ses pieds, je jouais avec sa barbe blanche qui descendait sur sa poitrine, et avec le cangiar à la poignée de diamant passé à sa ceinture; puis, de temps en temps venait à lui un Albanais qui lui disait quelques mots auxquels je ne faisais pas attention, et auxquels il répondait du même son de voix: «Tuez!» ou: «Faites grâce!» 

—C'est étrange, dit Albert, d'entendre sortir de pareilles choses de la bouche d'une jeune fille, autre part que sur un théâtre, et en se disant: Ceci n'est point une fiction. Et, demanda Albert, comment, avec cet horizon si poétique, comment, avec ce lointain merveilleux, trouvez-vous la France? 

—Je crois que c'est un beau pays, dit Haydée, mais je vois la France telle qu'elle est, car je la vois avec des yeux de femme, tandis qu'il me semble, au contraire, que mon pays, que je n'ai vu qu'avec des yeux d'enfant, est toujours enveloppé d'un brouillard lumineux ou sombre, selon que mes yeux le font une douce patrie ou un lieu d'amères souffrances. 

—Si jeune, signora, dit Albert cédant malgré lui à la puissance de la banalité, comment avez-vous pu souffrir?» 

Haydée tourna les yeux vers Monte-Cristo, qui, avec un signe imperceptible, murmura:  

 —Εἰπέ\footnote{Raconte.}   

—Rien ne compose le fond de l'âme comme les premiers souvenirs, et, à part les deux que je viens de vous dire, tous les souvenirs de ma jeunesse sont tristes. 

—Parlez, parlez, signora, dit Albert, je vous jure que je vous écoute avec un inexprimable bonheur.» 

Haydée sourit tristement. 

«Vous voulez donc que je passe à mes autres souvenirs? dit-elle. 

—Je vous en supplie, dit Albert. 

—Eh bien, j'avais quatre ans quand, un soir, je fus réveillée par ma mère. Nous étions au palais de Janina; elle me prit sur les coussins où je reposais, et, en ouvrant mes yeux, je vis les siens remplis de grosses larmes. 

«Elle m'emporta sans rien dire. 

«En la voyant pleurer, j'allais pleurer aussi. 

«—Silence! enfant, dit-elle. 

«Souvent, malgré les consolations ou les menaces maternelles, capricieuse comme tous les enfants, je continuais de pleurer; mais, cette fois, il y avait dans la voix de ma pauvre mère une telle intonation de terreur, que je me tus à l'instant même. 

«Elle m'emportait rapidement. 

«Je vis alors que nous descendions un large escalier; devant nous, toutes les femmes de ma mère, portant des coffres, des sachets, des objets de parure, des bijoux, des bourses d'or, descendaient le même escalier ou plutôt se précipitaient. 

«Derrière les femmes venait une garde de vingt hommes, armés de longs fusils et de pistolets, et revêtus de ce costume que vous connaissez en France depuis que la Grèce est redevenue une nation. 

«Il y avait quelque chose de sinistre, croyez-moi, ajouta Haydée en secouant la tête et en pâlissant à cette seule mémoire, dans cette longue file d'esclaves et de femmes à demi alourdies par le sommeil, ou du moins je me le figurais ainsi, moi, qui peut-être croyais les autres endormis parce que j'étais mal réveillée. 

«Dans l'escalier couraient des ombres gigantesques que les torches de sapin faisaient trembler aux voûtes. 

«—Qu'on se hâte! dit une voix au fond de la galerie. 

«Cette voix fit courber tout le monde, comme le vent en passant sur la plaine fait courber un champ d'épis. 

«Moi, elle me fit tressaillir. 

«Cette voix, c'était celle de mon père. 

«Il marchait le dernier, revêtu de ses splendides habits, tenant à la main sa carabine que votre empereur lui avait donnée; et, appuyé sur son favori Sélim, il nous poussait devant lui comme un pasteur fait d'un troupeau éperdu. 

«—Mon père, dit Haydée en relevant la tête, était un homme illustre que l'Europe a connu sous le nom d'Ali-Tebelin, pacha de Janina, et devant lequel la Turquie a tremblé.» 

Albert, sans savoir pourquoi, frissonna en entendant ces paroles prononcées avec un indéfinissable accent de hauteur et de dignité; il lui sembla que quelque chose de sombre et d'effrayant rayonnait dans les yeux de la jeune fille, lorsque, pareille à une pythonisse qui évoque un spectre, elle réveilla le souvenir de cette sanglante figure que sa mort terrible fit apparaître gigantesque aux yeux de l'Europe contemporaine. 

«Bientôt, continua Haydée, la marche s'arrêta; nous étions au bas de l'escalier et au bord d'un lac. Ma mère me pressait contre sa poitrine bondissante, et je vis, à deux pas derrière, mon père qui jetait de tous côtés des regards inquiets. 

«Devant nous s'étendaient quatre degrés de marbre, et au bas du dernier degré ondulait une barque. 

«D'où nous étions on voyait se dresser au milieu d'un lac une masse noire; c'était le kiosque où nous nous rendions. 

«Ce kiosque me paraissait à une distance considérable, peut-être à cause de l'obscurité. 

«Nous descendîmes dans la barque. Je me souviens que les rames ne faisaient aucun bruit en touchant l'eau; je me penchai pour les regarder: elles étaient enveloppées avec les ceintures de nos Palicares. 

«Il n'y avait, outre les rameurs, dans la barque, que des femmes, mon père, ma mère, Sélim et moi. 

«Les Palicares étaient restés au bord du lac, agenouillés sur le dernier degré, et se faisant, dans le cas où ils eussent été poursuivis, un rempart des trois autres. 

«Notre barque allait comme le vent. 

«—Pourquoi la barque va-t-elle si vite? demandai-je à ma mère. 

«—Chut! mon enfant, dit-elle, c'est que nous fuyons.» 

«Je ne compris pas. Pourquoi mon père fuyait-il, lui le tout-puissant, lui devant qui d'ordinaire fuyaient les autres, lui qui avait pris pour devise: \textit{Ils me haïssent, donc ils me craignent?}  

«En effet, c'était une fuite que mon père opérait sur le lac. Il m'a dit depuis que la garnison du château de Janina, fatiguée d'un long service\dots.» 

Ici Haydée arrêta son regard expressif sur Monte-Cristo, dont l'œil ne quitta plus ses yeux. La jeune fille continua donc lentement, comme quelqu'un qui invente ou qui supprime. 

«Vous disiez, signora, reprit Albert, qui accordait la plus grande attention à ce récit, que la garnison de Janina, fatiguée d'un long service. 

—Avait traité avec le séraskier Kourchid, envoyé par le sultan pour s'emparer de mon père; c'était alors que mon père avait pris la résolution de se retirer, après avoir envoyé au sultan un officier franc, auquel il avait toute confiance, dans l'asile que lui-même s'était préparé depuis longtemps, et qu'il appelait \textit{kataphygion}, c'est-à-dire son refuge. 

—Et cet officier, demanda Albert, vous rappelez-vous son nom, signora?» 

Monte-Cristo échangea avec la jeune fille un regard rapide comme un éclair, et qui resta inaperçu de Morcerf. 

«Non, dit-elle, je ne me le rappelle pas; mais peut-être plus tard me le rappellerai-je, et je le dirai.» 

Albert allait prononcer le nom de son père, lorsque Monte-Cristo leva doucement le doigt en signe de silence; le jeune homme se rappela son serment et se tut. 

«C'était vers ce kiosque que nous voguions. 

«Un rez-de-chaussée orné d'arabesques, baignant ses terrasses dans l'eau, et un premier étage donnant sur le lac, voici tout ce que le palais offrait de visible aux yeux. 

«Mais au-dessous du rez-de-chaussée, se prolongeant dans l'île, était un souterrain, vaste caverne où l'on nous conduisit, ma mère, moi et nos femmes, et où gisaient, formant un seul monceau, soixante mille bourses et deux cents tonneaux; il y avait dans ces bourses vingt-cinq millions en or, et dans les barils trente mille livres de poudre. 

«Près de ces barils se tenait Sélim, ce favori de mon père dont je vous ai parlé; il veillait jour et nuit, une lance au bout de laquelle brillait une mèche allumée à la main; il avait l'ordre de faire tout sauter, kiosque, gardes, pacha, femmes et or, au premier signe de mon père. 

«Je me rappelle que nos esclaves, connaissant ce redoutable voisinage, passaient les jours et les nuits à prier, à pleurer, à gémir. 

«Quant à moi, je vois toujours le jeune soldat au teint pâle et à l'œil noir; et quand l'ange de la mort descendra vers moi, je suis sûre que je reconnaîtrai Sélim. 

«Je ne pourrais dire combien de temps nous restâmes ainsi: à cette époque j'ignorais encore ce que c'était que le temps; quelquefois, mais rarement, mon père nous faisait appeler, ma mère et moi, sur la terrasse du palais; c'étaient mes heures de plaisir à moi qui ne voyais dans le souterrain que des ombres gémissantes et la lance enflammée de Sélim. Mon père, assis devant une grande ouverture, attachait un regard sombre sur les profondeurs de l'horizon, interrogeant chaque point noir qui apparaissait sur le lac, tandis que ma mère, à demi couchée près de lui, appuyait sa tête sur son épaule, et que, moi, je jouais à ses pieds, admirant, avec ces étonnements de l'enfance qui grandissent encore les objets, les escarpements du Pinde, qui se dressait à l'horizon, les châteaux de Janina, sortant blancs et anguleux des eaux bleues du lac, les touffes immenses de verdures noires, attachées comme des lichens aux rocs de la montagne, qui de loin semblaient des mousses, et qui de près sont des sapins gigantesques et des myrtes immenses. 

«Un matin, mon père nous envoya chercher, nous le trouvâmes assez calme, mais plus pâle que d'habitude. 

«—Prends patience, Vasiliki, aujourd'hui tout sera fini; aujourd'hui arrive le firman du maître, et mon sort sera décidé. Si la grâce est entière, nous retournerons triomphants à Janina; si la nouvelle est mauvaise, nous fuirons cette nuit. 

«—Mais s'ils ne nous laissent pas fuir? dit ma mère. 

«—Oh! sois tranquille, répondit Ali en souriant; Sélim et sa lance allumée me répondent d'eux. Ils voudraient que je fusse mort, mais pas à la condition de mourir avec moi. 

«Ma mère ne répondit que par des soupirs à ces consolations, qui ne partaient pas du cœur de mon père. 

«Elle lui prépara l'eau glacée qu'il buvait à chaque instant, car, depuis sa retraite dans le kiosque, il était brûlé par une fièvre ardente; elle parfuma sa barbe blanche et alluma la chibouque dont quelquefois, pendant des heures entières, il suivait distraitement des yeux la fumée se volatilisant dans l'air. 

«Tout à coup il fit un mouvement si brusque que je fus saisie de peur. 

«Puis, sans détourner les yeux du point qui fixait son attention, il demanda sa longue-vue. 

«Ma mère la lui passa, plus blanche que le stuc contre lequel elle s'appuyait. 

«Je vis la main de mon père trembler.  

«—Une barque!\dots deux!\dots trois!\dots murmura mon père; quatre!\dots 

«Et il se leva, saisissant ses armes, et versant, je m'en souviens, de la poudre dans le bassinet de ses pistolets. 

«—Vasiliki, dit-il à ma mère avec un tressaillement visible, voici l'instant qui va décider de nous, dans une demi-heure nous saurons la réponse du sublime empereur, retire-toi dans le souterrain avec Haydée. 

«—Je ne veux pas vous quitter, dit Vasiliki; si vous mourez, mon maître, je veux mourir avec vous. 

«—Allez près de Sélim! cria mon père. 

«—Adieu, seigneur! murmura ma mère, obéissante et pliée en deux comme par l'approche de la mort. 

«—Emmenez Vasiliki, dit mon père à ses Palicares. 

«Mais moi, qu'on oubliait, je courus à lui et j'étendis mes mains de son côté; il me vit, et, se penchant vers moi, il pressa mon front de ses lèvres. 

«Oh! ce baiser, ce fut le dernier, et il est là encore sur mon front. 

«En descendant, nous distinguions à travers les treilles de la terrasse les barques qui grandissaient sur le lac, et qui, pareilles naguère à des points noirs, semblaient déjà des oiseaux rasant la surface des ondes.  

«Pendant ce temps, dans le kiosque, vingt Palicares, assis aux pieds de mon père et cachés par la boiserie, épiaient d'un œil sanglant l'arrivée de ces bateaux, et tenaient prêts leurs longs fusils incrustés de nacre et d'argent: des cartouches en grand nombre étaient semées sur le parquet; mon père regardait sa montre et se promenait avec angoisse. 

«Voilà ce qui me frappa quand je quittai mon père après le dernier baiser que j'eus reçu de lui. 

«Nous traversâmes, ma mère et moi, le souterrain. Sélim était toujours à son poste; il nous sourit tristement. Nous allâmes chercher des coussins de l'autre côté de la caverne, et nous vînmes nous asseoir près de Sélim: dans les grands périls, les cœurs dévoués se cherchent, et, tout enfant que j'étais, je sentais instinctivement qu'un grand malheur planait sur nos têtes.» 

Albert avait souvent entendu raconter, non point par son père, qui n'en parlait jamais, mais par des étrangers, les derniers moments du vizir de Janina; il avait lu différents récits de sa mort; mais cette histoire, devenue vivante dans la personne et par la voix de la jeune fille, cet accent vivant et cette lamentable élégie, le pénétraient tout à la fois d'un charme et d'une horreur inexprimables. 

Quant à Haydée, toute à ces terribles souvenirs, elle avait cessé un instant de parler; son front, comme une fleur qui se penche un jour d'orage, s'était incliné sur sa main, et ses yeux, perdus vaguement, semblaient voir encore à l'horizon le Pinde verdoyant et les eaux bleues du lac de Janina, miroir magique qui reflétait le sombre tableau qu'elle esquissait. 

Monte-Cristo la regardait avec une indéfinissable expression d'intérêt et de pitié. 

«Continue, ma fille», dit le comte en langue romaïque. 

Haydée releva le front, comme si les mots sonores que venait de prononcer Monte-Cristo l'eussent tirée d'un rêve, et elle reprit: 

«Il était quatre heures du soir; mais bien que le jour fût pur et brillant au-dehors, nous étions, nous, plongés dans l'ombre du souterrain. 

«Une seule lueur brillait dans la caverne, pareille à une étoile tremblant au fond d'un ciel noir: c'était la mèche de Sélim. Ma mère était chrétienne, et elle priait. 

«Sélim répétait de temps en temps ces paroles consacrées: 

«—Dieu est grand! 

«Cependant ma mère avait encore quelque espérance. En descendant, elle avait cru reconnaître le Franc qui avait été envoyé à Constantinople, et dans lequel mon père avait toute confiance car il savait que les soldats du sultan français sont d'ordinaire nobles et généreux. Elle s'avança de quelques pas vers l'escalier et écouta. 

«—Ils approchent, dit-elle; pourvu qu'ils apportent la paix et la vie. 

«—Que crains-tu, Vasiliki?» répondit Sélim avec sa voix si suave et si fière à la fois; «s'ils n'apportent pas la paix, nous leur donnerons la mort.» 

«Et il ravivait la flamme de sa lance avec un geste qui le faisait ressembler au Dionysos de l'antique Crète. 

«Mais moi, qui étais si enfant et si naïve, j'avais peur de ce courage que je trouvais féroce et insensé, et je m'effrayais de cette mort épouvantable dans l'air et dans la flamme. 

«Ma mère éprouvait les mêmes impressions, car je la sentais frissonner. 

«—Mon Dieu! mon Dieu, maman! m'écriai-je, est-ce que nous allons mourir? 

«Et à ma voix les pleurs et les prières des esclaves redoublèrent. 

«—Enfant, me dit Vasiliki, Dieu te préserve d'en venir à désirer cette mort que tu crains aujourd'hui! 

«Puis tout bas: 

«—Sélim, dit-elle, quel est l'ordre du maître? 

«—S'il m'envoie son poignard, c'est que le sultan refuse de le recevoir en grâce, et je mets le feu; s'il m'envoie son anneau, c'est que le sultan lui pardonne, et je livre la poudrière. 

«—Ami, reprit ma mère, lorsque l'ordre du maître arrivera, si c'est le poignard qu'il envoie, au lieu de nous tuer toutes deux de cette mort qui nous épouvante, nous te tendrons la gorge et tu nous tueras avec ce poignard. 

«—Oui, Vasiliki, répondit tranquillement Sélim. 

«Soudain nous entendîmes comme de grands cris; nous écoutâmes: c'étaient des cris de joie; le nom du Franc qui avait été envoyé à Constantinople retentissait répété par nos Palicares; il était évident qu'il rapportait la réponse du sublime empereur, et que la réponse était favorable. 

—Et vous ne vous rappelez pas ce nom?» dit Morcerf, tout prêt à aider la mémoire de la narratrice. 

Monte-Cristo lui fit un signe. 

«Je ne me le rappelle pas, répondit Haydée. 

«Le bruit redoublait; des pas plus rapprochés retentirent; on descendait les marches du souterrain. 

«Sélim apprêta sa lance. 

«Bientôt une ombre apparut dans le crépuscule bleuâtre que formaient les rayons du jour pénétrant jusqu'à l'entrée du souterrain. 

«—Qui es-tu? cria Sélim. Mais, qui que tu sois, ne fais pas un pas de plus. 

«—Gloire au sultan! dit l'ombre. Toute grâce est accordée au vizir Ali; et non seulement il a la vie sauve, mais on lui rend sa fortune et ses biens. 

«Ma mère poussa un cri de joie et me serra contre son cœur. 

«—Arrête! lui dit Sélim, voyant qu'elle s'élançait déjà pour sortir; tu sais qu'il me faut l'anneau. 

«—C'est juste, dit ma mère, et elle tomba à genoux en me soulevant vers le ciel, comme si, en même temps qu'elle priait Dieu pour moi, elle voulait encore me soulever vers lui.» 

Et, pour la seconde fois, Haydée s'arrêta vaincue par une émotion telle que la sueur coulait sur son front pâli, et que sa voix étranglée semblait ne pouvoir franchir son gosier aride. 

Monte-Cristo versa un peu d'eau glacée dans un verre, et le lui présenta en disant avec une douceur où perçait une nuance de commandement: 

«Du courage, ma fille!» 

Haydée essuya ses yeux et son front, et continua: 

«Pendant ce temps, nos yeux, habitués à l'obscurité avaient reconnu l'envoyé du pacha: c'était un ami. 

«Sélim l'avait reconnu; mais le brave jeune homme ne savait qu'une chose: obéir! 

«—En quel nom viens-tu? dit-il. 

«—Je viens au nom de notre maître, Ali-Tebelin. 

«—Si tu viens au nom d'Ali, tu sais ce que tu dois me remettre? 

«—Oui, dit l'envoyé, et je t'apporte son anneau. 

«En même temps il éleva sa main au-dessus de sa tête; mais il était trop loin et il ne faisait pas assez clair pour que Sélim pût, d'où nous étions, distinguer et reconnaître l'objet qu'il lui présentait. 

«—Je ne vois pas ce que tu tiens, dit Sélim. 

«—Approche, dit le messager, ou je m'approcherai, moi. 

«—Ni l'un ni l'autre, répondit le jeune soldat; dépose à la place où tu es, et sous ce rayon de lumière, l'objet que tu me montres, et retire-toi jusqu'à ce que je l'aie vu. 

«—Soit, dit le messager. 

«Et il se retira après avoir déposé le signe de reconnaissance à l'endroit indiqué. 

«Et notre cœur palpitait: car l'objet nous paraissait être effectivement un anneau. Seulement, était-ce l'anneau de mon père? 

«Sélim, tenant toujours à la main sa mèche enflammée, vint à l'ouverture, s'inclina radieux sous le rayon de lumière et ramassa le signe. 

«—L'anneau du maître, dit-il en le baisant, c'est bien!  

«Et renversant la mèche contre terre, il marcha dessus et l'éteignit. 

«Le messager poussa un cri de joie et frappa dans ses mains. À ce signal, quatre soldats du séraskier Kourchid accoururent, et Sélim tomba percé de cinq coups de poignard. Chacun avait donné le sien. 

«Et cependant, ivres de leur crime, quoique encore pâles de peur, ils se ruèrent dans le souterrain, cherchant partout s'il y avait du feu, et se roulant sur les sacs d'or. 

«Pendant ce temps ma mère me saisit entre ses bras, et, agile, bondissant par des sinuosités connues de nous seules, elle arriva jusqu'à un escalier dérobé du kiosque dans lequel régnait un tumulte effrayant. 

«Les salles basses étaient entièrement peuplées par les Tchodoars de Kourchid, c'est-à-dire par nos ennemis. 

«Au moment où ma mère allait pousser la petite porte, nous entendîmes retentir, terrible et menaçante, la voix du pacha. 

«Ma mère colla son œil aux fentes des planches; une ouverture se trouva par hasard devant le mien, et je regardai. 

«—Que voulez-vous? disait mon père à des gens qui tenaient un papier avec des caractères d'or à la main. 

«—Ce que nous voulons, répondit l'un d'eux, c'est te communiquer la volonté de Sa Hautesse. Vois-tu ce firman?  

«—Je le vois, dit mon père. 

«—Eh bien, lis; il demande ta tête. 

«Mon père poussa un éclat de rire plus effrayant que n'eût été une menace; il n'avait pas encore cessé, que deux coups de pistolet étaient partis de ses mains et avaient tué deux hommes. 

«Les Palicares, qui étaient couchés tout autour de mon père la face contre le parquet, se levèrent alors et firent feu; la chambre se remplit de bruit, de flamme et de fumée. 

«À l'instant même le feu commença de l'autre côté, et les balles vinrent trouer les planches tout autour de nous. 

«Oh! qu'il était beau, qu'il était grand, le vizir Ali-Tebelin, mon père, au milieu des balles, le cimeterre au poing, le visage noir de poudre! Comme ses ennemis fuyaient! 

«—Sélim! Sélim! criait-il, gardien du feu, fais ton devoir! 

«—Sélim est mort! répondit une voix qui semblait sortir des profondeurs du kiosque, et toi, mon seigneur Ali, tu es perdu! 

«En même temps une détonation sourde se fit entendre, et le plancher vola en éclats tout autour de mon père. 

«Les Tchodoars tiraient à travers le parquet. Trois ou quatre Palicares tombèrent frappés de bas en haut par des blessures qui leur labouraient tout le corps. 

«Mon père rugit, enfonça ses doigts par les trous des balles et arracha une planche tout entière. 

«Mais en même temps, par cette ouverture, vingt coups de feu éclatèrent, et la flamme, sortant comme du cratère d'un volcan, gagna les tentures qu'elle dévora. 

«Au milieu de tout cet affreux tumulte, au milieu de ces cris terribles, deux coups plus distincts entre tous, deux cris plus déchirants par-dessus tous les cris, me glacèrent de terreur. Ces deux explosions avaient frappé mortellement mon père, et c'était lui qui avait poussé ces deux cris. 

«Cependant il était resté debout, cramponné à une fenêtre. Ma mère secouait la porte pour aller mourir avec lui; mais la porte était fermée en dedans. 

«Tout autour de lui, les Palicares se tordaient dans les convulsions de l'agonie; deux ou trois, qui étaient sans blessures ou blessés légèrement, s'élancèrent par les fenêtres. En même temps, le plancher tout entier craqua brisé en dessous. Mon père tomba sur un genou; en même temps vingt bras s'allongèrent, armés de sabres, de pistolets, de poignards, vingt coups frappèrent à la fois un seul homme, et mon père disparut dans un tourbillon de feu, attisé par ces démons rugissants comme si l'enfer se fût ouvert sous ses pieds. 

«Je me sentis rouler à terre: c'était ma mère qui s'abîmait évanouie.» 

Haydée laissa tomber ses deux bras en poussant un gémissement et en regardant le comte comme pour lui demander s'il était satisfait de son obéissance. 

Le comte se leva, vint à elle, lui prit la main et lui dit en remarque: 

«Repose-toi, chère enfant, et reprends courage en songeant qu'il y a un Dieu qui punit les traîtres. 

—Voilà une épouvantable histoire, comte, dit Albert tout effrayé de la pâleur d'Haydée, et je me reproche maintenant d'avoir été si cruellement indiscret. 

—Ce n'est rien», répondit Monte-Cristo. 

Puis posant sa main sur la tête de la jeune fille: 

«Haydée, continua-t-il, est une femme courageuse, elle a quelquefois trouvé du soulagement dans le récit de ses douleurs. 

—Parce que, mon seigneur, dit vivement la jeune fille, parce que mes douleurs me rappellent tes bienfaits.» 

Albert la regarda avec curiosité, car elle n'avait point encore raconté ce qu'il désirait le plus savoir, c'est-à-dire comment elle était devenue l'esclave du comte. 

Haydée vit à la fois dans les regards du comte et dans ceux d'Albert le même désir exprimé. 

Elle continua: 

«Quand ma mère reprit ses sens, dit-elle, nous étions devant le séraskier. 

«—Tuez-moi, dit-elle, mais épargnez l'honneur de la veuve d'Ali. 

«—Ce n'est point à moi qu'il faut t'adresser, dit Kourchid. 

«—À qui donc? 

«—C'est à ton nouveau maître. 

«—Quel est-il? 

«—Le voici. 

«Et Kourchid nous montra un de ceux qui avaient le plus contribué à la mort de mon père, continua la jeune fille avec une colère sombre. 

—Alors, demanda Albert, vous devîntes la propriété de cet homme? 

—Non, répondit Haydée; il n'osa nous garder, il nous vendit à des marchands d'esclaves qui allaient à Constantinople. Nous traversâmes la Grèce, et nous arrivâmes mourantes à la porte impériale, encombrée de curieux qui s'écartaient pour nous laisser passer, quand tout à coup ma mère suit des yeux la direction de leurs regards, jette un cri et tombe en me montrant une tête au-dessus de cette porte. 

«Au-dessous de cette tête étaient écrits ces mots: 

«Celle-ci est la tête d'Ali-Tebelin, pacha de Janina.» 

«J'essayai, en pleurant, de relever ma mère: elle était morte! 

«Je fus menée au bazar; un riche Arménien m'acheta, me fit instruire, me donna des maîtres et quand j'eus treize ans me vendit au sultan Mahmoud. 

—Auquel, dit Monte-Cristo, je la rachetai, comme je vous l'ai dit, Albert, pour cette émeraude pareille à celle où je mets mes pastilles de haschich. 

—Oh! tu es bon, tu es grand, mon seigneur, dit Haydée en baisant la main de Monte-Cristo, et je suis bien heureuse de t'appartenir!» 

Albert était resté tout étourdi de ce qu'il venait d'entendre. 

«Achevez donc votre tasse de café, lui dit le comte; l'histoire est finie.» 