\chapter{Le partage}

\lettrine{D}{ans} cet hôtel de la rue Saint-Germain-des-Prés qu'avait choisi pour sa mère et pour lui Albert de Morcerf, le premier étage, composé d'un petit appartement complet, était loué à un personnage fort mystérieux. 

Ce personnage était un homme dont jamais le concierge lui-même n'avait pu voir la figure, soit qu'il entrât ou qu'il sortît; car l'hiver il s'enfonçait le menton dans une de ces cravates rouges comme en ont les cochers de bonne maison qui attendent leurs maîtres à la sortie des spectacles, et l'été il se mouchait toujours précisément au moment où il eût pu être aperçu en passant devant la loge. Il faut dire que, contrairement à tous les usages reçus, cet habitant de l'hôtel n'était épié par personne, et que le bruit qui courait que son incognito cachait un individu très haut placé, et \textit{ayant le bras long}, avait fait respecter ses mystérieuses apparitions. 

Ses visites étaient ordinairement fixes, quoique parfois elles fussent avancées ou retardées; mais presque toujours, hiver ou été, c'était vers quatre heures qu'il prenait possession de son appartement, dans lequel il ne passait jamais la nuit. 

À trois heures et demie, l'hiver, le feu était allumé par la servante discrète qui avait l'intendance du petit appartement; à trois heures et demie, l'été, des glaces étaient montées par la même servante. 

À quatre heures, comme nous l'avons dit, le personnage mystérieux arrivait. 

Vingt minutes après lui, une voiture s'arrêtait devant l'hôtel; une femme vêtue de noir ou de bleu foncé, mais toujours enveloppée d'un grand voile, en descendait, passait comme une ombre devant la loge, montait l'escalier sans que l'on entendît craquer une seule marche sous son pied léger. 

Jamais il ne lui était arrivé qu'on lui demandât où elle allait. 

Son visage, comme celui de l'inconnu, était donc parfaitement étranger aux deux gardiens de la porte, ces concierges modèles, les seuls peut-être, dans l'immense confrérie des portiers de la capitale capables d'une pareille discrétion. 

Il va sans dire qu'elle ne montait pas plus haut que le premier. Elle grattait à une porte d'une façon particulière; la porte s'ouvrait, puis se refermait hermétiquement, et tout était dit. 

Pour quitter l'hôtel, même manœuvre que pour y entrer. 

L'inconnue sortait la première, toujours voilée, et remontait dans sa voiture, qui tantôt disparaissait par un bout de la rue, tantôt par l'autre; puis, vingt minutes après, l'inconnu sortait à son tour, enfoncé dans sa cravate ou caché par son mouchoir, et disparaissait également. 

Le lendemain du jour où le comte de Monte-Cristo avait été rendre visite à Danglars, jour de l'enterrement de Valentine, l'habitant mystérieux entra vers dix heures du matin, au lieu d'entrer comme d'habitude, vers quatre heures de l'après-midi. 

Presque aussitôt, et sans garder l'intervalle ordinaire, une voiture de place arriva, et la dame voilée monta rapidement l'escalier. 

La porte s'ouvrit et se referma. 

Mais, avant même que la porte fût refermée, la dame s'était écriée: 

«Ô Lucien! ô mon ami!» 

De sorte que le concierge, qui, sans le vouloir, avait entendu cette exclamation, sut alors pour la première fois que son locataire s'appelait Lucien; mais comme c'était un portier modèle, il se promit de ne pas même le dire à sa femme. 

«Eh bien, qu'y a-t-il, chère amie? demanda celui dont le trouble ou l'empressement de la dame voilée avait révélé le nom; parlez, dites. 

—Mon ami, puis-je compter sur vous? 

—Certainement, et vous le savez bien. 

«Mais qu'y a-t-il? 

«Votre billet de ce matin m'a jeté dans une perplexité terrible. 

«Cette précipitation, ce désordre dans votre écriture; voyons, rassurez-moi ou effrayez-moi tout à fait! 

—Lucien, un grand événement! dit la dame en attachant sur Lucien un regard interrogateur: M. Danglars est parti cette nuit. 

—Parti! M. Danglars parti! 

«Et où est-il allé? 

—Je l'ignore. 

—Comment! vous l'ignorez? Il est donc parti pour ne plus revenir? 

—Sans doute! 

«À dix heures du soir, ses chevaux l'ont conduit à la barrière de Charenton; là, il a trouvé une berline de poste tout attelée; il est monté dedans avec son valet de chambre, en disant à son cocher qu'il allait à Fontainebleau. 

—Eh bien, que disiez-vous donc? 

—Attendez, mon ami. Il m'avait laissé une lettre. 

—Une lettre? 

—Oui; lisez.» 

Et la baronne tira de sa poche une lettre décachetée qu'elle présenta à Debray. 

Debray, avant de la lire, hésita un instant, comme s'il eût cherché à deviner ce qu'elle contenait, ou plutôt comme si, quelque chose qu'elle contînt, il était décidé à prendre d'avance un parti. 

Au bout de quelques secondes ses idées étaient sans doute arrêtées, car il lut. 

Voici ce que contenait ce billet qui avait jeté un si grand trouble dans le cœur de Mme Danglars: 

\begin{mail}{}{Madame et très fidèle épouse.}
	\pausemail
	
Sans y songer, Debray s'arrêta et regarda la baronne, qui rougit jusqu'aux yeux. 

«Lisez», dit-elle. 

Debray continua: 

\resumemail

Quand vous recevrez cette lettre vous n'aurez plus de mari! Oh! ne prenez pas trop chaudement l'alarme, vous n'aurez plus de mari comme vous n'aurez plus de fille, c'est-à-dire que je serai sur une des trente ou quarante routes qui conduisent hors de France. 

Je vous dois des explications, et comme vous êtes femme à les comprendre parfaitement, je vous les donnerai. 

Écoutez donc: 

Un remboursement de cinq millions m'est survenu ce matin, je l'ai opéré; un autre de même somme l'a suivi presque immédiatement; je l'ajourne à demain: aujourd'hui je pars pour éviter ce demain qui me serait trop désagréable à supporter. 

Vous comprenez cela, n'est-ce pas, madame et très précieuse épouse? 

Je dis: 

Vous comprenez, parce que vous savez aussi bien que moi mes affaires; vous les savez même mieux que moi, attendu que s'il s'agissait de dire où a passé une bonne moitié de ma fortune, naguère encore assez belle, j'en serais incapable; tandis que vous, au contraire, j'en suis certain, vous vous en acquitteriez parfaitement. 

Car les femmes ont des instincts d'une sûreté infaillible, elles expliquent par une algèbre qu'elles ont inventée le merveilleux lui-même. Moi qui ne connaissais que mes chiffres, je n'ai plus rien su du jour où mes chiffres m'ont trompé. 

Avez-vous quelquefois admiré la rapidité de ma chute, madame? 

Avez-vous été un peu éblouie de cette incandescente fusion de mes lingots? 

Moi, je l'avoue, je n'y ai vu que du feu; espérons que vous avez retrouvé un peu d'or dans les cendres. 

C'est avec ce consolant espoir que je m'éloigne, madame et très prudente épouse, sans que ma conscience me reproche le moins du monde de vous abandonner; il vous reste des amis, les cendres en question, et, pour comble de bonheur, la liberté que je m'empresse de vous rendre. 

Cependant, madame, le moment est arrivé de placer dans ce paragraphe un mot d'explication intime. Tant que j'ai espéré que vous travailliez au bien-être de notre maison, à la fortune de notre fille, j'ai philosophiquement fermé les yeux; mais comme vous avez fait de la maison une vaste ruine, je ne veux pas servir de fondation à la fortune d'autrui. 

Je vous ai prise riche, mais peu honorée. 

Pardonnez-moi de vous parler avec cette franchise; mais comme je ne parle que pour nous deux probablement, je ne vois pas pourquoi je farderais mes paroles. 

J'ai augmenté notre fortune, qui pendant plus de quinze ans a été croissant, jusqu'au moment où des catastrophes inconnues et inintelligibles encore pour moi sont venues la prendre corps à corps et la renverser, sans que, je puis le dire, il y ait aucunement de ma faute. 

Vous, madame, vous avez travaillé seulement à accroître la vôtre, chose à laquelle vous avez réussi, j'en suis moralement convaincu. 

Je vous laisse donc comme je vous ai prise, riche, mais peu honorable. 

Adieu. 

Moi aussi, je vais, à partir d'aujourd'hui, travailler pour mon compte. 

Croyez à toute ma reconnaissance pour l'exemple que vous m'avez donné et que je vais suivre. 

\closeletter[Votre mari bien dévoué,]{Baron Danglars.}
\end{mail}

La baronne avait suivi des yeux Debray pendant cette longue et pénible lecture; elle avait vu, malgré sa puissance bien connue sur lui-même, le jeune homme changer de couleur une ou deux fois. 

Lorsqu'il eut fini, il ferma lentement le papier dans ses plis, et reprit son attitude pensive. 

«Eh bien? demanda Mme Danglars avec une anxiété facile à comprendre. 

—Eh bien, madame? répéta machinalement Debray. 

—Quelle idée vous inspire cette lettre? 

—C'est bien simple, madame; elle m'inspire l'idée que M. Danglars est parti avec des soupçons. 

—Sans doute; mais est-ce tout ce que vous avez à me dire? 

—Je ne comprends pas, dit Debray avec un froid glacial. 

—Il est parti! parti tout à fait! parti pour ne plus revenir. 

—Oh! fit Debray, ne croyez pas cela, baronne. 

—Non, vous dis-je, il ne reviendra pas; je le connais, c'est un homme inébranlable dans toutes les résolutions qui émanent de son intérêt. 

«S'il m'eût jugée utile à quelque chose, il m'eût emmenée. Il me laisse à Paris, c'est que notre séparation peut servir ses projets: elle est donc irrévocable et je suis libre à jamais», ajouta Mme Danglars avec la même expression de prière. 

Mais Debray, au lieu de répondre, la laissa dans cette anxieuse interrogation du regard et de la pensée. 

«Quoi! dit-elle enfin, vous ne me répondez pas, monsieur? 

—Mais je n'ai qu'une question à vous faire: que comptez-vous devenir? 

—J'allais vous le demander, répondit la baronne le cœur palpitant. 

—Ah! fit Debray, c'est donc un conseil que vous me demandez? 

—Oui, c'est un conseil que je vous demande, dit la baronne le cœur serré. 

—Alors, si c'est un conseil que vous me demandez, répondit froidement le jeune homme, je vous conseille de voyager. 

—De voyager! murmura madame Danglars. 

—Certainement. Comme l'a dit M. Danglars, vous êtes riche et parfaitement libre. Une absence de Paris sera nécessaire absolument, à ce que je crois du moins, après le double éclat du mariage rompu de Mlle Eugénie et de la disparition de M. Danglars. 

«Il importe seulement que tout le monde vous sache abandonnée et vous croie pauvre; car on ne pardonnerait pas à la femme du banqueroutier son opulence et son grand état de maison. 

«Pour le premier cas, il suffit que vous restiez seulement quinze jours à Paris, répétant à tout le monde que vous êtes abandonnée et racontant à vos meilleures amies, qui iront le répéter dans le monde, comment cet abandon a eu lieu. Puis vous quitterez votre hôtel, vous y laisserez vos bijoux, vous abandonnez votre douaire, et chacun vantera votre désintéressement et chantera vos louanges. 

«Alors on vous saura abandonnée, et l'on vous croira pauvre; car moi seul connais votre situation financière et suis prêt à vous rendre mes comptes en loyal associé.» 

La baronne, pâle, atterrée, avait écouté ce discours avec autant d'épouvante et de désespoir que Debray avait mis de calme et d'indifférence à le prononcer. 

«Abandonnée! répéta-t-elle, oh! bien abandonnée\dots Oui, vous avez raison, monsieur, et personne ne doutera de mon abandon.» 

Ce furent les seules paroles que cette femme, si fière et si violemment éprise, put répondre à Debray. 

«Mais riche, très riche même», poursuivit Debray en tirant de son portefeuille et en étalant sur la table quelques papiers qu'il renfermait. 

Mme Danglars le laissa faire, tout occupée d'étouffer les battements de son cœur et de retenir les larmes qu'elle sentait poindre au bord de ses paupières. Mais enfin le sentiment de la dignité l'emporta chez la baronne; et si elle ne réussit point à comprimer son cœur, elle parvint du moins à ne pas verser une larme. 

«Madame, dit Debray, il y a six mois à peu près que nous sommes associés. 

«Vous avez fourni une mise de fonds de cent mille francs. 

«C'est au mois d'avril de cette année qu'a eu lieu notre association. 

«En mai, nos opérations ont commencé. 

«En mai, nous avons gagné quatre cent cinquante mille francs. 

«En juin, le bénéfice a monté à neuf cent mille. 

«En juillet, nous y avons ajouté dix-sept cent mille francs; c'est, vous le savez, le mois des bons d'Espagne. 

«En août, nous perdîmes, au commencement du mois, trois cent mille francs; mais le 15 du mois nous nous étions rattrapés, et à la fin nous avions pris notre revanche; car nos comptes, mis au net depuis le jour de notre association jusqu'à hier où je les ai arrêtés, nous donnent un actif de deux millions quatre cent mille francs, c'est-à-dire de douze cent mille francs pour chacun de nous. 

«Maintenant, continua Debray, compulsant son carnet avec la méthode et la tranquillité d'un agent de change, nous trouvons quatre-vingt mille francs pour les intérêts composés de cette somme restée entre mes mains. 

—Mais, interrompit la baronne, que veulent dire ces intérêts, puisque jamais vous n'avez fait valoir cet argent? 

—Je vous demande pardon, madame, dit froidement Debray; j'avais vos pouvoirs pour le faire valoir, et j'ai usé de vos pouvoirs. 

«C'est donc quarante mille francs d'intérêts pour votre moitié, plus les cent mille francs de mise de fonds première, c'est-à-dire treize cent quarante mille francs pour votre part. 

«Or, madame, continua Debray, j'ai eu la précaution de mobiliser votre argent avant-hier, il n'y a pas longtemps, comme vous voyez, et l'on eût dit que je me doutais d'être incessamment appelé à vous rendre mes comptes. Votre argent est là, moitié en billets de banque, moitié en bons au porteur. 

«Je dis là, et c'est vrai: car comme je ne jugeais pas ma maison assez sûre, comme je ne trouvais pas les notaires assez discrets, et que les propriétés parlent encore plus haut que les notaires; comme enfin vous n'avez le droit de rien acheter ni de rien posséder en dehors de la communauté conjugale, j'ai gardé toute cette somme, aujourd'hui votre seule fortune, dans un coffre scellé au fond de cette armoire, et pour plus grande sécurité, j'ai fait le maçon moi-même. 

«Maintenant, continua Debray en ouvrant l'armoire d'abord, et la caisse ensuite, maintenant, madame voilà huit cents billets de mille francs chacun, qui ressemblent, comme vous voyez, à un gros album relié en fer; j'y joins un coupon de rente de vingt-cinq mille francs; puis pour l'appoint, qui fait quelque chose, je crois, comme cent dix mille francs, voici un bon à vue sur mon banquier, et comme mon banquier n'est pas M. Danglars, le bon sera payé, vous pouvez être tranquille.» 

Mme Danglars prit machinalement le bon à vue, le coupon de rente et la liasse de billets de banque. 

Cette énorme fortune paraissait bien peu de chose étalée là sur une table. 

Mme Danglars, les yeux secs, mais la poitrine gonflée de sanglots, la ramassa et enferma l'étui d'acier dans son sac, mit le coupon de rente et le bon à vue dans son portefeuille, et debout, pâle, muette, elle attendit une douce parole qui la consolât d'être si riche. 

Mais elle attendit vainement. 

«Maintenant, madame, dit Debray, vous avez une existence magnifique, quelque chose comme soixante mille livres de rente, ce qui est énorme pour une femme qui ne pourra pas tenir maison, d'ici à un an au moins. 

«C'est un privilège pour toutes les fantaisies qui vous passeront par l'esprit: sans compter que si vous trouvez votre part insuffisante, eu égard au passé qui vous échappe, vous pouvez puiser dans la mienne, madame; et je suis disposé à vous offrir, oh! à titre de prêt, bien entendu, tout ce que je possède, c'est-à-dire un million soixante mille francs. 

—Merci, monsieur, répondit la baronne, merci; vous comprenez que vous me remettez là beaucoup plus qu'il ne faut à une pauvre femme qui ne compte pas, d'ici à longtemps du moins, reparaître dans le monde.» 

Debray fut étonné un moment, mais il se remit et fit un geste qui pouvait se traduire par la formule la plus polie d'exprimer cette idée: 

«Comme il vous plaira!» 

Mme Danglars avait peut-être jusque-là espéré encore quelque chose; mais quand elle vit le geste insouciant qui venait d'échapper à Debray, et le regard oblique dont ce geste était accompagné, ainsi que la révérence profonde et le silence significatif qui les suivirent, elle releva la tête, ouvrit la porte, et sans fureur, sans secousse, mais aussi sans hésitation, elle s'élança dans l'escalier, dédaignant même d'adresser un dernier salut à celui qui la laissait partir de cette façon. 

«Bah! dit Debray lorsqu'elle fut partie: beaux projets que tout cela, elle restera dans son hôtel, lira des romans, et jouera au lansquenet, ne pouvant plus jouer à la bourse.» 

Et il reprit son carnet, biffant avec le plus grand soin les sommes qu'il venait de payer. 

«Il me reste un million soixante mille francs, dit-il. 

«Quel malheur que Mlle de Villefort soit morte! cette femme-là me convenait sous tous les rapports, et je l'eusse épousée.» 

Et flegmatiquement, selon son habitude, il attendit que Mme Danglars fût partie depuis vingt minutes pour se décider à partir à son tour. 

Pendant ces vingt minutes, Debray fit des chiffres, sa montre posée à côté de lui. 

Ce personnage diabolique que toute imagination aventureuse eût créé avec plus ou moins de bonheur si Le Sage n'en avait acquis la propriété dans son chef-d'œuvre, Asmodée, qui enlevait la croûte des maisons pour en voir l'intérieur, eût joui d'un singulier spectacle s'il eût enlevé, au moment où Debray faisait ses chiffres, la croûte du petit hôtel de la rue Saint-Germain-des-Prés. 

Au-dessus de cette chambre où Debray venait de partager avec Mme Danglars deux millions et demi, il y avait une autre chambre peuplée aussi d'habitants de notre connaissance, lesquels ont joué un rôle assez important dans les événements que nous venons de raconter pour que nous les retrouvions avec quelque intérêt. 

Il y avait dans cette chambre Mercédès et Albert. 

Mercédès était bien changée depuis quelques jours, non pas que, même au temps de sa plus grande fortune, elle eût jamais étalé le faste orgueilleux qui tranche visiblement avec toutes les conditions, et fait qu'on ne reconnaît plus la femme aussitôt qu'elle vous apparaît sous des habits plus simples; non pas davantage qu'elle fût tombée à cet état de dépression où l'on est contraint de revêtir la livrée de la misère; non, Mercédès était changée parce que son œil ne brillait plus, parce que sa bouche ne souriait plus, parce qu'enfin un perpétuel embarras arrêtait sur ses lèvres le mot rapide que lançait autrefois un esprit toujours préparé. 

Ce n'était pas la pauvreté qui avait flétri l'esprit de Mercédès, ce n'était pas le manque de courage qui lui rendait pesante sa pauvreté. 

Mercédès, descendue du milieu dans lequel elle vivait, perdue dans la nouvelle sphère qu'elle s'était choisie, comme ces personnes qui sortent d'un salon splendidement éclairé pour passer subitement dans les ténèbres; Mercédès semblait une reine descendue de son palais dans une chaumière, et qui, réduite au strict nécessaire, ne se reconnaît ni à la vaisselle d'argile qu'elle est obligée d'apporter elle-même sur sa table, ni au grabat qui a succédé à son lit. 

En effet, la belle Catalane ou la noble comtesse n'avait plus ni son regard fier, ni son charmant sourire, parce qu'en arrêtant ses yeux sur ce qui l'entourait elle ne voyait que d'affligeants objets: c'était une chambre tapissée d'un de ces papiers gris sur gris que les propriétaires économes choisissent de préférence comme étant les moins salissants; c'était un carreau sans tapis; c'étaient des meubles qui appelaient l'attention et forçaient la vue de s'arrêter sur la pauvreté d'un faux luxe, toutes choses enfin qui rompaient par leurs tons criards l'harmonie si nécessaire à des yeux habitués à un ensemble élégant. 

Mme de Morcerf vivait là depuis qu'elle avait quitté son hôtel; la tête lui tournait devant ce silence éternel comme elle tourne au voyageur arrivé sur le bord d'un abîme: s'apercevant qu'à toute minute Albert la regardait à la dérobée pour juger de l'état de son cœur, elle s'était astreinte à un monotone sourire des lèvres qui, en l'absence de ce feu si doux du sourire des yeux, fait l'effet d'une simple réverbération de lumière, c'est-à-dire d'une clarté sans chaleur. 

De son côté Albert était préoccupé, mal à l'aise, gêné par un reste de luxe qui l'empêchait d'être de sa condition actuelle; il voulait sortir sans gants, et trouvait ses mains trop blanches; il voulait courir la ville à pied, et trouvait ses bottes trop bien vernies. 

Cependant ces deux créatures si nobles et si intelligentes, réunies indissolublement par le lien de l'amour maternel et filial, avaient réussi à se comprendre sans parler de rien et à économiser toutes les privations que l'on se doit entre amis pour établir cette vérité matérielle d'où dépend la vie. 

Albert avait enfin pu dire à sa mère sans la faire pâlir: 

«Ma mère, nous n'avons plus d'argent.» 

Jamais Mercédès n'avait connu véritablement la misère; elle avait souvent, dans sa jeunesse, parlé elle-même de pauvreté, mais ce n'est point la même chose: besoin et nécessité sont deux synonymes entre lesquels il y a tout un monde d'intervalle. 

Aux Catalans, Mercédès avait besoin de mille choses, mais elle ne manquait jamais de certaines autres. Tant que les filets étaient bons, on prenait du poisson; tant qu'on vendait du poisson, on avait du fil pour entretenir les filets. 

Et puis, isolée d'amitié, n'ayant qu'un amour qui n'était pour rien dans les détails matériels de la situation, on pensait à soi, chacun à soi, rien qu'à soi. 

Mercédès, du peu qu'elle avait, faisait sa part aussi généreusement que possible: aujourd'hui elle avait deux parts à faire, et cela avec rien. 

L'hiver approchait: Mercédès, dans cette chambre nue et déjà froide, n'avait pas de feu, elle dont un calorifère aux mille branches chauffait autrefois la maison depuis les antichambres jusqu'au boudoir; elle n'avait pas une pauvre petite fleur, elle dont l'appartement était une serre chaude peuplée à prix d'or! 

Mais elle avait son fils\dots 

L'exaltation d'un devoir peut-être exagéré les avait soutenus jusque-là dans les sphères supérieures. 

L'exaltation est presque l'enthousiasme, et l'enthousiasme rend insensible aux choses de la terre. 

Mais l'enthousiasme s'était calmé, et il avait fallu redescendre peu à peu du pays des rêves au monde des réalités. 

Il fallait causer du positif, après avoir épuisé tout l'idéal. 

«Ma mère, disait Albert au moment même où Mme Danglars descendait l'escalier, comptons un peu toutes nos richesses, s'il vous plaît; j'ai besoin d'un total pour échafauder mes plans. 

—Total: rien, dit Mercédès avec un douloureux sourire. 

—Si fait, ma mère, total: trois mille francs, d'abord, et j'ai la prétention, avec ces trois mille francs, de mener à nous deux une adorable vie. 

—Enfant! soupira Mercédès. 

—Hélas! ma bonne mère, dit le jeune homme, je vous ai malheureusement dépensé assez d'argent pour en connaître le prix. 

«C'est énorme, voyez-vous, trois mille francs, et j'ai bâti sur cette somme un avenir miraculeux d'éternelle sécurité. 

—Vous dites cela, mon ami, continua la pauvre mère; mais d'abord acceptons-nous ces trois mille francs? dit Mercédès en rougissant. 

—Mais c'est convenu, ce me semble, dit Albert d'un ton ferme; nous les acceptons d'autant plus que nous ne les avons pas, car ils sont, comme vous le savez, enterrés dans le jardin de cette petite maison des Allées de Meilhan à Marseille. 

«Avec deux cents francs; dit Albert, nous irons tous deux à Marseille. 

—Avec deux cents francs! dit Mercédès, y songez-vous, Albert? 

—Oh! quant à ce point, je me suis renseigné aux diligences et aux bateaux à vapeur, et mes calculs sont faits. 

«Vous retenez vos places pour Chalon, dans le coupé: vous voyez, ma mère, que je vous traite en reine, trente-cinq francs.» 

Albert prit une plume, et écrivit: 

 \begin{tabular} {l@{\dotfill}r} 
 
%\multicolumn{1}{l}{~} & \multicolumn{1}{r}{Frs.} \\
Coupé, trente-cinq francs, ci:&35.\\
De Chalon à Lyon, vous allez par le bateau à vapeur, six francs, ci:&6.\\
De Lyon à Avignon, le bateau à vapeur encore, seize francs, ci:&16.\\
D'Avignon à Marseille, sept francs, ci:&7.\\
Dépenses de route, cinquante francs, ci:&50.\\
Total&114 F\\
\end{tabular}
\bigskip



«Mettons cent vingt, ajouta Albert en souriant, vous voyez que je suis généreux, n'est-ce pas, ma mère? 

—Mais toi, mon pauvre enfant? 

—Moi! n'avez-vous pas vu que je me réserve quatre-vingts francs? 

«Un jeune homme, ma mère, n'a pas besoin de toutes ses aises; d'ailleurs je sais ce que c'est que de voyager. 

—Avec ta chaise de poste et ton valet de chambre. 

—De toute façon, ma mère. 

—Eh bien, soit, dit Mercédès; mais ces deux cents francs? 

—Ces deux cents francs, les voici, et puis deux cents autres encore. 

«Tenez, j'ai vendu ma montre cent francs, et les breloques trois cents. 

«Comme c'est heureux! Des breloques qui valaient trois fois la montre. Toujours cette fameuse histoire du superflu! 

«Nous voilà donc riches, puisque, au lieu de cent quatorze francs qu'il vous fallait pour faire votre route, vous en avez deux cent cinquante. 

—Mais nous devons quelque chose dans cet hôtel? 

—Trente francs, mais je les paie sur mes cent cinquante francs. 

«Cela est convenu; et puisqu'il ne me faut à la rigueur que quatre-vingts francs pour faire ma route, vous voyez que je nage dans le luxe. 

«Mais ce n'est pas tout. 

«Que dites-vous de ceci, ma mère?» 

Et Albert tira d'un petit carnet à fermoir d'or, reste de ses anciennes fantaisies ou peut-être même tendre souvenir de quelqu'une de ces femmes mystérieuses et voilées qui frappaient à la petite porte, Albert tira d'un petit carnet un billet de mille francs. 

«Qu'est-ce que ceci? demanda Mercédès. 

—Mille francs, ma mère. Oh! il est parfaitement carré. 

—Mais d'où te viennent ces mille francs? 

—Écoutez ceci, ma mère, et ne vous émotionnez pas trop.» 

Et Albert, se levant, alla embrasser sa mère sur les deux joues, puis il s'arrêta à la regarder. 

«Vous n'avez pas idée, ma mère, comme je vous trouve belle! dit le jeune homme avec un profond sentiment d'amour filial, vous êtes en vérité la plus belle comme vous êtes la plus noble des femmes que j'aie jamais vues! 

—Cher enfant, dit Mercédès essayant en vain de retenir une larme qui pointait au coin de sa paupière. 

—En vérité, il ne vous manquait plus que d'être malheureuse pour changer mon amour en adoration. 

—Je ne suis pas malheureuse tant que j'ai mon fils, dit Mercédès; je ne serai point malheureuse tant que je l'aurai. 

—Ah! justement, dit Albert; mais voilà où commence l'épreuve, ma mère: vous savez ce qui est convenu! 

—Sommes-nous donc convenus de quelque chose? demanda Mercédès. 

—Oui, il est convenu que vous habiterez Marseille, et que, moi je partirai pour l'Afrique, où, en place du nom que j'ai quitté, je me ferai le nom que j'ai pris.» 

Mercédès poussa un soupir. 

«Eh bien, ma mère, depuis hier je suis engagé dans les spahis, ajouta le jeune homme en baissant les yeux avec une certaine honte, car il ne savait pas lui-même tout ce que son abaissement avait de sublime; ou plutôt j'ai cru que mon corps était bien à moi et que je pouvais le vendre; depuis hier je remplace quelqu'un. 

«Je me suis vendu, comme on dit, et, ajouta-t-il en essayant de sourire, plus cher que je ne croyais valoir, c'est-à-dire deux mille francs. 

—Ainsi ces mille francs?\dots dit en tressaillant Mercédès. 

—C'est la moitié de la somme, ma mère; l'autre viendra dans un an.» 

Mercédès leva les yeux au ciel avec une expression que rien ne saurait rendre, et les deux larmes arrêtées au coin de sa paupière, débordant sous l'émotion intérieure, coulèrent silencieusement le long de ses joues. 

«Le prix de son sang! murmura-t-elle. 

—Oui, si je suis tué, dit en riant Morcerf, mais je t'assure, bonne mère, que je suis au contraire dans l'intention de défendre cruellement ma peau; je ne me suis jamais senti si bonne envie de vivre que maintenant. 

—Mon Dieu! mon Dieu! fit Mercédès. 

—D'ailleurs, pourquoi donc voulez-vous que je sois tué, ma mère? 

«Est-ce que Lamoricière, cet autre Ney du Midi, a été tué? 

«Est-ce que Changarnier a été tué? 

«Est-ce que Bedeau a été tué? 

«Est-ce que Morrel, que nous connaissons, a été tué? 

«Songez donc à votre joie, ma mère, lorsque vous me verrez revenir avec mon uniforme brodé! 

«Je vous déclare que je compte être superbe là-dessous, et que j'ai choisi ce régiment-là par coquetterie.» 

Mercédès soupira, tout en essayant de sourire; elle comprenait, cette sainte mère, qu'il était mal à elle de laisser porter à son enfant tout le poids du sacrifice. 

«Eh bien, donc! reprit Albert, vous comprenez, ma mère, voilà déjà plus de quatre mille francs assurés pour vous: avec ces quatre mille francs vous vivrez deux bonnes années. 

—Crois-tu?» dit Mercédès. 

Ces mots étaient échappés à la comtesse, et avec une douleur si vraie que leur véritable sens n'échappa point à Albert; il sentit son cœur se serrer, et, prenant la main de sa mère, qu'il pressa tendrement dans les siennes: 

«Oui, vous vivrez! dit-il. 

—Je vivrai! s'écria Mercédès, mais tu ne partiras point, n'est-ce pas, mon fils? 

—Ma mère, je partirai, dit Albert d'une voix calme et ferme, vous m'aimez trop pour me laisser près de vous oisif et inutile; d'ailleurs j'ai signé. 

—Tu feras selon ta volonté, mon fils; moi, je ferai selon celle de Dieu. 

—Non pas selon ma volonté, ma mère, mais selon la raison, selon la nécessité. Nous sommes deux créatures désespérées, n'est-ce pas? Qu'est-ce que la vie pour vous aujourd'hui? rien. Qu'est-ce que la vie pour moi? oh! bien peu de chose sans vous, ma mère, croyez-le; car sans vous cette vie, je vous le jure, eût cessé du jour où j'ai douté de mon père et renié son nom! Enfin, je vis, si vous me promettez d'espérer encore; si vous me laissez le soin de votre bonheur à venir, vous doublez ma force. Alors je vais trouver là-bas le gouverneur de l'Algérie, c'est un cœur loyal et surtout essentiellement soldat; je lui conte ma lugubre histoire: je le prie de tourner de temps en temps les yeux du côté où je serai, et s'il me tient parole, s'il me regarde faire, avant six mois je suis officier ou mort. Si je suis officier, votre sort est assuré, ma mère, car j'aurai de l'argent pour vous et pour moi, de plus un nouveau nom dont nous serons fiers tous deux, puisque ce sera votre vrai nom. Si je suis tué\dots eh bien, si je suis tué, alors, chère mère, vous mourrez, s'il vous plaît, et alors nos malheurs auront leur terme dans leur excès même. 

—C'est bien, répondit Mercédès avec son noble et éloquent regard; tu as raison, mon fils: prouvons à certaines gens qui nous regardent et qui attendent nos actes pour nous juger, prouvons-leur que nous sommes au moins dignes d'être plaints. 

—Mais pas de funèbres idées, chère mère! s'écria le jeune homme; je vous jure que nous sommes, ou du moins que nous pouvons être très heureux. Vous êtes à la fois une femme pleine d'esprit et de résignation; moi, je suis devenu simple de goût et sans passion, je l'espère. Une fois au service, me voilà riche; une fois dans la maison de M. Dantès, vous voilà tranquille. Essayons! je vous en prie, ma mère, essayons. 

—Oui, essayons, mon fils, car tu dois vivre, car tu dois être heureux, répondit Mercédès. 

—Ainsi, ma mère, voilà notre partage fait, ajouta le jeune homme en affectant une grande aisance. Nous pouvons aujourd'hui même partir. Allons, je retiens, comme il est dit, votre place. 

—Mais la tienne, mon fils? 

—Moi, je dois rester deux ou trois jours encore, ma mère; c'est un commencement de séparation, et nous avons besoin de nous y habituer. J'ai besoin de quelques recommandations, de quelques renseignements sur l'Afrique, je vous rejoindrai à Marseille. 

—Eh bien, soit, partons! dit Mercédès en s'enveloppant dans le seul châle qu'elle eût emporté, et qui se trouvait par hasard être un cachemire noir d'un grand prix; partons!» 

Albert recueillit à la hâte ses papiers, sonna pour payer les trente francs qu'il devait au maître de l'hôtel, et, offrant son bras à sa mère, il descendit l'escalier. 

Quelqu'un descendait devant eux; ce quelqu'un, entendant le frôlement d'une robe de soie sur la rampe, se retourna. 

«Debray! murmura Albert. 

—Vous, Morcerf!» répondit le secrétaire du ministre en s'arrêtant sur la marche où il se trouvait. 

La curiosité l'emporta chez Debray sur le désir de garder l'incognito; d'ailleurs il était reconnu. 

Il semblait piquant, en effet, de retrouver dans cet hôtel ignoré le jeune homme dont la malheureuse aventure venait de faire un si grand éclat dans Paris. 

«Morcerf!» répéta Debray. 

Puis, apercevant dans la demi-obscurité la tournure jeune encore et le voile noir de Mme de Morcerf: 

«Oh! pardon, ajouta-t-il avec un sourire, je vous laisse, Albert.» 

Albert comprit la pensée de Debray. 

«Ma mère, dit-il en se retournant vers Mercédès, c'est M. Debray, secrétaire du ministre de l'Intérieur, un ancien ami à moi. 

—Comment! ancien, balbutia Debray; que voulez-vous dire? 

—Je dis cela, monsieur Debray, reprit Albert, parce qu'aujourd'hui je n'ai plus d'amis, et que je ne dois plus en avoir. Je vous remercie beaucoup d'avoir bien voulu me reconnaître, monsieur.» 

Debray remonta deux marches et vint donner une énergique poignée de main à son interlocuteur. 

«Croyez, mon cher Albert, dit-il avec l'émotion qu'il était susceptible d'avoir, croyez que j'ai pris une part profonde au malheur qui vous frappe, et que, pour toutes choses, je me mets à votre disposition. 

—Merci, monsieur, dit en souriant Albert, mais au milieu de ce malheur, nous sommes demeurés assez riches pour n'avoir besoin de recourir à personne; nous quittons Paris, et, notre voyage payé, il nous reste cinq mille francs.» 

Le rouge monta au front de Debray, qui tenait un million dans son portefeuille; et si peu poétique que fût cet esprit exact, il ne put s'empêcher de réfléchir que la même maison contenait naguère encore deux femmes, dont l'une, justement déshonorée, s'en allait pauvre avec quinze cent mille francs sous le pli de son manteau, et dont l'autre, injustement frappée, mais sublime en son malheur, se trouvait riche avec quelques deniers. 

Ce parallèle dérouta ses combinaisons de politesse, la philosophie de l'exemple l'écrasa; il balbutia quelques mots de civilité générale et descendit rapidement. 

Ce jour-là, les commis du ministère, ses subordonnés, eurent fort à souffrir de son humeur chagrine. 

Mais le soir il se rendit acquéreur d'une fort belle maison, sise boulevard de la Madeleine, et rapportant cinquante mille livres de rente. 

Le lendemain, à l'heure où Debray signait l'acte, c'est-à-dire sur les cinq heures du soir, Mme de Morcerf, après avoir tendrement embrassé son fils et après avoir été tendrement embrassée par lui, montait dans le coupé de la diligence, qui se refermait sur elle. 

Un homme était caché dans la cour des messageries Laffitte derrière une de ces fenêtres cintrées d'entresol qui surmontent chaque bureau; il vit Mercédès monter en voiture; il vit partir la diligence; il vit s'éloigner Albert. 

Alors il passa la main sur son front chargé de doute en disant: 

«Hélas! par quel moyen rendrai-je à ces deux innocents le bonheur que je leur ai ôté? Dieu m'aidera.» 