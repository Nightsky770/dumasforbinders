\chapter{La carte de Luigi Vampa} 

\lettrine{\accentletter[\gravebox]{A}}{} tout sommeil qui n'est pas celui que redoutait Danglars, il y a un réveil. 

\zz
Danglars se réveilla. 

\zz
Pour un Parisien habitué aux rideaux de soie, aux parois veloutées des murailles, au parfum qui monte du bois blanchissant dans la cheminée et qui descend des voûtes de satin, le réveil dans une grotte de pierre crayeuse doit être comme un rêve de mauvais aloi. 

En touchant ses courtines de peau de bouc, Danglars devait croire qu'il rêvait Samoïèdes ou Lapons. 

Mais en pareille circonstance une seconde suffit pour changer le doute le plus robuste en certitude. 

«Oui, oui, murmura-t-il, je suis aux mains des bandits dont nous a parlé Albert de Morcerf.» 

Son premier mouvement fut de respirer, afin de s'assurer qu'il n'était pas blessé: c'était un moyen qu'il avait trouvé dans \textit{Don Quichotte}, le seul livre, non pas qu'il eût lu, mais dont il eût retenu quelque chose. 

«Non, dit-il, ils ne m'ont tué ni blessé, mais ils m'ont volé peut-être?» 

Et il porta vivement ses mains à ses poches. Elles étaient intactes: les cent louis qu'il s'était réservés pour faire son voyage de Rome à Venise étaient bien dans la poche de son pantalon, et le portefeuille dans lequel se trouvait la lettre de crédit de cinq millions cinquante mille francs était bien dans la poche de sa redingote. 

«Singuliers bandits, se dit-il, qui m'ont laissé ma bourse et mon portefeuille! Comme je le disais hier en me couchant, ils vont me mettre à rançon. Tiens! j'ai aussi ma montre! Voyons un peu quelle heure il est.» 

La montre de Danglars, chef-d'œuvre de Bréguet, qu'il avait remontée avec soin la veille avant de se mettre en route, sonna cinq heures et demie du matin. Sans elle, Danglars fût resté complètement incertain sur l'heure, le jour ne pénétrant pas dans sa cellule. 

Fallait-il provoquer une explication des bandits? fallait-il attendre patiemment qu'ils la demandassent? La dernière alternative était la plus prudente: Danglars attendit. 

Il attendit jusqu'à midi. 

Pendant tout ce temps, une sentinelle avait veillé à sa porte. À huit heures du matin, la sentinelle avait été relevée. 

Il avait alors pris à Danglars l'envie de voir par qui il était gardé. 

Il avait remarqué que des rayons de lumière, non pas de jour, mais de lampe, filtraient à travers les ais de la porte mal jointe, il s'approcha d'une de ces ouvertures au moment juste où le bandit buvait quelques gorgées d'eau-de-vie, lesquelles, grâce à l'outre de peau qui les contenait, répandaient une odeur qui répugna fort à Danglars. 

«Pouah!» fit-il en reculant jusqu'au fond de sa cellule. 

À midi, l'homme à l'eau-de-vie fut remplacé par un autre factionnaire. Danglars eut la curiosité de voir son nouveau gardien; il s'approcha de nouveau de la jointure. 

Celui-là était un athlétique bandit, un Goliath aux gros yeux, aux lèvres épaisses, au nez écrasé; sa chevelure rousse pendait sur ses épaules en mèches tordues comme des couleuvres. 

«Oh! oh! dit Danglars, celui ici ressemble plus à un ogre qu'à une créature humaine; en tout cas, je suis vieux et assez coriace; gros blanc pas bon à manger.» 

Comme on voit, Danglars avait encore l'esprit assez présent pour plaisanter. 

Au même instant, comme pour lui donner la preuve qu'il n'était pas un ogre, son gardien s'assit en face de la porte de sa cellule, tira de son bissac du pain noir, des oignons et du fromage, qu'il se mit incontinent à dévorer. 

«Le diable m'emporte, dit Danglars en jetant à travers les fentes de sa porte un coup d'œil sur le dîner du bandit: le diable m'emporte si je comprends comment on peut manger de pareilles ordures.» 

Et il alla s'asseoir sur ses peaux de bouc, qui lui rappelaient l'odeur de l'eau-de-vie de la première sentinelle. 

Mais Danglars avait beau faire, et les secrets de la nature sont incompréhensibles, il y a bien de l'éloquence dans certaines invitations matérielles qu'adressent les plus grossières substances aux estomacs à jeun. 

Danglars sentit soudain que le sien n'avait pas de fonds en ce moment: il vit l'homme moins laid, le pain moins noir, le fromage plus frais. 

Enfin, ces oignons crus, affreuse alimentation du sauvage, lui rappelèrent certaines sauces Robert et certains mirotons que son cuisinier exécutait d'une façon supérieure, lorsque Danglars lui disait: «Monsieur Deniseau, faites-moi, pour aujourd'hui, un bon petit plat canaille.» 

Il se leva et alla frapper à la porte. 

Le bandit leva la tête. 

Danglars vit qu'il était entendu, et redoubla. 

«\textit{Che cosa}? demanda le bandit. 

—Dites donc! dites donc! l'ami, fit Danglars en tambourinant avec ses doigts contre sa porte, il me semble qu'il serait temps que l'on songeât à me nourrir aussi, moi!» 

Mais soit qu'il ne comprît pas, soit qu'il n'eût pas d'ordres à l'endroit de la nourriture de Danglars, le géant se remit à son dîner. 

Danglars sentit sa fierté humiliée, et, ne voulant pas davantage se commettre avec cette brute, il se recoucha sur ses peaux de bouc et ne souffla plus le mot. 

Quatre heures s'écoulèrent; le géant fut remplacé par un autre bandit. Danglars, qui éprouvait d'affreux tiraillements d'estomac, se leva doucement, appliqua derechef son oreille aux fentes de la porte, et reconnut la figure intelligente de son guide. 

C'était en effet Peppino qui se préparait à monter la garde la plus douce possible en s'asseyant en face de la porte, et en posant entre ses deux jambes une casserole de terre, laquelle contenait, chauds et parfumés, des pois chiches fricassés au lard. 

Près de ces pois chiches, Peppino posa encore un joli petit panier de raisin de Velletri et un fiasco de vin d'Orvietto. 

Décidément Peppino était un gourmet. 

En voyant ces préparatifs gastronomiques, l'eau vint à la bouche de Danglars. 

«Ah! ah! dit le prisonnier, voyons un peu si celui-ci sera plus traitable que l'autre.» 

Et il frappa gentiment à sa porte. 

«On y va, dit le bandit, qui, en fréquentant la maison de maître Pastrini, avait fini par apprendre le français jusque dans ses idiotismes.» 

En effet il vint ouvrir. 

Danglars le reconnut pour celui qui lui avait crié d'une si furieuse manière: «Rentrez la tête.» Mais ce n'était pas l'heure des récriminations. Il prit au contraire sa figure la plus agréable, et avec un sourire gracieux: 

«Pardon, monsieur, dit-il, mais est-ce que l'on ne me donnera pas à dîner, à moi aussi? 

—Comment donc! s'écria Peppino, Votre Excellence aurait-elle faim, par hasard? 

—Par hasard est charmant, murmura Danglars; il y a juste vingt-quatre heures que je n'ai mangé. 

«Mais oui, monsieur, ajouta-t-il en haussant la voix, j'ai faim, et même assez faim. 

—Et Votre Excellence veut manger? 

—À l'instant même, si c'est possible. 

—Rien de plus aisé, dit Peppino; ici l'on se procure tout ce que l'on désire, en payant, bien entendu comme cela se fait chez tous les honnêtes chrétiens. 

—Cela va sans dire! s'écria Danglars, quoique en vérité les gens qui vous arrêtent et qui vous emprisonnent devraient au moins nourrir leurs prisonniers. 

—Ah! Excellence, reprit Peppino, ce n'est pas l'usage. 

—C'est une assez mauvaise raison, reprit Danglars, qui comptait amadouer son gardien par son amabilité, et cependant je m'en contente. Voyons, qu'on me serve à manger. 

—À l'instant même, Excellence; que désirez-vous?» 

Et Peppino posa son écuelle à terre, de telle façon que la fumée en monta directement aux narines de Danglars. 

«Commandez, dit-il. 

—Vous avez donc des cuisines ici? demanda le banquier. 

—Comment! si nous avons des cuisines? des cuisines parfaites! 

—Et des cuisiniers? 

—Excellents! 

—Eh bien, un poulet, un poisson, du gibier, n'importe quoi, pourvu que je mange. 

—Comme il plaira à Votre Excellence; nous disons un poulet, n'est-ce pas? 

—Oui, un poulet.» 

Peppino, se redressant, cria de tous ses poumons: 

«Un poulet pour Son Excellence!» 

La voix de Peppino vibrait encore sous les voûtes que déjà paraissait un jeune homme, beau, svelte, et à moitié nu comme les porteurs de poissons antiques; il apportait le poulet sur un plat d'argent, et le poulet tenait seul sur sa tête. 

«On se croirait au \textit{Café de Paris}, murmura Danglars. 

—Voilà, Excellence», dit Peppino en prenant le poulet des mains du jeune bandit et en le posant sur une table vermoulue qui faisait, avec un escabeau et le lit de peaux de bouc, la totalité de l'ameublement de la cellule. 

Danglars demanda un couteau et une fourchette. 

«Voilà! Excellence», dit Peppino en offrant un petit couteau à la pointe émoussée et une fourchette de bois. 

Danglars prit le couteau d'une main, la fourchette de l'autre, et se mit en devoir de découper la volaille. 

«Pardon, Excellence, dit Peppino en posant une main sur l'épaule du banquier; ici on paie avant de manger; on pourrait n'être pas content en sortant\dots 

—Ah! ah! fit Danglars, ce n'est plus comme à Paris, sans compter qu'ils vont m'écorcher probablement; mais faisons les choses grandement. Voyons, j'ai toujours entendu parler du bon marché de la vie en Italie; un poulet doit valoir douze sous à Rome. 

«Voilà», dit-il, et il jeta un louis à Peppino. 

Peppino ramassa le louis, Danglars approcha le couteau du poulet. 

«Un moment, Excellence, dit Peppino en se relevant; un moment, Votre Excellence me redoit encore quelque chose. 

—Quand je disais qu'ils m'écorcheraient!» murmura Danglars. 

Puis, résolu de prendre son parti de cette extorsion: 

«Voyons, combien vous redoit-on pour cette volaille étique? demanda-t-il. 

—Votre Excellence a donné un louis d'acompte. 

—Un louis d'acompte sur un poulet? 

—Sans doute, d'acompte. 

—Bien\dots Allez! allez! 

—Ce n'est plus que quatre mille neuf cent quatre-vingt-dix-neuf louis que Votre Excellence me redoit.» 

Danglars ouvrit des yeux énormes à l'énoncé de cette gigantesque plaisanterie. 

«Ah! très drôle, murmura-t-il, en vérité.» 

Et il voulut se remettre à découper le poulet; mais Peppino lui arrêta la main droite avec la main gauche et tendit son autre main. 

«Allons, dit-il. 

—Quoi! vous ne riez point? dit Danglars. 

—Nous ne rions jamais, Excellence, reprit Peppino, sérieux comme un quaker. 

—Comment, cent mille francs ce poulet! 

—Excellence, c'est incroyable comme on a de la peine à élever la volaille dans ces maudites grottes. 

—Allons! allons! dit Danglars, je trouve cela très bouffon, très divertissant, en vérité; mais comme j'ai faim, laissez-moi manger. Tenez, voilà un autre louis pour vous, mon ami. 

—Alors cela ne fera plus que quatre mille neuf cent quatre-vingt-dix-huit louis, dit Peppino conservant le même sang-froid; avec de la patience, nous y viendrons. 

—Oh! quant à cela, dit Danglars révolté de cette persévérance à le railler, quant à cela, jamais. Allez au diable! Vous ne savez pas à qui vous avez affaire.» 

Peppino fit un signe, le jeune garçon allongea les deux mains et enleva prestement le poulet. Danglars se jeta sur son lit de peaux de bouc, Peppino referma la porte et se remit à manger ses pois au lard. 

Danglars ne pouvait voir ce que faisait Peppino, mais le claquement des dents du bandit ne devait laisser au prisonnier aucun doute sur l'exercice auquel il se livrait. 

Il était clair qu'il mangeait, même qu'il mangeait bruyamment, et comme un homme mal élevé. 

«Butor!» dit Danglars. 

Peppino fit semblant de ne pas entendre, et, sans même tourner la tête, continua de manger avec une sage lenteur. 

L'estomac de Danglars lui semblait à lui-même percé comme le tonneau des Danaïdes; il ne pouvait croire qu'il parviendrait à le remplir jamais. 

Cependant, il prit patience une demi-heure encore mais il est juste de dire que cette demi-heure lui parut un siècle. 

Il se leva et alla de nouveau à la porte. 

«Voyons, monsieur, dit-il, ne me faites pas languir plus longtemps, et dites-moi tout de suite ce que l'on veut de moi? 

—Mais, Excellence, dites plutôt ce que vous voulez de nous\dots Donnez vos ordres et nous les exécuterons. 

—Alors ouvrez-moi d'abord.» 

Peppino ouvrit. 

«Je veux, dit Danglars, pardieu! je veux manger! 

—Vous avez faim? 

—Et vous le savez, du reste. 

—Que désire manger Votre Excellence? 

—Un morceau de pain sec, puisque les poulets sont hors de prix dans ces maudites caves. 

—Du pain! soit, dit Peppino. 

«Holà! du pain!» cria-t-il. 

Le jeune garçon apporta un petit pain. 

«Voilà! dit Peppino. 

—Combien? demanda Danglars. 

—Quatre mille neuf cent quatre-vingt-dix-huit louis, il y a deux louis payés d'avance. 

—Comment, un pain, cent mille francs? 

—Cent mille francs, dit Peppino. 

—Mais vous ne demandiez que cent mille francs pour un poulet! 

—Nous ne servons pas à la carte, mais à prix fixe. Qu'on mange peu, qu'on mange beaucoup, qu'on demande dix plats ou un seul, c'est toujours le même chiffre. 

—Encore cette plaisanterie! Mon cher ami, je vous déclare que c'est absurde, que c'est stupide! Dites-moi tout de suite que vous voulez que je meure de faim, ce sera plus tôt fait. 

—Mais non, Excellence, c'est vous qui voulez vous suicider. Payez et mangez. 

—Avec quoi payer, triple animal? dit Danglars exaspéré. Est-ce que tu crois qu'on a cent mille francs dans sa poche? 

—Vous avez cinq millions cinquante mille francs dans la vôtre, Excellence, dit Peppino; cela fait cinquante poulets à cent mille francs et un demi-poulet à cinquante mille.» 

Danglars frissonna; le bandeau lui tomba des yeux: c'était bien toujours une plaisanterie, mais il la comprenait enfin. 

Il est même juste de dire qu'il ne la trouvait plus aussi plate que l'instant d'avant. 

«Voyons, dit-il, voyons: en donnant ces cent mille francs, me tiendrez-vous quitte au moins, et pourrai-je manger à mon aise? 

—Sans doute, dit Peppino. 

—Mais comment les donner? fit Danglars en respirant plus librement. 

—Rien de plus facile; vous avez un crédit ouvert chez MM. Thomson et French, via dei Banchi, à Rome, donnez-moi un bon de quatre mille neuf cent quatre-vingt-dix-huit louis sur ces messieurs, notre banquier nous le prendra.» 

Danglars voulut au moins se donner le mérite de la bonne volonté; il prit la plume et le papier que lui présentait Peppino, écrivit la cédule, et signa. 

«Tenez, dit-il, voilà votre bon au porteur. 

—Et vous, voici votre poulet.» 

Danglars découpa la volaille en soupirant: elle lui paraissait bien maigre pour une si grosse somme. 

Quant à Peppino, il lut attentivement le papier, le mit dans sa poche, et continua de manger ses pois chiches. 