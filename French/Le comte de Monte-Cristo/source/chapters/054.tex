\chapter{La hausse et la baisse}

\lettrine{Q}{uelques} jours après cette rencontre, Albert de Morcerf vint faire visite au comte de Monte-Cristo dans sa maison des Champs-Élysées, qui avait déjà pris cette allure de palais, que le comte, grâce à son immense fortune, donnait à ses habitations même les plus passagères. 

Il venait lui renouveler les remerciements de Mme Danglars, que lui avait déjà apportés une lettre signée baronne Danglars, née Herminie de Servieux. 

Albert était accompagné de Lucien Debray, lequel joignit aux paroles de son ami quelques compliments qui n'étaient pas officiels sans doute, mais dont, grâce à la finesse de son coup d'œil, le comte ne pouvait suspecter la source. 

Il lui sembla même que Lucien venait le voir, mû par un double sentiment de curiosité, et que la moitié de ce sentiment émanait de la rue de la Chaussée-d'Antin. En effet, il pouvait supposer, sans crainte de se tromper, que Mme Danglars, ne pouvant connaître par ses propres yeux l'intérieur d'un homme qui donnait des chevaux de trente mille francs, et qui allait à l'Opéra avec une esclave grecque portant un million de diamants, avait chargé les yeux par lesquels elle avait l'habitude de voir de lui donner des renseignements sur cet intérieur. 

Mais le comte ne parut pas soupçonner la moindre corrélation entre la visite de Lucien et la curiosité de la baronne. 

«Vous êtes en rapports presque continuels avec le baron Danglars? demanda-t-il à Albert de Morcerf. 

—Mais oui, monsieur le comte; vous savez ce que je vous ai dit. 

—Cela tient donc toujours? 

—Plus que jamais, dit Lucien; c'est une affaire arrangée.» 

Et Lucien, jugeant sans doute que ce mot mêlé à la conversation lui donnait le droit d'y demeurer étranger, plaça son lorgnon d'écaille dans son œil, et mordant la pomme d'or de sa badine, se mit à faire le tour de la chambre en examinant les armes et les tableaux. 

«Ah! dit Monte-Cristo; mais, à vous entendre, je n'avais pas cru à une si prompte solution. 

—Que voulez-vous? les choses marchent sans qu'on s'en doute; pendant que vous ne songez pas à elles, elles songent à vous; et quand vous vous retournez vous êtes étonné du chemin qu'elles ont fait. Mon père et M. Danglars ont servi ensemble en Espagne, mon père dans l'armée, M. Danglars dans les vivres. C'est là que mon père, ruiné par la Révolution, et M. Danglars, qui n'avait, lui, jamais eu de patrimoine, ont jeté les fondements, mon père, de sa fortune politique et militaire, qui est belle, M. Danglars, de sa fortune politique et financière, qui est admirable. 

—Oui, en effet, dit Monte-Cristo, je crois que, pendant la visite que je lui ai faite, M. Danglars m'a parlé de cela; et, continua-t-il en jetant un coup d'œil sur Lucien, qui feuilletait un album, et elle est jolie, Mlle Eugénie? car je crois me rappeler que c'est Eugénie qu'elle s'appelle.  

—Fort jolie, ou plutôt fort belle, répondit Albert, mais d'une beauté que je n'apprécie pas. Je suis un indigne! 

—Vous en parlez déjà comme si vous étiez son mari! 

—Oh! fit Albert, en regardant autour de lui pour voir à son tour ce que faisait Lucien. 

—Savez-vous, dit Monte-Cristo en baissant la voix, que vous ne me paraissez pas enthousiaste de ce mariage! 

—Mlle Danglars est trop riche pour moi, dit Morcerf, cela m'épouvante. 

—Bah! dit Monte-Cristo, voilà une belle raison; n'êtes-vous pas riche vous-même? 

—Mon père a quelque chose comme une cinquantaine de mille livres de rente, et m'en donnera peut-être dix ou douze en me mariant. 

—Le fait est que c'est modeste, dit le comte, à Paris surtout; mais tout n'est pas dans la fortune en ce monde, et c'est bien quelque chose aussi qu'un beau nom et une haute position sociale. Votre nom est célèbre, votre position magnifique, et puis le comte de Morcerf est un soldat, et l'on aime à voir s'allier cette intégrité de Bayard à la pauvreté de Duguesclin; le désintéressement est le plus beau rayon de soleil auquel puisse reluire une noble épée. Moi, tout au contraire, je trouve cette union on ne peut plus sortable: Mlle Danglars vous enrichira et vous l'anoblirez!» 

Albert secoua la tête et demeura pensif. 

«Il y a encore autre chose, dit-il. 

—J'avoue, reprit Monte-Cristo, que j'ai peine à comprendre cette répugnance pour une jeune fille riche et belle. 

—Oh! mon Dieu! dit Morcerf, cette répugnance, si répugnance il y a, ne vient pas toute de mon côté. 

—Mais de quel côté donc? car vous m'avez dit que votre père désirait ce mariage. 

—Du côté de ma mère, et ma mère est un œil prudent et sûr. Eh bien, elle ne sourit pas à cette union; elle a je ne sais quelle prévention contre les Danglars. 

—Oh! dit le comte avec un ton un peu forcé, cela se conçoit; Mme la comtesse de Morcerf, qui est la distinction, l'aristocratie, la finesse en personne, hésite un peu à toucher une main roturière, épaisse et brutale: c'est naturel. 

—Je ne sais si c'est cela, en effet, dit Albert; mais ce que je sais, c'est qu'il me semble que ce mariage, s'il se fait, la rendra malheureuse. Déjà l'on devait s'assembler pour parler d'affaires il y a six semaines mais j'ai été tellement pris de migraines\dots. 

—Réelles? dit le comte en souriant. 

—Oh! bien réelles, la peur sans doute\dots que l'on a remis le rendez-vous à deux mois. Rien ne presse, vous comprenez; je n'ai pas encore vingt et un ans, et Eugénie n'en a que dix-sept; mais les deux mois expirent la semaine prochaine. Il faudra s'exécuter. Vous ne pouvez vous imaginer, mon cher comte, combien je suis embarrassé\dots Ah! que vous êtes heureux d'être libre! 

—Eh bien, mais soyez libre aussi; qui vous en empêche, je vous le demande un peu? 

—Oh! ce serait une trop grande déception pour mon père si je n'épouse pas Mlle Danglars. 

—Épousez-la alors, dit le comte avec un singulier mouvement d'épaules. 

—Oui, dit Morcerf; mais pour ma mère ce ne sera pas de la déception, mais de la douleur. 

—Alors ne l'épousez pas, fit le comte. 

—Je verrai, j'essaierai, vous me donnerez un conseil, n'est-ce pas? et, s'il vous est possible, vous me tirerez de cet embarras. Oh! pour ne pas faire de peine à mon excellente mère, je me brouillerais avec le comte, je crois.» 

Monte-Cristo se détourna; il semblait ému. 

«Eh! dit-il à Debray, assis dans un fauteuil profond à l'extrémité du salon, et qui tenait de la main droite un crayon et de la gauche un carnet, que faites-vous donc, un croquis d'après le Poussin? 

—Moi? dit-il tranquillement, oh! bien oui! un croquis, j'aime trop la peinture pour cela! Non pas, je fais tout l'opposé de la peinture, je fais des chiffres. 

—Des chiffres?  

—Oui, je calcule; cela vous regarde indirectement, vicomte; je calcule ce que la maison Danglars a gagné sur la dernière hausse d'Haïti: de deux cent six le fonds est monté à quatre cent neuf en trois jours, et le prudent banquier avait acheté beaucoup à deux cent six. Il a dû gagner trois cent mille livres. 

—Ce n'est pas son meilleur coup, dit Morcerf; n'a-t-il pas gagné un million cette année avec les bons d'Espagne? 

—Écoutez, mon cher, dit Lucien, voici M. le comte de Monte-Cristo qui vous dira comme les Italiens: 

\begin{verse}\itshape
Danaro e santità\\Metà della metà
[Argent et sainteté,\\Moitié de la moitié.]
\end{verse}

Et c'est encore beaucoup. Aussi, quand on me fait de pareilles histoires, je hausse les épaules. 

—Mais vous parliez d'Haïti? dit Monte-Cristo. 

—Oh! Haïti, c'est autre chose; Haïti, c'est l'écarté de l'agiotage français. On peut aimer la bouillotte, chérir le whist, raffoler du boston, et se lasser cependant de tout cela; mais on en revient toujours à l'écarté: c'est un hors-d'œuvre. Ainsi M. Danglars a vendu hier à quatre cent six et empoché trois cent mille francs; s'il eût attendu à aujourd'hui, le fonds retombait à deux cent cinq, et au lieu de gagner trois cent mille francs, il en perdait vingt ou vingt-cinq mille. 

—Et pourquoi le fonds est-il retombé de quatre cent neuf à deux cent cinq? demanda Monte-Cristo. Je vous demande pardon, je suis fort ignorant de toutes ces intrigues de Bourse. 

—Parce que, répondit en riant Albert, les nouvelles se suivent et ne se ressemblent pas. 

—Ah! diable, fit le comte, M. Danglars joue à gagner ou à perdre trois cent mille francs en un jour. Ah çà! mais il est donc énormément riche? 

—Ce n'est pas lui qui joue! s'écria vivement Lucien, c'est Mme Danglars; elle est véritablement intrépide. 

—Mais vous qui êtes raisonnable, Lucien, et qui connaissez le peu de stabilité des nouvelles, puisque vous êtes à la source, vous devriez l'empêcher, dit Morcerf avec un sourire. 

—Comment le pourrais-je, si son mari ne réussit pas? demanda Lucien. Vous connaissez le caractère de la baronne, personne n'a d'influence sur elle, et elle ne fait absolument que ce qu'elle veut. 

—Oh! si j'étais à votre place! dit Albert. 

—Eh bien! 

—Je la guérirais, moi; ce serait un service à rendre à son futur gendre. 

—Comment cela? 

—Ah pardieu! c'est bien facile, je lui donnerais une leçon. 

—Une leçon? 

—Oui. Votre position de secrétaire du ministre vous donne une grande autorité pour les nouvelles; vous n'ouvrez pas la bouche que les agents de change ne sténographient au plus vite vos paroles; faites-lui perdre une centaine de mille francs coup sur coup, et cela la rendra prudente. 

—Je ne comprends pas, balbutia Lucien. 

—C'est cependant limpide, répondit le jeune homme avec une naïveté qui n'avait rien d'affecté; annoncez-lui un beau matin quelque chose d'inouï, une nouvelle télégraphique que vous seul puissiez savoir; que Henri IV, par exemple, a été vu hier chez Gabrielle; cela fera monter les fonds, elle établira son coup de bourse là-dessus, et elle perdra certainement lorsque Beauchamp écrira le lendemain dans son journal: «C'est à tort que les gens bien informés prétendent que le roi Henri IV a été vu avant-hier chez Gabrielle, ce fait est complètement inexact; le roi Henri IV n'a pas quitté le pont Neuf.» 

Lucien se mit à rire du bout des lèvres. Monte-Cristo, quoique indifférent en apparence, n'avait pas perdu un mot de cet entretien, et son œil perçant avait même cru lire un secret dans l'embarras du secrétaire intime. 

Il résulta de cet embarras de Lucien, qui avait complètement échappé à Albert, que Lucien abrégea sa visite. 

Il se sentait évidemment mal à l'aise. Le comte lui dit en le reconduisant quelques mots à voix basse auxquels il répondit: 

«Bien volontiers, monsieur le comte, j'accepte.» 

Le comte revint au jeune de Morcerf. 

«Ne pensez-vous pas, en y réfléchissant, lui dit-il, que vous avez eu tort de parler comme vous l'avez fait de votre belle-mère devant M. Debray? 

—Tenez, comte, dit Morcerf, je vous en prie, ne dites pas d'avance ce mot-là. 

—Vraiment, et sans exagération, la comtesse est à ce point contraire à ce mariage? 

—À ce point que la baronne vient rarement à la maison, et que ma mère, je crois, n'a pas été deux fois dans sa vie chez madame Danglars. 

—Alors, dit le comte, me voilà enhardi à vous parler à cœur ouvert: M. Danglars est mon banquier, M. de Villefort m'a comblé de politesse en remerciement d'un service qu'un heureux hasard m'a mis à même de lui rendre. Je devine sous tout cela une avalanche de dîners et de raouts. Or, pour ne pas paraître brocher fastueusement sur le tout, et même pour avoir le mérite de prendre les devants, si vous voulez, j'ai projeté de réunir dans ma maison de campagne d'Auteuil M. et Mme Danglars, M. et Mme de Villefort. Si je vous invite à ce dîner, ainsi que M. le comte et Mme la comtesse de Morcerf, cela n'aura-t-il pas l'air d'une espèce de rendez-vous matrimonial, ou du moins Mme la comtesse de Morcerf n'envisagera-t-elle point la chose ainsi, surtout si M. le baron Danglars me fait l'honneur d'amener sa fille? Alors votre mère me prendra en horreur, et je ne veux aucunement de cela, moi; je tiens, au contraire, et dites-le-lui toutes les fois que l'occasion s'en présentera, à rester au mieux dans son esprit. 

—Ma foi, comte, dit Morcerf, je vous remercie d'y mettre avec moi cette franchise, et j'accepte l'exclusion que vous me proposez. Vous dites que vous tenez à rester au mieux dans l'esprit de ma mère, où vous êtes déjà à merveille. 

—Vous croyez? fit Monte-Cristo avec intérêt. 

—Oh! j'en suis sûr. Quand vous nous avez quittés l'autre jour, nous avons causé une heure de vous mais j'en reviens à ce que nous disions. Eh bien, si ma mère pouvait savoir cette attention de votre part, et je me hasarderai à la lui dire, je suis sûr qu'elle vous en serait on ne peut plus reconnaissante. Il est vrai que de son côté, mon père serait furieux.» 

Le comte se mit à rire.  

«Eh bien, dit-il à Morcerf, vous voilà prévenu. Mais j'y pense, il n'y aura pas que votre père qui sera furieux; M. et Mme Danglars vont me considérer comme un homme de fort mauvaise façon. Ils savent que je vous vois avec une certaine intimité, que vous êtes même ma plus ancienne connaissance parisienne et ils ne vous trouveront pas chez moi; ils me demanderont pourquoi je ne vous ai pas invité. Songez au moins à vous munir d'un engagement antérieur qui ait quelque apparence de probabilité, et dont vous me ferez part au moyen d'un petit mot. Vous le savez, avec les banquiers les écrits sont seuls valables. 

—Je ferai mieux que cela, monsieur le comte, dit Albert. Ma mère veut aller respirer l'air de la mer. À quel jour est fixé votre dîner? 

—À samedi. 

—Nous sommes à mardi, bien; demain soir nous partons; après-demain nous serons au Tréport. Savez-vous, monsieur le comte, que vous êtes un homme charmant de mettre ainsi les gens à leur aise! 

—Moi! en vérité vous me tenez pour plus que je ne vaux; je désire vous être agréable, voilà tout. 

—Quel jour avez-vous fait vos invitations? 

—Aujourd'hui même. 

—Bien! Je cours chez M. Danglars, je lui annonce que nous quittons Paris demain, ma mère et moi. Je ne vous ai pas vu; par conséquent je ne sais rien de votre dîner. 

—Fou que vous êtes! et M. Debray, qui vient de vous voir chez moi, lui! 

—Ah! c'est juste. 

—Au contraire, je vous ai vu et invité ici sans cérémonie, et vous m'avez tout naïvement répondu que vous ne pouviez pas être mon convive, parce que vous partiez pour le Tréport. 

—Eh bien, voilà qui est conclu. Mais vous, viendrez-vous voir ma mère avant demain? 

—Avant demain, c'est difficile; puis je tomberais au milieu de vos préparatifs de départ. 

—Eh bien, faites mieux que cela; vous n'étiez qu'un homme charmant, vous serez un homme adorable.  

—Que faut-il que je fasse pour arriver à cette sublimité? 

—Ce qu'il faut que vous fassiez? 

—Je le demande. 

—Vous êtes aujourd'hui libre comme l'air; venez dîner avec moi: nous serons en petit comité, vous, ma mère et moi seulement. Vous avez à peine aperçu ma mère; mais vous la verrez de près. C'est une femme fort remarquable, et je ne regrette qu'une chose: c'est que sa pareille n'existe pas avec vingt ans de moins; il y aurait bientôt, je vous le jure, une comtesse et une vicomtesse de Morcerf. Quant à mon père, vous ne le trouverez pas: il est de commission ce soir et dîne chez le grand référendaire. Venez, nous causerons voyages. Vous qui avez vu le monde tout entier, vous nous raconterez vos aventures; vous nous direz l'histoire de cette belle Grecque qui était l'autre soir avec vous à l'Opéra, que vous appelez votre esclave et que vous traitez comme une princesse. Nous parlerons italien, espagnol. Voyons, acceptez; ma mère vous remerciera. 

—Mille grâces, dit le comte; l'invitation est des plus gracieuses, et je regrette vivement de ne pouvoir l'accepter. Je ne suis pas libre comme vous le pensiez, et j'ai au contraire un rendez-vous des plus importants. 

—Ah! prenez garde; vous m'avez appris tout à l'heure comment, en fait de dîner, on se décharge d'une chose désagréable. Il me faut une preuve. Je ne suis heureusement pas banquier comme M. Danglars; mais je suis, je vous en préviens, aussi incrédule que lui. 

—Aussi vais-je vous la donner», dit le comte. 

Et il sonna. 

«Hum! fit Morcerf, voilà déjà deux fois que vous refusez de dîner avec ma mère. C'est un parti pris, comte.» 

Monte-Cristo tressaillit. 

«Oh! vous ne le croyez pas, dit-il; d'ailleurs voici ma preuve qui vient.» 

Baptistin entra et se tint sur la porte debout et attendant. 

«Je n'étais pas prévenu de votre visite, n'est-ce pas? 

—Dame! vous êtes un homme si extraordinaire que je n'en répondrais pas. 

—Je ne pouvais point deviner que vous m'inviteriez à dîner, au moins. 

—Oh! quant à cela, c'est probable. 

—Eh bien, écoutez, Baptistin\dots que vous ai-je dit ce matin quand je vous ai appelé dans mon cabinet de travail? 

—De faire fermer la porte de M. le comte une fois cinq heures sonnées. 

—Ensuite? 

—Oh! monsieur le comte\dots dit Albert. 

—Non, non, je veux absolument me débarrasser de cette réputation mystérieuse que vous m'avez faite, mon cher vicomte. Il est trop difficile de jouer éternellement le Manfred. Je veux vivre dans une maison de verre. Ensuite\dots. Continuez, Baptistin. 

—Ensuite, de ne recevoir que M. le major Bartolomeo Cavalcanti et son fils. 

—Vous entendez, M. le major Bartolomeo Cavalcanti, un homme de la plus vieille noblesse d'Italie et dont Dante a pris la peine d'être le d'Hozier\dots. Vous vous rappelez ou vous ne vous rappelez pas, dans le dixième chant de l'Enfer; de plus, son fils, un charmant jeune homme de votre âge à peu près, vicomte, portant le même titre que vous, et qui fait son entrée dans le monde parisien avec les millions de son père. Le major m'amène ce soir son fils Andrea, le contino, comme nous disons en Italie. Il me le confie. Je le pousserai s'il a quelque mérite. Vous m'aiderez, n'est-ce pas? 

—Sans doute! C'est donc un ancien ami à vous que ce major Cavalcanti? demanda Albert. 

—Pas du tout, c'est un digne seigneur, très poli, très modeste, très discret, comme il y en a une foule en Italie, des descendants très descendus des vieilles familles. Je l'ai vu plusieurs fois, soit à Florence, soit à Bologne, soit à Lucques, et il m'a prévenu de son arrivée. Les connaissances de voyage sont exigeantes: elles réclament de vous, en tout lieu, l'amitié qu'on leur a témoignée une fois par hasard; comme si l'homme civilisé, qui sait vivre une heure avec n'importe qui, n'avait pas toujours son arrière-pensée! Ce bon major Cavalcanti va revoir Paris, qu'il n'a vu qu'en passant, sous l'Empire, en allant se faire geler à Moscou. Je lui donnerai un bon dîner, il me laissera son fils; je lui promettrai de veiller sur lui; je lui laisserai faire toutes les folies qu'il lui conviendra de faire, et nous serons quittes. 

—À merveille! dit Albert, et je vois que vous êtes un précieux mentor. Adieu donc, nous serons de retour dimanche. À propos, j'ai reçu des nouvelles de Franz. 

—Ah! vraiment! dit Monte-Cristo; et se plaît-il toujours en Italie? 

—Je pense que oui; cependant il vous y regrette. Il dit que vous étiez le soleil de Rome, et que sans vous il y fait gris. Je ne sais même pas s'il ne va point jusqu'à dire qu'il y pleut. 

—Il est donc revenu sur mon compte, votre ami Franz? 

—Au contraire, il persiste à vous croire fantastique au premier chef; voilà pourquoi il vous regrette. 

—Charmant jeune homme! dit Monte-Cristo, et pour lequel je me suis senti une vive sympathie le premier soir où je l'ai vu cherchant un souper quelconque, et il a bien voulu accepter le mien. C'est, je crois, le fils du général d'Épinay? 

—Justement. 

—Le même qui a été si misérablement assassiné en 1815? 

—Par les bonapartistes. 

—C'est cela! Ma foi, je l'aime! N'y a-t-il pas pour lui aussi des projets de mariage? 

—Oui, il doit épouser Mlle de Villefort. 

—C'est vrai? 

—Comme moi je dois épouser Mlle Danglars, reprit Albert en riant. 

—Vous riez\dots. 

—Oui. 

—Pourquoi riez-vous? 

—Je ris parce qu'il me semble voir de ce côté-là autant de sympathie pour le mariage qu'il y en a d'un autre côté entre Mlle Danglars et moi. Mais vraiment, mon cher comte, nous causons de femmes comme les femmes causent d'hommes; c'est impardonnable!» 

Albert se leva. 

«Vous vous en allez? 

—La question est bonne! il y a deux heures que je vous assomme, et vous avez la politesse de me demander si je m'en vais! En vérité, comte, vous êtes l'homme le plus poli de la terre! Et vos domestiques, comme ils sont dressés! M. Baptistin surtout! je n'ai jamais pu en avoir un comme cela. Les miens semblent tous prendre exemple sur ceux du Théâtre-Français, qui justement parce qu'ils n'ont qu'un mot à dire, viennent toujours le dire sur la rampe. Ainsi, si vous vous défaites de M. Baptistin, je vous demande la préférence. 

—C'est dit, vicomte. 

—Ce n'est pas tout, attendez: faites bien mes compliments à votre discret Lucquois, au seigneur Cavalcante dei Cavalcanti; et si par hasard il tenait à établir son fils, trouvez-lui une femme bien riche, bien noble, du chef de sa mère, du moins, et bien baronne du chef de son père. Je vous y aiderai, moi. 

—Oh! oh! répondit Monte-Cristo, en vérité, vous en êtes là? 

—Oui. 

—Ma foi, il ne faut jurer de rien. 

—Ah! comte, s'écria Morcerf, quel service vous me rendriez, et comme je vous aimerais cent fois davantage encore si, grâce à vous, je restais garçon, ne fût-ce que dix ans. 

—Tout est possible», répondit gravement Monte-Cristo. 

Et prenant congé d'Albert, il rentra chez lui et frappa trois fois sur son timbre. 

Bertuccio parut. 

«Monsieur Bertuccio, dit-il, vous saurez que je reçois samedi dans ma maison d'Auteuil.» 

Bertuccio eut un léger frisson. 

«Bien, monsieur, dit-il. 

—J'ai besoin de vous, continua le comte, pour que tout soit préparé convenablement. Cette maison est fort belle, ou du moins peut être fort belle. 

—Il faudrait tout changer pour en arriver là, monsieur le comte, car les tentures ont vieilli. 

—Changez donc tout, à l'exception d'une seule, celle de la chambre à coucher de damas rouge: vous la laisserez même absolument telle qu'elle est.» 

Bertuccio s'inclina. 

«Vous ne toucherez pas au jardin non plus; mais de la cour, par exemple, faites-en tout ce que vous voudrez; il me sera même agréable qu'on ne la puisse pas reconnaître. 

—Je ferai tout mon possible pour que monsieur le comte soit content; je serais plus rassuré cependant si monsieur le comte me voulait dire ses intentions pour le dîner. 

—En vérité, mon cher monsieur Bertuccio, dit le comte, depuis que vous êtes à Paris je vous trouve dépaysé, trembleur; mais vous ne me connaissez donc plus? 

—Mais enfin Son Excellence pourrait me dire qui elle reçoit! 

—Je n'en sais rien encore, et vous n'avez pas besoin de le savoir non plus. Lucullus dîne chez Lucullus, voilà tout.» 

Bertuccio s'inclina et sortit. 