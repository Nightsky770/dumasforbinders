\chapter{Le pain et le sel}

\lettrine{M}{adame} de Morcerf entra sous la voûte de feuillage avec son compagnon: cette voûte était une allée de tilleuls qui conduisait à une serre. 

\zz
«Il faisait trop chaud dans le salon, n'est-ce pas, monsieur le comte? dit-elle. 

—Oui madame; et votre idée de faire ouvrir les portes et les persiennes est une excellente idée.» 

En achevant ces mots, le comte s'aperçut que la main de Mercédès tremblait. 

«Mais vous, avec cette robe légère et sans autres préservatifs autour du cou que cette écharpe de gaze, vous aurez peut-être froid? dit-il.  

—Savez-vous où je vous mène? dit la comtesse, sans répondre à la question de Monte-Cristo. 

—Non, madame, répondit celui-ci; amis, vous le voyez, je ne fais pas de résistance. 

—À la serre, que vous voyez là, au bout de l'allée que nous suivons.» 

Le comte regarda Mercédès comme pour l'interroger; mais elle continua son chemin sans rien dire, et de son côté Monte-Cristo resta muet. 

On arriva dans le bâtiment, tout garni de fruits magnifiques qui, dès le commencement de juillet, atteignaient leur maturité sous cette température toujours calculée pour remplacer la chaleur du soleil, si souvent absente chez nous. 

La comtesse quitta le bras de Monte-Cristo, et alla cueillir à un cep une grappe de raisin muscat. 

«Tenez, monsieur le comte, dit-elle avec un sourire si triste que l'on eût pu voir poindre les larmes au bord de ses yeux, tenez, nos raisins de France ne sont point comparables, je le sais, à vos raisins de Sicile et de Chypre, mais vous serez indulgent pour notre pauvre soleil du Nord.» 

Le comte s'inclina, et fit un pas en arrière. 

«Vous me refusez? dit Mercédès d'une voix tremblante. 

—Madame, répondit Monte-Cristo, je vous prie bien humblement de m'excuser, mais je ne mange jamais de muscat.» 

Mercédès laissa tomber la grappe en soupirant. Une pêche magnifique pendait à un espalier voisin chauffé, comme le cep de vigne, par cette chaleur artificielle de la serre. Mercédès s'approcha du fruit velouté, et le cueillit. 

«Prenez cette pêche, alors», dit-elle. 

Mais le comte fit le même geste de refus. 

«Oh! encore! dit-elle avec un accent si douloureux qu'on sentait que cet accent étouffait un sanglot; en vérité, j'ai du malheur.» 

Un long silence suivit cette scène; la pêche, comme la grappe de raisin, avait roulé sur le sable. 

«Monsieur le comte, reprit enfin Mercédès en regardant Monte-Cristo d'un œil suppliant, il y a une touchante coutume arabe qui fait amis éternellement ceux qui ont partagé le pain et le sel sous le même toit. 

—Je la connais, madame, répondit le comte; mais nous sommes en France et non en Arabie, et en France, il n'y a pas plus d'amitiés éternelles que de partage du sel et du pain. 

—Mais enfin, dit la comtesse palpitante et les yeux attachés sur les yeux de Monte-Cristo, dont elle ressaisit presque convulsivement le bras avec ses deux mains, nous sommes amis, n'est-ce pas?» 

Le sang afflua au cœur du comte, qui devint pâle comme la mort, puis, remontant du cœur à la gorge, il envahit ses joues et ses yeux nagèrent dans le vague pendant quelques secondes, comme ceux d'un homme frappé d'éblouissement. 

«Certainement que nous sommes amis, madame, répliqua-t-il; d'ailleurs, pourquoi ne le serions-nous pas?» 

Ce ton était si loin de celui que désirait Mme de Morcerf, qu'elle se retourna pour laisser échapper un soupir qui ressemblait à un gémissement. 

«Merci», dit-elle. 

Et elle se remit à marcher. Ils firent ainsi le tour du jardin sans prononcer une seule parole. 

«Monsieur, reprit tout à coup la comtesse après dix minutes de promenade silencieuse, est-il vrai que vous ayez tant vu, tant voyagé, tant souffert? 

—J'ai beaucoup souffert, oui, madame, répondit Monte-Cristo. 

—Mais vous êtes heureux, maintenant? 

—Sans doute, répondit le comte, car personne ne m'entend me plaindre. 

—Et votre bonheur présent vous fait l'âme plus douce? 

—Mon bonheur présent égale ma misère passée, dit le comte. 

—N'êtes-vous pas marié? demanda la comtesse. 

—Moi, marié, répondit Monte-Cristo en tressaillant, qui a pu vous dire cela? 

—On ne me l'a pas dit, mais plusieurs fois on vous a vu conduire à l'Opéra une jeune et belle personne. 

—C'est une esclave que j'ai achetée à Constantinople, madame, une fille de prince dont j'ai fait ma fille, n'ayant pas d'autre affection au monde. 

—Vous vivez seul ainsi? 

—Je vis seul. 

—Vous n'avez pas de sœur\dots de fils\dots de père?\dots 

—Je n'ai personne. 

—Comment pouvez-vous vivre ainsi, sans rien qui vous attache à la vie? 

—Ce n'est pas ma faute, madame. À Malte, j'ai aimé une jeune fille et j'allais l'épouser, quand la guerre est venue et m'a enlevé loin d'elle comme un tourbillon. J'avais cru qu'elle m'aimait assez pour m'attendre, pour demeurer fidèle même à mon tombeau. Quand je suis revenu, elle était mariée. C'est l'histoire de tout homme qui a passé par l'âge de vingt ans. J'avais peut-être le cœur plus faible que les autres, et j'ai souffert plus qu'ils n'eussent fait à ma place, voilà tout.» 

La comtesse s'arrêta un moment, comme si elle eût eu besoin de cette halte pour respirer. 

«Oui, dit-elle, et cet amour vous est resté au cœur\dots. On n'aime bien qu'une fois\dots. Et avez-vous jamais revu cette femme? 

—Jamais. 

—Jamais! 

—Je ne suis point retourné dans le pays où elle était. 

—À Malte? 

—Oui, à Malte. 

—Elle est à Malte, alors? 

—Je le pense. 

—Et lui avez-vous pardonné ce qu'elle vous a fait souffrir? 

—À elle, oui. 

—Mais à elle seulement; vous haïssez toujours ceux qui vous ont séparé d'elle?» 

La comtesse se plaça en face de Monte-Cristo, elle tenait encore à la main un fragment de la grappe parfumée. 

«Prenez, dit-elle. 

—Jamais je ne mange de muscat, madame» répondit Monte-Cristo, comme s'il n'eût été question de rien entre eux à ce sujet. 

La comtesse lança la grappe dans le massif le plus proche avec un geste de désespoir. 

«Inflexible!» murmura-t-elle. 

Monte-Cristo demeura aussi impassible que si le reproche ne lui était pas adressé. Albert accourait en ce moment. 

«Oh! ma mère, dit-il, un grand malheur! 

—Quoi! qu'est-il arrivé? demanda la comtesse en se redressant comme si, après le rêve, elle eût été amenée à la réalité: un malheur, avez-vous dit? En effet, il doit arriver des malheurs. 

—M. de Villefort est ici. 

—Eh bien? 

—Il vient chercher sa femme et sa fille. 

—Et pourquoi cela? 

—Parce que Mme la marquise de Saint-Méran est arrivée à Paris, apportant la nouvelle que M. de Saint-Méran est mort en quittant Marseille, au premier relais. Mme de Villefort, qui était fort gaie, ne voulait ni comprendre, ni croire ce malheur; mais Mlle Valentine, aux premiers mots, et quelques précautions qu'ait prises son père, a tout deviné: ce coup l'a terrassée comme la foudre, et elle est tombée évanouie. 

—Et qu'est M. de Saint-Méran à Mlle de Villefort? demanda le comte. 

—Son grand-père maternel. Il venait pour hâter le mariage de Franz et de sa petite-fille. 

—Ah! vraiment! 

—Voilà Franz retardé. Pourquoi M. de Saint-Méran n'est-il pas aussi bien un aïeul de Mlle Danglars? 

—Albert! Albert! dit Mme de Morcerf du ton d'un doux reproche, que dites-vous là? Ah! monsieur le comte, vous pour qui il a une si grande considération, dites-lui qu'il a mal parlé!» 

Elle fit quelques pas en avant. 

Monte-Cristo la regarda si étrangement et avec une expression à la fois si rêveuse et si empreinte d'une affectueuse admiration, qu'elle revint sur ses pas. 

Alors elle lui prit la main en même temps qu'elle pressait celle de son fils, et les joignant toutes deux: 

«Nous sommes amis, n'est-ce pas? dit-elle. 

—Oh! votre ami, madame, je n'ai point cette prétention, dit le comte; mais, en tout cas, je suis votre bien respectueux serviteur.» 

La comtesse partit avec un inexprimable serrement de cœur; et avant qu'elle eût fait dix pas, le comte lui vit mettre son mouchoir à ses yeux. 

«Est-ce que vous n'êtes pas d'accord, ma mère et vous? demanda Albert avec étonnement. 

—Au contraire, répondit le comte, puisqu'elle vient de me dire devant vous que nous sommes amis.» 

Et ils regagnèrent le salon que venaient de quitter Valentine et M. et Mme de Villefort. Il va sans dire que Morrel était sorti derrière eux. 