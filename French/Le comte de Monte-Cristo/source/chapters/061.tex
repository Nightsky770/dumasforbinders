\chapter{Le moyen de délivrer un jardinier des loirs qui mangent ses pêches}
\chaptermark{Le moyen de délivrer un jardinier des loirs}

\lettrine{N}{on} pas le même soir, comme il l'avait dit, mais le lendemain matin, le comte de Monte-Cristo sortit par la barrière d'Enfer, prit la route d'Orléans, dépassa le village de Linas sans s'arrêter au télégraphe qui, justement au moment où le comte passait, faisait mouvoir ses longs bras décharnés, et gagna la tour de Montlhéry, située, comme chacun sait, sur l'endroit le plus élevé de la plaine de ce nom. 

Au pied de la colline, le comte mit pied à terre, et par un petit sentier circulaire, large de dix-huit pouces, commença de gravir la montagne; arrivé au sommet, il se trouva arrêté par une haie sur laquelle des fruits verts avaient succédé aux fleurs roses et blanches. 

Monte-Cristo chercha la porte du petit enclos, et ne tarda point à la trouver. C'était une petite herse en bois, roulant sur des gonds d'osier et se fermant avec un clou et une ficelle. En un instant le comte fut au courant du mécanisme et la porte s'ouvrit. 

Le comte se trouva alors dans un petit jardin de vingt pieds de long sur douze de large, borné d'un côté par la partie de la haie dans laquelle était encadrée l'ingénieuse machine que nous avons décrite sous le nom de porte, et de l'autre par la vieille tour ceinte de lierre, toute parsemée de ravenelles et de giroflées. 

On n'eût pas dit, à la voir ainsi ridée et fleurie comme une aïeule à qui ses petits-enfants viennent de souhaiter la fête, qu'elle pourrait raconter bien des drames terribles, si elle joignait une voix aux oreilles menaçantes qu'un vieux proverbe donne aux murailles. 

On parcourait ce jardin en suivant une allée sablée de sable rouge, sur lequel mordait, avec des tons qui eussent réjoui l'œil de Delacroix, notre Rubens moderne, une bordure de gros buis, vieille de plusieurs années. Cette allée avait la forme d'un 8, et tournait en s'élançant, de manière à faire dans un jardin de vingt pieds une promenade de soixante. Jamais Flore, la riante et fraîche déesse des bons jardiniers latins, n'avait été honorée d'un culte aussi minutieux et aussi pur que l'était celui qu'on lui rendait dans ce petit enclos. 

En effet, de vingt rosiers qui composaient le parterre, pas une feuille ne portait la trace de la mouche, pas un filet la petite grappe de pucerons verts qui désolent et rongent les plantes grandissant sur un terrain humide. Ce n'était cependant point l'humidité qui manquait à ce jardin: la terre noire comme de la suie, l'opaque feuillage des arbres, le disaient assez; d'ailleurs l'humidité factice eût promptement suppléé à l'humidité naturelle, grâce au tonneau plein d'eau croupissante qui creusait un des angles du jardin, et dans lequel stationnaient, sur une nappe verte, une grenouille et un crapaud qui, par incompatibilité d'humeur, sans doute, se tenaient toujours, en se tournant le dos, aux deux points opposés du cercle. 

D'ailleurs, pas une herbe dans les allées, pas un rejeton parasite dans les plates-bandes; une petite-maîtresse polit et émonde avec moins de soin les géraniums, les cactus et les rhododendrons de sa jardinière de porcelaine que ne le faisait le maître jusqu'alors invisible du petit enclos. 

Monte-Cristo arrêta après avoir refermé la porte en agrafant la ficelle à son clou, et embrassa d'un regard toute la propriété. 

«Il paraît, dit-il, que l'homme du télégraphe a des jardiniers à l'année, ou se livre passionnément à l'agriculture.» 

Tout à coup il se heurta à quelque chose, tapi derrière une brouette chargée de feuillage: ce quelque chose se redressa en laissant échapper une exclamation qui peignait son étonnement, et Monte-Cristo se trouva en face d'un bonhomme d'une cinquantaine d'années qui ramassait des fraises qu'il plaçait sur des feuilles de vigne. 

Il y avait douze feuilles de vigne et presque autant de fraises. 

Le bonhomme, en se relevant, faillit laisser choir fraises, feuilles et assiette. 

«Vous faites votre récolte, monsieur? dit Monte-Cristo en souriant. 

—Pardon, monsieur, répondit le bonhomme en portant la main à sa casquette, je ne suis pas là-haut c'est vrai, mais je viens d'en descendre à l'instant même. 

—Que je ne vous gêne en rien, mon ami, dit le comte; cueillez vos fraises, si toutefois il vous en reste encore.  

—J'en ai encore dix, dit l'homme, car en voici onze, et j'en avais vingt et une, cinq de plus que l'année dernière. Mais ce n'est pas étonnant, le printemps a été chaud cette année, et ce qu'il faut aux fraises, voyez-vous, monsieur, c'est la chaleur. Voilà pourquoi, au lieu de seize que j'ai eues l'année passée, j'en ai cette année, voyez-vous, onze déjà cueillies, douze, treize, quatorze, quinze, seize, dix-sept, dix-huit. Oh! mon Dieu! il m'en manque deux, elles y étaient encore hier, monsieur, elles y étaient, j'en suis sûr, je les ai comptées. Il faut que ce soit le fils de la mère Simon qui me les ait soufflées, je l'ai vu rôder par ici ce matin. Ah! le petit drôle, voler dans un enclos! il ne sait pas où cela peut le mener. 

—En effet, dit Monte-Cristo, c'est grave, mais vous ferez la part de la jeunesse du délinquant et de sa gourmandise. 

—Certainement, dit le jardinier; ce n'en est pas moins fort désagréable. Mais, encore une fois, pardon, monsieur: c'est peut-être un chef que je fais attendre ainsi?» 

Et il interrogeait d'un regard craintif le comte et son habit bleu. 

«Rassurez-vous, mon ami, dit le comte avec ce sourire qu'il faisait, à sa volonté, si terrible et si bienveillant, et qui cette fois n'exprimait que la bienveillance, je ne suis point un chef qui vient pour vous inspecter, mais un simple voyageur conduit par la curiosité et qui commence même à se reprocher sa visite en voyant qu'il vous fait perdre votre temps. 

—Oh! mon temps n'est pas cher, répliqua le bonhomme avec un sourire mélancolique. Cependant c'est le temps du gouvernement, et je ne devrais pas le perdre, mais j'avais reçu le signal qui m'annonçait que je pouvais me reposer une heure (il jeta les yeux sur le cadran solaire, car il y avait de tout dans l'enclos de la tour de Montlhéry, même un cadran solaire), et, vous le voyez. J'avais encore dix minutes devant moi, puis mes fraises étaient mûres, et un jour de plus\dots. D'ailleurs, croiriez-vous, monsieur, que les loirs me les mangent? 

—Ma foi, non, je ne l'aurais pas cru, répondit gravement Monte-Cristo; c'est un mauvais voisinage monsieur, que celui des loirs, pour nous qui ne les mangeons pas confits dans du miel comme faisaient les Romains. 

—Ah! les Romains les mangeaient? fit le jardinier; ils mangeaient les loirs? 

—J'ai lu cela dans Pétrone, dit le comte. 

—Vraiment? Ça ne doit pas être bon, quoi qu'on dise: Gras comme un loir. Et ce n'est pas étonnant monsieur, que les loirs soient gras, attendu qu'ils dorment toute la sainte journée, et qu'ils ne se réveillent que pour ronger toute la nuit. Tenez, l'an dernier, j'avais quatre abricots; ils m'en ont entamé un. J'avais un brugnon, un seul, il est vrai que c'est un fruit rare; eh bien, monsieur, ils me l'ont à moitié dévoré du côté de la muraille; un brugnon superbe et qui était excellent. Je n'en ai jamais mangé de meilleur. 

—Vous l'avez mangé? demanda Monte-Cristo. 

—C'est-à-dire la moitié qui restait, vous comprenez bien. C'était exquis, monsieur. Ah! dame, ces messieurs-là ne choisissent pas les pires morceaux. C'est comme le fils de la mère Simon, il n'a pas choisi les plus mauvaises fraises, allez! Mais, cette année, continua l'horticulteur, soyez tranquille, cela ne m'arrivera pas, dussé-je, quand les fruits seront près de mûrir, passer la nuit pour les garder.» 

Monte-Cristo en avait assez vu. Chaque homme a sa passion qui le mord au fond du cœur, comme chaque fruit son ver, celle de l'homme au télégraphe, c'était l'horticulture. Il se mit à cueillir les feuilles de vigne qui cachaient les grappes au soleil, et se conquit par là le cœur du jardinier. 

«Monsieur était venu pour voir le télégraphe? dit-il. 

—Oui, monsieur, si toutefois cela n'est pas défendu par les règlements. 

—Oh! pas défendu le moins du monde, dit le jardinier, attendu qu'il n'y a rien de dangereux, vu que personne ne sait ni ne peut savoir ce que nous disons. 

—On m'a dit, en effet, reprit le comte, que vous répétiez des signaux que vous ne compreniez pas vous-même. 

—Certainement, monsieur, et j'aime bien mieux cela, dit en riant l'homme du télégraphe. 

—Pourquoi aimez-vous mieux cela? 

—Parce que, de cette façon, je n'ai pas de responsabilité. Je suis une machine, moi, et pas autre chose, et pourvu que je fonctionne, on ne m'en demande pas davantage.»  

«Diable! fit Monte-Cristo en lui-même, est-ce que par hasard je serais tombé sur un homme qui n'aurait pas d'ambition! Morbleu! Ce serait jouer de malheur.» 

«Monsieur, dit le jardinier en jetant un coup d'œil sur son cadran solaire, les dix minutes vont expirer, je retourne à mon poste. Vous plaît-il de monter avec moi? 

—Je vous suis.» 

Monte-Cristo entra, en effet, dans la cour divisée en trois étages; celui du bas contenait quelques instruments aratoires, tels que bêches, râteaux, arrosoirs, dressés contre la muraille: c'était tout l'ameublement. 

Le second était l'habitation ordinaire ou plutôt nocturne de l'employé; il contenait quelques pauvres ustensiles de ménage, un lit, une table, deux chaises, une fontaine de grès, plus quelques herbes sèches pendues au plafond, et que le comte reconnut pour des pois de senteur et des haricots d'Espagne dont le bonhomme conservait la graine dans sa coque; il avait étiqueté tout cela avec le soin d'un maître botaniste du Jardin des plantes. 

«Faut-il passer beaucoup de temps à étudier la télégraphie, monsieur? demanda Monte-Cristo. 

—Ce n'est pas l'étude qui est longue, c'est le surnumérariat. 

—Et combien reçoit-on d'appointements? 

—Mille francs, monsieur.  

—Ce n'est guère. 

—Non; mais on est logé, comme vous voyez.» 

Monte-Cristo regarda la chambre. 

«Pourvu qu'il n'aille pas tenir à son logement», murmura-t-il. 

On passa au troisième étage: c'était la chambre du télégraphe. Monte-Cristo regarda tour à tour les deux poignées de fer à l'aide desquelles l'employé faisait jouer la machine. 

«C'est fort intéressant, dit-il, mais à la longue c'est une vie qui doit vous paraître un peu insipide? 

—Oui, dans le commencement cela donne le torticolis à force de regarder; mais au bout d'un an ou deux on s'y fait; puis nous avons nos heures de récréation et nos jours de congé. 

—Vos jours de congé? 

—Oui. 

—Lesquels? 

—Ceux où il fait du brouillard. 

—Ah! c'est juste. 

—Ce sont mes jours de fête, à moi; je descends dans le jardin ces jours-là, et je plante, je taille, je rogne, j'échenille: en somme, le temps passe. 

—Depuis combien de temps êtes-vous ici? 

—Depuis dix ans et cinq ans de surnumérariat, quinze. 

—Vous avez?\dots 

—Cinquante-cinq ans. 

—Combien de temps de service vous faut-il pour avoir la pension? 

—Oh! monsieur, vingt-cinq ans. 

—Et de combien est cette pension? 

—De cent écus. 

—Pauvre humanité! murmura Monte-Cristo. 

—Vous dites, monsieur?\dots demanda l'employé. 

—Je dis que c'est fort intéressant. 

—Quoi? 

—Tout ce que vous me montrez\dots. Et vous ne comprenez rien absolument à vos signes? 

—Rien absolument. 

—Vous n'avez jamais essayé de comprendre? 

—Jamais; pour quoi faire? 

—Cependant, il y a des signaux qui s'adressent à vous directement. 

—Sans doute. 

—Et ceux-là vous les comprenez? 

—Ce sont toujours les mêmes. 

—Et ils disent? 

—\textit{Rien de nouveau\dots vous avez une heure\dots ou à demain\dots} 

—Voilà qui est parfaitement innocent, dit le comte; mais regardez donc, ne voilà-t-il pas votre correspondant qui se met en mouvement. 

—Ah! c'est vrai; merci, monsieur. 

—Et que vous dit-il? est-ce quelque chose que vous comprenez? 

—Oui; il me demande si je suis prêt. 

—Et vous lui répondez?\dots 

—Par un signe qui apprend en même temps à mon correspondant de droite que je suis prêt, tandis qu'il invite mon correspondant de gauche à se préparer à son tour.  

—C'est très ingénieux, dit le comte. 

—Vous allez voir, reprit avec orgueil le bonhomme, dans cinq minutes il va parler. 

—J'ai cinq minutes alors, dit Monte-Cristo, c'est plus de temps qu'il ne m'en faut. Mon cher monsieur, dit-il, permettez-moi de vous faire une question. 

—Faites. 

—Vous aimez le jardinage? 

—Avec passion. 

—Et vous seriez heureux, au lieu d'avoir une terrasse de vingt pieds, d'avoir un enclos de deux arpents? 

—Monsieur, j'en ferais un paradis terrestre. 

—Avec vos mille francs, vous vivez mal? 

—Assez mal; mais enfin je vis. 

—Oui; mais vous n'avez qu'un jardin misérable. 

—Ah! c'est vrai, le jardin n'est pas grand. 

—Et encore, tel qu'il est, il est peuplé de loirs qui dévorent tout. 

—Ça, c'est mon fléau. 

—Dites-moi, si vous aviez le malheur de tourner la tête quand le correspondant de droite va marcher? 

—Je ne le verrais pas. 

—Alors qu'arriverait-il? 

—Que je ne pourrais pas répéter ses signaux. 

—Et après? 

—Il arriverait que, ne les ayant pas répétés par négligence, je serais mis à l'amende. 

—De combien? 

—De cent francs. 

—Le dixième de votre revenu, c'est joli! 

—Ah! fit l'employé. 

—Cela vous est arrivé? dit Monte-Cristo. 

—Une fois, monsieur, une fois que je greffais un rosier noisette. 

—Bien. Maintenant, si vous vous avisiez de changer quelque chose au signal, ou d'en transmettre un autre? 

—Alors, c'est différent, je serais renvoyé et je perdrais ma pension. 

—Trois cents francs? 

—Cent écus, oui, monsieur; aussi vous comprenez que jamais je ne ferai rien de tout cela. 

—Pas même pour quinze ans de vos appointements? Voyons, ceci mérite réflexion, hein? 

—Pour quinze mille francs? 

—Oui. 

—Monsieur, vous m'effrayez. 

—Bah! 

—Monsieur, vous voulez me tenter? 

—Justement! Quinze mille francs, comprenez? 

—Monsieur, laissez-moi regarder mon correspondant à droite! 

—Au contraire, ne le regardez pas et regardez ceci. 

—Qu'est-ce que c'est? 

—Comment? vous ne connaissez pas ces petits papiers-là? 

—Des billets de banque! 

—Carrés; il y en a quinze. 

—Et à qui sont-ils? 

—À vous, si vous voulez. 

—À moi! s'écria l'employé suffoqué. 

—Oh! mon Dieu, oui! à vous, en toute propriété. 

—Monsieur, voilà mon correspondant de droite qui marche. 

—Laissez-le marcher. 

—Monsieur, vous m'avez distrait, et je vais être à l'amende. 

—Cela vous coûtera cent francs; vous voyez bien que vous avez tout intérêt à prendre mes quinze billets de banque. 

—Monsieur, le correspondant de droite s'impatiente, il redouble ses signaux. 

—Laissez-le faire et prenez.» 

Le comte mit le paquet dans la main de l'employé. 

«Maintenant, dit-il, ce n'est pas tout: avec vos quinze mille francs vous ne vivrez pas. 

—J'aurai toujours ma place.  

—Non, vous la perdrez; car vous allez faire un autre signe que celui de votre correspondant. 

—Oh! monsieur, que me proposez-vous là? 

—Un enfantillage. 

—Monsieur, à moins que d'y être forcé\dots. 

—Je compte bien vous y forcer effectivement.» 

Et Monte-Cristo tira de sa poche un autre paquet. 

«Voici dix autres mille francs, dit-il; avec les quinze qui sont dans votre poche, cela fera vingt-cinq mille. Avec cinq mille francs, vous achèterez une jolie petite maison et deux arpents de terre; avec les vingt mille autres, vous vous ferez mille francs de rente. 

—Un jardin de deux arpents? 

—Et mille francs de rente. 

—Mon Dieu! mon Dieu! 

—Mais prenez donc!» 

Et Monte-Cristo mit de force les dix mille francs dans la main de l'employé. 

«Que dois-je faire?  

—Rien de bien difficile. 

—Mais enfin? 

—Répéter les signes que voici.» 

Monte-Cristo tira de sa poche un papier sur lequel il y avait trois signes tout tracés, des numéros indiquant l'ordre dans lequel ils devaient être faits. 

«Ce ne sera pas long, comme vous voyez. 

—Oui, mais\dots. 

—C'est pour le coup que vous aurez des brugnons, et de reste.» 

Le coup porta; rouge de fièvre et suant à grosses gouttes, le bonhomme exécuta les uns après les autres les trois signes donnés par le comte, malgré les effrayantes dislocations du correspondant de droite, qui, ne comprenant rien à ce changement, commençait à croire que l'homme aux brugnons était devenu fou. 

Quant au correspondant de gauche, il répéta consciencieusement les mêmes signaux qui furent recueillis définitivement au ministère de l'Intérieur. 

«Maintenant, vous voilà riche, dit Monte-Cristo. 

—Oui, répondit l'employé, mais à quel prix! 

—Écoutez, mon ami, dit Monte-Cristo, je ne veux pas que vous ayez des remords; croyez-moi donc, car, je vous jure, vous n'avez fait de tort à personne, et vous avez servi les projets de Dieu.» 

L'employé regardait les billets de banque, les palpait, les comptait; il était pâle, il était rouge; enfin, il se précipita vers sa chambre pour boire un verre d'eau; mais il n'eut pas le temps d'arriver jusqu'à la fontaine, et il s'évanouit au milieu de ses haricots secs. 

Cinq minutes après que la nouvelle télégraphique fut arrivée au ministère, Debray fit mettre les chevaux à son coupé, et courut chez Danglars. 

«Votre mari a des coupons de l'emprunt espagnol? dit-il à la baronne. 

—Je crois bien! il en a pour six millions. 

—Qu'il les vende à quelque prix que ce soit. 

—Pourquoi cela? 

—Parce que don Carlos s'est sauvé de Bourges et est rentré en Espagne. 

—Comment savez-vous cela? 

—Parbleu, dit Debray en haussant les épaules, comme je sais les nouvelles.» 

La baronne ne se le fit pas répéter deux fois: elle courut chez son mari, lequel courut à son tour chez son agent de change et lui ordonna de vendre à tout prix. 

Quand on vit que M. Danglars vendait, les fonds espagnols baissèrent aussitôt. Danglars y perdit cinq cent mille francs, mais il se débarrassa de tous ses coupons. 

Le soir on lut dans le \textit{Messager}: 

\begin{quotation}
	
\begin{flushright}\itshape
	Dépêche télégraphique.
\end{flushright}

Le roi don Carlos a échappé à la surveillance qu'on exerçait sur lui à Bourges, et est rentré en Espagne par la frontière de Catalogne. Barcelone s'est soulevée en sa faveur.
\end{quotation}

Pendant toute la soirée il ne fut bruit que de la prévoyance de Danglars, qui avait vendu ses coupons, et du bonheur de l'agioteur, qui ne perdait que cinq cent mille francs sur un pareil coup. 

Ceux qui avaient conservé leurs coupons ou acheté ceux de Danglars se regardèrent comme ruinés et passèrent une fort mauvaise nuit. 

Le lendemain on lut dans le \textit{Moniteur}: 

\begin{newspaper}{}{}
C'est sans aucun fondement que le \textit{Messager} a annoncé hier la fuite de don Carlos et la révolte de Barcelone. 

Le roi don Carlos n'a pas quitté Bourges, et la Péninsule jouit de la plus profonde tranquillité.  

Un signe télégraphique, mal interprété à cause du brouillard, a donné lieu à cette erreur.
\end{newspaper}

Les fonds remontèrent d'un chiffre double de celui où ils étaient descendus. 

Cela fit, en perte et en manque à gagner, un million de différence pour Danglars. 

«Bon! dit Monte-Cristo à Morrel, qui se trouvait chez lui au moment où on annonçait l'étrange revirement de Bourse dont Danglars avait été victime; je viens de faire pour vingt-cinq mille francs une découverte que j'eusse payée cent mille. 

—Que venez-vous donc de découvrir? demanda Maximilien. 

—Je viens de découvrir le moyen de délivrer un jardinier des loirs qui lui mangeaient ses pêches.» 