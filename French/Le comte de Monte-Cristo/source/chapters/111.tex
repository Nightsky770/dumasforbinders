\chapter{Expiation} 

\lettrine{M}{.} de Villefort avait vu s'ouvrir devant lui les rangs de la foule, si compacte qu'elle fût. Les grandes douleurs sont tellement vénérables, qu'il n'est pas d'exemple, même dans les temps les plus malheureux, que le premier mouvement de la foule réunie n'ait pas été un mouvement de sympathie pour une grande catastrophe. Beaucoup de gens haïs ont été assassinés dans une émeute; rarement un malheureux, fût-il criminel, a été insulté par les hommes qui assistaient à sa condamnation à mort. 

Villefort traversa donc la haie des spectateurs, des gardes, des gens du Palais, et s'éloigna, reconnu coupable de son propre aveu, mais protégé par sa douleur. 

Il est des situations que les hommes saisissent avec leur instinct, mais qu'ils ne peuvent commenter avec leur esprit; le plus grand poète, dans ce cas, est celui qui pousse le cri le plus véhément et le plus naturel. La foule prend ce cri pour un récit tout entier, et elle a raison de s'en contenter, et plus raison encore de le trouver sublime quand il est vrai. 

Du reste il serait difficile de dire l'état de stupeur dans lequel était Villefort en sortant du Palais, de peindre cette fièvre qui faisait battre chaque artère, raidissait chaque fibre, gonflait à la briser chaque veine, et disséquait chaque point du corps mortel en des millions de souffrances. 

Villefort se traîna le long des corridors, guidé seulement par l'habitude; il jeta de ses épaules la toge magistrale, non qu'il pensât à la quitter pour la convenance, mais parce qu'elle était à ses épaules un fardeau accablant, une tunique de Nessus féconde en tortures. 

Il arriva chancelant jusqu'à la cour Dauphine, aperçut sa voiture, réveilla le cocher en ouvrant la portière lui-même, et se laissa tomber sur les coussins en montrant du doigt la direction du faubourg Saint-Honoré. Le cocher partit. 

Tout le poids de sa fortune écroulée venait de retomber sur sa tête; ce poids l'écrasait, il n'en savait pas les conséquences; il ne les avait pas mesurées; il les sentait, il ne raisonnait pas son code comme le froid meurtrier qui commente un article connu. 

Il avait Dieu au fond du cœur. 

«Dieu! murmurait-il sans savoir même ce qu'il disait, Dieu! Dieu!» 

Il ne voyait que Dieu derrière l'éboulement qui venait de se faire. 

La voiture roulait avec vitesse; Villefort, en s'agitant sur ses coussins, sentit quelque chose qui le gênait. 

Il porta la main à cet objet: c'était un éventail oublié par Mme de Villefort entre le coussin et le dossier de la voiture; cet éventail éveilla un souvenir, et ce souvenir fut un éclair au milieu de la nuit. 

Villefort songea à sa femme\dots 

«Oh!» s'écria-t-il, comme si un fer rouge lui traversait le cœur. 

En effet, depuis une heure, il n'avait plus sous les yeux qu'une face de sa misère, et voilà que tout à coup il s'en offrait une autre à son esprit, et une autre non moins terrible. 

Cette femme, il venait de faire avec elle le juge inexorable, il venait de la condamner à mort; et elle, elle, frappée de terreur, écrasée par le remords, abîmée sous la honte qu'il venait de lui faire avec l'éloquence de son irréprochable vertu, elle, pauvre femme faible et sans défense contre un pouvoir absolu et suprême, elle se préparait peut-être en ce moment même à mourir! 

Une heure s'était déjà écoulée depuis sa condamnation; sans doute en ce moment elle repassait tous ses crimes dans sa mémoire, elle demandait grâce à Dieu, elle écrivait une lettre pour implorer à genoux le pardon de son vertueux époux, pardon qu'elle achetait de sa mort. 

Villefort poussa un second rugissement de douleur et de rage. 

«Ah! s'écria-t-il en se roulant sur le satin de son carrosse, cette femme n'est devenue criminelle que parce qu'elle m'a touché. Je sue le crime, moi! et elle a gagné le crime comme on gagne le typhus, comme on gagne le choléra, comme on gagne la peste!\dots et je la punis!\dots J'ai osé lui dire: Repentez-vous et mourez\dots moi! oh! non! non! elle vivra\dots elle me suivra\dots Nous allons fuir, quitter la France, aller devant nous tant que la terre pourra nous porter. Je lui parlais d'échafaud!\dots Grand Dieu! comment ai-je osé prononcer ce mot! Mais, moi aussi, l'échafaud m'attend!\dots Nous fuirons\dots Oui, je me confesserai à elle! oui, tous les jours je lui dirai, en m'humiliant, que, moi aussi, j'ai commis un crime\dots Oh! alliance du tigre et du serpent! oh! digne femme d'un mari tel que moi!\dots Il faut qu'elle vive, il faut que mon infamie fasse pâlir la sienne!» 

Et Villefort enfonça plutôt qu'il ne baissa la glace du devant de son coupé. 

«Vite, plus vite!» s'écria-t-il d'une voix qui fit bondir le cocher sur son siège. 

Les chevaux, emportés par la peur, volèrent jusqu'à la maison. 

«Oui, oui, se répétait Villefort à mesure qu'il se rapprochait de chez lui, oui, il faut que cette femme vive, il faut qu'elle se repente et qu'elle élève mon fils, mon pauvre enfant, le seul, avec l'indestructible vieillard, qui ait survécu à la destruction de la famille! Elle l'aimait; c'est pour lui qu'elle a tout fait. Il ne faut jamais désespérer du cœur d'une mère qui aime son enfant; elle se repentira; nul ne saura qu'elle fut coupable; ces crimes commis chez moi, et dont le monde s'inquiète déjà, ils seront oubliés avec le temps, ou, si quelques ennemis s'en souviennent, eh bien, je les prendrai sur ma liste de crimes. Un, deux, trois de plus, qu'importe! ma femme se sauvera emportant de l'or, et surtout emportant son fils, loin du gouffre où il me semble que le monde va tomber avec moi. Elle vivra, elle sera heureuse encore, puisque tout son amour est dans son fils, et que son fils ne la quittera point. J'aurai fait une bonne action; cela allège le cœur.» 

Et le procureur du roi respira plus librement qu'il n'avait fait depuis longtemps. 

La voiture s'arrêta dans la cour de l'hôtel. 

Villefort s'élança du marchepied sur le perron; il vit les domestiques surpris de le voir revenir si vite. Il ne lut pas autre chose sur leur physionomie; nul ne lui adressa la parole; on s'arrêta devant lui, comme d'habitude, pour le laisser passer; voilà tout. 

Il passa devant la chambre de Noirtier, et, par la porte il ne s'inquiéta point de la personne qui était avec son père; c'était ailleurs que son inquiétude le tirait. 

«Allons, dit-il en montant le petit escalier qui conduisait au palier où étaient l'appartement de sa femme et la chambre vide de Valentine; allons, rien n'est changé ici.» 

Avant tout il ferma la porte du palier. 

«Il faut que personne ne nous dérange, dit-il; il faut que je puisse lui parler librement, m'accuser devant elle, lui tout dire\dots» 

Il s'approcha de la porte, mit la main sur le bouton de cristal, la porte céda. 

«Pas fermée! oh! bien, très bien», murmura-t-il. 

Et il entra dans le petit salon où dans la soirée on dressait un lit pour Édouard; car, quoique en pension, Édouard rentrait tous les soirs: sa mère n'avait jamais voulu se séparer de lui. 

Il embrassa d'un coup d'œil tout le petit salon. 

«Personne, dit-il; elle est dans sa chambre à coucher sans doute.» 

Il s'élança vers la porte. Là, le verrou était mis. Il s'arrêta frissonnant. 

«Héloïse!» cria-t-il. 

Il lui sembla entendre remuer un meuble. 

«Héloïse! répéta-t-il. 

—Qui est là?» demanda la voix de celle qu'il appelait. 

Il lui sembla que cette voix était plus faible que de coutume. 

«Ouvrez! ouvrez! s'écria Villefort, c'est moi!» 

Mais malgré cet ordre, malgré le ton d'angoisse avec lequel il était donné, on n'ouvrit pas. 

Villefort enfonça la porte d'un coup de pied. 

À l'entrée de la chambre qui donnait dans son boudoir, Mme de Villefort était debout, pâle, les traits contractés, et le regardant avec des yeux d'une fixité effrayante. 

«Héloïse! Héloïse! dit-il, qu'avez-vous? Parlez!» 

La jeune femme étendit vers lui sa main raide et livide. 

«C'est fait, monsieur, dit-elle avec un râlement qui sembla déchirer son gosier; que voulez-vous donc encore de plus?» 

Et elle tomba de sa hauteur sur le tapis. 

Villefort courut à elle, lui saisit la main. Cette main serrait convulsivement un flacon de cristal à bouchon d'or. 

Mme de Villefort était morte. 

Villefort, ivre d'horreur, recula jusqu'au seuil de la chambre et regarda le cadavre. 

«Mon fils! s'écria-t-il tout à coup; où est mon fils? Édouard! Édouard!» 

Et il se précipita hors de l'appartement en criant: 

«Édouard! Édouard!» 

Ce nom était prononcé avec un tel accent d'angoisse, que les domestiques accoururent. 

«Mon fils! où est mon fils? demanda Villefort. Qu'on l'éloigne de la maison, qu'il ne voie pas\dots 

—M. Édouard n'est point en bas, monsieur, répondit le valet de chambre. 

—Il joue sans doute au jardin; voyez! voyez! 

—Non, monsieur. Madame a appelé son fils il y a une demi-heure à peu près; M. Édouard est entré chez madame et n'est point descendu depuis.» 

Une sueur glacée inonda le front de Villefort, ses pieds trébuchèrent sur la dalle, ses idées commencèrent à tourner dans sa tête comme les rouages désordonnés d'une montre qui se brise. 

«Chez madame! murmura-t-il, chez madame!» 

Et il revint lentement sur ses pas, s'essuyant le front d'une main, s'appuyant de l'autre aux parois de la muraille. 

En rentrant dans la chambre il fallait revoir le corps de la malheureuse femme. 

Pour appeler Édouard, il fallait réveiller l'écho de cet appartement changé en cercueil; parler, c'était violer le silence de la tombe. 

Villefort sentit sa langue paralysée dans sa gorge. 

«Édouard, Édouard», balbutia-t-il. 

L'enfant ne répondait pas; où donc était l'enfant qui, au dire des domestiques, était entré chez sa mère et n'en était pas sorti? 

Villefort fit un pas en avant. 

Le cadavre de Mme de Villefort était couché en travers de la porte du boudoir dans lequel se trouvait nécessairement Édouard; ce cadavre semblait veiller sur le seuil avec des yeux fixes et ouverts, avec une épouvantable et mystérieuse ironie sur les lèvres. 

Derrière le cadavre, la portière relevée laissait voir une partie du boudoir, un piano et le bout d'un divan de satin bleu. 

Villefort fit trois ou quatre pas en avant, et sur le canapé il aperçut son enfant couché. 

L'enfant dormait sans doute. 

Le malheureux eut un élan de joie indicible; un rayon de pure lumière descendit dans cet enfer où il se débattait. 

Il ne s'agissait donc que de passer par-dessus le cadavre, d'entrer dans le boudoir, de prendre l'enfant dans ses bras et de fuir avec lui, loin, bien loin. 

Villefort n'était plus cet homme dont son exquise corruption faisait le type de l'homme civilisé; c'était un tigre blessé à mort qui laisse ses dents brisées dans sa dernière blessure. 

Il n'avait plus peur des préjugés, mais des fantômes. Il prit son élan et bondit par-dessus le cadavre, comme s'il se fût agi de franchir un brasier dévorant. 

Il enleva l'enfant dans ses bras, le serrant, le secouant, l'appelant; l'enfant ne répondait point. Il colla ses lèvres avides à ses joues, ses joues étaient livides et glacées; il palpa ses membres raidis; il appuya sa main sur son cœur, son cœur ne battait plus. 

L'enfant était mort. 

Un papier plié en quatre tomba de la poitrine d'Édouard. 

Villefort, foudroyé, se laissa aller sur ses genoux; l'enfant s'échappa de ses bras inertes et roula du côté de sa mère. 

Villefort ramassa le papier, reconnut l'écriture de sa femme et le parcourut avidement. 

Voici ce qu'il contenait: 

«Vous savez si j'étais bonne mère, puisque c'est pour mon fils que je me suis faite criminelle! 

«Une bonne mère ne part pas sans son fils!» 

Villefort ne pouvait en croire ses yeux; Villefort ne pouvait en croire sa raison. Il se traîna vers le corps d'Édouard, qu'il examina encore une fois avec cette attention minutieuse que met la lionne à regarder son lionceau mort. 

Puis un cri déchirant s'échappa de sa poitrine. 

«Dieu! murmura-t-il, toujours Dieu!» 

Ces deux victimes l'épouvantaient, il sentait monter en lui l'horreur de cette solitude peuplée de deux cadavres. 

Tout à l'heure il était soutenu par la rage, cette immense faculté des hommes forts, par le désespoir, cette vertu suprême de l'agonie, qui poussait les Titans à escalader le ciel, Ajax à montrer le poing aux dieux. 

Villefort courba sa tête sous le poids des douleurs, il se releva sur ses genoux, secoua ses cheveux humides de sueur, hérissés d'effroi et celui-là, qui n'avait jamais eu pitié de personne s'en alla trouver le vieillard, son père, pour avoir, dans sa faiblesse, quelqu'un à qui raconter son malheur, quelqu'un près de qui pleurer. 

Il descendit l'escalier que nous connaissons et entra chez Noirtier. 

Quand Villefort entra, Noirtier paraissait attentif à écouter aussi affectueusement que le permettait son immobilité, l'abbé Busoni, toujours aussi calme et aussi froid que de coutume. 

Villefort, en apercevant l'abbé, porta la main à son front. Le passé lui revint comme une de ces vagues dont la colère soulève plus d'écume que les autres vagues. 

Il se souvint de la visite qu'il avait faite à l'abbé le surlendemain du dîner d'Auteuil et de la visite que lui avait faite l'abbé à lui-même le jour de la mort de Valentine. 

«Vous ici, monsieur! dit-il; mais vous n'apparaissez donc jamais que pour escorter la Mort?» 

Busoni se redressa; en voyant l'altération du visage du magistrat, l'éclat farouche de ses yeux, il comprit ou crut comprendre que la scène des assises était accomplie; il ignorait le reste. 

«J'y suis venu pour prier sur le corps de votre fille! répondit Busoni. 

—Et aujourd'hui, qu'y venez-vous faire? 

—Je viens vous dire que vous m'avez assez payé votre dette, et qu'à partir de ce moment je vais prier Dieu qu'il se contente comme moi. 

—Mon Dieu! fit Villefort en reculant, l'épouvante sur le front, cette voix, ce n'est pas celle de l'abbé Busoni! 

—Non.» 

L'abbé arracha sa fausse tonsure, secoua la tête, et ses longs cheveux noirs, cessant d'être comprimés, retombèrent sur ses épaules et encadrèrent son mâle visage. 

«C'est le visage de M. de Monte-Cristo! s'écria Villefort les yeux hagards. 

—Ce n'est pas encore cela, monsieur le procureur du roi, cherchez mieux et plus loin. 

—Cette voix! cette voix! où l'ai-je entendue pour la première fois? 

—Vous l'avez entendue pour la première fois à Marseille, il y a vingt-trois ans, le jour de votre mariage avec Mlle de Saint-Méran. Cherchez dans vos dossiers. 

—Vous n'êtes pas Busoni? vous n'êtes pas Monte-Cristo? Mon Dieu vous êtes cet ennemi caché, implacable, mortel! J'ai fait quelque chose contre vous à Marseille, oh! malheur à moi! 

—Oui, tu as raison, c'est bien cela, dit le comte en croisant les bras sur sa large poitrine; cherche, cherche! 

—Mais que t'ai-je donc fait? s'écria Villefort, dont l'esprit flottait déjà sur la limite où se confondent la raison et la démence, dans ce brouillard qui n'est plus le rêve et qui n'est pas encore le réveil; que t'ai-je fait? dis! parle! 

—Vous m'avez condamné à une mort lente et hideuse, vous avez tué mon père, vous m'avez ôté l'amour avec la liberté, et la fortune avec l'amour! 

—Qui êtes-vous? qui êtes-vous donc? mon Dieu! 

—Je suis le spectre d'un malheureux que vous avez enseveli dans les cachots du château d'If. À ce spectre sorti enfin de sa tombe Dieu a mis le masque du comte de Monte-Cristo, et il l'a couvert de diamants et d'or pour que vous ne le reconnaissiez qu'aujourd'hui. 

—Ah! je te reconnais, je te reconnais! dit le procureur du roi; tu es\dots 

—Je suis Edmond Dantès! 

—Tu es Edmond Dantès! s'écria le procureur du roi en saisissant le comte par le poignet; alors, viens!» 

Et il l'entraîna par l'escalier, dans lequel Monte-Cristo, étonné, le suivit, ignorant lui-même où le procureur du roi le conduisait, et pressentant quelque nouvelle catastrophe. 

«Tiens! Edmond Dantès, dit-il en montrant au comte le cadavre de sa femme et le corps de son fils, tiens! regarde, es-tu bien vengé?\dots» 

Monte-Cristo pâlit à cet effroyable spectacle; il comprit qu'il venait d'outrepasser les droits de la vengeance; il comprit qu'il ne pouvait plus dire: 

«Dieu est pour moi et avec moi.» 

Il se jeta avec un sentiment d'angoisse inexprimable sur le corps de l'enfant, rouvrit ses yeux, tâta le pouls, et s'élança avec lui dans la chambre de Valentine, qu'il referma à double tour\dots 

«Mon enfant! s'écria Villefort; il emporte le cadavre de mon enfant! Oh! malédiction! malheur! mort sur toi!» 

Et il voulut s'élancer après Monte-Cristo; mais, comme dans un rêve, il sentit ses pieds prendre racine, ses yeux se dilatèrent à briser leurs orbites, ses doigts recourbés sur la chair de sa poitrine s'y enfoncèrent graduellement jusqu'à ce que le sang rougît ses ongles; les veines de ses tempes se gonflèrent d'esprits bouillants qui allèrent soulever la voûte trop étroite de son crâne et noyèrent son cerveau dans un déluge de feu. 

Cette fixité dura plusieurs minutes, jusqu'à ce que l'effroyable bouleversement de la raison fût accompli. 

Alors il jeta un grand cri suivi d'un long éclat de rire et se précipita par les escaliers. 

Un quart d'heure après, la chambre de Valentine se rouvrit, et le comte de Monte-Cristo reparut. 

Pâle, l'œil morne, la poitrine oppressée, tous les traits de cette figure ordinairement si calme et si noble étaient bouleversés par la douleur. 

Il tenait dans ses bras l'enfant, auquel aucun secours n'avait pu rendre la vie. 

Il mit un genou en terre et le déposa religieusement près de sa mère, la tête posée sur sa poitrine. 

Puis, se relevant, il sortit, et rencontrant un domestique sur l'escalier: 

«Où est M. de Villefort?» demanda-t-il. 

Le domestique, sans lui répondre, étendit la main du côté du jardin. 

Monte-Cristo descendit le perron, s'avança vers l'endroit désigné, et vit, au milieu de ses serviteurs faisant cercle autour de lui, Villefort une bêche à la main, et fouillant la terre avec une espèce de rage. 

«Ce n'est pas encore ici, disait-il, ce n'est pas encore ici. 

Et il fouillait plus loin. 

Monte-Cristo s'approcha de lui, et tout bas: 

«Monsieur, lui dit-il d'un ton presque humble, vous avez perdu un fils, mais\dots» 

Villefort l'interrompit; il n'avait ni écouté ni entendu. 

«Oh! je le retrouverai, dit-il; vous avez beau prétendre qu'il n'y est pas, je le retrouverai, dussé-je le chercher jusqu'au jour du Jugement dernier. 

Monte-Cristo recula avec terreur. 

«Oh! dit-il, il est fou!» 

Et, comme s'il eût craint que les murs de la maison maudite ne s'écroulassent sur lui, il s'élança dans la rue, doutant pour la première fois qu'il eût le droit de faire ce qu'il avait fait. 

«Oh! assez, assez comme cela, dit-il, sauvons le dernier.» 

En rentrant chez lui, Monte-Cristo rencontra Morrel, qui errait dans l'hôtel des Champs-Élysées, silencieux comme une ombre qui attend le moment fixé par Dieu pour rentrer dans son tombeau. 

«Apprêtez-vous, Maximilien, lui dit-il avec un sourire, nous quittons Paris demain. 

—N'avez-vous plus rien à y faire? demanda Morrel. 

—Non, répondit Monte-Cristo, et Dieu veuille que je n'y aie pas trop fait!» 