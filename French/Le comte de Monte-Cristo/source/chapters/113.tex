\chapter{Le passé}

\lettrine{L}{e} comte sortit l'âme navrée de cette maison où il laissait Mercédès pour ne plus la revoir jamais, selon toute probabilité. 

\zz
Depuis la mort du petit Édouard, un grand changement s'était fait dans Monte-Cristo. Arrivé au sommet de sa vengeance par la pente lente et tortueuse qu'il avait suivie, il avait vu de l'autre côté de la montagne l'abîme du doute. 

Il y avait plus: cette conversation qu'il venait d'avoir avec Mercédès avait éveillé tant de souvenirs dans son cœur, que ces souvenirs eux-mêmes avaient besoin d'être combattus. 

Un homme de la trempe du comte ne pouvait flotter longtemps dans cette mélancolie qui peut faire vivre les esprits vulgaires en leur donnant une originalité apparente, mais qui tue les âmes supérieures. Le comte se dit que pour en être presque arrivé à se blâmer lui-même, il fallait qu'une erreur se fût glissée dans ses calculs. 

«Je regarde mal le passé, dit-il, et ne puis m'être trompé ainsi. 

«Quoi! continua-t-il, le but que je m'étais proposé serait un but insensé! Quoi! j'aurais fait fausse route depuis dix ans! Quoi! une heure aurait suffi pour prouver à l'architecte que l'œuvre de toutes ses espérances était une œuvre, sinon impossible, du moins sacrilège! 

«Je ne veux pas m'habituer à cette idée, elle me rendrait fou. Ce qui manque à mes raisonnements d'aujourd'hui, c'est l'appréciation exacte du passé parce que je revois ce passé de l'autre bout de l'horizon. En effet, à mesure qu'on s'avance, le passé, pareil au paysage à travers lequel on marche, s'efface à mesure qu'on s'éloigne. Il m'arrive ce qui arrive aux gens qui se sont blessés en rêve, ils regardent et sentent leur blessure, et ne se souviennent pas de l'avoir reçue. 

«Allons donc, homme régénéré; allons, riche extravagant; allons, dormeur éveillé; allons, visionnaire tout-puissant; allons, millionnaire invincible, reprends pour un instant cette funeste perspective de la vie misérable et affamée; repasse par les chemins où la fatalité t'a poussé, où le malheur t'a conduit, où le désespoir t'a reçu; trop de diamants, d'or et de bonheur rayonnent aujourd'hui sur les verres de ce miroir où Monte-Cristo regarde Dantès, cache ces diamants, souille cet or, efface ces rayons; riche, retrouve le pauvre; libre, retrouve le prisonnier, ressuscité, retrouve le cadavre.» 

Et tout en disant cela à lui-même, Monte-Cristo suivait la rue de la Caisserie. C'était la même par laquelle, vingt-quatre ans auparavant, il avait été conduit par une garde silencieuse et nocturne; ces maisons, à l'aspect riant et animé, elles étaient cette nuit-là sombres, muettes et fermées. 

«Ce sont cependant les mêmes, murmura Monte-Cristo, seulement alors il faisait nuit, aujourd'hui il fait grand jour; c'est le soleil qui éclaire tout cela et qui rend tout cela joyeux.» 

Il descendit sur le quai par la rue Saint-Laurent, et s'avança vers la Consigne: c'était le point du port où il avait été embarqué. Un bateau de promenade passait avec son dais de coutil; Monte-Cristo appela le patron, qui nagea aussitôt vers lui avec l'empressement que mettent à cet exercice les bateliers qui flairent une bonne aubaine. 

Le temps était magnifique, le voyage fut une fête. À l'horizon le soleil descendait, rouge et flamboyant, dans les flots qui s'embrasaient à son approche; la mer, unie comme un miroir, se ridait parfois sous les bonds des poissons qui, poursuivis par quelque ennemi caché, s'élançaient hors de l'eau pour demander leur salut à un autre élément; enfin, à l'horizon l'on voyait passer, blanches et gracieuses comme des mouettes voyageuses, les barques de pécheurs qui se rendent aux Martigues, ou les bâtiments marchands chargés pour la Corse ou pour l'Espagne. 

Malgré ce beau ciel, malgré ces barques aux gracieux contours, malgré cette lumière dorée qui inondait le paysage, le comte, enveloppé dans son manteau, se rappelait, un à un, tous les détails du terrible voyage: cette lumière unique et isolée, brûlant aux Catalans, cette vue du château d'If qui lui apprit où on le menait, cette lutte avec les gendarmes lorsqu'il voulut se précipiter dans la mer, son désespoir quand il se sentit vaincu, et cette sensation froide du bout du canon de la carabine appuyée sur sa tempe comme un anneau de glace. 

Et peu à peu, comme ces sources desséchées par l'été, qui lorsque s'amassent les nuages d'automne s'humectent peu à peu et commencent à sourdre goutte à goutte, le comte de Monte-Cristo sentit également sourdre dans sa poitrine ce vieux fiel extravasé qui avait autrefois inondé le cœur d'Edmond Dantès. 

Pour lui dès lors plus de beau ciel, plus de barques gracieuses, plus d'ardente lumière; le ciel se voila de crêpes funèbres, et l'apparition du noir géant qu'on appelle le château d'If le fit tressaillir, comme si lui fût apparu tout à coup le fantôme d'un ennemi mortel. 

On arriva. 

Instinctivement le comte se recula jusqu'à extrémité de la barque. Le patron avait beau lui dire de sa voix la plus caressante: 

«Nous abordons, monsieur.» 

Monte-Cristo se rappela qu'à ce même endroit, sur ce même rocher, il avait été violemment traîné par ses gardes, et qu'on l'avait forcé de monter cette rampe en lui piquant les reins avec la pointe d'une baïonnette. 

La route avait autrefois semblé bien longue à Dantès. Monte-Cristo l'avait trouvée bien courte; chaque coup de rame avait fait jaillir avec la poussière humide de la mer un million de pensées et de souvenirs. 

Depuis la révolution de Juillet, il n'y avait plus de prisonniers au château d'If; un poste destiné à empêcher de faire la contrebande habitait seul ses corps de garde; un concierge attendait les curieux à la porte pour leur montrer ce monument de terreur, devenu un monument de curiosité. 

Et cependant, quoiqu'il fût instruit de tous ces détails, lorsqu'il entra sous la voûte, lorsqu'il descendit l'escalier noir, lorsqu'il fut conduit aux cachots qu'il avait demandé à voir, une froide pâleur envahit son front, dont la sueur glacée fut refoulée jusqu'à son cœur. 

Le comte s'informa s'il restait encore quelque ancien guichetier du temps de la Restauration; tous avaient été mis à la retraite ou étaient passés à d'autres emplois. Le concierge qui le conduisait était là depuis 1830 seulement. 

On le conduisit dans son propre cachot. 

Il revit le jour blafard filtrant par l'étroit soupirail; il revit la place où était le lit, enlevé depuis, et, derrière le lit, quoique bouchée, mais visible encore par ses pierres plus neuves, l'ouverture percée par l'abbé Faria. 

Monte-Cristo sentit ses jambes faiblir; il prit un escabeau de bois et s'assit dessus. 

«Conte-t-on quelques histoires sur ce château autres que celle de l'emprisonnement de Mirabeau? demanda le comte; y a-t-il quelque tradition sur ces lugubres demeures où l'on hésite à croire que des hommes aient jamais enfermé un homme vivant? 

—Oui, monsieur, dit le concierge, et sur ce cachot même, le guichetier Antoine m'en a transmis une.» 

Monte-Cristo tressaillit. Ce guichetier Antoine était son guichetier. Il avait à peu près oublié son nom et son visage; mais, à son nom prononcé, il le revit tel qu'il était, avec sa figure cerclée de barbe, sa veste brune et son trousseau de clefs, dont il lui semblait encore entendre le tintement. 

Le comte se retourna et crut le voir dans l'ombre du corridor, rendue plus épaisse par la lumière de la torche qui brûlait aux mains du concierge. 

«Monsieur veut-il que je la lui raconte? demanda le concierge. 

—Oui, fit Monte-Cristo, dites.» 

Et il mit sa main sur sa poitrine pour comprimer un violent battement de cœur, effrayé d'entendre raconter sa propre histoire. 

«Dites, répéta-t-il. 

—Ce cachot, reprit le concierge, était habité par un prisonnier, il y a longtemps de cela, un homme fort dangereux, à ce qu'il paraît, et d'autant plus dangereux qu'il était plein d'industrie. Un autre homme habitait ce château en même temps que lui; celui-là n'était pas méchant; c'était un pauvre prêtre qui était fou. 

—Ah! oui, fou, répéta Monte-Cristo; et quelle était sa folie? 

—Il offrait des millions si on voulait lui rendre la liberté.» 

Monte-Cristo leva les yeux au ciel, mais il ne vit pas le ciel: il y avait un voile de pierre entre lui et le firmament. Il songea qu'il y avait eu un voile non moins épais entre les yeux de ceux à qui l'abbé Faria offrait des trésors et ces trésors qu'il leur offrait. 

«Les prisonniers pouvaient-ils se voir? demanda Monte-Cristo. 

—Oh! non, monsieur, c'était expressément défendu; mais ils éludèrent la défense en perçant une galerie qui allait d'un cachot à l'autre. 

—Et lequel des deux perça cette galerie? 

—Oh! ce fut le jeune homme, bien certainement, dit le concierge; le jeune homme était industrieux et fort, tandis que le pauvre abbé était vieux et faible; d'ailleurs il avait l'esprit trop vacillant pour suivre une idée. 

—Aveugles!\dots murmura Monte-Cristo. 

—Tant il y a, continua le concierge, que le jeune perça donc une galerie; avec quoi? l'on n'en sait rien mais il la perça, et la preuve, c'est qu'on en voit encore la trace; tenez, la voyez-vous?» 

Et il approcha sa torche de la muraille. 

«Ah! oui, vraiment, fit le comte d'une voix assourdie par l'émotion. 

—Il en résulta que les deux prisonniers communiquèrent ensemble. Combien de temps dura cette communication? on n'en sait rien. Or, un jour le vieux prisonnier tomba malade et mourut. Devinez ce que fit le jeune? fit le concierge en s'interrompant. 

—Dites. 

—Il emporta le défunt, qu'il coucha dans son propre lit, le nez tourné à la muraille, puis il revint dans le cachot vide, boucha le trou, et se glissa dans le sac du mort. Avez-vous jamais vu une idée pareille?» 

Monte-Cristo ferma les yeux et se sentit repasser par toutes les impressions qu'il avait éprouvées lorsque cette toile grossière, encore empreinte de ce froid que le cadavre lui avait communiqué, lui avait frotté le visage. 

Le guichetier continua: 

«Voyez-vous, voilà quel était son projet: il croyait qu'on enterrait les morts au château d'If, et comme il se doutait bien qu'on ne faisait pas de frais de cercueil pour les prisonniers, il comptait lever la terre avec ses épaules, mais il y avait malheureusement au château une coutume qui dérangeait son projet: on n'enterrait pas les morts; on se contentait de leur attacher un boulet aux pieds et de les lancer à la mer: c'est ce qui fut fait. Notre homme fut jeté à l'eau du haut de la galerie; le lendemain on retrouva le vrai mort dans son lit, et l'on devina tout, car les ensevelisseurs dirent alors ce qu'ils n'avaient pas osé dire jusque-là, c'est qu'au moment où le corps avait été lancé dans le vide ils avaient entendu un cri terrible, étouffé à l'instant même par l'eau dans laquelle il avait disparu. 

Le comte respira péniblement, la sueur coulait sur son front, l'angoisse serrait son cœur. 

«Non! murmura-t-il, non! ce doute que j'ai éprouvé, c'était un commencement d'oubli; mais ici le cœur se creuse de nouveau et redevient affamé de vengeance.» 

«Et le prisonnier, demanda-t-il, on n'en a jamais entendu parler? 

—Jamais, au grand jamais; vous comprenez, de deux choses l'une, ou il est tombé à plat, et, comme il tombait d'une cinquantaine de pieds, il se sera tué sur le coup. 

—Vous avez dit qu'on lui avait attaché un boulet aux pieds: il sera tombé debout. 

—Ou il est tombé debout, reprit le concierge, et alors le poids du boulet l'aura entraîné au fond, où il est resté, pauvre cher homme! 

—Vous le plaignez? 

—Ma foi, oui, quoiqu'il fût dans son élément. 

—Que voulez-vous dire? 

—Qu'il y avait un bruit qui courait que ce malheureux était, dans son temps, un officier de marine détenu pour bonapartisme.» 

«Vérité, murmura le comte, Dieu t'a faite pour surnager au-dessus des flots et des flammes. Ainsi le pauvre marin vit dans le souvenir de quelques conteurs; on récite sa terrible histoire au coin du foyer et l'on frissonne au moment où il fendit l'espace pour s'engloutir dans la profonde mer.» 

«On n'a jamais su son nom? demanda tout haut le comte. 

—Ah! bien oui, dit le gardien, comment? il n'était connu que sous le nom du numéro 34. 

—Villefort, Villefort! murmura Monte-Cristo, voilà ce que bien des fois tu as dû te dire quand mon spectre importunait tes insomnies. 

—Monsieur veut-il continuer la visite? demanda le concierge. 

—Oui, surtout si vous voulez me montrer la chambre du pauvre abbé. 

—Ah! du numéro 27» 

—Oui, du numéro 27», répéta Monte-Cristo. 

Et il lui sembla encore entendre la voix de l'abbé Faria lorsqu'il lui avait demandé son nom, et que celui-ci avait crié ce numéro à travers la muraille. 

«Venez. 

—Attendez, dit Monte-Cristo, que je jette un dernier regard sur toutes les faces de ce cachot. 

—Cela tombe bien, dit le guide, j'ai oublié la clef de l'autre. 

—Allez la chercher. 

—Je vous laisse la torche. 

—Non, emportez-la. 

—Mais vous allez rester sans lumière. 

—J'y vois la nuit. 

—Tiens, c'est comme lui. 

—Qui, lui? 

—Le numéro 34. On dit qu'il s'était tellement habitué à l'obscurité, qu'il eût vu une épingle dans le coin le plus obscur de son cachot. 

—Il lui a fallu dix ans pour en arriver là», murmura le comte. 

Le guide s'éloigna emportant la torche. 

Le comte avait dit vrai: à peine fut-il depuis quelques secondes dans l'obscurité, qu'il distingua tout comme en plein jour. 

Alors il regarda tout autour de lui, alors il reconnut bien réellement son cachot. 

«Oui, dit-il, voilà la pierre sur laquelle je m'asseyais! voilà la trace de mes épaules qui ont creusé leur empreinte dans la muraille! voilà la trace du sang qui a coulé de mon front, un jour que j'ai voulu me briser le front contre la muraille\dots Oh! ces chiffres\dots je me les rappelle\dots je les fis un jour que je calculais l'âge de mon père pour savoir si je le retrouverais vivant, et l'âge de Mercédès pour savoir si je la retrouverais libre\dots J'eus un instant d'espoir après avoir achevé ce calcul\dots Je comptais sans la faim et sans l'infidélité!» 

Et un rire amer s'échappa de la bouche du comte. Il venait de voir, comme dans un rêve, son père conduit à la tombe\dots Mercédès marchant à l'autel! 

Sur l'autre paroi de la muraille, une inscription frappa sa vue. Elle se détachait, blanche encore, sur le mur verdâtre: 

«MON DIEU! lut Monte-Cristo, CONSERVEZ-MOI LA MÉMOIRE!» 

«Oh! oui, s'écria-t-il, voilà la seule prière de mes derniers temps. Je ne demandais plus la liberté, je demandais la mémoire, je craignais de devenir fou et d'oublier. Mon Dieu! vous m'avez conservé la mémoire, et je me suis souvenu. Merci, merci, mon Dieu!» 

En ce moment, la lumière de la torche miroita sur les murailles; c'était le guide qui descendait. 

Monte-Cristo alla au-devant de lui. 

«Suivez-moi», dit-il. 

Et, sans avoir besoin de remonter vers le jour, il lui fit suivre un corridor souterrain qui le conduisit à une autre entrée. 

Là encore Monte-Cristo fut assailli par un monde de pensées. 

La première chose qui frappa ses yeux fut le méridien tracé sur la muraille, à l'aide duquel l'abbé Faria comptait les heures; puis les restes du lit sur lequel le pauvre prisonnier était mort. 

À cette vue, au lieu des angoisses que le comte avait éprouvées dans son cachot, un sentiment doux et tendre, un sentiment de reconnaissance gonfla son cœur, deux larmes roulèrent de ses yeux. 

«C'est ici, dit le guide, qu'était l'abbé fou; c'est par là que le jeune homme le venait trouver. (Et il montra à Monte-Cristo l'ouverture de la galerie qui, de ce côté était restée béante.) À la couleur de la pierre, continua-t-il, un savant a reconnu qu'il devait y avoir dix ans à peu près que les deux prisonniers communiquaient ensemble. Pauvres gens, ils ont dû bien s'ennuyer pendant ces dix ans.» 

Dantès prit quelques louis dans sa poche, et tendit la main vers cet homme qui, pour la seconde fois, le plaignait sans le connaître. 

Le concierge les accepta, croyant recevoir quelques menues pièces de monnaie, mais à la lueur de la torche, il reconnut la valeur de la somme que lui donnait le visiteur. 

«Monsieur, lui dit-il, vous vous êtes trompé. 

—Comment cela? 

—C'est de l'or que vous m'avez donné. 

—Je le sais bien. 

—Comment! vous le savez? 

—Oui. 

—Votre intention est de me donner cet or? 

—Oui. 

—Et je puis le garder en toute conscience? 

—Oui.» 

Le concierge regarda Monte-Cristo avec étonnement. 

«Et \textit{honnêteté}, dit le comte comme Hamlet. 

—Monsieur, reprit le concierge qui n'osait croire à son bonheur, monsieur, je ne comprends pas votre générosité. 

—Elle est facile à comprendre, cependant, mon ami, dit le comte: j'ai été marin, et votre histoire a dû me toucher plus qu'un autre. 

—Alors, monsieur, dit le guide, puisque vous êtes si généreux, vous méritez que je vous offre quelque chose. 

—Qu'as-tu à m'offrir, mon ami? des coquilles, des ouvrages de paille? merci. 

—Non pas, monsieur, non pas; quelque chose qui se rapporte à l'histoire de tout à l'heure. 

—En vérité! s'écria vivement le comte, qu'est-ce donc? 

—Écoutez, dit le concierge, voilà ce qui est arrivé: je me suis dit: On trouve toujours quelque chose dans une chambre où un prisonnier est resté quinze ans, et je me suis mis à sonder les murailles. 

—Ah! s'écria Monte-Cristo en se rappelant la double cachette de l'abbé, en effet. 

—À force de recherches, continua le concierge, j'ai découvert que cela sonnait le creux au chevet du lit et sous l'âtre de la cheminée. 

—Oui, dit Monte-Cristo, oui. 

—J'ai levé les pierres, et j'ai trouvé\dots 

—Une échelle de corde, des outils? s'écria le comte. 

—Comment savez-vous cela? demanda le concierge avec étonnement. 

—Je ne le sais pas, je le devine, dit le comte; c'est ordinairement ces sortes de choses que l'on trouve dans les cachettes des prisonniers. 

—Oui, monsieur, dit le guide, une échelle de corde, des outils. 

—Et tu les as encore? s'écria Monte-Cristo. 

—Non, monsieur; j'ai vendu ces différents objets, qui étaient fort curieux, à des visiteurs; mais il me reste autre chose. 

—Quoi donc? demanda le comte avec impatience. 

—Il me reste une espèce de livre écrit sur des bandes de toile. 

—Oh! s'écria Monte-Cristo, il te reste ce livre? 

—Je ne sais pas si c'est un livre, dit le concierge; mais il me reste ce que je vous dis. 

—Va me le chercher, mon ami, va, dit le comte; et, si c'est ce que je présume, sois tranquille. 

—J'y cours, monsieur. 

Et le guide sortit. 

Alors il alla s'agenouiller pieusement devant les débris de ce lit dont la mort avait fait pour lui un autel. 

«Ô mon second père, dit-il, toi qui m'as donné la liberté, la science, la richesse; toi qui, pareil aux créatures d'une essence supérieure à la nôtre, avais la science du bien et du mal, si au fond de la tombe il reste quelque chose de nous qui tressaille à la voix de ceux qui sont demeurés sur la terre, si dans la transfiguration que subit le cadavre quelque chose d'animé flotte aux lieux où nous avons beaucoup aimé ou beaucoup souffert, noble cœur, esprit suprême, âme profonde, par un mot, par un signe, par une révélation quelconque, je t'en conjure, au nom de cet amour paternel que tu m'accordais et de ce respect filial que je t'avais voué, enlève-moi ce reste de doute qui, s'il ne se change en conviction, deviendra un remords. 

Le comte baissa la tête et joignit les mains. 

«Tenez, monsieur!» dit une voix derrière lui. 

Monte-Cristo tressaillit et se retourna. 

Le concierge lui tendait ces bandes de toile sur lesquelles l'abbé Faria avait épanché tous les trésors de sa science. Ce manuscrit c'était le grand ouvrage de l'abbé Faria sur la royauté en Italie. 

Le comte s'en empara avec empressement, et ses yeux tout d'abord tombant sur l'épigraphe, il lut: «Tu arracheras les dents du dragon, et tu fouleras aux pieds les lions, a dit le Seigneur.» 

«Ah! s'écria-t-il, voilà la réponse! merci, mon père, merci!» 

En tirant de sa poche un petit portefeuille, qui contenait dix billets de banque de mille francs chacun: 

«Tiens, dit-il, prends ce portefeuille. 

—Vous me le donnez? 

—Oui, mais à la condition que tu ne regarderas dedans que lorsque je serai parti.» 

Et, plaçant sur sa poitrine la relique qu'il venait de retrouver et qui pour lui avait le prix du plus riche trésor, il s'élança hors du souterrain, et remontant dans la barque: 

«À Marseille!» dit-il. 

Puis en s'éloignant, les yeux fixés sur la sombre prison: 

«Malheur, dit-il, à ceux qui m'ont fait enfermer dans cette sombre prison, et à ceux qui ont oublié que j'y étais enfermé!» 

En repassant devant les Catalans, le comte se détourna, et s'enveloppant la tête dans son manteau, il murmura le nom d'une femme. 

La victoire était complète; le comte avait deux fois terrassé le doute. 

Ce nom, qu'il prononçait avec une expression de tendresse qui était presque de l'amour, c'était le nom d'Haydée. 

En mettant pied à terre, Monte-Cristo s'achemina vers le cimetière, où il savait retrouver Morrel. 

Lui aussi, dix ans auparavant, avait pieusement cherché une tombe dans ce cimetière, et l'avait cherchée inutilement. Lui, qui revenait en France avec des millions, n'avait pas pu retrouver la tombe de son père mort de faim. 

Morrel y avait bien fait mettre une croix, mais cette croix était tombée, et le fossoyeur en avait fait du feu, comme font les fossoyeurs de tous ces vieux bois gisant dans les cimetières. 

Le digne négociant avait été plus heureux: mort dans les bras de ses enfants, il avait été, conduit par eux, se coucher près de sa femme, qui l'avait précédé de deux ans dans l'éternité. 

Deux larges dalles de marbre, sur lesquelles étaient écrits leurs noms, étaient étendues l'une à côté de l'autre dans un petit enclos fermé d'une balustrade de fer et ombragé par quatre cyprès. 

Maximilien était appuyé à l'un de ces arbres, et fixait sur les deux tombes des yeux sans regard. 

Sa douleur était profonde, presque égarée. 

«Maximilien, lui dit le comte, ce n'est point là qu'il faut regarder, c'est là!» 

Et il lui montra le ciel. 

«Les morts sont partout, dit Morrel; n'est-ce pas ce que vous m'avez dit vous-même quand vous m'avez fait quitter Paris? 

—Maximilien, dit le comte, vous m'avez demandé pendant le voyage à vous arrêter quelques jours à Marseille: est-ce toujours votre désir? 

—Je n'ai plus de désir, comte, mais il me semble que j'attendrai moins péniblement ici qu'ailleurs. 

—Tant mieux, Maximilien, car je vous quitte et j'emporte votre parole, n'est-ce pas? 

—Ah! je l'oublierai, comte, dit Morrel, je l'oublierai! 

—Non! vous ne l'oublierez pas, parce que vous êtes homme d'honneur avant tout, Morrel, parce que vous avez juré, parce que vous allez jurer encore. 

—Ô Comte, ayez pitié de moi! Comte, je suis si malheureux! 

—J'ai connu un homme plus malheureux que vous, Morrel. 

—Impossible. 

—Hélas! dit Monte-Cristo, c'est un des orgueils de notre pauvre humanité, que chaque homme se croie plus malheureux qu'un autre malheureux qui pleure et qui gémit à côté de lui. 

—Qu'y a-t-il de plus malheureux que l'homme qui a perdu le seul bien qu'il aimât et désirât au monde? 

—Écoutez, Morrel, dit Monte-Cristo, et fixez un instant votre esprit sur ce que je vais vous dire. J'ai connu un homme qui, ainsi que vous, avait fait reposer toutes ses espérances de bonheur sur une femme. Cet homme était jeune, il avait un vieux père qu'il aimait, une fiancée qu'il adorait; il allait l'épouser quand tout à coup un de ces caprices du sort qui feraient douter de la bonté de Dieu, si Dieu ne se révélait plus tard en montrant que tout est pour lui un moyen de conduire à son unité infinie, quand tout à coup un caprice du sort lui enleva sa liberté, sa maîtresse, l'avenir qu'il rêvait et qu'il croyait le sien (car aveugle qu'il était, il ne pouvait lire que dans le présent) pour le plonger au fond d'un cachot. 

—Ah! fit Morrel, on sort d'un cachot au bout de huit jours, au bout d'un mois, au bout d'un an. 

—Il y resta quatorze ans, Morrel», dit le comte en posant sa main sur l'épaule du jeune homme. 

Maximilien tressaillit. 

«Quatorze ans! murmura-t-il. 

—Quatorze ans, répéta le comte; lui aussi, pendant ces quatorze années, il eut bien des moments de désespoir; lui aussi, comme vous, Morrel, se croyant le plus malheureux des hommes, il voulut se tuer. 

—Eh bien? demanda Morrel. 

—Eh bien, au moment suprême, Dieu se révéla à lui par un moyen humain; car Dieu ne fait plus de miracles: peut-être au premier abord (il faut du temps aux yeux voilés de larmes pour se dessiller tout à fait), ne comprit-il pas cette miséricorde infinie du Seigneur, mais enfin il prit patience et attendit. Un jour il sortit miraculeusement de la tombe, transfiguré, riche, puissant, presque dieu; son premier cri fut pour son père: son père était mort! 

—Et à moi aussi mon père est mort, dit Morrel. 

—Oui, mais votre père est mort dans vos bras, aimé, heureux, honoré, riche, plein de jours; son père à lui était mort pauvre, désespéré, doutant de Dieu; et lorsque, dix ans après sa mort, son fils chercha sa tombe, sa tombe même avait disparu, et nul n'a pu lui dire: «C'est là que repose dans le Seigneur le cœur qui t'a tant aimé.» 

—Oh! dit Morrel. 

—Celui-là était donc plus malheureux fils que vous, Morrel, car celui-là ne savait pas même où retrouver la tombe de son père. 

—Mais, dit Morrel, il lui restait la femme qu'il avait aimée, au moins. 

—Vous vous trompez Morrel; cette femme\dots 

—Elle était morte? s'écria Maximilien. 

—Pis que cela: elle avait été infidèle; elle avait épousé un des persécuteurs de son fiancé. Vous voyez donc, Morrel, que cet homme était plus malheureux amant que vous! 

—Et à cet homme, demanda Morrel, Dieu a envoyé la consolation? 

—Il lui a envoyé le calme du moins. 

—Et cet homme pourra encore être heureux un jour? 

—Il l'espère, Maximilien.» 

Le jeune homme laissa tomber sa tête sur sa poitrine. 

«Vous avez ma promesse, dit-il après un instant de silence, et tendant la main à Monte-Cristo: seulement rappelez-vous\dots 

—Le 5 octobre, Morrel, je vous attends à l'île de Monte-Cristo. Le 4, un yacht vous attendra dans le port de Bastia; ce yacht s'appellera \textit{l'Eurus}; vous vous nommerez au patron qui vous conduira près de moi. C'est dit, n'est-ce pas, Maximilien? 

—C'est dit, comte, et je ferai ce qui est dit; mais rappelez-vous que le 5 octobre\dots 

—Enfant, qui ne sait pas encore ce que c'est que la promesse d'un homme\dots Je vous ai dit vingt fois que ce jour-là, si vous vouliez encore mourir, je vous aiderais, Morrel. Adieu. 

—Vous me quittez?» 

—Oui, j'ai affaire en Italie; je vous laisse seul, seul aux prises avec le malheur, seul avec cet aigle aux puissantes ailes que le Seigneur envoie à ses élus pour les transporter, à ses pieds. L'histoire de Ganymède n'est pas une fable, Maximilien, c'est une allégorie. 

—Quand partez-vous? 

—À l'instant même; le bateau à vapeur m'attend, dans une heure je serai déjà loin de vous; m'accompagnerez-vous jusqu'au port, Morrel? 

—Je suis tout à vous, comte. 

—Embrassez-moi.» 

Morrel escorta le comte jusqu'au port; déjà la fumée sortait, comme un panache immense, du tube noir qui la lançait aux cieux. Bientôt le navire partit, et une heure après, comme l'avait dit Monte-Cristo, cette même aigrette de fumée blanchâtre rayait, à peine visible, l'horizon oriental, assombri par les premiers brouillards de la nuit. 