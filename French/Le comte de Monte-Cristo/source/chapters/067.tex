\chapter{Le cabinet du procureur du roi}

\lettrine{L}{aissons} le banquier revenir au grand trot de ses chevaux, et suivons Mme Danglars dans son excursion matinale. 

\zz
Nous avons dit qu'à midi et demi Mme Danglars avait demandé ses chevaux et était sortie en voiture. 

Elle se dirigea du côté du faubourg Saint-Germain, prit la rue Mazarine, et fit arrêter au passage du Pont-Neuf. 

Elle descendit et traversa le passage. Elle était vêtue fort simplement, comme il convient à une femme de goût qui sort le matin. 

Rue Guénégaud, elle monta en fiacre en désignant, comme le but de sa course, la rue du Harlay. 

À peine fut-elle dans la voiture, qu'elle tira de sa poche un voile noir très épais, qu'elle attacha sur son chapeau de paille; puis elle remit son chapeau sur sa tête, et vit avec plaisir, en regardant dans un petit miroir de poche, qu'on ne pouvait voir d'elle que sa peau blanche et la prunelle étincelante de son œil. 

Le fiacre prit le Pont-Neuf, et entra, par la place Dauphine, dans la cour du Harlay; il fut payé en ouvrant la portière, et Mme Danglars s'élançant vers l'escalier, qu'elle franchit légèrement, arriva bientôt à la salle des Pas-Perdus. 

Le matin, il y a beaucoup d'affaires et encore plus de gens affairés au Palais; les gens affairés ne regardent pas beaucoup les femmes; Mme Danglars traversa donc la salle des Pas-Perdus sans être plus remarquée que dix autres femmes qui guettaient leur avocat. 

Il y avait encombrement dans l'antichambre de M. de Villefort; mais Mme Danglars n'eut pas même besoin de prononcer son nom, dès qu'elle parut, un huissier se leva, vint à elle, lui demanda si elle n'était point la personne à laquelle M. le procureur du roi avait donné rendez-vous, et, sur sa réponse affirmative, il la conduisit, par un corridor réservé, au cabinet de M. de Villefort. 

Le magistrat écrivait, assis sur son fauteuil, le dos tourné à la porte: il entendit la porte s'ouvrir, l'huissier prononcer ces paroles: «Entrez, madame!» et la porte se refermer, sans faire un seul mouvement; mais à peine eut-il senti se perdre les pas de l'huissier, qui s'éloignait, qu'il se retourna vivement, alla pousser les verrous, tirer les rideaux et visiter chaque coin du cabinet. 

Puis lorsqu'il eut acquis la certitude qu'il ne pouvait être ni vu ni entendu, et que par conséquent il fut tranquillisé: 

«Merci, madame, dit-il, merci de votre exactitude.» 

Et il lui offrit un siège que Mme Danglars accepta, car le cœur lui battait si fortement qu'elle se sentait près de suffoquer. 

«Voilà, dit le procureur du roi en s'asseyant à son tour et en faisant décrire un demi-cercle à son fauteuil, afin de se trouver en face de Mme Danglars, voilà bien longtemps, madame, qu'il ne m'est arrivé d'avoir ce bonheur de causer seul avec vous; et, à mon grand regret, nous nous retrouvons pour entamer une conversation bien pénible. 

—Cependant, monsieur, vous voyez que je suis venue à votre premier appel, quoique certainement cette conversation soit encore plus pénible pour moi que pour vous.» 

Villefort sourit amèrement. 

«Il est donc vrai, dit-il, répondant à sa propre pensée bien plutôt qu'aux paroles de Mme Danglars, il est donc vrai que toutes nos actions laissent leurs traces, les unes sombres, les autres lumineuses, dans notre passé! Il est donc vrai que tous nos pas dans cette vie ressemblent à la marche du reptile sur le sable et font un sillon! Hélas! pour beaucoup, ce sillon est celui de leurs larmes! 

—Monsieur, dit Mme Danglars, vous comprenez mon émotion, n'est-ce pas? ménagez-moi donc, je vous prie. Cette chambre où tant de coupables ont passé tremblants et honteux, ce fauteuil où je m'assieds à mon tour honteuse et tremblante!\dots Oh! tenez, j'ai besoin de toute ma raison pour ne pas voir en moi une femme bien coupable et en vous un juge menaçant.» 

Villefort secoua la tête et poussa un soupir. 

«Et moi, reprit-il, et moi, je me dis que ma place n'est pas dans le fauteuil du juge, mais bien sur la sellette de l'accusé. 

—Vous? dit Mme Danglars étonnée. 

—Oui, moi. 

—Je crois que de votre part, monsieur, votre puritanisme s'exagère la situation, dit Mme Danglars, dont l'œil si beau s'illumina d'une fugitive lueur. Ces sillons dont vous parliez à l'instant même, ont été tracés par toutes les jeunesses ardentes. Au fond des passions au-delà du plaisir, il y a toujours un peu de remords; c'est pour cela que l'Évangile, cette ressource éternelle des malheureux, nous a donné pour soutien, à nous autres pauvres femmes, l'admirable parabole de la fille pécheresse et de la femme adultère. Aussi, je vous l'avoue, en me reportant à ces délires de ma jeunesse je pense quelquefois que Dieu me les pardonnera, car sinon l'excuse, du moins la compensation s'en est bien trouvée dans mes souffrances; mais vous, qu'avez-vous à craindre de tout cela, vous autres hommes que tout le monde excuse et que le scandale anoblit? 

—Madame, répliqua Villefort, vous me connaissez; je ne suis pas un hypocrite, ou du moins je ne fais pas de l'hypocrisie sans raison. Si mon front est sévère c'est que bien des malheurs l'ont assombri, si mon cœur s'est pétrifié, c'est afin de pouvoir supporter les chocs qu'il a reçus. Je n'étais pas ainsi dans ma jeunesse, je n'étais pas ainsi ce soir des fiançailles où nous étions tous assis autour d'une table de la rue du Cours à Marseille. Mais, depuis, tout a bien changé en moi et autour de moi; ma vie s'est usée à poursuivre des choses difficiles et à briser dans les difficultés ceux qui, volontairement ou involontairement, par leur libre arbitre ou par le hasard, se trouvaient placés sur mon chemin pour me susciter ces choses. Il est rare que ce qu'on désire ardemment ne soit pas défendu ardemment par ceux de qui on veut l'obtenir ou auxquels on tente de l'arracher. Ainsi, la plupart des mauvaises actions des hommes sont venues au-devant d'eux, déguisées sous la forme spécieuse de la nécessité; puis, la mauvaise action commise dans un moment d'exaltation, de crainte et de délire, on voit qu'on aurait pu passer auprès d'elle en l'évitant. Le moyen qu'il eût été bon d'employer, qu'on n'a pas vu, aveugle qu'on était, se présente à vos yeux facile et simple; vous vous dites: Comment n'ai-je pas fait cela au lieu de faire cela? Vous, mesdames, au contraire, bien rarement vous êtes tourmentées par des remords, car bien rarement la décision vient de vous, vos malheurs vous sont presque toujours imposés, vos fautes sont presque toujours le crime des autres. 

—En tout cas, monsieur, convenez-en, répondit Mme Danglars, si j'ai commis une faute, cette faute fût-elle personnelle, j'en ai reçu hier la sévère punition. 

—Pauvre femme! dit Villefort en lui serrant la main, trop sévère pour votre force car deux fois vous avez failli y succomber, et cependant\dots. 

—Eh bien? 

—Eh bien, je dois vous dire\dots rassemblez tout votre courage, madame, car vous n'êtes pas encore au bout. 

—Mon Dieu! s'écria Mme Danglars effrayée, qu'y a-t-il donc encore? 

—Vous ne voyez que le passé, madame, et certes il est sombre. Eh bien, figurez-vous un avenir plus sombre encore, un avenir\dots affreux certainement\dots sanglant peut-être!\dots» 

La baronne connaissait le calme de Villefort; elle fut si épouvantée de son exaltation, qu'elle ouvrit la bouche pour crier, mais que le cri mourut dans sa gorge. 

«Comment est-il ressuscité, ce passé terrible? s'écria Villefort; comment, du fond de la tombe et du fond de nos cœurs où il dormait, est-il sorti comme un fantôme pour faire pâlir nos joues et rougir nos fronts? 

—Hélas! dit Hermine, sans doute le hasard! 

—Le hasard! reprit Villefort; non, non, madame, il n'y a point de hasard! 

—Mais si; n'est-ce point un hasard, fatal il est vrai mais un hasard qui a fait tout cela? n'est-ce point par hasard que le comte de Monte-Cristo a acheté cette maison? n'est-ce point par hasard qu'il a fait creuser la terre? n'est-ce point par hasard, enfin, que ce malheureux enfant a été déterré sous les arbres? Pauvre innocente créature sortie de moi, à qui je n'ai jamais pu donner un baiser, mais à qui j'ai donné bien des larmes. Ah! tout mon cœur a volé au-devant du comte lorsqu'il a parlé de cette chère dépouille trouvée sous des fleurs. 

—Eh bien, non, madame; et voilà ce que j'avais de terrible à vous dire, répondit Villefort d'une voix sourde: non, il n'y a pas eu de dépouille trouvée sous les fleurs; non, il n'y a pas eu d'enfant déterré; non, il ne faut pas pleurer; non, il ne faut pas gémir: il faut trembler! 

—Que voulez-vous dire? s'écria Mme Danglars toute frémissante. 

—Je veux dire que M. Monte-Cristo, en creusant au pied de ces arbres, n'a pu trouver ni squelette d'enfant ni ferrure de coffre, parce que sous ces arbres il n'y avait ni l'un ni l'autre. 

—Il n'y avait ni l'un ni l'autre! redit Mme Danglars, en fixant sur le procureur du roi des yeux dont la prunelle, effroyablement dilatée, indiquait la terreur; il n'y avait ni l'un ni l'autre! répéta-t-elle encore comme une personne qui essaie de fixer par le son des paroles et par le bruit de la voix ses idées prêtes à lui échapper. 

—Non! dit Villefort, en laissant tomber son front dans ses mains, cent fois non!\dots 

—Mais ce n'est donc point là que vous aviez déposé le pauvre enfant, monsieur? Pourquoi me tromper? dans quel but, voyons, dites? 

—C'est là; mais écoutez-moi, écoutez-moi madame, et vous allez me plaindre, moi qui ai porté vingt ans, sans en rejeter la moindre part sur vous, le fardeau de douleurs que je vais vous dire. 

—Mon Dieu! vous m'effrayez! mais n'importe, parlez, je vous écoute. 

—Vous savez comment s'accomplit cette nuit douloureuse où vous étiez expirante sur votre lit, dans cette chambre de damas rouge, tandis que moi, presque aussi haletant que vous, j'attendais votre délivrance. L'enfant vint, me fut remis sans mouvement, sans souffle, sans voix: nous le crûmes mort.» 

Mme Danglars fit un mouvement rapide, comme si elle eût voulu s'élancer de sa chaise. 

Mais Villefort l'arrêta en joignant les mains comme pour implorer son attention. 

«Nous le crûmes mort, répéta-t-il; je le mis dans un coffre qui devait remplacer le cercueil, je descendis au jardin, je creusai une fosse et l'enfouis à la hâte. J'achevais à peine de le couvrir de terre, que le bras du Corse s'étendit vers moi. Je vis comme une ombre se dresser, comme un éclair reluire. Je sentis une douleur, je voulus crier, un frisson glacé me parcourut tout le corps et m'étreignit à la gorge\dots. Je tombai mourant, et je me crus tué. Je n'oublierai jamais votre sublime courage, quand, revenu à moi, je me traînai expirant jusqu'au bas de l'escalier, où, expirante vous-même, vous vîntes au-devant de moi. Il fallait garder le silence sur la terrible catastrophe; vous eûtes le courage de regagner votre maison, soutenue par votre nourrice; un duel fut le prétexte de ma blessure. Contre toute attente, le secret nous fut gardé à tous deux, on me transporta à Versailles; pendant trois mois, je luttai contre la mort; enfin comme je parus me rattacher à la vie, on m'ordonna le soleil et l'air du Midi. Quatre hommes me portèrent de Paris à Châlons, en faisant six lieues par jour. Mme de Villefort suivait le brancard dans sa voiture. À Châlons, on me mit sur la Saône, puis je passai sur le Rhône, et, par la seule vitesse du courant, je descendis jusqu'à Arles, puis d'Arles, je repris ma litière et continuai mon chemin pour Marseille. Ma convalescence dura six mois; je n'entendais plus parler de vous, je n'osai m'informer de ce que vous étiez devenue. Quand je revins à Paris, j'appris que, veuve de M. de Nargonne, vous aviez épousé M. Danglars. 

«À quoi avais-je pensé depuis que la connaissance m'était revenue? Toujours à la même chose, toujours à ce cadavre d'enfant qui, chaque nuit, dans mes rêves s'envolait du sein de la terre, et planait au-dessus de la fosse en me menaçant du regard et du geste. Aussi, à peine de retour à Paris, je m'informai; la maison n'avait pas été habitée depuis que nous en étions sortis, mais elle venait d'être louée pour neuf ans. J'allai trouver le locataire, je feignis d'avoir un grand désir de ne pas voir passer entre des mains étrangères cette maison qui appartenait au père et à la mère de ma femme; j'offris un dédommagement pour qu'on rompît le bail; on me demanda six mille francs: j'en eusse donné dix mille, j'en eusse donné vingt mille. Je les avais sur moi, je fis, séance tenante, signer la résiliation; puis, lorsque je tins cette cession tant désirée, je partis au galop pour Auteuil. Personne, depuis que j'en étais sorti, n'était entré dans la maison. 

«Il était cinq heures de l'après-midi, je montai dans la chambre rouge et j'attendis la nuit. 

«Là, tout ce que je me disais depuis un an dans mon agonie continuelle se représenta, bien plus menaçant que jamais, à ma pensée. 

«Ce Corse qui m'avait déclaré la vendetta, qui m'avait suivi de Nîmes à Paris; ce Corse, qui était caché dans le jardin, qui m'avait frappé, m'avait vu creuser la fosse, il m'avait vu enterrer l'enfant; il pouvait en arriver à vous connaître; peut-être vous connaissait-il\dots. Ne vous ferait-il pas payer un jour le secret de cette terrible affaire?\dots Ne serait-ce pas pour lui une bien douce vengeance, quand il apprendrait que je n'étais pas mort de son coup de poignard? Il était donc urgent qu'avant toute chose, et à tout hasard, je fisse disparaître les traces de ce passé, que j'en détruisisse tout vestige matériel; il n'y aurait toujours que trop de réalité dans mon souvenir. 

«C'était pour cela que j'avais annulé le bail, c'était pour cela que j'étais venu, c'était pour cela que j'attendais. 

«La nuit arriva, je la laissai bien s'épaissir; j'étais sans lumière dans cette chambre, où des souffles de vent faisaient trembler les portières derrière lesquelles je croyais toujours voir quelque espion embusqué; de temps en temps je tressaillais, il me semblait derrière moi, dans ce lit, entendre vos plaintes, et je n'osais me retourner. Mon cœur battait dans le silence, et je le sentais battre si violemment que je croyais que ma blessure allait se rouvrir; enfin, j'entendis s'éteindre, l'un après l'autre, tous ces bruits divers de la campagne. Je compris que je n'avais plus rien à craindre, que je ne pouvais être ni vu ni entendu, et je me décidai à descendre. 

«Écoutez, Hermine, je me crois aussi brave qu'un autre homme, mais lorsque je retirai de ma poitrine cette petite clef de l'escalier, que nous chérissions tous deux, et que vous aviez voulu faire attacher à un anneau d'or, lorsque j'ouvris la porte, lorsque, à travers les fenêtres, je vis une lune pâle jeter, sur les degrés en spirale, une longue bande de lumière blanche pareille à un spectre, je me retins au mur et je fus près de crier; il me semblait que j'allais devenir fou. 

«Enfin, je parvins à me rendre maître de moi-même. Je descendis l'escalier marche à marche; la seule chose que je n'avais pu vaincre, c'était un étrange tremblement dans les genoux. Je me cramponnai à la rampe; si je l'eusse lâchée un instant, je me fusse précipité. 

«J'arrivai à la porte d'en bas; en dehors de cette porte, une bêche était posée contre le mur. Je m'étais muni d'une lanterne sourde; au milieu de la pelouse, je m'arrêtai pour l'allumer, puis je continuai mon chemin. 

«Novembre finissait, toute la verdure du jardin avait disparu, les arbres n'étaient plus que des squelettes aux longs bras décharnés, et les feuilles mortes criaient avec le sable sous mes pas. 

«L'effroi m'étreignait si fortement le cœur, qu'en approchant du massif je tirai un pistolet de ma poche et l'armai. Je croyais toujours voir apparaître à travers les branches la figure du Corse. 

«J'éclairai le massif avec ma lanterne sourde; il était vide. Je jetai les yeux tout autour de moi; j'étais bien seul; aucun bruit ne troublait le silence de la nuit, si ce n'est le chant d'une chouette qui jetait son cri aigu et lugubre comme un appel aux fantômes de la nuit. 

«J'attachai ma lanterne à une branche fourchue que j'avais déjà remarquée un an auparavant, à l'endroit même où je m'arrêtai pour creuser la fosse. 

«L'herbe avait, pendant l'été, poussé bien épaisse à cet endroit, et, l'automne venu, personne ne s'était trouvé là pour la faucher. Cependant, une place moins garnie attira mon attention; il était évident que c'était là que j'avais retourné la terre. Je me mis à l'œuvre. 

«J'en étais donc arrivé à cette heure que j'attendais depuis plus d'un an! 

«Aussi, comme j'espérais, comme je travaillais, comme je sondais chaque touffe de gazon, croyant sentir de la résistance au bout de ma bêche; rien! et cependant je fis un trou deux fois plus grand que n'était le premier. Je crus m'être abusé, m'être trompé de place; je m'orientai, je regardai les arbres, je cherchai à reconnaître les détails qui m'avaient frappé. Une bise froide et aiguë sifflait à travers les branches dépouillées, et cependant la sueur ruisselait sur mon front. Je me rappelai que j'avais reçu le coup de poignard au moment où je piétinais la terre pour recouvrir la fosse; en piétinant cette terre, je m'appuyais à un faux ébénier; derrière moi était un rocher artificiel destiné à servir de banc aux promeneurs; car en tombant, ma main, qui venait de quitter l'ébénier, avait senti la fraîcheur de cette pierre. À ma droite était le faux ébénier, derrière moi était le rocher, je tombai en me plaçant de même, je me relevai et me mis à creuser et à élargir le trou: rien! toujours rien! le coffret n'y était pas.  

—Le coffret n'y était pas? murmura Mme Danglars suffoquée par l'épouvante. 

—Ne croyez pas que je me bornai à cette tentative, continua Villefort; non. Je fouillai tout le massif; je pensai que l'assassin, ayant déterré le coffre et croyant que c'était un trésor, avait voulu s'en emparer, l'avait emporté; puis s'apercevant de son erreur, avait fait à son tour un trou et l'y avait déposé; rien. Puis il me vint cette idée qu'il n'avait point pris tant de précautions, et l'avait purement et simplement jeté dans quelque coin. Dans cette dernière hypothèse, il me fallait, pour faire mes recherches, attendre le jour. Je remontai dans la chambre et j'attendis. 

—Oh! mon Dieu! 

—Le jour venu, je descendis de nouveau. Ma première visite fut pour le massif; j'espérais y retrouver des traces qui m'auraient échappé pendant l'obscurité. J'avais retourné la terre sur une superficie de plus de vingt pieds carrés, et sur une profondeur de plus de deux pieds. Une journée eût à peine suffi à un homme salarié pour faire ce que j'avais fait, moi, en une heure. Rien, je ne vis absolument rien. 

«Alors, je me mis à la recherche du coffre, selon la supposition que j'avais faite qu'il avait été jeté dans quelque coin. Ce devait être sur le chemin qui conduisait à la petite porte de sortie; mais cette nouvelle investigation fut aussi inutile que la première, et, le cœur serré, je revins au massif, qui lui-même ne me laissait plus aucun espoir. 

—Oh! s'écria Mme Danglars, il y avait de quoi devenir fou. 

—Je l'espérai un instant, dit Villefort, mais je n'eus pas ce bonheur; cependant, rappelant ma force et par conséquent mes idées: Pourquoi cet homme aurait-il emporté ce cadavre? me demandai-je. 

—Mais vous l'avez dit, reprit Mme Danglars, pour avoir une preuve. 

—Eh! non, madame, ce ne pouvait plus être cela; on ne garde pas un cadavre pendant un an, on le montre à un magistrat, et l'on fait sa déposition. Or, rien de tout cela n'était arrivé. 

—Eh bien, alors?\dots demanda Hermine toute palpitante. 

—Alors, il y a quelque chose de plus terrible, de plus fatal, de plus effrayant pour nous: il y a que l'enfant était vivant peut-être, et que l'assassin l'a sauvé.» 

Mme Danglars poussa un cri terrible, et saisissant les mains de Villefort: 

«Mon enfant était vivant! dit-elle; vous avez enterré mon enfant vivant, monsieur! Vous n'étiez pas sûr que mon enfant était mort, et vous l'avez enterré! ah!\dots» 

Mme Danglars s'était redressée et elle se tenait devant le procureur du roi, dont elle serrait les poignets entre ses mains délicates, debout et presque menaçante. 

«Que sais-je? Je vous dis cela comme je vous dirais autre chose», répondit Villefort avec une fixité de regard qui indiquait que cet homme si puissant était près d'atteindre les limites du désespoir et de la folie. 

«Ah! mon enfant, mon pauvre enfant!» s'écria la baronne, retombant sur sa chaise et étouffant ses sanglots dans son mouchoir. 

Villefort revint à lui, et comprit que pour détourner l'orage maternel qui s'amassait sur sa tête, il fallait faire passer chez Mme Danglars la terreur qu'il éprouvait lui-même. 

«Vous comprenez alors que si cela est ainsi, dit-il en se levant à son tour et en s'approchant de la baronne pour lui parler d'une voix plus basse, nous sommes perdus: cet enfant vit, et quelqu'un sait qu'il vit, quelqu'un a notre secret; et puisque Monte-Cristo parle devant nous d'un enfant déterré où cet enfant n'était plus, ce secret c'est lui qui l'a. 

—Dieu, Dieu juste, Dieu vengeur!» murmura Mme Danglars. 

Villefort ne répondit que par une espèce de rugissement. 

«Mais cet enfant, cet enfant, monsieur? reprit la mère obstinée. 

—Oh! que je l'ai cherché! reprit Villefort en se tordant les bras: que de fois je l'ai appelé dans mes longues nuits sans sommeil! que de fois j'ai désiré une richesse royale pour acheter un million de secrets à un million d'hommes, et pour trouver mon secret dans les leurs! Enfin, un jour que pour la centième fois je reprenais la bêche, je me demandai pour la centième fois ce que le Corse avait pu faire de l'enfant: un enfant embarrasse un fugitif; peut-être en s'apercevant qu'il était vivant encore, l'avait-il jeté dans la rivière. 

—Oh! impossible! s'écria Mme Danglars; on assassine un homme par vengeance, on ne noie pas de sang-froid un enfant! 

—Peut-être, continua Villefort, l'avait-il mis aux Enfants-Trouvés. 

—Oh! oui, oui! s'écria la baronne, mon enfant est là! monsieur! 

—Je courus à l'hospice, et j'appris que cette nuit même, la nuit du 20 septembre, un enfant avait été déposé dans le tour; il était enveloppé d'une moitié de serviette en toile fine, déchirée avec intention. Cette moitié de serviette portait une moitié de couronne de baron et la lettre H. 

—C'est cela, c'est cela! s'écria Mme Danglars, tout mon linge était marqué ainsi; M. de Nargonne était baron, et je m'appelle Hermine. Merci, mon Dieu! mon enfant n'était pas mort! 

—Non, il n'était pas mort! 

—Et vous me le dites! vous me dites cela sans craindre de me faire mourir de joie, monsieur! Où est-il? où est mon enfant?» 

Villefort haussa les épaules. 

«Le sais-je? dit-il; et croyez-vous que si je le savais je vous ferais passer par toutes ces gradations, comme le ferait un dramaturge ou un romancier? Non, hélas! non! je ne le sais pas. Une femme, il y avait six mois environ, était venue réclamer l'enfant avec l'autre moitié de la serviette. Cette femme avait fourni toutes les garanties que la loi exige, et on le lui avait remis. 

—Mais il fallait vous informer de cette femme, il fallait la découvrir. 

—Et de quoi pensez-vous donc que je me sois occupé, madame? J'ai feint une instruction criminelle, et tout ce que la police a de fins limiers, d'adroits agents, je les mis à sa recherche. On a retrouvé ses traces jusqu'à Châlons; à Châlons, on les a perdues. 

—Perdues? 

—Oui, perdues; perdues à jamais.» 

Mme Danglars avait écouté ce récit avec un soupir, une larme, un cri pour chaque circonstance. 

«Et c'est tout, dit-elle; et vous vous êtes borné là? 

—Oh! non, dit Villefort, je n'ai jamais cessé de chercher, de m'enquérir, de m'informer. Cependant, depuis deux ou trois ans, j'ai donné quelque relâche. Mais, aujourd'hui, je vais recommencer avec plus de persévérance et d'acharnement que jamais; et je réussirai, voyez-vous; car ce n'est plus la conscience qui me pousse, c'est la peur. 

—Mais, reprit Mme Danglars, le comte de Monte-Cristo ne sait rien; sans quoi, ce me semble, il ne nous rechercherait point comme il le fait. 

—Oh! la méchanceté des hommes est bien profonde, dit Villefort, puisqu'elle est plus profonde que la bonté de Dieu. Avez-vous remarqué les yeux de cet homme, tandis qu'il nous parlait? 

—Non. 

—Mais l'avez-vous examiné profondément parfois? 

—Sans doute. Il est bizarre, mais voilà tout. Une chose qui m'a frappée seulement, c'est que de tout ce repas exquis qu'il nous a donné, il n'a rien touché, c'est que d'aucun plat il n'a voulu prendre sa part. 

—Oui, oui! dit Villefort, j'ai remarqué cela aussi. Si j'avais su ce que je sais maintenant, moi non plus je n'eusse touché à rien; j'aurais cru qu'il voulait nous empoisonner. 

—Et vous vous seriez trompé, vous le voyez bien. 

—Oui, sans doute; mais, croyez-moi, cet homme a d'autres projets. Voilà pourquoi j'ai voulu vous voir, voilà pourquoi j'ai demandé à vous parler, voilà pourquoi j'ai voulu vous prémunir contre tout le monde, mais contre lui surtout. Dites-moi, continua Villefort en fixant plus profondément encore qu'il ne l'avait fait jusque-là ses yeux sur la baronne, vous n'avez parlé de notre liaison à personne? 

—Jamais, à personne. 

—Vous me comprenez, reprit affectueusement Villefort, quand je dis à personne, pardonnez-moi cette insistance, à personne au monde, n'est-ce pas? 

—Oh! oui, oui, je comprends très bien, dit la baronne en rougissant; jamais! je vous le jure. 

—Vous n'avez point l'habitude d'écrire le soir ce qui s'est passé dans la matinée? vous ne faites pas de journal? 

—Non! Hélas! ma vie passe emportée par la frivolité; moi-même, je l'oublie. 

—Vous ne rêvez pas haut, que vous sachiez? 

—J'ai un sommeil d'enfant; ne vous le rappelez-vous pas?» 

Le pourpre monta au visage de la baronne, et la pâleur envahit celui de Villefort. 

«C'est vrai, dit-il si bas qu'on l'entendit à peine. 

—Eh bien? demanda la baronne. 

—Eh bien, je comprends ce qu'il me reste à faire, reprit Villefort. Avant huit jours d'ici, je saurai ce que c'est que M. de Monte-Cristo, d'où il vient, où il va, et pourquoi il parle devant nous des enfants qu'on déterre dans son jardin.» 

Villefort prononça ces mots avec un accent qui eût fait frissonner le comte s'il eût pu les entendre. 

Puis il serra la main que la baronne répugnait à lui donner et la reconduisit avec respect jusqu'à la porte. 

Mme Danglars reprit un autre fiacre, qui la ramena au passage, de l'autre côté duquel elle retrouva sa voiture et son cocher, qui, en l'attendant, dormait paisiblement sur son siège. 