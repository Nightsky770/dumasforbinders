\chapter{Andrea Cavalcanti}

\lettrine{L}{e} comte de Monte-Cristo entra dans le salon voisin que Baptistin avait désigné sous le nom de salon bleu, et où venait de le précéder un jeune homme de tournure dégagée, assez élégamment vêtu, et qu'un cabriolet de place avait, une demi-heure auparavant, jeté à la porte de l'hôtel. Baptistin n'avait pas eu de peine à le reconnaître; c'était bien ce grand jeune homme aux cheveux blonds, à la barbe rousse, aux yeux noirs, dont le teint vermeil et la peau éblouissante de blancheur lui avaient été signalés par son maître. 

Quand le comte entra dans le salon, le jeune homme était négligemment étendu sur un sofa, fouettant avec distraction sa botte d'un petit jonc à pomme d'or. 

En apercevant Monte-Cristo, il se leva vivement. 

«Monsieur est le comte de Monte-Cristo? dit-il. 

—Oui, monsieur, répondit celui-ci, et j'ai l'honneur de parler, je crois, à monsieur le vicomte Andrea Cavalcanti? 

—Le vicomte Andrea Cavalcanti, répéta le jeune homme en accompagnant ces mots d'un salut plein de désinvolture. 

—Vous devez avoir une lettre qui vous accrédite près de moi? dit Monte-Cristo. 

—Je ne vous en parlais pas à cause de la signature, qui m'a paru étrange. 

—Simbad le marin, n'est-ce pas? 

—Justement. Or, comme je n'ai jamais connu d'autre Simbad le marin que celui des \textit{Mille et une Nuits}\dots. 

—Eh bien, c'est un de ses descendants, un de mes amis fort riche, un Anglais plus qu'original, presque fou, dont le véritable nom est Lord Wilmore. 

—Ah! voilà qui m'explique tout, dit Andrea. Alors cela va à merveille. C'est ce même Anglais que j'ai connu\dots à\dots oui, très bien!\dots Monsieur le comte, je suis votre serviteur. 

—Si ce que vous me faites l'honneur de me dire est vrai, répliqua en souriant le comte, j'espère que vous serez assez bon pour me donner quelques détails sur vous et votre famille. 

—Volontiers, monsieur le comte, répondit le jeune homme avec une volubilité qui prouvait la solidité de sa mémoire. Je suis, comme vous l'avez dit, le vicomte Andrea Cavalcanti, fils du major Bartolomeo Cavalcanti descendant des Cavalcanti inscrits au livre d'or de Florence. Notre famille, quoique très riche encore puisque mon père possède un demi-million de rente, a éprouvé bien des malheurs, et moi-même, monsieur, j'ai été à l'âge de cinq ou six ans enlevé par un gouverneur infidèle; de sorte que depuis quinze ans je n'ai point revu l'auteur de mes jours. Depuis que j'ai l'âge de raison, depuis que je suis libre et maître de moi, je le cherche, mais inutilement. Enfin cette lettre de votre ami Simbad m'annonce qu'il est à Paris, et m'autorise à m'adresser à vous pour en obtenir des nouvelles. 

—En vérité, monsieur, tout ce que vous me racontez là est fort intéressant, dit le comte, regardant avec une sombre satisfaction cette mine dégagée, empreinte d'une beauté pareille à celle du mauvais ange, et vous avez fort bien fait de vos conformer en toutes choses à l'invitation de mon ami Simbad, car votre père est en effet ici et vous cherche.» 

Le comte, depuis son entrée au salon, n'avait pas perdu de vue le jeune homme, il avait admiré l'assurance de son regard et la sûreté de sa voix; mais à ces mots si naturels: \textit{Votre père est en effet ici et vous cherche}, le jeune Andrea fit un bond et s'écria: 

«Mon père! mon père ici? 

—Sans doute, répondit Monte-Cristo, votre père, le major Bartolomeo Cavalcanti.» 

L'impression de terreur répandue sur les traits du jeune homme s'effaça presque aussitôt. 

«Ah! oui, c'est vrai, dit-il, le major Bartolomeo Cavalcanti. Et vous dites, monsieur le comte, qu'il est ici, ce cher père. 

—Oui, monsieur. J'ajouterai même que je le quitte à l'instant, que l'histoire qu'il m'a contée de ce fils chéri, perdu autrefois, m'a fort touché; en vérité, ses douleurs, ses craintes, ses espérances à ce sujet composeraient un poème attendrissant. Enfin il reçut un jour des nouvelles qui lui annonçaient que les ravisseurs de son fils offraient de le rendre, ou d'indiquer où il était, moyennant une somme assez forte. Mais rien ne retint ce bon père; cette somme fut envoyée à la frontière du Piémont, avec un passeport tout visé pour l'Italie. Vous étiez dans le Midi de la France, je crois? 

—Oui, monsieur, répondit Andrea d'un air assez embarrassé; oui, j'étais dans le Midi de la France. 

—Une voiture devait vous attendre à Nice? 

—C'est bien cela, monsieur; elle m'a conduit de Nice à Gênes, de Gênes à Turin, de Turin à Chambéry, de Chambéry à Pont-de-Beauvoisin, et de Pont-de-Beauvoisin à Paris. 

—À merveille! il espérait toujours vous rencontrer en chemin, car c'était la route qu'il suivait lui-même; voilà pourquoi votre itinéraire avait été tracé ainsi. 

—Mais, dit Andrea, s'il m'eût rencontré, ce cher père, je doute qu'il m'eût reconnu; je suis quelque peu changé depuis que je l'ai perdu de vue. 

—Oh! la voix du sang, dit Monte-Cristo. 

—Ah! oui, c'est vrai, reprit le jeune homme, je n'y songeais pas à la voix du sang. 

—Maintenant, reprit Monte-Cristo, une seule chose inquiète le marquis Cavalcanti, c'est ce que vous avez fait pendant que vous avez été éloigné de lui; c'est de quelle façon vous avez été traité par vos persécuteurs; c'est si l'on a conservé pour votre naissance tous les égards qui lui étaient dus; c'est enfin s'il ne vous est pas resté de cette souffrance morale à laquelle vous avez été exposé, souffrance pire cent fois que la souffrance physique, quelque affaiblissement des facultés dont la nature vous a si largement doué, et si vous croyez vous-même pouvoir reprendre et soutenir dignement dans le monde le rang qui vous appartient. 

—Monsieur, balbutia le jeune homme étourdi, j'espère qu'aucun faux rapport\dots. 

—Moi! J'ai entendu parler de vous pour la première fois par mon ami Wilmore, le philanthrope. J'ai su qu'il vous avait trouvé dans une position fâcheuse, j'ignore laquelle, et ne lui ai fait aucune question: je ne suis pas curieux. Vos malheurs l'ont intéressé, donc vous étiez intéressant. Il m'a dit qu'il voulait vous rendre dans le monde la position que vous aviez perdue, qu'il chercherait votre père, qu'il le trouverait; l'a cherché, il l'a trouvé, à ce qu'il paraît, puisqu'il est là; enfin il m'a prévenu hier de votre arrivée, en me donnant encore quelques autres instructions relatives à votre fortune; voilà tout. Je sais que c'est un original, mon ami Wilmore, mais en même temps, comme c'est un homme sûr, riche comme une mine d'or, qui, par conséquent, peut se passer ses originalités sans qu'elles le ruinent, j'ai promis de suivre ses instructions. Maintenant, monsieur, ne vous blessez pas de ma question: comme je serai obligé de vous patronner quelque peu, je désirerais savoir si les malheurs qui vous sont arrivés, malheurs indépendants de votre volonté et qui ne diminuent en aucune façon la considération que je vous porte, ne vous ont pas rendu quelque peu étranger à ce monde dans lequel votre fortune et votre nom vous appelaient à faire si bonne figure. 

—Monsieur, répondit le jeune homme reprenant son aplomb au fur et à mesure que le comte parlait, rassurez-vous sur ce point: les ravisseurs qui m'ont éloigné de mon père, et qui, sans doute, avaient pour but de me vendre plus tard à lui comme ils l'ont fait ont calculé que, pour tirer un bon parti de moi, il fallait me laisser toute ma valeur personnelle, et même l'augmenter encore, s'il était possible; j'ai donc reçu une assez bonne éducation, et j'ai été traité par les larrons d'enfants à peu près comme l'étaient dans l'Asie Mineure les esclaves dont leurs maîtres faisaient des grammairiens, des médecins et des philosophes, pour les vendre plus cher au marché de Rome.» 

Monte-Cristo sourit avec satisfaction; il n'avait pas tant espéré, à ce qu'il paraît, de M. Andrea Cavalcanti. 

«D'ailleurs, reprit le jeune homme, s'il y avait en moi quelque défaut d'éducation ou plutôt d'habitude du monde, on aurait, je suppose, l'indulgence de les excuser, en considération des malheurs qui ont accompagné ma naissance et poursuivi ma jeunesse. 

—Eh bien, dit négligemment Monte-Cristo, vous en ferez ce que vous voudrez, vicomte, car vous êtes le maître, et cela vous regarde; mais, ma parole, au contraire, je ne dirais pas un mot de toutes ces aventures, c'est un roman que votre histoire, et le monde, qui adore les romans serrés entre deux couvertures de papier jaune, se défie étrangement de ceux qu'il voit reliés en vélin vivant, fussent-ils dorés comme vous pouvez l'être. Voilà la difficulté que je me permettrai de vous signaler, monsieur le vicomte; à peine aurez-vous raconté à quelqu'un votre touchante histoire, qu'elle courra dans le monde complètement dénaturée. Vous serez obligé de vous poser en Antony, et le temps des Antony est un peu passé. Peut-être aurez-vous un succès de curiosité, mais tout le monde n'aime pas à se faire centre d'observations et cible à commentaires. Cela vous fatiguera peut-être. 

—Je crois que vous avez raison, monsieur le comte, dit le jeune homme en pâlissant malgré lui, sous l'inflexible regard de Monte-Cristo; c'est là un grave inconvénient. 

—Oh! il ne faut pas non plus se l'exagérer, dit Monte-Cristo; car, pour éviter une faute, on tomberait dans une folie. Non, c'est un simple plan de conduite à arrêter; et, pour un homme intelligent comme vous, ce plan est d'autant plus facile à adopter qu'il est conforme à vos intérêts; il faudra combattre, par des témoignages et par d'honorables amitiés, tout ce que votre passé peut avoir d'obscur.» 

Andrea perdit visiblement contenance. 

«Je m'offrirais bien à vous comme répondant et caution, dit Monte-Cristo; mais c'est chez moi une habitude morale de douter de mes meilleurs amis, et un besoin de chercher à faire douter les autres; aussi jouerais-je là un rôle hors de mon emploi, comme disent les tragédiens, et je risquerais de me faire siffler, ce qui est inutile. 

—Cependant, monsieur le comte, dit Andrea avec audace, en considération de Lord Wilmore qui m'a recommandé à vous\dots. 

—Oui, certainement, reprit Monte-Cristo; mais Lord Wilmore ne m'a pas laissé ignorer, cher monsieur Andrea, que vous aviez eu une jeunesse quelque peu orageuse. Oh! dit le comte en voyant le mouvement que faisait Andrea, je ne vous demande pas de confession; d'ailleurs, c'est pour que vous n'ayez besoin de personne que l'on a fait venir de Lucques M. le marquis Cavalcanti, votre père. Vous allez le voir, il est un peu raide, un peu guindé; mais c'est une question d'uniforme, et quand on saura que depuis dix-huit ans il est au service de l'Autriche, tout s'excusera; nous ne sommes pas, en général, exigeants pour les Autrichiens. En somme, c'est un père fort suffisant, je vous assure. 

—Ah! vous me rassurez, monsieur; je l'avais quitté depuis si longtemps, que je n'avais de lui aucun souvenir. 

—Et puis, vous savez, une grande fortune fait passer sur bien des choses. 

—Mon père est donc réellement riche, monsieur? 

—Millionnaire\dots cinq cent mille livres de rente. 

—Alors, demanda le jeune homme avec anxiété, je vais me trouver dans une position\dots agréable? 

—Des plus agréables, mon cher monsieur; il vous fait cinquante mille livres de rente par an pendant tout le temps que vous resterez à Paris. 

—Mais j'y resterai toujours, en ce cas. 

—Heu! qui peut répondre des circonstances, mon cher monsieur? l'homme propose et Dieu dispose\dots.» 

Andrea poussa un soupir. 

«Mais enfin, dit-il, tout le temps que je resterai à Paris, et\dots qu'aucune circonstance ne me forcera pas de m'éloigner, cet argent dont vous me parliez tout à l'heure m'est-il assuré? 

—Oh! parfaitement. 

—Par mon père? demanda Andrea avec inquiétude. 

—Oui, mais garanti par Lord Wilmore, qui vous a, sur la demande de votre père, ouvert un crédit de cinq mille francs par mois chez M. Danglars, un des plus sûrs banquiers de Paris. 

—Et mon père compte rester longtemps à Paris? demanda Andrea avec inquiétude. 

—Quelque jours seulement, répondit Monte-Cristo, son service ne lui permet pas de s'absenter plus de deux ou trois semaines. 

—Oh! ce cher père! dit Andrea visiblement enchanté de ce prompt départ. 

—Aussi, dit Monte-Cristo, faisant semblant de se tromper à l'accent de ces paroles; aussi je ne veux pas retarder d'un instant l'heure de votre réunion. Êtes-vous préparé à embrasser ce digne M. Cavalcanti? 

—Vous n'en doutez pas, je l'espère? 

—Eh bien, entrez donc dans le salon, mon cher ami, et vous trouverez votre père, qui vous attend.» 

Andrea fit un profond salut au comte et entra dans le salon. 

Le comte le suivit des yeux, et, l'ayant vu disparaître, poussa un ressort correspondant à un tableau, lequel, en s'écartant du cadre, laissait, par un interstice habilement ménagé, pénétrer la vue dans le salon. 

Andrea referma la porte derrière lui et s'avança vers le major, qui se leva dès qu'il entendit le bruit des pas qui s'approchaient. 

«Ah! monsieur et cher père, dit Andrea à haute voix et de manière que le comte l'entendit à travers la porte fermée, est-ce bien vous? 

—Bonjour, mon cher fils, fit gravement le major. 

—Après tant d'années de séparation, dit Andrea en continuant de regarder du côté de la porte, quel bonheur de nous revoir! 

—En effet, la séparation a été longue. 

—Ne nous embrassons-nous pas, monsieur? reprit Andrea. 

—Comme vous voudrez, mon fils», dit le major. 

Et les deux hommes s'embrassèrent comme on s'embrasse au Théâtre-Français, c'est-à-dire en se passant la tête par-dessus l'épaule.  

«Ainsi donc nous voici réunis! dit Andrea. 

—Nous voici réunis, reprit le major. 

—Pour ne plus nous séparer? 

—Si fait; je crois, mon cher fils, que vous regardez maintenant la France comme une seconde patrie? 

—Le fait est, dit le jeune homme, que je serais désespéré de quitter Paris. 

—Et moi, vous comprenez, je ne saurais vivre hors de Lucques. Je retournerai donc en Italie aussitôt que je pourrai. 

—Mais avant de partir, très cher père, vous me remettrez sans doute des papiers à l'aide desquels il me sera facile de constater le sang dont je sors. 

—Sans aucun doute, car je viens exprès pour cela, et j'ai eu trop de peine à vous rencontrer, afin de vous les remettre, pour que nous recommencions encore à nous chercher; cela prendrait la dernière partie de ma vie. 

—Et ces papiers? 

—Les voici.» 

Andrea saisit avidement l'acte de mariage de son père, son certificat de baptême à lui, et, après avoir ouvert le tout avec une avidité naturelle à un bon fils, il parcourut les deux pièces avec une rapidité et une habitude qui dénotaient le coup d'œil le plus exercé en même temps que l'intérêt le plus vif. 

Lorsqu'il eut fini, une indéfinissable expression de joie brilla sur son front; et regardant le major avec un étrange sourire: 

«Ah çà! dit-il en excellent toscan, il n'y a donc pas de galère en Italie?\dots» 

Le major se redressa. 

«Et pourquoi cela? dit-il. 

—Qu'on y fabrique impunément de pareilles pièces? Pour la moitié de cela, mon très cher père, en France on nous enverrait prendre l'air à Toulon pour cinq ans. 

—Plaît-il? dit le Lucquois en essayant de conquérir un air majestueux. 

—Mon cher monsieur Cavalcanti, dit Andrea en pressant le bras du major, combien vous donne-t-on pour être mon père?» 

Le major voulut parler. 

«Chut! dit Andrea en baissant la voix, je vais vous donner l'exemple de la confiance; on me donne cinquante mille francs par an pour être votre fils: par conséquent, vous comprenez bien que ce n'est pas moi qui serai disposé à nier que vous soyez mon père.» 

Le major regarda avec inquiétude autour de lui. 

«Eh! soyez tranquille, nous sommes seuls, dit Andrea, d'ailleurs nous parlons italien. 

—Eh bien, à moi, dit le Lucquois, on me donne cinquante mille francs une fois payés. 

—Monsieur Cavalcanti, dit Andrea, avez-vous foi aux contes de fées? 

—Non, pas autrefois, mais maintenant il faut bien que j'y croie. 

—Vous avez donc eu des preuves?» 

Le major tira de son gousset une poignée d'or. 

«Palpables, comme vous voyez. 

—Vous pensez donc que je puis croire aux promesses qu'on m'a faites? 

—Je le crois. 

—Et que ce brave homme de comte les tiendra? 

—De point en point; mais, vous comprenez, pour arriver à ce but, il faut jouer notre rôle. 

—Comment donc?\dots 

—Moi de tendre père\dots. 

—Moi, de fils respectueux. 

—Puisqu'ils désirent que vous descendiez de moi\dots. 

—Qui, \textit{ils}? 

—Dame, je n'en sais rien, ceux qui vous ont écrit; n'avez vous pas reçu une lettre? 

—Si fait. 

—De qui? 

—D'un certain abbé Busoni. 

—Que vous ne connaissez pas? 

—Que je n'ai jamais vu. 

—Que vous disait cette lettre? 

—Vous ne me trahirez pas? 

—Je m'en garderai bien, nos intérêts sont les mêmes.  

—Alors lisez.» 

Et le major passa une lettre au jeune homme. 

Andrea lut à voix basse: 

\begin{mail}{}{}
Vous êtes pauvre, une vieillesse malheureuse vous attend. Voulez-vous devenir sinon riche, du moins indépendant? 

Partez pour Paris à l'instant même, et allez réclamer à M. le comte de Monte-Cristo, avenue des Champs-Élysées, n°30, le fils que vous avez eu de la marquise de Corsinari, et qui vous a été enlevé à l'âge de cinq ans. 

Ce fils se nomme Andrea Cavalcanti. 

Pour que vous ne révoquiez pas en doute l'attention qu'a le soussigné de vous être agréable, vous trouverez ci-joint: 

\begin{enumerate}
\item Un bon de deux mille quatre cents livres toscanes, payable chez M. Gozzi, à Florence; 

\item Une lettre d'introduction près de M. le comte de Monte-Cristo sur lequel je vous crédite d'une somme de quarante-huit mille francs. 
\end{enumerate}

Soyez chez le comte le 26 mai, à sept heures du soir. 

\closeletter[Signé]{Abbé Busoni.}

\end{mail}

—C'est cela. 

—Comment, c'est cela? Que voulez-vous dire? demanda le major. 

—Je dis que j'ai reçu la pareille à peu près. 

—Vous? 

—Oui, moi. 

—De l'abbé Busoni? 

—Non. 

—De qui donc? 

—D'un Anglais, d'un certain Lord Wilmore, qui prend le nom de Simbad le marin. 

—Et que vous ne connaissez pas plus que je ne connais l'abbé Busoni? 

—Si fait; moi, je suis plus avancé que vous. 

—Vous l'avez vu? 

—Oui, une fois. 

—Où cela? 

—Ah! justement voici ce que je ne puis pas vous dire; vous seriez aussi savant que moi, et c'est inutile. 

—Et cette lettre vous disait?\dots 

—Lisez.» 

\begin{mail}{}{}
Vous êtes pauvre, et vous n'avez qu'un avenir misérable: voulez-vous avoir un nom, être libre, être riche?
\pausemail
—Parbleu! fit le jeune homme en se balançant sur ses talons, comme si une pareille question se faisait! 
\resumemail

Prenez la chaise de poste que vous trouverez tout attelée en sortant de Nice par la porte de Gênes. Passez par Turin, Chambéry et Pont-de-Beauvoisin. Présentez-vous chez M. le comte de Monte-Cristo, avenue des Champs-Élysées, le 26 mai, à sept heures du soir, et demandez-lui votre père. 

Vous êtes le fils du marquis Bartolomeo Cavalcanti et de la marquise Olivia Corsinari, ainsi que le constateront les papiers qui vous seront remis par le marquis, et qui vous permettront de vous présenter sous ce nom dans le monde parisien. 

Quant à votre rang, un revenu de cinquante mille livres par an vous mettra à même de le soutenir. 

Ci-joint un bon de cinq mille livres payable sur M. Ferrea, banquier à Nice, et une lettre d'introduction près du comte de Monte-Cristo, chargé par moi de pourvoir à vos besoins. 
\closeletter{Simbad le Marin.}
\end{mail}

«Hum! fit le major, c'est fort beau! 

—N'est-ce pas? 

—Vous avez vu le comte? 

—Je le quitte. 

—Et il a ratifié? 

—Tout. 

—Y comprenez-vous quelque chose? 

—Ma foi, non. 

—Il y a une dupe dans tout cela. 

—En tout cas, ce n'est ni vous ni moi? 

—Non, certainement. 

—Et bien, alors!\dots 

—Peu nous importe, n'est-ce pas?  

—Justement, c'est ce que je voulais dire, allons jusqu'au bout et jouons serré. 

—Soit; vous verrez que je suis digne de faire votre partie. 

—Je n'en ai pas douté un seul instant, mon cher père. 

—Vous me faites honneur, mon cher fils.» 

Monte-Cristo choisit ce moment pour rentrer dans le salon. En entendant le bruit de ses pas, les deux hommes se jetèrent dans les bras l'un de l'autre; le comte les trouva embrassés. 

«Eh bien! monsieur le marquis, dit Monte-Cristo, il paraît que vous avez retrouvé un fils selon votre cœur? 

—Ah! monsieur le comte, je suffoque de joie. 

—Et vous, jeune homme? 

—Ah! monsieur le comte, j'étouffe de bonheur. 

—Heureux père! heureux enfant! dit le comte. 

—Une seule chose m'attriste, dit le major; c'est la nécessité où je suis de quitter Paris si vite. 

—Oh! cher monsieur Cavalcanti, dit Monte-Cristo, vous ne partirez pas, je l'espère, que je ne vous aie présenté à quelques amis. 

—Je suis aux ordres de monsieur le comte, dit le major. 

—Maintenant, voyons, jeune homme, confessez-vous. 

—À qui? 

—Mais à monsieur votre père; dites-lui quelques mots de l'état de vos finances. 

—Ah! diable, fit Andrea, vous touchez la corde sensible. 

—Entendez-vous, major? dit Monte-Cristo. 

—Sans doute que je l'entends. 

—Oui, mais comprenez-vous? 

—À merveille. 

—Il dit qu'il a besoin d'argent, ce cher enfant. 

—Que voulez-vous que j'y fasse? 

—Que vous lui en donniez, parbleu! 

—Moi? 

—Oui, vous.» 

Monte-Cristo passa entre les deux hommes. 

«Tenez! dit-il à Andrea en lui glissant un paquet de billets de banque à la main. 

—Qu'est-ce que cela? 

—La réponse de votre père. 

—De mon père? 

—Oui. Ne venez-vous pas de laisser entendre que vous aviez besoin d'argent? 

—Oui. Eh bien? 

—Eh bien! il me charge de vous remettre cela. 

—A compte sur mes revenus? 

—Non, pour vos frais d'installation. 

—Oh! cher père! 

—Silence, dit Monte-Cristo, vous voyez bien qu'il ne veut pas que je dise que cela vient de lui. 

—J'apprécie cette délicatesse, dit Andrea, en enfonçant ses billets de banque dans le gousset de son pantalon. 

—C'est bien, dit Monte-Cristo, maintenant, allez! 

—Et quand aurons-nous l'honneur de revoir M. le comte? demanda Cavalcanti. 

—Ah! oui, demanda Andrea, quand aurons-nous cet honneur? 

—Samedi, si vous voulez\dots oui\dots tenez\dots samedi. J'ai à dîner à ma maison d'Auteuil, rue de la Fontaine, n°28, plusieurs personnes, et entre autres M. Danglars, votre banquier, je vous présenterai à lui, il faut bien qu'il vous connaisse tous les deux pour vous compter votre argent. 

—Grande tenue? demanda à demi-voix le major. 

—Grande tenue: uniforme, croix, culotte courte. 

—Et moi? demanda Andrea. 

—Oh! vous, très simplement: pantalon noir, bottes vernies, gilet blanc, habit noir ou bleu, cravate longue; prenez Blin ou Véronique pour vous habiller. Si vous ne connaissez pas leurs adresses, Baptistin vous les donnera. Moins vous affecterez de prétention dans votre mise, étant riche comme vous l'êtes, meilleur effet cela fera. Si vous achetez des chevaux, prenez-les chez Devedeux; si vous achetez un phaéton, allez chez Baptiste. 

—À quelle heure pourrons-nous nous présenter? demanda le jeune homme. 

—Mais vers six heures et demie. 

—C'est bien, on y sera», dit le major en portant la main à son chapeau. 

Les deux Cavalcanti saluèrent le comte et sortirent. Le comte s'approcha de la fenêtre, et les vit qui traversaient la cour bras dessus, bras dessous. 

«En vérité, dit-il, voilà deux grands misérables! Quel malheur que ce ne soit pas véritablement le père et le fils!» 

Puis après un instant de sombre réflexion: 

«Allons chez les Morrel, dit-il; je crois que le dégoût m'écœure encore plus que la haine.» 