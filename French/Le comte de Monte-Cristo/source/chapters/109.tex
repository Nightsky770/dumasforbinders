\chapter{Les assises} 

\lettrine{L}{'affaire} Benedetto, comme on disait alors au Palais et dans le monde, avait produit une énorme sensation. Habitué du Café de Paris, du boulevard de Gand et du Bois de Boulogne, le faux Cavalcanti, pendant qu'il était resté à Paris et pendant les deux ou trois mois qu'avait duré sa splendeur, avait fait une foule de connaissances. Les journaux avaient raconté les diverses stations du prévenu dans sa vie élégante et dans sa vie de bagne; il en résultait la plus vive curiosité chez ceux-là surtout qui avaient personnellement connu le prince Andrea Cavalcanti; aussi ceux-là surtout étaient-ils décidés à tout risquer pour aller voir sur le banc des accusés M. Benedetto, l'assassin de son camarade de chaîne. 

Pour beaucoup de gens, Benedetto était, sinon une victime, du moins une erreur de la justice: on avait vu M. Cavalcanti père à Paris, et l'on s'attendait à le voir de nouveau apparaître pour réclamer son illustre rejeton. Bon nombre de personnes qui n'avaient jamais entendu parler de la fameuse polonaise avec laquelle il avait débarqué chez le comte de Monte-Cristo s'étaient senties frappées de l'air digne, de la gentilhommerie et de la science du monde qu'avait montrés le vieux patricien, lequel, il faut le dire, semblait un seigneur parfait toutes les fois qu'il ne parlait point et ne faisait point d'arithmétique. 

Quant à l'accusé lui-même, beaucoup de gens se rappelaient l'avoir vu si aimable, si beau, si prodigue, qu'ils aimaient mieux croire à quelque machination de la part d'un ennemi comme on en trouve en ce monde, où les grandes fortunes élèvent les moyens de faire le mal et le bien à la hauteur du merveilleux, et la puissance à la hauteur de l'inouï. 

Chacun accourut donc à la séance de la cour d'assises, les uns pour savourer le spectacle, les autres pour le commenter. Dès sept heures du matin on faisait queue à la grille, et une heure avant l'ouverture de la séance la salle était déjà pleine de privilégiés. 

Avant l'entrée de la cour, et même souvent après, une salle d'audience, les jours de grands procès, ressemble fort à un salon où beaucoup de gens se reconnaissent, s'abordent quand ils sont assez près les uns des autres pour ne pas perdre leurs places, se font des signes quand ils sont séparés par un trop grand nombre de populaire, d'avocats et de gendarmes. 

Il faisait une de ces magnifiques journées d'automne qui nous dédommagent parfois d'un été absent ou écourté; les nuages que M. de Villefort avait vus le matin rayer le soleil levant s'étaient dissipés comme par magie, et laissaient luire dans toute sa pureté un des derniers, un des plus doux jours de septembre. 

Beauchamp, un des rois de la presse, et par conséquent ayant son trône partout, lorgnait à droite et à gauche. Il aperçut Château-Renaud et Debray qui venaient de gagner les bonnes grâces d'un sergent de ville, et qui l'avaient décidé à se mettre derrière eux au lieu de les masquer, comme c'était son droit. Le digne agent avait flairé le secrétaire du ministre et le millionnaire; il se montra plein d'égards pour ses nobles voisins et leur permit même d'aller rendre visite à Beauchamp, en leur promettant de leur garder leurs places. 

«Eh bien, dit Beauchamp, nous venons donc voir notre ami? 

—Eh! mon Dieu, oui, répondit Debray: ce digne prince! Que le diable soit des princes italiens, va! 

—Un homme qui avait eu Dante pour généalogiste, et qui remontait à \textit{La Divine Comédie}! 

—Noblesse de corde, dit flegmatiquement Château-Renaud. 

—Il sera condamné, n'est-ce pas? demanda Debray à Beauchamp. 

—Eh! mon cher, répondit le journaliste, c'est à vous, ce me semble, qu'il faut demander cela: vous connaissez mieux que nous autres l'air du bureau; avez-vous vu le président à la dernière soirée de votre ministre? 

—Oui. 

—Que vous a-t-il dit? 

—Une chose qui va vous étonner. 

—Ah! parlez donc vite, alors, cher ami, il y a si longtemps qu'on ne me dit plus rien de ce genre-là. 

—Eh bien, il m'a dit que Benedetto, qu'on regarde comme un phénix de subtilité, comme un géant d'astuce, n'est qu'un filou très subalterne, très niais, et tout à fait indigne des expériences qu'on fera après sa mort sur ses organes phrénologiques. 

—Bah! fit Beauchamp; il jouait cependant très passablement le prince. 

—Pour vous, Beauchamp, qui les détestez, ces malheureux princes et qui êtes enchanté de leur trouver de mauvaises façons, mais pas pour moi, qui flaire d'instinct le gentilhomme et qui lève une famille aristocratique, quelle qu'elle soit, en vrai limier du blason. 

—Ainsi, vous n'avez jamais cru à sa principauté? 

—À sa principauté? si\dots à son principat? non. 

—Pas mal, dit Debray; je vous assure cependant que pour tout autre que vous il pouvait passer\dots Je l'ai vu chez les ministres. 

—Ah! oui, dit Château-Renaud; avec cela que vos ministres se connaissent en princes! 

—Il y a du bon dans ce que vous venez de dire, Château-Renaud, répondit Beauchamp en éclatant de rire; la phrase est courte, mais agréable. Je vous demande la permission d'en user dans mon compte rendu. 

—Prenez, mon cher monsieur Beauchamp, dit Château-Renaud; prenez; je vous donne ma phrase pour ce qu'elle vaut. 

—Mais, dit Debray à Beauchamp, si j'ai parlé au président, vous avez dû parler au procureur du roi, vous? 

—Impossible; depuis huit jours M. de Villefort se cèle; c'est tout naturel: cette suite étrange de chagrins domestiques couronnée par la mort étrange de sa fille\dots 

—La mort étrange! Que dites-vous donc là, Beauchamp? 

—Oh! oui, faites donc l'ignorant, sous prétexte que tout cela se passe chez la noblesse de robe, dit Beauchamp en appliquant son lorgnon à son œil et en le forçant de tenir tout seul. 

—Mon cher monsieur, dit Château-Renaud, permettez-moi de vous dire que, pour le lorgnon, vous n'êtes pas de la force de Debray. Debray, donnez donc une leçon à M. Beauchamp. 

—Tiens, dit Beauchamp, je ne me trompe pas. 

—Quoi donc? 

—C'est elle. 

—Qui, elle? 

—On la disait partie. 

—Mlle Eugénie? demanda Château-Renaud; serait-elle déjà revenue? 

—Non, mais sa mère. 

—Mme Danglars? 

—Allons donc! fit Château-Renaud, impossible; dix jours après la fuite de sa fille, trois jours après la banqueroute de son mari!» 

Debray rougit légèrement et suivit la direction du regard de Beauchamp. 

«Allons donc! dit-il, c'est une femme voilée, une dame inconnue, quelque princesse étrangère, la mère du prince Cavalcanti peut-être; mais vous disiez, ou plutôt vous alliez dire des choses fort intéressantes, Beauchamp, ce me semble. 

—Moi? 

—Oui. Vous parliez de la mort étrange de Valentine. 

—Ah! oui, c'est vrai; mais pourquoi donc Mme de Villefort, n'est-elle pas ici? 

—Pauvre chère femme! dit Debray, elle est sans doute occupée à distiller de l'eau de mélisse pour les hôpitaux, et à composer des cosmétiques pour elle et pour ses amies. Vous savez qu'elle dépense à cet amusement deux ou trois mille écus par an, à ce que l'on assure. Au fait, vous avez raison, pourquoi n'est-elle pas ici, Mme de Villefort? Je l'aurais vue avec un grand plaisir; j'aime beaucoup cette femme. 

—Et moi, dit Château-Renaud, je la déteste. 

—Pourquoi? 

—Je n'en sais rien. Pourquoi aime-t-on? pourquoi déteste-t-on? Je la déteste par antipathie. 

—Ou par instinct, toujours. 

—Peut-être\dots Mais revenons à ce que vous disiez, Beauchamp. 

—Eh bien, reprit Beauchamp, n'êtes-vous pas curieux de savoir, messieurs, pourquoi l'on meurt si dru dans la maison Villefort? 

—Dru est joli, dit Château-Renaud. 

—Mon cher, le mot se trouve dans Saint-Simon. 

—Mais la chose se trouve chez M. de Villefort; allons-y donc. 

—Ma foi! dit Debray, j'avoue que je ne perds pas de vue cette maison tendue de deuil depuis trois mois et avant-hier encore, à propos de Valentine, madame m'en parlait. 

—Qu'est-ce que madame?\dots demanda Château-Renaud. 

—La femme du ministre, pardieu! 

—Ah! pardon, fit Château-Renaud, je ne vais pas chez les ministres, moi, je laisse cela aux princes. 

—Vous n'étiez que beau, vous devenez flamboyant, baron; prenez pitié de vous, ou vous allez nous brûler comme un autre Jupiter. 

—Je ne dirai plus rien, dit Château-Renaud; mais que diable, ayez pitié de moi, ne me donnez pas la réplique. 

—Voyons, tâchons d'arriver au bout de notre dialogue, Beauchamp; je vous disais donc que madame me demandait avant-hier des renseignements là-dessus; instruisez-moi, je l'instruirai. 

—Eh bien, messieurs, si l'on meurt si dru, je maintiens le mot, dans la maison Villefort, c'est qu'il y a un assassin dans la maison!» 

Les deux jeunes gens tressaillirent, car déjà plus d'une fois la même idée leur était venue. 

«Et quel est cet assassin? demandèrent-ils. 

—Le jeune Édouard.» 

Un éclat de rire des deux auditeurs ne déconcerta aucunement l'orateur, qui continua: 

«Oui, messieurs, le jeune Édouard, enfant phénoménal, qui tue déjà comme père et mère. 

—C'est une plaisanterie? 

—Pas du tout; j'ai pris hier un domestique qui sort de chez M. de Villefort: écoutez bien ceci. 

—Nous écoutons. 

—Et que je vais renvoyer demain, parce qu'il mange énormément pour se remettre du jeûne de terreur qu'il s'imposait là-bas. Eh bien, il parait que ce cher enfant a mis la main sur quelque flacon de drogue dont il use de temps en temps contre ceux qui lui déplaisent. D'abord ce fut bon papa et bonne maman de Saint-Méran qui lui déplurent, et il leur a versé trois gouttes de son élixir: trois gouttes suffisent; puis ce fut le brave Barrois, vieux serviteur de bon papa Noirtier, lequel rudoyait de temps en temps l'aimable espiègle que vous connaissez. L'aimable espiègle lui a versé trois gouttes de son élixir. Ainsi fut fait de la pauvre Valentine, qui ne le rudoyait pas, elle, mais dont il était jaloux: il lui a versé trois gouttes de son élixir, et pour elle comme pour les autres tout a été fini. 

—Mais quel diable de conte nous faites-vous là? dit Château-Renaud. 

—Oui, dit Beauchamp, un conte de l'autre monde, n'est-ce pas? 

—C'est absurde, dit Debray. 

—Ah! reprit Beauchamp, voilà déjà que vous cherchez des moyens dilatoires! Que diable! demandez à mon domestique, ou plutôt à celui qui demain ne sera plus mon domestique: c'était le bruit de la maison. 

—Mais cet élixir, où est-il? quel est-il? 

—Dame! l'enfant le cache. 

—Où l'a-t-il pris? 

—Dans le laboratoire de madame sa mère. 

—Sa mère a donc des poisons dans son laboratoire? 

—Est-ce que je sais, moi! vous venez me faire là des questions de procureur du roi. Je répète ce qu'on m'a dit, voilà tout; je vous cite mon auteur: je ne puis faire davantage. Le pauvre diable ne mangeait plus d'épouvante. 

—C'est incroyable! 

—Mais non, mon cher, ce n'est pas incroyable du tout, vous avez vu l'an passé cet enfant de la rue de Richelieu, qui s'amusait à tuer ses frères et ses sœurs en leur enfonçant une épingle dans l'oreille, tandis qu'ils dormaient. La génération qui nous suit est très précoce, mon cher. 

—Mon cher, dit Château-Renaud, je parie que vous ne croyez pas un seul mot de ce que vous nous contez là?\dots Mais je ne vois pas le comte de Monte-Cristo; comment donc n'est-il pas ici? 

—Il est blasé, lui, fit Debray, et puis il ne voudra point paraître devant tout le monde, lui qui a été la dupe de tous les Cavalcanti, lesquels sont venus à lui, à ce qu'il paraît, avec de fausses lettres de créance; de sorte qu'il en est pour une centaine de mille francs hypothéqués sur la principauté. 

—À propos, monsieur de Château-Renaud, demanda Beauchamp, comment se porte Morrel? 

—Ma foi, dit le gentilhomme, voici trois fois que je vais chez lui, et pas plus de Morrel que sur la main. Cependant sa sœur ne m'a point paru inquiète, et elle m'a dit avec un fort bon visage qu'elle ne l'avait pas vu non plus depuis deux ou trois jours, mais qu'elle était certaine qu'il se portait bien. 

—Ah! j'y pense! le comte de Monte-Cristo ne peut venir dans la salle, dit Beauchamp. 

—Pourquoi cela? 

—Parce qu'il est acteur dans le drame. 

—Est-ce qu'il a aussi assassiné quelqu'un? demanda Debray. 

—Mais non, c'est lui, au contraire, qu'on a voulu assassiner. Vous savez bien que c'est en sortant de chez lui que ce bon M. de Caderousse a été assassiné par son petit Benedetto. Vous savez bien que c'est chez lui qu'on a retrouvé ce fameux gilet dans lequel était la lettre qui est venue déranger la signature du contrat. Voyez-vous le fameux gilet? Il est là tout sanglant, sur le bureau, comme pièce de conviction. 

—Ah! fort bien. 

—Chut! messieurs, voici la cour; à nos places!» 

En effet un grand bruit se fit entendre dans le prétoire; le sergent de ville appela ses deux protégés par un hem! énergique, et l'huissier, paraissant au seuil de la salle des délibérations, cria de cette voix glapissante que les huissiers avaient déjà du temps de Beaumarchais: 

«La cour, messieurs!» 