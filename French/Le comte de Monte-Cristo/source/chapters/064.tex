\chapter{Le mendiant}

\lettrine{L}{a} soirée s'avançait; Mme de Villefort avait manifesté le désir de regagner Paris, ce que n'avait point osé faire Mme Danglars, malgré le malaise évident qu'elle éprouvait. 

\zz
Sur la demande de sa femme, M. de Villefort donna donc le premier le signal du départ. Il offrit une place dans son landau à Mme Danglars, afin qu'elle eût les soins de sa femme. Quant à M. Danglars, absorbé dans une conversation industrielle des plus intéressantes avec M. Cavalcanti, il ne faisait aucune attention à tout ce qui se passait. 

Monte-Cristo, tout en demandant son flacon à Mme de Villefort, avait remarqué que M. de Villefort s'était approché de Mme Danglars, et guidé par sa situation, il avait deviné ce qu'il lui avait dit, quoiqu'il eût parlé si bas qu'à peine si Mme Danglars elle-même l'avait entendu. 

Il laissa, sans s'opposer à aucun arrangement, partir Morrel, Debray et Château-Renaud à cheval, et monter les deux dames dans le landau de M. de Villefort; de son côté, Danglars, de plus en plus enchanté de Cavalcanti père, l'invita à monter avec lui dans son coupé. 

Quant à Andrea Cavalcanti, il gagna son tilbury, qui l'attendait devant la porte, et dont un groom, qui exagérait les agréments de la fashion anglaise, lui tenait, en se hissant sur la pointe de ses bottes, l'énorme cheval gris de fer. 

Andrea n'avait pas beaucoup parlé durant le dîner, par cela même que c'était un garçon fort intelligent, et qu'il avait tout naturellement éprouvé la crainte de dire quelque sottise au milieu de ces convives riches et puissants, parmi lesquels son œil dilaté n'apercevait peut-être pas sans crainte un procureur du roi. 

Ensuite il avait été accaparé par M. Danglars, qui, après un rapide coup d'œil sur le vieux major au cou raide et sur son fils encore un peu timide, en rapprochant tous ces symptômes de l'hospitalité de Monte-Cristo, avait pensé qu'il avait affaire à quelque nabab venu à Paris pour perfectionner son fils unique dans la vie mondaine. 

Il avait donc contemplé avec une complaisance indicible l'énorme diamant qui brillait au petit doigt du major, car le major, en homme prudent et expérimenté, de peur qu'il n'arrivât quelque accident à ses billets de banque, les avait convertis à l'instant même en un objet de valeur. Puis, après le dîner, toujours sous prétexte d'industrie et de voyages, il avait questionné le père et le fils sur leur manière de vivre; et le père et le fils, prévenus que c'était chez Danglars que devaient leur être ouverts, à l'un, son crédit de quarante-huit mille francs, une fois donnés, à l'autre, son crédit annuel de cinquante mille livres, avaient été charmants et plein d'affabilité pour le banquier, aux domestiques duquel, s'ils ne se fussent retenus, ils eussent serré la main, tant leur reconnaissance éprouvait le besoin de l'expansion. 

Une chose surtout augmenta la considération, nous dirons presque la vénération de Danglars pour Cavalcanti. Celui-ci, fidèle au principe d'Horace: \textit{nil admirari}, s'était contenté, comme on l'a vu, de faire preuve de science, en disant de quel lac on tirait les meilleures lamproies. Puis il avait mangé sa part de celle-là sans dire un seul mot. Danglars en avait conclu que ces sortes de somptuosités étaient familières à l'illustre descendant des Cavalcanti, lequel se nourrissait probablement, à Lucques, de truites qu'il faisait venir de Suisse, et de langoustes qu'on lui envoyait de Bretagne, par des procédés pareils à ceux dont le comte s'était servi pour faire venir des lamproies du lac Fusaro, et des sterlets du fleuve Volga. Aussi, avait-il accueilli avec une bienveillance très prononcée ces paroles de Cavalcanti: 

«Demain, monsieur, j'aurai l'honneur de vous rendre visite pour affaires. 

—Et moi, monsieur, avait répondu Danglars, je serai heureux de vous recevoir.» 

Sur quoi il avait proposé à Cavalcanti, si cependant cela ne le privait pas trop de se séparer de son fils, de le reconduire à l'hôtel des Princes. 

Cavalcanti avait répondu que, depuis longtemps, son fils avait l'habitude de mener la vie de jeune homme; qu'en conséquence, il avait ses chevaux et ses équipages à lui, et que, n'étant pas venus ensemble, il ne voyait pas de difficulté à ce qu'ils s'en allassent séparément. 

Le major était donc monté dans la voiture de Danglars, et le banquier s'était assis à ses côtés, de plus en plus charmé des idées d'ordre et d'économie de cet homme, qui, cependant, donnait à son fils cinquante mille francs par an, ce qui supposait une fortune de cinq ou six cent mille livres de rente. 

Quant à Andrea, il commença, pour se donner bon air, à gronder son groom de ce qu'au lieu de le venir prendre au perron il l'attendait à la porte de sortie, ce qui lui avait donné la peine de faire trente pas pour aller chercher son tilbury. 

Le groom reçut la semonce avec humilité, prit, pour retenir le cheval impatient et qui frappait du pied, le mors de la main gauche, tendit de la droite les rênes à Andrea, qui les prit et posa légèrement sa botte vernie sur le marchepied. 

En ce moment, une main s'appuya sur son épaule. Le jeune homme se retourna, pensant que Danglars ou Monte-Cristo avait oublié quelque chose à lui dire, et revenait à la charge au moment du départ. 

Mais, au lieu de l'un et de l'autre, il n'aperçut qu'une figure étrange, hâlée par le soleil, encadrée dans une barbe de modèle, des yeux brillants comme des escarboucles et un sourire railleur s'épanouissant sur une bouche où brillaient, rangées à leur place et sans qu'il en manquât une seule, trente-deux dents blanches, aiguës et affamées comme celles d'un loup ou d'un chacal. 

Un mouchoir à carreaux rouges coiffait cette tête aux cheveux grisâtres et terreux; un bourgeron des plus crasseux et des plus déchirés couvrait ce grand corps maigre et osseux, dont il semblait que les os, comme ceux d'un squelette, dussent cliqueter en marchant. Enfin, la main qui s'appuya sur l'épaule d'Andrea, et qui fut la première chose que vit le jeune homme, lui parut d'une dimension gigantesque. Le jeune homme, reconnut-il cette figure à la lueur de la lanterne de son tilbury, ou fut-il seulement frappé de l'horrible aspect de cet interlocuteur? Nous ne saurions le dire; mais le fait est qu'il tressaillit et se recula vivement. 

«Que me voulez-vous? dit-il. 

—Pardon! notre bourgeois, répondit l'homme en portant la main à son mouchoir rouge, je vous dérange peut-être, mais c'est que j'ai à vous parler. 

—On ne mendie pas le soir, dit le groom en faisant un mouvement pour débarrasser son maître de cet importun. 

—Je ne mendie pas, mon joli garçon, dit l'homme inconnu au domestique avec un sourire ironique, et un sourire si effrayant que celui-ci s'écarta: je désire seulement dire deux mots à votre bourgeois, qui m'a chargé d'une commission il y a quinze jours à peu près. 

—Voyons, dit à son tour Andrea avec assez de force pour que le domestique ne s'aperçût point de son trouble, que voulez-vous? dites vite, mon ami. 

—Je voudrais\dots je voudrais\dots dit tout bas l'homme au mouchoir rouge, que vous voulussiez bien m'épargner la peine de retourner à Paris à pied. Je suis très fatigué, et, comme je n'ai pas si bien dîné que toi, à peine, si je puis me tenir.» 

Le jeune homme tressaillit à cette étrange familiarité. 

«Mais enfin, lui dit-il, voyons, que voulez-vous? 

—Eh bien, je veux que tu me laisses monter dans ta belle voiture, et que tu me reconduises. 

Andrea pâlit, mais ne répondit point.  

«Oh! mon Dieu, oui, dit l'homme au mouchoir rouge en enfonçant ses mains dans ses poches, et en regardant le jeune homme avec des yeux provocateurs, c'est une idée que j'ai comme cela; entends-tu, mon petit Benedetto?» 

À ce nom, le jeune homme réfléchit sans doute, car il s'approcha de son groom, et lui dit: 

«Cet homme a effectivement été chargé par moi d'une commission dont il a à me rendre compte. Allez à pied jusqu'à la barrière; là, vous prendrez un cabriolet, afin de n'être point trop en retard.» 

Le valet, surpris, s'éloigna. 

«Laissez-moi au moins gagner l'ombre, dit Andrea. 

—Oh! quant à cela, je vais moi-même te conduire en belle place; attends», dit l'homme au mouchoir rouge. 

Et il prit le cheval par le mors, et conduisit le tilbury dans un endroit où il était effectivement impossible à qui que ce fût au monde de voir l'honneur que lui accordait Andrea. 

«Oh! moi, lui dit-il, ce n'est pas pour la gloire de monter dans une belle voiture non, c'est seulement parce que je suis fatigué, et puis, un petit peu, parce que j'ai à causer d'affaires avec toi. 

—Voyons, montez», dit le jeune homme. 

Il était fâcheux qu'il ne fît pas jour, car ç'eût été un spectacle curieux que celui de ce gueux, assis carrément sur les coussins brochés, près du jeune et élégant conducteur du tilbury. 

Andrea poussa son cheval jusqu'à la dernière maison du village sans dire un seul mot à son compagnon, qui, de son côté, souriait et gardait le silence, comme s'il eût été ravi de se promener dans une si bonne locomotive. 

Une fois hors d'Auteuil, Andrea regarda autour de lui pour s'assurer sans doute que nul ne pouvait ni les voir ni les entendre; et alors, arrêtant son cheval et se croisant les bras devant l'homme au mouchoir rouge: 

«Ah çà! lui dit-il, pourquoi venez-vous me troubler dans ma tranquillité? 

—Mais, toi-même, mon garçon, pourquoi te défies-tu de moi? 

—Et en quoi me suis-je défié de vous? 

—En quoi? tu le demandes? nous nous quittons au pont du Var, tu me dis que tu vas voyager en Piémont et en Toscane, et pas du tout, tu viens à Paris. 

—En quoi cela vous gêne-t-il? 

—En rien; au contraire, j'espère même que cela va m'aider. 

—Ah! ah! dit Andrea, c'est-à-dire que vous spéculez sur moi. 

—Allons! voilà les gros mots qui arrivent. 

—C'est que vous auriez tort, maître Caderousse, je vous en préviens. 

—Eh! mon Dieu! ne te fâche pas, le petit; tu dois pourtant savoir ce que c'est que le malheur; eh bien, le malheur, ça rend jaloux. Je te crois courant le Piémont et la Toscane, obligé de te faire \textit{faccino} ou \textit{cicerone}; je te plains du fond de mon cœur, comme je plaindrais mon enfant. Tu sais que je t'ai toujours appelé mon enfant. 

—Après? après? 

—Patience donc, salpêtre! 

—J'en ai de la patience; voyons, achevez. Et je te vois tout à coup passer à la barrière des Bons-Hommes avec un groom, avec un tilbury, avec des habits tout flambant neufs. Ah çà! mais tu as donc découvert une mine, ou acheté une charge d'agent de change? 

—De sorte que, comme vous l'avouez, vous êtes jaloux? 

—Non, je suis content, si content, que j'ai voulu te faire mes compliments, le petit! mais, comme je n'étais pas vêtu régulièrement, j'ai pris mes précautions pour ne pas te compromettre. 

—Belles précautions! dit Andrea, vous m'abordez devant mon domestique. 

—Eh! que veux-tu, mon enfant! je t'aborde quand je puis te saisir. Tu as un cheval très vif, un tilbury très léger; tu es naturellement glissant comme une anguille; si je t'avais manqué ce soir, je courais risque de ne pas te rejoindre. 

—Vous voyez bien que je ne me cache pas. 

—Tu es bien heureux, et j'en voudrais bien dire autant; moi, je me cache: sans compter que j'avais peur que tu ne me reconnusses pas; mais tu m'as reconnu, ajouta Caderousse avec son mauvais sourire; allons, tu es bien gentil. 

—Voyons, dit Andrea, que vous faut-il? 

—Tu ne me tutoies plus, c'est mal, Benedetto, un ancien camarade; prends garde, tu vas me rendre exigeant.» 

Cette menace fit tomber la colère du jeune homme: le vent de la contrainte venait de souffler dessus. Il remit son cheval au trot. 

«C'est mal à toi-même, Caderousse, dit-il, de t'y prendre ainsi envers un ancien camarade, comme tu disais tout à l'heure; tu es Marseillais, je suis\dots. 

—Tu le sais donc ce que tu es maintenant? 

—Non, mais j'ai été élevé en Corse; tu es vieux et entêté; je suis jeune et têtu. Entre gens comme nous, la menace est mauvaise, et tout doit se faire à l'amiable. Est-ce ma faute si la chance, qui continue d'être mauvaise pour toi, est bonne pour moi au contraire? 

—Elle est donc bonne, la chance? ce n'est donc pas un groom d'emprunt, ce n'est donc pas un tilbury d'emprunt, ce ne sont donc pas des habits d'emprunt que nous avons là? Bon, tant mieux! dit Caderousse avec des yeux brillants de convoitise. 

—Oh! tu le vois bien et tu le sais bien, puisque tu m'abordes, dit Andrea s'animant de plus en plus. Si j'avais un mouchoir comme le tien sur ma tête, un bourgeron crasseux sur les épaules et des souliers percés aux pieds, tu ne me reconnaîtrais pas. 

—Tu vois bien que tu me méprises, le petit, et tu as tort; maintenant que je t'ai retrouvé, rien ne m'empêche d'être vêtu d'elbeuf comme un autre, attendu que je te connais bon cœur: si tu as deux habits, tu m'en donneras bien un; je te donnais bien ma portion de soupe et de haricots, moi, quand tu avais trop faim. 

—C'est vrai, dit Andrea. 

—Quel appétit tu avais! Est-ce que tu as toujours bon appétit? 

—Mais oui, dit Andrea en riant. 

—Comme tu as dû dîner chez ce prince d'où tu sors. 

—Ce n'est pas un prince, mais tout bonnement un comte. 

—Un comte? et un riche, hein? 

—Oui, mais ne t'y fie pas; c'est un monsieur qui n'a pas l'air commode. 

—Oh! mon Dieu! sois donc tranquille! On n'a pas de projets sur ton comte, et on te le laissera pour toi tout seul. Mais, ajouta Caderousse en reprenant ce mauvais sourire qui avait déjà effleuré ses lèvres, il faut donner quelque chose pour cela, tu comprends. 

—Voyons, que te faut-il? 

—Je crois qu'avec cent francs par mois\dots. 

—Eh bien? 

—Je vivrais\dots. 

—Avec cent francs? 

—Mais mal, tu comprends bien; mais avec\dots. 

—Avec? 

—Cent cinquante francs, je serais fort heureux. 

—En voilà deux cents», dit Andrea. 

Et il mit dans la main de Caderousse dix louis d'or. 

«Bon, fit Caderousse. 

—Présente-toi chez le concierge tous les premiers du mois et tu en trouveras autant. 

—Allons! voilà encore que tu m'humilies! 

—Comment cela? 

—Tu me mets en rapport avec de la valetaille, non, vois-tu, je ne veux avoir affaire qu'à toi. 

—Eh bien, soit, demande-moi, et tous les premiers du mois, du moins tant que je toucherai ma rente, toi, tu toucheras la tienne. 

—Allons, allons! je vois que je ne m'étais pas trompé, tu es un brave garçon, et c'est une bénédiction quand le bonheur arrive à des gens comme toi. Voyons, conte-moi ta bonne chance. 

—Qu'as-tu besoin de savoir cela? demanda Cavalcanti. 

—Bon! encore de la défiance! 

—Non. Eh bien, j'ai retrouvé mon père. 

—Un vrai père? 

—Dame! tant qu'il paiera\dots. 

—Tu croiras et tu honoreras; c'est juste. Comment l'appelles-tu ton père? 

—Le major Cavalcanti. 

—Et il se contente de toi? 

—Jusqu'à présent il paraît que je lui suffis. 

—Et qui t'a fait retrouver ce père-là? 

—Le comte de Monte-Cristo. 

—Celui de chez qui tu sors? 

—Oui. 

—Dis donc, tâche de me placer chez lui comme grand-parent, puisqu'il tient bureau. 

—Soit, je lui parlerai de toi; mais en attendant que vas-tu faire? 

—Moi? 

—Oui, toi. 

—Tu es bien bon de t'occuper de cela, dit Caderousse. 

—Il me semble, puisque tu prends intérêt à moi, reprit Andrea, que je puis bien à mon tour prendre quelques informations. 

—C'est juste\dots je vais louer une chambre dans une maison honnête, me couvrir d'un habit décent, me faire raser tous les jours, et aller lire les journaux au café. Le soir, j'entrerai dans quelque spectacle avec un chef de claque, j'aurai l'air d'un boulanger retiré, c'est mon rêve. 

—Allons, c'est bon! Si tu veux mettre ce projet à exécution et être sage, tout ira à merveille. 

—Voyez-vous M. Bossuet!\dots et toi, que vas-tu devenir?\dots pair de France?  

—Eh! eh! dit Andrea, qui sait? 

—M. le major Cavalcanti l'est peut-être\dots mais malheureusement l'hérédité est abolie. 

—Pas de politique, Caderousse!\dots Et maintenant que tu as ce que tu veux et que nous sommes arrivés, saute en bas de ma voiture et disparais. 

—Non pas, cher ami! 

—Comment, non pas? 

—Mais songes-y donc, le petit, un mouchoir rouge sur la tête, presque pas de souliers, pas de papier du tout et dix napoléons en or dans ma poche, sans compter ce qu'il y avait déjà, ce qui fait juste deux cents francs; mais on m'arrêterait immanquablement à la barrière! Alors je serais forcé, pour me justifier, de dire que c'est toi qui m'as donné ces dix napoléons: de là information, enquête; on apprend que j'ai quitté Toulon sans donner congé, et l'on me reconduit de brigade en brigade jusqu'au bord de la Méditerranée. Je redeviens purement et simplement le n°106, et adieu mon rêve de ressembler à un boulanger retiré! Non pas, mon fils; je préfère rester honorablement dans la capitale.» 

Andrea fronça le sourcil; c'était, comme il s'en était vanté lui-même, une assez mauvaise tête que le fils putatif de M. le major Cavalcanti. Il s'arrêta un instant, jeta un coup d'œil rapide autour de lui, et comme son regard achevait de décrire le cercle investigateur, sa main descendit innocemment dans son gousset, ou elle commença de caresser la sous-garde d'un pistolet de poche. 

Mais pendant ce temps, Caderousse, qui ne perdait pas de vue son compagnon, passait ses mains derrière son dos, et ouvrait tout doucement un long couteau espagnol qu'il portait sur lui à tout événement. 

Les deux amis, comme on le voit, étaient dignes de se comprendre, et se comprirent; la main d'Andrea sortit inoffensive de sa poche, et remonta jusqu'à sa moustache rousse, qu'elle caressa quelque temps. 

«Bon Caderousse, dit-il, tu vas donc être heureux? 

—Je ferai tout mon possible, répondit l'aubergiste du pont du Gard en renfonçant son couteau dans sa manche. 

—Allons, voyons, rentrons donc dans Paris. Mais comment vas-tu faire pour passer la barrière sans éveiller les soupçons? Il me semble qu'avec ton costume tu risques encore plus en voiture qu'à pied. 

—Attends, dit Caderousse, tu vas voir.» 

Il prit le chapeau d'Andrea, la houppelande à grand collet que le groom exilé du tilbury avait laissée à sa place, et la mit sur son dos, après quoi, il prit la pose renfrognée d'un domestique de bonne maison dont le maître conduit lui-même. 

«Et moi, dit Andrea, je vais donc rester nu-tête? 

—Peuh! dit Caderousse, il fait tant de vent que la bise peut bien t'avoir enlevé ton chapeau. 

—Allons donc, dit Andrea, et finissons-en. 

—Qui est-ce qui t'arrête? dit Caderousse, ce n'est pas moi, je l'espère? 

—Chut!» fit Cavalcanti. 

On traversa la barrière sans accident. 

À la première rue transversale, Andrea arrêta son cheval, et Caderousse sauta à terre. 

«Eh bien, dit Andrea, et le manteau de mon domestique, et mon chapeau? 

—Ah! répondit Caderousse, tu ne voudrais pas que je risquasse de m'enrhumer? 

—Mais moi? 

—Toi, tu es jeune, tandis que, moi, je commence à me faire vieux; au revoir, Benedetto!» 

Et il s'enfonça dans la ruelle, où il disparut. 

«Hélas! dit Andrea en poussant un soupir, on ne peut donc pas être complètement heureux en ce monde!» 