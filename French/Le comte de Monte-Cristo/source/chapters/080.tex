\chapter{L'accusation}

\lettrine{M}{.} d'Avrigny eut bientôt rappelé à lui le magistrat, qui semblait un second cadavre dans cette chambre funèbre. 

\zz
«Oh! la mort est dans ma maison! s'écria Villefort. 

\zz
—Dites le crime, répondit le docteur. 

—Monsieur d'Avrigny! s'écria Villefort, je ne puis vous exprimer tout ce qui se passe en moi en ce moment; c'est de l'effroi, c'est de la douleur, c'est de la folie. 

—Oui, dit M. d'Avrigny avec un calme imposant: mais je crois qu'il est temps que nous agissions; je crois qu'il est temps que nous opposions une digue à ce torrent de mortalité. Quant à moi, je ne me sens point capable de porter plus longtemps de pareils secrets, sans espoir d'en faire bientôt sortir la vengeance pour la société et les victimes.» 

Villefort jeta autour de lui un sombre regard. 

«Dans ma maison! murmura-t-il, dans ma maison! 

—Voyons, magistrat, dit d'Avrigny, soyez homme; interprète de la loi, honorez-vous par une immolation complète. 

—Vous me faites frémir, docteur, une immolation! 

—J'ai dit le mot. 

—Vous soupçonnez donc quelqu'un? 

—Je ne soupçonne personne; la mort frappe à votre porte, elle entre, elle va, non pas aveugle, mais intelligente qu'elle est, de chambre en chambre. Eh bien, moi, je suis sa trace, je reconnais son passage, j'adopte la sagesse des anciens: je tâtonne; car mon amitié pour votre famille, car mon respect pour vous sont deux bandeaux appliqués sur mes yeux; eh bien\dots. 

—Oh! parlez, parlez, docteur, j'aurai du courage. 

—Eh bien, monsieur, vous avez chez vous, dans le sein de votre maison, dans votre famille peut-être, un de ces affreux phénomènes comme chaque siècle en produit quelqu'un. Locuste et Agrippine, vivant en même temps, sont une exception qui prouve la fureur de la Providence à perdre l'empire romain, souillé par tant de crimes. Brunehaut et Frédégonde sont les résultats du travail pénible d'une civilisation à sa genèse, dans laquelle l'homme apprenait à dominer l'esprit, fût-ce par l'envoyé des ténèbres. Eh bien, toutes ces femmes avaient été ou étaient encore jeunes et belles. On avait vu fleurir sur leur front, ou sur leur front fleurissait encore, cette même fleur d'innocence que l'on retrouve aussi sur le front de la coupable qui est dans votre maison.» 

Villefort poussa un cri, joignit les mains, et regarda le docteur avec un geste suppliant. 

Mais celui-ci poursuivit sans pitié: 

«Cherche à qui le crime profite, dit un axiome de jurisprudence\dots. 

—Docteur! s'écria Villefort, hélas! docteur, combien de fois la justice des hommes n'a-t-elle pas été trompée par ces funestes paroles! Je ne sais, mais il me semble que ce crime\dots. 

—Ah! vous avouez donc enfin que le crime existe? 

—Oui, je le reconnais. Que voulez-vous? il le faut bien. Mais laissez-moi continuer. Il me semble, dis-je, que ce crime tombe sur moi seul et non sur les victimes. Je soupçonne quelque désastre pour moi sous tous ces désastres étranges. 

—Ô homme! murmura d'Avrigny; le plus égoïste de tous les animaux, la plus personnelle de toutes les créatures, qui croit toujours que la terre tourne, que le soleil brille, que la mort fauche pour lui tout seul; fourmi maudissant Dieu du haut d'un brin d'herbe! Et ceux qui ont perdu la vie, n'ont-ils rien perdu, eux? M. de Saint-Méran, Mme de Saint-Méran, M. Noirtier\dots. 

—Comment? M. Noirtier! 

—Eh oui! Croyez-vous, par exemple, que ce soit à ce malheureux domestique qu'on en voulait? Non, non: comme le Polonius de Shakespeare, il est mort pour un autre. C'était Noirtier qui devait boire la limonade, c'est Noirtier qui l'a bue selon l'ordre logique des choses: l'autre ne l'a bue que par accident; et, quoique ce soit Barrois qui soit mort, c'est Noirtier qui devait mourir. 

—Mais alors comment mon père n'a-t-il pas succombé? 

—Je vous l'ai déjà dit, un soir, dans le jardin, après la mort de Mme de Saint-Méran; parce que son corps est fait à l'usage de ce poison même; parce que la dose insignifiante pour lui était mortelle pour tout autre; parce qu'enfin personne ne sait, et pas même l'assassin, que depuis un an je traite avec la brucine la paralysie de M. Noirtier, tandis que l'assassin n'ignore pas, et il s'en est assuré par expérience, que la brucine est un poison violent. 

—Mon Dieu! mon Dieu! murmura Villefort en se tordant les bras. 

—Suivez la marche du criminel; il tue M. de Saint-Méran. 

—Oh! docteur! 

—Je le jurerais; ce que l'on m'a dit des symptômes s'accorde trop bien avec ce que j'ai vu de mes yeux.» 

Villefort cessa de combattre, et poussa un gémissement. 

«Il tue M. de Saint-Méran, répéta le docteur, il tue Mme de Saint-Méran: double héritage à recueillir.» 

Villefort essuya la sueur qui coulait sur son front. 

«Écoutez bien. 

—Hélas! balbutia Villefort, je ne perds pas un mot, pas un seul. 

—M. Noirtier, reprit de sa voix impitoyable M. d'Avrigny, M. Noirtier avait testé naguère contre vous, contre votre famille, en faveur des pauvres enfin; M. Noirtier est épargné, on n'attend rien de lui. Mais il n'a pas plus tôt détruit son premier testament, il n'a pas plus tôt fait le second, que, de peur qu'il n'en fasse sans doute un troisième, on le frappe: le testament est d'avant-hier, je crois; vous le voyez, il n'y a pas de temps de perdu. 

—Oh! grâce! monsieur d'Avrigny. 

—Pas de grâce, monsieur; le médecin a une mission sacrée sur la terre, c'est pour la remplir qu'il a remonté jusqu'aux sources de la vie et descendu dans les mystérieuses ténèbres de la mort. Quand le crime a été commis, et que Dieu, épouvanté sans doute, détourne son regard du criminel, c'est au médecin de dire: Le voilà! 

—Grâce pour ma fille, monsieur! murmura Villefort. 

—Vous voyez bien que c'est vous qui l'avez nommée, vous, son père! 

—Grâce pour Valentine! Écoutez, c'est impossible. J'aimerais autant m'accuser moi-même! Valentine, un cœur de diamant, un lis d'innocence! 

—Pas de grâce, monsieur le procureur du roi, le crime est flagrant: Mlle de Villefort a emballé elle-même les médicaments qu'on a envoyés à M. de Saint-Méran, et M. de Saint-Méran est mort. 

«Mlle de Villefort a préparé les tisanes de Mme de Saint-Méran, et Mme de Saint-Méran est morte. 

«Mlle de Villefort a pris des mains de Barrois, que l'on a envoyé dehors, le carafon de limonade que le vieillard vide ordinairement dans la matinée, et le vieillard n'a échappé que par miracle.  

«Mlle de Villefort est la coupable! c'est l'empoisonneuse! Monsieur le procureur du roi, je vous dénonce Mlle de Villefort, faites votre devoir. 

—Docteur, je ne résiste plus, je ne me défends plus, je vous crois, mais, par pitié, épargnez ma vie, mon honneur! 

—Monsieur de Villefort, reprit le docteur avec une force croissante, il est des circonstances où je franchis toutes les limites de la sotte circonspection humaine. Si votre fille avait commis seulement un premier crime, et que je la visse en méditer un second, je vous dirais: Avertissez-la, punissez-la, qu'elle passe le reste de sa vie dans quelque cloître, dans quelque couvent, à pleurer, à prier. Si elle avait commis un second crime, je vous dirais: «Tenez, monsieur de Villefort, voilà un poison qui n'a pas d'antidote connu, prompt comme la pensée, rapide comme l'éclair, mortel comme la foudre, donnez-lui ce poison en recommandant son âme à Dieu, et sauvez ainsi votre honneur et vos jours, car c'est à vous qu'elle en veut.» Et je la vois s'approcher de votre chevet avec ses sourires hypocrites et ses douces exhortations! Malheur à vous, monsieur de Villefort, si vous ne vous hâtez pas de frapper le premier! Voilà ce que je vous dirais si elle n'avait tué que deux personnes; mais elle a vu trois agonies, elle a contemplé trois moribonds, s'est agenouillée près de trois cadavres; au bourreau l'empoisonneuse! au bourreau! Vous parlez de votre honneur, faites ce que je vous dis, et c'est l'immortalité qui vous attend!» 

Villefort tomba à genoux. 

«Écoutez, dit-il, je n'ai pas cette force que vous avez, ou plutôt que vous n'auriez pas si, au lieu de ma fille Valentine, il s'agissait de votre fille Madeleine.» 

Le docteur pâlit. 

«Docteur, tout homme fils de la femme est né pour souffrir et mourir; docteur, je souffrirai, et j'attendrai la mort. 

—Prenez garde, dit M. d'Avrigny, elle sera lente\dots cette mort; vous la verrez s'approcher après avoir frappé votre père, votre femme, votre fils peut-être.» 

Villefort, suffoquant, étreignit le bras du docteur. 

«Écoutez-moi! s'écria-t-il, plaignez-moi, secourez-moi\dots. Non, ma fille n'est pas coupable\dots. Traînez-nous devant un tribunal, je dirai encore: «Non, ma fille n'est pas coupable» il n'y a pas de crime dans ma maison\dots. Je ne veux pas, entendez-vous, qu'il y ait un crime dans ma maison; car lorsque le crime entre quelque part, c'est comme la mort, il n'entre pas seul. Écoutez, que vous importe à vous que je meure assassiné?\dots êtes-vous mon ami? êtes-vous un homme? avez-vous un cœur?\dots Non, vous êtes médecin!\dots Eh bien, je vous dis: «Non, ma fille ne sera pas traînée par moi aux mains du bourreau!\dots» Ah! voilà une idée qui me dévore, qui me pousse comme un insensé à creuser ma poitrine avec mes ongles!\dots Et si vous vous trompiez, docteur! si c'était un autre que ma fille! Si, un jour, je venais, pâle comme un spectre vous dire: Assassin! tu as tué ma fille\dots. Tenez, si cela arrivait, je suis chrétien, monsieur d'Avrigny, et cependant je me tuerais! 

—C'est bien, dit le docteur après un instant de silence, j'attendrai.» 

Villefort le regarda comme s'il doutait encore de ses paroles. 

«Seulement, continua M. d'Avrigny d'une voix lente et solennelle, si quelque personne de votre maison tombe malade, si vous-même vous vous sentez frappé, ne m'appelez pas, car je ne viendrai plus. Je veux bien partager avec vous ce secret terrible, mais je ne veux pas que la honte et le remords aillent chez moi en fructifiant et en grandissant dans ma conscience, comme le crime et le malheur vont grandir et fructifier dans votre maison. 

—Ainsi, vous m'abandonnez, docteur? 

—Oui, car je ne puis pas vous suivre plus loin, et je ne m'arrête qu'au pied de l'échafaud. Quelque autre révélation viendra qui amènera la fin de cette terrible tragédie. Adieu. 

—Docteur, je vous en supplie! 

—Toutes les horreurs qui souillent ma pensée font votre maison odieuse et fatale. Adieu, monsieur. 

—Un mot, un mot seulement encore, docteur! Vous vous retirez me laissant toute l'horreur de la situation, horreur que vous avez augmentée par ce que vous m'avez révélé. Mais de la mort instantanée, subite, de ce pauvre vieux serviteur, que va-t-on dire? 

—C'est juste, dit M. d'Avrigny, reconduisez-moi.» 

Le docteur sortit le premier, M. de Villefort le suivit; les domestiques, inquiets, étaient dans les corridors et sur les escaliers par où devait passer le médecin. 

«Monsieur, dit d'Avrigny à Villefort, en parlant à haute voix de façon que tout le monde l'entendît, le pauvre Barrois était trop sédentaire depuis quelques années: lui, qui aimait tant avec son maître à courir à cheval ou en voiture les quatre coins de l'Europe, il s'est tué à ce service monotone autour d'un fauteuil. Le sang est devenu lourd. Il était replet, il avait le cou gros et court, il a été frappé d'une apoplexie foudroyante, et l'on m'est venu avertir trop tard. 

«À propos, ajouta-t-il tout bas, ayez bien soin de jeter cette tasse de violettes dans les cendres.» 

Et le docteur, sans toucher la main de Villefort, sans revenir un seul instant sur ce qu'il avait dit, sortit escorté par les larmes et les lamentations de tous les gens de la maison. 

Le soir même, tous les domestiques de Villefort, qui s'étaient réunis dans la cuisine et qui avaient longuement causé entre eux, vinrent demander à Mme de Villefort la permission de se retirer. Aucune instance, aucune proposition d'augmentation de gages ne les put retenir; à toutes paroles ils répondaient: 

«Nous voulons nous en aller parce que la mort est dans la maison.» 

Ils partirent donc, malgré les prières qu'on leur fit, témoignant que leurs regrets étaient vifs de quitter de si bons maîtres, et surtout Mlle Valentine, si bonne, si bienfaisante et si douce. 

Villefort, à ces mots, regarda Valentine. 

Elle pleurait. 

Chose étrange! à travers l'émotion que lui firent éprouver ces larmes, il regarda aussi Mme de Villefort, et il lui sembla qu'un sourire fugitif et sombre avait passé sur ses lèvres minces, comme ces météores qu'on voit glisser, sinistres, entre deux nuages, au fond d'un ciel orageux. 