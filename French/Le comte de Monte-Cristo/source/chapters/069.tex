\chapter{Les informations} 

\lettrine{M}{.} de Villefort tint parole à Mme Danglars, et surtout à lui-même, en cherchant à savoir de quelle façon M. le comte de Monte-Cristo avait pu apprendre l'histoire de la maison d'Auteuil. 

Il écrivit le même jour à un certain M. de Boville, qui, après avoir été autrefois inspecteur des prisons, avait été attaché, dans un grade supérieur, à la police de sûreté, pour avoir les renseignements qu'il désirait, et celui-ci demanda deux jours pour savoir au juste près de qui l'on pourrait se renseigner. 

Les deux jours expirés, M. de Villefort reçut la note suivante: 

«La personne que l'on appelle M. le comte de Monte-Cristo est connue particulièrement de Lord Wilmore, riche étranger, que l'on voit quelquefois à Paris et qui s'y trouve en ce moment; il est connu également de l'abbé Busoni, prêtre sicilien d'une grande réputation en Orient, où il a fait beaucoup de bonnes œuvres.» 

M. de Villefort répondit par un ordre de prendre sur ces deux étrangers les informations les plus promptes et les plus précises; le lendemain soir, ses ordres étaient exécutés, et voici les renseignements qu'il recevait: 

L'abbé, qui n'était que pour un mois à Paris, habitait, derrière Saint-Sulpice, une petite maison composée d'un seul étage au-dessus d'un rez-de-chaussée; quatre pièces, deux pièces en haut et deux pièces en bas, formaient tout le logement, dont il était l'unique locataire. 

Les deux pièces d'en bas se composaient d'une salle à manger avec table, deux chaises et buffet en noyer, et d'un salon boisé peint en blanc, sans ornements, sans tapis et sans pendule. On voyait que, pour lui-même, l'abbé se bornait aux objets de stricte nécessité. 

Il est vrai que l'abbé habitait de préférence le salon du premier. Ce salon, tout meublé de livres de théologie et de parchemins, au milieu desquels on le voyait s'ensevelir, disait son valet de chambre, pendant des mois entiers, était en réalité moins un salon qu'une bibliothèque. 

Ce valet regardait les visiteurs au travers d'une sorte de guichet, et lorsque leur figure lui était inconnue ou ne lui plaisait pas, il répondait que M. l'abbé n'était point à Paris, ce dont beaucoup se contentaient, sachant que l'abbé voyageait souvent et restait quelquefois fort longtemps en voyage. 

Au reste, qu'il fût au logis ou qu'il n'y fût pas, qu'il se trouvât à Paris ou au Caire, l'abbé donnait toujours, et le guichet servait de tour aux aumônes que le valet distribuait incessamment au nom de son maître. 

L'autre chambre, située près de la bibliothèque, était une chambre à coucher. Un lit sans rideaux, quatre fauteuils et un canapé de velours d'Utrecht jaune formaient, avec un prie-Dieu, tout son ameublement. 

Quant à Lord Wilmore, il demeurait rue Fontaine-Saint-Georges. C'était un de ces Anglais touristes qui mangent toute leur fortune en voyages. Il louait en garni l'appartement qu'il habitait dans lequel il venait passer seulement deux ou trois heures par jour, et où il ne couchait que rarement. Une de ses manies était de ne vouloir pas absolument parler la langue française, qu'il écrivait cependant, assurait-on, avec une assez grande pureté.  

Le lendemain du jour où ces précieux renseignements étaient parvenus à M. le procureur du roi, un homme, qui descendait de voiture au coin de la rue Férou, vint frapper à une porte peinte en vert olive et demanda l'abbé Busoni. 

«M. l'abbé est sorti dès le matin, répondit le valet. 

—Je pourrais ne pas me contenter de cette réponse, dit le visiteur, car je viens de la part d'une personne pour laquelle on est toujours chez soi. Mais veuillez remettre à l'abbé Busoni\dots. 

—Je vous ai déjà dit qu'il n'y était pas, répéta le valet. 

—Alors quand il sera rentré, remettez-lui cette carte et ce papier cacheté. Ce soir, à huit heures M. l'abbé sera-t-il chez lui? 

—Oh! sans faute, monsieur, à moins que M. l'abbé ne travaille, et alors c'est comme s'il était sorti. 

—Je reviendrai donc ce soir à l'heure convenue», reprit le visiteur. 

Et il se retira. 

En effet, à l'heure indiquée, le même homme revint dans la même voiture, qui cette fois, au lieu de s'arrêter au coin de la rue Férou, s'arrêta devant la porte verte. Il frappa, on lui ouvrit, et il entra. 

Aux signes de respect dont le valet fut prodigue envers lui, il comprit que sa lettre avait fait l'effet désiré. 

«M. l'abbé est chez lui? demanda-t-il. 

—Oui, il travaille dans sa bibliothèque; mais il attend monsieur», répondit le serviteur. 

L'étranger monta un escalier assez rude, et, devant une table dont la superficie était inondée de la lumière que concentrait un vaste abat-jour, tandis que le reste de l'appartement était dans l'ombre, il aperçut l'abbé, en habit ecclésiastique, la tête couverte de ces coqueluchons sous lesquels s'ensevelissait le crâne des savants en \textit{us} du Moyen Âge. 

«C'est à monsieur Busoni que j'ai l'honneur de parler? demanda le visiteur. 

—Oui, monsieur, répondit l'abbé, et vous êtes la personne que M. de Boville, ancien intendant des prisons, m'envoie de la part de M. le préfet de Police? 

—Justement, monsieur. 

—Un des agents préposés à la sûreté de Paris? 

—Oui, monsieur», répondit l'étranger avec une espèce d'hésitation, et surtout un peu de rougeur. 

L'abbé rajusta les grandes lunettes qui lui couvraient non seulement les yeux, mais encore les tempes, et, se rasseyant, fit signe au visiteur de s'asseoir à son tour. 

«Je vous écoute, monsieur, dit l'abbé avec un accent italien des plus prononcés. 

—La mission dont je me suis chargé, monsieur, reprit le visiteur en pesant chacune de ses paroles comme si elles avaient peine à sortir, est une mission de confiance pour celui qui la remplit et pour celui près duquel on la remplit. 

L'abbé s'inclina. 

«Oui, reprit l'étranger, votre probité, monsieur l'abbé, est si connue de M. le préfet de Police, qu'il veut savoir de vous, comme magistrat, une chose qui intéresse cette sûreté publique au nom de laquelle je vous suis député. Nous espérons donc, monsieur l'abbé, qu'il n'y aura ni liens d'amitié ni considération humaine qui puissent vous engager à déguiser la vérité à la justice. 

—Pourvu, monsieur, que les choses qu'il vous importe de savoir ne touchent en rien aux scrupules de ma conscience. Je suis prêtre, monsieur, et les secrets de la confession, par exemple, doivent rester entre moi et la justice de Dieu, et non entre moi et la justice humaine. 

—Oh! soyez tranquille, monsieur l'abbé, dit l'étranger, dans tous les cas nous mettrons votre conscience à couvert.» 

À ces mots l'abbé, en pesant de son côté sur l'abat-jour, leva ce même abat-jour du côté opposé, de sorte que, tout en éclairant en plein le visage de l'étranger, le sien restait toujours dans l'ombre. 

«Pardon, monsieur l'abbé, dit l'envoyé de M. le préfet de Police, mais cette lumière me fatigue horriblement la vue.» 

L'abbé baissa le carton vert. 

«Maintenant, monsieur, je vous écoute, parlez. 

—J'arrive au fait. Vous connaissez M. le comte de Monte-Cristo? 

—Vous voulez parler de M. Zaccone, je présume? 

—Zaccone!\dots Ne s'appelle-t-il donc pas Monte-Cristo! 

—Monte-Cristo est un nom de terre, ou plutôt un nom de rocher, et non pas un nom de famille. 

—Eh bien, soit; ne discutons pas sur les mots, et puisque M. de Monte-Cristo et M. Zaccone c'est le même homme\dots. 

—Absolument le même. 

—Parlons de M. Zaccone. 

—Soit. 

—Je vous demandais si vous le connaissiez? 

—Beaucoup. 

—Qu'est-il? 

—C'est le fils d'un riche armateur de Malte.  

—Oui, je le sais bien, c'est ce qu'on dit; mais, comme vous le comprenez, la police ne peut pas se contenter d'un \textit{on-dit}. 

—Cependant, reprit l'abbé avec un sourire tout affable, quand cet \textit{on-dit} est la vérité, il faut bien que tout le monde s'en contente, et que la police fasse comme tout le monde. 

—Mais vous êtes sûr de ce que vous dites? 

—Comment! si j'en suis sûr! 

—Remarquez, monsieur, que je ne suspecte en aucune façon votre bonne foi. Je vous dis: Êtes-vous sûr? 

—Écoutez, j'ai connu M. Zaccone le père. 

—Ah! ah! 

—Oui, et tout enfant j'ai joué dix fois avec son fils dans leurs chantiers de construction. 

—Mais cependant ce titre de comte? 

—Vous savez, cela s'achète. 

—En Italie? 

—Partout. 

—Mais ces richesses qui sont immenses à ce qu'on dit toujours\dots. 

—Oh! quant à cela, répondit l'abbé, immenses c'est le mot. 

—Combien croyez-vous qu'il possède, vous qui le connaissez? 

—Oh! il a bien cent cinquante à deux cent mille livres de rente. 

—Ah! voilà qui est raisonnable, dit le visiteur, mais on parle de trois, de quatre millions! 

—Deux cent mille livres de rente, monsieur, font juste quatre millions de capital. 

—Mais on parlait de trois à quatre millions de rente! 

—Oh! cela n'est pas croyable. 

—Et vous connaissez son île de Monte-Cristo? 

—Certainement; tout homme qui est venu de Palerme, de Naples ou de Rome en France, par mer, la connaît, puisqu'il est passé à côté d'elle et l'a vue en passant. 

—C'est un séjour enchanteur, à ce que l'on assure. 

—C'est un rocher. 

—Et pourquoi donc le comte a-t-il acheté un rocher? 

—Justement pour être comte. En Italie, pour être comte, on a encore besoin d'un comté. 

—Vous avez sans doute entendu parler des aventures de jeunesse de M. Zaccone. 

—Le père? 

—Non, le fils. 

—Ah! voici où commencent mes incertitudes, car voici où j'ai perdu mon jeune camarade de vue. 

—Il a fait la guerre? 

—Je crois qu'il a servi. 

—Dans quelle arme? 

—Dans la marine. 

—Voyons, vous n'êtes pas son confesseur? 

—Non, monsieur; je le crois luthérien. 

—Comment, luthérien? 

—Je dis que je crois; je n'affirme pas. D'ailleurs, je croyais la liberté des cultes établie en France. 

—Sans doute, aussi n'est-ce point de ses croyances que nous nous occupons en ce moment, c'est de ses actions; au nom de M. le préfet de Police, je vous somme de dire ce que vous savez. 

—Il passe pour un homme fort charitable. Notre saint-père le pape l'a fait chevalier du Christ, faveur qu'il n'accorde guère qu'aux princes, pour les services éminents qu'il a rendus aux chrétiens d'Orient; il a cinq ou six grands cordons conquis par des services rendus ainsi aux princes ou aux États. 

—Et il les porte? 

—Non, mais il en est fier, il dit qu'il aime mieux les récompenses accordées aux bienfaiteurs de l'humanité que celles accordées aux destructeurs des hommes. 

—C'est donc un quaker que cet homme-là? 

—Justement, c'est un quaker, moins le grand chapeau et l'habit marron, bien entendu. 

—Lui connaît-on des amis? 

—Oui, car il a pour amis tous ceux qui le connaissent. 

—Mais enfin, il a bien quelque ennemi? 

—Un seul. 

—Comment le nommez-vous? 

—Lord Wilmore. 

—Où est-il? 

—À Paris dans ce moment même. 

—Et il peut me donner des renseignements? 

—Précieux. Il était dans l'Inde en même temps que Zaccone. 

—Savez-vous où il demeure? 

—Quelque part dans la Chaussée-d'Antin; mais j'ignore la rue et le numéro. 

—Vous êtes mal avec cet Anglais? 

—J'aime Zaccone et lui le déteste; nous sommes en froid à cause de cela. 

—Monsieur l'abbé, pensez-vous que le comte de Monte-Cristo soit jamais venu en France avant le voyage qu'il vient de faire à Paris? 

—Ah! pour cela, je puis vous répondre pertinemment. Non, monsieur, il n'y est jamais venu, puisqu'il s'est adressé à moi, il y a six mois, pour avoir les renseignements qu'il désirait. De mon côté, comme j'ignorais à quelle époque je serais moi-même de retour à Paris, je lui ai adressé M. Cavalcanti. 

—Andrea? 

—Non; Bartolomeo, le père. 

—Très bien, monsieur; je n'ai plus à vous demander qu'une chose, et je vous somme, au nom de l'honneur, de l'humanité et de la religion, de me répondre sans détour. 

—Dites, monsieur. 

—Savez-vous dans quel but M. le comte de Monte-Cristo a acheté une maison à Auteuil? 

—Certainement, car il me l'a dit. 

—Dans quel but, monsieur? 

—Dans celui d'en faire un hospice d'aliénés dans le style de celui fondé par le baron de Pisani, à Palerme. Connaissez-vous cet hospice? 

—De réputation, oui, monsieur. 

—C'est une institution magnifique.» 

Et là-dessus, l'abbé salua l'étranger en homme qui désire faire comprendre qu'il ne serait pas fâché de se remettre au travail interrompu. Le visiteur, soit qu'il comprît le désir de l'abbé, soit qu'il fût au bout de ses questions, se leva à son tour. 

L'abbé le reconduisit jusqu'à la porte. 

«Vous faites de riches aumônes, dit le visiteur, et quoiqu'on vous dise riche, j'oserai vous offrir, quelque chose pour vos pauvres; de votre côté, daignerez-vous accepter mon offrande? 

—Merci, monsieur, il n'y a qu'une seule chose dont je sois jaloux au monde, c'est que le bien que je fais vienne de moi. 

—Mais cependant\dots.  

—C'est une résolution invariable. Mais cherchez, monsieur, et vous trouverez: hélas! sur le chemin de chaque homme riche, il y a bien des misères à coudoyer!» 

L'abbé salua une dernière fois en ouvrant la porte; l'étranger salua à son tour et sortit. 

La voiture le conduisit droit chez M. de Villefort. 

Une heure après, la voiture sortit de nouveau, et cette fois se dirigea vers la rue Fontaine-Saint-Georges. Au n°5, elle s'arrêta. C'était là que demeurait Lord Wilmore. 

L'étranger avait écrit à Lord Wilmore pour lui demander un rendez-vous que celui-ci avait fixé à dix heures. Aussi, comme l'envoyé de M. le préfet de Police arriva à dix heures moins dix minutes, lui fut-il répondu que Lord Wilmore, qui était l'exactitude et la ponctualité en personne, n'était pas encore rentré, mais qu'il rentrerait pour sûr à dix heures sonnantes. 

Le visiteur attendit dans le salon. Ce salon n'avait rien de remarquable et était comme tous les salons d'hôtel garni. 

Une cheminée avec deux vases de Sèvres modernes, une pendule avec un Amour tendant son arc, une glace en deux morceaux; de chaque côté de cette glace une gravure représentant, l'une Homère portant son guide, l'autre Bélisaire demandant l'aumône, un papier gris sur gris, un meuble en drap rouge imprimé de noir: tel était le salon de Lord Wilmore. 

Il était éclairé par des globes de verre dépoli qui ne répandaient qu'une faible lumière, laquelle semblait ménagée exprès pour les yeux fatigués de l'envoyé de M. le préfet de Police. 

Au bout de dix minutes d'attente, la pendule sonna dix heures; au cinquième coup, la porte s'ouvrit, et Lord Wilmore parut. 

Lord Wilmore était un homme plutôt grand que petit, avec des favoris rares et roux, le teint blanc et les cheveux blonds grisonnants. Il était vêtu avec toute l'excentricité anglaise, c'est-à-dire qu'il portait un habit bleu à boutons d'or et haut collet piqué, comme on les portait en 1811: un gilet de casimir blanc et un pantalon de nankin de trois pouces trop court, mais que des sous-pieds de même étoffe empêchaient de remonter jusqu'aux genoux. 

Son premier mot en entrant fut: 

«Vous savez, monsieur, que je ne parle pas français. 

—Je sais, du moins, que vous n'aimez pas à parler notre langue, répondit l'envoyé de M. le préfet de Police. 

—Mais vous pouvez la parler, vous, reprit Lord Wilmore, car, si je ne la parle pas, je la comprends. 

—Et moi, reprit le visiteur en changeant d'idiome, je parle assez facilement l'anglais pour soutenir la conversation dans cette langue. Ne vous gênez donc pas, monsieur. 

—Hao!» fit Lord Wilmore avec cette intonation qui n'appartient qu'aux naturels les plus purs de la Grande-Bretagne. 

L'envoyé du préfet de Police présenta à Lord Wilmore sa lettre d'introduction. Celui-ci la lut avec un flegme tout anglican; puis, lorsqu'il eut terminé sa lecture: 

«Je comprends, dit-il en anglais; je comprends très bien.» 

Alors commencèrent les interrogations. 

Elles furent à peu près les mêmes que celles qui avaient été adressées à l'abbé Busoni. Mais comme Lord Wilmore, en sa qualité d'ennemi du comte de Monte-Cristo, n'y mettait pas la même retenue que l'abbé, elles furent beaucoup plus étendues; il raconta la jeunesse de Monte-Cristo, qui, selon lui, était, à l'âge de dix ans, entré au service d'un de ces petits souverains de l'Inde qui font la guerre aux Anglais; c'est là qu'il l'avait, lui Wilmore, rencontré pour la première fois, et qu'ils avaient combattu l'un contre l'autre. Dans cette guerre, Zaccone avait été fait prisonnier, avait été envoyé en Angleterre, mis sur les pontons, d'où il s'était enfui à la nage. Alors avaient commencé ses voyages, ses duels, ses passions; alors était arrivée l'insurrection de Grèce, il avait servi dans les rangs des Grecs. Tandis qu'il était à leur service, il avait découvert une mine d'argent dans les montagnes de la Thessalie, mais il s'était bien gardé de parler de cette découverte à personne. Après Navarin, et lorsque le gouvernement grec fut consolidé, il demanda au roi Othon un privilège d'exploitation pour cette mine, ce privilège lui fut accordé. De là cette fortune immense qui pouvait, selon Lord Wilmore monter à un ou deux millions de revenu, fortune qui néanmoins, pouvait tarir tout à coup, si la mine elle-même tarissait. 

«Mais, demanda le visiteur, savez-vous pourquoi il est venu en France? 

—Il veut spéculer sur les chemins de fer, dit Lord Wilmore; et puis, comme il est chimiste habile et physicien non moins distingué, il a découvert un nouveau télégraphe dont il poursuit l'application. 

—Combien dépense-t-il à peu près par an? demanda l'envoyé de M. le préfet de Police. 

—Oh! cinq ou six cent mille francs, tout au plus, dit Lord Wilmore; il est avare.» 

Il était évident que la haine faisait parler l'Anglais, et que, ne sachant quelle chose reprocher au comte, il lui reprochait son avarice. 

«Savez-vous quelque chose de sa maison d'Auteuil? 

—Oui, certainement. 

—Eh bien, qu'en savez-vous? 

—Vous demandez dans quel but il l'a achetée? 

—Oui. 

—Eh bien, le comte est un spéculateur qui se ruinera certainement en essais et en utopies: il prétend qu'il y a à Auteuil, dans les environs de la maison qu'il vient d'acquérir, un courant d'eau minérale qui peut rivaliser avec les eaux de Bagnères, de Luchon et de Cauterets. Il veut faire de son acquisition un \textit{badhaus} comme disent les Allemands. Il a déjà deux ou trois fois retourné tout son jardin pour retrouver le fameux cours d'eau; et comme il n'a pas pu le découvrir, vous allez le voir, d'ici à peu de temps, acheter les maisons qui environnent la sienne. Or, comme je lui en veux, j'espère que dans son chemin de fer, dans son télégraphe électrique ou dans son exploitation de bains, il va se ruiner; je le suis pour jouir de sa déconfiture, qui ne peut manquer d'arriver un jour ou l'autre. 

—Et pourquoi lui en voulez-vous? demanda le visiteur. 

—Je lui en veux, répondit Lord Wilmore, parce qu'en passant en Angleterre il a séduit la femme d'un de mes amis. 

—Mais si vous lui en voulez, pourquoi ne cherchez-vous pas à vous venger de lui? 

—Je me suis déjà battu trois fois avec le comte, dit l'Anglais: la première fois au pistolet; la seconde à l'épée; la troisième à l'espadon. 

—Et le résultat de ces duels a été? 

—La première fois, il m'a cassé le bras; la seconde fois, il m'a traversé le poumon; et la troisième, il m'a fait cette blessure.» 

L'Anglais rabattit un col de chemise qui lui montait jusqu'aux oreilles, et montra une cicatrice dont la rougeur indiquait la date peu ancienne. 

«De sorte que je lui en veux beaucoup, répéta l'Anglais, et qu'il ne mourra, bien sûr, que de ma main.  

—Mais, dit l'envoyé de la préfecture, vous ne prenez pas le chemin de le tuer, ce me semble. 

—Hao! fit l'Anglais, tous les jours je vais au tir, et tous les deux jours Grisier vient chez moi.» 

C'était ce que voulait savoir le visiteur, ou plutôt c'était tout ce que paraissait savoir l'Anglais. L'agent se leva donc, et après avoir salué Lord Wilmore, qui lui répondit avec la raideur et la politesse anglaises, il se retira. 

De son côté, Lord Wilmore, après avoir entendu se refermer sur lui la porte de la rue, rentra dans sa chambre à coucher, où, en un tour de main, il perdit ses cheveux blonds, ses favoris roux, sa fausse mâchoire et sa cicatrice pour retrouver les cheveux noirs, le teint mat et les dents de perles du comte de Monte-Cristo. 

Il est vrai que, de son côté, ce fut M. de Villefort, et non l'envoyé de M. le préfet de Police, qui rentra chez M. de Villefort. 

Le procureur du roi était un peu tranquillisé par cette double visite, qui, au reste, ne lui avait rien appris de rassurant, mais qui ne lui avait rien appris non plus d'inquiétant. Il en résulta que, pour la première fois depuis le dîner d'Auteuil, il dormit la nuit suivante avec quelque tranquillité. 