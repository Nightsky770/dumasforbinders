\chapter{Le départ}

\lettrine{L}{es} événements qui venaient de se passer préoccupaient tout Paris. Emmanuel et sa femme se les racontaient, avec une surprise bien naturelle, dans leur petit salon de la rue Meslay; ils rapprochaient ces trois catastrophes aussi soudaines qu'inattendues de Morcerf, de Danglars et de Villefort. 

Maximilien, qui était venu leur faire une visite, les écoutait ou plutôt assistait à leur conversation, plongé dans son insensibilité habituelle. 

«En vérité, disait Julie, ne dirait-on pas, Emmanuel que tous ces gens riches, si heureux hier, avaient oublié, dans le calcul sur lequel ils avaient établi leur fortune, leur bonheur et leur considération, la part du mauvais génie, et que celui-ci, comme les méchantes fées des contes de Perrault qu'on a négligé d'inviter à quelque noce ou à quelque baptême, est apparu tout à coup pour se venger de ce fatal oubli? 

—Que de désastres! disait Emmanuel pensant à Morcerf et à Danglars. 

—Que de souffrances! disait Julie, en se rappelant Valentine, que par instinct de femme elle ne voulait pas nommer devant son frère. 

—Si c'est Dieu qui les a frappés, disait Emmanuel, c'est que Dieu, qui est la suprême bonté, n'a rien trouvé dans le passé de ces gens-là qui méritât l'atténuation de la peine; c'est que ces gens-là étaient maudits. 

—N'es-tu pas bien téméraire dans ton jugement, Emmanuel? dit Julie. Quand mon père, le pistolet à la main, était prêt à se brûler la cervelle, si quelqu'un eût dit comme tu le dis à cette heure: «Cet homme a mérité sa peine», ce quelqu'un-là ne se serait-il point trompé? 

—Oui, mais Dieu n'a pas permis que notre père succombât, comme il n'a pas permis qu'Abraham sacrifiât son fils. Au patriarche, comme à nous, il a envoyé un ange qui a coupé à moitié chemin les ailes de la Mort.» 

Il achevait à peine de prononcer ces paroles que le bruit de la cloche retentit. 

C'était le signal donné par le concierge qu'une visite arrivait. 

Presque au même instant la porte du salon s'ouvrit, et le comte de Monte-Cristo parut sur le seuil. 

Ce fut un double cri de joie de la part des deux jeunes gens. 

Maximilien releva la tête et la laissa retomber. 

«Maximilien, dit le comte sans paraître remarquer les différentes impressions que sa présence produisait sur ses hôtes, je viens vous chercher. 

—Me chercher? dit Morrel comme sortant d'un rêve. 

—Oui, dit Monte-Cristo; n'est-il pas convenu que je vous emmène, et ne vous ai-je pas prévenu de vous tenir prêt? 

—Me voici, dit Maximilien, j'étais venu leur dire adieu. 

—Et où allez-vous, monsieur le comte? demanda Julie. 

—À Marseille d'abord, madame. 

—À Marseille? répétèrent ensemble les deux jeunes gens. 

—Oui, et je vous prends votre frère. 

—Hélas! monsieur le comte, dit Julie, rendez-nous-le guéri!» 

Morrel se détourna pour cacher sa rougeur. 

«Vous vous êtes donc aperçue qu'il était souffrant? dit le comte. 

—Oui, répondit la jeune femme, et j'ai peur qu'il ne s'ennuie avec nous. 

—Je le distrairai, reprit le comte. 

—Je suis prêt, monsieur, dit Maximilien. Adieu, mes bons amis! Adieu, Emmanuel! Adieu, Julie! 

—Comment! adieu? s'écria Julie; vous partez ainsi tout de suite, sans préparations, sans passeports? 

—Ce sont les délais qui doublent le chagrin des séparations, dit Monte-Cristo, et Maximilien, j'en suis sûr, a dû se précautionner de toutes choses: je le lui avais recommandé. 

—J'ai mon passeport, et mes malles sont faites, dit Morrel avec sa tranquillité monotone. 

—Fort bien, dit Monte-Cristo en souriant, on reconnaît là l'exactitude d'un bon soldat. 

—Et vous nous quittez comme cela, dit Julie, à l'instant? Vous ne nous donnez pas un jour, pas une heure? 

—Ma voiture est à la porte, madame; il faut que je sois à Rome dans cinq jours. 

—Mais Maximilien ne va pas à Rome? dit Emmanuel. 

—Je vais où il plaira au comte de me mener, dit Morrel avec un triste sourire; je lui appartiens pour un mois encore. 

—Oh! mon Dieu! comme il dit cela, monsieur le comte! 

—Maximilien m'accompagne, dit le comte avec sa persuasive affabilité, tranquillisez-vous donc sur votre frère. 

—Adieu, ma sœur! répéta Morrel; adieu, Emmanuel! 

—Il me navre le cœur avec sa nonchalance, dit Julie. Oh! Maximilien, Maximilien, tu nous caches quelque chose. 

—Bah! dit Monte-Cristo, vous le verrez revenir gai, riant et joyeux.» 

Maximilien lança à Monte-Cristo un regard presque dédaigneux, presque irrité. 

«Partons! dit le comte. 

—Avant que vous partiez, monsieur le comte, dit Julie, me permettez-vous de vous dire tout ce que l'autre jour\dots 

—Madame, répliqua le comte en lui prenant les deux mains, tout ce que vous me diriez ne vaudra jamais ce que je lis dans vos yeux, ce que votre cœur a pensé, ce que le mien a ressenti. Comme les bienfaiteurs de roman, j'eusse dû partir sans vous revoir; mais cette vertu était au-dessus de mes forces, parce que je suis un homme faible et vaniteux, parce que le regard humide, joyeux et tendre de mes semblables me fait du bien. Maintenant je pars, et je pousse l'égoïsme jusqu'à vous dire: Ne m'oubliez pas, mes amis, car probablement vous ne me reverrez jamais. 

—Ne plus vous revoir! s'écria Emmanuel, tandis que deux grosses larmes roulaient sur les joues de Julie: ne plus vous revoir! mais ce n'est donc pas un homme, c'est donc un dieu qui nous quitte, et ce dieu va donc remonter au ciel après être apparu sur la terre pour y faire le bien! 

—Ne dites pas cela, reprit vivement Monte-Cristo, ne dites jamais cela, mes amis; les dieux ne font jamais le mal, les dieux s'arrêtent où ils veulent s'arrêter; le hasard n'est pas plus fort qu'eux, et ce sont eux au contraire, qui maîtrisent le hasard. Non, je suis un homme, Emmanuel, et votre admiration est aussi injuste que vos paroles sont sacrilèges.» 

Et serrant sur ses lèvres la main de Julie, qui se précipita dans ses bras, il tendit l'autre main à Emmanuel; puis, s'arrachant de cette maison, doux nid dont le bonheur était l'hôte, il attira derrière lui d'un signe Maximilien, passif, insensible et consterné comme il l'était depuis la mort de Valentine. 

«Rendez la joie à mon frère!» dit Julie à l'oreille de Monte-Cristo. 

Monte-Cristo lui serra la main comme il la lui avait serrée onze ans auparavant sur l'escalier qui conduisait au cabinet de Morrel. 

«Vous fiez-vous toujours à Simbad le marin? lui demanda-t-il en souriant. 

—Oh! oui! 

—Eh bien, donc, endormez-vous dans la paix et dans la confiance du Seigneur.» 

Comme nous l'avons dit, la chaise de poste attendait; quatre chevaux vigoureux hérissaient leurs crins et frappaient le pavé avec impatience. 

Au bas du perron, Ali attendait le visage luisant de sueur; il paraissait arriver d'une longue course. 

«Eh bien, lui demanda le comte en arabe, as-tu été chez le vieillard?» 

Ali fit signe que oui. 

«Et tu lui as déployé la lettre sous les yeux, ainsi que je te l'avais ordonné? 

—Oui, fit encore respectueusement l'esclave. 

—Et qu'a-t-il dit, ou plutôt qu'a-t-il fait?» 

Ali se plaça sous la lumière, de façon que son maître pût le voir, et, imitant avec son intelligence si dévouée la physionomie du vieillard, il ferma les yeux comme faisait Noirtier lorsqu'il voulait dire: Oui. 

«Bien, il accepte, dit Monte-Cristo; partons!» 

Il avait à peine laissé échapper ce mot, que déjà la voiture roulait et que les chevaux faisaient jaillir du pavé une poussière d'étincelles. Maximilien s'accommoda dans son coin sans dire un seul mot. 

Une demi-heure s'écoula; la calèche s'arrêta tout à coup; le comte venait de tirer le cordonnet de soie qui correspondait au doigt d'Ali. 

Le Nubien descendit et ouvrit la portière. La nuit étincelait d'étoiles. On était au haut de la montée de Villejuif, sur le plateau d'où Paris, comme une sombre mer, agite ses millions de lumières qui paraissent des flots phosphorescents; flots en effet, flots plus bruyants, plus passionnés, plus mobiles, plus furieux, plus avides que ceux de l'Océan irrité, flots qui ne connaissent pas le calme comme ceux de la vaste mer, flots qui se heurtent toujours, écument toujours, engloutissent toujours!\dots 

Le comte demeura seul, et sur un signe de sa main la voiture fit quelques pas en avant. 

Alors il considéra longtemps, les bras croisés, cette fournaise où viennent se fondre, se tordre et se modeler toutes ces idées qui s'élancent du gouffre bouillonnant pour aller agiter le monde. Puis, lorsqu'il eut bien arrêté son regard puissant sur cette Babylone qui fait rêver les poètes religieux comme les railleurs matérialistes: 

«Grande ville! murmura-t-il en inclinant la tête et en joignant les mains comme s'il eût prié, voilà moins de six mois que j'ai franchi tes portes. Je crois que l'esprit de Dieu m'y avait conduit, il m'en ramène triomphant; le secret de ma présence dans tes murs, je l'ai confié à ce Dieu qui seul a pu lire que dans mon cœur; seul il connaît que je me retire sans haine et sans orgueil, mais non sans regrets; seul il sait que je n'ai fait usage ni pour moi, ni pour de vaines causes, de la puissance qu'il m'avait confiée. Ô grande ville! c'est dans ton sein palpitant que j'ai trouvé ce que je cherchais; mineur patient, j'ai remué tes entrailles pour en faire sortir le mal; maintenant, mon œuvre est accomplie, ma mission est terminée; maintenant tu ne peux plus m'offrir ni joies, ni douleurs. Adieu, Paris! adieu!» 

Son regard se promena encore sur la vaste plaine comme celui d'un génie nocturne; puis, passant la main sur son front, il remonta dans sa voiture, qui se referma sur lui, et qui disparut bientôt de l'autre côté de la montée dans un tourbillon de poussière et de bruit. 

Ils firent deux lieues sans prononcer une seule parole. Morrel rêvait, Monte-Cristo le regardait rêver. 

«Morrel, lui dit le comte, vous repentiriez-vous de m'avoir suivi? 

—Non, monsieur le comte; mais quitter Paris\dots 

—Si j'avais cru que le bonheur vous attendît à Paris, Morrel, je vous y eusse laissé. 

—C'est à Paris que Valentine repose, et quitter Paris, c'est la perdre une seconde fois. 

—Maximilien, dit le comte, les amis que nous avons perdus ne reposent pas dans la terre, ils sont ensevelis dans notre cœur, et c'est Dieu qui l'a voulu ainsi pour que nous en fussions toujours accompagnés. Moi, j'ai deux amis qui m'accompagnent toujours ainsi: l'un est celui qui m'a donné la vie, l'autre est celui qui m'a donné l'intelligence. Leur esprit à tous deux vit en moi. Je les consulte dans le doute, et si j'ai fait quelque bien, c'est à leurs conseils que je le dois. Consultez la voix de votre cœur, Morrel, et demandez-lui si vous devez continuer de me faire ce méchant visage. 

—Mon ami, dit Maximilien, la voix de mon cœur est bien triste et ne me promet que des malheurs. 

—C'est le propre des esprits affaiblis de voir toutes choses à travers un crêpe; c'est l'âme qui se fait à elle-même ses horizons; votre âme est sombre, c'est elle qui vous fait un ciel orageux. 

—Cela est peut-être vrai», dit Maximilien. 

Et il retomba dans sa rêverie. 

Le voyage se fit avec cette merveilleuse rapidité qui était une des puissances du comte; les villes passaient comme des ombres sur leur route; les arbres, secoués par les premiers vents de l'automne, semblaient venir au-devant d'eux comme des géants échevelés, et s'enfuyaient rapidement dès qu'ils les avaient rejoints. Le lendemain, dans la matinée, ils arrivèrent à Châlons, où les attendait le bateau à vapeur du comte; sans perdre un instant, la voiture fut transportée à bord; les deux voyageurs étaient déjà embarqués. 

Le bateau était taillé pour la course, on eût dit une pirogue indienne; ses deux roues semblaient deux ailes avec lesquelles il rasait l'eau comme un oiseau voyageur; Morrel lui-même éprouvait cette espèce d'enivrement de la vitesse; et parfois le vent qui faisait flotter ses cheveux semblait prêt pour un moment à écarter les nuages de son front. 

Quant au comte, à mesure qu'il s'éloignait de Paris, une sérénité presque surhumaine semblait l'envelopper comme une auréole. On eût dit d'un exilé qui regagne sa patrie. 

Bientôt Marseille, blanche, tiède, vivante; Marseille, la sœur cadette de Tyr et de Carthage, et qui leur a succédé à l'empire de la Méditerranée; Marseille, toujours plus jeune à mesure qu'elle vieillit, apparut à leurs yeux. C'était pour tous deux des aspects féconds en souvenirs que cette tour ronde, ce fort Saint-Nicolas, cet hôtel de ville de Puget, ce port aux quais de briques où tous deux avaient joué enfants. 

Aussi, d'un commun accord, s'arrêtèrent-ils tous deux sur la Canebière. 

Un navire partait pour Alger; les colis, les passagers entassés sur le pont, la foule des parents, des amis qui disaient adieu, qui criaient et pleuraient, spectacle toujours émouvant, même pour ceux qui assistent tous les jours à ce spectacle, ce mouvement ne put distraire Maximilien d'une idée qui l'avait saisi du moment où il avait posé le pied sur les larges dalles du quai. 

«Tenez, dit-il, prenant le bras de Monte-Cristo, voici l'endroit où s'arrêta mon père quand Le \textit{Pharaon} entra dans le port; ici le brave homme que vous sauviez de la mort et du déshonneur se jeta dans mes bras; je sens encore l'impression de ses larmes sur mon visage, et il ne pleurait pas seul, bien des gens aussi pleuraient en nous voyant. 

Monte-Cristo sourit. 

«J'étais là», dit-il en montrant à Morrel l'angle d'une rue. 

Comme il disait cela, et dans la direction qu'indiquait le comte, on entendit un gémissement douloureux, et l'on vit une femme qui faisait signe à un passager du navire en partance. Cette femme était voilée, Monte-Cristo la suivit des yeux avec une émotion que Morrel eût facilement remarquée, si, tout au contraire du comte, ses yeux à lui n'eussent été fixés sur le bâtiment. 

«Oh! mon Dieu! s'écria Morrel, je ne me trompe pas! ce jeune homme qui salue avec son chapeau, ce jeune homme en uniforme, c'est Albert de Morcerf! 

—Oui, dit Monte-Cristo, je l'avais reconnu. 

—Comment cela? vous regardiez du côté opposé.» 

Le comte sourit, comme il faisait quand il ne voulait pas répondre. 

Et ses yeux se reportèrent sur la femme voilée, qui disparut au coin de la rue. 

Alors il se retourna. 

«Cher ami, dit-il à Maximilien, n'avez-vous point quelque chose à faire dans ce pays? 

—J'ai à pleurer sur la tombe de mon père, répondit sourdement Morrel. 

—C'est bien, allez et attendez-moi là-bas; je vous y rejoindrai. 

—Vous me quittez? 

—Oui\dots moi aussi, j'ai une pieuse visite à faire.» 

Morrel laissa tomber sa main dans la main que lui tendait le comte; puis, avec un mouvement de tête dont il serait impossible d'exprimer la mélancolie, il quitta le comte et se dirigea vers l'est de la ville. 

Monte-Cristo laissa s'éloigner Maximilien, demeurant au même endroit jusqu'à ce qu'il eût disparu, puis alors il s'achemina vers les Allées de Meilhan, afin de retrouver la petite maison que les commencements de cette histoire ont dû rendre familière à nos lecteurs. 

Cette maison s'élevait encore à l'ombre de la grande allée de tilleuls qui sert de promenade aux Marseillais oisifs, tapissée de vastes rideaux de vigne qui croisaient, sur la pierre jaunie par l'ardent soleil du Midi, leurs bras noircis et déchiquetés par l'âge. Deux marches de pierre, usées par le frottement des pieds, conduisaient à la porte d'entrée, porte faite de trois planches qui jamais, malgré leurs réparations annuelles, n'avaient connu le mastic et la peinture, attendant patiemment que l'humidité revînt pour les approcher. 

Cette maison, toute charmante malgré sa vétusté, toute joyeuse malgré son apparente misère, était bien la même qu'habitait autrefois le père Dantès. Seulement le vieillard habitait la mansarde, et le comte avait mis la maison tout entière à la disposition de Mercédès. 

Ce fut là qu'entra cette femme au long voile que Monte-Cristo avait vue s'éloigner du navire en partance, elle en fermait la porte au moment même où il apparaissait à l'angle d'une rue, de sorte qu'il la vit disparaître presque aussitôt qu'il la retrouva. 

Pour lui, les marches usées étaient d'anciennes connaissances; il savait mieux que personne ouvrir cette vieille porte, dont un clou à large tête soulevait le loquet intérieur. 

Aussi entra-t-il sans frapper, sans prévenir, comme un ami, comme un hôte. 

Au bout d'une allée pavée de briques s'ouvrait, riche de chaleur, de soleil et de lumière, un petit jardin, le même où, à la place indiquée, Mercédès avait trouvé la somme dont la délicatesse du comte avait fait remonter le dépôt à vingt-quatre ans; du seuil de la porte de la rue on apercevait les premiers arbres de ce jardin. 

Arrivé sur le seuil, Monte-Cristo entendit un soupir qui ressemblait à un sanglot: ce soupir guida son regard, et sous un berceau de jasmin de Virginie au feuillage épais et aux longues fleurs de pourpre, il aperçut Mercédès assise, inclinée et pleurant. 

Elle avait relevé son voile, et seule à la face du ciel, le visage caché par ses deux mains, elle donnait librement l'essor à ses soupirs et à ses sanglots, si longtemps contenus par la présence de son fils. 

Monte-Cristo fit quelques pas en avant; le sable cria sous ses pieds. 

Mercédès releva la tête et poussa un cri d'effroi en voyant un homme devant elle. 

«Madame, dit le comte, il n'est plus en mon pouvoir de vous apporter le bonheur, mais je vous offre la consolation: daignerez-vous l'accepter comme vous venant d'un ami? 

—Je suis, en effet, bien malheureuse, répondit Mercédès; seule au monde\dots Je n'avais que mon fils, et il m'a quittée. 

—Il a bien fait, madame, répliqua le comte, c'est un noble cœur. Il a compris que tout homme doit un tribut à la patrie: les uns leurs talents, les autres leur industrie; ceux-ci leurs veilles, ceux-là leur sang. En restant avec vous, il eût usé près de vous sa vie devenue inutile, il n'aurait pu s'accoutumer à vos douleurs. Il serait devenu haineux par impuissance: il deviendra grand et fort en luttant contre son adversité qu'il changera en fortune. Laissez-le reconstituer votre avenir à tous deux, madame; j'ose vous promettre qu'il est en de sûres mains. 

—Oh! dit la pauvre femme en secouant tristement la tête, cette fortune dont vous parlez, et que du fond de mon âme je prie Dieu de lui accorder, je n'en jouirai pas, moi. Tant de choses se sont brisées en moi et autour de moi, que je me sens près de ma tombe. Vous avez bien fait, monsieur le comte, de me rapprocher de l'endroit où j'ai été si heureuse: c'est là où l'on a été heureux que l'on doit mourir. 

—Hélas! dit Monte-Cristo, toutes vos paroles, madame, tombent amères et brûlantes sur mon cœur, d'autant plus amères et plus brûlantes que vous avez raison de me haïr; c'est moi qui ai causé tous vos maux: que ne me plaignez-vous au lieu de m'accuser? vous me rendriez bien plus malheureux encore\dots 

—Vous haïr, vous accuser, vous, Edmond\dots Haïr, accuser l'homme qui a sauvé la vie de mon fils, car c'était votre intention fatale et sanglante, n'est-ce pas, de tuer à M. de Morcerf ce fils dont il était fier? Oh! regardez-moi, et vous verrez s'il y a en moi l'apparence d'un reproche.» 

Le comte souleva son regard et l'arrêta sur Mercédès qui, à moitié debout, étendait ses deux mains vers lui. 

«Oh! regardez-moi, continua-t-elle avec un sentiment de profonde mélancolie; on peut supporter l'éclat de mes yeux aujourd'hui, ce n'est plus le temps où je venais sourire à Edmond Dantès, qui m'attendait là-haut, à la fenêtre de cette mansarde qu'habitait son vieux père\dots Depuis ce temps, bien des jours douloureux se sont écoulés, qui ont creusé comme un abîme entre moi et ce temps. Vous accuser, Edmond, vous haïr, mon ami! non, c'est moi que j'accuse et que je hais! Oh! misérable que je suis! s'écria-t-elle en joignant les mains et en levant les yeux au ciel. Ai-je été punie!\dots J'avais la religion, l'innocence, l'amour, ces trois bonheurs qui font les anges, et, misérable que je suis, j'ai douté de Dieu!» 

Monte-Cristo fit un pas vers elle et silencieusement lui tendit la main. 

«Non, dit-elle en retirant doucement la sienne, non, mon ami, ne me touchez pas. Vous m'avez épargnée, et cependant de tous ceux que vous avez frappés, j'étais la plus coupable. Tous les autres ont agi par haine, par cupidité, par égoïsme; moi, j'ai agi par lâcheté. Eux désiraient, moi, j'ai eu peur. Non, ne me pressez pas ma main. Edmond, vous méditez quelque parole affectueuse, je le sens, ne la dites pas: gardez-la pour une autre, je n'en suis plus digne, moi. Voyez\dots (elle découvrit tout à fait son visage), voyez, le malheur a fait mes cheveux gris; mes yeux ont tant versé de larmes qu'ils sont cerclés de veines violettes; mon front se ride. Vous, au contraire, Edmond, vous êtes toujours jeune, toujours beau, toujours fier. C'est que vous avez eu la foi, vous; c'est que vous avez eu la force; c'est que vous vous êtes reposé en Dieu, et que Dieu vous a soutenu. Moi, j'ai été lâche, moi, j'ai renié; Dieu m'a abandonnée, et me voilà.» 

Mercédès fondit en larmes, le cœur de la femme se brisait au choc des souvenirs. 

Monte-Cristo prit sa main et la baisa respectueusement, mais elle sentit elle-même que ce baiser était sans ardeur, comme celui que le comte eût déposé sur la main de marbre de la statue d'une sainte. 

«Il y a, continua-t-elle, des existences prédestinées dont une première faute brise tout l'avenir. Je vous croyais mort, j'eusse dû mourir; car à quoi a-t-il servi que j'aie porté éternellement votre deuil dans mon cœur? à faire d'une femme de trente-neuf ans une femme de cinquante, voilà tout. À quoi a-t-il servi que, seule entre tous, vous ayant reconnu, j'aie seulement sauvé mon fils? Ne devais-je pas aussi sauver l'homme, si coupable qu'il fût, que j'avais accepté pour époux? cependant je l'ai laissé mourir; que dis-je mon Dieu! j'ai contribué à sa mort par ma lâche insensibilité, par mon mépris, ne me rappelant pas, ne voulant pas me rappeler que c'était pour moi qu'il s'était fait parjure et traître! À quoi sert enfin que j'aie accompagné mon fils jusqu'ici, puisque ici je l'abandonne, puisque je le laisse partir seul, puisque je le livre à cette terre dévorante d'Afrique? Oh! j'ai été lâche, vous dis-je; j'ai renié mon amour, et, comme les renégats, je porte malheur à tout ce qui m'environne! 

—Non, Mercédès, dit Monte-Cristo, non; reprenez meilleure opinion de vous-même. Non; vous êtes une noble et sainte femme, et vous m'aviez désarmé par votre douleur; mais, derrière moi, invisible, inconnu, irrité, il y avait Dieu, dont je n'étais que le mandataire et qui n'a pas voulu retenir la foudre que j'avais lancée. Oh! j'adjure ce Dieu, aux pieds duquel depuis dix ans je me prosterne chaque jour, j'atteste ce Dieu que je vous avais fait le sacrifice de ma vie, et avec ma vie celui des projets qui y étaient enchaînés. Mais, je le dis avec orgueil, Mercédès, Dieu avait besoin de moi, et j'ai vécu. Examinez le passé, examinez le présent, tâchez de deviner l'avenir, et voyez si je ne suis pas l'instrument du Seigneur; les plus affreux malheurs, les plus cruelles souffrances, l'abandon de tous ceux qui m'aimaient, la persécution de ceux qui ne me connaissaient pas, voilà la première partie de ma vie; puis, tout à coup, après la captivité, la solitude, là misère, l'air, la liberté, une fortune si éclatante, si prestigieuse, si démesurée, que, à moins d'être aveugle, j'ai dû penser que Dieu me l'envoyait dans de grands desseins. Dès lors, cette fortune m'a semblé être un sacerdoce; dès lors, plus une pensée en moi pour cette vie dont vous, pauvre femme, vous avez parfois savouré la douceur; pas une heure de calme, pas une: je me sentais poussé comme le nuage de feu passant dans le ciel pour aller brûler les villes maudites. Comme ces aventureux capitaines qui s'embarquent pour un dangereux voyage, qui méditent une périlleuse expédition, je préparais les vivres, je chargeais les armes, j'amassais les moyens d'attaque et de défense, habituant mon corps aux exercices les plus violents, mon âme aux chocs les plus rudes, instruisant mon bras à tuer, mes yeux à voir souffrir, ma bouche à sourire aux aspects les plus terribles; de bon, de confiant, d'oublieux que j'étais, je me suis fait vindicatif, dissimulé, méchant, ou plutôt impassible comme la sourde et aveugle fatalité. Alors, je me suis lancé dans la voie qui m'était ouverte, j'ai franchi l'espace, j'ai touché au but: malheur à ceux que j'ai rencontrés sur mon chemin! 

—Assez! dit Mercédès, assez, Edmond! croyez que celle qui a pu seule vous reconnaître a pu seule aussi vous comprendre. Or, Edmond, celle qui a su vous reconnaître, celle qui a pu vous comprendre, celle-là, l'eussiez-vous rencontrée sur votre route et l'eussiez-vous brisée comme verre, celle-là a dû vous admirer, Edmond! Comme il y a un abîme entre moi et le passé, il y a un abîme entre vous et les autres hommes, et ma plus douloureuse torture, je vous le dis, c'est de comparer; car il n'y a rien au monde qui vous vaille, rien qui vous ressemble. Maintenant, dites-moi adieu, Edmond, et séparons-nous. 

—Avant que je vous quitte, que désirez-vous, Mercédès? demanda Monte-Cristo. 

—Je ne désire qu'une chose, Edmond: que mon fils soit heureux. 

—Priez le Seigneur, qui seul tient l'existence des hommes entre ses mains, d'écarter la mort de lui, moi, je me charge du reste. 

—Merci, Edmond. 

—Mais vous, Mercédès? 

—Moi je n'ai besoin de rien, je vis entre deux tombes: l'une est celle d'Edmond Dantès, mort il y a si longtemps; je l'aimais! Ce mot ne sied plus à ma lèvre flétrie, mais mon cœur se souvient encore, et pour rien au monde je ne voudrais perdre cette mémoire du cœur. L'autre est celle d'un homme qu'Edmond Dantès a tué; j'approuve le meurtre, mais je dois prier pour le mort. 

—Votre fils sera heureux, madame, répéta le comte. 

—Alors je serai aussi heureuse que je puis l'être. 

—Mais\dots enfin\dots que ferez-vous?» 

Mercédès sourit tristement. 

«Vous dire que je vivrai dans ce pays comme la Mercédès d'autrefois, c'est-à-dire en travaillant, vous ne le croiriez pas; je ne sais plus que prier, mais je n'ai point besoin de travailler; le petit trésor enfoui par vous s'est retrouvé à la place que vous avez indiquée; on cherchera qui je suis, on demandera ce que je fais, on ignorera comment je vis, qu'importe! c'est une affaire entre Dieu, vous et moi. 

—Mercédès, dit le comte, je ne vous en fais pas un reproche, mais vous avez exagéré le sacrifice en abandonnant toute cette fortune amassée par M. de Morcerf, et dont la moitié revenait de droit à votre économie et à votre vigilance. 

—Je vois ce que vous m'allez proposer; mais je ne puis accepter, Edmond, mon fils me le défendrait. 

—Aussi me garderai-je de rien faire pour vous qui n'ait l'approbation de M. Albert de Morcerf. Je saurai ses intentions et m'y soumettrai. Mais, s'il accepte ce que je veux faire, l'imiterez-vous sans répugnance? 

—Vous savez, Edmond, que je ne suis plus une créature pensante; de détermination, je n'en ai pas sinon celle de n'en prendre jamais. Dieu m'a tellement secouée dans ses orages que j'en ai perdu la volonté. Je suis entre ses mains comme un passereau aux serres de l'aigle. Il ne veut pas que je meure puisque je vis. S'il m'envoie des secours, c'est qu'il le voudra et je les prendrai. 

—Prenez garde, madame, dit Monte-Cristo, ce n'est pas ainsi qu'on adore Dieu! Dieu veut qu'on le comprenne et qu'on discute sa puissance: c'est pour cela qu'il nous a donné le libre arbitre. 

—Malheureux! s'écria Mercédès, ne me parlez pas ainsi; si je croyais que Dieu m'eût donné le libre arbitre, que me resterait-il donc pour me sauver du désespoir!» 

Monte-Cristo pâlit légèrement et baissa la tête, écrasé par cette véhémence de la douleur. 

«Ne voulez-vous pas me dire au revoir? fit-il en lui tendant la main. 

—Au contraire, je vous dis au revoir, répliqua Mercédès en lui montrant le ciel avec solennité; c'est vous prouver que j'espère encore.» 

Et après avoir touché la main du comte de sa main frissonnante, Mercédès s'élança dans l'escalier et disparut aux yeux du comte. 

Monte-Cristo alors sortit lentement de la maison et reprit le chemin du port. 

Mais Mercédès ne le vit point s'éloigner, quoiqu'elle fût à la fenêtre de la petite chambre du père de Dantès. Ses yeux cherchaient au loin le bâtiment qui emportait son fils vers la vaste mer. 

Il est vrai que sa voix, comme malgré elle, murmurait tout bas: 

«Edmond, Edmond, Edmond!» 