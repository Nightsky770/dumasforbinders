\chapter{La mère et le fils}

\lettrine{L}{e} comte de Monte-Cristo salua les cinq jeunes gens avec un sourire plein de mélancolie et de dignité, et remonta dans sa voiture avec Maximilien et Emmanuel. 

\zz
Albert, Beauchamp et Château-Renaud restèrent seuls sur le champ de bataille. 

Le jeune homme attacha sur ses deux témoins un regard qui, sans être timide, semblait pourtant leur demander leur avis sur ce qui venait de se passer. 

«Ma foi! mon cher ami, dit Beauchamp le premier, soit qu'il eût plus de sensibilité, soit qu'il eût moins de dissimulation, permettez-moi de vous féliciter: voilà un dénouement bien inespéré à une bien désagréable affaire.» 

Albert resta muet et concentré dans sa rêverie. Château-Renaud se contenta de battre sa botte avec sa canne flexible. 

«Ne partons-nous pas? dit-il après ce silence embarrassant. 

—Quand il vous plaira, répondit Beauchamp; laissez-moi seulement le temps de complimenter M. de Morcerf; il a fait preuve aujourd'hui d'une générosité si chevaleresque\dots si rare! 

—Oh! oui, dit Château-Renaud. 

—C'est magnifique, continua Beauchamp, de pouvoir conserver sur soi-même un empire aussi grand! 

—Assurément: quant à moi, j'en eusse été incapable, dit Château-Renaud avec une froideur des plus significatives. 

—Messieurs, interrompit Albert, je crois que vous n'avez pas compris qu'entre M. de Monte-Cristo et moi il s'est passé quelque chose de bien grave\dots 

—Si fait, si fait, dit aussitôt Beauchamp, mais tous nos badauds ne seraient pas à portée de comprendre votre héroïsme, et, tôt ou tard, vous vous verriez forcé de le leur expliquer plus énergiquement qu'il ne convient à la santé de votre corps et à la durée de votre vie. Voulez-vous que je vous donne un conseil d'ami? Partez pour Naples, La Haye ou Saint-Pétersbourg, pays calmes, où l'on est plus intelligent du point d'honneur que chez nos cerveaux brûlés de Parisiens. Une fois là, faites pas mal de mouches au pistolet, et infiniment de contres de quarte et de contres de tierce; rendez-vous assez oublié pour revenir paisiblement en France dans quelques années, ou assez respectable, quant aux exercices académiques, pour conquérir votre tranquillité. N'est-ce pas, monsieur de Château-Renaud, que j'ai raison? 

—C'est parfaitement mon avis, dit le gentilhomme. Rien n'appelle les duels sérieux comme un duel sans résultat. 

—Merci, messieurs, répondit Albert avec un froid sourire; je suivrai votre conseil, non parce que vous me le donnez, mais parce que mon intention était de quitter la France. Je vous remercie également du service que vous m'avez rendu en me servant de témoins. Il est bien profondément gravé dans mon cœur, puisque, après les paroles que je viens d'entendre, je ne me souviens plus que de lui.» 

Château-Renaud et Beauchamp se regardèrent. L'impression était la même sur tous deux, et l'accent avec lequel Morcerf venait de prononcer son remerciement était empreint d'une telle résolution, que la position fût devenue embarrassante pour tous si la conversation eût continué. 

«Adieu, Albert», fit tout à coup Beauchamp en tendant négligemment la main au jeune homme, sans que celui-ci parût sortir de sa léthargie. 

En effet, il ne répondit rien à l'offre de cette main. 

«Adieu», dit à son tour Château-Renaud, gardant à la main gauche sa petite canne, et saluant de la main droite. 

Les lèvres d'Albert murmurèrent à peine: «Adieu!» Son regard était plus explicite; il renfermait tout un poème de colères contenues, de fiers dédains, de généreuse indignation. 

Lorsque ses deux témoins furent remontés en voiture, il garda quelque temps sa pose immobile et mélancolique; puis soudain, détachant son cheval du petit arbre autour duquel son domestique avait noué le bridon, il sauta légèrement en selle, et reprit au galop le chemin de Paris. Un quart d'heure après, il rentrait à l'hôtel de la rue du Helder. 

En descendant de cheval, il lui sembla, derrière le rideau de la chambre à coucher du comte, apercevoir le visage pâle de son père; Albert détourna la tête avec un soupir et rentra dans son petit pavillon. 

Arrivé là, il jeta un dernier regard sur toutes ces richesses qui lui avaient fait la vie si douce et si heureuse depuis son enfance; il regarda encore une fois ces tableaux, dont les figures semblaient lui sourire, et dont les paysages parurent s'animer de vivantes couleurs. 

Puis il enleva de son châssis de chêne le portrait de sa mère, qu'il roula, laissant vide et noir le cadre d'or qui l'entourait. 

Puis il mit en ordre ses belles armes turques, ses beaux fusils anglais, ses porcelaines japonaises, ses coupes montées, ses bronzes artistiques, signés Feuchères ou Barye, visita les armoires et plaça les clefs à chacune d'elles; jeta dans un tiroir de son secrétaire qu'il laissa ouvert, tout l'argent de poche qu'il avait sur lui, y joignit les mille bijoux de fantaisie qui peuplaient ses coupes, ses écrins, ses étagères; fit un inventaire exact et précis de tout, et plaça cet inventaire à l'endroit le plus apparent d'une table, après avoir débarrassé cette table des livres et des papiers qui l'encombraient. 

Au commencement de ce travail, son domestique malgré l'ordre que lui avait donné Albert de le laisser seul, était entré dans sa chambre. 

«Que voulez-vous? lui demanda Morcerf d'un accent plus triste que courroucé. 

—Pardon, monsieur, dit le valet de chambre, monsieur m'avait bien défendu de le déranger, c'est vrai mais M. le comte de Morcerf m'a fait appeler. 

—Eh bien? demanda Albert. 

—Je n'ai pas voulu me rendre chez M. le comte sans prendre les ordres de monsieur. 

—Pourquoi cela? 

—Parce que M. le comte sait sans doute que j'ai accompagné monsieur sur le terrain. 

—C'est probable, dit Albert. 

—Et s'il me fait demander, c'est sans doute pour m'interroger sur ce qui s'est passé là-bas. Que dois-je répondre? 

—La vérité. 

—Alors je dirai que la rencontre n'a pas eu lieu! 

—Vous direz que j'ai fait des excuses à M. le comte de Monte-Cristo, allez.» 

Le valet s'inclina et sortit. 

Albert s'était alors remis à son inventaire. 

Comme il terminait ce travail, le bruit de chevaux piétinant dans la cour et des roues d'une voiture ébranlant les vitres attira son attention, il s'approcha de la fenêtre, et vit son père monter dans sa calèche et partir. 

À peine la porte de l'hôtel fut-elle refermée derrière le comte, qu'Albert se dirigea vers l'appartement de sa mère, et comme personne n'était là pour l'annoncer, il pénétra jusqu'à la chambre de Mercédès, et, le cœur gonflé de ce qu'il voyait et de ce qu'il devinait, il s'arrêta sur le seuil. 

Comme si la même âme eût animé ces deux corps, Mercédès faisait chez elle ce qu'Albert venait de faire chez lui. Tout était mis en ordre: les dentelles, les parures, les bijoux, le linge, l'argent, allaient se ranger au fond des tiroirs, dont la comtesse assemblait soigneusement les clefs. 

Albert vit tous ces préparatifs; il les comprit, et s'écriant: «Ma mère!» il alla jeter ses bras au cou de Mercédès. 

Le peintre qui eût pu rendre l'expression de ces deux figures eût fait certes un beau tableau. 

En effet, tout cet appareil d'une résolution énergique qui n'avait point fait peur à Albert pour lui-même l'effrayait pour sa mère. 

«Que faites-vous donc? demanda-t-il. 

—Que faisiez-vous? répondit-elle. 

—Ô ma mère! s'écria Albert, ému au point de ne pouvoir parler, il n'est point de vous comme de moi! Non, vous ne pouvez pas avoir résolu ce que j'ai décidé, car je viens vous prévenir que je dis adieu à votre maison, et\dots et à vous. 

—Moi aussi, Albert, répondit Mercédès; moi aussi, je pars. J'avais compté, je l'avoue, que mon fils m'accompagnerait; me suis-je trompée? 

—Ma mère, dit Albert avec fermeté, je ne puis vous faire partager le sort que je me destine: il faut que je vive désormais sans nom et sans fortune; il faut, pour commencer l'apprentissage de cette rude existence, que j'emprunte à un ami le pain que je mangerai d'ici au moment où j'en gagnerai d'autre. Ainsi, ma bonne mère, je vais de ce pas chez Franz le prier de me prêter la petite somme que j'ai calculé m'être nécessaire. 

—Toi, mon pauvre enfant! s'écria Mercédès; toi souffrir de la misère, souffrir de la faim! Oh! ne dis pas cela, tu briseras toutes mes résolutions. 

—Mais non pas les miennes, ma mère, répondit Albert. Je suis jeune, je suis fort, je crois que je suis brave, et depuis hier j'ai appris ce que peut la volonté. Hélas! ma mère, il y a des gens qui ont tant souffert, et qui non seulement ne sont pas morts mais qui encore ont édifié une nouvelle fortune sur la ruine de toutes les promesses de bonheur que le ciel leur avait faites, sur les débris de toutes les espérances que Dieu leur avait données! J'ai appris cela, ma mère, j'ai vu ces hommes; je sais que du fond de l'abîme où les avait plongés leur ennemi, ils se sont relevés avec tant de vigueur et de gloire, qu'ils ont dominé leur ancien vainqueur et l'ont précipité à son tour. Non, ma mère, non; j'ai rompu, à partir d'aujourd'hui, avec le passé et je n'en accepte plus rien, pas même mon nom, parce que, vous le comprenez, vous, n'est-ce pas, ma mère? votre fils ne peut porter le nom d'un homme qui doit rougir devant un autre homme! 

—Albert, mon enfant, dit Mercédès, si j'avais eu un cœur plus fort, c'est là le conseil que je t'eusse donné; ta conscience a parlé quand ma voix éteinte se taisait; écoute ta conscience, mon fils. Tu avais des amis, Albert, romps momentanément avec eux, mais ne désespère pas, au nom de ta mère! La vie est belle encore à ton âge, mon cher Albert, car à peine as-tu vingt-deux ans; et comme à un cœur aussi pur que le tien il faut un nom sans tache, prends celui de mon père: il s'appelait Herrera. Je te connais, mon Albert; quelque carrière que tu suives, tu rendras en peu de temps ce nom illustre. Alors, mon ami, reparais dans le monde plus brillant encore de tes malheurs passés; et si cela ne doit pas être ainsi, malgré toutes mes prévisions, laisse-moi du moins cet espoir, à moi qui n'aurai plus que cette seule pensée, à moi qui n'ai plus d'avenir, et pour qui la tombe commence au seuil de cette maison. 

—Je ferai selon vos désirs, ma mère, dit le jeune homme; oui, je partage votre espoir: la colère du ciel ne nous poursuivra pas, vous si pure, moi si innocent. Mais puisque nous sommes résolus, agissons promptement. M. de Morcerf a quitté l'hôtel voilà une demi-heure à peu près; l'occasion, comme vous le voyez, est favorable pour éviter le bruit et l'explication. 

—Je vous attends, mon fils», dit Mercédès. 

Albert courut aussitôt jusqu'au boulevard, d'où il ramena un fiacre qui devait les conduire hors de l'hôtel, il se rappelait certaine petite maison garnie dans la rue des Saints-Pères, où sa mère trouverait un logement modeste, mais décent; il revint donc chercher la comtesse. 

Au moment où le fiacre s'arrêta devant la porte, et comme Albert en descendait, un homme s'approcha de lui et lui remit une lettre. 

Albert reconnut l'intendant. 

«Du comte», dit Bertuccio. 

Albert prit la lettre, l'ouvrit, la lut. 

Après l'avoir lue, il chercha des yeux Bertuccio, mais, pendant que le jeune homme lisait, Bertuccio avait disparu. 

Alors Albert, les larmes aux yeux, la poitrine toute gonflée d'émotion, rentra chez Mercédès, et, sans prononcer une parole, lui présenta la lettre. 

Mercédès lut: 

«Albert, 

«En vous montrant que j'ai pénétré le projet auquel vous êtes sur le point de vous abandonner, je crois vous montrer aussi que je comprends la délicatesse. Vous voilà libre, vous quittez l'hôtel du comte, et vous allez retirer chez vous votre mère, libre comme vous; mais, réfléchissez-y, Albert, vous lui devez plus que vous ne pouvez lui payer, pauvre noble cœur que vous êtes. Gardez pour vous la lutte, réclamez pour vous la souffrance, mais épargnez-lui cette première misère qui accompagnera inévitablement vos premiers efforts; car elle ne mérite pas même le reflet du malheur qui la frappe aujourd'hui, et la Providence ne veut pas que l'innocent paie pour le coupable. 

«Je sais que vous allez quitter tous deux la maison de la rue du Helder sans rien emporter. Comment je l'ai appris, ne cherchez point à le découvrir. Je le sais: voilà tout. 

«Écoutez, Albert. 

«Il y a vingt-quatre ans, je revenais bien joyeux et bien fier dans ma patrie. J'avais une fiancée, Albert, une sainte jeune fille que j'adorais, et je rapportais à ma fiancée cent cinquante louis amassés péniblement par un travail sans relâche. Cet argent était pour elle, je le lui destinais, et sachant combien la mer est perfide, j'avais enterré notre trésor dans le petit jardin de la maison que mon père habitait à Marseille, sur les Allées de Meilhan. 

«Votre mère, Albert, connaît bien cette pauvre chère maison. 

«Dernièrement, en venant à Paris, j'ai passé par Marseille. Je suis allé voir cette maison aux douloureux souvenirs; et le soir, une bêche à la main, j'ai sondé le coin où j'avais enfoui mon trésor. La cassette de fer était encore à la même place, personne n'y avait touché; elle est dans l'angle qu'un beau figuier, planté par mon père le jour de ma naissance, couvre de son ombre. 

«Eh bien, Albert, cet argent qui autrefois devait aider à la vie et à la tranquillité de cette femme que j'adorais, voilà qu'aujourd'hui, par un hasard étrange et douloureux, il a retrouvé le même emploi. Oh! comprenez bien ma pensée, à moi qui pourrais offrir des millions à cette pauvre femme, et qui lui rends seulement le morceau de pain noir oublié sous mon pauvre toit depuis le jour où j'ai été séparé de celle que j'aimais. 

«Vous êtes un homme généreux, Albert, mais peut-être êtes-vous néanmoins aveuglé par la fierté ou par le ressentiment; si vous me refusez, si vous demandez à un autre ce que j'ai le droit de vous offrir, je dirai qu'il est peu généreux à vous de refuser la vie de votre mère offerte par un homme dont votre père a fait mourir le père dans les horreurs de la faim et du désespoir.» 

Cette lecture finie, Albert demeura pâle et immobile en attendant ce que déciderait sa mère. 

Mercédès leva au ciel un regard d'une ineffable expression. 

«J'accepte, dit-elle; il a le droit de payer la dot que j'apporterai dans un couvent!» 

Et, mettant la lettre sur son cœur, elle prit le bras de son fils, et d'un pas plus ferme qu'elle ne s'y attendait peut-être elle-même, elle prit le chemin de l'escalier. 