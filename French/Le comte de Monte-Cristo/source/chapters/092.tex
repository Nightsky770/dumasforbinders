\chapter{Le suicide}

\lettrine{C}{ependant} Monte-Cristo, lui aussi, était rentré en ville avec Emmanuel et Maximilien. 

\zz
Le retour fut gai. Emmanuel ne dissimulait pas sa joie d'avoir vu succéder la paix à la guerre, et avouait hautement ses goûts philanthropiques. Morrel, dans un coin de la voiture, laissait la gaieté de son beau-frère s'évaporer en paroles, et gardait pour lui une joie tout aussi sincère, mais qui brillait seulement dans ses regards. 

À la barrière du Trône, on rencontra Bertuccio: il attendait là, immobile comme une sentinelle à son poste. 

Monte-Cristo passa la tête par la portière, échangea avec lui quelques paroles à voix basse, et l'intendant disparut. 

«Monsieur le comte, dit Emmanuel en arrivant à la hauteur de la place Royale, faites-moi jeter, je vous prie, à ma porte, afin que ma femme ne puisse avoir un seul moment d'inquiétude ni pour vous ni pour moi. 

—S'il n'était ridicule d'aller faire montre de son triomphe, dit Morrel, j'inviterais M. le comte à entrer chez nous, mais M. le comte aussi a sans doute des cœurs tremblants à rassurer. Nous voici arrivés, Emmanuel, saluons notre ami, et laissons-le continuer son chemin. 

—Un moment, dit Monte-Cristo, ne me privez pas ainsi d'un seul coup de mes deux compagnons; rentrez auprès de votre charmante femme, à laquelle je vous charge de présenter tous mes compliments, et accompagnez-moi jusqu'aux Champs-Élysées, Morrel. 

—À merveille, dit Maximilien, d'autant plus que j'ai affaire dans votre quartier, comte. 

—T'attendra-t-on pour déjeuner? demanda Emmanuel. 

—Non», dit le jeune homme. 

La portière se referma, la voiture continua sa route. 

«Voyez comme je vous ai porté bonheur, dit Morrel lorsqu'il fut seul avec le comte. N'y avez-vous pas pensé? 

—Si fait, dit Monte-Cristo, voilà pourquoi je voudrais toujours vous tenir près de moi. 

—C'est miraculeux! continua Morrel, répondant à sa propre pensée. 

—Quoi donc? dit Monte-Cristo. 

—Ce qui vient de se passer. 

—Oui, répondit le comte avec un sourire; vous avez dit le mot, Morrel, c'est miraculeux! 

—Car enfin, reprit Morrel, Albert est brave. 

—Très brave, dit Monte-Cristo, je l'ai vu dormir le poignard suspendu sur sa tête. 

—Et, moi, je sais qu'il s'est battu deux fois, et très bien battu, dit Morrel; conciliez donc cela avec la conduite de ce matin. 

—Votre influence, toujours, reprit en souriant Monte-Cristo. 

—C'est heureux pour Albert qu'il ne soit point soldat, dit Morrel. 

—Pourquoi cela? 

—Des excuses sur le terrain! fit le jeune capitaine en secouant la tête. 

—Allons, dit le comte avec douceur, n'allez-vous point tomber dans les préjugés des hommes ordinaires, Morrel? Ne conviendrez-vous pas que puisque Albert est brave, il ne peut être lâche; qu'il faut qu'il ait eu quelque raison d'agir comme il l'a fait ce matin, et que partant sa conduite est plutôt héroïque qu'autre chose? 

—Sans doute, sans doute, répondit Morrel, mais je dirai comme l'Espagnol: il a été moins brave aujourd'hui qu'hier. 

—Vous déjeunez avec moi, n'est-ce pas, Morrel? dit le comte pour couper court à la conversation. 

—Non pas, je vous quitte à dix heures. 

—Votre rendez-vous était donc pour déjeuner?» 

Morrel sourit et secoua la tête. 

«Mais, enfin, faut-il toujours que vous déjeuniez quelque part? 

—Cependant, si je n'ai pas faim? dit le jeune homme. 

—Oh! fit le comte, je ne connais que deux sentiments qui coupent ainsi l'appétit: la douleur (et comme heureusement je vous vois très gai, ce n'est point cela) et l'amour. Or, d'après ce que vous m'avez dit à propos de votre cœur, il m'est permis de croire\dots 

—Ma foi, comte, répliqua gaiement Morrel, je ne dis pas non. 

—Et vous ne me contez pas cela, Maximilien? reprit le comte d'un ton si vif, que l'on voyait tout l'intérêt qu'il eût pris à connaître ce secret. 

—Je vous ai montré ce matin que j'avais un cœur, n'est-ce pas, comte?» 

Pour toute réponse Monte-Cristo tendit la main au jeune homme. 

«Eh bien, continua celui-ci, depuis que ce cœur n'est plus avec vous au bois de Vincennes, il est autre part où je vais le retrouver. 

—Allez, dit lentement le comte, allez, cher ami, mais par grâce, si vous éprouviez quelque obstacle, rappelez-vous que j'ai quelque pouvoir en ce monde, que je suis heureux d'employer ce pouvoir au profit des gens que j'aime, et que je vous aime, vous, Morrel. 

—Bien, dit le jeune homme, je m'en souviendrai comme les enfants égoïstes se souviennent de leurs parents quand ils ont besoin d'eux. Quand j'aurai besoin de vous, et peut-être ce moment viendra-t-il, je m'adresserai à vous, comte. 

—Bien, je retiens votre parole. Adieu donc. 

—Au revoir.» 

On était arrivé à la porte de la maison des Champs-Élysées, Monte-Cristo ouvrit la portière. Morrel sauta sur le pavé. 

Bertuccio attendait sur le perron. 

Morrel disparut par l'avenue de Marigny et Monte-Cristo marcha vivement au-devant de Bertuccio. 

«Eh bien? demanda-t-il. 

—Eh bien, répondit l'intendant, elle va quitter sa maison. 

—Et son fils? 

—Florentin, son valet de chambre, pense qu'il en va faire autant. 

—Venez.» 

Monte-Cristo emmena Bertuccio dans son cabinet, écrivit la lettre que nous avons vue, et la remit à l'intendant. 

«Allez, dit-il, et faites diligence; à propos, faites prévenir Haydée que je suis rentré. 

—Me voilà», dit la jeune fille, qui, au bruit de la voiture, était déjà descendue, et dont le visage rayonnait de joie en revoyant le comte sain et sauf. 

Bertuccio sortit. 

Tous les transports d'une fille revoyant un père chéri, tous les délires d'une maîtresse revoyant un amant adoré, Haydée les éprouva pendant les premiers instants de ce retour attendu par elle avec tant d'impatience. 

Certes, pour être moins expansive, la joie de Monte-Cristo n'était pas moins grande; la joie pour les cœurs qui ont longtemps souffert est pareille à la rosée pour les terres desséchées par le soleil; cœur et terre absorbent cette pluie bienfaisante qui tombe sur eux, et rien n'en apparaît au-dehors. Depuis quelques jours, Monte-Cristo comprenait une chose que depuis longtemps il n'osait plus croire, c'est qu'il y avait deux Mercédès au monde, c'est qu'il pouvait encore être heureux. 

Son œil ardent de bonheur se plongeait avidement dans les regards humides d'Haydée, quand tout à coup la porte s'ouvrit. Le comte fronça le sourcil. 

«M. de Morcerf!» dit Baptistin, comme si ce mot seul renfermait son excuse. 

En effet, le visage du comte s'éclaira. 

«Lequel, demanda-t-il, le vicomte ou le comte? 

—Le comte. 

—Mon Dieu! s'écria Haydée, n'est-ce donc point fini encore? 

—Je ne sais si c'est fini, mon enfant bien-aimée, dit Monte-Cristo en prenant les mains de la jeune fille, mais ce que je sais, c'est que tu n'as rien à craindre. 

—Oh! c'est cependant le misérable\dots 

—Cet homme ne peut rien sur moi, Haydée, dit Monte-Cristo; c'est quand j'avais affaire à son fils qu'il fallait craindre. 

—Aussi, ce que j'ai souffert, dit la jeune fille, tu ne le sauras jamais, mon seigneur.» 

Monte-Cristo sourit. 

«Par la tombe de mon père! dit Monte-Cristo en étendant la main sur la tête de la jeune fille, je te jure que s'il arrive malheur, ce ne sera point à moi. 

—Je te crois, mon seigneur, comme si Dieu me parlait», dit la jeune fille en présentant son front au comte. 

Monte-Cristo déposa sur ce front si pur et si beau un baiser qui fit battre à la fois deux cœurs, l'un avec violence, l'autre sourdement. 

«Oh! mon Dieu! murmura le comte, permettriez-vous donc que je puisse aimer encore!\dots Faites entrer M. le comte de Morcerf au salon», dit-il à Baptistin, tout en conduisant la belle Grecque vers un escalier dérobé. 

Un mot d'explication sur cette visite, attendue peut-être de Monte-Cristo, mais inattendue sans doute pour nos lecteurs. 

Tandis que Mercédès, comme nous l'avons dit, faisait chez elle l'espèce d'inventaire qu'Albert avait fait chez lui; tandis qu'elle classait ses bijoux, fermait ses tiroirs, réunissait ses clefs, afin de laisser toutes choses dans un ordre parfait, elle ne s'était pas aperçue qu'une tête pâle et sinistre était venue apparaître au vitrage d'une porte qui laissait entrer le jour dans le corridor; de là, non seulement on pouvait voir, mais on pouvait entendre. Celui qui regardait ainsi, selon toute probabilité, sans être vu ni entendu, vit donc et entendit donc tout ce qui se passait chez Mme de Morcerf. 

De cette porte vitrée, l'homme au visage pâle se transporta dans la chambre à coucher du comte de Morcerf, et, arrivé là, souleva d'une main contractée le rideau d'une fenêtre donnant sur la cour. Il resta là dix minutes ainsi immobile, muet, écoutant les battements de son propre cœur. Pour lui c'était bien long, dix minutes. 

Ce fut alors qu'Albert, revenant de son rendez-vous, aperçut son père, qui guettait son retour derrière un rideau et détourna la tête. 

L'œil du comte se dilata: il savait que l'insulte d'Albert à Monte-Cristo avait été terrible, qu'une pareille insulte, dans tous les pays du monde, entraînait un duel à mort. Or, Albert rentrait sain et sauf, donc le comte était vengé. 

Un éclair de joie indicible illumina ce visage lugubre, comme fait un dernier rayon de soleil avant de se perdre dans les nuages qui semblent moins sa couche que son tombeau. 

Mais, nous l'avons dit, il attendit en vain que le jeune homme montât à son appartement pour lui rendre compte de son triomphe. Que son fils, avant de combattre, n'ait pas voulu voir le père dont il allait venger l'honneur, cela se comprend; mais, l'honneur du père vengé, pourquoi ce fils ne venait-il point se jeter dans ses bras? 

Ce fut alors que le comte, ne pouvant voir Albert, envoya chercher son domestique. On sait qu'Albert l'avait autorisé à ne rien cacher au comte. 

Dix minutes après on vit apparaître sur le perron le général de Morcerf, vêtu d'une redingote noire, ayant un col militaire, un pantalon noir, des gants noirs. Il avait donné, à ce qu'il paraît, des ordres antérieurs; car, à peine eut-il touché le dernier degré du perron, que sa voiture tout attelée sortit de la remise et vint s'arrêter devant lui. 

Son valet de chambre vint alors jeter dans la voiture un caban militaire, raidi par les deux épées qu'il enveloppait; puis fermant la portière, il s'assit près du cocher. 

Le cocher se pencha devant la calèche pour demander l'ordre: 

«Aux Champs-Élysées, dit le général, chez le comte de Monte-Cristo. Vite!» 

Les chevaux bondirent sous le coup de fouet qui les enveloppa; cinq minutes après, ils s'arrêtèrent devant la maison du comte. 

M. de Morcerf ouvrit lui-même la portière, et, la voiture roulant encore, il sauta comme un jeune homme dans la contre-allée, sonna et disparut dans la porte béante avec son domestique. 

Une seconde après, Baptistin annonçait à M. de Monte-Cristo le comte de Morcerf, et Monte-Cristo, reconduisant Haydée, donna l'ordre qu'on fît entrer le comte de Morcerf dans le salon. 

Le général arpentait pour la troisième fois le salon dans toute sa longueur, lorsqu'en se retournant il aperçut Monte-Cristo debout sur le seuil. 

«Eh! c'est monsieur de Morcerf, dit tranquillement Monte-Cristo; je croyais avoir mal entendu. 

—Oui c'est moi-même, dit le comte avec une effroyable contraction des lèvres qui l'empêchait d'articuler nettement. 

—Il ne me reste donc qu'à savoir maintenant, dit Monte-Cristo, la cause qui me procure le plaisir de voir monsieur le comte de Morcerf de si bonne heure. 

—Vous avez eu ce matin une rencontre avec mon fils, monsieur? dit le général. 

—Vous savez cela? répondit le comte. 

—Et je sais aussi que mon fils avait de bonnes raisons pour désirer se battre contre vous et faire tout ce qu'il pourrait pour vous tuer. 

—En effet, monsieur, il en avait de fort bonnes! mais vous voyez que, malgré ces raisons-là, il ne m'a pas tué, et même qu'il ne s'est pas battu. 

—Et cependant il vous regardait comme la cause du déshonneur de son père, comme la cause de la ruine effroyable qui, en ce moment-ci, accable ma maison. 

—C'est vrai, monsieur, dit Monte-Cristo avec son calme terrible; cause secondaire, par exemple, et non principale. 

—Sans doute vous lui avez fait quelque excuse ou donné quelque explication? 

—Je ne lui ai donné aucune explication, et c'est lui qui m'a fait des excuses. 

—Mais à quoi attribuez-vous cette conduite? 

—À la conviction, probablement, qu'il y avait dans tout ceci un homme plus coupable que moi. 

—Et quel était cet homme? 

—Son père. 

—Soit, dit le comte en pâlissant; mais vous savez que le coupable n'aime pas à s'entendre convaincre de culpabilité. 

—Je sais\dots Aussi je m'attendais à ce qui arrive en ce moment. 

—Vous vous attendiez à ce que mon fils fût un lâche! s'écria le comte. 

—M. Albert de Morcerf n'est point un lâche, dit Monte-Cristo. 

—Un homme qui tient à la main une épée, un homme qui, à la portée de cette épée, tient un ennemi mortel, cet homme, s'il ne se bat pas, est un lâche! Que n'est-il ici pour que je le lui dise! 

—Monsieur, répondit froidement Monte-Cristo, je ne présume pas que vous soyez venu me trouver pour me conter vos petites affaires de famille. Allez dire cela à M. Albert, peut-être saura-t-il que vous répondre. 

—Oh! non, non, répliqua le général avec un sourire aussitôt disparu qu'éclos, non, vous avez raison, je ne suis pas venu pour cela! Je suis venu pour vous dire que, moi aussi, je vous regarde comme mon ennemi! Je suis venu pour vous dire que je vous hais d'instinct! qu'il me semble que je vous ai toujours connu, toujours haï! Et qu'enfin, puisque les jeunes gens de ce siècle ne se battent plus, c'est à nous de nous battre\dots Est-ce votre avis, monsieur? 

—Parfaitement. Aussi, quand je vous ai dit que j'avais prévu ce qui m'arrivait, c'est de l'honneur de votre visite que je voulais parler. 

—Tant mieux\dots vos préparatifs sont faits, alors? 

—Ils le sont toujours, monsieur. 

—Vous savez que nous nous battrons jusqu'à la mort de l'un de nous deux? dit le général, les dents serrées par la rage. 

—Jusqu'à la mort de l'un de nous deux, répéta le comte de Monte-Cristo en faisant un léger mouvement de tête de haut en bas. 

—Partons alors, nous n'avons pas besoin de témoins. 

—En effet, dit Monte-Cristo, c'est inutile, nous nous connaissons si bien! 

—Au contraire, dit le comte, c'est que nous ne nous connaissons pas. 

—Bah! dit Monte-Cristo avec le même flegme désespérant, voyons un peu. N'êtes-vous pas le soldat Fernand qui a déserté la veille de la bataille de Waterloo? N'êtes-vous pas le lieutenant Fernand qui a servi de guide et d'espion à l'armée française en Espagne? N'êtes-vous pas le colonel Fernand qui a trahi, vendu, assassiné son bienfaiteur Ali? Et tous ces Fernand-là réunis n'ont-ils pas fait le lieutenant général comte de Morcerf, pair de France? 

—Oh! s'écria le général, frappé par ces paroles comme par un fer rouge; oh! misérable, qui me reproches ma honte au moment peut-être où tu vas me tuer, non, je n'ai point dit que je t'étais inconnu; je sais bien, démon, que tu as pénétré dans la nuit du passé, et que tu y as lu, à la lueur de quel flambeau, je l'ignorais, chaque page de ma vie! mais peut-être y a-t-il encore plus d'honneur en moi, dans mon opprobre, qu'en toi sous tes dehors pompeux. Non, non, je te suis connu, je le sais, mais c'est toi que je ne connais pas, aventurier cousu d'or et de pierreries! Tu t'es fait appeler à Paris le comte de Monte-Cristo; en Italie, Simbad le Marin; à Malte, que sais-je? moi, je l'ai oublié. Mais c'est ton nom réel que je te demande, c'est ton vrai nom que je veux savoir, au milieu de tes cent noms, afin que je le prononce sur le terrain du combat au moment où je t'enfoncerai mon épée dans le cœur.» 

Le comte de Monte-Cristo pâlit d'une façon terrible; son œil fauve s'embrasa d'un feu dévorant; il fit un bond vers le cabinet attenant à sa chambre, et en moins d'une seconde, arrachant sa cravate, sa redingote et son gilet, il endossa une petite veste de marin et se coiffa d'un chapeau de matelot, sous lequel se déroulèrent ses longs cheveux noirs. 

Il revint ainsi, effrayant, implacable, marchant les bras croisés au-devant du général, qui n'avait rien compris à sa disparition, qui l'attendait, et qui, sentant ses dents claquer et ses jambes se dérober sous lui, recula d'un pas et ne s'arrêta qu'en trouvant sur une table un point d'appui pour sa main crispée. 

«Fernand! lui cria-t-il, de mes cent noms, je n'aurais besoin de t'en dire qu'un seul pour te foudroyer; mais ce nom, tu le devines, n'est-ce pas? ou plutôt tu te le rappelles? car, malgré tous mes chagrins, toutes mes tortures, je te montre aujourd'hui un visage que le bonheur de la vengeance rajeunit, un visage que tu dois avoir vu bien souvent dans tes rêves depuis ton mariage\dots avec Mercédès, ma fiancée!» 

Le général, la tête renversée en arrière, les mains étendues, le regard fixe, dévora en silence ce terrible spectacle; puis, allant chercher la muraille comme point d'appui, il s'y glissa lentement jusqu'à la porte par laquelle il sortit à reculons, en laissant échapper ce seul cri lugubre, lamentable, déchirant: 

«Edmond Dantès!» 

Puis, avec des soupirs qui n'avaient rien d'humain, il se traîna jusqu'au péristyle de la maison, traversa la cour en homme ivre, et tomba dans les bras de son valet de chambre en murmurant seulement d'une voix inintelligible: 

«À l'hôtel! à l'hôtel!» 

En chemin, l'air frais et la honte que lui causait l'attention de ses gens le remirent en état d'assembler ses idées; mais le trajet fut court, et, à mesure qu'il se rapprochait de chez lui, le comte sentait se renouveler toutes ses douleurs. 

À quelques pas de la maison, le comte fit arrêter et descendit. La porte de l'hôtel était toute grande ouverte; un fiacre, tout surpris d'être appelé dans cette magnifique demeure, stationnait au milieu de la cour; le comte regarda ce fiacre avec effroi, mais sans oser interroger personne, et s'élança dans son appartement. 

Deux personnes descendaient l'escalier, il n'eut que le temps de se jeter dans un cabinet pour les éviter. 

C'était Mercédès, appuyée au bras de son fils, qui tous deux quittaient l'hôtel. 

Ils passèrent à deux lignes du malheureux, qui, caché derrière la portière de damas, fut effleuré en quelque sorte par la robe de soie de Mercédès, et qui sentit à son visage la tiède haleine de ces paroles prononcées par son fils: 

«Du courage, ma mère! Venez, venez, nous ne sommes plus ici chez nous.» 

Les paroles s'éteignirent, les pas s'éloignèrent. 

Le général se redressa, suspendu par ses mains crispées au rideau de damas; il comprimait le plus horrible sanglot qui fût jamais sorti de la poitrine d'un père, abandonné à la fois par sa femme et par son fils\dots 

Bientôt il entendit claquer la portière en fer du fiacre, puis la voix du cocher, puis le roulement de la lourde machine ébranla les vitres; alors il s'élança dans sa chambre à coucher pour voir encore une fois tout ce qu'il avait aimé dans le monde; mais le fiacre partit sans que la tête de Mercédès ou celle d'Albert eût paru à la portière, pour donner à la maison solitaire, pour donner au père et à l'époux abandonné le dernier regard, l'adieu et le regret, c'est-à-dire le pardon. 

Aussi, au moment même où les roues du fiacre ébranlaient le pavé de la voûte, un coup de feu retentit, et une fumée sombre sortit par une des vitres de cette fenêtre de la chambre à coucher, brisée par la force de l'explosion. 