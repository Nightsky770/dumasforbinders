\chapter{L'attelage gris pommelé}

\lettrine{L}{e} baron, suivi du comte, traversa une longue file d'appartements remarquables par leur lourde somptuosité et leur fastueux mauvais goût, et arriva jusqu'au boudoir de Mme Danglars, petite pièce octogone tendue de satin rose recouvert de mousseline des Indes; les fauteuils étaient en vieux bois doré et en vieilles étoffes; les dessus des portes représentaient des bergeries dans le genre de Boucher; enfin deux jolis pastels en médaillon, en harmonie avec le reste de l'ameublement, faisaient de cette petite chambre la seule de l'hôtel qui eût quelque caractère; il est vrai qu'elle avait échappé au plan général arrêté entre M. Danglars et son architecte, une des plus hautes et des plus éminentes célébrités de l'Empire, et que c'était la baronne et Lucien Debray seulement qui s'en étaient réservé la décoration. Aussi M. Danglars, grand admirateur de l'antique à la manière dont le comprenait le Directoire, méprisait-il fort ce coquet petit réduit, où, au reste, il n'était admis en général qu'à la condition qu'il ferait excuser sa présence en amenant quelqu'un; ce n'était donc pas en réalité Danglars qui présentait, c'était au contraire lui qui était présenté et qui était bien ou mal reçu selon que le visage du visiteur était agréable ou désagréable à la baronne. 

Mme Danglars, dont la beauté pouvait encore être citée, malgré ses trente-six ans, était à son piano, petit chef-d'œuvre de marqueterie, tandis que Lucien Debray, assis devant une table à ouvrage, feuilletait un album. 

Lucien avait déjà, avant son arrivée, eu le temps de raconter à la baronne bien des choses relatives au comte. On sait combien, pendant le déjeuner chez Albert, Monte-Cristo avait fait impression sur ses convives; cette impression, si peu impressionnable qu'il fût, n'était pas encore effacée chez Debray, et les renseignements qu'il avait donnés à la baronne sur le comte s'en étaient ressentis. La curiosité de Mme Danglars, excitée par les anciens détails venus de Morcerf et les nouveaux détails venus de Lucien, était donc portée à son comble. Aussi cet arrangement de piano et d'album n'était-il qu'une de ces petites ruses du monde à l'aide desquelles on voile les plus fortes précautions. La baronne reçut en conséquence M. Danglars avec un sourire, ce qui de sa part n'était pas chose habituelle. Quant au comte, il eut, en échange de son salut, une cérémonieuse, mais en même temps gracieuse révérence. 

Lucien, de son côté, échangea avec le comte un salut de demi-connaissance, et avec Danglars un geste d'intimité.  

«Madame la baronne, dit Danglars, permettez que je vous présente M. le comte de Monte-Cristo, qui m'est adressé par mes correspondants de Rome avec les recommandations les plus instantes: je n'ai qu'un mot à en dire et qui va en un instant le rendre la coqueluche de toutes nos belles dames; il vient à Paris avec l'intention d'y rester un an et de dépenser six millions pendant cette année; cela promet une série de bals, de dîners, de médianoches, dans lesquels j'espère que M. le comte ne nous oubliera pas plus que nous ne l'oublierons nous-mêmes dans nos petites fêtes.» 

Quoique la présentation fût assez grossièrement louangeuse, c'est, en général, une chose si rare qu'un homme venant à Paris pour dépenser en une année la fortune d'un prince, que Mme Danglars jeta sur le comte un coup d'œil qui n'était pas dépourvu d'un certain intérêt. 

«Et vous êtes arrivé, monsieur?\dots demanda la baronne. 

—Depuis hier matin, madame. 

—Et vous venez, selon votre habitude, à ce qu'on m'a dit, du bout du monde? 

—De Cadix cette fois, madame, purement et simplement. 

—Oh! vous arrivez dans une affreuse saison. Paris est détestable l'été; il n'y a plus ni bals, ni réunions, ni fêtes. L'Opéra italien est à Londres, l'Opéra français est partout, excepté à Paris; et quant au Théâtre-Français, vous savez qu'il n'est plus nulle part. Il nous reste donc pour toute distraction quelques malheureuses courses au Champ-de-Mars et à Satory. Ferez-vous courir, monsieur le comte? 

—Moi, madame, dit Monte-Cristo, je ferai tout ce qu'on fait à Paris, si j'ai le bonheur de trouver quelqu'un qui me renseigne convenablement sur les habitudes françaises. 

—Vous êtes amateur de chevaux, monsieur le comte? 

—J'ai passé une partie de ma vie en Orient, madame, et les Orientaux, vous le savez, n'estiment que deux choses au monde: la noblesse des chevaux et la beauté des femmes.  

—Ah! monsieur le comte, dit la baronne, vous auriez dû avoir la galanterie de mettre les femmes les premières. 

—Vous voyez, madame, que j'avais bien raison quand tout à l'heure je souhaitais un précepteur qui pût me guider dans les habitudes françaises.» 

En ce moment la camériste favorite de Mme la baronne Danglars entra, et s'approchant de sa maîtresse, lui glissa quelques mots à l'oreille. 

Mme Danglars pâlit. 

«Impossible! dit-elle. 

—C'est l'exacte vérité, cependant, madame», répondit la camériste. 

Mme Danglars se retourna du côté de son mari. 

«Est-ce vrai, monsieur? 

—Quoi, madame? demanda Danglars visiblement agité. 

—Ce que me dit cette fille\dots. 

—Et que vous dit-elle? 

—Elle me dit qu'au moment où mon cocher a été pour mettre mes chevaux à ma voiture, il ne les a pas trouvés à l'écurie; que signifie cela, je vous le demande? 

—Madame, dit Danglars, écoutez-moi. 

—Oh! je vous écoute, monsieur, car je suis curieuse de savoir ce que vous allez me dire; je ferai ces messieurs juges entre nous, et je vais commencer par leur dire ce qu'il en est. Messieurs, continua la baronne, M. le baron Danglars a dix chevaux à l'écurie; parmi ces dix chevaux, il y en a deux qui sont à moi, des chevaux charmants, les plus beaux chevaux de Paris; vous les connaissez, monsieur Debray, mes gris pommelé! Eh bien, au moment où Mme de Villefort m'emprunte ma voiture, où je la lui promets pour aller demain au Bois, voilà les deux chevaux qui ne se retrouvent plus! M. Danglars aura trouvé à gagner dessus quelques milliers de francs, et il les aura vendus. Oh! la vilaine race, mon Dieu! que celle des spéculateurs! 

—Madame, répondit Danglars, les chevaux étaient trop vifs, ils avaient quatre ans à peine, ils me faisaient pour vous des peurs horribles. 

—Eh! monsieur, dit la baronne, vous savez bien que j'ai depuis un mois à mon service le meilleur cocher de Paris, à moins toutefois que vous ne l'ayez vendu avec les chevaux. 

—Chère amie je vous trouverai les pareils, de plus beaux même, s'il y en a; mais des chevaux doux calmes, et qui ne m'inspirent plus pareille terreur.» 

La baronne haussa les épaules avec un air de profond mépris. Danglars ne parut point s'apercevoir de ce geste plus que conjugal, et se retournant vers Monte-Cristo: 

«En vérité, je regrette de ne pas vous avoir connu plus tôt, monsieur le comte, dit-il; vous montez votre maison? 

—Mais oui, dit le comte. 

—Je vous les eusse proposés. Imaginez-vous que je les ai donnés pour rien, mais, comme je vous l'ai dit, je voulais m'en défaire: ce sont des chevaux de jeune homme.  

—Monsieur, dit le comte, je vous remercie; j'en ai acheté ce matin d'assez bons et pas trop cher. Tenez, voyez, monsieur Debray, vous êtes amateur, je crois?» 

Pendant que Debray s'approchait de la fenêtre, Danglars s'approcha de sa femme. 

«Imaginez-vous, madame, lui dit-il tout bas, qu'on est venu m'offrir un prix exorbitant de ces chevaux. Je ne sais quel est le fou en train de se ruiner qui m'a envoyé ce matin son intendant, mais le fait est que j'ai gagné seize mille francs dessus; ne me boudez pas, et je vous en donnerai quatre mille, et deux mille à Eugénie.» 

Mme Danglars laissa tomber sur son mari un regard écrasant. 

«Oh! mon Dieu! s'écria Debray. 

—Quoi donc? demanda la baronne. 

—Mais je ne me trompe pas, ce sont vos chevaux, vos propres chevaux attelés à la voiture du comte. 

—Mes gris pommelé!» s'écria Mme Danglars. 

Et elle s'élança vers la fenêtre. 

«En effet, ce sont eux», dit-elle.  

Danglars était stupéfait. 

«Est-ce possible? dit Monte-Cristo en jouant l'étonnement. 

—C'est incroyable!» murmura le banquier. 

La baronne dit deux mots à l'oreille de Debray, qui s'approcha à son tour de Monte-Cristo. 

«La baronne vous fait demander combien son mari vous a vendu son attelage. 

—Mais je ne sais trop, dit le comte, c'est une surprise que mon intendant m'a faite, et\dots qui m'a coûté trente mille francs, je crois.» 

Debray alla reporter la réponse à la baronne. 

Danglars était si pâle et si décontenancé, que le comte eut l'air de le prendre en pitié. 

«Voyez, lui dit-il, combien les femmes sont ingrates: cette prévenance de votre part n'a pas touché un instant la baronne; ingrate n'est pas le mot, c'est folle que je devrais dire. Mais que voulez-vous, on aime toujours ce qui nuit; aussi, le plus court, croyez-moi, cher baron, est toujours de les laisser faire à leur tête; si elles se la brisent, au moins, ma foi! elles ne peuvent s'en prendre qu'à elles.»  

Danglars ne répondit rien, il prévoyait dans un prochain avenir une scène désastreuse; déjà le sourcil de Mme la baronne s'était froncé, et comme celui de Jupiter olympien, présageait un orage; Debray, qui le sentait grossir prétexta une affaire et partit. Monte-Cristo, qui ne voulait pas gâter la position qu'il voulait conquérir en demeurant plus longtemps, salua Mme Danglars et se retira, livrant le baron à la colère de sa femme. 

«Bon! pensa Monte-Cristo en se retirant, j'en suis arrivé où j'en voulais venir; voilà que je tiens dans mes mains la paix du ménage et que je vais gagner d'un seul coup le cœur de monsieur et le cœur de madame; quel bonheur! Mais, ajouta-t-il, dans tout cela, je n'ai point été présenté à Mlle Eugénie Danglars, que j'eusse été cependant fort aise de connaître. Mais, reprit-il avec ce sourire qui lui était particulier, nous voici à Paris, et nous avons du temps devant nous\dots. Ce sera pour plus tard!\dots» 

Sur cette réflexion, le comte monta en voiture et rentra chez lui. 

Deux heures après, Mme Danglars reçut une lettre charmante du comte de Monte-Cristo, dans laquelle il lui déclarait que, ne voulant pas commencer ses débuts dans le monde parisien en désespérant une jolie femme, il la suppliait de reprendre ses chevaux. 

Ils avaient le même harnais qu'elle leur avait vu le matin; seulement au centre de chaque rosette qu'ils portaient sur l'oreille, le comte avait fait coudre un diamant.  

Danglars, aussi, eut sa lettre. 

Le comte lui demandait la permission de passer à la baronne ce caprice de millionnaire, le priant d'excuser les façons orientales dont le renvoi des chevaux était accompagné. 

Pendant la soirée, Monte-Cristo partit pour Auteuil, accompagné d'Ali. 

Le lendemain vers trois heures, Ali, appelé par un coup de timbre entra dans le cabinet du comte. 

«Ali, lui dit-il, tu m'as souvent parlé de ton adresse à lancer le lasso?» 

Ali fit signe que oui et se redressa fièrement. 

«Bien!\dots Ainsi, avec le lasso, tu arrêterais un bœuf?» 

Ali fit signe de la tête que oui. 

«Un tigre?» 

Ali fit le même signe. 

«Un lion?» 

Ali fit le geste d'un homme qui lance le lasso, et imita un rugissement étranglé. 

«Bien, je comprends, dit Monte-Cristo, tu as chassé le lion?» 

Ali fit un signe de tête orgueilleux. 

«Mais arrêterais-tu, dans leur course, deux chevaux?» 

Ali sourit. 

«Eh bien, écoute, dit Monte-Cristo. Tout à l'heure une voiture passera emportée par deux chevaux gris pommelé, les mêmes que j'avais hier. Dusses-tu te faire écraser, il faut que tu arrêtes cette voiture devant ma porte.»  

Ali descendit dans la rue et traça devant la porte une ligne sur le pavé: puis il rentra et montra la ligne au comte, qui l'avait suivi des yeux. 

Le comte lui frappa doucement sur l'épaule: c'était sa manière de remercier Ali. Puis le Nubien alla fumer sa chibouque sur la borne qui formait l'angle de la maison et de la rue, tandis que Monte-Cristo rentrait sans plus s'occuper de rien. 

Cependant, vers cinq heures, c'est-à-dire l'heure où le comte attendait la voiture, on eût pu voir naître en lui les signes presque imperceptibles d'une légère impatience: il se promenait dans une chambre donnant sur la rue, prêtant l'oreille par intervalles, et de temps en temps se rapprochant de la fenêtre, par laquelle il apercevait Ali poussant des bouffées de tabac avec une régularité indiquant que le Nubien était tout à cette importante occupation. 

Tout à coup on entendit un roulement lointain, mais qui se rapprochait avec la rapidité de la foudre; puis une calèche apparut dont le cocher essayait inutilement de retenir les chevaux, qui s'avançaient furieux, hérissés, bondissant avec des élans insensés. 

Dans la calèche, une jeune femme et un enfant de sept à huit ans, se tenant embrassés, avaient perdu par l'excès de la terreur jusqu'à la force de pousser un cri; il eût suffi d'une pierre sous la roue ou d'un arbre accroché pour briser tout à fait la voiture, qui craquait. La voiture tenait le milieu du pavé, et on entendait dans la rue les cris de terreur de ceux qui la voyaient venir. 

Soudain Ali pose sa chibouque, tire de sa poche le lasso, le lance, enveloppe d'un triple tour les jambes de devant du cheval de gauche, se laisse entraîner trois ou quatre pas par la violence de l'impulsion; mais, au bout de trois ou quatre pas, le cheval enchaîné s'abat, tombe sur la flèche, qu'il brise, et paralyse les efforts que fait le cheval resté debout pour continuer sa course. Le cocher saisit cet instant de répit pour sauter en bas de son siège; mais déjà Ali a saisi les naseaux du second cheval avec ses doigts de fer, et l'animal, hennissant de douleur, s'est allongé convulsivement près de son compagnon. 

Il a fallu à tout cela le temps qu'il faut à la balle pour frapper le but. 

Cependant il a suffi pour que de la maison en face de laquelle l'accident est arrivé, un homme se soit élancé suivi de plusieurs serviteurs. Au moment où le cocher ouvre la portière, il enlève de la calèche la dame, qui d'une main se cramponne au coussin, tandis que de l'autre elle serre contre sa poitrine son fils évanoui. Monte-Cristo les emporta tous les deux dans le salon, et les déposant sur un canapé: 

«Ne craignez plus rien, madame, dit-il; vous êtes sauvée.» 

La femme revint à elle, et pour réponse elle lui présenta son fils, avec un regard plus éloquent que toutes les prières.  

En effet, l'enfant était toujours évanoui. 

«Oui, madame, je comprends, dit le comte en examinant l'enfant; mais, soyez tranquille, il ne lui est arrivé aucun mal, et c'est la peur seule qui l'a mis dans cet état. 

—Oh! monsieur, s'écria la mère, ne me dites-vous pas cela pour me rassurer? Voyez comme il est pâle! Mon fils, mon enfant! mon Édouard! réponds donc à ta mère! Ah! monsieur! envoyez chercher un médecin. Ma fortune à qui me rend mon fils!» 

Monte-Cristo fit de la main un geste pour calmer la mère éplorée; et, ouvrant un coffret, il en tira un flacon de Bohème, incrusté d'or, contenant une liqueur rouge comme du sang et dont il laissa tomber une seule goutte sur les lèvres de l'enfant. 

L'enfant, quoique toujours pâle, rouvrit aussitôt les yeux. 

À cette vue, la joie de la mère fut presque un délire. 

«Où suis-je? s'écria-t-elle, et à qui dois-je tant de bonheur après une si cruelle épreuve? 

—Vous êtes, madame, répondit Monte-Cristo, chez l'homme le plus heureux d'avoir pu vous épargner un chagrin. 

—Oh! maudite curiosité! dit la dame. Tout Paris parlait de ces magnifiques chevaux de Mme Danglars, et j'ai eu la folie de vouloir les essayer. 

—Comment! s'écria le comte avec une surprise admirablement jouée, ces chevaux sont ceux de la baronne? 

—Oui, monsieur, la connaissez-vous? 

—Mme Danglars?\dots j'ai cet honneur, et ma joie est double de vous voir sauvée du péril que ces chevaux vous ont fait courir; car ce péril, c'est à moi que vous eussiez pu l'attribuer: j'avais acheté hier ces chevaux au baron; mais la baronne a paru tellement les regretter, que je les lui ai renvoyés hier en la priant de les accepter de ma main. 

—Mais alors vous êtes donc le comte de Monte-Cristo dont Hermine m'a tant parlé hier? 

—Oui, madame, fit le comte. 

—Moi, monsieur, je suis Mme Héloïse de Villefort.» 

Le comte salua en homme devant lequel on prononce un nom parfaitement inconnu. 

«Oh! que M. de Villefort sera reconnaissant! reprit Héloïse car enfin il vous devra notre vie à tous deux: vous lui avez rendu sa femme et son fils. Assurément, sans votre généreux serviteur, ce cher enfant et moi, nous étions tués. 

—Hélas! madame! je frémis encore du péril que vous avez couru. 

—Oh! j'espère que vous me permettrez de récompenser dignement le dévouement de cet homme. 

—Madame, répondit Monte-Cristo, ne me gâtez pas Ali, je vous prie, ni par des louanges, ni par des récompenses: ce sont des habitudes que je ne veux pas qu'il prenne. Ali est mon esclave; en vous sauvant la vie il me sert, et c'est son devoir de me servir. 

—Mais il a risqué sa vie, dit Mme de Villefort, à qui ce ton de maître imposait singulièrement. 

—J'ai sauvé cette vie, madame, répondit Monte-Cristo, par conséquent elle m'appartient.»  

Mme de Villefort se tut: peut-être réfléchissait-elle à cet homme qui, du premier abord, faisait une si profonde impression sur les esprits. 

Pendant cet instant de silence, le comte put considérer à son aise l'enfant que sa mère couvrait de baisers. Il était petit, grêle, blanc de peau comme les enfants roux, et cependant une forêt de cheveux noirs, rebelles à toute frisure, couvrait son front bombé, et, tombant sur ses épaules en encadrant son visage, redoublait la vivacité de ses yeux pleins de malice sournoise et de juvénile méchanceté; sa bouche, à peine redevenue vermeille, était fine de lèvres et large d'ouverture; les traits de cet enfant de huit ans annonçaient déjà douze ans au moins. Son premier mouvement fut de se débarrasser par une brusque secousse des bras de sa mère, et d'aller ouvrir le coffret d'où le comte avait tiré le flacon d'élixir; puis aussitôt, sans en demander la permission à personne, et en enfant habitué à satisfaire tous ses caprices, il se mit à déboucher les fioles. 

«Ne touchez pas à cela, mon ami, dit vivement le comte, quelques-unes de ces liqueurs sont dangereuses, non seulement à boire, mais même à respirer.» 

Mme de Villefort pâlit et arrêta le bras de son fils qu'elle ramena vers elle; mais, sa crainte calmée, elle jeta aussitôt sur le coffret un court mais expressif regard que le comte saisit au passage. 

En ce moment Ali entra. 

Mme de Villefort fit un mouvement de joie, et ramena l'enfant plus près d'elle encore: 

«Édouard, dit-elle, vois-tu ce bon serviteur: il a été bien courageux, car il a exposé sa vie pour arrêter les chevaux qui nous emportaient et la voiture qui allait se briser. Remercie-le donc, car probablement sans lui, à cette heure, serions-nous morts tous les deux.» 

L'enfant allongea les lèvres et tourna dédaigneusement la tête. 

«Il est trop laid», dit-il. 

Le comte sourit comme si l'enfant venait de remplir une de ses espérances; quant à Mme de Villefort, elle gourmanda son fils avec une modération qui n'eût, certes, pas été du goût de Jean-Jacques Rousseau si le petit Édouard se fût appelé Émile. 

«Vois-tu, dit en arabe le comte à Ali, cette dame prie son fils de te remercier pour la vie que tu leur as sauvée à tous deux, et l'enfant répond que tu es trop laid.» 

Ali détourna un instant sa tête intelligente et regarda l'enfant sans expression apparente; mais un simple frémissement de sa narine apprit à Monte-Cristo que l'Arabe venait d'être blessé au cœur. 

«Monsieur, demanda Mme de Villefort en se levant pour se retirer, est-ce votre demeure habituelle que cette maison? 

—Non, madame, répondit le comte, c'est une espèce de pied-à-terre que j'ai acheté: j'habite avenue des Champs-Élysées, n° 30. Mais je vois que vous êtes tout à fait remise, et que vous désirez vous retirer. Je viens d'ordonner qu'on attelle ces mêmes chevaux à ma voiture, et Ali, ce garçon si laid, dit-il en souriant à l'enfant, va avoir l'honneur de vous reconduire chez vous, tandis que votre cocher restera ici pour faire raccommoder la calèche. Aussitôt cette besogne indispensable terminée, un de mes attelages la reconduira directement chez Mme Danglars. 

—Mais, dit Mme de Villefort, avec ces mêmes chevaux je n'oserai jamais m'en aller.  

—Oh! vous allez voir, madame, dit Monte-Cristo; sous la main d'Ali, ils vont devenir doux comme des agneaux.» 

En effet, Ali s'était approché des chevaux qu'on avait remis sur leurs jambes avec beaucoup de peine. Il tenait à la main une petite éponge imbibée de vinaigre aromatique; il en frotta les naseaux et les tempes des chevaux, couverts de sueur et d'écume, et presque aussitôt ils se mirent à souffler bruyamment et à frissonner de tout leur corps durant quelques secondes. 

Puis, au milieu d'une foule nombreuse que les débris de la voiture et le bruit de l'événement avaient attirée devant la maison, Ali fit atteler les chevaux au coupé du comte, rassembla les rênes, monta sur le siège, et, au grand étonnement des assistants qui avaient vu ces chevaux emportés comme par un tourbillon, il fut obligé d'user vigoureusement du fouet pour les faire partir et encore ne put-il obtenir des fameux gris pommelé, maintenant stupides, pétrifiés, morts, qu'un trot si mal assuré et si languissant qu'il fallut près de deux heures à Mme de Villefort pour regagner le faubourg Saint-Honoré, où elle demeurait. 

À peine arrivée chez elle, et les premières émotions de famille apaisées, elle écrivit le billet suivant à Mme Danglars: 

\begin{mail}{}{Chère Hermine,}

Je viens d'être miraculeusement sauvée avec mon fils par ce même comte de Monte-Cristo dont nous avons tant parlé hier soir, et que j'étais loin de me douter que je verrais aujourd'hui. Hier vous m'avez parlé de lui avec un enthousiasme que je n'ai pu m'empêcher de railler de toute la force de mon pauvre petit esprit, mais aujourd'hui je trouve cet enthousiasme bien au-dessous de l'homme qui l'inspirait. Vos chevaux s'étaient emportés au Ranelagh comme s'ils eussent été pris de frénésie, et nous allions probablement être mis en morceaux, mon pauvre Édouard et moi, contre le premier arbre de la route ou la première borne du village, quand un Arabe, un Nègre, un Nubien, un homme noir enfin, au service du comte, a, sur un signe de lui, je crois, arrêté l'élan des chevaux, au risque d'être brisé lui-même, et c'est vraiment un miracle qu'il ne l'ait pas été. Alors le comte est accouru, nous a emportés chez lui, Édouard et moi, et là a rappelé mon fils à la vie. C'est dans sa propre voiture que j'ai été ramenée à l'hôtel; la vôtre vous sera renvoyée demain. Vous trouverez vos chevaux bien affaiblis depuis cet accident; ils sont comme hébétés; on dirait qu'ils ne peuvent se pardonner à eux-mêmes de s'être laissé dompter par un homme. Le comte m'a chargée de vous dire que deux jours de repos sur la litière et de l'orge pour toute nourriture les remettront dans un état aussi florissant, ce qui veut dire aussi effrayant qu'hier. 

Adieu! Je ne vous remercie pas de ma promenade, et, quand je réfléchis, c'est pourtant de l'ingratitude que de vous garder rancune pour les caprices de votre attelage; car c'est à l'un de ces caprices que je dois d'avoir vu le comte de Monte-Cristo, et l'illustre étranger me paraît, à part les millions dont il dispose, un problème si curieux et si intéressant, que je compte l'étudier à tout prix, dussé-je recommencer une promenade au Bois avec vos propres chevaux. 

Édouard a supporté l'accident avec un courage miraculeux. Il s'est évanoui, mais il n'a pas poussé un cri auparavant et n'a pas versé une larme après. Vous me direz encore que mon amour maternel m'aveugle; mais il y a une âme de fer dans ce pauvre petit corps si frêle et si délicat. 

Notre chère Valentine dit bien des choses à votre chère Eugénie; moi, je vous embrasse de tout cœur. 

\addPS{Faites-moi donc trouver chez vous d'une façon quelconque avec ce comte de Monte-Cristo, je veux absolument le revoir. Au reste, je viens d'obtenir de M. de Villefort qu'il lui fasse une visite; j'espère bien qu'il la lui rendra.}

\closeletter{Héloïse de Villefort.} 

\end{mail}

Le soir, l'événement d'Auteuil faisait le sujet de toutes les conversations: Albert le racontait à sa mère, Château-Renaud au Jockey-Club, Debray dans le salon du ministre; Beauchamp lui-même fit au comte la galanterie, dans son journal, d'un \textit{fait divers} de vingt lignes, qui posa le noble étranger en héros auprès de toutes les femmes de l'aristocratie. 

Beaucoup de gens allèrent se faire inscrire chez Mme de Villefort afin d'avoir le droit de renouveler leur visite en temps utile et d'entendre alors de sa bouche tous les détails de cette pittoresque aventure.  

Quant à M. de Villefort, comme l'avait dit Héloïse, il prit un habit noir, des gants blancs, sa plus belle livrée, et monta dans son carrosse qui vint, le même soir, s'arrêter à la porte du numéro 30 de la maison des Champs-Élysées. 