\chapter{Le bal}

\lettrine{O}{n} en était arrivé aux plus chaudes journées de juillet, lorsque vint se présenter à son tour, dans l'ordre des temps, ce samedi où devait avoir lieu le bal de M. de Morcerf. 

\zz
Il était dix heures du soir: les grands arbres du jardin de l'hôtel du comte se détachaient en vigueur sur un ciel où glissaient, découvrant, une tenture d'azur parsemée d'étoiles d'or, les dernières vapeurs d'un orage qui avait grondé menaçant toute la journée. 

Dans les salles du rez-de-chaussée, on entendait bruire la musique et tourbillonner la valse et le galop tandis que des bandes éclatantes de lumière passaient tranchantes à travers les ouvertures des persiennes. 

Le jardin était livré en ce moment à une dizaine de serviteurs, à qui la maîtresse de maison, rassurée par le temps qui se rassérénait de plus en plus, venait de donner l'ordre de dresser le souper. 

Jusque-là on avait hésité si l'on souperait dans la salle à manger ou sous une longue tente de coutil dressée sur la pelouse. Ce beau ciel bleu, tout parsemé d'étoiles, venait de décider le procès en faveur de la tente et de la pelouse. 

On illuminait les allées du jardin avec les lanternes de couleur, comme c'est l'habitude en Italie, et l'on surchargeait de bougies et de fleurs la table du souper, comme c'est l'usage dans tous les pays où l'on comprend un peu ce luxe de la table, le plus rare de tous les luxes, quand on veut le rencontrer complet. 

Au moment où la comtesse de Morcerf rentrait dans ses salons, après avoir donné ses derniers ordres, les salons commençaient à se remplir d'invités qu'attirait la charmante hospitalité de la comtesse, bien plus que la position distinguée du comte; car on était sûr d'avance que cette fête offrirait, grâce au bon goût de Mercédès, quelques détails dignes d'être racontés ou copiés au besoin. 

Mme Danglars, à qui les événements que nous avons racontés avaient inspiré une profonde inquiétude, hésitait à aller chez Mme de Morcerf, lorsque dans la matinée sa voiture avait croisé celle de Villefort. Villefort lui avait fait un signe, les deux voitures s'étaient rapprochées, et à travers les portières: 

«Vous allez chez Mme de Morcerf, n'est-ce pas? avait demandé le procureur du roi. 

—Non, avait répondu Mme Danglars, je suis trop souffrante. 

—Vous avez tort, reprit Villefort avec un regard significatif; il serait important que l'on vous y vît. 

—Ah! croyez-vous? demanda la baronne. 

—Je le crois. 

—En ce cas, j'irai.» 

Et les deux voitures avaient repris leur course divergente. Mme Danglars était donc venue, non seulement belle de sa propre beauté, mais encore éblouissante de luxe; elle entrait par une porte au moment où Mercédès entrait par l'autre. 

La comtesse détacha Albert au-devant de Mme Danglars; Albert s'avança, fit à la baronne, sur sa toilette, les compliments mérités, et lui prit le bras pour la conduire à la place qu'il lui plairait de choisir. 

Albert regarda autour de lui. 

«Vous cherchez ma fille? dit en souriant la baronne. 

—Je l'avoue, dit Albert; auriez-vous eu la cruauté de ne pas nous l'amener?» 

—Rassurez-vous, elle a rencontré Mlle de Villefort et a pris son bras; tenez, les voici qui nous suivent toutes les deux en robes blanches, l'une avec un bouquet de camélias, l'autre avec un bouquet de myosotis; mais dites-moi donc?\dots 

—Que cherchez-vous à votre tour? demanda Albert en souriant. 

—Est-ce que vous n'aurez pas ce soir le comte de Monte-Cristo? 

—Dix-sept! répondit Albert. 

—Que voulez-vous dire? 

—Je veux dire que cela va bien, reprit le vicomte en riant, et que vous êtes la dix-septième personne qui me fait la même question; il va bien le comte!\dots je lui en fais mon compliment\dots. 

—Et répondez-vous à tout le monde comme à moi? 

—Ah! c'est vrai, je ne vous ai pas répondu; rassurez-vous, madame, nous aurons l'homme à la mode, nous sommes des privilégiés. 

—Étiez-vous hier à l'Opéra? 

—Non. 

—Il y était, lui. 

—Ah! vraiment! Et l'\textit{excentric man} a-t-il fait quelque nouvelle originalité? 

—Peut-il se montrer sans cela? Elssler dansait dans le \textit{Diable boiteux}; la princesse grecque était dans le ravissement. Après la cachucha, il a passé une bague magnifique dans la queue du bouquet, et l'a jeté à la charmante danseuse, qui au troisième acte a reparu, pour lui faire honneur, avec sa bague au doigt. Et sa princesse grecque, l'aurez-vous? 

—Non, il faut que vous vous en priviez; sa position dans la maison du comte n'est pas assez fixée. 

—Tenez, laissez-moi ici et allez saluer Mme de Villefort, dit la baronne: je vois qu'elle meurt d'envie de vous parler.» 

Albert salua Mme Danglars et s'avança vers Mme de Villefort, qui ouvrit la bouche à mesure qu'il approchait. 

«Je parie, dit Albert en l'interrompant, que je sais ce que vous allez me dire? 

—Ah! par exemple! dit Mme de Villefort. 

—Si je devine juste, me l'avouerez-vous? 

—Oui. 

—D'honneur? 

—D'honneur. 

—Vous alliez me demander si le comte de Monte-Cristo était arrivé ou allait venir? 

—Pas du tout. Ce n'est pas de lui que je m'occupe en ce moment. J'allais vous demander si vous aviez reçu des nouvelles de M. Franz. 

—Oui, hier. 

—Que vous disait-il? 

—Qu'il partait en même temps que sa lettre. 

—Bien! Maintenant, le comte? 

—Le comte viendra, soyez tranquille. 

—Vous savez qu'il a un autre nom que Monte-Cristo? 

—Non, je ne savais pas. 

—Monte-Cristo est un nom d'île, et il a un nom de famille. 

—Je ne l'ai jamais entendu prononcer. 

—Eh bien, je suis plus avancée que vous; il s'appelle Zaccone. 

—C'est possible. 

—Il est Maltais. 

—C'est possible encore. 

—Fils d'un armateur. 

—Oh! mais, en vérité, vous devriez raconter ces choses-là tout haut, vous auriez le plus grand succès. 

—Il a servi dans l'Inde, exploite une mine d'argent en Thessalie, et vient à Paris pour faire un établissement d'eaux minérales à Auteuil. 

—Eh bien, à la bonne heure, dit Morcerf, voilà des nouvelles! Me permettez-vous de les répéter? 

—Oui, mais petit à petit, une à une, sans dire qu'elles viennent de moi. 

—Pourquoi cela?  

—Parce que c'est presque un secret surpris. 

—À qui? 

—À la police. 

—Alors ces nouvelles se débitaient\dots. 

—Hier soir, chez le préfet. Paris s'est ému, vous le comprenez bien, à la vue de ce luxe inusité, et la police a pris des informations. 

—Bien! il ne manquait plus que d'arrêter le comte comme vagabond, sous prétexte qu'il est trop riche. 

—Ma foi, c'est ce qui aurait bien pu lui arriver si les renseignements n'avaient pas été si favorables. 

—Pauvre comte, et se doute-t-il du péril qu'il a couru? 

—Je ne crois pas. 

—Alors, c'est charité que de l'en avertir. À son arrivée je n'y manquerai pas.» 

En ce moment un beau jeune homme aux yeux vifs, aux cheveux noirs, à la moustache luisante, vint saluer respectueusement Mme de Villefort. Albert lui tendit la main. 

«Madame, dit Albert, j'ai l'honneur de vous présenter M. Maximilien Morrel, capitaine aux spahis, l'un de nos bons et surtout de nos braves officiers. 

—J'ai déjà eu le plaisir de rencontrer monsieur à Auteuil, chez M. le comte de Monte-Cristo», répondit Mme de Villefort en se détournant avec une froideur marquée. 

Cette réponse, et surtout le ton dont elle était faite, serrèrent le cœur du pauvre Morrel; mais une compensation lui était ménagée: en se retournant, il vit à l'encoignure de la porte une belle et blanche figure dont les yeux dilatés et sans expression apparente s'attachaient sur lui, tandis que le bouquet de myosotis montait lentement à ses lèvres. 

Ce salut fut si bien compris que Morrel, avec la même expression de regard, approcha à son tour son mouchoir de sa bouche; et les deux statues vivantes dont le cœur battait si rapidement sous le marbre apparent de leur visage, séparées l'une de l'autre par toute la largeur de la salle, s'oublièrent un instant, ou plutôt un instant oublièrent tout le monde dans cette muette contemplation. 

Elles eussent pu rester plus longtemps ainsi perdues l'une dans l'autre, sans que personne remarquât leur oubli de toutes choses: le comte de Monte-Cristo venait d'entrer. 

Nous l'avons déjà dit, le comte, soit prestige factice, soit prestige naturel, attirait l'attention partout où il se présentait; ce n'était pas son habit noir, irréprochable il est vrai dans sa coupe, mais simple et sans décorations; ce n'était pas son gilet blanc sans aucune broderie; ce n'était pas son pantalon emboîtant un pied de la forme la plus délicate, qui attiraient l'attention: c'étaient son teint mat, ses cheveux noirs ondés, c'était son visage calme et pur, c'était son œil profond et mélancolique, c'était enfin sa bouche dessinée avec une finesse merveilleuse, et qui prenait si facilement l'expression d'un haut dédain, qui faisaient que tous les yeux se fixaient sur lui. 

Il pouvait y avoir des hommes plus beaux, mais il n'y en avait certes pas de plus \textit{significatifs}, qu'on nous passe cette expression: tout dans le comte voulait dire quelque chose et avait sa valeur; car l'habitude de la pensée utile avait donné à ses traits, à l'expression de son visage et au plus insignifiant de ses gestes une souplesse et une fermeté incomparables. 

Et puis notre monde parisien est si étrange, qu'il n'eût peut-être point fait attention à tout cela, s'il n'y eût eu sous tout cela une mystérieuse histoire dorée par une immense fortune. 

Quoi qu'il en soit, il s'avança, sous le poids des regards et à travers l'échange des petits saluts jusqu'à Mme de Morcerf, qui, debout devant la cheminée garnie de fleurs, l'avait vu apparaître dans une glace placée en face de la porte, et s'était préparée pour le recevoir. 

Elle se retourna donc vers lui avec un sourire composé au moment même où il s'inclinait devant elle. 

Sans doute elle crut que le comte allait lui parler; sans doute, de son côté, le comte crut qu'elle allait lui adresser la parole; mais des deux côtés ils restèrent muets, tant une banalité leur semblait sans doute indigne de tous deux; et, après un échange de saluts, Monte-Cristo se dirigea vers Albert, qui venait à lui la main ouverte. 

«Vous avez vu ma mère? demanda Albert. 

—Je viens d'avoir l'honneur de la saluer, dit le comte, mais je n'ai point aperçu votre père. 

—Tenez! il cause politique, là-bas, dans ce petit groupe de grandes célébrités. 

—En vérité, dit Monte-Cristo, ces messieurs que je vois là-bas sont des célébrités? je ne m'en serais pas douté! Et de quel genre? Il y a des célébrités de toute espèce, comme vous savez. 

—Il y a d'abord un savant, ce grand monsieur sec; il a découvert dans la campagne de Rome une espèce de lézard qui a une vertèbre de plus que les autres, et il est revenu faire part à l'Institut de cette découverte. La chose a été longtemps contestée: mais force est restée au grand monsieur sec. La vertèbre avait fait beaucoup de bruit dans le monde savant; le grand monsieur sec n'était que chevalier de la Légion d'honneur, on l'a nommé officier. 

—À la bonne heure! dit Monte-Cristo, voilà une croix qui me paraît sagement donnée; alors, s'il trouve une seconde vertèbre, on le fera commandeur? 

—C'est probable, dit Morcerf. 

—Et cet autre qui a eu la singulière idée de s'affubler d'un habit bleu brodé de vert, quel peut-il être? 

—Ce n'est pas lui qui a eu l'idée de s'affubler de cet habit: c'est la République, laquelle, comme vous le savez, était un peu artiste, et qui, voulant donner un uniforme aux académiciens, a prié David de leur dessiner un habit. 

—Ah! vraiment, dit Monte-Cristo; ainsi ce monsieur est académicien? 

—Depuis huit jours il fait partie de la docte assemblée. 

—Et quel est son mérite, sa spécialité? 

—Sa spécialité? Je crois qu'il enfonce des épingles dans la tête des lapins, qu'il fait manger de la garance aux poules et qu'il repousse avec des baleines la moelle épinière des chiens. 

—Et il est de l'Académie des sciences pour cela? 

—Non pas, de l'Académie française. 

—Mais qu'a donc à faire l'Académie française là-dedans? 

—Je vais vous dire, il paraît\dots. 

—Que ses expériences ont fait faire un grand pas à la science, sans doute? 

—Non, mais qu'il écrit en fort bon style. 

—Cela doit, dit Monte-Cristo, flatter énormément l'amour-propre des lapins à qui il enfonce des épingles dans la tête, des poules dont il teint les os en rouge, et des chiens dont il repousse la moelle épinière.»  

Albert se mit à rire. 

«Et cet autre? demanda le comte. 

—Cet autre? 

—Oui, le troisième. 

—Ah! l'habit bleu barbeau? 

—Oui. 

—C'est un collègue du comte, qui vient de s'opposer le plus chaudement à ce que la Chambre des pairs ait un uniforme; il a eu un grand succès de tribune à ce propos-là; il était mal avec les gazettes libérales, mais sa noble opposition aux désirs de la cour vient de le raccommoder avec elles; on parle de le nommer ambassadeur. 

—Et quels sont ses titres à la pairie? 

—Il a fait deux ou trois opéras-comiques, pris quatre ou cinq actions au \textit{Siècle}, et voté cinq ou six ans pour le ministère. 

—Bravo! vicomte, dit Monte-Cristo en riant, vous êtes un charmant cicérone; maintenant vous me rendrez un service, n'est-ce pas? 

—Lequel? 

—Vous ne me présenterez pas à ces messieurs, et s'ils demandent à m'être présentés, vous me préviendrez.» 

En ce moment le comte sentit qu'on lui posait la main sur le bras; il se retourna, c'était Danglars. 

«Ah! c'est vous, baron! dit-il. 

—Pourquoi m'appelez-vous baron? dit Danglars; vous savez bien que je ne tiens pas à mon titre. Ce n'est pas comme vous, vicomte; vous y tenez, n'est-ce pas, vous?». 

—Certainement, répondit Albert, attendu que si je n'étais pas vicomte, je ne serais plus rien, tandis que vous, vous pouvez sacrifier votre titre de baron, vous resterez encore millionnaire. 

—Ce qui me paraît le plus beau titre sous la royauté de Juillet, reprit Danglars. 

—Malheureusement, dit Monte-Cristo, on n'est pas millionnaire à vie comme on est baron, pair de France ou académicien; témoins les millionnaires Frank et Poulmann, de Francfort, qui viennent de faire banqueroute. 

—Vraiment? dit Danglars en pâlissant. 

—Ma foi, j'en ai reçu la nouvelle ce soir par un courrier; j'avais quelque chose comme un million chez eux; mais, averti à temps, j'en ai exigé le remboursement voici un mois à peu près. 

—Ah! mon Dieu! reprit Danglars; ils ont tiré sur moi pour deux cent mille francs.  

—Eh bien, vous voilà prévenu; leur signature vaut cinq pour cent. 

—Oui, mais je suis prévenu trop tard, dit Danglars, j'ai fait honneur à leur signature. 

—Bon! dit Monte-Cristo, voilà deux cent mille francs qui sont allés rejoindre\dots. 

—Chut! dit Danglars; ne parlez donc pas de ces choses-là\dots.» 

Puis, s'approchant de Monte-Cristo: «surtout devant M. Cavalcanti fils», ajouta le banquier, qui, en prononçant ces mots, se tourna en souriant du côté du jeune homme. 

Morcerf avait quitté le comte pour aller parler à sa mère. Danglars le quitta pour saluer Cavalcanti fils. Monte-Cristo se trouva un instant seul. 

Cependant la chaleur commençait à devenir excessive. 

Les valets circulaient dans les salons avec des plateaux chargés de fruits et de glaces. 

Monte-Cristo essuya avec son mouchoir son visage mouillé de sueur; mais il se recula quand le plateau passa devant lui, et ne prit rien pour se rafraîchir. 

Mme de Morcerf ne perdait pas du regard Monte-Cristo. Elle vit passer le plateau sans qu'il y touchât; elle saisit même le mouvement par lequel il s'en éloigna.  

«Albert, dit-elle, avez-vous remarqué une chose? 

—Laquelle, ma mère? 

—C'est que le comte n'a jamais voulu accepter de dîner chez M. de Morcerf. 

—Oui, mais il a accepté de déjeuner chez moi, puisque c'est par ce déjeuner qu'il a fait son entrée dans le monde. 

—Chez vous n'est pas chez le comte, murmura Mercédès, et, depuis qu'il est ici, je l'examine. 

—Eh bien? 

—Eh bien, il n'a encore rien pris. 

—Le comte est très sobre.» 

Mercédès sourit tristement. 

«Rapprochez-vous de lui, dit-elle, et, au premier plateau qui passera, insistez. 

—Pourquoi cela, ma mère? 

—Faites-moi ce plaisir, Albert», dit Mercédès. 

Albert baisa la main de sa mère, et alla se placer près du comte. 

Un autre plateau passa chargé comme les précédents; elle vit Albert insister près du comte, prendre même une glace et la lui présenter, mais il refusa obstinément. 

Albert revint près de sa mère; la comtesse était très pâle. 

«Eh bien, dit-elle, vous voyez, il a refusé. 

—Oui; mais en quoi cela peut-il vous préoccuper? 

—Vous le savez, Albert, les femmes sont singulières. J'aurais vu avec plaisir le comte prendre quelque chose chez moi, ne fût-ce qu'un grain de grenade. Peut-être au reste ne s'accommode-t-il pas des coutumes françaises, peut-être a-t-il des préférences pour quelque chose. 

—Mon Dieu, non! je l'ai vu en Italie prendre de tout; sans doute qu'il est mal disposé ce soir. 

—Puis, dit la comtesse, ayant toujours habité des climats brillants, peut-être est-il moins sensible qu'un autre à la chaleur? 

—Je ne crois pas, car il se plaignait d'étouffer, demandait pourquoi, puisqu'on a déjà ouvert les fenêtres, on n'a pas aussi ouvert les jalousies. 

—En effet, dit Mercédès, c'est un moyen de m'assurer si cette abstinence est un parti pris.» 

Et elle sortit du salon. 

Un instant après, les persiennes s'ouvrirent, et l'on put, à travers les jasmins et les clématites qui garnissaient les fenêtres, voir tout le jardin illuminé avec les lanternes et le souper servi sous la tente. 

Danseurs et danseuses, joueurs et causeurs poussèrent un cri de joie: tous ces poumons altérés aspiraient avec délices l'air qui entrait à flots. 

Au même moment, Mercédès reparut, plus pâle qu'elle n'était sortie, mais avec cette fermeté de visage qui était remarquable chez elle dans certaines circonstances. Elle alla droit au groupe dont son mari formait le centre: 

«N'enchaînez pas ces messieurs ici, monsieur le comte, dit-elle, ils aimeront autant, s'ils ne jouent pas, respirer au jardin qu'étouffer ici. 

—Ah! madame, dit un vieux général fort galant, qui avait chanté: \textit{Partons pour la Syrie}! en 1809, nous n'irons pas seuls au jardin. 

—Soit, dit Mercédès, je vais donc donner l'exemple.» 

Et se retournant vers Monte-Cristo: 

«Monsieur le comte, dit-elle, faites-moi l'honneur de m'offrir votre bras.» 

Le comte chancela presque à ces simples paroles; puis il regarda un moment Mercédès. Ce moment eut la rapidité de l'éclair, et cependant il parut à la comtesse qu'il durait un siècle, tant Monte-Cristo avait mis de pensées dans ce seul regard. Il offrit son bras à la comtesse; elle s'y appuya, ou, pour mieux dire, elle l'effleura de sa petite main, et tous deux descendirent un des escaliers du perron bordé de rhododendrons et de camélias. Derrière eux, et par l'autre escalier, s'élancèrent dans le jardin, avec de bruyantes exclamations de plaisir, une vingtaine de promeneurs. 