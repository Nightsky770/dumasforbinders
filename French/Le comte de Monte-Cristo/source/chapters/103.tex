\chapter{Maximilien}

\lettrine{V}{illefort} se releva presque honteux d'avoir été surpris dans l'accès de cette douleur. 

\zz
Le terrible état qu'il exerçait depuis vingt-cinq ans était arrivé à en faire plus ou moins qu'un homme. 

Son regard, un instant égaré, se fixa sur Morrel. 

«Qui êtes-vous, monsieur, dit-il, vous qui oubliez qu'on n'entre pas ainsi dans une maison qu'habite la mort? 

«Sortez, monsieur! sortez!» 

Mais Morrel demeurait immobile, il ne pouvait détacher ses yeux du spectacle effrayant de ce lit en désordre et de la pâle figure qui était couchée dessus. 

«Sortez, entendez-vous!» cria Villefort, tandis que d'Avrigny s'avançait de son côté pour faire sortir Morrel. 

Celui-ci regarda d'un air égaré ce cadavre, ces deux hommes, toute la chambre, sembla hésiter un instant, ouvrit la bouche; puis enfin, ne trouvant pas un mot à répondre, malgré l'innombrable essaim d'idées fatales qui envahissaient son cerveau, il rebroussa chemin en enfonçant ses mains dans ses cheveux; de telle sorte que Villefort et d'Avrigny, un instant distraits de leurs préoccupations, échangèrent, après l'avoir suivi des yeux, un regard qui voulait dire: 

«Il est fou!» 

Mais avant que cinq minutes se fussent écoulées, on entendit gémir l'escalier sous un poids considérable, et l'on vit Morrel qui, avec une force surhumaine, soulevant le fauteuil de Noirtier entre ses bras, apportait le vieillard au premier étage de la maison. 

Arrivé au haut de l'escalier, Morrel posa le fauteuil à terre et le roula rapidement jusque dans la chambre de Valentine. 

Toute cette manœuvre s'exécuta avec une force décuplée par l'exaltation frénétique du jeune homme. 

Mais une chose était effrayante surtout, c'était la figure de Noirtier s'avançant vers le lit de Valentine, poussé par Morrel, la figure de Noirtier en qui l'intelligence déployait toutes ses ressources, dont les yeux réunissaient toute leur puissance pour suppléer aux autres facultés. 

Aussi ce visage pâle, au regard enflammé, fut-il pour Villefort une effrayante apparition. 

Chaque fois qu'il s'était trouvé en contact avec son père, il s'était toujours passé quelque chose de terrible. 

«Voyez ce qu'ils en ont fait! cria Morrel, une main encore appuyée au dossier du fauteuil qu'il venait de pousser jusqu'au lit, et l'autre étendue vers Valentine; voyez, mon père, voyez!» 

Villefort recula d'un pas et regarda avec étonnement ce jeune homme qui lui était presque inconnu, et qui appelait Noirtier son père. 

En ce moment toute l'âme du vieillard sembla passer dans ses yeux, qui s'injectèrent de sang; puis les veines de son cou se gonflèrent, une teinte bleuâtre comme celle qui envahit la peau de l'épileptique, couvrit son cou, ses joues et ses tempes; il ne manquait à cette explosion intérieure de tout l'être qu'un cri. 

Ce cri sortit pour ainsi dire de tous les pores, effrayant dans son mutisme, déchirant dans son silence. 

D'Avrigny se précipita vers le vieillard et lui fit respirer un violent révulsif. 

«Monsieur! s'écria alors Morrel, en saisissant la main inerte du paralytique, on me demande ce que je suis, et quel droit j'ai d'être ici. Ô vous qui le savez, dites-le, vous! dites-le!» 

Et la voix du jeune homme s'éteignit dans les sanglots. 

Quant au vieillard, sa respiration haletante secouait sa poitrine. On eût dit qu'il était en proie à ces agitations qui précèdent l'agonie. 

Enfin, les larmes vinrent jaillir des yeux de Noirtier, plus heureux que le jeune homme qui sanglotait sans pleurer. Sa tête ne pouvant se pencher, ses yeux se fermèrent. 

«Dites, continua Morrel d'une voix étranglée, dites que j'étais son fiancé! 

«Dites qu'elle était ma noble amie, mon seul amour sur la terre! 

«Dites, dites, dites, que ce cadavre m'appartient!» 

Et le jeune homme, donnant le terrible spectacle d'une grande force qui se brise, tomba lourdement à genoux devant ce lit que ses doigts crispés étreignirent avec violence. 

Cette douleur était si poignante que d'Avrigny se détourna pour cacher son émotion, et que Villefort, sans demander d'autre explication, attiré par ce magnétisme qui nous pousse vers ceux qui ont aimé ceux que nous pleurons, tendit sa main au jeune homme. 

Mais Morrel ne voyait rien; il avait saisi la main glacée de Valentine, et, ne pouvant parvenir à pleurer, il mordait les draps en rugissant. 

Pendant quelque temps, on n'entendit dans cette chambre que le conflit des sanglots, des imprécations et de la prière. Et cependant un bruit dominait tous ceux-là, c'était l'aspiration rauque et déchirante qui semblait, à chaque reprise d'air, rompre un des ressorts de la vie dans la poitrine de Noirtier. 

Enfin, Villefort, le plus maître de tous, après avoir pour ainsi dire cédé pendant quelque temps sa place à Maximilien, Villefort prit la parole. 

«Monsieur, dit-il à Maximilien, vous aimiez Valentine, dites-vous: vous étiez son fiancé; j'ignorais cet amour, j'ignorais cet engagement; et cependant, moi, son père, je vous le pardonne, car, je le vois, votre douleur est grande, réelle et vraie. 

«D'ailleurs, chez moi aussi la douleur est trop grande pour qu'il reste en mon cœur place pour la colère.» 

«Mais, vous le voyez, l'ange que vous espériez a quitté la terre: elle n'a plus que faire des adorations des hommes, elle qui, à cette heure, adore le Seigneur; faites donc vos adieux, monsieur, à la triste dépouille qu'elle a oubliée parmi nous; prenez une dernière fois sa main que vous attendiez, et séparez-vous d'elle à jamais: Valentine n'a plus besoin maintenant que du prêtre qui doit la bénir. 

—Vous vous trompez, monsieur, s'écria Morrel en se relevant sur un genou, le cœur traversé par une douleur plus aiguë qu'aucune de celles qu'il eût encore ressenties; vous vous trompez: Valentine, morte comme elle est morte, a non seulement besoin d'un prêtre, mais encore d'un vengeur. 

«Monsieur de Villefort, envoyez chercher le prêtre; moi, je serai le vengeur. 

—Que voulez-vous dire, monsieur? murmura Villefort tremblant à cette nouvelle inspiration du délire de Morrel. 

—Je veux dire, continua Morrel, qu'il y a deux hommes en vous, monsieur. Le père a assez pleuré; que le procureur du roi commence son office.» 

Les yeux de Noirtier étincelèrent, d'Avrigny se rapprocha. 

«Monsieur, continua le jeune homme, en recueillant des yeux tous les sentiments qui se révélaient sur les visages des assistants, je sais ce que je dis, et vous savez tous aussi bien que moi ce que je vais dire. 

«Valentine est morte assassinée!» 

Villefort baissa la tête; d'Avrigny avança d'un pas encore; Noirtier fit oui des yeux. 

«Or, monsieur, continua Morrel, au temps où nous vivons, une créature, ne fût-elle pas jeune, ne fût-elle pas belle, ne fût-elle pas adorable comme était Valentine, une créature ne disparaît pas violemment du monde sans que l'on demande compte de sa disparition. 

«Allons, monsieur le procureur du roi, ajouta Morrel avec une véhémence croissante, pas de pitié! je vous dénonce le crime, cherchez l'assassin!» 

Et son œil implacable interrogeait Villefort, qui de son côté sollicitait du regard tantôt Noirtier, tantôt d'Avrigny. 

Mais au lieu de trouver secours dans son père et dans le docteur, Villefort ne rencontra en eux qu'un regard aussi inflexible que celui de Morrel. 

«Oui! fit le vieillard. 

—Certes! dit d'Avrigny. 

—Monsieur, répliqua Villefort, essayant de lutter contre cette triple volonté et contre sa propre émotion, monsieur, vous vous trompez, il ne se commet pas de crimes chez moi; la fatalité me frappe, Dieu m'éprouve; c'est horrible à penser; mais on n'assassine personne!» 

Les yeux de Noirtier flamboyèrent, d'Avrigny ouvrit la bouche pour parler. 

Morrel étendit le bras en commandant le silence. 

«Et moi, je vous dis que l'on tue ici! s'écria Morrel dont la voix baissa sans rien perdre de sa vibration terrible. 

«Je vous dis que voilà la quatrième victime frappée depuis quatre mois. 

«Je vous dis qu'on avait déjà une fois, il y a quatre jours de cela, essayé d'empoisonner Valentine, et que l'on avait échoué grâce aux précautions qu'avait prises M. Noirtier! 

«Je vous dis que l'on a doublé la dose ou changé la nature du poison, et que cette fois on a réussi! 

«Je vous dis que vous savez tout cela aussi bien que moi, enfin, puisque monsieur que voilà vous en a prévenu, et comme médecin et comme ami. 

—Oh, vous êtes en délire! monsieur, dit Villefort, essayant vainement de se débattre dans le cercle où il se sentait pris. 

—Je suis en délire! s'écria Morrel; eh bien, j'en appelle à M. d'Avrigny lui-même. 

«Demandez-lui, monsieur, s'il se souvient encore des paroles qu'il a prononcées dans votre jardin, dans le jardin de cet hôtel, le soir même de la mort de Mme de Saint-Méran, alors que tous deux, vous et lui, vous croyant seuls, vous vous entreteniez de cette mort tragique, dans laquelle cette fatalité dont vous parlez et Dieu, que vous accusez injustement, ne peuvent être comptés que pour une chose; c'est-à-dire pour avoir créé l'assassin de Valentine!» 

Villefort et d'Avrigny se regardèrent. 

«Oui, oui, rappelez-vous, dit Morrel, car ces paroles, que vous croyiez livrées au silence et à la solitude sont tombées dans mon oreille. Certes, de ce soir-là, en voyant la coupable complaisance de M. de Villefort pour les siens, j'eusse dû tout découvrir à l'autorité; je ne serais pas complice comme je le suis en ce moment de ta mort, Valentine! ma Valentine bien-aimée! mais le complice deviendra le vengeur; ce quatrième meurtre est flagrant et visible aux yeux de tous, et si ton père t'abandonne, Valentine, c'est moi, c'est moi, je te le jure, qui poursuivrai l'assassin.» 

Et cette fois, comme si la nature avait enfin pitié de cette vigoureuse organisation prête à se briser par sa propre force, les dernières paroles de Morrel s'éteignirent dans sa gorge; sa poitrine éclata en sanglots, les larmes, si longtemps rebelles, jaillirent de ses yeux, il s'affaissa sur lui-même, et retomba à genoux pleurant près du lit de Valentine. 

Alors ce fut le tour de d'Avrigny. 

«Et moi aussi, dit-il d'une voix forte, moi aussi, je me joins à M. Morrel pour demander justice du crime; car mon cœur se soulève à l'idée que ma lâche complaisance a encouragé l'assassin! 

—Ô mon Dieu! mon Dieu!» murmura Villefort anéanti. 

Morrel releva la tête, en lisant dans les yeux du vieillard qui lançaient une flamme surnaturelle: 

«Tenez, dit-il, tenez, M. Noirtier veut parler. 

—Oui, fit Noirtier avec une expression d'autant plus terrible que toutes les facultés de ce pauvre vieillard impuissant étaient concentrées dans son regard. 

—Vous connaissez l'assassin? dit Morrel. 

—Oui, répliqua Noirtier. 

—Et vous allez nous guider? s'écria le jeune homme. Écoutons! M. d'Avrigny, écoutons!» 

Noirtier adressa au malheureux Morrel un sourire mélancolique, un de ces doux sourires des yeux qui tant de fois avaient rendu Valentine heureuse, et fixa son attention. 

Puis, ayant rivé pour ainsi dire les yeux de son interlocuteur aux siens, il les détourna vers la porte. 

«Voulez-vous que je sorte, monsieur? s'écria douloureusement Morrel. 

—Oui, fit Noirtier. 

—Hélas! hélas! monsieur; mais ayez donc pitié de moi!» 

Les yeux du vieillard demeurèrent impitoyablement fixés vers la porte. 

«Pourrais-je revenir, au moins? demanda Morrel. 

—Oui. 

—Dois-je sortir seul? 

—Non. 

—Qui dois-je emmener avec moi? M. le procureur au roi? 

—Non. 

—Le docteur? 

—Oui. 

—Vous voulez rester seul avec M. de Villefort? 

—Oui. 

—Mais pourrait-il vous comprendre, lui? 

—Oui. 

—Oh! dit Villefort presque joyeux de ce que l'enquête allait se faire en tête-à-tête, oh! soyez tranquille, je comprends très bien mon père.» 

Et tout en disant cela avec cette expression de joie que nous avons signalée, les dents du procureur du roi s'entrechoquaient avec violence. 

D'Avrigny prit le bras de Morrel et entraîna le jeune homme dans la chambre voisine. 

Il se fit alors dans toute cette maison un silence plus profond que celui de la mort. 

Enfin, au bout d'un quart d'heure, un pas chancelant se fit entendre, et Villefort parut sur le seuil du salon où se tenaient d'Avrigny et Morrel, l'un absorbé et l'autre suffoquant. 

«Venez», dit-il. 

Et il les ramena près du fauteuil de Noirtier. 

Morrel, alors, regarda attentivement Villefort. 

La figure du procureur du roi était livide; de larges taches de couleur de rouille sillonnaient son front; entre ses doigts, une plume tordue de mille façons criait en se déchiquetant en lambeaux. 

«Messieurs, dit-il d'une voix étranglée à d'Avrigny et à Morrel, messieurs, votre parole d'honneur que l'horrible secret demeurera enseveli entre nous!» 

Les deux hommes firent un mouvement. 

«Je vous en conjure!\dots continua Villefort. 

—Mais, dit Morrel, le coupable!\dots le meurtrier!\dots l'assassin!\dots 

—Soyez tranquille, monsieur, justice sera faite, dit Villefort. Mon père m'a révélé le nom du coupable; mon père a soif de vengeance comme vous, et cependant mon père vous conjure, comme moi de garder le secret du crime. 

«N'est-ce pas, mon père? 

—Oui», fit résolument Noirtier. 

Morrel laissa échapper un mouvement d'horreur et d'incrédulité. 

«Oh! s'écria Villefort, en arrêtant Maximilien par le bras, oh! monsieur, si mon père, l'homme inflexible que vous connaissez, vous fait cette demande, c'est qu'il sait que Valentine sera terriblement vengée. 

«N'est-ce pas, mon père?» 

Le vieillard fit signe que oui. 

Villefort continua. 

«Il me connaît, lui, et c'est à lui que j'ai engagé ma parole. Rassurez-vous donc, messieurs; trois jours, je vous demande trois jours, c'est moins que ne vous demanderait la justice, et dans trois jours la vengeance que j'aurai tirée du meurtre de mon enfant fera frissonner jusqu'au fond de leur cœur les plus indifférents des hommes. 

«N'est-ce pas, mon père?» 

Et en disant ces paroles, il grinçait des dents et secouait la main engourdie du vieillard. 

«Tout ce qui est promis sera-t-il tenu, monsieur Noirtier? demanda Morrel, tandis que d'Avrigny interrogeait du regard. 

—Oui, fit Noirtier, avec un regard de sinistre joie. 

—Jurez donc, messieurs, dit Villefort en joignant les mains de d'Avrigny et de Morrel, jurez que vous aurez pitié de l'honneur de ma maison, et que vous me laisserez le soin de le venger?» 

D'Avrigny se détourna et murmura un oui bien faible, mais Morrel arracha sa main du magistrat, se précipita vers le lit, imprima ses lèvres sur les lèvres glacées de Valentine, et s'enfuit avec le long gémissement d'une âme qui s'engloutit dans le désespoir. 

Nous avons dit que tous les domestiques avaient disparu. 

M. de Villefort fut donc forcé de prier d'Avrigny de se charger des démarches, si nombreuses et si délicates, qu'entraîne la mort dans nos grandes villes, et surtout la mort accompagnée de circonstances aussi suspectes. 

Quant à Noirtier, c'était quelque chose de terrible à voir que cette douleur sans mouvement, que ce désespoir sans gestes, que ces larmes sans voix. 

Villefort rentra dans son cabinet; d'Avrigny alla chercher le médecin de la mairie qui remplit les fonctions d'inspecteur après décès, et que l'on nomme assez énergiquement le médecin des morts. 

Noirtier ne voulut point quitter sa petite-fille. 

Au bout d'une demi-heure, M. d'Avrigny revint avec son confrère; on avait fermé les portes de la rue, et comme le concierge avait disparu avec les autres serviteurs, ce fut Villefort lui-même qui alla ouvrir. 

Mais il s'arrêta sur le palier; il n'avait plus le courage d'entrer dans la chambre mortuaire. 

Les deux docteurs pénétrèrent donc seuls jusqu'à la chambre de Valentine. 

Noirtier était près du lit, pâle comme la morte, immobile et muet comme elle. 

Le médecin des morts s'approcha avec l'indifférence de l'homme qui passe la moitié de sa vie avec les cadavres, souleva le drap qui recouvrait la jeune fille, et entrouvrit seulement les lèvres. 

«Oh! dit d'Avrigny en soupirant, pauvre jeune fille, elle est bien morte, allez. 

—Oui», répondit laconiquement le médecin en laissant retomber le drap qui recouvrait le visage de Valentine. 

Noirtier fit entendre un sourd râlement. 

D'Avrigny se retourna, les yeux du vieillard étincelaient. Le bon docteur comprit que Noirtier réclamait la vue de son enfant, il le rapprocha du lit, et tandis que le médecin des morts trempait dans de l'eau chlorurée les doigts qui avaient touché les lèvres de la trépassée, il découvrit ce calme et pâle visage qui semblait celui d'un ange endormi. 

Une larme qui reparut au coin de l'œil de Noirtier fut le remerciement que reçut le bon docteur. 

Le médecin des morts dressa son procès-verbal sur le coin d'une table, dans la chambre même de Valentine, et, cette formalité suprême accomplie, sortit reconduit par le docteur. 

Villefort les entendit descendre et reparut à la porte de son cabinet. 

En quelques mots il remercia le médecin, et, se retournant vers d'Avrigny: 

«Et maintenant! dit-il, le prêtre? 

—Avez-vous un ecclésiastique que vous désirez plus particulièrement charger de prier près de Valentine? demanda d'Avrigny. 

—Non, dit Villefort, allez chez le plus proche. 

—Le plus proche, fit le médecin est un bon abbé italien qui est venu demeurer dans la maison voisine de la vôtre. Voulez-vous que je le prévienne en passant? 

—D'Avrigny, dit Villefort, veuillez, je vous prie, accompagner monsieur. 

«Voici la clef pour que vous puissiez entrer et sortir à volonté. 

«Vous ramènerez le prêtre, et vous vous chargerez de l'installer dans la chambre de ma pauvre enfant. 

—Désirez-vous lui parler, mon ami? 

—Je désire être seul. Vous m'excuserez, n'est-ce pas? Un prêtre doit comprendre toutes les douleurs, même la douleur paternelle.» 

Et M. de Villefort, donnant un passe-partout à d'Avrigny, salua une dernière fois le docteur étranger et rentra dans son cabinet, où il se mit à travailler. 

Pour certaines organisations, le travail est le remède à toutes les douleurs. 

Au moment où ils descendaient dans la rue, ils aperçurent un homme vêtu d'une soutane, qui se tenait sur le seuil de la porte voisine. 

«Voici celui dont je vous parlais», dit le médecin des morts à d'Avrigny. 

D'Avrigny aborda l'ecclésiastique. 

«Monsieur, lui dit-il, seriez-vous disposé à rendre un grand service à un malheureux père qui vient de perdre sa fille, à M. le procureur du roi Villefort? 

—Ah! monsieur, répondit le prêtre avec un accent italien des plus prononcés, oui, je sais, la mort est dans sa maison. 

—Alors, je n'ai point à vous apprendre quel genre de service il ose attendre de vous. 

—J'allais aller m'offrir, monsieur, dit le prêtre; c'est notre mission d'aller au-devant de nos devoirs. 

—C'est une jeune fille. 

—Oui, je sais cela, je l'ai appris des domestiques que j'ai vus fuyant la maison. J'ai su qu'elle s'appelait Valentine; et j'ai déjà prié pour elle. 

—Merci, merci, monsieur, dit d'Avrigny, et puisque vous avez déjà commencé d'exercer votre saint ministère, daignez le continuer. Venez vous asseoir près de la morte, et toute une famille plongée dans le deuil vous sera bien reconnaissante. 

—J'y vais, monsieur, répondit l'abbé, et j'ose dire que jamais prières ne seront plus ardentes que les miennes.» 

D'Avrigny prit l'abbé par la main, et sans rencontrer Villefort, enfermé dans son cabinet, il le conduisit jusqu'à la chambre de Valentine, dont les ensevelisseurs devaient s'emparer seulement la nuit suivante. 

En entrant dans la chambre, le regard de Noirtier avait rencontré celui de l'abbé, et sans doute il crut y lire quelque chose de particulier, car il ne le quitta plus. 

D'Avrigny recommanda au prêtre non seulement la morte, mais le vivant, et le prêtre promit à d'Avrigny de donner ses prières à Valentine et ses soins à Noirtier. 

L'abbé s'y engagea solennellement, et, sans doute pour n'être pas dérangé dans ses prières, et pour que Noirtier ne fût pas dérangé dans sa douleur, il alla, dès que M. d'Avrigny eut quitté la chambre, fermer non seulement les verrous de la porte par laquelle le docteur venait de sortir, mais encore les verrous de celle qui conduisait chez Mme de Villefort. 