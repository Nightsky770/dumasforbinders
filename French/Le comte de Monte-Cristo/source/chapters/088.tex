\chapter{L'insulte} 

\lettrine{\accentletter[\gravebox]{A}}{} la porte du banquier, Beauchamp arrêta Morcerf. 

\zz
«Écoutez, lui dit-il, tout à l'heure je vous ai dit, chez M. Danglars, que c'était à M. de Monte-Cristo que vous deviez demander une explication? 

—Oui, et nous allons chez lui. 

—Un instant, Morcerf; avant d'aller chez le comte, réfléchissez. 

—À quoi voulez-vous que je réfléchisse? 

—À la gravité de la démarche. 

—Est-elle plus grave que d'aller chez M. Danglars? 

—Oui; M. Danglars était un homme d'argent, et vous le savez, les hommes d'argent savent trop le capital qu'ils risquent pour se battre facilement. L'autre au contraire, est un gentilhomme, en apparence du moins; mais ne craignez-vous pas, sous le gentilhomme, de rencontrer le bravo? 

—Je ne crains qu'une chose, c'est de trouver un homme qui ne se batte pas. 

—Oh! soyez tranquille, dit Beauchamp, celui-là se battra. J'ai même peur d'une chose, c'est qu'il ne se batte trop bien; prenez garde! 

—Ami, dit Morcerf avec un beau sourire, c'est ce que je demande; et ce qui peut m'arriver de plus heureux, c'est d'être tué pour mon père: cela nous sauvera tous. 

—Votre mère en mourra! 

—Pauvre mère! dit Albert en passant la main sur ses yeux, je le sais bien; mais mieux vaut qu'elle meure de cela que de mourir de honte. 

—Vous êtes bien décidé, Albert? 

—Oui. 

—Allez donc! Mais croyez-vous que nous le trouvions? 

—Il devait revenir quelques heures après moi, et certainement il sera revenu.» 

Ils montèrent, et se firent conduire avenue des Champs-Élysées, n° 30. 

Beauchamp voulait descendre seul, mais Albert lui fit observer que cette affaire, sortant des règles ordinaires, lui permettait de s'écarter de l'étiquette du duel. 

Le jeune homme agissait dans tout ceci pour une cause si sainte, que Beauchamp n'avait autre chose à faire qu'à se prêter à toutes ses volontés: il céda donc à Morcerf et se contenta de le suivre. 

Albert ne fit qu'un bond de la loge du concierge au perron. Ce fut Baptistin qui le reçut. 

Le comte venait d'arriver effectivement, mais il était au bain, et avait défendu de recevoir qui que ce fût au monde. 

«Mais, après le bain? demanda Morcerf. 

—Monsieur dînera. 

—Et après le dîner? 

—Monsieur dormira une heure. 

—Ensuite? 

—Ensuite il ira à l'Opéra. 

—Vous en êtes sûr? demanda Albert. 

—Parfaitement sûr; monsieur a commandé ses chevaux pour huit heures précises. 

—Fort bien, répliqua Albert; voilà tout ce que je voulais savoir.» 

Puis, se retournant vers Beauchamp: 

«Si vous avez quelque chose à faire, Beauchamp, faites-le tout de suite; si vous avez rendez-vous ce soir, remettez-le à demain. Vous comprenez que je compte sur vous pour aller à l'Opéra. Si vous le pouvez, amenez-moi Château-Renaud.» 

Beauchamp profita de la permission et quitta Albert après lui avoir promis de le venir prendre à huit heures moins un quart. 

Rentré chez lui, Albert prévint Franz, Debray et Morrel du désir qu'il avait de les voir le soir même à l'Opéra. 

Puis il alla visiter sa mère, qui, depuis les événements de la veille, avait fait défendre sa porte et gardait la chambre. Il la trouva au lit, écrasée par la douleur de cette humiliation publique. 

La vue d'Albert produisit sur Mercédès l'effet qu'on en pouvait attendre; elle serra la main de son fils et éclata en sanglots. Cependant ces larmes la soulagèrent. 

Albert demeura un instant debout et muet près du visage de sa mère. On voyait à sa mine pâle et à ses sourcils froncés que sa résolution de vengeance s'émoussait de plus en plus dans son cœur. 

«Ma mère, demanda Albert, est-ce que vous connaissez quelque ennemi à M. de Morcerf?» 

Mercédès tressaillit; elle avait remarqué que le jeune homme n'avait pas dit: à mon père. 

«Mon ami, dit-elle, les gens dans la position du comte ont beaucoup d'ennemis qu'ils ne connaissent point. D'ailleurs, les ennemis qu'on connaît ne sont point, vous le savez, les plus dangereux. 

—Oui, je sais cela, aussi j'en appelle à toute votre perspicacité. Ma mère, vous êtes une femme si supérieure que rien ne vous échappe, à vous! 

—Pourquoi me dites-vous cela? 

—Parce que vous aviez remarqué, par exemple, que le soir du bal que nous avons donné, M. de Monte-Cristo n'avait rien voulu prendre chez nous.» 

Mercédès se soulevant toute tremblante sur son bras brûlé par la fièvre: 

«M. de Monte-Cristo! s'écria-t-elle, et quel rapport cela aurait-il avec la question que vous me faites? 

—Vous le savez, ma mère, M. de Monte-Cristo est presque un homme d'Orient, et les Orientaux, pour conserver toute liberté de vengeance, ne mangent ni ne boivent jamais chez leurs ennemis. 

—M. de Monte-Cristo, notre ennemi, dites-vous, Albert? reprit Mercédès en devenant plus pâle que le drap qui la couvrait. Qui vous a dit cela? pourquoi? Vous êtes fou, Albert. M. de Monte-Cristo n'a eu pour nous que des politesses. M. de Monte-Cristo vous a sauvé la vie, c'est vous-même qui nous l'avez présenté. Oh! je vous en prie, mon fils, si vous aviez une pareille idée, écartez-la, et si j'ai une recommandation à vous faire, je dirai plus, si j'ai une prière à vous adresser, tenez-vous bien avec lui. 

—Ma mère, répliqua le jeune homme avec un sombre regard, vous avez vos raisons pour me dire de ménager cet homme. 

—Moi! s'écria Mercédès, rougissant avec la même rapidité qu'elle avait pâli, et redevenant presque aussitôt plus pâle encore qu'auparavant. 

—Oui, sans doute, et cette raison, n'est-ce pas, reprit Albert, est que cet homme ne peut nous faire du mal?» 

Mercédès frissonna; et attachant sur son fils un regard scrutateur: 

«Vous me parlez étrangement, dit-elle à Albert, et vous avez de singulières préventions, ce me semble. Que vous a donc fait le comte? Il y a trois jours vous étiez avec lui en Normandie; il y a trois jours je le regardais et vous le regardiez vous-même comme votre meilleur ami.» 

Un sourire ironique effleura les lèvres d'Albert. Mercédès vit ce sourire, et avec son double instinct de femme et de mère elle devina tout; mais, prudente et forte, elle cacha son trouble et ses frémissements. 

Albert laissa tomber la conversation; au bout d'un instant la comtesse la renoua. 

«Vous veniez me demander comment j'allais, dit-elle, je vous répondrai franchement, mon ami, que je ne me sens pas bien. Vous devriez vous installer ici, Albert, vous me tiendriez compagnie; j'ai besoin de n'être pas seule. 

—Ma mère, dit le jeune homme, je serais à vos ordres, et vous savez avec quel bonheur, si une affaire pressée et importante ne me forçait à vous quitter toute la soirée. 

—Ah! fort bien, répondit Mercédès avec un soupir; allez, Albert, je ne veux point vous rendre esclave de votre piété filiale.» 

Albert fit semblant de ne point entendre, salua sa mère et sortit. À peine le jeune homme eut-il refermé la porte que Mercédès fit appeler un domestique de confiance et lui ordonna de suivre Albert partout où il irait dans la soirée, et de lui en venir rendre compte à l'instant même. 

Puis elle sonna sa femme de chambre, et, si faible qu'elle fût, se fit habiller pour être prête à tout événement. 

La mission donnée au laquais n'était pas difficile à exécuter. Albert rentra chez lui et s'habilla avec une sorte de recherche sévère. À huit heures moins dix minutes Beauchamp arriva: il avait vu Château-Renaud, lequel avait promis de se trouver à l'orchestre avant le lever du rideau. 

Tous deux montèrent dans le coupé d'Albert, qui n'ayant aucune raison de cacher où il allait, dit tout haut: 

«À l'Opéra!» 

Dans son impatience, il avait devancé le lever du rideau. Château-Renaud était à sa stalle: prévenu de tout par Beauchamp, Albert n'avait aucune explication à lui donner. La conduite de ce fils cherchant à venger son père était si simple, que Château-Renaud ne tenta en rien de le dissuader, et se contenta de lui renouveler l'assurance qu'il était à sa disposition. 

Debray n'était pas encore arrivé, mais Albert savait qu'il manquait rarement une représentation de l'Opéra. Albert erra dans le théâtre jusqu'au lever du rideau. Il espérait rencontrer Monte-Cristo, soit dans le couloir, soit dans l'escalier. La sonnette l'appela à sa place, et il vint s'asseoir à l'orchestre, entre Château-Renaud et Beauchamp. 

Mais ses yeux ne quittaient pas cette loge d'entre-colonnes qui, pendant tout le premier acte, semblait s'obstiner à rester fermée. 

Enfin, comme Albert, pour la centième fois, interrogeait sa montre, au commencement du deuxième acte, la porte de la loge s'ouvrit, et Monte-Cristo, vêtu de noir, entra et s'appuya à la rampe pour regarder dans la salle; Morrel le suivait, cherchant des yeux sa sœur et son beau-frère. Il les aperçut dans une loge du second rang, et leur fit signe. 

Le comte, en jetant son coup d'œil circulaire dans la salle, aperçut une tête pâle et des yeux étincelants qui semblaient attirer avidement ses regards; il reconnut bien Albert, mais l'expression qu'il remarqua sur ce visage bouleversé lui conseilla sans doute de ne point l'avoir remarqué. Sans faire donc aucun mouvement qui décelât sa pensée, il s'assit, tira sa jumelle de son étui, et lorgna d'un autre côté. 

Mais, sans paraître voir Albert, le comte ne le perdait pas de vue, et, lorsque la toile tomba sur la fin du second acte, son coup d'œil infaillible et sûr suivit le jeune homme sortant de l'orchestre et accompagné de ses deux amis. 

Puis, la même tête reparut aux carreaux d'une première loge, en face de la sienne. Le comte sentait venir à lui la tempête, et lorsqu'il entendit la clef tourner dans la serrure de sa loge, quoiqu'il parlât en ce moment même à Morrel avec son visage le plus riant, le comte savait à quoi s'en tenir, et il s'était préparé à tout. 

La porte s'ouvrit. 

Seulement alors, Monte-Cristo se retourna et aperçut Albert, livide et tremblant; derrière lui étaient Beauchamp et Château-Renaud. 

«Tiens! s'écria-t-il avec cette bienveillante politesse qui distinguait d'habitude son salut des banales civilités du monde, voilà mon cavalier arrivé au but! Bonsoir, monsieur de Morcerf.» 

Et le visage de cet homme, si singulièrement maître de lui-même, exprimait la plus parfaite cordialité. 

Morrel alors se rappela seulement la lettre qu'il avait reçue du vicomte, et dans laquelle, sans autre explication, celui-ci le priait de se trouver à l'Opéra; et il comprit qu'il allait se passer quelque chose de terrible. 

«Nous ne venons point ici pour échanger d'hypocrites politesses ou de faux-semblants d'amitié, dit le jeune homme; nous venons vous demander une explication, monsieur le comte.» 

La voix tremblante du jeune homme avait peine à passer entre ses dents serrées. 

«Une explication à l'Opéra? dit le comte avec ce ton si calme et avec ce coup d'œil si pénétrant, qu'on reconnaît à ce double caractère l'homme éternellement sûr de lui-même. Si peu familier que je sois avec les habitudes parisiennes, je n'aurais pas cru, monsieur, que ce fût là que les explications se demandaient. 

—Cependant, lorsque les gens se font celer, dit Albert, lorsqu'on ne peut pénétrer jusqu'à eux sous prétexte qu'ils sont au bain, à table ou au lit, il faut bien s'adresser là où on les rencontre. 

—Je ne suis pas difficile à rencontrer, dit Monte-Cristo, car hier encore, monsieur, si j'ai bonne mémoire, vous étiez chez moi. 

—Hier, monsieur, dit le jeune homme, dont la tête s'embarrassait, j'étais chez vous parce que j'ignorais qui vous étiez.» 

Et en prononçant ces paroles, Albert avait élevé la voix de manière à ce que les personnes placées dans les loges voisines l'entendissent, ainsi que celles qui passaient dans le couloir. Aussi les personnes des loges se retournèrent-elles, et celles du couloir s'arrêtèrent-elles derrière Beauchamp et Château-Renaud au bruit de cette altercation. 

«D'où sortez-vous donc, monsieur? dit Monte-Cristo sans la moindre émotion apparente. Vous ne semblez pas jouir de votre bon sens. 

—Pourvu que je comprenne vos perfidies, monsieur, et que je parvienne à vous faire comprendre que je veux m'en venger, je serai toujours assez raisonnable, dit Albert furieux. 

—Monsieur, je ne vous comprends point, répliqua Monte-Cristo, et, quand même je vous comprendrais, vous n'en parleriez encore que trop haut. Je suis ici chez moi, monsieur, et moi seul ai le droit d'y élever la voix au-dessus des autres. Sortez, monsieur!» 

Et Monte-Cristo montra la porte à Albert avec un geste admirable de commandement. 

«Ah! je vous en ferai bien sortir, de chez vous! reprit Albert en froissant dans ses mains convulsives son gant, que le comte ne perdait pas de vue. 

—Bien, bien! dit flegmatiquement Monte-Cristo; vous me cherchez querelle, monsieur; je vois cela; mais un conseil, vicomte, et retenez-le bien: c'est une coutume mauvaise que de faire du bruit en provoquant. Le bruit ne va pas à tout le monde, monsieur de Morcerf.» 

À ce nom, un murmure d'étonnement passa comme un frisson parmi les auditeurs de cette scène. Depuis la veille le nom de Morcerf était dans toutes les bouches. 

Albert mieux que tous, et le premier de tous, comprit l'allusion, et fit un geste pour lancer son gant au visage du comte; mais Morrel lui saisit le poignet, tandis que Beauchamp et Château-Renaud, craignant que la scène ne dépassât la limite d'une provocation, le retenaient par-derrière. 

Mais Monte-Cristo, sans se lever, en inclinant sa chaise, étendit la main seulement, et saisissant entre les doigts crispés du jeune homme le gant humide et écrasé: 

«Monsieur, dit-il avec un accent terrible, je tiens votre gant pour jeté, et je vous l'enverrai roulé autour d'une balle. Maintenant, sortez de chez moi, ou j'appelle mes domestiques et je vous fais jeter à la porte.» 

Ivre, effaré, les yeux sanglants, Albert fit deux pas en arrière. 

Morrel en profita pour refermer la porte. 

Monte-Cristo reprit sa jumelle et se remit à lorgner, comme si rien d'extraordinaire ne venait de se passer. 

Cet homme avait un cœur de bronze et un visage de marbre. Morrel se pencha à son oreille. 

«Que lui avez-vous fait? dit-il. 

—Moi? rien, personnellement du moins, dit Monte-Cristo. 

—Cependant cette scène étrange doit avoir une cause? 

—L'aventure du comte de Morcerf exaspère le malheureux jeune homme. 

—Y êtes-vous pour quelque chose? 

—C'est par Haydée que la Chambre a été instruite de la trahison de son père. 

—En effet, dit Morrel, on m'a dit, mais je n'avais pas voulu le croire, que cette esclave grecque que j'ai vue avec vous ici, dans cette loge même, était la fille d'Ali-Pacha. 

—C'est la vérité, cependant. 

—Oh! mon Dieu! dit Morrel, je comprends tout alors, et cette scène était préméditée. 

—Comment cela? 

—Oui, Albert m'a écrit de me trouver ce soir à l'opéra; c'était pour me rendre témoin de l'insulte qu'il voulait vous faire. 

—Probablement, dit Monte-Cristo avec son imperturbable tranquillité. 

—Mais que ferez-vous de lui? 

—De qui? 

—D'Albert! 

—D'Albert? reprit Monte-Cristo du même ton, ce que j'en ferai, Maximilien? Aussi vrai que vous êtes ici et que je vous serre la main, je le tuerai demain avant dix heures du matin. Voilà ce que j'en ferai.» 

Morrel, à son tour, prit la main de Monte-Cristo dans les deux siennes, et il frémit en sentant cette main froide et calme. 

«Ah! comte, dit-il, son père l'aime tant! 

—Ne me dites pas ces choses-là! s'écria Monte-Cristo avec le premier mouvement de colère qu'il eût paru éprouver; je le ferais souffrir!» 

Morrel, stupéfait, laissa tomber la main de Monte-Cristo. 

«Comte! comte! dit-il. 

—Cher Maximilien, interrompit le comte, écoutez de quelle adorable façon Duprez chante cette phrase: \textit{Ô Mathilde! idole de mon âme.} Tenez, j'ai deviné le premier Duprez à Naples et j'ai applaudi le premier. Bravo! bravo!» 

Morrel comprit qu'il n'y avait plus rien à dire, et il attendit. 

La toile, qui s'était levée à la fin de la scène d'Albert, retomba presque aussitôt. On frappa à la porte. 

«Entrez», dit Monte-Cristo sans que sa voix décelât la moindre émotion. 

Beauchamp parut. 

«Bonsoir, monsieur Beauchamp, dit Monte-Cristo, comme s'il voyait le journaliste pour la première fois de la soirée; asseyez-vous donc.» 

Beauchamp salua, entra et s'assit. 

«Monsieur, dit-il à Monte-Cristo, j'accompagnais tout à l'heure, comme vous avez pu le voir, M. de Morcerf. 

—Ce qui veut dire, reprit Monte-Cristo en riant, que vous venez probablement de dîner ensemble. Je suis heureux de voir, monsieur Beauchamp, que vous êtes plus sobre que lui. 

—Monsieur, dit Beauchamp, Albert a eu, j'en conviens, le tort de s'emporter, et je viens pour mon propre compte vous faire des excuses. Maintenant que mes excuses sont faites, les miennes, entendez-vous, monsieur le comte, je viens vous dire que je vous crois trop galant homme pour refuser de me donner quelque explication au sujet de vos relations avec les gens de Janina; puis j'ajouterai deux mots sur cette jeune Grecque.» 

Monte-Cristo fit de la lèvre et des yeux un petit geste qui commandait le silence. 

«Allons! ajouta-t-il en riant, voilà toutes mes espérances détruites. 

—Comment cela? demanda Beauchamp. 

—Sans doute, vous vous empressez de me faire une réputation d'excentricité: je suis, selon vous, un Lara, un Manfred, un Lord Ruthwen; puis, le moment de me voir excentrique passé, vous gâtez votre type, vous essayez de faire de moi un homme banal. Vous me voulez commun, vulgaire; vous me demandez des explications enfin. Allons donc! monsieur Beauchamp, vous voulez rire. 

—Cependant, reprit Beauchamp avec hauteur, il est des occasions où la probité commande\dots 

—Monsieur Beauchamp, interrompit l'homme étrange, ce qui commande à M. le comte de Monte-Cristo, c'est M. le comte de Monte-Cristo. Ainsi donc pas un mot de tout cela, s'il vous plaît. Je fais ce que je veux, monsieur Beauchamp, et, croyez-moi, c'est toujours fort bien fait. 

—Monsieur, répondit le jeune homme, on ne paie pas d'honnêtes gens avec cette monnaie; il faut des garanties à l'honneur. 

—Monsieur, je suis une garantie vivante, reprit Monte-Cristo impassible, mais dont les yeux s'enflammaient d'éclairs menaçants. Nous avons tous deux dans les veines du sang que nous avons envie de verser, voilà notre garantie mutuelle. Reportez cette réponse au vicomte, et dites-lui que demain, avant dix heures, j'aurai vu la couleur du sien. 

—Il ne me reste donc, dit Beauchamp, qu'à fixer les arrangements du combat. 

—Cela m'est parfaitement indifférent, monsieur, dit le comte de Monte-Cristo; il était donc inutile de venir me déranger au spectacle pour si peu de chose. En France, on se bat à l'épée ou au pistolet, aux colonies, on prend la carabine, en Arabie, on a le poignard. Dites à votre client que, quoique insulté pour être excentrique jusqu'au bout, je lui laisse le choix des armes, et que j'accepterai tout sans discussion, sans conteste; tout, entendez-vous bien? tout, même le combat par voie du sort, ce qui est toujours stupide. Mais moi, c'est autre chose: je suis sûr de gagner. 

—Sûr de gagner! répéta Beauchamp en regardant le comte d'un œil effaré. 

—Eh! certainement, dit Monte-Cristo en haussant légèrement les épaules. Sans cela je ne me battrais pas avec M. de Morcerf. Je le tuerai, il le faut, cela sera. Seulement, par un mot ce soir chez moi, indiquez-moi l'arme et l'heure; je n'aime pas à me faire attendre. 

—Au pistolet, à huit heures du matin au bois de Vincennes, dit Beauchamp, décontenancé ne sachant pas s'il avait affaire à un fanfaron outrecuidant ou à un être surnaturel. 

—C'est bien, monsieur, dit Monte-Cristo. Maintenant que tout est réglé, laissez-moi entendre le spectacle, je vous prie, et dites à votre ami Albert de ne pas revenir ce soir: il se ferait tort avec toutes ses brutalités de mauvais goût. Qu'il rentre et qu'il dorme.» 

Beauchamp sortit tout étonné. 

«Maintenant, dit Monte-Cristo en se retournant vers Morrel, je compte sur vous, n'est-ce pas? 

—Certainement, dit Morrel, et vous pouvez disposer de moi, comte; cependant\dots 

—Quoi? 

—Il serait important, comte, que je connusse la véritable cause\dots 

—C'est-à-dire, que vous me refusez? 

—Non pas. 

—La véritable cause, Morrel? dit le comte; ce jeune homme lui-même marche en aveugle et ne la connaît pas. La véritable cause, elle n'est connue que de moi et de Dieu; mais je vous donne ma parole d'honneur, Morrel, que Dieu, qui la connaît, sera pour nous. 

—Cela suffit, comte, dit Morrel. Quel est votre second témoin? 

—Je ne connais personne à Paris à qui je veuille faire cet honneur, que vous, Morrel, et votre beau-frère Emmanuel. Croyez-vous qu'Emmanuel veuille me rendre ce service. 

—Je vous réponds de lui, comme de moi, comte. 

—Bien! c'est tout ce qu'il me faut. Demain, à sept heures du matin chez moi, n'est-ce pas? 

—Nous y serons. 

—Chut! voici la toile qui se lève, écoutons. J'ai l'habitude de ne pas perdre une note de cet opéra; c'est une si adorable musique que celle de \textit{Guillaume Tell}!» 