\chapter{L'effraction}

\lettrine{L}{e} lendemain du jour où avait eu lieu la conversation que nous venons de rapporter, le comte de Monte-Cristo était en effet parti pour Auteuil avec Ali, plusieurs domestiques et des chevaux qu'il voulait essayer. Ce qui avait surtout déterminé ce départ, auquel il ne songeait même pas la veille, et auquel Andrea ne songeait pas plus que lui, c'était l'arrivée de Bertuccio, qui, revenu de Normandie, rapportait des nouvelles de la maison et de la corvette. La maison était prête, et la corvette, arrivée depuis huit jours et à l'ancre dans une petite anse où elle se tenait avec son équipage de six hommes, après avoir rempli toutes les formalités exigées, était déjà en état de reprendre la mer. 

Le comte loua le zèle de Bertuccio et l'invita à se préparer à un prompt départ, son séjour en France ne devant plus se prolonger au-delà d'un mois. 

«Maintenant, lui dit-il, je puis avoir besoin d'aller en une nuit de Paris au Tréport; je veux huit relais échelonnés sur la route qui me permettent de faire cinquante lieues en dix heures. 

—Votre Excellence avait déjà manifesté ce désir, répondit Bertuccio, et les chevaux sont prêts. Je les ai achetés et cantonnés moi-même aux endroits les plus commodes, c'est-à-dire dans des villages où personne ne s'arrête ordinairement. 

—C'est bien, dit Monte-Cristo, je reste ici un jour ou deux, arrangez-vous en conséquence.» 

Comme Bertuccio allait sortir pour ordonner tout ce qui avait rapport à ce séjour, Baptistin ouvrit la porte; il tenait une lettre sur un plateau de vermeil. 

«Que venez-vous faire ici? demanda le comte en le voyant tout couvert de poussière, je ne vous ai pas demandé, ce me semble?» 

Baptistin, sans répondre, s'approcha du comte et lui présenta la lettre. 

«Importante et pressée», dit-il. 

Le comte ouvrit la lettre et lut: 

«M. de Monte-Cristo est prévenu que cette nuit même un homme s'introduira dans sa maison des Champs-Élysées, pour soustraire des papiers qu'il croit enfermés dans le secrétaire du cabinet de toilette: on sait M. le comte de Monte-Cristo assez brave pour ne pas recourir à l'intervention de la police, intervention qui pourrait compromettre fortement celui qui donne cet avis. M. le comte, soit par une ouverture qui donnera de la chambre à coucher dans le cabinet, soit s'embusquant dans le cabinet, pourra se faire justice lui-même. Beaucoup de gens et de précautions apparentes éloigneraient certainement le malfaiteur, et feraient perdre à M. de Monte-Cristo cette occasion de connaître un ennemi que le hasard a fait découvrir à la personne qui donne cet avis au comte, avis qu'elle n'aurait peut-être pas l'occasion de renouveler si, cette première entreprise échouant, le malfaiteur en renouvelait une autre.» 

Le premier mouvement du comte fut de croire à une ruse de voleurs, piège grossier qui lui signalait un danger médiocre pour l'exposer à un danger plus grave. Il allait donc faire porter la lettre à un commissaire de police, malgré la recommandation, et peut-être même à cause de la recommandation de l'ami anonyme, quand tout à coup l'idée lui vint que ce pouvait être, en effet, quelque ennemi particulier à lui, que lui seul pouvait reconnaître et dont, le cas échéant, lui seul pouvait tirer parti, comme avait fait Fiesque du Maure qui avait voulu l'assassiner. On connaît le comte; nous n'avons donc pas besoin de dire que c'était un esprit plein d'audace et de vigueur qui se raidissait contre l'impossible avec cette énergie qui fait seule les hommes supérieurs. Par la vie qu'il avait menée, par la décision qu'il avait prise et qu'il avait tenue de ne reculer devant rien, le comte en était venu à savourer des jouissances inconnues dans les luttes qu'il entreprenait parfois contre la nature, qui est Dieu, et contre le monde qui peut bien passer pour le diable. 

«Ils ne veulent pas me voler mes papiers, dit Monte-Cristo, ils veulent me tuer; ce ne sont pas des voleurs, ce sont des assassins. Je ne veux pas que M. le préfet de Police se mêle de mes affaires particulières. Je suis assez riche, ma foi, pour dégrever en ceci le budget de son administration.» 

Le comte rappela Baptistin, qui était sorti de la chambre après avoir apporté la lettre. 

«Vous allez retourner à Paris, dit-il, vous ramènerez ici tous les domestiques qui restent. J'ai besoin de tout mon monde à Auteuil. 

—Mais ne restera-t-il donc personne à la maison, monsieur le comte? demanda Baptistin. 

—Si fait, le concierge. 

—Monsieur le comte réfléchira qu'il y a loin de la loge à la maison. 

—Eh bien? 

—Eh bien, on pourrait dévaliser tout le logis, sans qu'il entendît le moindre bruit. 

—Qui cela? 

—Mais des voleurs. 

—Vous êtes un niais, monsieur Baptistin; les voleurs dévalisassent-ils tout le logement, ne m'occasionneront jamais le désagrément que m'occasionnerait un service mal fait.» 

Baptistin s'inclina. 

«Vous m'entendez, dit le comte, ramenez vos camarades depuis le premier jusqu'au dernier; mais que tout reste dans l'état habituel; vous fermerez les volets du rez-de-chaussée, voilà tout. 

—Et ceux du premier? 

—Vous savez qu'on ne les ferme jamais. Allez.» 

Le comte fit dire qu'il dînerait seul chez lui et ne voulait être servi que par Ali. 

Il dîna avec sa tranquillité et sa sobriété habituelles, et après le dîner, faisant signe à Ali de le suivre, il sortit par la petite porte, gagna le bois de Boulogne comme s'il se promenait, prit sans affectation le chemin de Paris, et à la nuit tombante se trouva en face de la maison des Champs-Élysées. 

Tout était sombre, seule une faible lumière brillait dans la loge du concierge, distante d'une quarantaine de pas de la maison, comme l'avait dit Baptistin. 

Monte-Cristo s'adossa à un arbre, et, de cet œil qui se trompait si rarement, sonda la double allée, examina les passants, et plongea son regard dans les rues voisines, afin de voir si quelqu'un n'était point embusqué. Au bout de dix minutes, il fut convaincu que personne ne le guettait. Il courut aussitôt à la petite porte avec Ali, entra précipitamment, et, par l'escalier de service, dont il avait la clef, rentra dans sa chambre à coucher, sans ouvrir ou déranger un seul rideau, sans que le concierge lui-même pût se douter que la maison, qu'il croyait vide, avait retrouvé son principal habitant. 

Arrivé dans la chambre à coucher, le comte fit signe à Ali de s'arrêter, puis il passa dans le cabinet, qu'il examina; tout était dans l'état habituel: le précieux secrétaire à sa place, et la clef au secrétaire. Il le ferma à double tour, prit la clef, revint à la porte de la chambre à coucher, enleva la double gâche du verrou, et rentra. 

Pendant ce temps, Ali apportait sur une table les armes que le comte lui avait demandées, c'est-à-dire une carabine courte et une paire de pistolets doubles, dont les canons superposés permettaient de viser aussi sûrement qu'avec des pistolets de tir. Armé ainsi, le comte tenait la vie de cinq hommes entre ses mains. 

Il était neuf heures et demie à peu près; le comte et Ali mangèrent à la hâte un morceau de pain et burent un verre de vin d'Espagne; puis Monte-Cristo fit glisser un de ces panneaux mobiles qui lui permettaient de voir d'une pièce dans l'autre. Il avait à sa portée ses pistolets et sa carabine, et Ali, debout près de lui tenait à la main une de ces petites haches arabes qui n'ont pas changé de forme depuis les croisades. 

Par une des fenêtres de la chambre à coucher, parallèle à celle du cabinet, le comte pouvait voir dans la rue. 

Deux heures se passèrent ainsi; il faisait l'obscurité la plus profonde, et cependant Ali, grâce à sa nature sauvage, et cependant le comte, grâce sans doute à une qualité acquise, distinguaient dans cette nuit jusqu'aux plus faibles oscillations des arbres de la cour. 

Depuis longtemps la petite lumière de la loge du concierge s'était éteinte. 

Il était à présumer que l'attaque, si réellement il y avait une attaque projetée, aurait lieu par l'escalier du rez-de-chaussée et non par une fenêtre. Dans les idées de Monte-Cristo, les malfaiteurs en voulaient à sa vie et non à son argent. C'était donc à sa chambre à coucher qu'ils s'attaqueraient, et ils parviendraient à sa chambre à coucher soit par l'escalier dérobé, soi par la fenêtre du cabinet. 

Il plaça Ali devant la porte de l'escalier et continua de surveiller le cabinet. 

Onze heures trois quarts sonnèrent à l'horloge des Invalides; le vent d'ouest apportait sur ses humides bouffées la lugubre vibration des trois coups. 

Comme le dernier coup s'éteignait, le comte crut entendre un léger bruit du côté du cabinet; ce premier bruit, ou plutôt ce premier grincement, fut suivi d'un second, puis d'un troisième; au quatrième, le comte savait à quoi s'en tenir. Une main ferme et exercée était occupée à couper les quatre côtés d'une vitre avec un diamant. 

Le comte sentit battre plus rapidement son cœur. Si endurcis au danger que soient les hommes, si bien prévenus qu'ils soient du péril, ils comprennent toujours, au frémissement de leur cœur et au frissonnement de leur chair, la différence énorme qui existe entre le rêve et la réalité, entre le projet et l'exécution. 

Cependant Monte-Cristo ne fit qu'un signe pour prévenir Ali; celui-ci, comprenant que le danger était du côté du cabinet, fit un pas pour se rapprocher de son maître. 

Monte-Cristo était avide de savoir à quels ennemis et à combien d'ennemis il avait affaire. 

La fenêtre où l'on travaillait était en face de l'ouverture par laquelle le comte plongeait son regard dans le cabinet. Ses yeux se fixèrent donc vers cette fenêtre: il vit une ombre se dessiner plus épaisse sur l'obscurité; puis un des carreaux devint tout à fait opaque, comme si l'on y collait du dehors une feuille de papier, puis le carreau craqua sans tomber. Par l'ouverture pratiquée, un bras passa qui chercha l'espagnolette; une seconde après la fenêtre tourna sur ses gonds, et un homme entra.  

L'homme était seul. 

«Voilà un hardi coquin», murmura le comte. 

En ce moment il sentit qu'Ali lui touchait doucement l'épaule; il se retourna: Ali lui montrait la fenêtre de la chambre où ils étaient, et qui donnait sur la rue. 

Monte-Cristo fit trois pas vers cette fenêtre, il connaissait l'exquise délicatesse des sens du fidèle serviteur. En effet, il vit un autre homme qui se détachait d'une porte, et, montant sur une borne, semblait chercher à voir ce qui se passait chez le comte. 

«Bon! dit-il, ils sont deux: l'un agit, l'autre guette!» 

Il fit signe à Ali de ne pas perdre des yeux l'homme de la rue, et revint à celui du cabinet. 

Le coupeur de vitres était entré et s'orientait, les bras tendus en avant. 

Enfin il parut s'être rendu compte de toutes choses; il y avait deux portes dans le cabinet, il alla pousser les verrous de toutes deux. 

Lorsqu'il s'approcha de celle de la chambre à coucher, Monte-Cristo crut qu'il venait pour entrer, et prépara un de ses pistolets; mais il entendit simplement le bruit des verrous glissant dans leurs anneaux de cuivre. C'était une précaution, voilà tout; le nocturne visiteur, ignorant le soin qu'avait pris le comte d'enlever les gâches, pouvait désormais se croire chez lui et agir en toute tranquillité. 

Seul et libre de tous ses mouvements, l'homme alors tira de sa large poche quelque chose, que le comte ne put distinguer, posa ce quelque chose sur un guéridon, puis il alla droit au secrétaire, le palpa à l'endroit de la serrure, et s'aperçut que, contre son attente, la clef manquait. 

Mais le casseur de vitres était un homme de précaution et qui avait tout prévu; le comte entendit bientôt ce froissement du fer contre le fer que produit, quand on le remue, ce trousseau de clefs informes qu'apportent les serruriers quand on les envoie chercher pour ouvrir une porte, et auxquels les voleurs ont donné le nom de rossignols, sans doute à cause du plaisir qu'ils éprouvent à entendre leur chant nocturne, lorsqu'ils grincent contre le pêne de la serrure. 

«Ah! ah! murmura Monte-Cristo avec un sourire de désappointement, ce n'est qu'un voleur.» 

Mais l'homme, dans l'obscurité, ne pouvait choisir l'instrument convenable. Il eut alors recours à l'objet qu'il avait posé sur le guéridon; il fit jouer un ressort, et aussitôt une lumière pâle, mais assez vive cependant pour qu'on pût voir, envoya son reflet doré sur les mains et sur le visage de cet homme. 

«Tiens! fit tout à coup Monte-Cristo en se reculant avec un mouvement de surprise, c'est\dots.» 

Ali leva sa hache. 

«Ne bouge pas, lui dit Monte-Cristo tout bas, et laisse là ta hache, nous n'avons plus besoin d'armes ici.» 

Puis il ajouta quelques mots en baissant encore la voix, car l'exclamation, si faible qu'elle fût, que la surprise avait arrachée au comte, avait suffi pour faire tressaillir l'homme, qui était resté dans la pose du rémouleur antique. C'était un ordre que venait de donner le comte, car aussitôt Ali s'éloigna sur la pointe du pied, détacha de la muraille de l'alcôve un vêtement noir et un chapeau triangulaire. Pendant ce temps, Monte-Cristo ôtait rapidement sa redingote, son gilet et sa chemise, et l'on pouvait, grâce au rayon de lumière filtrant par la fente du panneau, reconnaître sur la poitrine du comte une de ces souples et fines tuniques de mailles d'acier, dont la dernière, dans cette France où l'on ne craint plus les poignards, fut peut-être portée par le roi Louis XVI, qui craignait le couteau pour sa poitrine, et qui fut frappé d'une hache à la tête. 

Cette tunique disparut bientôt sous une longue soutane comme les cheveux du comte sous une perruque à tonsure; le chapeau triangulaire, placé sur la perruque, acheva de changer le comte en abbé. 

Cependant l'homme n'entendant plus rien, s'était relevé, et pendant le temps que Monte-Cristo opérait sa métamorphose, était allé droit au secrétaire, dont la serrure commençait à craquer sous son \textit{rossignol}. 

«Bon! murmura le comte, lequel se reposait sans doute sur quelque secret de serrurerie qui devait être inconnu au crocheteur de portes, si habile qu'il fût bon! tu en as pour quelques minutes.» Et il alla à la fenêtre. 

L'homme qu'il avait vu monter sur une borne en était descendu, et se promenait toujours dans la rue; mais, chose singulière, au lieu de s'inquiéter de ceux qui pouvaient venir, soit par l'avenue des Champs-Élysées, soit par le faubourg Saint-Honoré, il ne paraissait préoccupé que de ce qui se passait chez le comte, et tous ses mouvements avaient pour but de voir ce qui se passait dans le cabinet. 

Monte-Cristo, tout à coup, se frappa le front et laissa errer sur ses lèvres entrouvertes un rire silencieux. 

Puis se rapprochant d'Ali: 

«Demeure ici, lui dit-il tout bas, caché dans l'obscurité, et quel que soit le bruit que tu entendes, quelque chose qui se passe, n'entre et ne te montre que si je t'appelle par ton nom.» 

Ali fit signe de la tête qu'il avait compris et qu'il obéirait. 

Alors Monte-Cristo tira d'une armoire une bougie tout allumée, et au moment où le voleur était le plus occupé à sa serrure, il ouvrit doucement la porte ayant soin que la lumière qu'il tenait à la main donnât tout entière sur son visage. 

La porte tourna si doucement que le voleur n'entendit pas le bruit. Mais, à son grand étonnement, il vit tout à coup la chambre s'éclairer. 

Il se retourna. 

«Eh! bonsoir, cher monsieur Caderousse, dit Monte-Cristo; que diable venez-vous donc faire ici à une pareille heure! 

—L'abbé Busoni!» s'écria Caderousse. 

Et ne sachant comment cette étrange apparition était venue jusqu'à lui, puisqu'il avait fermé les portes, il laissa tomber son trousseau de fausses clefs, et resta immobile et comme frappé de stupeur. 

Le comte alla se placer entre Caderousse et la fenêtre, coupant ainsi au voleur terrifié son seul moyen de retraite. 

«L'abbé Busoni! répéta Caderousse en fixant sur le comte des yeux hagards. 

—Eh bien, sans doute, l'abbé Busoni, reprit Monte-Cristo, lui-même en personne, et je suis bien aise que vous me reconnaissiez, mon cher monsieur Caderousse, cela prouve que nous avons bonne mémoire, car, si je ne me trompe, voilà tantôt dix ans que nous ne nous sommes vus.» 

Ce calme, cette ironie, cette puissance, frappèrent l'esprit de Caderousse d'une terreur vertigineuse. 

«L'abbé! l'abbé! murmura-t-il en crispant ses poings et en faisant claquer ses dents. 

—Nous voulons donc voler le comte de Monte-Cristo? continua le prétendu abbé. 

—Monsieur l'abbé, murmura Caderousse cherchant à gagner la fenêtre que lui interceptait impitoyablement le comte, monsieur l'abbé, je ne sais\dots je vous prie de croire\dots je vous jure\dots. 

—Un carreau coupé, continua le comte, une lanterne sourde, un trousseau de rossignols, un secrétaire à demi forcé, c'est clair cependant.» 

Caderousse s'étranglait avec sa cravate, il cherchait un angle où se cacher, un trou par où disparaître. 

«Allons, dit le comte, je vois que vous êtes toujours le même, monsieur l'assassin. 

—Monsieur l'abbé, puisque vous savez tout, vous savez que ce n'est pas moi, que c'est la Carconte; ç'a été reconnu au procès, puisqu'ils ne m'ont condamné qu'aux galères. 

—Vous avez donc fini votre temps, que je vous retrouve en train de vous y faire ramener? 

—Non, monsieur l'abbé, j'ai été délivré par quelqu'un. 

—Ce quelqu'un-là a rendu un charmant service à la société. 

—Ah! dit Caderousse, j'avais cependant bien promis\dots. 

—Ainsi, vous êtes en rupture de ban? interrompit Monte-Cristo. 

—Hélas! oui, fit Caderousse, très inquiet. 

—Mauvaise récidive\dots. Cela vous conduira, si je ne me trompe, à la place de Grève. Tant pis, tant pis, diavolo! comme disent les mondains de mon pays. 

—Monsieur l'abbé, je cède à un entraînement\dots. 

—Tous les criminels disent cela. 

—Le besoin\dots. 

—Laissez donc, dit dédaigneusement Busoni, le besoin peut conduire à demander l'aumône, à voler un pain à la porte d'un boulanger, mais non à venir forcer un secrétaire dans une maison que l'on croit inhabitée. Et lorsque le bijoutier Joannès venait de vous compter quarante-cinq mille francs en échange du diamant que je vous avais donné, et que vous l'avez tué pour avoir le diamant et l'argent, était-ce aussi le besoin? 

—Pardon, monsieur l'abbé, dit Caderousse; vous m'avez déjà sauvé une fois, sauvez-moi encore une seconde. 

—Cela ne m'encourage pas. 

—Êtes-vous seul, monsieur l'abbé? demanda Caderousse en joignant les mains, ou bien avez-vous là des gendarmes tout prêts à me prendre? 

—Je suis tout seul, dit l'abbé, et j'aurai encore pitié de vous et je vous laisserai aller au risque des nouveaux malheurs que peut amener ma faiblesse, si vous me dites toute la vérité. 

—Ah! monsieur l'abbé! s'écria Caderousse en joignant les mains et en se rapprochant d'un pas de Monte-Cristo, je puis bien vous dire que vous êtes mon sauveur, vous! 

—Vous prétendez qu'on vous a délivré du bagne? 

—Oh! ça, foi de Caderousse, monsieur l'abbé! 

—Qui cela? 

—Un Anglais. 

—Comment s'appelait-il? 

—Lord Wilmore. 

—Je le connais; je saurai donc si vous mentez. 

—Monsieur l'abbé, je dis la vérité pure. 

—Cet Anglais vous protégeait donc? 

—Non pas moi, mais un jeune Corse qui était mon compagnon de chaîne. 

—Comment se nommait ce jeune Corse? 

—Benedetto. 

—C'est un nom de baptême. 

—Il n'en avait pas d'autre, c'était un enfant trouvé. 

—Alors ce jeune homme s'est évadé avec vous? 

—Oui. 

—Comment cela? 

—Nous travaillions à Saint-Mandrier, près de Toulon. Connaissez-vous Saint-Mandrier? 

—Je le connais. 

—Eh bien, pendant qu'on dormait, de midi à une heure\dots. 

—Des forçats qui font la sieste! Plaignez donc ces gaillards-là, dit l'abbé.  

—Dame! fit Caderousse, on ne peut pas toujours travailler, on n'est pas des chiens. 

—Heureusement pour les chiens, dit Monte-Cristo. 

—Pendant que les autres faisaient donc la sieste, nous nous sommes éloignés un petit peu, nous avons scié nos fers avec une lime que nous avait fait parvenir l'Anglais, et nous nous sommes sauvés à la nage. 

—Et qu'est devenu ce Benedetto? 

—Je n'en sais rien. 

—Vous devez le savoir cependant. 

—Non, en vérité. Nous nous sommes séparés à Hyères.» 

Et, pour donner plus de poids à sa protestation, Caderousse fit encore un pas vers l'abbé qui demeura immobile à sa place, toujours calme et interrogateur. 

«Vous mentez! dit l'abbé Busoni, avec un accent d'irrésistible autorité. 

—Monsieur l'abbé!\dots 

—Vous mentez! cet homme est encore votre ami, et vous vous servez de lui comme d'un complice peut-être? 

—Oh! monsieur l'abbé!\dots 

—Depuis que vous avez quitté Toulon, comment avez-vous vécu? Répondez. 

—Comme j'ai pu. 

—Vous mentez!» reprit une troisième fois l'abbé avec un accent plus impératif encore. 

Caderousse terrifié, regarda le comte. 

«Vous avez vécu, reprit celui-ci, de l'argent qu'il vous a donné. 

—Eh bien, c'est vrai, dit Caderousse; Benedetto est devenu un fils de grand seigneur. 

—Comment peut-il être fils de grand seigneur? 

—Fils naturel. 

—Et comment nommez-vous ce grand seigneur? 

—Le comte de Monte-Cristo, celui-là même chez qui nous sommes. 

—Benedetto le fils du comte? reprit Monte-Cristo étonné à son tour. 

—Dame! il faut bien croire, puisque le comte lui a trouvé un faux père, puisque le comte lui fait quatre mille francs par mois, puisque le comte lui laisse cinq cent mille francs par son testament. 

—Ah! ah! dit le faux abbé, qui commençait à comprendre; et quel nom porte, en attendant, ce jeune homme? 

—Il s'appelle Andrea Cavalcanti. 

—Alors c'est ce jeune homme que mon ami le comte de Monte-Cristo reçoit chez lui, et qui va épouser Mlle Danglars? 

—Justement. 

—Et vous souffrez cela, misérable! vous qui connaissez sa vie et sa flétrissure? 

—Pourquoi voulez-vous que j'empêche un camarade de réussir? dit Caderousse. 

—C'est juste, ce n'est pas à vous de prévenir M. Danglars, c'est à moi. 

—Ne faites pas cela, monsieur l'abbé!\dots 

—Et pourquoi? 

—Parce que c'est notre pain que vous nous feriez perdre. 

—Et vous croyez que, pour conserver le pain à des misérables comme vous, je me ferai le fauteur de leur ruse, le complice de leurs crimes? 

—Monsieur l'abbé! dit Caderousse en se rapprochant encore. 

—Je dirai tout. 

—À qui? 

—À M. Danglars. 

—Tron de l'air! s'écria Caderousse en tirant un couteau tout ouvert de son gilet, et en frappant le comte au milieu de la poitrine, tu ne diras rien, l'abbé!» 

Au grand étonnement de Caderousse, le poignard, au lieu de pénétrer dans la poitrine du comte, rebroussa émoussé. 

En même temps le comte saisit de la main gauche le poignet de l'assassin, et le tordit avec une telle force que le couteau tomba de ses doigts raidis et que Caderousse poussa un cri de douleur. 

Mais le comte, sans s'arrêter à ce cri, continua de tordre le poignet du bandit jusqu'à ce que, le bras disloqué, il tombât d'abord à genoux, puis ensuite la face contre terre. 

Le comte appuya son pied sur sa tête et dit: 

«Je ne sais qui me retient de te briser le crâne, scélérat! 

—Ah! grâce! grâce!» cria Caderousse. 

Le comte retira son pied. 

«Relève-toi!» dit-il. 

Caderousse se releva. 

«Tudieu! quel poignet vous avez, monsieur l'abbé! dit Caderousse, caressant son bras tout meurtri par les tenailles de chair qui l'avaient étreint; tudieu! quel poignet! 

—Silence. Dieu me donne la force de dompter une bête féroce comme toi; c'est au nom de ce Dieu que j'agis; souviens-toi de cela, misérable, et t'épargner en ce moment, c'est encore servir les desseins de Dieu. 

—Ouf! fit Caderousse, tout endolori. 

—Prends cette plume et ce papier, et écris ce que je vais te dicter. 

—Je ne sais pas écrire, monsieur l'abbé. 

—Tu mens, prends cette plume et écris!» 

Caderousse, subjugué par cette puissance supérieure, s'assit et écrivit: 

«Monsieur, l'homme que vous recevez chez vous et à qui vous destinez votre fille est un ancien forçat échappé avec moi du bagne de Toulon; il portait le n°59 et moi le n°58. 

«Il se nommait Benedetto; mais il ignore lui-même son véritable nom, n'ayant jamais connu ses parents. 

«Signe! continua le comte. 

—Mais vous voulez donc me perdre? 

—Si je voulais te perdre, imbécile, je te traînerais jusqu'au premier corps de garde; d'ailleurs, à l'heure où le billet sera rendu à son adresse, il est probable que tu n'auras plus rien à craindre; signe donc.» 

Caderousse signa. 

«L'adresse: \textit{À monsieur le baron Danglars, banquier, rue de la Chaussée-d'Antin}.» 

Caderousse écrivit l'adresse. 

L'abbé prit le billet. 

«Maintenant, dit-il, c'est bien, va-t'en. 

—Par où? 

—Par où tu es venu. 

—Vous voulez que je sorte par cette fenêtre? 

—Tu y es bien entré. 

—Vous méditez quelque chose contre moi, monsieur l'abbé? 

—Imbécile, que veux-tu que je médite? 

—Pourquoi ne pas m'ouvrir la porte? 

—À quoi bon réveiller le concierge? 

—Monsieur l'abbé, dites-moi que vous ne voulez pas ma mort. 

—Je veux ce que Dieu veut. 

—Mais jurez-moi que vous ne me frapperez pas tandis que je descendrai. 

—Sot et lâche que tu es!  

—Que voulez-vous faire de moi? 

—Je te le demande. J'ai essayé d'en faire un homme heureux, et je n'en ai fait qu'un assassin! 

—Monsieur l'abbé, dit Caderousse, tentez une dernière épreuve. 

—Soit, dit le comte. Écoute, tu sais que je suis un homme de parole? 

—Oui, dit Caderousse. 

—Si tu rentres chez toi sain et sauf\dots. 

—À moins que ce ne soit de vous, qu'ai-je à craindre? 

—Si tu rentres chez toi sain et sauf, quitte Paris, quitte la France, et partout où tu seras, tant que tu te conduiras honnêtement, je te ferai passer une petite pension; car si tu rentres chez toi sain et sauf, eh bien\dots. 

—Eh bien? demanda Caderousse en frémissant. 

—Eh bien, je croirai que Dieu t'a pardonné, et je te pardonnerai aussi. 

—Vrai comme je suis chrétien, balbutia Caderousse en reculant, vous me faites mourir de peur! 

—Allons, va-t'en!» dit le comte en montrant du doigt la fenêtre à Caderousse. 

Caderousse, encore mal rassuré par cette promesse, enjamba la fenêtre et mit le pied sur l'échelle. 

Là, il s'arrêta tremblant. 

«Maintenant descends», dit l'abbé en se croisant les bras. 

Caderousse commença de comprendre qu'il n'y avait rien à craindre de ce côté, et descendit. 

Alors le comte s'approcha avec la bougie, de sorte qu'on pût distinguer des Champs-Élysées cet homme qui descendait d'une fenêtre, éclairé par un autre homme. 

—Que faites-vous donc, monsieur l'abbé? dit Caderousse; s'il passait une patrouille\dots.» 

Et il souffla la bougie. Puis il continua de descendre; mais ce ne fut que lorsqu'il sentit le sol du jardin sous son pied qu'il fut suffisamment rassuré. 

Monte-Cristo rentra dans sa chambre à coucher, et jetant un coup d'œil rapide du jardin à la rue, il vit d'abord Caderousse qui, après être descendu, faisait un détour dans le jardin et allait planter son échelle à l'extrémité de la muraille, afin de sortir à une autre place que celle par laquelle il était entré. 

Puis, passant du jardin à la rue, il vit l'homme qui semblait attendre courir parallèlement dans la rue et se placer derrière l'angle même près duquel Caderousse allait descendre. 

Caderousse monta lentement sur l'échelle, et, arrivé aux derniers échelons, passa sa tête par-dessus le chaperon pour s'assurer que la rue était bien solitaire. 

On ne voyait personne, on n'entendait aucun bruit. 

Une heure sonna aux Invalides. 

Alors Caderousse se mit à cheval sur le perron, et, tirant à lui son échelle, la passa par-dessus le mur, puis il se mit en devoir de descendre, ou plutôt de se laisser glisser le long des deux montants, manœuvre qu'il opéra avec une adresse qui prouva l'habitude qu'il avait de cet exercice. 

Mais, une fois lancé sur la pente, il ne put s'arrêter. Vainement il vit un homme s'élancer dans l'ombre au moment où il était à moitié chemin; vainement il vit un bras se lever au moment où il touchait la terre; avant qu'il eût pu se mettre en défense, ce bras le frappa si furieusement dans le dos, qu'il lâcha l'échelle en criant: 

«Au secours!» 

Un second coup lui arriva presque aussitôt dans le flanc, et il tomba en criant: 

«Au meurtre!» 

Enfin, comme il se roulait sur la terre, son adversaire le saisit aux cheveux et lui porta un troisième coup dans la poitrine. 

Cette fois Caderousse voulut crier encore, mais il ne put pousser qu'un gémissement, et laissa couler en gémissant les trois ruisseaux de sang qui sortaient de ses trois blessures. 

L'assassin, voyant qu'il ne criait plus, lui souleva la tête par les cheveux; Caderousse avait les yeux fermés et la bouche tordue. L'assassin le crut mort, laissa retomber la tête et disparut. 

Alors Caderousse, le sentant s'éloigner, se redressa sur son coude, et, d'une voix mourante, cria dans un suprême effort: 

«À l'assassin! je meurs! à moi, monsieur l'abbé, à moi!» 

Ce lugubre appel perça l'ombre de la nuit. La porte de l'escalier dérobé s'ouvrit, puis la petite porte du jardin, et Ali et son maître accoururent avec des lumières. 