\chapter{Le contrat}

\lettrine{T}{rois} jours après la scène que nous venons de raconter, c'est-à-dire vers les cinq heures de l'après-midi du jour fixé pour la signature du contrat de Mlle Eugénie Danglars et d'Andrea Cavalcanti, que le banquier s'était obstiné à maintenir prince, comme une brise fraîche faisait frissonner toutes les feuilles du petit jardin situé en avant de la maison du comte de Monte-Cristo, au moment où celui-ci se préparait à sortir, et tandis que ses chevaux l'attendaient en frappant du pied, maintenus par la main du cocher assis déjà depuis un quart d'heure sur le siège, l'élégant phaéton avec lequel nous avons déjà plusieurs fois fait connaissance, et notamment pendant la soirée d'Auteuil, vint tourner rapidement l'angle de la porte d'entrée, et lança plutôt qu'il ne déposa sur les degrés du perron M. Andrea Cavalcanti, aussi doré, aussi rayonnant que si lui, de son côté, eût été sur le point d'épouser une princesse. 

Il s'informa de la santé du comte avec cette familiarité qui lui était habituelle, et, escaladant légèrement le premier étage, le rencontra lui-même au haut de l'escalier. 

À la vue du jeune homme, le comte s'arrêta. Quant à Andrea Cavalcanti, il était lancé, et quand il était lancé, rien ne l'arrêtait. 

«Eh! bonjour, cher monsieur de Monte-Cristo, dit-il au comte. 

—Ah! monsieur Andrea! fit celui-ci avec sa voix demi-railleuse, comment vous portez-vous? 

—À merveille, comme vous voyez. Je viens causer avec vous de mille choses; mais d'abord sortiez-vous ou rentriez-vous? 

—Je sortais, monsieur. 

—Alors, pour ne point vous retarder, je monterai, si vous le voulez bien, dans votre calèche, et Tom nous suivra, conduisant mon phaéton à la remorque. 

—Non, dit avec un imperceptible sourire de mépris le comte, qui ne se souciait pas d'être vu en compagnie du jeune homme; non, je préfère vous donner audience ici, cher monsieur Andrea; on cause mieux dans une chambre, et l'on n'a pas de cocher qui surprenne vos paroles au vol.» 

Le comte rentra donc dans un petit salon faisant partie du premier étage, s'assit, et fit, en croisant ses jambes l'une sur l'autre, signe au jeune homme de s'asseoir à son tour. 

Andrea prit son air le plus riant. 

«Vous savez, cher comte, dit-il, que la cérémonie a lieu ce soir; à neuf heures on signe le contrat chez le beau-père. 

—Ah! vraiment? dit Monte-Cristo. 

—Comment! est-ce une nouvelle que je vous apprends? et n'étiez-vous pas prévenu de cette solennité par M. Danglars? 

—Si fait, dit le comte, j'ai reçu une lettre de lui hier; mais je ne crois pas que l'heure y fût indiquée. 

—C'est possible; le beau-père aura compté sur la notoriété publique. 

—Eh bien, dit Monte-Cristo, vous voilà heureux monsieur Cavalcanti; c'est une alliance des plus sortables que vous contractez là; et puis, Mlle Danglars est jolie. 

—Mais oui, répondit Cavalcanti avec un accent plein de modestie. 

—Elle est surtout fort riche, à ce que je crois du moins, dit Monte-Cristo. 

—Fort riche, vous croyez? répéta le jeune homme. 

—Sans doute; on dit que M. Danglars cache pour le moins la moitié de sa fortune. 

—Et il avoue quinze ou vingt millions, dit Andrea avec un regard étincelant de joie. 

—Sans compter, ajouta Monte-Cristo, qu'il est à la veille d'entrer dans un genre de spéculation déjà un peu usé aux États-Unis et en Angleterre, mais tout à fait neuf en France. 

—Oui, oui, je sais ce dont vous voulez parler: le chemin de fer dont il vient d'obtenir l'adjudication, n'est-ce pas? 

—Justement! il gagnera au moins, c'est l'avis général, au moins dix millions dans cette affaire. 

—Dix millions! vous croyez? c'est magnifique, dit Cavalcanti, qui se grisait à ce bruit métallique de paroles dorées. 

—Sans compter, reprit Monte-Cristo, que toute cette fortune vous reviendra, et que c'est justice, puisque Mlle Danglars est fille unique. D'ailleurs, votre fortune à vous, votre père me l'a dit du moins, est presque égale à celle de votre fiancée. Mais laissons là un peu les affaires d'argent. Savez-vous, monsieur Andrea, que vous avez un peu lestement et habilement mené toute cette affaire! 

—Mais pas mal, pas mal, dit le jeune homme; j'étais né pour être diplomate. 

—Eh bien, on vous fera entrer dans la diplomatie; la diplomatie, vous le savez, ne s'apprend pas; c'est une chose d'instinct\dots Le cœur est donc pris? 

—En vérité, j'en ai peur, répondit Andrea du ton dont il avait vu au Théâtre-Français Dorante ou Valère répondre à Alceste. 

—Vous aime-t-on un peu? 

—Il le faut bien, dit Andrea avec un sourire vainqueur, puisqu'on m'épouse. Mais cependant, n'oublions pas un grand point. 

—Lequel? 

—C'est que j'ai été singulièrement aidé dans tout ceci. 

—Bah! 

—Certainement. 

—Par les circonstances? 

—Non, par vous. 

—Par moi? Laissez donc, prince, dit Monte-Cristo en appuyant avec affectation sur le titre. Qu'ai-je pu faire pour vous? Est-ce que votre nom, votre position sociale et votre mérite ne suffisaient point? 

—Non, dit Andrea, non; et vous avez beau dire, monsieur le comte, je maintiens, moi, que la position d'un homme tel que vous a plus fait que mon nom, ma position sociale et mon mérite. 

—Vous vous abusez complètement, monsieur, dit Monte-Cristo, qui sentit l'adresse perfide du jeune homme, et qui comprit la portée de ses paroles; ma protection ne vous a été acquise qu'après connaissance prise de l'influence et de la fortune de monsieur votre père; car enfin qui m'a procuré, à moi qui ne vous avais jamais vu, ni vous, ni l'illustre auteur de vos jours, le bonheur de votre connaissance? Ce sont deux de mes bons amis, Lord Wilmore et l'abbé Busoni. Qui m'a encouragé, non pas à vous servir de garantie, mais à vous patronner? C'est le nom de votre père, si connu et si honoré en Italie; personnellement, moi, je ne vous connais pas.» 

Ce calme, cette parfaite aisance firent comprendre à Andrea qu'il était pour le moment étreint par une main plus musculeuse que la sienne, et que l'étreinte n'en pouvait être facilement brisée. 

«Ah çà! mais, dit-il, mon père a donc vraiment une bien grande fortune, monsieur le comte? 

—Il paraît que oui, monsieur, répondit Monte-Cristo. 

—Savez-vous si la dot qu'il m'a promise est arrivée? 

—J'en ai reçu la lettre d'avis. 

—Mais les trois millions? 

—Les trois millions sont en route, selon toute probabilité. 

—Je les toucherai donc réellement? 

—Mais dame! reprit le comte, il me semble que jusqu'à présent, monsieur, l'argent ne vous a pas fait faute!» 

Andrea fut tellement surpris, qu'il ne put s'empêcher de rêver un moment. 

«Alors, dit-il en sortant de sa rêverie, il me reste, monsieur, à vous adresser une demande, et celle-là vous la comprendrez, même quand elle devrait vous être désagréable. 

—Parlez, dit Monte-Cristo. 

—Je me suis mis en relation, grâce à ma fortune, avec beaucoup de gens distingués, et j'ai même, pour le moment du moins, une foule d'amis. Mais en me mariant comme je le fais, en face de toute la société parisienne, je dois être soutenu par un nom illustre, et à défaut de la main paternelle, c'est une main puissante qui doit me conduire à l'autel; or, mon père ne vient point à Paris, n'est-ce pas? 

—Il est vieux, couvert de blessures, et il souffre, dit-il, à en mourir, chaque fois qu'il voyage. 

—Je comprends. Eh bien, je viens vous faire une demande. 

—À moi? 

—Oui, à vous. 

—Et laquelle? mon Dieu! 

—Eh bien, c'est de le remplacer. 

—Ah! mon cher monsieur! quoi! après les nombreuses relations que j'ai eu le bonheur d'avoir avec vous, vous me connaissez si mal que de me faire une pareille demande? 

«Demandez-moi un demi-million à emprunter, et, quoiqu'un pareil prêt soit assez rare, parole d'honneur! vous me serez moins gênant. Sachez donc, je croyais vous l'avoir déjà dit, que dans sa participation, morale surtout, aux choses de ce monde, jamais le comte de Monte-Cristo n'a cessé d'apporter les scrupules, je dirai plus, les superstitions d'un homme de l'Orient. 

«Moi qui ai un sérail au Caire, un à Smyrne et un à Constantinople, présider à un mariage! jamais. 

—Ainsi, vous me refusez? 

—Net; et fussiez-vous mon fils, fussiez-vous mon frère, je vous refuserais de même. 

—Ah! par exemple! s'écria Andrea désappointé, mais comment faire alors? 

—Vous avez cent amis, vous l'avez dit vous-même. 

—D'accord, mais c'est vous qui m'avez présenté chez M. Danglars. 

—Point! Rétablissons les faits dans toute la vérité: c'est moi qui vous ai fait dîner avec lui à Auteuil, et c'est vous qui vous êtes présenté vous-même; diable! c'est tout différent. 

—Oui, mais mon mariage: vous avez aidé\dots 

—Moi! en aucune chose, je vous prie de le croire; mais rappelez-vous donc ce que je vous ai répondu quand vous êtes venu me prier de faire la demande: Oh! je ne fais jamais de mariage, moi, mon cher prince, c'est un principe arrêté chez moi.» 

Andrea se mordit les lèvres. 

«Mais enfin, dit-il, vous serez là au moins? 

—Tout Paris y sera? 

—Oh! certainement. 

—Eh bien, j'y serai comme tout Paris, dit le comte. 

—Vous signerez au contrat? 

—Oh! je n'y vois aucun inconvénient, et mes scrupules ne vont point jusque-là. 

—Enfin, puisque vous ne voulez pas m'accorder davantage, je dois me contenter de ce que vous me donnez. Mais un dernier mot, comte. 

—Comment donc? 

—Un conseil. 

—Prenez garde; un conseil, c'est pis qu'un service. 

—Oh! celui-ci, vous pouvez me le donner sans vous compromettre. 

—Dites. 

—La dot de ma femme est de cinq cent mille livres. 

—C'est le chiffre que M. Danglars m'a annoncé à moi-même. 

—Faut-il que je la reçoive ou que je la laisse aux mains du notaire? 

—Voici, en général, comment les choses se passent quand on veut qu'elles se passent galamment: vos deux notaires prennent rendez-vous au contrat pour le lendemain ou le surlendemain; le lendemain ou le surlendemain, ils échangent les deux dots, dont ils se donnent mutuellement reçu, puis, le mariage célébré, ils mettent les millions à votre disposition, comme chef de la communauté. 

—C'est que, dit Andrea avec une certaine inquiétude mal dissimulée, je croyais avoir entendu dire à mon beau-père qu'il avait l'intention de placer nos fonds dans cette fameuse affaire de chemin de fer dont vous me parliez tout à l'heure. 

—Eh bien, mais, reprit Monte-Cristo, c'est, à ce que tout le monde assure, un moyen que vos capitaux soient triplés dans l'année. M. le baron Danglars est bon père et sait compter. 

—Allons donc, dit Andrea, tout va bien, sans votre refus, toutefois, qui me perce le cœur. 

—Ne l'attribuez qu'à des scrupules fort naturels en pareille circonstance. 

—Allons, dit Andrea, qu'il soit donc fait comme vous le voulez; à ce soir, neuf heures. 

—À ce soir.» 

Et malgré une légère résistance de Monte-Cristo, dont les lèvres pâlirent, mais qui cependant conserva son sourire de cérémonie, Andrea saisit la main du comte, la serra, sauta dans son phaéton et disparut. 

Les quatre ou cinq heures qui lui restaient jusqu'à neuf heures, Andrea les employa en courses, en visites destinées à intéresser ces amis dont il avait parlé, à paraître chez le banquier avec tout le luxe de leurs équipages, les éblouissant par ces promesses d'actions qui, depuis, ont fait tourner toutes les têtes, et dont Danglars, en ce moment-là, avait l'initiative. 

En effet, à huit heures et demie du soir, le grand salon de Danglars, la galerie attenante à ce salon et les trois autres salons de l'étage étaient pleins d'une foule parfumée qu'attirait fort peu la sympathie, mais beaucoup cet irrésistible besoin d'être là où l'on sait qu'il y a du nouveau. 

Un académicien dirait que les soirées du monde sont des collections de fleurs qui attirent papillons inconstants, abeilles affamées et frelons bourdonnants. 

Il va sans dire que les salons étaient resplendissants de bougies, la lumière roulait à flots des moulures d'or sur les tentures de soie, et tout le mauvais goût de cet ameublement, qui n'avait pour lui que la richesse, resplendissait de tout son éclat. 

Mlle Eugénie était vêtue avec la simplicité la plus élégante: une robe de soie blanche brochée de blanc, une rose blanche à moitié perdue dans ses cheveux d'un noir de jais, composaient toute sa parure que ne venait pas enrichir le plus petit bijou. 

Seulement on pouvait lire que dans ses yeux cette assurance parfaite destinée à démentir ce que cette candide toilette avait de vulgairement virginal à ses propres yeux. 

Mme Danglars, à trente pas d'elle, causait avec Debray, Beauchamp et Château-Renaud. Debray avait fait sa rentrée dans cette maison pour cette grande solennité, mais comme tout le monde et sans aucun privilège particulier. 

M. Danglars, entouré de députés, d'hommes de finance, expliquait une théorie de contributions nouvelles qu'il comptait mettre en exercice quand la force des choses aurait contraint le gouvernement à l'appeler au ministère. 

Andrea, tenant sous son bras un des plus fringants dandys de l'Opéra, lui expliquait assez impertinemment, attendu qu'il avait besoin d'être hardi pour paraître à l'aise, ses projets de vie à venir, et les progrès de luxe qu'il comptait faire faire avec ses cent soixante-quinze mille livres de rente à la fashion parisienne. 

La foule générale roulait dans ces salons comme un flux et un reflux de turquoises, de rubis, d'émeraudes, d'opales et de diamants. 

Comme partout, on remarquait que c'étaient les plus vieilles femmes qui étaient les plus parées, et les plus laides qui se montraient avec le plus d'obstination. 

S'il y avait quelque beau lis blanc, quelque rose suave et parfumée, il fallait la chercher et la découvrir cachée dans quelque coin par une mère à turban, ou par une tante à oiseau de paradis. 

À chaque instant, au milieu de cette cohue, de ce bourdonnement, de ces rires, la voix des huissiers lançait un nom connu dans les finances, respecté dans l'armée ou illustre dans les lettres; alors un faible mouvement des groupes accueillait ce nom. 

Mais pour un qui avait le privilège de faire frémir cet océan de vagues humaines, combien passaient accueillis par l'indifférence ou le ricanement du dédain! 

Au moment où l'aiguille de la pendule massive, de la pendule représentant Endymion endormi, marquait neuf heures sur un cadran d'or, et où le timbre, fidèle reproducteur de la pensée machinale, retentissait neuf fois, le nom du comte de Monte-Cristo retentit à son tour, et, comme poussée par la flamme électrique, toute l'assemblée se tourna vers la porte. 

Le comte était vêtu de noir et avec sa simplicité habituelle; son gilet blanc dessinait sa vaste et noble poitrine; son col noir paraissait d'une fraîcheur singulière, tant il ressortait sur la mâle pâleur de son teint; pour tout bijou, il portait une chaîne de gilet si fine qu'à peine le mince filet d'or tranchait sur le piqué blanc. 

Il se fit à l'instant même un cercle autour de la porte. 

Le comte, d'un seul coup d'œil, aperçut Mme Danglars à un bout du salon, M. Danglars à l'autre, et Mlle Eugénie devant lui. 

Il s'approcha d'abord de la baronne, qui causait avec Mme de Villefort, qui était venue seule, Valentine étant toujours souffrante; et sans dévier, tant le chemin se frayait devant lui, il passa de la baronne à Eugénie, qu'il complimenta en termes si rapides et si réservés, que la fière artiste en fut frappée. 

Près d'elle était Mlle Louise d'Armilly, qui remercia le comte des lettres de recommandation qu'il lui avait si gracieusement données pour l'Italie, et dont elle comptait, lui dit-elle, faire incessamment usage. 

En quittant ces dames, il se retourna et se trouva près de Danglars, qui s'était approché pour lui donner la main. 

Ces trois devoirs sociaux accomplis, Monte-Cristo s'arrêta, promenant autour de lui ce regard assuré empreint de cette expression particulière aux gens d'un certain monde et surtout d'une certaine portée, regard qui semble dire: 

«J'ai fait ce que j'ai dû; maintenant que les autres fassent ce qu'ils me doivent.» 

Andrea, qui était dans un salon contigu, sentit cette espèce de frémissement que Monte-Cristo avait imprimé à la foule, et il accourut saluer le comte. 

Il le trouva complètement entouré; on se disputait ses paroles, comme il arrive toujours pour les gens qui parlent peu et qui ne disent jamais un mot sans valeur. 

Les notaires firent leur entrée en ce moment, et vinrent installer leurs pancartes griffonnées sur le velours brodé d'or qui couvrait la table préparée pour la signature, table en bois doré. 

Un des notaires s'assit, l'autre resta debout. 

On allait procéder à la lecture du contrat que la moitié de Paris, présente à cette solennité, devait signer. 

Chacun prit place, ou plutôt les femmes firent cercle, tandis que les hommes, plus indifférents à l'endroit du \textit{style énergique}, comme dit Boileau, firent leurs commentaires sur l'agitation fébrile d'Andrea, sur l'attention de M. Danglars, sur l'impassibilité d'Eugénie et sur la façon leste et enjouée dont la baronne traitait cette importante affaire. 

Le contrat fut lu au milieu d'un profond silence. Mais, aussitôt la lecture achevée, la rumeur recommença dans les salons, double de ce qu'elle était auparavant: ces sommes brillantes, ces millions roulant dans l'avenir des deux jeunes gens et qui venaient compléter l'exposition qu'on avait faite, dans une chambre exclusivement consacrée à cet objet, du trousseau de la mariée et des diamants de la jeune femme, avaient retenti avec tout leur prestige dans la jalouse assemblée. 

Les charmes de Mlle Danglars en étaient doubles aux yeux des jeunes gens, et pour le moment ils effaçaient l'éclat du soleil. 

Quant aux femmes, il va sans dire que, tout en jalousant ces millions, elles ne croyaient pas en avoir besoin pour être belles. 

Andrea, serré par ses amis, complimenté, adulé, commençant à croire à la réalité du rêve qu'il faisait, Andrea était sur le point de perdre la tête. 

Le notaire prit solennellement la plume, l'éleva au-dessus de sa tête et dit: 

«Messieurs, on va signer le contrat.» 

Le baron devait signer le premier, puis le fondé de pouvoir de M. Cavalcanti père, puis la baronne, puis les futurs conjoints, comme on dit dans cet abominable style qui a cours sur papier timbré. 

Le baron prit la plume et signa, puis le chargé de pouvoir. 

La baronne s'approcha, au bras de Mme de Villefort. 

«Mon ami, dit-elle en prenant la plume, n'est-ce pas une chose désespérante? Un incident inattendu, arrivé dans cette affaire d'assassinat et de vol dont M. le comte de Monte-Cristo a failli être victime, nous prive d'avoir M. de Villefort. 

—Oh! mon Dieu! fit Danglars, du même ton dont il aurait dit: Ma foi, la chose m'est bien indifférente! 

—Mon Dieu! dit Monte-Cristo en s'approchant, j'ai bien peur d'être la cause involontaire de cette absence. 

—Comment! vous, comte? dit Mme Danglars en signant. S'il en est ainsi, prenez garde, je ne vous le pardonnerai jamais.» 

Andrea dressait les oreilles. 

«Il n'y aurait cependant point de ma faute, dit le comte; aussi je tiens à le constater.» 

On écoutait avidement: Monte-Cristo, qui desserrait si rarement les lèvres, allait parler. 

«Vous vous rappelez, dit le comte au milieu du plus profond silence, que c'est chez moi qu'est mort ce malheureux qui était venu pour me voler, et qui, en sortant de chez moi a été tué, à ce que l'on croit, par son complice? 

—Oui, dit Danglars. 

—Eh bien, pour lui porter secours, on l'avait déshabillé et l'on avait jeté ses habits dans un coin où la justice les a ramassés; mais la justice, en prenant l'habit et le pantalon pour les déposer au greffe, avait oublié le gilet.» 

Andrea pâlit visiblement et tira tout doucement du côté de la porte; il voyait paraître un nuage à l'horizon, et ce nuage lui semblait renfermer la tempête dans ses flancs. 

«Eh bien, ce malheureux gilet, on l'a trouvé aujourd'hui tout couvert de sang et troué à l'endroit du cœur.» 

Les dames poussèrent un cri, et deux ou trois se préparèrent à s'évanouir. 

«On me l'a apporté. Personne ne pouvait deviner d'où venait cette guenille; moi seul songeai que c'était probablement le gilet de la victime. Tout à coup mon valet de chambre, en fouillant avec dégoût et précaution cette funèbre relique, a senti un papier dans la poche et l'en a tiré: c'était une lettre adressée à qui? à vous, baron. 

—À moi? s'écria Danglars. 

—Oh! mon Dieu! oui, à vous; je suis parvenu à lire votre nom sous le sang dont le billet était maculé, répondit Monte-Cristo au milieu des éclats de surprise générale. 

—Mais, demanda Mme Danglars regardant son mari avec inquiétude, comment cela empêche-t-il M. de Villefort? 

—C'est tout simple, madame, répondit Monte-Cristo; ce gilet et cette lettre étaient ce qu'on appelle des pièces de conviction; lettre et gilet, j'ai tout envoyé à M. le procureur du roi. Vous comprenez, mon cher baron, la voie légale est la plus sûre en matière criminelle: c'était peut-être quelque machination contre vous.» 

Andrea regarda fixement Monte-Cristo et disparut dans le deuxième salon. 

«C'est possible, dit Danglars; cet homme assassiné n'était-il point un ancien forçat? 

—Oui, répondit le comte, un ancien forçat nommé Caderousse.» 

Danglars pâlit légèrement; Andrea quitta le second salon et gagna l'antichambre. 

«Mais signez donc, signez donc! dit Monte-Cristo; je m'aperçois que mon récit a mis tout le monde en émoi et j'en demande bien humblement pardon à vous, madame la baronne et à Mlle Danglars.» 

La baronne, qui venait de signer, remit la plume au notaire. 

«Monsieur le prince Cavalcanti, dit le tabellion, monsieur le prince Cavalcanti, où êtes-vous? 

—Andrea! Andrea! répétèrent plusieurs voix de jeunes gens qui en étaient déjà arrivés avec le noble Italien à ce degré d'intimité de l'appeler par son nom de baptême. 

—Appelez donc le prince, prévenez-le donc que c'est à lui de signer!» cria Danglars à un huissier. 

Mais au même instant la foule des assistants reflua, terrifiée, dans le salon principal, comme si quelque monstre effroyable fût entré dans les appartements, \textit{quaerens quem devoret}. 

Il y avait en effet de quoi reculer, s'effrayer, crier. 

Un officier de gendarmerie plaçait deux gendarmes à la porte de chaque salon, et s'avançait vers Danglars, précédé d'un commissaire de police ceint de son écharpe. 

Mme Danglars poussa un cri et s'évanouit. 

Danglars, qui se croyait menacé (certaines consciences ne sont jamais calmes), Danglars offrit aux yeux de ses conviés un visage décomposé par la terreur. 

«Qu'y a-t-il donc, monsieur? demanda Monte-Cristo s'avançant au-devant du commissaire. 

—Lequel de vous, messieurs, demanda le magistrat sans répondre au comte, s'appelle Andrea Cavalcanti?» 

Un cri de stupeur partit de tous les coins du salon. On chercha; on interrogea. 

«Mais quel est donc cet Andrea Cavalcanti? demanda Danglars presque égaré. 

—Un ancien forçat échappé du bagne de Toulon. 

—Et quel crime a-t-il commis? 

—Il est prévenu, dit le commissaire de sa voix impassible, d'avoir assassiné le nommé Caderousse, son ancien compagnon de chaîne, au moment où il sortait de chez le comte de Monte-Cristo.» 

Monte-Cristo jeta un regard rapide autour de lui. 

Andrea avait disparu. 