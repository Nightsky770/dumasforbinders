\chapter{La nuit} 

\lettrine{M}{.} de Monte-Cristo attendit, selon son habitude, que Duprez eût chanté son fameux \textit{Suivez-moi}! et alors seulement il se leva et sortit. 

\zz
À la porte, Morrel le quitta en renouvelant la promesse d'être chez lui, avec Emmanuel, le lendemain matin à sept heures précises. Puis il monta dans son coupé, toujours calme et souriant. Cinq minutes après il était chez lui. Seulement il eût fallu ne pas connaître le comte pour se laisser tromper à l'expression avec laquelle il dit en entrant à Ali: 

«Ali, mes pistolets à crosse d'ivoire!» 

Ali apporta la boîte à son maître, et celui-ci se mit à examiner ces armes avec une sollicitude bien naturelle à un homme qui va confier sa vie à un peu de fer et de plomb. C'étaient des pistolets particuliers que Monte-Cristo avait fait faire pour tirer à la cible dans ses appartements. Une capsule suffisait pour chasser la balle, et de la chambre à côté on n'aurait pas pu se douter que le comte, comme on dit en termes de tir, était occupé à s'entretenir la main. 

Il en était à emboîter l'arme dans sa main, et à chercher le point de mire sur une petite plaque de tôle qui lui servait de cible, lorsque la porte de son cabinet s'ouvrit et que Baptistin entra. 

Mais, avant même qu'il eût ouvert la bouche, le comte aperçut dans la porte, demeurée ouverte, une femme voilée, debout, dans la pénombre de la pièce voisine, et qui avait suivi Baptistin. 

Elle avait aperçu le comte le pistolet à la main, elle voyait deux épées sur une table, elle s'élança. 

Baptistin consultait son maître du regard. Le comte fit un signe, Baptistin sortit, et referma la porte derrière lui. 

«Qui êtes-vous, madame?» dit le comte à la femme voilée. 

L'inconnue jeta un regard autour d'elle pour s'assurer qu'elle était bien seule, puis s'inclinant comme si elle eût voulu s'agenouiller, et joignant les mains avec accent du désespoir: 

«Edmond, dit-elle, vous ne tuerez pas mon fils!» 

Le comte fit un pas en arrière, jeta un faible cri et laissa tomber l'arme qu'il tenait. 

«Quel nom avez-vous prononcé, là, madame de Morcerf? dit-il. 

—Le vôtre! s'écria-t-elle en rejetant son voile, le vôtre que seule, peut-être, je n'ai pas oublié. Edmond, ce n'est pas Mme de Morcerf qui vient à vous, c'est Mercédès. 

—Mercédès est morte, madame, dit Monte-Cristo, et je ne connais plus personne de ce nom. 

—Mercédès vit, monsieur, et Mercédès se souvient, car seule elle vous a reconnu lorsqu'elle vous a vu, et même sans vous voir, à votre voix, Edmond, au seul accent de votre voix; et depuis ce temps elle vous suit pas à pas, elle vous surveille, elle vous redoute, et elle n'a pas eu besoin, elle, de chercher la main d'où partait le coup qui frappait M. de Morcerf. 

—Fernand, voulez-vous dire, madame, reprit Monte-Cristo avec une ironie amère; puisque nous sommes en train de nous rappeler nos noms, rappelons-nous-les tous.» 

Et Monte-Cristo avait prononcé ce nom de Fernand avec une telle expression de haine, que Mercédès sentit le frisson de l'effroi courir par tout son corps. 

«Vous voyez bien, Edmond, que je ne me suis pas trompée! s'écria Mercédès, et que j'ai raison de vous dire: Épargnez mon fils! 

—Et qui vous a dit, madame, que j'en voulais à votre fils? 

—Personne, mon Dieu! mais une mère est douée de la double vue. J'ai tout deviné; je l'ai suivi ce soir à l'Opéra, et, cachée dans une baignoire, j'ai tout vu. 

—Alors, si vous avez tout vu, madame, vous avez vu que le fils de Fernand m'a insulté publiquement? dit Monte-Cristo avec un calme terrible. 

—Oh! par pitié! 

—Vous avez vu, continua le comte, qu'il m'eût jeté son gant à la figure si un de mes amis, M. Morrel, ne lui eût arrêté le bras. 

—Écoutez-moi. Mon fils vous a deviné aussi, lui; il vous attribue les malheurs qui frappent son père. 

—Madame, dit Monte-Cristo, vous confondez: ce ne sont point des malheurs, c'est un châtiment. Ce n'est pas moi qui frappe M. de Morcerf, c'est la Providence qui le punit. 

—Et pourquoi vous substituez-vous à la Providence? s'écria Mercédès. Pourquoi vous souvenez-vous quand elle oublie? Que vous importent, à vous, Edmond, Janina et son vizir? Quel tort vous a fait Fernand Mondego en trahissant Ali-Tebelin? 

—Aussi, madame, répondit Monte-Cristo, tout ceci est-il une affaire entre le capitaine franc et la fille de Vasiliki. Cela ne me regarde point, vous avez raison, et si j'ai juré de me venger, ce n'est ni du capitaine franc, ni du comte de Morcerf: c'est du pécheur Fernand, mari de la Catalane Mercédès. 

—Ah! monsieur! s'écria la comtesse, quelle terrible vengeance pour une faute que la fatalité m'a fait commettre! Car la coupable, c'est moi, Edmond, et si vous avez à vous venger de quelqu'un, c'est de moi, qui ai manqué de force contre votre absence et mon isolement. 

—Mais, s'écria Monte-Cristo, pourquoi étais-je absent? pourquoi étiez-vous isolée? 

—Parce qu'on vous a arrêté, Edmond, parce que vous étiez prisonnier. 

—Et pourquoi étais-je arrêté? pourquoi étais-je prisonnier? 

—Je l'ignore, dit Mercédès. 

—Oui, vous l'ignorez, madame, je l'espère du moins. Eh bien, je vais vous le dire, moi. J'étais arrêté, j'étais prisonnier, parce que sous la tonnelle de la Réserve, la veille même du jour où je devais vous épouser, un homme, nommé Danglars, avait écrit cette lettre que le pêcheur Fernand se chargea lui-même de mettre à la poste.» 

Et Monte-Cristo, allant à un secrétaire, ouvrit un tiroir où il prit un papier qui avait perdu sa couleur première, et dont l'encre était devenue couleur de rouille, qu'il mit sous les yeux de Mercédès. 

C'était la lettre de Danglars au procureur du roi que, le jour où il avait payé les deux cent mille francs à M. de Boville, le comte de Monte-Cristo, déguisé en mandataire de la maison Thomson et French, avait soustraite au dossier d'Edmond Dantès. 

Mercédès lut avec effroi les lignes suivantes: 

«Monsieur le procureur du roi est prévenu, par un ami du trône et de la religion, que le nommé Edmond Dantès, second du navire \textit{Le Pharaon}, arrivé ce matin de Smyrne, après avoir touché à Naples et à Porto-Ferrajo, a été chargé par Murat d'une lettre pour l'usurpateur, et, par l'usurpateur, d'une lettre pour le comité bonapartiste de Paris. 

«On aura la preuve de ce crime en l'arrêtant, car on trouvera cette lettre, ou sur lui, ou chez son père, ou dans sa cabine à bord du \textit{Pharaon}.» 

«Oh! mon Dieu! fit Mercédès en passant la main sur son front mouillé de sueur; et cette lettre\dots 

—Je l'ai achetée deux cent mille francs, madame, dit Monte-Cristo; mais c'est bon marché encore, puisqu'elle me permet aujourd'hui de me disculper à vos yeux. 

—Et le résultat de cette lettre? 

—Vous le savez, madame, a été mon arrestation; mais ce que vous ne savez pas, madame, c'est le temps qu'elle a duré, cette arrestation. Ce que vous ne savez pas, c'est que je suis resté quatorze ans à un quart de lieue de vous, dans un cachot du château d'If. Ce que vous ne savez pas, c'est que chaque jour de ces quatorze ans j'ai renouvelé le vœu de vengeance que j'avais fait le premier jour, et cependant j'ignorais que vous aviez épousé Fernand, mon dénonciateur, et que mon père était mort, et mort de faim! 

—Juste Dieu! s'écria Mercédès chancelante. 

—Mais voilà ce que j'ai su en sortant de prison, quatorze ans après y être entré, et voilà ce qui fait que, sur Mercédès vivante et sur mon père mort, j'ai juré de me venger de Fernand, et\dots et je me venge. 

—Et vous êtes sûr que le malheureux Fernand a fait cela? 

—Sur mon âme, madame, et il l'a fait comme je vous le dis; d'ailleurs ce n'est pas beaucoup plus odieux que d'avoir, Français d'adoption, passé aux Anglais! Espagnol de naissance, avoir combattu contre les Espagnols; stipendiaire d'Ali, trahi et assassiné Ali. En face de pareilles choses, qu'était-ce que la lettre que vous venez de lire? une mystification galante que doit pardonner, je l'avoue et le comprends, la femme qui a épousé cet homme, mais que ne pardonne pas l'amant qui devait l'épouser. Eh bien, les Français ne se sont pas vengés du traître, les Espagnols n'ont pas fusillé le traître, Ali, couché dans sa tombe, a laissé impuni le traître; mais moi, trahi, assassiné, jeté aussi dans une tombe, je suis sorti de cette tombe par la grâce de Dieu, je dois à Dieu de me venger; il m'envoie pour cela, et me voici.» 

La pauvre femme laissa retomber sa tête entre ses mains; ses jambes plièrent sous elle, et elle tomba à genoux. 

«Pardonnez, Edmond, dit-elle, pardonnez pour moi, qui vous aime encore!» 

La dignité de l'épouse arrêta l'élan de l'amante et de la mère. Son front s'inclina presque à toucher le tapis. Le comte s'élança au-devant d'elle et la releva. Alors, assise sur un fauteuil, elle put, à travers ses larmes, regarder le mâle visage de Monte-Cristo, sur lequel la douleur et la haine imprimaient encore un caractère menaçant. 

«Que je n'écrase pas cette race maudite! murmura-t-il; que je désobéisse à Dieu, qui m'a suscité pour sa punition! impossible, madame, impossible! 

—Edmond, dit la pauvre mère, essayant de tous les moyens: mon Dieu! quand je vous appelle Edmond, pourquoi ne m'appelez-vous pas Mercédès? 

—Mercédès, répéta Monte-Cristo, Mercédès! Eh bien! oui, vous avez raison, ce nom m'est doux encore à prononcer, et voilà la première fois, depuis bien longtemps, qu'il retentit si clairement au sortir de mes lèvres. Ô Mercédès, votre nom, je l'ai prononcé avec les soupirs de la mélancolie, avec les gémissements de la douleur, avec le râle du désespoir; je l'ai prononcé, glacé par le froid, accroupi sur la paille de mon cachot; je l'ai prononcé, dévoré par la chaleur, en me roulant sur les dalles de ma prison. Mercédès, il faut que je me venge, car quatorze ans j'ai souffert, quatorze ans j'ai pleuré, j'ai maudit; maintenant, je vous le dis, Mercédès, il faut que je me venge!» 

Et le comte, tremblant de céder aux prières de celle qu'il avait tant aimée, appelait ses souvenirs au secours de sa haine. 

«Vengez-vous, Edmond! s'écria la pauvre mère, mais vengez-vous sur les coupables; vengez-vous sur lui, vengez-vous sur moi, mais ne vous vengez pas sur mon fils! 

—Il est écrit dans le Livre saint, répondit Monte-Cristo: «Les fautes des pères retomberont sur les enfants jusqu'à la troisième et quatrième génération.» Puisque Dieu a dicté ces propres paroles à son prophète, pourquoi serais-je meilleur que Dieu? 

—Parce que Dieu a le temps et l'éternité, ces deux choses qui échappent aux hommes.» 

Monte-Cristo poussa un soupir qui ressemblait à un rugissement, et saisit ses beaux cheveux à pleines mains. 

«Edmond, continua Mercédès, les bras tendus vers le comte, Edmond, depuis que je vous connais j'ai adoré votre nom, j'ai respecté votre mémoire. Edmond, mon ami, ne me forcez pas à ternir cette image noble et pure reflétée sans cesse dans le miroir de mon cœur. Edmond, si vous saviez toutes les prières que j'ai adressées pour vous à Dieu, tant que je vous ai espéré vivant et depuis que je vous ai cru mort, oui, mort, hélas! Je croyais votre cadavre enseveli au fond de quelque sombre tour; je croyais votre corps précipité au fond de quelqu'un de ces abîmes où les geôliers laissent rouler les prisonniers morts, et je pleurais! Moi, que pouvais-je pour vous, Edmond, sinon prier ou pleurer? Écoutez-moi; pendant dix ans j'ai fait chaque nuit le même rêve. On a dit que vous aviez voulu fuir, que vous aviez pris la place d'un prisonnier que vous vous étiez glissé dans le suaire d'un mort et qu'alors on avait lancé le cadavre vivant du haut en bas du château d'If; et que le cri que vous aviez poussé en vous brisant sur les rochers avait seul révélé la substitution à vos ensevelisseurs, devenus vos bourreaux. Eh bien, Edmond, je vous le jure sur la tête de ce fils pour lequel je vous implore, Edmond, pendant dix ans j'ai vu chaque nuit des hommes qui balançaient quelque chose d'informe et d'inconnu au haut d'un rocher; pendant dix ans j'ai, chaque nuit, entendu un cri terrible qui m'a réveillée frissonnante et glacée. Et moi aussi, Edmond, oh! croyez-moi, toute criminelle que je fusse, oh! oui, moi aussi, j'ai bien souffert. 

—Avez-vous senti mourir votre père en votre absence? s'écria Monte-Cristo enfonçant ses mains dans ses cheveux; avez-vous vu la femme que vous aimiez tendre sa main à votre rival, tandis que vous râliez au fond du gouffre?\dots 

—Non, interrompit Mercédès; mais j'ai vu celui que j'aimais prêt à devenir le meurtrier de mon fils!» 

Mercédès prononça ces paroles avec une douleur si puissante, avec un accent si désespéré, qu'à ces paroles et à cet accent un sanglot déchira la gorge du comte. 

Le lion était dompté; le vengeur était vaincu. 

«Que demandez-vous? dit-il; que votre fils vive? eh bien, il vivra!» 

Mercédès jeta un cri qui fit jaillir deux larmes des paupières de Monte-Cristo, mais ces deux larmes disparurent presque aussitôt, car sans doute Dieu avait envoyé quelque ange pour les recueillir, bien autrement précieuses qu'elles étaient aux yeux du Seigneur que les plus riches perles de Gusarate et d'Ophir. 

«Oh! s'écria-t-elle en saisissant la main du comte et en la portant à ses lèvres, oh! merci, merci, Edmond! te voilà bien tel que je t'ai toujours rêvé, tel que je t'ai toujours aimé. Oh! maintenant je puis le dire. 

—D'autant mieux, répondit Monte-Cristo, que le pauvre Edmond n'aura pas longtemps à être aimé par vous. Le mort va rentrer dans la tombe, le fantôme va rentrer dans la nuit. 

—Que dites-vous, Edmond? 

—Je dis que puisque vous l'ordonnez, Mercédès, il faut mourir. 

—Mourir! et qui est-ce qui dit cela? Qui parle de mourir? d'où vous reviennent ces idées de mort? 

—Vous ne supposez pas qu'outragé publiquement, en face de toute une salle, en présence de vos amis et de ceux de votre fils, provoqué par un enfant qui se glorifiera de mon pardon comme d'une victoire, vous ne supposez pas, dis-je, que j'aie un instant le désir de vivre. Ce que j'ai le plus aimé après vous, Mercédès, c'est moi-même, c'est-à-dire ma dignité, c'est-à-dire cette force qui me rendait supérieur aux autres hommes; cette force, c'était ma vie. D'un mot vous la brisez. Je meurs. 

—Mais ce duel n'aura pas lieu, Edmond, puisque vous pardonnez. 

—Il aura lieu, madame, dit solennellement Monte-Cristo, seulement, au lieu du sang de votre fils, que devait boire la terre, ce sera le mien qui coulera.» 

Mercédès poussa un grand cri et s'élança vers Monte-Cristo; mais tout à coup elle s'arrêta. 

«Edmond, dit-elle, il y a un Dieu au-dessus de nous, puisque vous vivez, puisque je vous ai revu, et je me fie à lui du plus profond de mon cœur. En attendant son appui, je me repose sur votre parole. Vous avez dit que mon fils vivrait; il vivra, n'est-ce pas? 

—Il vivra, oui, madame», dit Monte-Cristo, étonné que, sans autre exclamation, sans autre surprise, Mercédès eût accepté l'héroïque sacrifice qu'il lui faisait. 

Mercédès tendit la main au comte. 

«Edmond, dit-elle, tandis que ses yeux se mouillaient de larmes en regardant celui auquel elle adressait la parole, comme c'est beau de votre part, comme c'est grand ce que vous venez de faire là, comme c'est sublime d'avoir eu pitié d'une pauvre femme qui s'offrait à vous avec toutes les chances contraires à ses espérances! Hélas! je suis vieillie par les chagrins plus encore que par l'âge, et je ne puis même plus rappeler à mon Edmond par un sourire, par un regard, cette Mercédès qu'autrefois il a passé tant d'heures à contempler. Ah! croyez-moi, Edmond, je vous ai dit que, moi aussi, j'avais bien souffert; je vous le répète, cela est bien lugubre de voir passer sa vie sans se rappeler une seule joie, sans conserver une seule espérance, mais cela prouve que tout n'est point fini sur la terre. Non! tout n'est pas fini, je le sens à ce qui me reste encore dans le cœur. Oh! je vous le répète, Edmond, c'est beau, c'est grand, c'est sublime de pardonner comme vous venez de le faire! 

—Vous dites cela, Mercédès; et que diriez-vous donc si vous saviez l'étendue du sacrifice que je vous fais? Supposez que le Maître suprême, après avoir créé le monde, après avoir fertilisé le chaos, se fût arrêté au tiers de la création pour épargner à un ange les larmes que nos crimes devaient faire couler un jour de ses yeux immortels; supposez qu'après avoir tout préparé, tout pétri, tout fécondé, au moment d'admirer son œuvre, Dieu ait éteint le soleil et repoussé du pied le monde dans la nuit éternelle, alors vous aurez une idée, ou plutôt non, non, vous ne pourrez pas encore vous faire une idée de ce que je perds en perdant la vie en ce moment.» 

Mercédès regarda le comte d'un air qui peignait à la fois son étonnement, son admiration et sa reconnaissance. 

Monte-Cristo appuya son front sur ses mains brûlantes, comme si son front ne pouvait plus porter seul le poids de ses pensées. 

«Edmond, dit Mercédès, je n'ai plus qu'un mot à vous dire.» 

Le comte sourit amèrement. 

«Edmond, continua-t-elle, vous verrez que si mon front est pâli, que si mes yeux sont éteints, que si ma beauté est perdue, que si Mercédès enfin ne ressemble plus à elle-même pour les traits du visage, vous verrez que c'est toujours le même cœur!\dots Adieu donc, Edmond; je n'ai plus rien à demander au Ciel\dots Je vous ai revu aussi noble et aussi grand qu'autrefois. Adieu, Edmond\dots adieu et merci!» 

Mais le comte ne répondit pas. 

Mercédès ouvrit la porte du cabinet, et elle avait disparu avant qu'il fût revenu de la rêverie douloureuse et profonde où sa vengeance perdue l'avait plongé. 

Une heure sonnait à l'horloge des Invalides quand la voiture qui emportait Mme de Morcerf, en roulant sur le pavé des Champs-Élysées, fit relever la tête au comte de Monte-Cristo. 

«Insensé, dit-il, le jour où j'avais résolu de me venger, de ne pas m'être arraché le cœur!» 