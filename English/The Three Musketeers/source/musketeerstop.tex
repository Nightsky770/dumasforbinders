\documentclass[
%paper=5.5in:8.5in,
a5paper,
]{scrbook} 

\usepackage{bindery}
\usepackage{dropcaps}
\usepackage{epistolary}

\setdropcaps{FloralCapitals.ttf}
\renewcommand{\thechapter}{\Roman{chapter}}



\BeforeTOCHead[toc]{%
  \KOMAoptions{parskip=false}% no parskip in ToC
%  \RedeclareSectionCommand[afterskip=1sp minus 1sp]{chapter}% no skip after ToC title
}

\DeclareTOCStyleEntry[beforeskip=.1cm,dynnumwidth=true,linefill=\TOCLineLeaderFill]{chapter}{chapter}


\automark{chapter}
\lehead{The Three Musketeers}
%\renewcommand{\chaptermark}[1]{\markboth{\chaptername\ \arabic{chapter}:\ #1}{}}
\rohead{\leftmark}
\flushbottom


\begin{document}
\frontmatter
 \includepdf[width=1.3\textwidth]{titlepage.jpg}
\tableofcontents
 
 \renewcommand{\chaptermark}[1]{\markboth{#1}{}}
 \include{chapters/prologue.tex}
 \renewcommand{\chaptermark}[1]{\markboth{\chaptername\ \arabic{chapter}:\ #1}{}}
 
\mainmatter
\pagestyle{headings}
\KOMAoptions{headings=openright}

\include{chapters/01.tex}
\include{chapters/02.tex}
\include{chapters/03.tex}
\include{chapters/04.tex}
%!TeX root=../musketeersfr.tex 


\chapter[Les Mousquetaires et les Gardes]{Les Mousquetaires Du Roi et les Gardes De M. Le Cardinal} 
	
\lettrine{D}{'Artagnan} ne connaissait personne à Paris. Il alla donc au rendez-vous d'Athos sans amener de second, résolu de se contenter de ceux qu'aurait choisis son adversaire. D'ailleurs son intention était formelle de faire au brave mousquetaire toutes les excuses convenables, mais sans faiblesse, craignant qu'il ne résultât de ce duel ce qui résulte toujours de fâcheux, dans une affaire de ce genre, quand un homme jeune et vigoureux se bat contre un adversaire blessé et affaibli: vaincu, il double le triomphe de son antagoniste; vainqueur, il est accusé de forfaiture et de facile audace. 

Au reste, ou nous avons mal exposé le caractère de notre chercheur d'aventures, ou notre lecteur a déjà dû remarquer que d'Artagnan n'était point un homme ordinaire. Aussi, tout en se répétant à lui-même que sa mort était inévitable, il ne se résigna point à mourir tout doucettement, comme un autre moins courageux et moins modéré que lui eût fait à sa place. Il réfléchit aux différents caractères de ceux avec lesquels il allait se battre, et commença à voir plus clair dans sa situation. Il espérait, grâce aux excuses loyales qu'il lui réservait, se faire un ami d'Athos, dont l'air grand seigneur et la mine austère lui agréaient fort. Il se flattait de faire peur à Porthos avec l'aventure du baudrier, qu'il pouvait, s'il n'était pas tué sur le coup, raconter à tout le monde, récit qui, poussé adroitement à l'effet, devait couvrir Porthos de ridicule; enfin, quant au sournois Aramis, il n'en avait pas très grand-peur, et en supposant qu'il arrivât jusqu'à lui, il se chargeait de l'expédier bel et bien, ou du moins en le frappant au visage, comme César avait recommandé de faire aux soldats de Pompée, d'endommager à tout jamais cette beauté dont il était si fier. 

Ensuite il y avait chez d'Artagnan ce fonds inébranlable de résolution qu'avaient déposé dans son cœur les conseils de son père, conseils dont la substance était: «Ne rien souffrir de personne que du roi, du cardinal et de M. de Tréville.» Il vola donc plutôt qu'il ne marcha vers le couvent des Carmes Déchaussés, ou plutôt Deschaux, comme on disait à cette époque, sorte de bâtiment sans fenêtres, bordé de prés arides, succursale du Pré-aux-Clercs, et qui servait d'ordinaire aux rencontres des gens qui n'avaient pas de temps à perdre. 

Lorsque d'Artagnan arriva en vue du petit terrain vague qui s'étendait au pied de ce monastère, Athos attendait depuis cinq minutes seulement, et midi sonnait. Il était donc ponctuel comme la Samaritaine, et le plus rigoureux casuiste à l'égard des duels n'avait rien a dire. 

Athos, qui souffrait toujours cruellement de sa blessure, quoiqu'elle eût été pansée à neuf par le chirurgien de M. de Tréville, s'était assis sur une borne et attendait son adversaire avec cette contenance paisible et cet air digne qui ne l'abandonnaient jamais. À l'aspect de d'Artagnan, il se leva et fit poliment quelques pas au-devant de lui. Celui-ci, de son côté, n'aborda son adversaire que le chapeau à la main et sa plume traînant jusqu'à terre. 

«Monsieur, dit Athos, j'ai fait prévenir deux de mes amis qui me serviront de seconds, mais ces deux amis ne sont point encore arrivés. Je m'étonne qu'ils tardent: ce n'est pas leur habitude. 

\speak  Je n'ai pas de seconds, moi, monsieur, dit d'Artagnan, car arrivé d'hier seulement à Paris, je n'y connais encore personne que M. de Tréville, auquel j'ai été recommandé par mon père qui a l'honneur d'être quelque peu de ses amis.» 

Athos réfléchit un instant. 

«Vous ne connaissez que M. de Tréville? demanda-t-il. 

\speak  Oui, monsieur, je ne connais que lui. 

\speak  Ah çà, mais\dots, continua Athos parlant moitié à lui-même, moitié à d'Artagnan, ah\dots çà, mais si je vous tue, j'aurai l'air d'un mangeur d'enfants, moi! 

\speak  Pas trop, monsieur, répondit d'Artagnan avec un salut qui ne manquait pas de dignité; pas trop, puisque vous me faites l'honneur de tirer l'épée contre moi avec une blessure dont vous devez être fort incommodé. 

\speak  Très incommodé, sur ma parole, et vous m'avez fait un mal du diable, je dois le dire; mais je prendrai la main gauche, c'est mon habitude en pareille circonstance. Ne croyez donc pas que je vous fasse une grâce, je tire proprement des deux mains; et il y aura même désavantage pour vous: un gaucher est très gênant pour les gens qui ne sont pas prévenus. Je regrette de ne pas vous avoir fait part plus tôt de cette circonstance. 

\speak  Vous êtes vraiment, monsieur, dit d'Artagnan en s'inclinant de nouveau, d'une courtoisie dont je vous suis on ne peut plus reconnaissant. 

\speak  Vous me rendez confus, répondit Athos avec son air de gentilhomme; causons donc d'autre chose, je vous prie, à moins que cela ne vous soit désagréable. Ah! sangbleu! que vous m'avez fait mal! l'épaule me brûle. 

\speak  Si vous vouliez permettre\dots, dit d'Artagnan avec timidité. 

\speak  Quoi, monsieur? 

\speak  J'ai un baume miraculeux pour les blessures, un baume qui me vient de ma mère, et dont j'ai fait l'épreuve sur moi-même. 

\speak  Eh bien? 

\speak  Eh bien, je suis sûr qu'en moins de trois jours ce baume vous guérirait, et au bout de trois jours, quand vous seriez guéri: eh bien, monsieur, ce me serait toujours un grand honneur d'être votre homme.» 

D'Artagnan dit ces mots avec une simplicité qui faisait honneur à sa courtoisie, sans porter aucunement atteinte à son courage. 

«Pardieu, monsieur, dit Athos, voici une proposition qui me plaît, non pas que je l'accepte, mais elle sent son gentilhomme d'une lieue. C'est ainsi que parlaient et faisaient ces preux du temps de Charlemagne, sur lesquels tout cavalier doit chercher à se modeler. Malheureusement, nous ne sommes plus au temps du grand empereur. Nous sommes au temps de M. le cardinal, et d'ici à trois jours on saurait, si bien gardé que soit le secret, on saurait, dis-je, que nous devons nous battre, et l'on s'opposerait à notre combat. Ah çà, mais! ces flâneurs ne viendront donc pas? 

\speak  Si vous êtes pressé, monsieur, dit d'Artagnan à Athos avec la même simplicité qu'un instant auparavant il lui avait proposé de remettre le duel à trois jours, si vous êtes pressé et qu'il vous plaise de m'expédier tout de suite, ne vous gênez pas, je vous en prie. 

\speak  Voilà encore un mot qui me plaît, dit Athos en faisant un gracieux signe de tête à d'Artagnan, il n'est point d'un homme sans cervelle, et il est à coup sûr d'un homme de cœur. Monsieur, j'aime les hommes de votre trempe, et je vois que si nous ne nous tuons pas l'un l'autre, j'aurai plus tard un vrai plaisir dans votre conversation. Attendons ces messieurs, je vous prie, j'ai tout le temps, et cela sera plus correct. Ah! en voici un, je crois.» 

En effet, au bout de la rue de Vaugirard commençait à apparaître le gigantesque Porthos. 

«Quoi! s'écria d'Artagnan, votre premier témoin est M. Porthos? 

\speak  Oui, cela vous contrarie-t-il? 

\speak  Non, aucunement. 

\speak  Et voici le second.» 

D'Artagnan se retourna du côté indiqué par Athos, et reconnut Aramis. 

«Quoi! s'écria-t-il d'un accent plus étonné que la première fois, votre second témoin est M. Aramis? 

\speak  Sans doute, ne savez-vous pas qu'on ne nous voit jamais l'un sans l'autre, et qu'on nous appelle, dans les mousquetaires et dans les gardes, à la cour et à la ville, Athos, Porthos et Aramis ou les trois inséparables? Après cela, comme vous arrivez de Dax ou de Pau\dots 

\speak  De Tarbes, dit d'Artagnan. 

\speak \dots Il vous est permis d'ignorer ce détail, dit Athos. 

\speak  Ma foi, dit d'Artagnan, vous êtes bien nommés, messieurs, et mon aventure, si elle fait quelque bruit, prouvera du moins que votre union n'est point fondée sur les contrastes.» 

Pendant ce temps, Porthos s'était rapproché, avait salué de la main Athos; puis, se retournant vers d'Artagnan, il était resté tout étonné. 

Disons, en passant, qu'il avait changé de baudrier et quitté son manteau. 

«Ah! ah! fit-il, qu'est-ce que cela? 

\speak  C'est avec monsieur que je me bats, dit Athos en montrant de la main d'Artagnan, et en le saluant du même geste. 

\speak  C'est avec lui que je me bats aussi, dit Porthos. 

\speak  Mais à une heure seulement, répondit d'Artagnan. 

\speak  Et moi aussi, c'est avec monsieur que je me bats, dit Aramis en arrivant à son tour sur le terrain. 

\speak  Mais à deux heures seulement, fit d'Artagnan avec le même calme. 

\speak  Mais à propos de quoi te bats-tu, toi, Athos? demanda Aramis. 

\speak  Ma foi, je ne sais pas trop, il m'a fait mal à l'épaule; et toi, Porthos? 

\speak  Ma foi, je me bats parce que je me bats», répondit Porthos en rougissant. 

Athos, qui ne perdait rien, vit passer un fin sourire sur les lèvres du Gascon. 

«Nous avons eu une discussion sur la toilette, dit le jeune homme. 

\speak  Et toi, Aramis? demanda Athos. 

\speak  Moi, je me bats pour cause de théologie», répondit Aramis tout en faisant signe à d'Artagnan qu'il le priait de tenir secrète la cause de son duel. 

Athos vit passer un second sourire sur les lèvres de d'Artagnan. 

«Vraiment, dit Athos. 

\speak  Oui, un point de saint Augustin sur lequel nous ne sommes pas d'accord, dit le Gascon. 

\speak  Décidément c'est un homme d'esprit, murmura Athos. 

\speak  Et maintenant que vous êtes rassemblés, messieurs, dit d'Artagnan, permettez-moi de vous faire mes excuses.» 

À ce mot d'\textit{excuses}, un nuage passa sur le front d'Athos, un sourire hautain glissa sur les lèvres de Porthos, et un signe négatif fut la réponse d'Aramis. 

«Vous ne me comprenez pas, messieurs, dit d'Artagnan en relevant sa tête, sur laquelle jouait en ce moment un rayon de soleil qui en dorait les lignes fines et hardies: je vous demande excuse dans le cas où je ne pourrais vous payer ma dette à tous trois, car M. Athos a le droit de me tuer le premier, ce qui ôte beaucoup de sa valeur à votre créance, monsieur Porthos, et ce qui rend la vôtre à peu près nulle, monsieur Aramis. Et maintenant, messieurs, je vous le répète, excusez-moi, mais de cela seulement, et en garde!» 

À ces mots, du geste le plus cavalier qui se puisse voir, d'Artagnan tira son épée. 

Le sang était monté à la tête de d'Artagnan, et dans ce moment il eût tiré son épée contre tous les mousquetaires du royaume, comme il venait de faire contre Athos, Porthos et Aramis. 

Il était midi et un quart. Le soleil était à son zénith et l'emplacement choisi pour être le théâtre du duel se trouvait exposé à toute son ardeur. 

«Il fait très chaud, dit Athos en tirant son épée à son tour, et cependant je ne saurais ôter mon pourpoint; car, tout à l'heure encore, j'ai senti que ma blessure saignait, et je craindrais de gêner monsieur en lui montrant du sang qu'il ne m'aurait pas tiré lui-même. 

\speak  C'est vrai, monsieur, dit d'Artagnan, et tiré par un autre ou par moi, je vous assure que je verrai toujours avec bien du regret le sang d'un aussi brave gentilhomme; je me battrai donc en pourpoint comme vous. 

\speak  Voyons, voyons, dit Porthos, assez de compliments comme cela, et songez que nous attendons notre tour. 

\speak  Parlez pour vous seul, Porthos, quand vous aurez à dire de pareilles incongruités, interrompit Aramis. Quant à moi, je trouve les choses que ces messieurs se disent fort bien dites et tout à fait dignes de deux gentilshommes. 

\speak  Quand vous voudrez, monsieur, dit Athos en se mettant en garde. 

\speak  J'attendais vos ordres», dit d'Artagnan en croisant le fer. 

Mais les deux rapières avaient à peine résonné en se touchant, qu'une escouade des gardes de Son Éminence, commandée par M. de Jussac, se montra à l'angle du couvent. 

«Les gardes du cardinal! s'écrièrent à la fois Porthos et Aramis. L'épée au fourreau, messieurs! l'épée au fourreau! 

Mais il était trop tard. Les deux combattants avaient été vus dans une pose qui ne permettait pas de douter de leurs intentions. 

«Holà! cria Jussac en s'avançant vers eux et en faisant signe à ses hommes d'en faire autant, holà! mousquetaires, on se bat donc ici? Et les édits, qu'en faisons-nous? 

\speak  Vous êtes bien généreux, messieurs les gardes, dit Athos plein de rancune, car Jussac était l'un des agresseurs de l'avant-veille. Si nous vous voyions battre, je vous réponds, moi, que nous nous garderions bien de vous en empêcher. Laissez-nous donc faire, et vous allez avoir du plaisir sans prendre aucune peine. 

\speak  Messieurs, dit Jussac, c'est avec grand regret que je vous déclare que la chose est impossible. Notre devoir avant tout. Rengainez donc, s'il vous plaît, et nous suivez. 

\speak  Monsieur, dit Aramis parodiant Jussac, ce serait avec un grand plaisir que nous obéirions à votre gracieuse invitation, si cela dépendait de nous; mais malheureusement la chose est impossible: M. de Tréville nous l'a défendu. Passez donc votre chemin, c'est ce que vous avez de mieux à faire.» 

Cette raillerie exaspéra Jussac. 

«Nous vous chargerons donc, dit-il, si vous désobéissez. 

\speak  Ils sont cinq, dit Athos à demi-voix, et nous ne sommes que trois; nous serons encore battus, et il nous faudra mourir ici, car je le déclare, je ne reparais pas vaincu devant le capitaine.» 

Alors Porthos et Aramis se rapprochèrent à l'instant les uns des autres, pendant que Jussac alignait ses soldats. 

Ce seul moment suffit à d'Artagnan pour prendre son parti: c'était là un de ces événements qui décident de la vie d'un homme, c'était un choix à faire entre le roi et le cardinal; ce choix fait, il allait y persévérer. Se battre, c'est-à-dire désobéir à la loi, c'est-à-dire risquer sa tête, c'est-à-dire se faire d'un seul coup l'ennemi d'un ministre plus puissant que le roi lui-même: voilà ce qu'entrevit le jeune homme, et, disons-le à sa louange, il n'hésita point une seconde. Se tournant donc vers Athos et ses amis: 

«Messieurs, dit-il, je reprendrai, s'il vous plaît, quelque chose à vos paroles. Vous avez dit que vous n'étiez que trois, mais il me semble, à moi, que nous sommes quatre. 

\speak  Mais vous n'êtes pas des nôtres, dit Porthos. 

\speak  C'est vrai, répondit d'Artagnan; je n'ai pas l'habit, mais j'ai l'âme. Mon cœur est mousquetaire, je le sens bien, monsieur, et cela m'entraîne. 

\speak  Écartez-vous, jeune homme, cria Jussac, qui sans doute à ses gestes et à l'expression de son visage avait deviné le dessein de d'Artagnan. Vous pouvez vous retirer, nous y consentons. Sauvez votre peau; allez vite.» 

D'Artagnan ne bougea point. 

«Décidément vous êtes un joli garçon, dit Athos en serrant la main du jeune homme. 

\speak  Allons! allons! prenons un parti, reprit Jussac. 

\speak  Voyons, dirent Porthos et Aramis, faisons quelque chose. 

\speak  Monsieur est plein de générosité», dit Athos. 

Mais tous trois pensaient à la jeunesse de d'Artagnan et redoutaient son inexpérience. 

«Nous ne serons que trois, dont un blessé, plus un enfant, reprit Athos, et l'on n'en dira pas moins que nous étions quatre hommes. 

\speak  Oui, mais reculer! dit Porthos. 

\speak  C'est difficile», reprit Athos. 

D'Artagnan comprit leur irrésolution. 

«Messieurs, essayez-moi toujours, dit-il, et je vous jure sur l'honneur que je ne veux pas m'en aller d'ici si nous sommes vaincus. 

\speak  Comment vous appelle-t-on, mon brave? dit Athos. 

\speak  D'Artagnan, monsieur. 

\speak  Eh bien, Athos, Porthos, Aramis et d'Artagnan, en avant! cria Athos. 

\speak  Eh bien, voyons, messieurs, vous décidez-vous à vous décider? cria pour la troisième fois Jussac. 

\speak  C'est fait, messieurs, dit Athos. 

\speak  Et quel parti prenez-vous? demanda Jussac. 

Nous allons avoir l'honneur de vous charger, répondit Aramis en levant son chapeau d'une main et tirant son épée de l'autre. 

\speak  Ah! vous résistez! s'écria Jussac. 

\speak  Sangdieu! cela vous étonne?» 

Et les neuf combattants se précipitèrent les uns sur les autres avec une furie qui n'excluait pas une certaine méthode. 

Athos prit un certain Cahusac, favori du cardinal; Porthos eut Biscarat, et Aramis se vit en face de deux adversaires. 

Quant à d'Artagnan, il se trouva lancé contre Jussac lui-même. 

Le cœur du jeune Gascon battait à lui briser la poitrine, non pas de peur, Dieu merci! il n'en avait pas l'ombre, mais d'émulation; il se battait comme un tigre en fureur, tournant dix fois autour de son adversaire, changeant vingt fois ses gardes et son terrain. Jussac était, comme on le disait alors, friand de la lame, et avait fort pratiqué; cependant il avait toutes les peines du monde à se défendre contre un adversaire qui, agile et bondissant, s'écartait à tout moment des règles reçues, attaquant de tous côtés à la fois, et tout cela en parant en homme qui a le plus grand respect pour son épiderme. 

Enfin cette lutte finit par faire perdre patience à Jussac. Furieux d'être tenu en échec par celui qu'il avait regardé comme un enfant, il s'échauffa et commença à faire des fautes. D'Artagnan, qui, à défaut de la pratique, avait une profonde théorie, redoubla d'agilité. Jussac, voulant en finir, porta un coup terrible à son adversaire en se fendant à fond; mais celui-ci para prime, et tandis que Jussac se relevait, se glissant comme un serpent sous son fer, il lui passa son épée au travers du corps. Jussac tomba comme une masse. 

D'Artagnan jeta alors un coup d'œil inquiet et rapide sur le champ de bataille. 

Aramis avait déjà tué un de ses adversaires; mais l'autre le pressait vivement. Cependant Aramis était en bonne situation et pouvait encore se défendre. 

Biscarat et Porthos venaient de faire coup fourré: Porthos avait reçu un coup d'épée au travers du bras, et Biscarat au travers de la cuisse. Mais comme ni l'une ni l'autre des deux blessures n'était grave, ils ne s'en escrimaient qu'avec plus d'acharnement. 

Athos, blessé de nouveau par Cahusac, pâlissait à vue d'œil, mais il ne reculait pas d'une semelle: il avait seulement changé son épée de main, et se battait de la main gauche. 

D'Artagnan, selon les lois du duel de cette époque, pouvait secourir quelqu'un; pendant qu'il cherchait du regard celui de ses compagnons qui avait besoin de son aide, il surprit un coup d'œil d'Athos. Ce coup d'œil était d'une éloquence sublime. Athos serait mort plutôt que d'appeler au secours; mais il pouvait regarder, et du regard demander un appui. D'Artagnan le devina, fit un bond terrible et tomba sur le flanc de Cahusac en criant: 

«À moi, monsieur le garde, je vous tue!» 

Cahusac se retourna; il était temps. Athos, que son extrême courage soutenait seul, tomba sur un genou. 

«Sangdieu! criait-il à d'Artagnan, ne le tuez pas, jeune homme, je vous en prie; j'ai une vieille affaire à terminer avec lui, quand je serai guéri et bien portant. Désarmez-le seulement, liez-lui l'épée. C'est cela. Bien! très bien!» 

Cette exclamation était arrachée à Athos par l'épée de Cahusac qui sautait à vingt pas de lui. D'Artagnan et Cahusac s'élancèrent ensemble, l'un pour la ressaisir, l'autre pour s'en emparer; mais d'Artagnan, plus leste, arriva le premier et mit le pied dessus. 

Cahusac courut à celui des gardes qu'avait tué Aramis, s'empara de sa rapière, et voulut revenir à d'Artagnan; mais sur son chemin il rencontra Athos, qui, pendant cette pause d'un instant que lui avait procurée d'Artagnan, avait repris haleine, et qui, de crainte que d'Artagnan ne lui tuât son ennemi, voulait recommencer le combat. 

D'Artagnan comprit que ce serait désobliger Athos que de ne pas le laisser faire. En effet, quelques secondes après, Cahusac tomba la gorge traversée d'un coup d'épée. 

Au même instant, Aramis appuyait son épée contre la poitrine de son adversaire renversé, et le forçait à demander merci. 

Restaient Porthos et Biscarat. Porthos faisait mille fanfaronnades, demandant à Biscarat quelle heure il pouvait bien être, et lui faisait ses compliments sur la compagnie que venait d'obtenir son frère dans le régiment de Navarre; mais tout en raillant, il ne gagnait rien. Biscarat était un de ces hommes de fer qui ne tombent que morts. 

Cependant il fallait en finir. Le guet pouvait arriver et prendre tous les combattants, blessés ou non, royalistes ou cardinalistes. Athos, Aramis et d'Artagnan entourèrent Biscarat et le sommèrent de se rendre. Quoique seul contre tous, et avec un coup d'épée qui lui traversait la cuisse, Biscarat voulait tenir; mais Jussac, qui s'était élevé sur son coude, lui cria de se rendre. Biscarat était un Gascon comme d'Artagnan; il fit la sourde oreille et se contenta de rire, et entre deux parades, trouvant le temps de désigner, du bout de son épée, une place à terre: 

«Ici, dit-il, parodiant un verset de la Bible, ici mourra Biscarat, seul de ceux qui sont avec lui. 

\speak  Mais ils sont quatre contre toi; finis-en, je te l'ordonne. 

\speak  Ah! si tu l'ordonnes, c'est autre chose, dit Biscarat, comme tu es mon brigadier, je dois obéir.» 

Et, faisant un bond en arrière, il cassa son épée sur son genou pour ne pas la rendre, en jeta les morceaux pardessus le mur du couvent et se croisa les bras en sifflant un air cardinaliste. 

La bravoure est toujours respectée, même dans un ennemi. Les mousquetaires saluèrent Biscarat de leurs épées et les remirent au fourreau. D'Artagnan en fit autant, puis, aidé de Biscarat, le seul qui fut resté debout, il porta sous le porche du couvent Jussac, Cahusac et celui des adversaires d'Aramis qui n'était que blessé. Le quatrième, comme nous l'avons dit, était mort. Puis ils sonnèrent la cloche, et, emportant quatre épées sur cinq, ils s'acheminèrent ivres de joie vers l'hôtel de M. de Tréville. On les voyait entrelacés, tenant toute la largeur de la rue, et accostant chaque mousquetaire qu'ils rencontraient, si bien qu'à la fin ce fut une marche triomphale. Le cœur de d'Artagnan nageait dans l'ivresse, il marchait entre Athos et Porthos en les étreignant tendrement. 

«Si je ne suis pas encore mousquetaire, dit-il à ses nouveaux amis en franchissant la porte de l'hôtel de M. de Tréville, au moins me voilà reçu apprenti, n'est-ce pas?»
%!TeX root=../musketeersfr.tex 

\chapter{Sa Majesté Le Roi Louis Treizième} 

\lettrine{L}{'affaire} fit grand bruit. M. de Tréville gronda beaucoup tout haut contre ses mousquetaires, et les félicita tout bas; mais comme il n'y avait pas de temps à perdre pour prévenir le roi, M. de Tréville s'empressa de se rendre au Louvre. Il était déjà trop tard, le roi était enfermé avec le cardinal, et l'on dit à M. de Tréville que le roi travaillait et ne pouvait recevoir en ce moment. Le soir, M. de Tréville vint au jeu du roi. Le roi gagnait, et comme Sa Majesté était fort avare, elle était d'excellente humeur; aussi, du plus loin que le roi aperçut Tréville: 

«Venez ici, monsieur le capitaine, dit-il, venez que je vous gronde; savez-vous que Son Éminence est venue me faire des plaintes sur vos mousquetaires, et cela avec une telle émotion, que ce soir Son Éminence en est malade? Ah çà, mais ce sont des diables à quatre, des gens à pendre, que vos mousquetaires! 

\speak  Non, Sire, répondit Tréville, qui vit du premier coup d'œil comment la chose allait tourner; non, tout au contraire, ce sont de bonnes créatures, douces comme des agneaux, et qui n'ont qu'un désir, je m'en ferais garant: c'est que leur épée ne sorte du fourreau que pour le service de Votre Majesté. Mais, que voulez-vous, les gardes de M. le cardinal sont sans cesse à leur chercher querelle, et, pour l'honneur même du corps, les pauvres jeunes gens sont obligés de se défendre. 

\speak  Écoutez M. de Tréville! dit le roi, écoutez-le! ne dirait-on pas qu'il parle d'une communauté religieuse! En vérité, mon cher capitaine, j'ai envie de vous ôter votre brevet et de le donner à Mlle de Chémerault, à laquelle j'ai promis une abbaye. Mais ne pensez pas que je vous croirai ainsi sur parole. On m'appelle Louis le Juste, monsieur de Tréville, et tout à l'heure, tout à l'heure nous verrons. 

\speak  Ah! c'est parce que je me fie à cette justice, Sire, que j'attendrai patiemment et tranquillement le bon plaisir de Votre Majesté. 

\speak  Attendez donc, monsieur, attendez donc, dit le roi, je ne vous ferai pas longtemps attendre.» 

En effet, la chance tournait, et comme le roi commençait à perdre ce qu'il avait gagné, il n'était pas fâché de trouver un prétexte pour faire --- qu'on nous passe cette expression de joueur, dont, nous l'avouons, nous ne connaissons pas l'origine ---, pour faire charlemagne. Le roi se leva donc au bout d'un instant, et mettant dans sa poche l'argent qui était devant lui et dont la majeure partie venait de son gain: 

«La Vieuville, dit-il, prenez ma place, il faut que je parle à M. de Tréville pour affaire d'importance. Ah!\dots j'avais quatre-vingts louis devant moi; mettez la même somme, afin que ceux qui ont perdu n'aient point à se plaindre. La justice avant tout.» 

Puis, se retournant vers M. de Tréville et marchant avec lui vers l'embrasure d'une fenêtre: 

«Eh bien, monsieur, continua-t-il, vous dites que ce sont les gardes de l'Éminentissime qui ont été chercher querelle à vos mousquetaires? 

\speak  Oui, Sire, comme toujours. 

\speak  Et comment la chose est-elle venue, voyons? car, vous le savez, mon cher capitaine, il faut qu'un juge écoute les deux parties. 

\speak  Ah! mon Dieu! de la façon la plus simple et la plus naturelle. Trois de mes meilleurs soldats, que Votre Majesté connaît de nom et dont elle a plus d'une fois apprécié le dévouement, et qui ont, je puis l'affirmer au roi, son service fort à cœur; --- trois de mes meilleurs soldats, dis-je, MM. Athos, Porthos et Aramis, avaient fait une partie de plaisir avec un jeune cadet de Gascogne que je leur avais recommandé le matin même. La partie allait avoir lieu à Saint-Germain, je crois, et ils s'étaient donné rendez-vous aux Carmes-Deschaux, lorsqu'elle fut troublée par M. de Jussac et MM. Cahusac, Biscarat, et deux autres gardes qui ne venaient certes pas là en si nombreuse compagnie sans mauvaise intention contre les édits. 

\speak  Ah! ah! vous m'y faites penser, dit le roi: sans doute, ils venaient pour se battre eux-mêmes. 

\speak  Je ne les accuse pas, Sire, mais je laisse Votre Majesté apprécier ce que peuvent aller faire cinq hommes armés dans un lieu aussi désert que le sont les environs du couvent des Carmes. 

\speak  Oui, vous avez raison, Tréville, vous avez raison. 

\speak  Alors, quand ils ont vu mes mousquetaires, ils ont changé d'idée et ils ont oublié leur haine particulière pour la haine de corps; car Votre Majesté n'ignore pas que les mousquetaires, qui sont au roi et rien qu'au roi, sont les ennemis naturels des gardes, qui sont à M. le cardinal. 

\speak  Oui, Tréville, oui, dit le roi mélancoliquement, et c'est bien triste, croyez-moi, de voir ainsi deux partis en France, deux têtes à la royauté; mais tout cela finira, Tréville, tout cela finira. Vous dites donc que les gardes ont cherché querelle aux mousquetaires? 

\speak  Je dis qu'il est probable que les choses se sont passées ainsi, mais je n'en jure pas, Sire. Vous savez combien la vérité est difficile à connaître, et à moins d'être doué de cet instinct admirable qui a fait nommer Louis XIII le Juste\dots 

\speak  Et vous avez raison, Tréville; mais ils n'étaient pas seuls, vos mousquetaires, il y avait avec eux un enfant? 

\speak  Oui, Sire, et un homme blessé, de sorte que trois mousquetaires du roi, dont un blessé, et un enfant, non seulement ont tenu tête à cinq des plus terribles gardes de M. le cardinal, mais encore en ont porté quatre à terre. 

\speak  Mais c'est une victoire, cela! s'écria le roi tout rayonnant; une victoire complète! 

\speak  Oui, Sire, aussi complète que celle du pont de Cé. 

\speak  Quatre hommes, dont un blessé, et un enfant, dites-vous? 

\speak  Un jeune homme à peine; lequel s'est même si parfaitement conduit en cette occasion, que je prendrai la liberté de le recommander à Votre Majesté. 

\speak  Comment s'appelle-t-il? 

\speak  D'Artagnan, Sire. C'est le fils d'un de mes plus anciens amis; le fils d'un homme qui a fait avec le roi votre père, de glorieuse mémoire, la guerre de partisan. 

\speak  Et vous dites qu'il s'est bien conduit, ce jeune homme? Racontez-moi cela, Tréville; vous savez que j'aime les récits de guerre et de combat.» 

Et le roi Louis XIII releva fièrement sa moustache en se posant sur la hanche. 

«Sire, reprit Tréville, comme je vous l'ai dit M. d'Artagnan est presque un enfant, et comme il n'a pas l'honneur d'être mousquetaire, il était en habit bourgeois; les gardes de M. le cardinal, reconnaissant sa grande jeunesse et, de plus, qu'il était étranger au corps, l'invitèrent donc à se retirer avant qu'ils attaquassent. 

\speak  Alors, vous voyez bien, Tréville, interrompit le roi, que ce sont eux qui ont attaqué. 

\speak  C'est juste, Sire: ainsi, plus de doute; ils le sommèrent donc de se retirer; mais il répondit qu'il était mousquetaire de cœur et tout à Sa Majesté, qu'ainsi donc il resterait avec messieurs les mousquetaires. 

\speak  Brave jeune homme! murmura le roi. 

\speak  En effet, il demeura avec eux; et Votre Majesté a là un si ferme champion, que ce fut lui qui donna à Jussac ce terrible coup d'épée qui met si fort en colère M. le cardinal. 

\speak  C'est lui qui a blessé Jussac? s'écria le roi; lui, un enfant! Ceci, Tréville, c'est impossible. 

\speak  C'est comme j'ai l'honneur de le dire à Votre Majesté. 

\speak  Jussac, une des premières lames du royaume! 

\speak  Eh bien, Sire! il a trouvé son maître. 

\speak  Je veux voir ce jeune homme, Tréville, je veux le voir, et si l'on peut faire quelque chose, eh bien, nous nous en occuperons. 

\speak  Quand Votre Majesté daignera-t-elle le recevoir? 

\speak  Demain à midi, Tréville. 

\speak  L'amènerai-je seul? 

\speak  Non, amenez-les-moi tous les quatre ensemble. Je veux les remercier tous à la fois; les hommes dévoués sont rares, Tréville, et il faut récompenser le dévouement. 

\speak  À midi, Sire, nous serons au Louvre. 

\speak  Ah! par le petit escalier, Tréville, par le petit escalier. Il est inutile que le cardinal sache\dots 

\speak  Oui, Sire. 

\speak  Vous comprenez, Tréville, un édit est toujours un édit; il est défendu de se battre, au bout du compte. 

\speak  Mais cette rencontre, Sire, sort tout à fait des conditions ordinaires d'un duel: c'est une rixe, et la preuve, c'est qu'ils étaient cinq gardes du cardinal contre mes trois mousquetaires et M. d'Artagnan. 

\speak  C'est juste, dit le roi; mais n'importe, Tréville, venez toujours par le petit escalier.» 

Tréville sourit. Mais comme c'était déjà beaucoup pour lui d'avoir obtenu de cet enfant qu'il se révoltât contre son maître, il salua respectueusement le roi, et avec son agrément prit congé de lui. 

Dès le soir même, les trois mousquetaires furent prévenus de l'honneur qui leur était accordé. Comme ils connaissaient depuis longtemps le roi, ils n'en furent pas trop échauffés: mais d'Artagnan, avec son imagination gasconne, y vit sa fortune à venir, et passa la nuit à faire des rêves d'or. Aussi, dès huit heures du matin, était-il chez Athos. 

D'Artagnan trouva le mousquetaire tout habillé et prêt à sortir. Comme on n'avait rendez-vous chez le roi qu'à midi, il avait formé le projet, avec Porthos et Aramis, d'aller faire une partie de paume dans un tripot situé tout près des écuries du Luxembourg. Athos invita d'Artagnan à les suivre, et malgré son ignorance de ce jeu, auquel il n'avait jamais joué, celui-ci accepta, ne sachant que faire de son temps, depuis neuf heures du matin qu'il était à peine jusqu'à midi. 

Les deux mousquetaires étaient déjà arrivés et pelotaient ensemble. Athos, qui était très fort à tous les exercices du corps, passa avec d'Artagnan du côté opposé, et leur fit défi. Mais au premier mouvement qu'il essaya, quoiqu'il jouât de la main gauche, il comprit que sa blessure était encore trop récente pour lui permettre un pareil exercice. D'Artagnan resta donc seul, et comme il déclara qu'il était trop maladroit pour soutenir une partie en règle, on continua seulement à s'envoyer des balles sans compter le jeu. Mais une de ces balles, lancée par le poignet herculéen de Porthos, passa si près du visage de d'Artagnan, qu'il pensa que si, au lieu de passer à côté, elle eût donné dedans, son audience était probablement perdue, attendu qu'il lui eût été de toute impossibilité de se présenter chez le roi. Or, comme de cette audience, dans son imagination gasconne, dépendait tout son avenir, il salua poliment Porthos et Aramis, déclarant qu'il ne reprendrait la partie que lorsqu'il serait en état de leur tenir tête, et il s'en revint prendre place près de la corde et dans la galerie. 

Malheureusement pour d'Artagnan, parmi les spectateurs se trouvait un garde de Son Éminence, lequel, tout échauffé encore de la défaite de ses compagnons, arrivée la veille seulement, s'était promis de saisir la première occasion de la venger. Il crut donc que cette occasion était venue, et s'adressant à son voisin: 

«Il n'est pas étonnant, dit-il, que ce jeune homme ait eu peur d'une balle, c'est sans doute un apprenti mousquetaire.» 

D'Artagnan se retourna comme si un serpent l'eût mordu, et regarda fixement le garde qui venait de tenir cet insolent propos. 

«Pardieu! reprit celui-ci en frisant insolemment, sa moustache, regardez-moi tant que vous voudrez, mon petit monsieur, j'ai dit ce que j'ai dit. 

\speak  Et comme ce que vous avez dit est trop clair pour que vos paroles aient besoin d'explication, répondit d'Artagnan à voix basse, je vous prierai de me suivre. 

\speak  Et quand cela? demanda le garde avec le même air railleur. 

\speak  Tout de suite, s'il vous plaît. 

\speak  Et vous savez qui je suis, sans doute? 

\speak Moi, je l'ignore complètement, et je ne m'en inquiète guère. 

\speak  Et vous avez tort, car, si vous saviez mon nom, peut-être seriez-vous moins pressé. 

\speak  Comment vous appelez-vous? 

\speak  Bernajoux, pour vous servir. 

\speak  Eh bien, monsieur Bernajoux, dit tranquillement d'Artagnan, je vais vous attendre sur la porte. 

\speak  Allez, monsieur, je vous suis. 

\speak  Ne vous pressez pas trop, monsieur, qu'on ne s'aperçoive pas que nous sortons ensemble; vous comprenez que pour ce que nous allons faire, trop de monde nous gênerait. 

\speak  C'est bien», répondit le garde, étonné que son nom n'eût pas produit plus d'effet sur le jeune homme. 

En effet, le nom de Bernajoux était connu de tout le monde, de d'Artagnan seul excepté, peut-être; car c'était un de ceux qui figuraient le plus souvent dans les rixes journalières que tous les édits du roi et du cardinal n'avaient pu réprimer. 

Porthos et Aramis étaient si occupés de leur partie, et Athos les regardait avec tant d'attention, qu'ils ne virent pas même sortir leur jeune compagnon, lequel, ainsi qu'il l'avait dit au garde de Son Éminence, s'arrêta sur la porte; un instant après, celui-ci descendit à son tour. Comme d'Artagnan n'avait pas de temps à perdre, vu l'audience du roi qui était fixée à midi, il jeta les yeux autour de lui, et voyant que la rue était déserte: 

«Ma foi, dit-il à son adversaire, il est bien heureux pour vous, quoique vous vous appeliez Bernajoux, de n'avoir affaire qu'à un apprenti mousquetaire; cependant, soyez tranquille, je ferai de mon mieux. En garde! 

\speak  Mais, dit celui que d'Artagnan provoquait ainsi, il me semble que le lieu est assez mal choisi, et que nous serions mieux derrière l'abbaye de Saint-Germain ou dans le Pré-aux-Clercs. 

\speak  Ce que vous dites est plein de sens, répondit d'Artagnan; malheureusement j'ai peu de temps à moi, ayant un rendez-vous à midi juste. En garde donc, monsieur, en garde!» 

Bernajoux n'était pas homme à se faire répéter deux fois un pareil compliment. Au même instant son épée brilla à sa main, et il fondit sur son adversaire que, grâce à sa grande jeunesse, il espérait intimider. 

Mais d'Artagnan avait fait la veille son apprentissage, et tout frais émoulu de sa victoire, tout gonflé de sa future faveur, il était résolu à ne pas reculer d'un pas: aussi les deux fers se trouvèrent-ils engagés jusqu'à la garde, et comme d'Artagnan tenait ferme à sa place, ce fut son adversaire qui fit un pas de retraite. Mais d'Artagnan saisit le moment où, dans ce mouvement, le fer de Bernajoux déviait de la ligne, il dégagea, se fendit et toucha son adversaire à l'épaule. Aussitôt d'Artagnan, à son tour, fit un pas de retraite et releva son épée; mais Bernajoux lui cria que ce n'était rien, et se fendant aveuglément sur lui, il s'enferra de lui-même. Cependant, comme il ne tombait pas, comme il ne se déclarait pas vaincu, mais que seulement il rompait du côté de l'hôtel de M. de La Trémouille au service duquel il avait un parent, d'Artagnan, ignorant lui-même la gravité de la dernière blessure que son adversaire avait reçue, le pressait vivement, et sans doute allait l'achever d'un troisième coup, lorsque la rumeur qui s'élevait de la rue s'étant étendue jusqu'au jeu de paume, deux des amis du garde, qui l'avaient entendu échanger quelques paroles avec d'Artagnan et qui l'avaient vu sortir à la suite de ces paroles, se précipitèrent l'épée à la main hors du tripot et tombèrent sur le vainqueur. Mais aussitôt Athos, Porthos et Aramis parurent à leur tour et au moment où les deux gardes attaquaient leur jeune camarade, les forcèrent à se retourner. En ce moment Bernajoux tomba; et comme les gardes étaient seulement deux contre quatre, ils se mirent à crier: «À nous, l'hôtel de La Trémouille!» À ces cris, tout ce qui était dans l'hôtel sortit, se ruant sur les quatre compagnons, qui de leur côté se mirent à crier: «À nous, mousquetaires!» 

Ce cri était ordinairement entendu; car on savait les mousquetaires ennemis de Son Éminence, et on les aimait pour la haine qu'ils portaient au cardinal. Aussi les gardes des autres compagnies que celles appartenant au duc Rouge, comme l'avait appelé Aramis, prenaient-ils en général parti dans ces sortes de querelles pour les mousquetaires du roi. De trois gardes de la compagnie de M. des Essarts qui passaient, deux vinrent donc en aide aux quatre compagnons, tandis que l'autre courait à l'hôtel de M. de Tréville, criant: «À nous, mousquetaires, à nous!» Comme d'habitude, l'hôtel de M. de Tréville était plein de soldats de cette arme, qui accoururent au secours de leurs camarades; la mêlée devint générale, mais la force était aux mousquetaires: les gardes du cardinal et les gens de M. de La Trémouille se retirèrent dans l'hôtel, dont ils fermèrent les portes assez à temps pour empêcher que leurs ennemis n'y fissent irruption en même temps qu'eux. Quant au blessé, il y avait été tout d'abord transporté et, comme nous l'avons dit, en fort mauvais état. 

L'agitation était à son comble parmi les mousquetaires et leurs alliés, et l'on délibérait déjà si, pour punir l'insolence qu'avaient eue les domestiques de M. de La Trémouille de faire une sortie sur les mousquetaires du roi, on ne mettrait pas le feu à son hôtel. La proposition en avait été faite et accueillie avec enthousiasme, lorsque heureusement onze heures sonnèrent; d'Artagnan et ses compagnons se souvinrent de leur audience, et comme ils eussent regretté que l'on fît un si beau coup sans eux, ils parvinrent à calmer les têtes. On se contenta donc de jeter quelques pavés dans les portes, mais les portes résistèrent: alors on se lassa; d'ailleurs ceux qui devaient être regardés comme les chefs de l'entreprise avaient depuis un instant quitté le groupe et s'acheminaient vers l'hôtel de M. de Tréville, qui les attendait, déjà au courant de cette algarade. 

«Vite, au Louvre, dit-il, au Louvre sans perdre un instant, et tâchons de voir le roi avant qu'il soit prévenu par le cardinal; nous lui raconterons la chose comme une suite de l'affaire d'hier, et les deux passeront ensemble.» 

M. de Tréville, accompagné des quatre jeunes gens, s'achemina donc vers le Louvre; mais, au grand étonnement du capitaine des mousquetaires, on lui annonça que le roi était allé courre le cerf dans la forêt de Saint-Germain. M. de Tréville se fit répéter deux fois cette nouvelle, et à chaque fois ses compagnons virent son visage se rembrunir. 

«Est-ce que Sa Majesté, demanda-t-il, avait dès hier le projet de faire cette chasse? 

\speak  Non, Votre Excellence, répondit le valet de chambre, c'est le grand veneur qui est venu lui annoncer ce matin qu'on avait détourné cette nuit un cerf à son intention. Il a d'abord répondu qu'il n'irait pas, puis il n'a pas su résister au plaisir que lui promettait cette chasse, et après le dîner il est parti. 

\speak  Et le roi a-t-il vu le cardinal? demanda M. de Tréville. 

\speak  Selon toute probabilité, répondit le valet de chambre, car j'ai vu ce matin les chevaux au carrosse de Son Éminence, j'ai demandé où elle allait, et l'on m'a répondu: “À Saint-Germain.” 

\speak  Nous sommes prévenus, dit M. de Tréville, messieurs, je verrai le roi ce soir; mais quant à vous, je ne vous conseille pas de vous y hasarder.» 

L'avis était trop raisonnable et surtout venait d'un homme qui connaissait trop bien le roi, pour que les quatre jeunes gens essayassent de le combattre. M. de Tréville les invita donc à rentrer chacun chez eux et à attendre de ses nouvelles. 

En entrant à son hôtel, M. de Tréville songea qu'il fallait prendre date en portant plainte le premier. Il envoya un de ses domestiques chez M. de La Trémouille avec une lettre dans laquelle il le priait de mettre hors de chez lui le garde de M. le cardinal, et de réprimander ses gens de l'audace qu'ils avaient eue de faire leur sortie contre les mousquetaires. Mais M. de La Trémouille, déjà prévenu par son écuyer dont, comme on le sait, Bernajoux était le parent, lui fit répondre que ce n'était ni à M. de Tréville, ni à ses mousquetaires de se plaindre, mais bien au contraire à lui dont les mousquetaires avaient chargé les gens et voulu brûler l'hôtel. Or, comme le débat entre ces deux seigneurs eût pu durer longtemps, chacun devant naturellement s'entêter dans son opinion, M. de Tréville avisa un expédient qui avait pour but de tout terminer: c'était d'aller trouver lui-même M. de La Trémouille. 

Il se rendit donc aussitôt à son hôtel et se fit annoncer. 

Les deux seigneurs se saluèrent poliment, car, s'il n'y avait pas amitié entre eux, il y avait du moins estime. Tous deux étaient gens de cœur et d'honneur; et comme M. de La Trémouille, protestant, et voyant rarement le roi, n'était d'aucun parti, il n'apportait en général dans ses relations sociales aucune prévention. Cette fois, néanmoins, son accueil quoique poli fut plus froid que d'habitude. 

«Monsieur, dit M. de Tréville, nous croyons avoir à nous plaindre chacun l'un de l'autre, et je suis venu moi-même pour que nous tirions de compagnie cette affaire au clair. 

\speak  Volontiers, répondit M. de La Trémouille; mais je vous préviens que je suis bien renseigné, et tout le tort est à vos mousquetaires. 

\speak  Vous êtes un homme trop juste et trop raisonnable, monsieur, dit M. de Tréville, pour ne pas accepter la proposition que je vais faire. 

\speak  Faites, monsieur, j'écoute. 

\speak  Comment se trouve M. Bernajoux, le parent de votre écuyer? 

\speak  Mais, monsieur, fort mal. Outre le coup d'épée qu'il a reçu dans le bras, et qui n'est pas autrement dangereux, il en a encore ramassé un autre qui lui a traversé le poumon, de sorte que le médecin en dit de pauvres choses. 

\speak  Mais le blessé a-t-il conservé sa connaissance? 

\speak  Parfaitement. 

\speak  Parle-t-il? 

\speak  Avec difficulté, mais il parle. 

\speak  Eh bien, monsieur! rendons-nous près de lui; adjurons-le, au nom du Dieu devant lequel il va être appelé peut-être, de dire la vérité. Je le prends pour juge dans sa propre cause, monsieur, et ce qu'il dira je le croirai.» 

M. de La Trémouille réfléchit un instant, puis, comme il était difficile de faire une proposition plus raisonnable, il accepta. 

Tous deux descendirent dans la chambre où était le blessé. Celui-ci, en voyant entrer ces deux nobles seigneurs qui venaient lui faire visite, essaya de se relever sur son lit, mais il était trop faible, et, épuisé par l'effort qu'il avait fait, il retomba presque sans connaissance. 

M. de La Trémouille s'approcha de lui et lui fit respirer des sels qui le rappelèrent à la vie. Alors M. de Tréville, ne voulant pas qu'on pût l'accuser d'avoir influencé le malade, invita M. de La Trémouille à l'interroger lui-même. 

Ce qu'avait prévu M. de Tréville arriva. Placé entre la vie et la mort comme l'était Bernajoux, il n'eut pas même l'idée de taire un instant la vérité, et il raconta aux deux seigneurs les choses exactement, telles qu'elles s'étaient passées. 

C'était tout ce que voulait M. de Tréville; il souhaita à Bernajoux une prompte convalescence, prit congé de M. de La Trémouille, rentra à son hôtel et fit aussitôt prévenir les quatre amis qu'il les attendait à dîner. 

M. de Tréville recevait fort bonne compagnie, toute anticardinaliste d'ailleurs. On comprend donc que la conversation roula pendant tout le dîner sur les deux échecs que venaient d'éprouver les gardes de Son Éminence. Or, comme d'Artagnan avait été le héros de ces deux journées, ce fut sur lui que tombèrent toutes les félicitations, qu'Athos, Porthos et Aramis lui abandonnèrent non seulement en bons camarades, mais en hommes qui avaient eu assez souvent leur tour pour qu'ils lui laissassent le sien. 

Vers six heures, M. de Tréville annonça qu'il était tenu d'aller au Louvre; mais comme l'heure de l'audience accordée par Sa Majesté était passée, au lieu de réclamer l'entrée par le petit escalier, il se plaça avec les quatre jeunes gens dans l'antichambre. Le roi n'était pas encore revenu de la chasse. Nos jeunes gens attendaient depuis une demi-heure à peine, mêlés à la foule des courtisans, lorsque toutes les portes s'ouvrirent et qu'on annonça Sa Majesté. 

À cette annonce, d'Artagnan se sentit frémir jusqu'à la moelle des os. L'instant qui allait suivre devait, selon toute probabilité, décider du reste de sa vie. Aussi ses yeux se fixèrent-ils avec angoisse sur la porte par laquelle devait entrer le roi. 

Louis XIII parut, marchant le premier; il était en costume de chasse, encore tout poudreux, ayant de grandes bottes et tenant un fouet à la main. Au premier coup d'œil, d'Artagnan jugea que l'esprit du roi était à l'orage. 

Cette disposition, toute visible qu'elle était chez Sa Majesté, n'empêcha pas les courtisans de se ranger sur son passage: dans les antichambres royales, mieux vaut encore être vu d'un œil irrité que de n'être pas vu du tout. Les trois mousquetaires n'hésitèrent donc pas, et firent un pas en avant, tandis que d'Artagnan au contraire restait caché derrière eux; mais quoique le roi connût personnellement Athos, Porthos et Aramis, il passa devant eux sans les regarder, sans leur parler et comme s'il ne les avait jamais vus. Quant à M. de Tréville, lorsque les yeux du roi s'arrêtèrent un instant sur lui, il soutint ce regard avec tant de fermeté, que ce fut le roi qui détourna la vue; après quoi, tout en grommelant, Sa Majesté rentra dans son appartement. 

«Les affaires vont mal, dit Athos en souriant, et nous ne serons pas encore fait chevaliers de l'ordre cette fois-ci. 

\speak  Attendez ici dix minutes, dit M. de Tréville; et si au bout de dix minutes vous ne me voyez pas sortir, retournez à mon hôtel: car il sera inutile que vous m'attendiez plus longtemps.» 

Les quatre jeunes gens attendirent dix minutes, un quart d'heure, vingt minutes; et voyant que M. de Tréville ne reparaissait point, ils sortirent fort inquiets de ce qui allait arriver. 

M. de Tréville était entré hardiment dans le cabinet du roi, et avait trouvé Sa Majesté de très méchante humeur, assise sur un fauteuil et battant ses bottes du manche de son fouet, ce qui ne l'avait pas empêché de lui demander avec le plus grand flegme des nouvelles de sa santé. 

«Mauvaise, monsieur, mauvaise, répondit le roi, je m'ennuie.» 

C'était en effet la pire maladie de Louis XIII, qui souvent prenait un de ses courtisans, l'attirait à une fenêtre et lui disait: «Monsieur un tel, ennuyons-nous ensemble.» 

«Comment! Votre Majesté s'ennuie! dit M. de Tréville. N'a-t-elle donc pas pris aujourd'hui le plaisir de la chasse? 

\speak  Beau plaisir, monsieur! Tout dégénère, sur mon âme, et je ne sais si c'est le gibier qui n'a plus de voie ou les chiens qui n'ont plus de nez. Nous lançons un cerf dix cors, nous le courons six heures, et quand il est prêt à tenir, quand Saint-Simon met déjà le cor à sa bouche pour sonner l'hallali, crac! toute la meute prend le change et s'emporte sur un daguet. Vous verrez que je serai obligé de renoncer à la chasse à courre comme j'ai renoncé à la chasse au vol. Ah! je suis un roi bien malheureux, monsieur de Tréville! je n'avais plus qu'un gerfaut, et il est mort avant-hier. 

\speak  En effet, Sire, je comprends votre désespoir, et le malheur est grand; mais il vous reste encore, ce me semble, bon nombre de faucons, d'éperviers et de tiercelets. 

\speak  Et pas un homme pour les instruire, les fauconniers s'en vont, il n'y a plus que moi qui connaisse l'art de la vénerie. Après moi tout sera dit, et l'on chassera avec des traquenards, des pièges, des trappes. Si j'avais le temps encore de former des élèves! mais oui, M. le cardinal est là qui ne me laisse pas un instant de repos, qui me parle de l'Espagne, qui me parle de l'Autriche, qui me parle de l'Angleterre! Ah! à propos de M. le cardinal, monsieur de Tréville, je suis mécontent de vous.» 

M. de Tréville attendait le roi à cette chute. Il connaissait le roi de longue main; il avait compris que toutes ses plaintes n'étaient qu'une préface, une espèce d'excitation pour s'encourager lui-même, et que c'était où il était arrivé enfin qu'il en voulait venir. 

«Et en quoi ai-je été assez malheureux pour déplaire à Votre Majesté? demanda M. de Tréville en feignant le plus profond étonnement. 

\speak  Est-ce ainsi que vous faites votre charge, monsieur? continua le roi sans répondre directement à la question de M. de Tréville; est-ce pour cela que je vous ai nommé capitaine de mes mousquetaires, que ceux-ci assassinent un homme, émeuvent tout un quartier et veulent brûler Paris sans que vous en disiez un mot? Mais, au reste, continua le roi, sans doute que je me hâte de vous accuser, sans doute que les perturbateurs sont en prison et que vous venez m'annoncer que justice est faite. 

\speak  Sire, répondit tranquillement M. de Tréville, je viens vous la demander au contraire. 

\speak  Et contre qui? s'écria le roi. 

\speak  Contre les calomniateurs, dit M. de Tréville. 

\speak  Ah! voilà qui est nouveau, reprit le roi. N'allez-vous pas dire que vos trois mousquetaires damnés, Athos, Porthos et Aramis et votre cadet de Béarn, ne se sont pas jetés comme des furieux sur le pauvre Bernajoux, et ne l'ont pas maltraité de telle façon qu'il est probable qu'il est en train de trépasser à cette heure! N'allez-vous pas dire qu'ensuite ils n'ont pas fait le siège de l'hôtel du duc de La Trémouille, et qu'ils n'ont point voulu le brûler! ce qui n'aurait peut-être pas été un très grand malheur en temps de guerre, vu que c'est un nid de huguenots, mais ce qui, en temps de paix, est un fâcheux exemple. Dites, n'allez-vous pas nier tout cela? 

\speak  Et qui vous a fait ce beau récit, Sire? demanda tranquillement M. de Tréville. 

\speak  Qui m'a fait ce beau récit, monsieur! et qui voulez-vous que ce soit, si ce n'est celui qui veille quand je dors, qui travaille quand je m'amuse, qui mène tout au-dedans et au-dehors du royaume, en France comme en Europe? 

\speak  Sa Majesté veut parler de Dieu, sans doute, dit M. de Tréville, car je ne connais que Dieu qui soit si fort au-dessus de Sa Majesté. 

\speak  Non monsieur; je veux parler du soutien de l'État, de mon seul serviteur, de mon seul ami, de M. le cardinal. 

\speak  Son Éminence n'est pas Sa Sainteté, Sire. 

\speak  Qu'entendez-vous par là, monsieur? 

\speak  Qu'il n'y a que le pape qui soit infaillible, et que cette infaillibilité ne s'étend pas aux cardinaux. 

\speak  Vous voulez dire qu'il me trompe, vous voulez dire qu'il me trahit. Vous l'accusez alors. Voyons, dites, avouez franchement que vous l'accusez. 

\speak  Non, Sire; mais je dis qu'il se trompe lui-même, je dis qu'il a été mal renseigné; je dis qu'il a eu hâte d'accuser les mousquetaires de Votre Majesté, pour lesquels il est injuste, et qu'il n'a pas été puiser ses renseignements aux bonnes sources. 

\speak  L'accusation vient de M. de La Trémouille, du duc lui-même. Que répondrez-vous à cela? 

\speak  Je pourrais répondre, Sire, qu'il est trop intéressé dans la question pour être un témoin bien impartial; mais loin de là, Sire, je connais le duc pour un loyal gentilhomme, et je m'en rapporterai à lui, mais à une condition, Sire. 

\speak  Laquelle? 

\speak  C'est que Votre Majesté le fera venir, l'interrogera, mais elle-même, en tête-à-tête, sans témoins, et que je reverrai Votre Majesté aussitôt qu'elle aura reçu le duc. 

\speak  Oui-da! fit le roi, et vous vous en rapporterez à ce que dira M. de La Trémouille? 

\speak  Oui, Sire. 

\speak  Vous accepterez son jugement? 

\speak  Sans doute. 

\speak  Et vous vous soumettrez aux réparations qu'il exigera? 

\speak  Parfaitement. 

\speak  La Chesnaye! fit le roi. La Chesnaye!» 

Le valet de chambre de confiance de Louis XIII, qui se tenait toujours à la porte, entra. 

«La Chesnaye, dit le roi, qu'on aille à l'instant même me quérir M. de La Trémouille; je veux lui parler ce soir. 

\speak  Votre Majesté me donne sa parole qu'elle ne verra personne entre M. de La Trémouille et moi? 

\speak  Personne, foi de gentilhomme. 

\speak  À demain, Sire, alors. 

\speak  À demain, monsieur. 

\speak  À quelle heure, s'il plaît à Votre Majesté? 

\speak  À l'heure que vous voudrez. 

\speak  Mais, en venant par trop matin, je crains de réveiller votre Majesté. 

\speak  Me réveiller? Est-ce que je dors? Je ne dors plus, monsieur; je rêve quelquefois, voilà tout. Venez donc d'aussi bon matin que vous voudrez, à sept heures; mais gare à vous, si vos mousquetaires sont coupables! 

\speak  Si mes mousquetaires sont coupables, Sire, les coupables seront remis aux mains de Votre Majesté, qui ordonnera d'eux selon son bon plaisir. Votre Majesté exige-t-elle quelque chose de plus? qu'elle parle, je suis prêt à lui obéir. 

\speak  Non, monsieur, non, et ce n'est pas sans raison qu'on m'a appelé Louis le Juste. À demain donc, monsieur, à demain. 

\speak  Dieu garde jusque-là Votre Majesté!» 

Si peu que dormit le roi, M. de Tréville dormit plus mal encore; il avait fait prévenir dès le soir même ses trois mousquetaires et leur compagnon de se trouver chez lui à six heures et demie du matin. Il les emmena avec lui sans rien leur affirmer, sans leur rien promettre, et ne leur cachant pas que leur faveur et même la sienne tenaient à un coup de dés. 

Arrivé au bas du petit escalier, il les fit attendre. Si le roi était toujours irrité contre eux, ils s'éloigneraient sans être vus; si le roi consentait à les recevoir, on n'aurait qu'à les faire appeler. 

En arrivant dans l'antichambre particulière du roi, M. de Tréville trouva La Chesnaye, qui lui apprit qu'on n'avait pas rencontré le duc de La Trémouille la veille au soir à son hôtel, qu'il était rentré trop tard pour se présenter au Louvre, qu'il venait seulement d'arriver, et qu'il était à cette heure chez le roi. 

Cette circonstance plut beaucoup à M. de Tréville, qui, de cette façon, fut certain qu'aucune suggestion étrangère ne se glisserait entre la déposition de M. de La Trémouille et lui. 

En effet, dix minutes s'étaient à peine écoulées, que la porte du cabinet s'ouvrit et que M. de Tréville en vit sortir le duc de La Trémouille, lequel vint à lui et lui dit: 

«Monsieur de Tréville, Sa Majesté vient de m'envoyer quérir pour savoir comment les choses s'étaient passées hier matin à mon hôtel. Je lui ai dit la vérité, c'est-à-dire que la faute était à mes gens, et que j'étais prêt à vous en faire mes excuses. Puisque je vous rencontre, veuillez les recevoir, et me tenir toujours pour un de vos amis. 

\speak  Monsieur le duc, dit M. de Tréville, j'étais si plein de confiance dans votre loyauté, que je n'avais pas voulu près de Sa Majesté d'autre défenseur que vous-même. Je vois que je ne m'étais pas abusé, et je vous remercie de ce qu'il y a encore en France un homme de qui on puisse dire sans se tromper ce que j'ai dit de vous. 

\speak  C'est bien, c'est bien! dit le roi qui avait écouté tous ces compliments entre les deux portes; seulement, dites-lui, Tréville, puisqu'il se prétend un de vos amis, que moi aussi je voudrais être des siens, mais qu'il me néglige; qu'il y a tantôt trois ans que je ne l'ai vu, et que je ne le vois que quand je l'envoie chercher. Dites-lui tout cela de ma part, car ce sont de ces choses qu'un roi ne peut dire lui-même. 

\speak  Merci, Sire, merci, dit le duc; mais que Votre Majesté croie bien que ce ne sont pas ceux, je ne dis point cela pour M. de Tréville, que ce ne sont point ceux qu'elle voit à toute heure du jour qui lui sont le plus dévoués. 

\speak  Ah! vous avez entendu ce que j'ai dit; tant mieux, duc, tant mieux, dit le roi en s'avançant jusque sur la porte. Ah! c'est vous, Tréville! où sont vos mousquetaires? Je vous avais dit avant-hier de me les amener, pourquoi ne l'avez-vous pas fait? 

\speak  Ils sont en bas, Sire, et avec votre congé La Chesnaye va leur dire de monter. 

\speak  Oui, oui, qu'ils viennent tout de suite; il va être huit heures, et à neuf heures j'attends une visite. Allez, monsieur le duc, et revenez surtout. Entrez, Tréville.» 

Le duc salua et sortit. Au moment où il ouvrait la porte, les trois mousquetaires et d'Artagnan, conduits par La Chesnaye, apparaissaient au haut de l'escalier. 

«Venez, mes braves, dit le roi, venez; j'ai à vous gronder.» 

Les mousquetaires s'approchèrent en s'inclinant; d'Artagnan les suivait par-derrière. 

«Comment diable! continua le roi; à vous quatre, sept gardes de Son Éminence mis hors de combat en deux jours! C'est trop, messieurs, c'est trop. À ce compte-là, Son Éminence serait forcée de renouveler sa compagnie dans trois semaines, et moi de faire appliquer les édits dans toute leur rigueur. Un par hasard, je ne dis pas; mais sept en deux jours, je le répète, c'est trop, c'est beaucoup trop. 

\speak  Aussi, Sire, Votre Majesté voit qu'ils viennent tout contrits et tout repentants lui faire leurs excuses. 

\speak  Tout contrits et tout repentants! Hum! fit le roi, je ne me fie point à leurs faces hypocrites; il y a surtout là-bas une figure de Gascon. Venez ici, monsieur.» 

D'Artagnan, qui comprit que c'était à lui que le compliment s'adressait, s'approcha en prenant son air le plus désespéré. 

«Eh bien, que me disiez-vous donc que c'était un jeune homme? c'est un enfant, monsieur de Tréville, un véritable enfant! Et c'est celui-là qui a donné ce rude coup d'épée à Jussac? 

\speak  Et ces deux beaux coups d'épée à Bernajoux. 

\speak  Véritablement! 

\speak  Sans compter, dit Athos, que s'il ne m'avait pas tiré des mains de Biscarat, je n'aurais très certainement pas l'honneur de faire en ce moment-ci ma très humble révérence à Votre Majesté. 

\speak  Mais c'est donc un véritable démon que ce Béarnais, ventre-saint-gris! monsieur de Tréville comme eût dit le roi mon père. À ce métier-là, on doit trouer force pourpoints et briser force épées. Or les Gascons sont toujours pauvres, n'est-ce pas? 

\speak  Sire, je dois dire qu'on n'a pas encore trouvé des mines d'or dans leurs montagnes, quoique le Seigneur dût bien ce miracle en récompense de la manière dont ils ont soutenu les prétentions du roi votre père. 

\speak  Ce qui veut dire que ce sont les Gascons qui m'ont fait roi moi-même, n'est-ce pas, Tréville, puisque je suis le fils de mon père? Eh bien, à la bonne heure, je ne dis pas non. La Chesnaye, allez voir si, en fouillant dans toutes mes poches, vous trouverez quarante pistoles; et si vous les trouvez, apportez-les-moi. Et maintenant, voyons, jeune homme, la main sur la conscience, comment cela s'est-il passé?» 

D'Artagnan raconta l'aventure de la veille dans tous ses détails: comment, n'ayant pas pu dormir de la joie qu'il éprouvait à voir Sa Majesté, il était arrivé chez ses amis trois heures avant l'heure de l'audience; comment ils étaient allés ensemble au tripot, et comment, sur la crainte qu'il avait manifestée de recevoir une balle au visage, il avait été raillé par Bernajoux, lequel avait failli payer cette raillerie de la perte de la vie, et M. de La Trémouille, qui n'y était pour rien, de la perte de son hôtel. 

«C'est bien cela, murmurait le roi; oui, c'est ainsi que le duc m'a raconté la chose. Pauvre cardinal! sept hommes en deux jours, et de ses plus chers; mais c'est assez comme cela, messieurs, entendez-vous! c'est assez: vous avez pris votre revanche de la rue Férou, et au-delà; vous devez être satisfaits. 

\speak  Si Votre Majesté l'est, dit Tréville, nous le sommes. 

\speak  Oui, je le suis, ajouta le roi en prenant une poignée d'or de la main de La Chesnaye, et la mettant dans celle de d'Artagnan. Voici, dit-il, une preuve de ma satisfaction.» 

À cette époque, les idées de fierté qui sont de mise de nos jours n'étaient point encore de mode. Un gentilhomme recevait de la main à la main de l'argent du roi, et n'en était pas le moins du monde humilié. D'Artagnan mit donc les quarante pistoles dans sa poche sans faire aucune façon, et en remerciant tout au contraire grandement Sa Majesté. 

«Là, dit le roi en regardant sa pendule, là, et maintenant qu'il est huit heures et demie, retirez-vous; car, je vous l'ai dit, j'attends quelqu'un à neuf heures. Merci de votre dévouement, messieurs. J'y puis compter, n'est-ce pas? 

\speak  Oh! Sire, s'écrièrent d'une même voix les quatre compagnons, nous nous ferions couper en morceaux pour Votre Majesté. 

\speak  Bien, bien; mais restez entiers: cela vaut mieux, et vous me serez plus utiles. Tréville, ajouta le roi à demi-voix pendant que les autres se retiraient, comme vous n'avez pas de place dans les mousquetaires et que d'ailleurs pour entrer dans ce corps nous avons décidé qu'il fallait faire un noviciat, placez ce jeune homme dans la compagnie des gardes de M. des Essarts, votre beau-frère. Ah! pardieu! Tréville, je me réjouis de la grimace que va faire le cardinal: il sera furieux, mais cela m'est égal; je suis dans mon droit.» 

Et le roi salua de la main Tréville, qui sortit et s'en vint rejoindre ses mousquetaires, qu'il trouva partageant avec d'Artagnan les quarante pistoles. 

Et le cardinal, comme l'avait dit Sa Majesté, fut effectivement furieux, si furieux que pendant huit jours il abandonna le jeu du roi, ce qui n'empêchait pas le roi de lui faire la plus charmante mine du monde, et toutes les fois qu'il le rencontrait de lui demander de sa voix la plus caressante: 

«Eh bien, monsieur le cardinal, comment vont ce pauvre Bernajoux et ce pauvre Jussac, qui sont à vous?» 
\include{chapters/07.tex}
%!TeX root=../musketeersfr.tex 

\chapter{Une Intrigue De Cœur}

\lettrine{C}{ependant} les quarante pistoles du roi Louis XIII, ainsi que toutes les choses de ce monde, après avoir eu un commencement avaient eu une fin, et depuis cette fin nos quatre compagnons étaient tombés dans la gêne. D'abord Athos avait soutenu pendant quelque temps l'association de ses propres deniers. Porthos lui avait succédé, et, grâce à une de ces disparitions auxquelles on était habitué, il avait pendant près de quinze jours encore subvenu aux besoins de tout le monde; enfin était arrivé le tour d'Aramis, qui s'était exécuté de bonne grâce, et qui était parvenu, disait-il, en vendant ses livres de théologie, à se procurer quelques pistoles. 

On eut alors, comme d'habitude, recours à M. de Tréville, qui fit quelques avances sur la solde; mais ces avances ne pouvaient conduire bien loin trois mousquetaires qui avaient déjà force comptes arriérés, et un garde qui n'en avait pas encore. 

Enfin, quand on vit qu'on allait manquer tout à fait, on rassembla par un dernier effort huit ou dix pistoles que Porthos joua. Malheureusement, il était dans une mauvaise veine: il perdit tout, plus vingt-cinq pistoles sur parole. 

Alors la gêne devint de la détresse, on vit les affamés suivis de leurs laquais courir les quais et les corps de garde, ramassant chez leurs amis du dehors tous les dîners qu'ils purent trouver; car, suivant l'avis d'Aramis, on devait dans la prospérité semer des repas à droite et à gauche pour en récolter quelques-uns dans la disgrâce. 

Athos fut invité quatre fois et mena chaque fois ses amis avec leurs laquais. Porthos eut six occasions et en fit également jouir ses camarades; Aramis en eut huit. C'était un homme, comme on a déjà pu s'en apercevoir, qui faisait peu de bruit et beaucoup de besogne. 

Quant à d'Artagnan, qui ne connaissait encore personne dans la capitale, il ne trouva qu'un déjeuner de chocolat chez un prêtre de son pays, et un dîner chez un cornette des gardes. Il mena son armée chez le prêtre, auquel on dévora sa provision de deux mois, et chez le cornette, qui fit des merveilles; mais, comme le disait Planchet, on ne mange toujours qu'une fois, même quand on mange beaucoup. 

D'Artagnan se trouva donc assez humilié de n'avoir eu qu'un repas et demi, car le déjeuner chez le prêtre ne pouvait compter que pour un demi-repas, à offrir à ses compagnons en échange des festins que s'étaient procurés Athos, Porthos et Aramis. Il se croyait à charge à la société, oubliant dans sa bonne foi toute juvénile qu'il avait nourri cette société pendant un mois, et son esprit préoccupé se mit à travailler activement. Il réfléchit que cette coalition de quatre hommes jeunes, braves, entreprenants et actifs devait avoir un autre but que des promenades déhanchées, des leçons d'escrime et des lazzi plus ou moins spirituels. 

En effet, quatre hommes comme eux, quatre hommes dévoués les uns aux autres depuis la bourse jusqu'à la vie, quatre hommes se soutenant toujours, ne reculant jamais, exécutant isolément ou ensemble les résolutions prises en commun; quatre bras menaçant les quatre points cardinaux ou se tournant vers un seul point, devaient inévitablement, soit souterrainement, soit au jour, soit par la mine, soit par la tranchée, soit par la ruse, soit par la force, s'ouvrir un chemin vers le but qu'ils voulaient atteindre, si bien défendu ou si éloigné qu'il fût. La seule chose qui étonnât d'Artagnan, c'est que ses compagnons n'eussent point songé à cela. 

Il y songeait, lui, et sérieusement même, se creusant la cervelle pour trouver une direction à cette force unique quatre fois multipliée avec laquelle il ne doutait pas que, comme avec le levier que cherchait Archimède, on ne parvînt à soulever le monde, --- lorsque l'on frappa doucement à la porte. D'Artagnan réveilla Planchet et lui ordonna d'aller ouvrir. 

Que de cette phrase: d'Artagnan réveilla Planchet, le lecteur n'aille pas augurer qu'il faisait nuit ou que le jour n'était point encore venu. Non! quatre heures venaient de sonner. Planchet, deux heures auparavant, était venu demander à dîner à son maître, lequel lui avait répondu par le proverbe: «Qui dort dîne.» Et Planchet dînait en dormant. 

Un homme fut introduit, de mine assez simple et qui avait l'air d'un bourgeois. 

Planchet, pour son dessert, eût bien voulu entendre la conversation; mais le bourgeois déclara à d'Artagnan que ce qu'il avait à lui dire étant important et confidentiel, il désirait demeurer en tête-à-tête avec lui. 

D'Artagnan congédia Planchet et fit asseoir son visiteur. 

Il y eut un moment de silence pendant lequel les deux hommes se regardèrent comme pour faire une connaissance préalable, après quoi d'Artagnan s'inclina en signe qu'il écoutait. 

«J'ai entendu parler de M. d'Artagnan comme d'un jeune homme fort brave, dit le bourgeois, et cette réputation dont il jouit à juste titre m'a décidé à lui confier un secret. 

\speak  Parlez, monsieur, parlez», dit d'Artagnan, qui d'instinct flaira quelque chose d'avantageux. 

Le bourgeois fit une nouvelle pause et continua: 

«J'ai ma femme qui est lingère chez la reine, monsieur, et qui ne manque ni de sagesse, ni de beauté. On me l'a fait épouser voilà bientôt trois ans, quoiqu'elle n'eût qu'un petit avoir, parce que M. de La Porte, le portemanteau de la reine, est son parrain et la protège\dots 

\speak  Eh bien, monsieur? demanda d'Artagnan. 

\speak  Eh bien, reprit le bourgeois, eh bien, monsieur, ma femme a été enlevée hier matin, comme elle sortait de sa chambre de travail. 

\speak  Et par qui votre femme a-t-elle été enlevée? 

\speak  Je n'en sais rien sûrement, monsieur, mais je soupçonne quelqu'un. 

\speak  Et quelle est cette personne que vous soupçonnez? 

\speak  Un homme qui la poursuivait depuis longtemps. 

\speak  Diable! 

\speak  Mais voulez-vous que je vous dise, monsieur, continua le bourgeois, je suis convaincu, moi, qu'il y a moins d'amour que de politique dans tout cela. 

\speak  Moins d'amour que de politique, reprit d'Artagnan d'un air fort réfléchi, et que soupçonnez-vous? 

\speak  Je ne sais pas si je devrais vous dire ce que je soupçonne\dots 

\speak  Monsieur, je vous ferai observer que je ne vous demande absolument rien, moi. C'est vous qui êtes venu. C'est vous qui m'avez dit que vous aviez un secret à me confier. Faites donc à votre guise, il est encore temps de vous retirer. 

\speak  Non, monsieur, non; vous m'avez l'air d'un honnête jeune homme, et j'aurai confiance en vous. Je crois donc que ce n'est pas à cause de ses amours que ma femme a été arrêtée, mais à cause de celles d'une plus grande dame qu'elle. 

\speak  Ah! ah! serait-ce à cause des amours de Mme de Bois-Tracy? fit d'Artagnan, qui voulut avoir l'air, vis-à-vis de son bourgeois, d'être au courant des affaires de la cour. 

\speak  Plus haut, monsieur, plus haut. 

\speak  De Mme d'Aiguillon? 

\speak  Plus haut encore. 

\speak  De Mme de Chevreuse? 

\speak  Plus haut, beaucoup plus haut! 

\speak  De la\dots d'Artagnan s'arrêta. 

\speak  Oui, monsieur, répondit si bas, qu'à peine si on put l'entendre, le bourgeois épouvanté. 

\speak  Et avec qui? 

\speak  Avec qui cela peut-il être, si ce n'est avec le duc de\dots 

\speak  Le duc de\dots 

\speak  Oui, monsieur! répondit le bourgeois, en donnant à sa voix une intonation plus sourde encore. 

\speak  Mais comment savez-vous tout cela, vous? 

\speak  Ah! comment je le sais? 

\speak  Oui, comment le savez-vous? Pas de demi-confidence, ou\dots vous comprenez. 

\speak  Je le sais par ma femme, monsieur, par ma femme elle-même. 

\speak  Qui le sait, elle, par qui? 

\speak  Par M. de La Porte. Ne vous ai-je pas dit qu'elle était la filleule de M. de La Porte, l'homme de confiance de la reine? Eh bien, M. de La Porte l'avait mise près de Sa Majesté pour que notre pauvre reine au moins eût quelqu'un à qui se fier, abandonnée comme elle l'est par le roi, espionnée comme elle l'est par le cardinal, trahie comme elle l'est par tous. 

\speak  Ah! ah! voilà qui se dessine, dit d'Artagnan. 

\speak  Or ma femme est venue il y a quatre jours, monsieur; une de ses conditions était qu'elle devait me venir voir deux fois la semaine; car, ainsi que j'ai eu l'honneur de vous le dire, ma femme m'aime beaucoup; ma femme est donc venue, et m'a confié que la reine, en ce moment-ci, avait de grandes craintes. 

\speak  Vraiment? 

\speak  Oui, M. le cardinal, à ce qu'il paraît, la poursuit et la persécute plus que jamais. Il ne peut pas lui pardonner l'histoire de la sarabande. Vous savez l'histoire de la sarabande? 

\speak  Pardieu, si je la sais! répondit d'Artagnan, qui ne savait rien du tout, mais qui voulait avoir l'air d'être au courant. 

\speak  De sorte que, maintenant, ce n'est plus de la haine, c'est de la vengeance. 

\speak  Vraiment? 

\speak  Et la reine croit\dots 

\speak  Eh bien, que croit la reine? 

\speak  Elle croit qu'on a écrit à M. le duc de Buckingham en son nom. 

\speak  Au nom de la reine? 

\speak  Oui, pour le faire venir à Paris, et une fois venu à Paris, pour l'attirer dans quelque piège. 

\speak  Diable! mais votre femme, mon cher monsieur, qu'a-t-elle à faire dans tout cela? 

\speak  On connaît son dévouement pour la reine, et l'on veut ou l'éloigner de sa maîtresse, ou l'intimider pour avoir les secrets de Sa Majesté, ou la séduire pour se servir d'elle comme d'un espion. 

\speak  C'est probable, dit d'Artagnan; mais l'homme qui l'a enlevée, le connaissez-vous? 

\speak  Je vous ai dit que je croyais le connaître. 

\speak  Son nom? 

\speak  Je ne le sais pas; ce que je sais seulement, c'est que c'est une créature du cardinal, son âme damnée. 

\speak  Mais vous l'avez vu? 

\speak  Oui, ma femme me l'a montré un jour. 

\speak  A-t-il un signalement auquel on puisse le reconnaître? 

\speak  Oh! certainement, c'est un seigneur de haute mine, poil noir, teint basané, œil perçant, dents blanches et une cicatrice à la tempe. 

\speak  Une cicatrice à la tempe! s'écria d'Artagnan, et avec cela dents blanches, œil perçant, teint basané, poil noir, et haute mine; c'est mon homme de Meung! 

\speak  C'est votre homme, dites-vous? 

\speak  Oui, oui; mais cela ne fait rien à la chose. Non, je me trompe, cela la simplifie beaucoup, au contraire: si votre homme est le mien, je ferai d'un coup deux vengeances, voilà tout; mais où rejoindre cet homme? 

\speak  Je n'en sais rien. 

\speak  Vous n'avez aucun renseignement sur sa demeure? 

\speak  Aucun; un jour que je reconduisais ma femme au Louvre, il en sortait comme elle allait y entrer, et elle me l'a fait voir. 

\speak  Diable! diable! murmura d'Artagnan, tout ceci est bien vague; par qui avez-vous su l'enlèvement de votre femme? 

\speak  Par M. de La Porte. 

\speak  Vous a-t-il donné quelque détail? 

\speak  Il n'en avait aucun. 

\speak  Et vous n'avez rien appris d'un autre côté? 

\speak  Si fait, j'ai reçu\dots 

\speak  Quoi? 

\speak  Mais je ne sais pas si je ne commets pas une grande imprudence? 

\speak  Vous revenez encore là-dessus; cependant je vous ferai observer que, cette fois, il est un peu tard pour reculer. 

\speak  Aussi je ne recule pas, mordieu! s'écria le bourgeois en jurant pour se monter la tête. D'ailleurs, foi de Bonacieux\dots 

\speak  Vous vous appelez Bonacieux? interrompit d'Artagnan. 

\speak  Oui, c'est mon nom. 

\speak  Vous disiez donc: foi de Bonacieux! pardon si je vous ai interrompu; mais il me semblait que ce nom ne m'était pas inconnu. 

\speak  C'est possible, monsieur. Je suis votre propriétaire. 

\speak  Ah! ah! fit d'Artagnan en se soulevant à demi et en saluant, vous êtes mon propriétaire? 

\speak  Oui, monsieur, oui. Et comme depuis trois mois que vous êtes chez moi, et que distrait sans doute par vos grandes occupations vous avez oublié de me payer mon loyer; comme, dis-je, je ne vous ai pas tourmenté un seul instant, j'ai pensé que vous auriez égard à ma délicatesse. 

\speak  Comment donc! mon cher monsieur Bonacieux, reprit d'Artagnan, croyez que je suis plein de reconnaissance pour un pareil procédé, et que, comme je vous l'ai dit, si je puis vous être bon à quelque chose\dots 

\speak  Je vous crois, monsieur, je vous crois, et comme j'allais vous le dire, foi de Bonacieux, j'ai confiance en vous. 

\speak  Achevez donc ce que vous avez commencé à me dire.» 

Le bourgeois tira un papier de sa poche, et le présenta à d'Artagnan. 

«Une lettre! fit le jeune homme. 

\speak  Que j'ai reçue ce matin.» 

D'Artagnan l'ouvrit, et comme le jour commençait à baisser, il s'approcha de la fenêtre. Le bourgeois le suivit. 

«Ne cherchez pas votre femme, lut d'Artagnan, elle vous sera rendue quand on n'aura plus besoin d'elle. Si vous faites une seule démarche pour la retrouver, vous êtes perdu.» 

«Voilà qui est positif, continua d'Artagnan; mais après tout, ce n'est qu'une menace. 

\speak  Oui, mais cette menace m'épouvante; moi, monsieur, je ne suis pas homme d'épée du tout, et j'ai peur de la Bastille. 

\speak  Hum! fit d'Artagnan; mais c'est que je ne me soucie pas plus de la Bastille que vous, moi. S'il ne s'agissait que d'un coup d'épée, passe encore. 

\speak  Cependant, monsieur, j'avais bien compté sur vous dans cette occasion. 

\speak  Oui? 

\speak  Vous voyant sans cesse entouré de mousquetaires à l'air fort superbe, et reconnaissant que ces mousquetaires étaient ceux de M. de Tréville, et par conséquent des ennemis du cardinal, j'avais pensé que vous et vos amis, tout en rendant justice à notre pauvre reine, seriez enchantés de jouer un mauvais tour à Son Éminence. 

\speak  Sans doute. 

\speak  Et puis j'avais pensé que, me devant trois mois de loyer dont je ne vous ai jamais parlé\dots 

\speak  Oui, oui, vous m'avez déjà donné cette raison, et je la trouve excellente. 

\speak  Comptant de plus, tant que vous me ferez l'honneur de rester chez moi, ne jamais vous parler de votre loyer à venir\dots 

\speak  Très bien. 

\speak  Et ajoutez à cela, si besoin est, comptant vous offrir une cinquantaine de pistoles si, contre toute probabilité, vous vous trouviez gêné en ce moment. 

\speak  À merveille; mais vous êtes donc riche, mon cher monsieur Bonacieux? 

\speak  Je suis à mon aise, monsieur, c'est le mot; j'ai amassé quelque chose comme deux ou trois mille écus de rente dans le commerce de la mercerie, et surtout en plaçant quelques fonds sur le dernier voyage du célèbre navigateur Jean Mocquet; de sorte que, vous comprenez, monsieur\dots Ah! mais\dots s'écria le bourgeois. 

\speak  Quoi? demanda d'Artagnan. 

\speak  Que vois-je là? 

\speak  Où? 

\speak  Dans la rue, en face de vos fenêtres, dans l'embrasure de cette porte: un homme enveloppé dans un manteau. 

\speak  C'est lui! s'écrièrent à la fois d'Artagnan et le bourgeois, chacun d'eux en même temps ayant reconnu son homme. 

\speak  Ah! cette fois-ci, s'écria d'Artagnan en sautant sur son épée, cette fois-ci, il ne m'échappera pas.» 

Et tirant son épée du fourreau, il se précipita hors de l'appartement. 

Sur l'escalier, il rencontra Athos et Porthos qui le venaient voir. Ils s'écartèrent, d'Artagnan passa entre eux comme un trait. 

«Ah çà, où cours-tu ainsi? lui crièrent à la fois les deux mousquetaires. 

\speak  L'homme de Meung!» répondit d'Artagnan, et il disparut. 

D'Artagnan avait plus d'une fois raconté à ses amis son aventure avec l'inconnu, ainsi que l'apparition de la belle voyageuse à laquelle cet homme avait paru confier une si importante missive. 

L'avis d'Athos avait été que d'Artagnan avait perdu sa lettre dans la bagarre. Un gentilhomme, selon lui --- et, au portrait que d'Artagnan avait fait de l'inconnu, ce ne pouvait être qu'un gentilhomme ---, un gentilhomme devait être incapable de cette bassesse, de voler une lettre. 

Porthos n'avait vu dans tout cela qu'un rendez-vous amoureux donné par une dame à un cavalier ou par un cavalier à une dame, et qu'était venu troubler la présence de d'Artagnan et de son cheval jaune. 

Aramis avait dit que ces sortes de choses étant mystérieuses, mieux valait ne les point approfondir. 

Ils comprirent donc, sur les quelques mots échappés à d'Artagnan, de quelle affaire il était question, et comme ils pensèrent qu'après avoir rejoint son homme ou l'avoir perdu de vue, d'Artagnan finirait toujours par remonter chez lui, ils continuèrent leur chemin. 

Lorsqu'ils entrèrent dans la chambre de d'Artagnan, la chambre était vide: le propriétaire, craignant les suites de la rencontre qui allait sans doute avoir lieu entre le jeune homme et l'inconnu, avait, par suite de l'exposition qu'il avait faite lui-même de son caractère, jugé qu'il était prudent de décamper.
%!TeX root=../musketeersfr.tex 

\chapter{D'Artagnan Se Dessine}

\lettrine{C}{omme} l'avaient prévu Athos et Porthos, au bout d'une demi-heure d'Artagnan rentra. Cette fois encore il avait manqué son homme, qui avait disparu comme par enchantement. D'Artagnan avait couru, l'épée à la main, toutes les rues environnantes, mais il n'avait rien trouvé qui ressemblât à celui qu'il cherchait, puis enfin il en était revenu à la chose par laquelle il aurait dû commencer peut-être, et qui était de frapper à la porte contre laquelle l'inconnu était appuyé; mais c'était inutilement qu'il avait dix ou douze fois de suite fait résonner le marteau, personne n'avait répondu, et des voisins qui, attirés par le bruit, étaient accourus sur le seuil de leur porte ou avaient mis le nez à leurs fenêtres, lui avaient assuré que cette maison, dont au reste toutes les ouvertures étaient closes, était depuis six mois complètement inhabitée. 

Pendant que d'Artagnan courait les rues et frappait aux portes, Aramis avait rejoint ses deux compagnons, de sorte qu'en revenant chez lui, d'Artagnan trouva la réunion au grand complet. 

«Eh bien? dirent ensemble les trois mousquetaires en voyant entrer d'Artagnan, la sueur sur le front et la figure bouleversée par la colère. 

\speak  Eh bien, s'écria celui-ci en jetant son épée sur le lit, il faut que cet homme soit le diable en personne; il a disparu comme un fantôme, comme une ombre, comme un spectre. 

\speak  Croyez-vous aux apparitions? demanda Athos à Porthos. 

\speak  Moi, je ne crois que ce que j'ai vu, et comme je n'ai jamais vu d'apparitions, je n'y crois pas. 

\speak  La Bible, dit Aramis, nous fait une loi d'y croire: l'ombre de Samuel apparut à Saül, et c'est un article de foi que je serais fâché de voir mettre en doute, Porthos. 

\speak  Dans tous les cas, homme ou diable, corps ou ombre, illusion ou réalité, cet homme est né pour ma damnation, car sa fuite nous fait manquer une affaire superbe, messieurs, une affaire dans laquelle il y avait cent pistoles et peut-être plus à gagner. 

\speak  Comment cela?» dirent à la fois Porthos et Aramis. 

Quant à Athos, fidèle à son système de mutisme, il se contenta d'interroger d'Artagnan du regard. 

«Planchet, dit d'Artagnan à son domestique, qui passait en ce moment la tête par la porte entrebâillée pour tâcher de surprendre quelques bribes de la conversation, descendez chez mon propriétaire, M. Bonacieux, et dites-lui de nous envoyer une demi-douzaine de bouteilles de vin de Beaugency: c'est celui que je préfère. 

\speak  Ah çà, mais vous avez donc crédit ouvert chez votre propriétaire? demanda Porthos. 

\speak  Oui, répondit d'Artagnan, à compter d'aujourd'hui, et soyez tranquilles, si son vin est mauvais, nous lui en enverrons quérir d'autre. 

\speak  Il faut user et non abuser, dit sentencieusement Aramis. 

\speak  J'ai toujours dit que d'Artagnan était la forte tête de nous quatre, fit Athos, qui, après avoir émis cette opinion à laquelle d'Artagnan répondit par un salut, retomba aussitôt dans son silence accoutumé. 

\speak  Mais enfin, voyons, qu'y a-t-il? demanda Porthos. 

\speak  Oui, dit Aramis, confiez-nous cela, mon cher ami, à moins que l'honneur de quelque dame ne se trouve intéressé à cette confidence, à ce quel cas vous feriez mieux de la garder pour vous. 

\speak  Soyez tranquilles, répondit d'Artagnan, l'honneur de personne n'aura à se plaindre de ce que j'ai à vous dire.» 

Et alors il raconta mot à mot à ses amis ce qui venait de se passer entre lui et son hôte, et comment l'homme qui avait enlevé la femme du digne propriétaire était le même avec lequel il avait eu maille à partir à l'hôtellerie du Franc Meunier. 

«Votre affaire n'est pas mauvaise, dit Athos après avoir goûté le vin en connaisseur et indiqué d'un signe de tête qu'il le trouvait bon, et l'on pourra tirer de ce brave homme cinquante à soixante pistoles. Maintenant, reste à savoir si cinquante à soixante pistoles valent la peine de risquer quatre têtes. 

\speak  Mais faites attention, s'écria d'Artagnan qu'il y a une femme dans cette affaire, une femme enlevée, une femme qu'on menace sans doute, qu'on torture peut-être, et tout cela parce qu'elle est fidèle à sa maîtresse! 

\speak  Prenez garde, d'Artagnan, prenez garde, dit Aramis, vous vous échauffez un peu trop, à mon avis, sur le sort de Mme Bonacieux. La femme a été créée pour notre perte, et c'est d'elle que nous viennent toutes nos misères.» 

Athos, à cette sentence d'Aramis, fronça le sourcil et se mordit les lèvres. 

«Ce n'est point de Mme Bonacieux que je m'inquiète, s'écria d'Artagnan, mais de la reine, que le roi abandonne, que le cardinal persécute, et qui voit tomber, les unes après les autres, les têtes de tous ses amis. 

\speak  Pourquoi aime-t-elle ce que nous détestons le plus au monde, les Espagnols et les Anglais? 

\speak  L'Espagne est sa patrie, répondit d'Artagnan, et il est tout simple qu'elle aime les Espagnols, qui sont enfants de la même terre qu'elle. Quant au second reproche que vous lui faites, j'ai entendu dire qu'elle aimait non pas les Anglais, mais un Anglais. 

\speak  Eh! ma foi, dit Athos, il faut avouer que cet Anglais était bien digne d'être aimé. Je n'ai jamais vu un plus grand air que le sien. 

\speak  Sans compter qu'il s'habille comme personne, dit Porthos. J'étais au Louvre le jour où il a semé ses perles, et pardieu! j'en ai ramassé deux que j'ai bien vendues dix pistoles pièce. Et toi, Aramis, le connais-tu? 

\speak  Aussi bien que vous, messieurs, car j'étais de ceux qui l'ont arrêté dans le jardin d'Amiens, où m'avait introduit M. de Putange, l'écuyer de la reine. J'étais au séminaire à cette époque, et l'aventure me parut cruelle pour le roi. 

\speak  Ce qui ne m'empêcherait pas, dit d'Artagnan, si je savais où est le duc de Buckingham, de le prendre par la main et de le conduire près de la reine, ne fût-ce que pour faire enrager M. le cardinal; car notre véritable, notre seul, notre éternel ennemi, messieurs, c'est le cardinal, et si nous pouvions trouver moyen de lui jouer quelque tour bien cruel, j'avoue que j'y engagerais volontiers ma tête. 

\speak  Et, reprit Athos, le mercier vous a dit, d'Artagnan, que la reine pensait qu'on avait fait venir Buckingham sur un faux avis? 

\speak  Elle en a peur. 

\speak  Attendez donc, dit Aramis. 

\speak  Quoi? demanda Porthos. 

\speak  Allez toujours, je cherche à me rappeler des circonstances. 

\speak  Et maintenant je suis convaincu, dit d'Artagnan, que l'enlèvement de cette femme de la reine se rattache aux événements dont nous parlons, et peut-être à la présence de M. de Buckingham à Paris. 

\speak  Le Gascon est plein d'idées, dit Porthos avec admiration. 

\speak  J'aime beaucoup l'entendre parler, dit Athos, son patois m'amuse. 

\speak  Messieurs, reprit Aramis, écoutez ceci. 

\speak  Écoutons Aramis, dirent les trois amis. 

\speak  Hier je me trouvais chez un savant docteur en théologie que je consulte quelquefois pour mes études\dots» 

Athos sourit. 

«Il habite un quartier désert, continua Aramis: ses goûts, sa profession l'exigent. Or, au moment où je sortais de chez lui\dots» 

Ici Aramis s'arrêta. 

«Eh bien? demandèrent ses auditeurs, au moment où vous sortiez de chez lui?» 

Aramis parut faire un effort sur lui-même, comme un homme qui, en plein courant de mensonge, se voit arrêter par quelque obstacle imprévu; mais les yeux de ses trois compagnons étaient fixés sur lui, leurs oreilles attendaient béantes, il n'y avait pas moyen de reculer. 

«Ce docteur a une nièce, continua Aramis. 

\speak  Ah! il a une nièce! interrompit Porthos. 

\speak  Dame fort respectable», dit Aramis. 

Les trois amis se mirent à rire. 

«Ah! si vous riez ou si vous doutez, reprit Aramis, vous ne saurez rien. 

\speak  Nous sommes croyants comme des mahométistes et muets comme des catafalques, dit Athos. 

\speak  Je continue donc, reprit Aramis. Cette nièce vient quelquefois voir son oncle; or elle s'y trouvait hier en même temps que moi, par hasard, et je dus m'offrir pour la conduire à son carrosse. 

\speak  Ah! elle a un carrosse, la nièce du docteur? interrompit Porthos, dont un des défauts était une grande incontinence de langue; belle connaissance, mon ami. 

\speak  Porthos, reprit Aramis, je vous ai déjà fait observer plus d'une fois que vous êtes fort indiscret, et que cela vous nuit près des femmes. 

\speak  Messieurs, messieurs, s'écria d'Artagnan, qui entrevoyait le fond de l'aventure, la chose est sérieuse; tâchons donc de ne pas plaisanter si nous pouvons. Allez, Aramis, allez. 

\speak  Tout à coup, un homme grand, brun, aux manières de gentilhomme\dots, tenez, dans le genre du vôtre, d'Artagnan. 

\speak  Le même peut-être, dit celui-ci. 

\speak  C'est possible, continua Aramis,\dots s'approcha de moi, accompagné de cinq ou six hommes qui le suivaient à dix pas en arrière, et du ton le plus poli: “Monsieur le duc, me dit-il, et vous, madame”, continua-t-il en s'adressant à la dame que j'avais sous le bras\dots 

\speak  À la nièce du docteur? 

\speak  Silence donc, Porthos! dit Athos, vous êtes insupportable. 

\speak  Veuillez monter dans ce carrosse, et cela sans essayer la moindre résistance, sans faire le moindre bruit.» 

\speak  Il vous avait pris pour Buckingham! s'écria d'Artagnan. 

\speak  Je le crois, répondit Aramis. 

\speak  Mais cette dame? demanda Porthos. 

\speak  Il l'avait prise pour la reine! dit d'Artagnan. 

\speak  Justement, répondit Aramis. 

\speak  Le Gascon est le diable! s'écria Athos, rien ne lui échappe. 

\speak  Le fait est, dit Porthos, qu'Aramis est de la taille et a quelque chose de la tournure du beau duc; mais cependant, il me semble que l'habit de mousquetaire\dots 

\speak  J'avais un manteau énorme, dit Aramis. 

\speak  Au mois de juillet, diable! fit Porthos, est-ce que le docteur craint que tu ne sois reconnu? 

\speak  Je comprends encore, dit Athos, que l'espion se soit laissé prendre par la tournure; mais le visage\dots 

\speak  J'avais un grand chapeau, dit Aramis. 

\speak  Oh! mon Dieu, s'écria Porthos, que de précautions pour étudier la théologie! 

\speak  Messieurs, messieurs, dit d'Artagnan, ne perdons pas notre temps à badiner; éparpillons-nous et cherchons la femme du mercier, c'est la clef de l'intrigue. 

\speak  Une femme de condition si inférieure! vous croyez, d'Artagnan? fit Porthos en allongeant les lèvres avec mépris. 

\speak  C'est la filleule de La Porte, le valet de confiance de la reine. Ne vous l'ai-je pas dit, messieurs? Et d'ailleurs, c'est peut-être un calcul de Sa Majesté d'avoir été, cette fois, chercher ses appuis si bas. Les hautes têtes se voient de loin, et le cardinal a bonne vue. 

\speak  Eh bien, dit Porthos, faites d'abord prix avec le mercier, et bon prix. 

\speak  C'est inutile, dit d'Artagnan, car je crois que s'il ne nous paie pas, nous serons assez payés d'un autre côté.» 

En ce moment, un bruit précipité de pas retentit dans l'escalier, la porte s'ouvrit avec fracas, et le malheureux mercier s'élança dans la chambre où se tenait le conseil. 

«Ah! messieurs, s'écria-t-il, sauvez-moi, au nom du Ciel, sauvez-moi! Il y a quatre hommes qui viennent pour m'arrêter; sauvez-moi, sauvez-moi!» 

Porthos et Aramis se levèrent. 

«Un moment, s'écria d'Artagnan en leur faisant signe de repousser au fourreau leurs épées à demi tirées; un moment, ce n'est pas du courage qu'il faut ici, c'est de la prudence. 

\speak  Cependant, s'écria Porthos, nous ne laisserons pas\dots 

\speak  Vous laisserez faire d'Artagnan, dit Athos, c'est, je le répète, la forte tête de nous tous, et moi, pour mon compte, je déclare que je lui obéis. Fais ce que tu voudras, d'Artagnan.» 

En ce moment, les quatre gardes apparurent à la porte de l'antichambre, et voyant quatre mousquetaires debout et l'épée au côté, hésitèrent à aller plus loin. 

«Entrez, messieurs, entrez, cria d'Artagnan; vous êtes ici chez moi, et nous sommes tous de fidèles serviteurs du roi et de M. le cardinal. 

\speak  Alors, messieurs, vous ne vous opposerez pas à ce que nous exécutions les ordres que nous avons reçus? demanda celui qui paraissait le chef de l'escouade. 

\speak  Au contraire, messieurs, et nous vous prêterions main-forte, si besoin était. 

\speak  Mais que dit-il donc? marmotta Porthos. 

\speak  Tu es un niais, dit Athos, silence! 

\speak  Mais vous m'avez promis\dots, dit tout bas le pauvre mercier. 

\speak  Nous ne pouvons vous sauver qu'en restant libres, répondit rapidement et tout bas d'Artagnan, et si nous faisons mine de vous défendre, on nous arrête avec vous. 

\speak  Il me semble, cependant\dots 

\speak  Venez, messieurs, venez, dit tout haut d'Artagnan; je n'ai aucun motif de défendre monsieur. Je l'ai vu aujourd'hui pour la première fois, et encore à quelle occasion, il vous le dira lui-même, pour me venir réclamer le prix de mon loyer. Est-ce vrai, monsieur Bonacieux? Répondez! 

\speak  C'est la vérité pure, s'écria le mercier, mais monsieur ne vous dit pas\dots 

\speak  Silence sur moi, silence sur mes amis, silence sur la reine surtout, ou vous perdriez tout le monde sans vous sauver. Allez, allez, messieurs, emmenez cet homme!» 

Et d'Artagnan poussa le mercier tout étourdi aux mains des gardes, en lui disant: 

«Vous êtes un maraud, mon cher; vous venez me demander de l'argent, à moi! à un mousquetaire! En prison, messieurs, encore une fois, emmenez-le en prison et gardez-le sous clef le plus longtemps possible, cela me donnera du temps pour payer.» 

Les sbires se confondirent en remerciements et emmenèrent leur proie. 

Au moment où ils descendaient, d'Artagnan frappa sur l'épaule du chef: 

«Ne boirai-je pas à votre santé et vous à la mienne? dit-il, en remplissant deux verres du vin de Beaugency qu'il tenait de la libéralité de M. Bonacieux. 

\speak  Ce sera bien de l'honneur pour moi, dit le chef des sbires, et j'accepte avec reconnaissance. 

\speak  Donc, à la vôtre, monsieur\dots comment vous nommez-vous? 

\speak  Boisrenard. 

\speak  Monsieur Boisrenard! 

\speak  À la vôtre, mon gentilhomme: comment vous nommez-vous, à votre tour, s'il vous plaît? 

\speak  D'Artagnan. 

\speak  À la vôtre, monsieur d'Artagnan! 

\speak  Et par-dessus toutes celles-là, s'écria d'Artagnan comme emporté par son enthousiasme, à celle du roi et du cardinal.» 

Le chef des sbires eût peut-être douté de la sincérité de d'Artagnan, si le vin eût été mauvais; mais le vin était bon, il fut convaincu. 

«Mais quelle diable de vilenie avez-vous donc faite là? dit Porthos lorsque l'alguazil en chef eut rejoint ses compagnons, et que les quatre amis se retrouvèrent seuls. Fi donc! quatre mousquetaires laisser arrêter au milieu d'eux un malheureux qui crie à l'aide! Un gentilhomme trinquer avec un recors! 

\speak  Porthos, dit Aramis, Athos t'a déjà prévenu que tu étais un niais, et je me range de son avis. D'Artagnan, tu es un grand homme, et quand tu seras à la place de M. de Tréville, je te demande ta protection pour me faire avoir une abbaye. 

\speak  Ah çà, je m'y perds, dit Porthos, vous approuvez ce que d'Artagnan vient de faire? 

\speak  Je le crois parbleu bien, dit Athos; non seulement j'approuve ce qu'il vient de faire, mais encore je l'en félicite. 

\speak  Et maintenant, messieurs, dit d'Artagnan sans se donner la peine d'expliquer sa conduite à Porthos, tous pour un, un pour tous, c'est notre devise, n'est-ce pas? 

\speak  Cependant\dots dit Porthos. 

\speak  Étends la main et jure!» s'écrièrent à la fois Athos et Aramis. 

Vaincu par l'exemple, maugréant tout bas, Porthos étendit la main, et les quatre amis répétèrent d'une seule voix la formule dictée par d'Artagnan: 

«Tous pour un, un pour tous.» 

«C'est bien, que chacun se retire maintenant chez soi, dit d'Artagnan comme s'il n'avait fait autre chose que de commander toute sa vie, et attention, car à partir de ce moment, nous voilà aux prises avec le cardinal.»
%!TeX root=../musketeersfr.tex 

\chapter{Une Souricière Au XVII\ieme\ Siècle} 
	
	\lettrine{L}{'invention} de la souricière ne date pas de nos jours; dès que les sociétés, en se formant, eurent inventé une police quelconque, cette police, à son tour, inventa les souricières. 

\zz
Comme peut-être nos lecteurs ne sont pas familiarisés encore avec l'argot de la rue de Jérusalem, et que c'est, depuis que nous écrivons --- et il y a quelque quinze ans de cela ---, la première fois que nous employons ce mot appliqué à cette chose, expliquons-leur ce que c'est qu'une souricière. 

Quand, dans une maison quelle qu'elle soit, on a arrêté un individu soupçonné d'un crime quelconque, on tient secrète l'arrestation; on place quatre ou cinq hommes en embuscade dans la première pièce, on ouvre la porte à tous ceux qui frappent, on la referme sur eux et on les arrête; de cette façon, au bout de deux ou trois jours, on tient à peu près tous les familiers de l'établissement. 

Voilà ce que c'est qu'une souricière. 

On fit donc une souricière de l'appartement de maître Bonacieux, et quiconque y apparut fut pris et interrogé par les gens de M. le cardinal. Il va sans dire que, comme une allée particulière conduisait au premier étage qu'habitait d'Artagnan, ceux qui venaient chez lui étaient exemptés de toutes visites. 

D'ailleurs les trois mousquetaires y venaient seuls; ils s'étaient mis en quête chacun de son côté, et n'avaient rien trouvé, rien découvert. Athos avait été même jusqu'à questionner M. de Tréville, chose qui, vu le mutisme habituel du digne mousquetaire, avait fort étonné son capitaine. Mais M. de Tréville ne savait rien, sinon que, la dernière fois qu'il avait vu le cardinal, le roi et la reine, le cardinal avait l'air fort soucieux, que le roi était inquiet, et que les yeux rouges de la reine indiquaient qu'elle avait veillé ou pleuré. Mais cette dernière circonstance l'avait peu frappé, la reine, depuis son mariage, veillant et pleurant beaucoup. 

M. de Tréville recommanda en tout cas à Athos le service du roi et surtout celui de la reine, le priant de faire la même recommandation à ses camarades. 

Quant à d'Artagnan, il ne bougeait pas de chez lui. Il avait converti sa chambre en observatoire. Des fenêtres il voyait arriver ceux qui venaient se faire prendre; puis, comme il avait ôté les carreaux du plancher, qu'il avait creusé le parquet et qu'un simple plafond le séparait de la chambre au-dessous, où se faisaient les interrogatoires, il entendait tout ce qui se passait entre les inquisiteurs et les accusés. 

Les interrogatoires, précédés d'une perquisition minutieuse opérée sur la personne arrêtée, étaient presque toujours ainsi conçus: 

«Mme Bonacieux vous a-t-elle remis quelque chose pour son mari ou pour quelque autre personne? 

\speak  M. Bonacieux vous a-t-il remis quelque chose pour sa femme ou pour quelque autre personne? 

\speak  L'un et l'autre vous ont-ils fait quelque confidence de vive voix?» 

«S'ils savaient quelque chose, ils ne questionneraient pas ainsi, se dit à lui-même d'Artagnan. Maintenant, que cherchent-ils à savoir? Si le duc de Buckingham ne se trouve point à Paris et s'il n'a pas eu ou s'il ne doit point avoir quelque entrevue avec la reine.» 

D'Artagnan s'arrêta à cette idée, qui, d'après tout ce qu'il avait entendu, ne manquait pas de probabilité. 

En attendant, la souricière était en permanence, et la vigilance de d'Artagnan aussi. 

Le soir du lendemain de l'arrestation du pauvre Bonacieux, comme Athos venait de quitter d'Artagnan pour se rendre chez M. de Tréville, comme neuf heures venaient de sonner, et comme Planchet, qui n'avait pas encore fait le lit, commençait sa besogne, on entendit frapper à la porte de la rue; aussitôt cette porte s'ouvrit et se referma: quelqu'un venait de se prendre à la souricière. 

D'Artagnan s'élança vers l'endroit décarrelé, se coucha ventre à terre et écouta. 

Des cris retentirent bientôt, puis des gémissements qu'on cherchait à étouffer. D'interrogatoire, il n'en était pas question. 

«Diable! se dit d'Artagnan, il me semble que c'est une femme: on la fouille, elle résiste, --- on la violente, --- les misérables!» 

Et d'Artagnan, malgré sa prudence, se tenait à quatre pour ne pas se mêler à la scène qui se passait au-dessous de lui. 

«Mais je vous dis que je suis la maîtresse de la maison, messieurs; je vous dis que je suis Mme Bonacieux, je vous dis que j'appartiens à la reine!» s'écriait la malheureuse femme. 

«Mme Bonacieux! murmura d'Artagnan; serais-je assez heureux pour avoir trouvé ce que tout le monde cherche?» 

«C'est justement vous que nous attendions», reprirent les interrogateurs. 

La voix devint de plus en plus étouffée: un mouvement tumultueux fit retentir les boiseries. La victime résistait autant qu'une femme peut résister à quatre hommes. 

«Pardon, messieurs, par\dots», murmura la voix, qui ne fit plus entendre que des sons inarticulés. 

«Ils la bâillonnent, ils vont l'entraîner, s'écria d'Artagnan en se redressant comme par un ressort. Mon épée; bon, elle est à mon côté. Planchet! 

\speak  Monsieur? 

\speak  Cours chercher Athos, Porthos et Aramis. L'un des trois sera sûrement chez lui, peut-être tous les trois seront-ils rentrés. Qu'ils prennent des armes, qu'ils viennent, qu'ils accourent. Ah! je me souviens, Athos est chez M. de Tréville. 

\speak  Mais où allez-vous, monsieur, où allez-vous? 

\speak  Je descends par la fenêtre, s'écria d'Artagnan, afin d'être plus tôt arrivé; toi, remets les carreaux, balaie le plancher, sors par la porte et cours où je te dis. 

\speak  Oh! monsieur, monsieur, vous allez vous tuer, s'écria Planchet. 

\speak  Tais-toi, imbécile», dit d'Artagnan. Et s'accrochant de la main au rebord de sa fenêtre, il se laissa tomber du premier étage, qui heureusement n'était pas élevé, sans se faire une écorchure. 

Puis il alla aussitôt frapper à la porte en murmurant: 

«Je vais me faire prendre à mon tour dans la souricière, et malheur aux chats qui se frotteront à pareille souris.» 

À peine le marteau eut-il résonné sous la main du jeune homme, que le tumulte cessa, que des pas s'approchèrent, que la porte s'ouvrit, et que d'Artagnan, l'épée nue, s'élança dans l'appartement de maître Bonacieux, dont la porte, sans doute mue par un ressort, se referma d'elle-même sur lui. 

Alors ceux qui habitaient encore la malheureuse maison de Bonacieux et les voisins les plus proches entendirent de grands cris, des trépignements, un cliquetis d'épées et un bruit prolongé de meubles. Puis, un moment après, ceux qui, surpris par ce bruit, s'étaient mis aux fenêtres pour en connaître la cause, purent voir la porte se rouvrir et quatre hommes vêtus de noir non pas en sortir, mais s'envoler comme des corbeaux effarouchés, laissant par terre et aux angles des tables des plumes de leurs ailes, c'est-à-dire des loques de leurs habits et des bribes de leurs manteaux. 

D'Artagnan était vainqueur sans beaucoup de peine, il faut le dire, car un seul des alguazils était armé, encore se défendit-il pour la forme. Il est vrai que les trois autres avaient essayé d'assommer le jeune homme avec les chaises, les tabourets et les poteries; mais deux ou trois égratignures faites par la flamberge du Gascon les avaient épouvantés. Dix minutes avaient suffi à leur défaite et d'Artagnan était resté maître du champ de bataille. 

Les voisins, qui avaient ouvert leurs fenêtres avec le sang-froid particulier aux habitants de Paris dans ces temps d'émeutes et de rixes perpétuelles, les refermèrent dès qu'ils eurent vu s'enfuir les quatre hommes noirs: leur instinct leur disait que, pour le moment, tout était fini. 

D'ailleurs il se faisait tard, et alors comme aujourd'hui on se couchait de bonne heure dans le quartier du Luxembourg. 

D'Artagnan, resté seul avec Mme Bonacieux, se retourna vers elle: la pauvre femme était renversée sur un fauteuil et à demi évanouie. D'Artagnan l'examina d'un coup d'œil rapide. 

C'était une charmante femme de vingt-cinq à vingt-six ans, brune avec des yeux bleus, ayant un nez légèrement retroussé, des dents admirables, un teint marbré de rose et d'opale. Là cependant s'arrêtaient les signes qui pouvaient la faire confondre avec une grande dame. Les mains étaient blanches, mais sans finesse: les pieds n'annonçaient pas la femme de qualité. Heureusement d'Artagnan n'en était pas encore à se préoccuper de ces détails. 

Tandis que d'Artagnan examinait Mme Bonacieux, et en était aux pieds, comme nous l'avons dit, il vit à terre un fin mouchoir de batiste, qu'il ramassa selon son habitude, et au coin duquel il reconnut le même chiffre qu'il avait vu au mouchoir qui avait failli lui faire couper la gorge avec Aramis. 

Depuis ce temps, d'Artagnan se méfiait des mouchoirs armoriés; il remit donc sans rien dire celui qu'il avait ramassé dans la poche de Mme Bonacieux. En ce moment, Mme Bonacieux reprenait ses sens. Elle ouvrit les yeux, regarda avec terreur autour d'elle, vit que l'appartement était vide, et qu'elle était seule avec son libérateur. Elle lui tendit aussitôt les mains en souriant. Mme Bonacieux avait le plus charmant sourire du monde. 

«Ah! monsieur! dit-elle, c'est vous qui m'avez sauvée; permettez-moi que je vous remercie. 

\speak  Madame, dit d'Artagnan, je n'ai fait que ce que tout gentilhomme eût fait à ma place, vous ne me devez donc aucun remerciement. 

\speak  Si fait, monsieur, si fait, et j'espère vous prouver que vous n'avez pas rendu service à une ingrate. Mais que me voulaient donc ces hommes, que j'ai pris d'abord pour des voleurs, et pourquoi M. Bonacieux n'est-il point ici? 

\speak  Madame, ces hommes étaient bien autrement dangereux que ne pourraient être des voleurs, car ce sont des agents de M. le cardinal, et quant à votre mari, M. Bonacieux, il n'est point ici parce qu'hier on est venu le prendre pour le conduire à la Bastille. 

\speak  Mon mari à la Bastille! s'écria Mme Bonacieux, oh! mon Dieu! qu'a-t-il donc fait? pauvre cher homme! lui, l'innocence même!» 

Et quelque chose comme un sourire perçait sur la figure encore tout effrayée de la jeune femme. 

«Ce qu'il a fait, madame? dit d'Artagnan. Je crois que son seul crime est d'avoir à la fois le bonheur et le malheur d'être votre mari. 

\speak  Mais, monsieur, vous savez donc\dots 

\speak  Je sais que vous avez été enlevée, madame. 

\speak  Et par qui? Le savez-vous? Oh! si vous le savez, dites-le-moi. 

\speak  Par un homme de quarante à quarante-cinq ans, aux cheveux noirs, au teint basané, avec une cicatrice à la tempe gauche. 

\speak  C'est cela, c'est cela; mais son nom? 

\speak  Ah! son nom? c'est ce que j'ignore. 

\speak  Et mon mari savait-il que j'avais été enlevée? 

\speak  Il en avait été prévenu par une lettre que lui avait écrite le ravisseur lui-même. 

\speak  Et soupçonne-t-il, demanda Mme Bonacieux avec embarras, la cause de cet événement? 

\speak  Il l'attribuait, je crois, à une cause politique. 

\speak  J'en ai douté d'abord, et maintenant je le pense comme lui. Ainsi donc, ce cher M. Bonacieux ne m'a pas soupçonnée un seul instant\dots? 

\speak  Ah! loin de là, madame, il était trop fier de votre sagesse et surtout de votre amour.» 

Un second sourire presque imperceptible effleura les lèvres rosées de la belle jeune femme. 

«Mais, continua d'Artagnan, comment vous êtes-vous enfuie? 

\speak  J'ai profité d'un moment où l'on m'a laissée seule, et comme je savais depuis ce matin à quoi m'en tenir sur mon enlèvement, à l'aide de mes draps je suis descendue par la fenêtre; alors, comme je croyais mon mari ici, je suis accourue. 

\speak  Pour vous mettre sous sa protection? 

\speak  Oh! non, pauvre cher homme, je savais bien qu'il était incapable de me défendre; mais comme il pouvait nous servir à autre chose, je voulais le prévenir. 

\speak  De quoi? 

\speak  Oh! ceci n'est pas mon secret, je ne puis donc pas vous le dire. 

\speak  D'ailleurs, dit d'Artagnan (pardon, madame, si, tout garde que je suis, je vous rappelle à la prudence), d'ailleurs je crois que nous ne sommes pas ici en lieu opportun pour faire des confidences. Les hommes que j'ai mis en fuite vont revenir avec main-forte; s'ils nous retrouvent ici nous sommes perdus. J'ai bien fait prévenir trois de mes amis, mais qui sait si on les aura trouvés chez eux! 

\speak  Oui, oui, vous avez raison, s'écria Mme Bonacieux effrayée; fuyons, sauvons-nous.» 

À ces mots, elle passa son bras sous celui de d'Artagnan et l'entraîna vivement. 

«Mais où fuir? dit d'Artagnan, où nous sauver? 

\speak  Éloignons-nous d'abord de cette maison, puis après nous verrons.» 

Et la jeune femme et le jeune homme, sans se donner la peine de refermer la porte, descendirent rapidement la rue des Fossoyeurs, s'engagèrent dans la rue des Fossés-Monsieur-le-Prince et ne s'arrêtèrent qu'à la place Saint-Sulpice. 

«Et maintenant, qu'allons-nous faire, demanda d'Artagnan, et où voulez-vous que je vous conduise? 

\speak  Je suis fort embarrassée de vous répondre, je vous l'avoue, dit Mme Bonacieux; mon intention était de faire prévenir M. de La Porte par mon mari, afin que M. de La Porte pût nous dire précisément ce qui s'était passé au Louvre depuis trois jours, et s'il n'y avait pas danger pour moi de m'y présenter. 

\speak  Mais moi, dit d'Artagnan, je puis aller prévenir M. de La Porte. 

\speak  Sans doute; seulement il n'y a qu'un malheur: c'est qu'on connaît M. Bonacieux au Louvre et qu'on le laisserait passer, lui, tandis qu'on ne vous connaît pas, vous, et que l'on vous fermera la porte. 

\speak  Ah! bah, dit d'Artagnan, vous avez bien à quelque guichet du Louvre un concierge qui vous est dévoué, et qui grâce à un mot d'ordre\dots» 

Mme Bonacieux regarda fixement le jeune homme. 

«Et si je vous donnais ce mot d'ordre, dit-elle, l'oublieriez-vous aussitôt que vous vous en seriez servi? 

\speak  Parole d'honneur, foi de gentilhomme! dit d'Artagnan avec un accent à la vérité duquel il n'y avait pas à se tromper. 

\speak  Tenez, je vous crois; vous avez l'air d'un brave jeune homme, d'ailleurs votre fortune est peut-être au bout de votre dévouement. 

\speak  Je ferai sans promesse et de conscience tout ce que je pourrai pour servir le roi et être agréable à la reine, dit d'Artagnan; disposez donc de moi comme d'un ami. 

\speak  Mais moi, où me mettrez-vous pendant ce temps-là? 

\speak  N'avez-vous pas une personne chez laquelle M. de La Porte puisse revenir vous prendre? 

\speak  Non, je ne veux me fier à personne. 

\speak  Attendez, dit d'Artagnan; nous sommes à la porte d'Athos. Oui, c'est cela. 

\speak  Qu'est-ce qu'Athos? 

\speak  Un de mes amis. 

\speak  Mais s'il est chez lui et qu'il me voie? 

\speak  Il n'y est pas, et j'emporterai la clef après vous avoir fait entrer dans son appartement. 

\speak  Mais s'il revient? 

\speak  Il ne reviendra pas; d'ailleurs on lui dirait que j'ai amené une femme, et que cette femme est chez lui. 

\speak  Mais cela me compromettra très fort, savez-vous! 

\speak  Que vous importe! on ne vous connaît pas; d'ailleurs nous sommes dans une situation à passer par-dessus quelques convenances! 

\speak  Allons donc chez votre ami. Où demeure-t-il? 

\speak  Rue Férou, à deux pas d'ici. 

\speak  Allons.» 

Et tous deux reprirent leur course. Comme l'avait prévu d'Artagnan, Athos n'était pas chez lui: il prit la clef, qu'on avait l'habitude de lui donner comme à un ami de la maison, monta l'escalier et introduisit Mme Bonacieux dans le petit appartement dont nous avons déjà fait la description. 

«Vous êtes chez vous, dit-il; attendez, fermez la porte en dedans et n'ouvrez à personne, à moins que vous n'entendiez frapper trois coups ainsi: tenez; et il frappa trois fois: deux coups rapprochés l'un de l'autre et assez forts, un coup plus distant et plus léger. 

\speak  C'est bien, dit Mme Bonacieux; maintenant, à mon tour de vous donner mes instructions. 

\speak  J'écoute. 

\speak  Présentez-vous au guichet du Louvre, du côté de la rue de l'Échelle, et demandez Germain. 

\speak  C'est bien. Après? 

\speak  Il vous demandera ce que vous voulez, et alors vous lui répondrez par ces deux mots: Tours et Bruxelles. Aussitôt il se mettra à vos ordres. 

\speak  Et que lui ordonnerai-je? 

\speak  D'aller chercher M. de La Porte, le valet de chambre de la reine. 

\speak  Et quand il l'aura été chercher et que M. de La Porte sera venu? 

\speak  Vous me l'enverrez. 

\speak  C'est bien, mais où et comment vous reverrai-je? 

\speak  Y tenez-vous beaucoup à me revoir? 

\speak  Certainement. 

\speak  Eh bien, reposez-vous sur moi de ce soin, et soyez tranquille. 

\speak  Je compte sur votre parole. 

\speak  Comptez-y.» 

D'Artagnan salua Mme Bonacieux en lui lançant le coup d'œil le plus amoureux qu'il lui fût possible de concentrer sur sa charmante petite personne, et tandis qu'il descendait l'escalier, il entendit la porte se fermer derrière lui à double tour. En deux bonds il fut au Louvre: comme il entrait au guichet de Échelle, dix heures sonnaient. Tous les événements que nous venons de raconter s'étaient succédé en une demi-heure. 

Tout s'exécuta comme l'avait annoncé Mme Bonacieux. Au mot d'ordre convenu, Germain s'inclina; dix minutes après, La Porte était dans la loge; en deux mots, d'Artagnan le mit au fait et lui indiqua où était Mme Bonacieux. La Porte s'assura par deux fois de l'exactitude de l'adresse, et partit en courant. Cependant, à peine eut-il fait dix pas, qu'il revint. 

«Jeune homme, dit-il à d'Artagnan, un conseil. 

\speak  Lequel? 

\speak  Vous pourriez être inquiété pour ce qui vient de se passer. 

\speak  Vous croyez? 

\speak  Oui. Avez-vous quelque ami dont la pendule retarde? 

\speak  Eh bien? 

\speak  Allez le voir pour qu'il puisse témoigner que vous étiez chez lui à neuf heures et demie. En justice, cela s'appelle un alibi.» 

D'Artagnan trouva le conseil prudent; il prit ses jambes à son cou, il arriva chez M. de Tréville, mais, au lieu de passer au salon avec tout le monde, il demanda à entrer dans son cabinet. Comme d'Artagnan était un des habitués de l'hôtel, on ne fit aucune difficulté d'accéder à sa demande; et l'on alla prévenir M. de Tréville que son jeune compatriote, ayant quelque chose d'important à lui dire, sollicitait une audience particulière. Cinq minutes après, M. de Tréville demandait à d'Artagnan ce qu'il pouvait faire pour son service et ce qui lui valait sa visite à une heure si avancée. 

«Pardon, monsieur! dit d'Artagnan, qui avait profité du moment où il était resté seul pour retarder l'horloge de trois quarts d'heure; j'ai pensé que, comme il n'était que neuf heures vingt-cinq minutes, il était encore temps de me présenter chez vous. 

\speak  Neuf heures vingt-cinq minutes! s'écria M. de Tréville en regardant sa pendule; mais c'est impossible! 

\speak  Voyez plutôt, monsieur, dit d'Artagnan, voilà qui fait foi. 

\speak  C'est juste, dit M. de Tréville, j'aurais cru qu'il était plus tard. Mais voyons, que me voulez-vous?» 

Alors d'Artagnan fit à M. de Tréville une longue histoire sur la reine. Il lui exposa les craintes qu'il avait conçues à l'égard de Sa Majesté; il lui raconta ce qu'il avait entendu dire des projets du cardinal à l'endroit de Buckingham, et tout cela avec une tranquillité et un aplomb dont M. de Tréville fut d'autant mieux la dupe, que lui-même, comme nous l'avons dit, avait remarqué quelque chose de nouveau entre le cardinal, le roi et la reine. 

À dix heures sonnant, d'Artagnan quitta M. de Tréville, qui le remercia de ses renseignements, lui recommanda d'avoir toujours à cœur le service du roi et de la reine, et qui rentra dans le salon. Mais, au bas de l'escalier, d'Artagnan se souvint qu'il avait oublié sa canne: en conséquence, il remonta précipitamment, rentra dans le cabinet, d'un tour de doigt remit la pendule à son heure, pour qu'on ne pût pas s'apercevoir, le lendemain, qu'elle avait été dérangée, et sûr désormais qu'il y avait un témoin pour prouver son alibi, il descendit l'escalier et se trouva bientôt dans la rue.
%!TeX root=../musketeersfr.tex 

\chapter{L'Intrigue Se Noue}

\lettrine{S}{a} visite faite à M. de Tréville, d'Artagnan prit, tout pensif, le plus long pour rentrer chez lui. 

\zz
À quoi pensait d'Artagnan, qu'il s'écartait ainsi de sa route, regardant les étoiles du ciel, et tantôt soupirant tantôt souriant? 

Il pensait à Mme Bonacieux. Pour un apprenti mousquetaire, la jeune femme était presque une idéalité amoureuse. Jolie, mystérieuse, initiée à presque tous les secrets de cour, qui reflétaient tant de charmante gravité sur ses traits gracieux, elle était soupçonnée de n'être pas insensible, ce qui est un attrait irrésistible pour les amants novices; de plus, d'Artagnan l'avait délivrée des mains de ces démons qui voulaient la fouiller et la maltraiter, et cet important service avait établi entre elle et lui un de ces sentiments de reconnaissance qui prennent si facilement un plus tendre caractère. 

D'Artagnan se voyait déjà, tant les rêves marchent vite sur les ailes de l'imagination, accosté par un messager de la jeune femme qui lui remettait quelque billet de rendez-vous, une chaîne d'or ou un diamant. Nous avons dit que les jeunes cavaliers recevaient sans honte de leur roi; ajoutons qu'en ce temps de facile morale, ils n'avaient pas plus de vergogne à l'endroit de leurs maîtresses, et que celles-ci leur laissaient presque toujours de précieux et durables souvenirs, comme si elles eussent essayé de conquérir la fragilité de leurs sentiments par la solidité de leurs dons. 

On faisait alors son chemin par les femmes, sans en rougir. Celles qui n'étaient que belles donnaient leur beauté, et de là vient sans doute le proverbe, que la plus belle fille du monde ne peut donner que ce qu'elle a. Celles qui étaient riches donnaient en outre une partie de leur argent, et l'on pourrait citer bon nombre de héros de cette galante époque qui n'eussent gagné ni leurs éperons d'abord, ni leurs batailles ensuite, sans la bourse plus ou moins garnie que leur maîtresse attachait à l'arçon de leur selle. 

D'Artagnan ne possédait rien; l'hésitation du provincial, vernis léger, fleur éphémère, duvet de la pêche, s'était évaporée au vent des conseils peu orthodoxes que les trois mousquetaires donnaient à leur ami. D'Artagnan, suivant l'étrange coutume du temps, se regardait à Paris comme en campagne, et cela ni plus ni moins que dans les Flandres: l'Espagnol là-bas, la femme ici. C'était partout un ennemi à combattre, des contributions à frapper. 

Mais, disons-le, pour le moment d'Artagnan était mû d'un sentiment plus noble et plus désintéressé. Le mercier lui avait dit qu'il était riche; le jeune homme avait pu deviner qu'avec un niais comme l'était M. Bonacieux, ce devait être la femme qui tenait la clef de la bourse. Mais tout cela n'avait influé en rien sur le sentiment produit par la vue de Mme Bonacieux, et l'intérêt était resté à peu près étranger à ce commencement d'amour qui en avait été la suite. Nous disons: à peu près, car l'idée qu'une jeune femme, belle, gracieuse, spirituelle, est riche en même temps, n'ôte rien à ce commencement d'amour, et tout au contraire le corrobore. 

Il y a dans l'aisance une foule de soins et de caprices aristocratiques qui vont bien à la beauté. Un bas fin et blanc, une robe de soie, une guimpe de dentelle, un joli soulier au pied, un frais ruban sur la tête, ne font point jolie une femme laide, mais font belle une femme jolie, sans compter les mains qui gagnent à tout cela; les mains, chez les femmes surtout, ont besoin de rester oisives pour rester belles. 

Puis d'Artagnan, comme le sait bien le lecteur, auquel nous n'avons pas caché l'état de sa fortune, d'Artagnan n'était pas un millionnaire; il espérait bien le devenir un jour, mais le temps qu'il se fixait lui-même pour cet heureux changement était assez éloigné. En attendant, quel désespoir que de voir une femme qu'on aime désirer ces mille riens dont les femmes composent leur bonheur, et de ne pouvoir lui donner ces mille riens! Au moins, quand la femme est riche et que l'amant ne l'est pas, ce qu'il ne peut lui offrir elle se l'offre elle-même; et quoique ce soit ordinairement avec l'argent du mari qu'elle se passe cette jouissance, il est rare que ce soit à lui qu'en revienne la reconnaissance. 

Puis d'Artagnan, disposé à être l'amant le plus tendre, était en attendant un ami très dévoué. Au milieu de ses projets amoureux sur la femme du mercier, il n'oubliait pas les siens. La jolie Mme Bonacieux était femme à promener dans la plaine Saint-Denis ou dans la foire Saint-Germain en compagnie d'Athos, de Porthos et d'Aramis, auxquels d'Artagnan serait fier de montrer une telle conquête. Puis, quand on a marché longtemps, la faim arrive; d'Artagnan depuis quelque temps avait remarqué cela. On ferait de ces petits dîners charmants où l'on touche d'un côté la main d'un ami, et de l'autre le pied d'une maîtresse. Enfin, dans les moments pressants, dans les positions extrêmes, d'Artagnan serait le sauveur de ses amis. 

Et M. Bonacieux, que d'Artagnan avait poussé dans les mains des sbires en le reniant bien haut et à qui il avait promis tout bas de le sauver? Nous devons avouer à nos lecteurs que d'Artagnan n'y songeait en aucune façon, ou que, s'il y songeait, c'était pour se dire qu'il était bien où il était, quelque part qu'il fût. L'amour est la plus égoïste de toutes les passions. 

Cependant, que nos lecteurs se rassurent: si d'Artagnan oublie son hôte ou fait semblant de l'oublier, sous prétexte qu'il ne sait pas où on l'a conduit, nous ne l'oublions pas, nous, et nous savons où il est. Mais pour le moment faisons comme le Gascon amoureux. Quant au digne mercier, nous reviendrons à lui plus tard. 

D'Artagnan, tout en réfléchissant à ses futures amours, tout en parlant à la nuit, tout en souriant aux étoiles, remontait la rue du Cherche-Midi ou Chasse-Midi, ainsi qu'on l'appelait alors. Comme il se trouvait dans le quartier d'Aramis, l'idée lui était venue d'aller faire une visite à son ami, pour lui donner quelques explications sur les motifs qui lui avaient fait envoyer Planchet avec invitation de se rendre immédiatement à la souricière. Or, si Aramis s'était trouvé chez lui lorsque Planchet y était venu, il avait sans aucun doute couru rue des Fossoyeurs, et n'y trouvant personne que ses deux autres compagnons peut-être, ils n'avaient dû savoir, ni les uns ni les autres, ce que cela voulait dire. Ce dérangement méritait donc une explication, voilà ce que disait tout haut d'Artagnan. 

Puis, tout bas, il pensait que c'était pour lui une occasion de parler de la jolie petite Mme Bonacieux, dont son esprit, sinon son cœur, était déjà tout plein. Ce n'est pas à propos d'un premier amour qu'il faut demander de la discrétion. Ce premier amour est accompagné d'une si grande joie, qu'il faut que cette joie déborde, sans cela elle vous étoufferait. 

Paris depuis deux heures était sombre et commençait à se faire désert. Onze heures sonnaient à toutes les horloges du faubourg Saint-Germain, il faisait un temps doux. D'Artagnan suivait une ruelle située sur l'emplacement où passe aujourd'hui la rue d'Assas, respirant les émanations embaumées qui venaient avec le vent de la rue de Vaugirard et qu'envoyaient les jardins rafraîchis par la rosée du soir et par la brise de la nuit. Au loin résonnaient, assourdis cependant par de bons volets, les chants des buveurs dans quelques cabarets perdus dans la plaine. Arrivé au bout de la ruelle, d'Artagnan tourna à gauche. La maison qu'habitait Aramis se trouvait située entre la rue Cassette et la rue Servandoni. 

D'Artagnan venait de dépasser la rue Cassette et reconnaissait déjà la porte de la maison de son ami, enfouie sous un massif de sycomores et de clématites qui formaient un vaste bourrelet au-dessus d'elle lorsqu'il aperçut quelque chose comme une ombre qui sortait de la rue Servandoni. Ce quelque chose était enveloppé d'un manteau, et d'Artagnan crut d'abord que c'était un homme; mais, à la petitesse de la taille, à l'incertitude de la démarche, à l'embarras du pas, il reconnut bientôt une femme. De plus, cette femme, comme si elle n'eût pas été bien sûre de la maison qu'elle cherchait, levait les yeux pour se reconnaître, s'arrêtait, retournait en arrière, puis revenait encore. D'Artagnan fut intrigué. 

«Si j'allais lui offrir mes services! pensa-t-il. À son allure, on voit qu'elle est jeune; peut-être jolie. Oh! oui. Mais une femme qui court les rues à cette heure ne sort guère que pour aller rejoindre son amant. Peste! si j'allais troubler les rendez-vous, ce serait une mauvaise porte pour entrer en relations.» 

Cependant, la jeune femme s'avançait toujours, comptant les maisons et les fenêtres. Ce n'était, au reste, chose ni longue, ni difficile. Il n'y avait que trois hôtels dans cette partie de la rue, et deux fenêtres ayant vue sur cette rue; l'une était celle d'un pavillon parallèle à celui qu'occupait Aramis, l'autre était celle d'Aramis lui-même. 

«Pardieu! se dit d'Artagnan, auquel la nièce du théologien revenait à l'esprit; pardieu! il serait drôle que cette colombe attardée cherchât la maison de notre ami. Mais sur mon âme, cela y ressemble fort. Ah! mon cher Aramis, pour cette fois, j'en veux avoir le cœur net.» 

Et d'Artagnan, se faisant le plus mince qu'il put, s'abrita dans le côté le plus obscur de la rue, près d'un banc de pierre situé au fond d'une niche. 

La jeune femme continua de s'avancer, car outre la légèreté de son allure, qui l'avait trahie, elle venait de faire entendre une petite toux qui dénonçait une voix des plus fraîches. D'Artagnan pensa que cette toux était un signal. 

Cependant, soit qu'on eût répondu à cette toux par un signe équivalent qui avait fixé les irrésolutions de la nocturne chercheuse, soit que sans secours étranger elle eût reconnu qu'elle était arrivée au bout de sa course, elle s'approcha résolument du volet d'Aramis et frappa à trois intervalles égaux avec son doigt recourbé. 

«C'est bien chez Aramis, murmura d'Artagnan. Ah! monsieur l'hypocrite! je vous y prends à faire de la théologie!» 

Les trois coups étaient à peine frappés, que la croisée intérieure s'ouvrit et qu'une lumière parut à travers les vitres du volet. 

«Ah! ah! fit l'écouteur non pas aux portes, mais aux fenêtres, ah! la visite était attendue. Allons, le volet va s'ouvrir et la dame entrera par escalade. Très bien!» 

Mais, au grand étonnement de d'Artagnan, le volet resta fermé. De plus, la lumière qui avait flamboyé un instant, disparut, et tout rentra dans l'obscurité. 

D'Artagnan pensa que cela ne pouvait durer ainsi, et continua de regarder de tous ses yeux et d'écouter de toutes ses oreilles. 

Il avait raison: au bout de quelques secondes, deux coups secs retentirent dans l'intérieur. 

La jeune femme de la rue répondit par un seul coup, et le volet s'entrouvrit. 

On juge si d'Artagnan regardait et écoutait avec avidité. 

Malheureusement, la lumière avait été transportée dans un autre appartement. Mais les yeux du jeune homme s'étaient habitués à la nuit. D'ailleurs les yeux des Gascons ont, à ce qu'on assure, comme ceux des chats, la propriété de voir pendant la nuit. 

D'Artagnan vit donc que la jeune femme tirait de sa poche un objet blanc qu'elle déploya vivement et qui prit la forme d'un mouchoir. Cet objet déployé, elle en fit remarquer le coin à son interlocuteur. 

Cela rappela à d'Artagnan ce mouchoir qu'il avait trouvé aux pieds de Mme Bonacieux, lequel lui avait rappelé celui qu'il avait trouvé aux pieds d'Aramis. 

«Que diable pouvait donc signifier ce mouchoir?» 

Placé où il était, d'Artagnan ne pouvait voir le visage d'Aramis, nous disons d'Aramis, parce que le jeune homme ne faisait aucun doute que ce fût son ami qui dialoguât de l'intérieur avec la dame de l'extérieur; la curiosité l'emporta donc sur la prudence, et, profitant de la préoccupation dans laquelle la vue du mouchoir paraissait plonger les deux personnages que nous avons mis en scène, il sortit de sa cachette, et prompt comme l'éclair, mais étouffant le bruit de ses pas, il alla se coller à un angle de la muraille, d'où son œil pouvait parfaitement plonger dans l'intérieur de l'appartement d'Aramis. 

Arrivé là, d'Artagnan pensa jeter un cri de surprise: ce n'était pas Aramis qui causait avec la nocturne visiteuse, c'était une femme. Seulement, d'Artagnan y voyait assez pour reconnaître la forme de ses vêtements, mais pas assez pour distinguer ses traits. 

Au même instant, la femme de l'appartement tira un second mouchoir de sa poche, et l'échangea avec celui qu'on venait de lui montrer. Puis, quelques mots furent prononcés entre les deux femmes. Enfin le volet se referma; la femme qui se trouvait à l'extérieur de la fenêtre se retourna, et vint passer à quatre pas de d'Artagnan en abaissant la coiffe de sa mante; mais la précaution avait été prise trop tard, d'Artagnan avait déjà reconnu Mme Bonacieux. 

Mme Bonacieux! Le soupçon que c'était elle lui avait déjà traversé l'esprit quand elle avait tiré le mouchoir de sa poche; mais quelle probabilité que Mme Bonacieux qui avait envoyé chercher M. de La Porte pour se faire reconduire par lui au Louvre, courût les rues de Paris seule à onze heures et demie du soir, au risque de se faire enlever une seconde fois? 

Il fallait donc que ce fût pour une affaire bien importante; et quelle est l'affaire importante d'une femme de vingt-cinq ans? L'amour. 

Mais était-ce pour son compte ou pour le compte d'une autre personne qu'elle s'exposait à de semblables hasards? Voilà ce que se demandait à lui-même le jeune homme, que le démon de la jalousie mordait au cœur ni plus ni moins qu'un amant en titre. 

Il y avait, au reste, un moyen bien simple de s'assurer où allait Mme Bonacieux: c'était de la suivre. Ce moyen était si simple, que d'Artagnan l'employa tout naturellement et d'instinct. 

Mais, à la vue du jeune homme qui se détachait de la muraille comme une statue de sa niche, et au bruit des pas qu'elle entendit retentir derrière elle, Mme Bonacieux jeta un petit cri et s'enfuit. 

D'Artagnan courut après elle. Ce n'était pas une chose difficile pour lui que de rejoindre une femme embarrassée dans son manteau. Il la rejoignit donc au tiers de la rue dans laquelle elle s'était engagée. La malheureuse était épuisée, non pas de fatigue, mais de terreur, et quand d'Artagnan lui posa la main sur l'épaule, elle tomba sur un genou en criant d'une voix étranglée: 

«Tuez-moi si vous voulez, mais vous ne saurez rien.» 

D'Artagnan la releva en lui passant le bras autour de la taille; mais comme il sentait à son poids qu'elle était sur le point de se trouver mal, il s'empressa de la rassurer par des protestations de dévouement. Ces protestations n'étaient rien pour Mme Bonacieux; car de pareilles protestations peuvent se faire avec les plus mauvaises intentions du monde; mais la voix était tout. La jeune femme crut reconnaître le son de cette voix: elle rouvrit les yeux, jeta un regard sur l'homme qui lui avait fait si grand-peur, et, reconnaissant d'Artagnan, elle poussa un cri de joie. 

«Oh! c'est vous, c'est vous! dit-elle; merci, mon Dieu! 

\speak  Oui, c'est moi, dit d'Artagnan, moi que Dieu a envoyé pour veiller sur vous. 

\speak  Était-ce dans cette intention que vous me suiviez?» demanda avec un sourire plein de coquetterie la jeune femme, dont le caractère un peu railleur reprenait le dessus, et chez laquelle toute crainte avait disparu du moment où elle avait reconnu un ami dans celui qu'elle avait pris pour un ennemi. 

«Non, dit d'Artagnan, non, je l'avoue; c'est le hasard qui m'a mis sur votre route; j'ai vu une femme frapper à la fenêtre d'un de mes amis\dots 

\speak  D'un de vos amis? interrompit Mme Bonacieux. 

\speak  Sans doute; Aramis est de mes meilleurs amis. 

\speak  Aramis! qu'est-ce que cela? 

\speak  Allons donc! allez-vous me dire que vous ne connaissez pas Aramis? 

\speak  C'est la première fois que j'entends prononcer ce nom. 

\speak  C'est donc la première fois que vous venez à cette maison? 

\speak  Sans doute. 

\speak  Et vous ne saviez pas qu'elle fût habitée par un jeune homme? 

\speak  Non. 

\speak  Par un mousquetaire? 

\speak  Nullement. 

\speak  Ce n'est donc pas lui que vous veniez chercher? 

\speak  Pas le moins du monde. D'ailleurs, vous l'avez bien vu, la personne à qui j'ai parlé est une femme. 

\speak  C'est vrai; mais cette femme est des amies d'Aramis. 

\speak  Je n'en sais rien. 

\speak  Puisqu'elle loge chez lui. 

\speak  Cela ne me regarde pas. 

\speak  Mais qui est-elle? 

\speak  Oh! cela n'est point mon secret. 

\speak  Chère madame Bonacieux, vous êtes charmante; mais en même temps vous êtes la femme la plus mystérieuse\dots 

\speak  Est-ce que je perds à cela? 

\speak  Non; vous êtes, au contraire, adorable. 

\speak  Alors, donnez-moi le bras. 

\speak  Bien volontiers. Et maintenant? 

\speak  Maintenant, conduisez-moi. 

\speak  Où cela? 

\speak  Où je vais. 

\speak  Mais où allez-vous? 

\speak  Vous le verrez, puisque vous me laisserez à la porte. 

\speak  Faudra-t-il vous attendre? 

\speak  Ce sera inutile. 

\speak  Vous reviendrez donc seule? 

\speak  Peut-être oui, peut-être non. 

\speak  Mais la personne qui vous accompagnera ensuite sera-t-elle un homme, sera-t-elle une femme? 

\speak  Je n'en sais rien encore. 

\speak  Je le saurai bien, moi! 

\speak  Comment cela? 

\speak  Je vous attendrai pour vous voir sortir. 

\speak  En ce cas, adieu! 

\speak  Comment cela? 

\speak  Je n'ai pas besoin de vous. 

\speak  Mais vous aviez réclamé\dots 

\speak  L'aide d'un gentilhomme, et non la surveillance d'un espion. 

\speak  Le mot est un peu dur! 

\speak  Comment appelle-t-on ceux qui suivent les gens malgré eux? 

\speak  Des indiscrets. 

\speak  Le mot est trop doux. 

\speak  Allons, madame, je vois bien qu'il faut faire tout ce que vous voulez. 

\speak  Pourquoi vous être privé du mérite de le faire tout de suite? 

\speak  N'y en a-t-il donc aucun à se repentir? 

\speak  Et vous repentez-vous réellement? 

\speak  Je n'en sais rien moi-même. Mais ce que je sais, c'est que je vous promets de faire tout ce que vous voudrez si vous me laissez vous accompagner jusqu'où vous allez. 

\speak  Et vous me quitterez après? 

\speak  Oui. 

\speak  Sans m'épier à ma sortie? 

\speak  Non. 

\speak  Parole d'honneur? 

\speak  Foi de gentilhomme! 

\speak  Prenez mon bras et marchons alors.» 

D'Artagnan offrit son bras à Mme Bonacieux, qui s'y suspendit, moitié rieuse, moitié tremblante, et tous deux gagnèrent le haut de la rue de La Harpe. Arrivée là, la jeune femme parut hésiter, comme elle avait déjà fait dans la rue de Vaugirard. Cependant, à de certains signes, elle sembla reconnaître une porte; et s'approchant de cette porte: 

«Et maintenant, monsieur, dit-elle, c'est ici que j'ai affaire; mille fois merci de votre honorable compagnie, qui m'a sauvée de tous les dangers auxquels, seule, j'eusse été exposée. Mais le moment est venu de tenir votre parole: je suis arrivée à ma destination. 

\speak  Et vous n'aurez plus rien à craindre en revenant? 

\speak  Je n'aurai à craindre que les voleurs. 

\speak  N'est-ce donc rien? 

\speak  Que pourraient-ils me prendre? je n'ai pas un denier sur moi. 

\speak  Vous oubliez ce beau mouchoir brodé, armorié. 

\speak  Lequel? 

\speak  Celui que j'ai trouvé à vos pieds et que j'ai remis dans votre poche. 

\speak  Taisez-vous, taisez-vous, malheureux! s'écria la jeune femme, voulez-vous me perdre? 

\speak  Vous voyez bien qu'il y a encore du danger pour vous, puisqu'un seul mot vous fait trembler, et que vous avouez que, si on entendait ce mot, vous seriez perdue. Ah! tenez, madame, s'écria d'Artagnan en lui saisissant la main et la couvrant d'un ardent regard, tenez! soyez plus généreuse, confiez-vous à moi; n'avez-vous donc pas lu dans mes yeux qu'il n'y a que dévouement et sympathie dans mon cœur? 

\speak  Si fait, répondit Mme Bonacieux; aussi demandez-moi mes secrets, et je vous les dirai; mais ceux des autres, c'est autre chose. 

\speak  C'est bien, dit d'Artagnan, je les découvrirai; puisque ces secrets peuvent avoir une influence sur votre vie, il faut que ces secrets deviennent les miens. 

\speak  Gardez-vous-en bien, s'écria la jeune femme avec un sérieux qui fit frissonner d'Artagnan malgré lui. Oh! ne vous mêlez en rien de ce qui me regarde, ne cherchez point à m'aider dans ce que j'accomplis; et cela, je vous le demande au nom de l'intérêt que je vous inspire, au nom du service que vous m'avez rendu! et que je n'oublierai de ma vie. Croyez bien plutôt à ce que je vous dis. Ne vous occupez plus de moi, je n'existe plus pour vous, que ce soit comme si vous ne m'aviez jamais vue. 

\speak  Aramis doit-il en faire autant que moi, madame? dit d'Artagnan piqué. 

\speak  Voilà deux ou trois fois que vous avez prononcé ce nom, monsieur, et cependant je vous ai dit que je ne le connaissais pas. 

\speak  Vous ne connaissez pas l'homme au volet duquel vous avez été frapper. Allons donc, madame! vous me croyez par trop crédule, aussi! 

\speak  Avouez que c'est pour me faire parler que vous inventez cette histoire, et que vous créez ce personnage. 

\speak  Je n'invente rien, madame, je ne crée rien, je dis l'exacte vérité. 

\speak  Et vous dites qu'un de vos amis demeure dans cette maison? 

\speak  Je le dis et je le répète pour la troisième fois, cette maison est celle qu'habite mon ami, et cet ami est Aramis. 

\speak  Tout cela s'éclaircira plus tard, murmura la jeune femme: maintenant, monsieur, taisez-vous. 

\speak  Si vous pouviez voir mon cœur tout à découvert, dit d'Artagnan, vous y liriez tant de curiosité, que vous auriez pitié de moi, et tant d'amour, que vous satisferiez à l'instant même ma curiosité. On n'a rien à craindre de ceux qui vous aiment. 

\speak  Vous parlez bien vite d'amour, monsieur! dit la jeune femme en secouant la tête. 

\speak  C'est que l'amour m'est venu vite et pour la première fois, et que je n'ai pas vingt ans.» 

La jeune femme le regarda à la dérobée. 

«Écoutez, je suis déjà sur la trace, dit d'Artagnan. Il y a trois mois, j'ai manqué avoir un duel avec Aramis pour un mouchoir pareil à celui que vous avez montré à cette femme qui était chez lui, pour un mouchoir marqué de la même manière, j'en suis sûr. 

\speak  Monsieur, dit la jeune femme, vous me fatiguez fort, je vous le jure, avec ces questions. 

\speak  Mais vous, si prudente, madame, songez-y, si vous étiez arrêtée avec ce mouchoir, et que ce mouchoir fût saisi, ne seriez-vous pas compromise? 

\speak  Pourquoi cela, les initiales ne sont-elles pas les miennes: C.B., Constance Bonacieux? 

\speak  Ou Camille de Bois-Tracy. 

\speak  Silence, monsieur, encore une fois silence! Ah! puisque les dangers que je cours pour moi-même ne vous arrêtent pas, songez à ceux que vous pouvez courir, vous! 

\speak  Moi? 

\speak  Oui, vous. Il y a danger de la prison, il y a danger de la vie à me connaître. 

\speak  Alors, je ne vous quitte plus. 

\speak  Monsieur, dit la jeune femme suppliant et joignant les mains, monsieur, au nom du Ciel, au nom de l'honneur d'un militaire, au nom de la courtoisie d'un gentilhomme, éloignez-vous; tenez, voilà minuit qui sonne, c'est l'heure où l'on m'attend. 

\speak  Madame, dit le jeune homme en s'inclinant, je ne sais rien refuser à qui me demande ainsi; soyez contente, je m'éloigne. 

\speak  Mais vous ne me suivrez pas, vous ne m'épierez pas? 

\speak  Je rentre chez moi à l'instant. 

\speak  Ah! je le savais bien, que vous étiez un brave jeune homme!» s'écria Mme Bonacieux en lui tendant une main et en posant l'autre sur le marteau d'une petite porte presque perdue dans la muraille. 

D'Artagnan saisit la main qu'on lui tendait et la baisa ardemment. 

«Ah! j'aimerais mieux ne vous avoir jamais vue, s'écria d'Artagnan avec cette brutalité naïve que les femmes préfèrent souvent aux afféteries de la politesse, parce qu'elle découvre le fond de la pensée et qu'elle prouve que le sentiment l'emporte sur la raison. 

\speak  Eh bien, reprit Mme Bonacieux d'une voix presque caressante, et en serrant la main de d'Artagnan qui n'avait pas abandonné la sienne; eh bien, je n'en dirai pas autant que vous: ce qui est perdu pour aujourd'hui n'est pas perdu pour l'avenir. Qui sait, si lorsque je serai déliée un jour, je ne satisferai pas votre curiosité? 

\speak  Et faites-vous la même promesse à mon amour? s'écria d'Artagnan au comble de la joie. 

\speak  Oh! de ce côté, je ne veux point m'engager, cela dépendra des sentiments que vous saurez m'inspirer. 

\speak  Ainsi, aujourd'hui, madame\dots 

\speak  Aujourd'hui, monsieur, je n'en suis encore qu'à la reconnaissance. 

\speak  Ah! vous êtes trop charmante, dit d'Artagnan avec tristesse, et vous abusez de mon amour. 

\speak  Non, j'use de votre générosité, voilà tout. Mais croyez-le bien, avec certaines gens tout se retrouve. 

\speak  Oh! vous me rendez le plus heureux des hommes. N'oubliez pas cette soirée, n'oubliez pas cette promesse. 

\speak  Soyez tranquille, en temps et lieu je me souviendrai de tout. Eh bien, partez donc, partez, au nom du Ciel! On m'attendait à minuit juste, et je suis en retard. 

\speak  De cinq minutes. 

\speak  Oui; mais dans certaines circonstances, cinq minutes sont cinq siècles. 

\speak  Quand on aime. 

\speak  Eh bien, qui vous dit que je n'ai pas affaire à un amoureux? 

\speak  C'est un homme qui vous attend? s'écria d'Artagnan, un homme! 

\speak  Allons, voilà la discussion qui va recommencer, fit Mme Bonacieux avec un demi-sourire qui n'était pas exempt d'une certaine teinte d'impatience. 

\speak  Non, non, je m'en vais, je pars; je crois en vous, je veux avoir tout le mérite de mon dévouement, ce dévouement dût-il être une stupidité. Adieu, madame, adieu!» 

Et comme s'il ne se fût senti la force de se détacher de la main qu'il tenait que par une secousse, il s'éloigna tout courant, tandis que Mme Bonacieux frappait, comme au volet, trois coups lents et réguliers; puis, arrivé à l'angle de la rue, il se retourna: la porte s'était ouverte et refermée, la jolie mercière avait disparu. 

D'Artagnan continua son chemin, il avait donné sa parole de ne pas épier Mme Bonacieux, et sa vie eût-elle dépendu de l'endroit où elle allait se rendre, ou de la personne qui devait l'accompagner, d'Artagnan serait rentré chez lui, puisqu'il avait dit qu'il y rentrait. Cinq minutes après, il était dans la rue des Fossoyeurs. 

«Pauvre Athos, disait-il, il ne saura pas ce que cela veut dire. Il se sera endormi en m'attendant, ou il sera retourné chez lui, et en rentrant il aura appris qu'une femme y était venue. Une femme chez Athos! Après tout, continua d'Artagnan, il y en avait bien une chez Aramis. Tout cela est fort étrange, et je serais bien curieux de savoir comment cela finira. 

\speak  Mal, monsieur, mal», répondit une voix que le jeune homme reconnut pour celle de Planchet; car tout en monologuant tout haut, à la manière des gens très préoccupés, il s'était engagé dans l'allée au fond de laquelle était l'escalier qui conduisait à sa chambre. 

«Comment, mal? que veux-tu dire, imbécile? demanda d'Artagnan, qu'est-il donc arrivé? 

\speak  Toutes sortes de malheurs. 

\speak  Lesquels? 

\speak  D'abord M. Athos est arrêté. 

\speak  Arrêté! Athos! arrêté! pourquoi? 

\speak  On l'a trouvé chez vous; on l'a pris pour vous. 

\speak  Et par qui a-t-il été arrêté? 

\speak  Par la garde qu'ont été chercher les hommes noirs que vous avez mis en fuite. 

\speak  Pourquoi ne s'est-il pas nommé? pourquoi n'a-t-il pas dit qu'il était étranger à cette affaire? 

\speak  Il s'en est bien gardé, monsieur; il s'est au contraire approché de moi et m'a dit: «C'est ton maître qui a besoin de sa liberté en ce moment, et non pas moi, puisqu'il sait tout et que je ne sais rien. On le croira arrêté, et cela lui donnera du temps; dans trois jours je dirai qui je suis, et il faudra bien qu'on me fasse sortir.» 

\speak  Bravo, Athos! noble cœur, murmura d'Artagnan, je le reconnais bien là! Et qu'ont fait les sbires? 

\speak  Quatre l'ont emmené je ne sais où, à la Bastille ou au For-l'Évêque; deux sont restés avec les hommes noirs, qui ont fouillé partout et qui ont pris tous les papiers. Enfin les deux derniers, pendant cette expédition, montaient la garde à la porte; puis, quand tout a été fini, ils sont partis, laissant la maison vide et tout ouvert. 

\speak  Et Porthos et Aramis? 

\speak  Je ne les avais pas trouvés, ils ne sont pas venus. 

\speak  Mais ils peuvent venir d'un moment à l'autre, car tu leur as fait dire que je les attendais? 

\speak  Oui, monsieur. 

\speak  Eh bien, ne bouge pas d'ici; s'ils viennent, préviens-les de ce qui m'est arrivé, qu'ils m'attendent au cabaret de la Pomme de Pin; ici il y aurait danger, la maison peut être espionnée. Je cours chez M. de Tréville pour lui annoncer tout cela, et je les y rejoins. 

\speak  C'est bien, monsieur, dit Planchet. 

\speak  Mais tu resteras, tu n'auras pas peur! dit d'Artagnan en revenant sur ses pas pour recommander le courage à son laquais. 

\speak  Soyez tranquille, monsieur, dit Planchet, vous ne me connaissez pas encore; je suis brave quand je m'y mets, allez; c'est le tout de m'y mettre; d'ailleurs je suis Picard. 

\speak  Alors, c'est convenu, dit d'Artagnan, tu te fais tuer plutôt que de quitter ton poste. 

\speak  Oui, monsieur, et il n'y a rien que je ne fasse pour prouver à monsieur que je lui suis attaché.» 

«Bon, dit en lui-même d'Artagnan, il paraît que la méthode que j'ai employée à l'égard de ce garçon est décidément la bonne: j'en userai dans l'occasion.» 

Et de toute la vitesse de ses jambes, déjà quelque peu fatiguées cependant par les courses de la journée, d'Artagnan se dirigea vers la rue du Colombier. 

M. de Tréville n'était point à son hôtel; sa compagnie était de garde au Louvre; il était au Louvre avec sa compagnie. 

Il fallait arriver jusqu'à M. de Tréville; il était important qu'il fût prévenu de ce qui se passait. D'Artagnan résolut d'essayer d'entrer au Louvre. Son costume de garde dans la compagnie de M. des Essarts lui devait être un passeport. 

Il descendit donc la rue des Petits-Augustins, et remonta le quai pour prendre le Pont-Neuf. Il avait eu un instant l'idée de passer le bac; mais en arrivant au bord de l'eau, il avait machinalement introduit sa main dans sa poche et s'était aperçu qu'il n'avait pas de quoi payer le passeur. 

Comme il arrivait à la hauteur de la rue Guénégaud, il vit déboucher de la rue Dauphine un groupe composé de deux personnes et dont l'allure le frappa. 

Les deux personnes qui composaient le groupe étaient: l'un, un homme; l'autre, une femme. 

La femme avait la tournure de Mme Bonacieux, et l'homme ressemblait à s'y méprendre à Aramis. 

En outre, la femme avait cette mante noire que d'Artagnan voyait encore se dessiner sur le volet de la rue de Vaugirard et sur la porte de la rue de La Harpe. 

De plus, l'homme portait l'uniforme des mousquetaires. 

Le capuchon de la femme était rabattu, l'homme tenait son mouchoir sur son visage; tous deux, cette double précaution l'indiquait, tous deux avaient donc intérêt à n'être point reconnus. 

Ils prirent le pont: c'était le chemin de d'Artagnan, puisque d'Artagnan se rendait au Louvre; d'Artagnan les suivit. 

D'Artagnan n'avait pas fait vingt pas, qu'il fut convaincu que cette femme, c'était Mme Bonacieux, et que cet homme, c'était Aramis. 

Il sentit à l'instant même tous les soupçons de la jalousie qui s'agitaient dans son cœur. 

Il était doublement trahi et par son ami et par celle qu'il aimait déjà comme une maîtresse. Mme Bonacieux lui avait juré ses grands dieux qu'elle ne connaissait pas Aramis, et un quart d'heure après qu'elle lui avait fait ce serment, il la retrouvait au bras d'Aramis. 

D'Artagnan ne réfléchit pas seulement qu'il connaissait la jolie mercière depuis trois heures seulement, qu'elle ne lui devait rien qu'un peu de reconnaissance pour l'avoir délivrée des hommes noirs qui voulaient l'enlever, et qu'elle ne lui avait rien promis. Il se regarda comme un amant outragé, trahi, bafoué; le sang et la colère lui montèrent au visage, il résolut de tout éclaircir. 

La jeune femme et le jeune homme s'étaient aperçus qu'ils étaient suivis, et ils avaient doublé le pas. D'Artagnan prit sa course, les dépassa, puis revint sur eux au moment où ils se trouvaient devant la Samaritaine, éclairée par un réverbère qui projetait sa lueur sur toute cette partie du pont. 

D'Artagnan s'arrêta devant eux, et ils s'arrêtèrent devant lui. 

«Que voulez-vous, monsieur? demanda le mousquetaire en reculant d'un pas et avec un accent étranger qui prouvait à d'Artagnan qu'il s'était trompé dans une partie de ses conjectures. 

\speak  Ce n'est pas Aramis! s'écria-t-il. 

\speak  Non, monsieur, ce n'est point Aramis, et à votre exclamation je vois que vous m'avez pris pour un autre, et je vous pardonne. 

\speak  Vous me pardonnez! s'écria d'Artagnan. 

\speak  Oui, répondit l'inconnu. Laissez-moi donc passer, puisque ce n'est pas à moi que vous avez affaire. 

\speak  Vous avez raison, monsieur, dit d'Artagnan, ce n'est pas à vous que j'ai affaire, c'est à madame. 

\speak  À madame! vous ne la connaissez pas, dit l'étranger. 

\speak  Vous vous trompez, monsieur, je la connais. 

\speak  Ah! fit Mme Bonacieux d'un ton de reproche, ah monsieur! j'avais votre parole de militaire et votre foi de gentilhomme; j'espérais pouvoir compter dessus. 

\speak  Et moi, madame, dit d'Artagnan embarrassé, vous m'aviez promis\dots 

\speak  Prenez mon bras, madame, dit l'étranger, et continuons notre chemin.» 

Cependant d'Artagnan, étourdi, atterré, anéanti par tout ce qui lui arrivait, restait debout et les bras croisés devant le mousquetaire et Mme Bonacieux. 

Le mousquetaire fit deux pas en avant et écarta d'Artagnan avec la main. 

D'Artagnan fit un bond en arrière et tira son épée. 

En même temps et avec la rapidité de l'éclair, l'inconnu tira la sienne. 

«Au nom du Ciel, Milord! s'écria Mme Bonacieux en se jetant entre les combattants et prenant les épées à pleines mains. 

\speak  Milord! s'écria d'Artagnan illuminé d'une idée subite, Milord! pardon, monsieur; mais est-ce que vous seriez\dots 

\speak  Milord duc de Buckingham, dit Mme Bonacieux à demi-voix; et maintenant vous pouvez nous perdre tous. 

\speak  Milord, madame, pardon, cent fois pardon; mais je l'aimais, Milord, et j'étais jaloux; vous savez ce que c'est que d'aimer, Milord; pardonnez-moi, et dites-moi comment je puis me faire tuer pour Votre Grâce. 

\speak  Vous êtes un brave jeune homme, dit Buckingham en tendant à d'Artagnan une main que celui-ci serra respectueusement; vous m'offrez vos services, je les accepte; suivez-nous à vingt pas jusqu'au Louvre; et si quelqu'un nous épie, tuez-le!» 

D'Artagnan mit son épée nue sous son bras, laissa prendre à Mme Bonacieux et au duc vingt pas d'avance et les suivit, prêt à exécuter à la lettre les instructions du noble et élégant ministre de Charles I\ier. 

Mais heureusement le jeune séide n'eut aucune occasion de donner au duc cette preuve de son dévouement, et la jeune femme et le beau mousquetaire rentrèrent au Louvre par le guichet de l'Échelle sans avoir été inquiétés\dots 

Quant à d'Artagnan, il se rendit aussitôt au cabaret de la Pomme de Pin, où il trouva Porthos et Aramis qui l'attendaient. 

Mais, sans leur donner d'autre explication sur le dérangement qu'il leur avait causé, il leur dit qu'il avait terminé seul l'affaire pour laquelle il avait cru un instant avoir besoin de leur intervention. Et maintenant, emportés que nous sommes par notre récit, laissons nos trois amis rentrer chacun chez soi, et suivons, dans les détours du Louvre, le duc de Buckingham et son guide.
\include{chapters/12.tex}
%!TeX root=../musketeersfr.tex 

\chapter{Monsieur Bonacieux}
	
	\lettrine{I}{l} y avait dans tout cela, comme on a pu le remarquer, un personnage dont, malgré sa position précaire, on n'avait paru s'inquiéter que fort médiocrement; ce personnage était M. Bonacieux, respectable martyr des intrigues politiques et amoureuses qui s'enchevêtraient si bien les unes aux autres, dans cette époque à la fois si chevaleresque et si galante. 

Heureusement --- le lecteur se le rappelle ou ne se le rappelle pas --- heureusement que nous avons promis de ne pas le perdre de vue. 

Les estafiers qui l'avaient arrêté le conduisirent droit à la Bastille, où on le fit passer tout tremblant devant un peloton de soldats qui chargeaient leurs mousquets. 

De là, introduit dans une galerie demi-souterraine, il fut, de la part de ceux qui l'avaient amené, l'objet des plus grossières injures et des plus farouches traitements. Les sbires voyaient qu'ils n'avaient pas affaire à un gentilhomme, et ils le traitaient en véritable croquant. 

Au bout d'une demi-heure à peu près, un greffier vint mettre fin à ses tortures, mais non pas à ses inquiétudes, en donnant l'ordre de conduire M. Bonacieux dans la chambre des interrogatoires. Ordinairement on interrogeait les prisonniers chez eux, mais avec M. Bonacieux on n'y faisait pas tant de façons. 

Deux gardes s'emparèrent du mercier, lui firent traverser une cour, le firent entrer dans un corridor où il y avait trois sentinelles, ouvrirent une porte et le poussèrent dans une chambre basse, où il n'y avait pour tous meubles qu'une table, une chaise et un commissaire. Le commissaire était assis sur la chaise et occupé à écrire sur la table. 

Les deux gardes conduisirent le prisonnier devant la table et, sur un signe du commissaire, s'éloignèrent hors de la portée de la voix. 

Le commissaire, qui jusque-là avait tenu sa tête baissée sur ses papiers, la releva pour voir à qui il avait affaire. Ce commissaire était un homme à la mine rébarbative, au nez pointu, aux pommettes jaunes et saillantes, aux yeux petits mais investigateurs et vifs, à la physionomie tenant à la fois de la fouine et du renard. Sa tête, supportée par un cou long et mobile, sortait de sa large robe noire en se balançant avec un mouvement à peu près pareil à celui de la tortue tirant sa tête hors de sa carapace. 

Il commença par demander à M. Bonacieux ses nom et prénoms, son âge, son état et son domicile. 

L'accusé répondit qu'il s'appelait Jacques-Michel Bonacieux, qu'il était âgé de cinquante et un ans, mercier retiré et qu'il demeurait rue des Fossoyeurs, n° 11. 

Le commissaire alors, au lieu de continuer à l'interroger, lui fit un grand discours sur le danger qu'il y a pour un bourgeois obscur à se mêler des choses publiques. 

Il compliqua cet exorde d'une exposition dans laquelle il raconta la puissance et les actes de M. le cardinal, ce ministre incomparable, ce vainqueur des ministres passés, cet exemple des ministres à venir: actes et puissance que nul ne contrecarrait impunément. 

Après cette deuxième partie de son discours, fixant son regard d'épervier sur le pauvre Bonacieux, il l'invita à réfléchir à la gravité de sa situation. 

Les réflexions du mercier étaient toutes faites: il donnait au diable l'instant où M. de La Porte avait eu l'idée de le marier avec sa filleule, et l'instant surtout où cette filleule avait été reçue dame de la lingerie chez la reine. 

Le fond du caractère de maître Bonacieux était un profond égoïsme mêlé à une avarice sordide, le tout assaisonné d'une poltronnerie extrême. L'amour que lui avait inspiré sa jeune femme, étant un sentiment tout secondaire, ne pouvait lutter avec les sentiments primitifs que nous venons d'énumérer. 

Bonacieux réfléchit, en effet, sur ce qu'on venait de lui dire. 

«Mais, monsieur le commissaire, dit-il timidement, croyez bien que je connais et que j'apprécie plus que personne le mérite de l'incomparable Éminence par laquelle nous avons l'honneur d'être gouvernés. 

\speak  Vraiment? demanda le commissaire d'un air de doute; mais s'il en était véritablement ainsi, comment seriez-vous à la Bastille? 

\speak  Comment j'y suis, ou plutôt pourquoi j'y suis, répliqua M. Bonacieux, voilà ce qu'il m'est parfaitement impossible de vous dire, vu que je l'ignore moi-même; mais, à coup sûr, ce n'est pas pour avoir désobligé, sciemment du moins, M. le cardinal. 

\speak  Il faut cependant que vous ayez commis un crime, puisque vous êtes ici accusé de haute trahison. 

\speak  De haute trahison! s'écria Bonacieux épouvanté, de haute trahison! et comment voulez-vous qu'un pauvre mercier qui déteste les huguenots et qui abhorre les Espagnols soit accusé de haute trahison? Réfléchissez, monsieur, la chose est matériellement impossible. 

\speak  Monsieur Bonacieux, dit le commissaire en regardant l'accusé comme si ses petits yeux avaient la faculté de lire jusqu'au plus profond des cœurs, monsieur Bonacieux, vous avez une femme? 

\speak  Oui, monsieur, répondit le mercier tout tremblant, sentant que c'était là où les affaires allaient s'embrouiller; c'est-à-dire, j'en avais une. 

\speak  Comment? vous en aviez une! qu'en avez-vous fait, si vous ne l'avez plus? 

\speak  On me l'a enlevée, monsieur. 

\speak  On vous l'a enlevée? dit le commissaire. Ah!» 

Bonacieux sentit à ce «ah!» que l'affaire s'embrouillait de plus en plus. 

«On vous l'a enlevée! reprit le commissaire, et savez-vous quel est l'homme qui a commis ce rapt? 

\speak  Je crois le connaître. 

\speak  Quel est-il? 

\speak  Songez que je n'affirme rien, monsieur le commissaire, et que je soupçonne seulement. 

\speak  Qui soupçonnez-vous? Voyons, répondez franchement.» 

M. Bonacieux était dans la plus grande perplexité: devait-il tout nier ou tout dire? En niant tout, on pouvait croire qu'il en savait trop long pour avouer; en disant tout, il faisait preuve de bonne volonté. Il se décida donc à tout dire. 

«Je soupçonne, dit-il, un grand brun, de haute mine, lequel a tout à fait l'air d'un grand seigneur; il nous a suivis plusieurs fois, à ce qu'il m'a semblé, quand j'attendais ma femme devant le guichet du Louvre pour la ramener chez moi.» 

Le commissaire parut éprouver quelque inquiétude. 

«Et son nom? dit-il. 

\speak  Oh! quant à son nom, je n'en sais rien, mais si je le rencontre jamais, je le reconnaîtrai à l'instant même, je vous en réponds, fût-il entre mille personnes.» 

Le front du commissaire se rembrunit. 

«Vous le reconnaîtriez entre mille, dites-vous? continua-t-il\dots 

\speak  C'est-à-dire, reprit Bonacieux, qui vit qu'il avait fait fausse route, c'est-à-dire\dots 

\speak  Vous avez répondu que vous le reconnaîtriez, dit le commissaire; c'est bien, en voici assez pour aujourd'hui; il faut, avant que nous allions plus loin, que quelqu'un soit prévenu que vous connaissez le ravisseur de votre femme. 

\speak  Mais je ne vous ai pas dit que je le connaissais! s'écria Bonacieux au désespoir. Je vous ai dit au contraire\dots 

\speak  Emmenez le prisonnier, dit le commissaire aux deux gardes. 

\speak  Et où faut-il le conduire? demanda le greffier. 

\speak  Dans un cachot. 

\speak  Dans lequel? 

\speak  Oh! mon Dieu, dans le premier venu, pourvu qu'il ferme bien», répondit le commissaire avec une indifférence qui pénétra d'horreur le pauvre Bonacieux. 

«Hélas! hélas! se dit-il, le malheur est sur ma tête; ma femme aura commis quelque crime effroyable; on me croit son complice, et l'on me punira avec elle: elle en aura parlé, elle aura avoué qu'elle m'avait tout dit; une femme, c'est si faible! Un cachot, le premier venu! c'est cela! une nuit est bientôt passée; et demain, à la roue, à la potence! Oh! mon Dieu! mon Dieu! ayez pitié de moi!» 

Sans écouter le moins du monde les lamentations de maître Bonacieux, lamentations auxquelles d'ailleurs ils devaient être habitués, les deux gardes prirent le prisonnier par un bras, et l'emmenèrent, tandis que le commissaire écrivait en hâte une lettre que son greffier attendait. 

Bonacieux ne ferma pas l'œil, non pas que son cachot fût par trop désagréable, mais parce que ses inquiétudes étaient trop grandes. Il resta toute la nuit sur son escabeau, tressaillant au moindre bruit; et quand les premiers rayons du jour se glissèrent dans sa chambre, l'aurore lui parut avoir pris des teintes funèbres. 

Tout à coup, il entendit tirer les verrous, et il fit un soubresaut terrible. Il croyait qu'on venait le chercher pour le conduire à l'échafaud; aussi, lorsqu'il vit purement et simplement paraître, au lieu de l'exécuteur qu'il attendait, son commissaire et son greffier de la veille, il fut tout près de leur sauter au cou. 

«Votre affaire s'est fort compliquée depuis hier au soir, mon brave homme, lui dit le commissaire, et je vous conseille de dire toute la vérité; car votre repentir peut seul conjurer la colère du cardinal. 

\speak  Mais je suis prêt à tout dire, s'écria Bonacieux, du moins tout ce que je sais. Interrogez, je vous prie. 

\speak  Où est votre femme, d'abord? 

\speak  Mais puisque je vous ai dit qu'on me l'avait enlevée. 

\speak  Oui, mais depuis hier cinq heures de l'après-midi, grâce à vous, elle s'est échappée. 

\speak  Ma femme s'est échappée! s'écria Bonacieux. Oh! la malheureuse! monsieur, si elle s'est échappée, ce n'est pas ma faute, je vous le jure. 

\speak  Qu'alliez-vous donc alors faire chez M. d'Artagnan votre voisin, avec lequel vous avez eu une longue conférence dans la journée? 

\speak  Ah! oui, monsieur le commissaire, oui, cela est vrai, et j'avoue que j'ai eu tort. J'ai été chez M. d'Artagnan. 

\speak  Quel était le but de cette visite? 

\speak  De le prier de m'aider à retrouver ma femme. Je croyais que j'avais droit de la réclamer; je me trompais, à ce qu'il paraît, et je vous en demande bien pardon. 

\speak  Et qu'a répondu M. d'Artagnan? 

\speak  M. d'Artagnan m'a promis son aide; mais je me suis bientôt aperçu qu'il me trahissait. 

\speak  Vous en imposez à la justice! M. d'Artagnan a fait un pacte avec vous, et en vertu de ce pacte il a mis en fuite les hommes de police qui avaient arrêté votre femme, et l'a soustraite à toutes les recherches. 

\speak  M. d'Artagnan a enlevé ma femme! Ah çà, mais que me dites-vous là? 

\speak  Heureusement M. d'Artagnan est entre nos mains, et vous allez lui être confronté. 

\speak  Ah! ma foi, je ne demande pas mieux, s'écria Bonacieux; je ne serais pas fâché de voir une figure de connaissance. 

\speak  Faites entrer M. d'Artagnan», dit le commissaire aux deux gardes. 

Les deux gardes firent entrer Athos. 

«Monsieur d'Artagnan, dit le commissaire en s'adressant à Athos, déclarez ce qui s'est passé entre vous et monsieur. 

\speak  Mais! s'écria Bonacieux, ce n'est pas M. d'Artagnan que vous me montrez là! 

\speak  Comment! ce n'est pas M. d'Artagnan? s'écria le commissaire. 

\speak  Pas le moins du monde, répondit Bonacieux. 

\speak  Comment se nomme monsieur? demanda le commissaire. 

\speak  Je ne puis vous le dire, je ne le connais pas. 

\speak  Comment! vous ne le connaissez pas? 

\speak  Non. 

\speak  Vous ne l'avez jamais vu? 

\speak  Si fait; mais je ne sais comment il s'appelle. 

\speak  Votre nom? demanda le commissaire. 

\speak  Athos, répondit le mousquetaire. 

\speak  Mais ce n'est pas un nom d'homme, ça, c'est un nom de montagne! s'écria le pauvre interrogateur qui commençait à perdre la tête. 

\speak  C'est mon nom, dit tranquillement Athos. 

\speak  Mais vous avez dit que vous vous nommiez d'Artagnan. 

\speak  Moi? 

\speak  Oui, vous. 

\speak  C'est-à-dire que c'est à moi qu'on a dit: «Vous êtes M. d'Artagnan?» J'ai répondu: «Vous croyez?» Mes gardes se sont écriés qu'ils en étaient sûrs. Je n'ai pas voulu les contrarier. D'ailleurs je pouvais me tromper. 

\speak  Monsieur, vous insultez à la majesté de la justice. 

\speak  Aucunement, fit tranquillement Athos. 

\speak  Vous êtes M. d'Artagnan. 

\speak  Vous voyez bien que vous me le dites encore. 

\speak  Mais, s'écria à son tour M. Bonacieux, je vous dis, monsieur le commissaire, qu'il n'y a pas un instant de doute à avoir. M. d'Artagnan est mon hôte, et par conséquent, quoiqu'il ne me paie pas mes loyers, et justement même à cause de cela, je dois le connaître. M. d'Artagnan est un jeune homme de dix-neuf à vingt ans à peine, et monsieur en a trente au moins. M. d'Artagnan est dans les gardes de M. des Essarts, et monsieur est dans la compagnie des mousquetaires de M. de Tréville: regardez l'uniforme, monsieur le commissaire, regardez l'uniforme. 

\speak  C'est vrai, murmura le commissaire; c'est pardieu vrai.» 

En ce moment la porte s'ouvrit vivement, et un messager, introduit par un des guichetiers de la Bastille, remit une lettre au commissaire. 

«Oh! la malheureuse! s'écria le commissaire. 

\speak  Comment? que dites-vous? de qui parlez-vous? Ce n'est pas de ma femme, j'espère! 

\speak  Au contraire, c'est d'elle. Votre affaire est bonne, allez. 

\speak  Ah çà, s'écria le mercier exaspéré, faites-moi le plaisir de me dire, monsieur, comment mon affaire à moi peut s'empirer de ce que fait ma femme pendant que je suis en prison! 

\speak  Parce que ce qu'elle fait est la suite d'un plan arrêté entre vous, plan infernal! 

\speak  Je vous jure, monsieur le commissaire, que vous êtes dans la plus profonde erreur, que je ne sais rien au monde de ce que devait faire ma femme, que je suis entièrement étranger à ce qu'elle a fait, et que, si elle a fait des sottises, je la renie, je la démens, je la maudis. 

\speak  Ah çà, dit Athos au commissaire, si vous n'avez plus besoin de moi ici, renvoyez-moi quelque part, il est très ennuyeux, votre monsieur Bonacieux. 

\speak  Reconduisez les prisonniers dans leurs cachots, dit le commissaire en désignant d'un même geste Athos et Bonacieux, et qu'ils soient gardés plus sévèrement que jamais. 

\speak  Cependant, dit Athos avec son calme habituel, si c'est à M. d'Artagnan que vous avez affaire, je ne vois pas trop en quoi je puis le remplacer. 

\speak  Faites ce que j'ai dit! s'écria le commissaire, et le secret le plus absolu! Vous entendez!» 

Athos suivit ses gardes en levant les épaules, et M. Bonacieux en poussant des lamentations à fendre le cœur d'un tigre. 

On ramena le mercier dans le même cachot où il avait passé la nuit, et l'on l'y laissa toute la journée. Toute la journée Bonacieux pleura comme un véritable mercier, n'étant pas du tout homme d'épée, il nous l'a dit lui-même. 

Le soir, vers les neuf heures, au moment où il allait se décider à se mettre au lit, il entendit des pas dans son corridor. Ces pas se rapprochèrent de son cachot, sa porte s'ouvrit, des gardes parurent. 

«Suivez-moi, dit un exempt qui venait à la suite des gardes. 

\speak  Vous suivre! s'écria Bonacieux; vous suivre à cette heure-ci! et où cela, mon Dieu? 

\speak  Où nous avons l'ordre de vous conduire. 

\speak  Mais ce n'est pas une réponse, cela. 

\speak  C'est cependant la seule que nous puissions vous faire. 

\speak  Ah! mon Dieu, mon Dieu, murmura le pauvre mercier, pour cette fois je suis perdu!» 

Et il suivit machinalement et sans résistance les gardes qui venaient le quérir. 

Il prit le même corridor qu'il avait déjà pris, traversa une première cour, puis un second corps de logis; enfin, à la porte de la cour d'entrée, il trouva une voiture entourée de quatre gardes à cheval. On le fit monter dans cette voiture, l'exempt se plaça près de lui, on ferma la portière à clef, et tous deux se trouvèrent dans une prison roulante. 

La voiture se mit en mouvement, lente comme un char funèbre. À travers la grille cadenassée, le prisonnier apercevait les maisons et le pavé, voilà tout; mais, en véritable Parisien qu'il était, Bonacieux reconnaissait chaque rue aux bornes, aux enseignes, aux réverbères. Au moment d'arriver à Saint-Paul, lieu où l'on exécutait les condamnés de la Bastille, il faillit s'évanouir et se signa deux fois. Il avait cru que la voiture devait s'arrêter là. La voiture passa cependant. 

Plus loin, une grande terreur le prit encore, ce fut en côtoyant le cimetière Saint-Jean où on enterrait les criminels d'État. Une seule chose le rassura un peu, c'est qu'avant de les enterrer on leur coupait généralement la tête, et que sa tête à lui était encore sur ses épaules. Mais lorsqu'il vit que la voiture prenait la route de la Grève, qu'il aperçut les toits aigus de l'hôtel de ville, que la voiture s'engagea sous l'arcade, il crut que tout était fini pour lui, voulut se confesser à l'exempt, et, sur son refus, poussa des cris si pitoyables que l'exempt annonça que, s'il continuait à l'assourdir ainsi, il lui mettrait un bâillon. 

Cette menace rassura quelque peu Bonacieux: si l'on eût dû l'exécuter en Grève, ce n'était pas la peine de le bâillonner, puisqu'on était presque arrivé au lieu de l'exécution. En effet, la voiture traversa la place fatale sans s'arrêter. Il ne restait plus à craindre que la Croix-du-Trahoir: la voiture en prit justement le chemin. 

Cette fois, il n'y avait plus de doute, c'était à la Croix-du-Trahoir qu'on exécutait les criminels subalternes. Bonacieux s'était flatté en se croyant digne de Saint-Paul ou de la place de Grève: c'était à la Croix-du-Trahoir qu'allaient finir son voyage et sa destinée! Il ne pouvait voir encore cette malheureuse croix, mais il la sentait en quelque sorte venir au-devant de lui. Lorsqu'il n'en fut plus qu'à une vingtaine de pas, il entendit une rumeur, et la voiture s'arrêta. C'était plus que n'en pouvait supporter le pauvre Bonacieux, déjà écrasé par les émotions successives qu'il avait éprouvées; il poussa un faible gémissement, qu'on eût pu prendre pour le dernier soupir d'un moribond, et il s'évanouit.
%!TeX root=../musketeersfr.tex 

\chapter{L'Homme De Meung}

\lettrine{C}{e} rassemblement était produit non point par l'attente d'un homme qu'on devait pendre, mais par la contemplation d'un pendu. 

\zz
La voiture, arrêtée un instant, reprit donc sa marche, traversa la foule, continua son chemin, enfila la rue Saint-Honoré, tourna la rue des Bons-Enfants et s'arrêta devant une porte basse. 

La porte s'ouvrit, deux gardes reçurent dans leurs bras Bonacieux, soutenu par l'exempt; on le poussa dans une allée, on lui fit monter un escalier, et on le déposa dans une antichambre. 

Tous ces mouvements s'étaient opérés pour lui d'une façon machinale. 

Il avait marché comme on marche en rêve; il avait entrevu les objets à travers un brouillard; ses oreilles avaient perçu des sons sans les comprendre; on eût pu l'exécuter dans ce moment qu'il n'eût pas fait un geste pour entreprendre sa défense, qu'il n'eût pas poussé un cri pour implorer la pitié. 

Il resta donc ainsi sur la banquette, le dos appuyé au mur et les bras pendants, à l'endroit même où les gardes l'avaient déposé. 

Cependant, comme, en regardant autour de lui, il ne voyait aucun objet menaçant, comme rien n'indiquait qu'il courût un danger réel, comme la banquette était convenablement rembourrée, comme la muraille était recouverte d'un beau cuir de Cordoue, comme de grands rideaux de damas rouge flottaient devant la fenêtre, retenus par des embrasses d'or, il comprit peu à peu que sa frayeur était exagérée, et il commença de remuer la tête à droite et à gauche et de bas en haut. 

À ce mouvement, auquel personne ne s'opposa, il reprit un peu de courage et se risqua à ramener une jambe, puis l'autre; enfin, en s'aidant de ses deux mains, il se souleva sur sa banquette et se trouva sur ses pieds. 

En ce moment, un officier de bonne mine ouvrit une portière, continua d'échanger encore quelques paroles avec une personne qui se trouvait dans la pièce voisine, et se retournant vers le prisonnier: 

«C'est vous qui vous nommez Bonacieux? dit-il. 

\speak  Oui, monsieur l'officier, balbutia le mercier, plus mort que vif, pour vous servir. 

\speak  Entrez», dit l'officier. 

Et il s'effaça pour que le mercier pût passer. Celui-ci obéit sans réplique, et entra dans la chambre où il paraissait être attendu. 

C'était un grand cabinet, aux murailles garnies d'armes offensives et défensives, clos et étouffé, et dans lequel il y avait déjà du feu, quoique l'on fût à peine à la fin du mois de septembre. Une table carrée, couverte de livres et de papiers sur lesquels était déroulé un plan immense de la ville de La Rochelle, tenait le milieu de l'appartement. 

Debout devant la cheminée était un homme de moyenne taille, à la mine haute et fière, aux yeux perçants, au front large, à la figure amaigrie qu'allongeait encore une royale surmontée d'une paire de moustaches. Quoique cet homme eût trente-six à trente-sept ans à peine, cheveux, moustache et royale s'en allaient grisonnant. Cet homme, moins l'épée, avait toute la mine d'un homme de guerre, et ses bottes de buffle encore légèrement couvertes de poussière indiquaient qu'il avait monté à cheval dans la journée. 

Cet homme, c'était Armand-Jean Duplessis, cardinal de Richelieu, non point tel qu'on nous le représente, cassé comme un vieillard, souffrant comme un martyr, le corps brisé, la voix éteinte, enterré dans un grand fauteuil comme dans une tombe anticipée, ne vivant plus que par la force de son génie, et ne soutenant plus la lutte avec l'Europe que par l'éternelle application de sa pensée, mais tel qu'il était réellement à cette époque, c'est-à-dire adroit et galant cavalier, faible de corps déjà, mais soutenu par cette puissance morale qui a fait de lui un des hommes les plus extraordinaires qui aient existé; se préparant enfin, après avoir soutenu le duc de Nevers dans son duché de Mantoue, après avoir pris Nîmes, Castres et Uzès, à chasser les Anglais de l'île de Ré et à faire le siège de La Rochelle. 

À la première vue, rien ne dénotait donc le cardinal, et il était impossible à ceux-là qui ne connaissaient point son visage de deviner devant qui ils se trouvaient. 

Le pauvre mercier demeura debout à la porte, tandis que les yeux du personnage que nous venons de décrire se fixaient sur lui, et semblaient vouloir pénétrer jusqu'au fond du passé. 

«C'est là ce Bonacieux? demanda-t-il après un moment de silence. 

\speak  Oui, Monseigneur, reprit l'officier. 

\speak  C'est bien, donnez-moi ces papiers et laissez-nous.» 

L'officier prit sur la table les papiers désignés, les remit à celui qui les demandait, s'inclina jusqu'à terre, et sortit. 

Bonacieux reconnut dans ces papiers ses interrogatoires de la Bastille. De temps en temps, l'homme de la cheminée levait les yeux de dessus les écritures, et les plongeait comme deux poignards jusqu'au fond du cœur du pauvre mercier. 

Au bout de dix minutes de lecture et dix secondes d'examen, le cardinal était fixé. 

«Cette tête-là n'a jamais conspiré», murmura-t-il; mais n'importe, voyons toujours. 

\speak  Vous êtes accusé de haute trahison, dit lentement le cardinal. 

\speak  C'est ce qu'on m'a déjà appris, Monseigneur, s'écria Bonacieux, donnant à son interrogateur le titre qu'il avait entendu l'officier lui donner; mais je vous jure que je n'en savais rien.» 

Le cardinal réprima un sourire. 

«Vous avez conspiré avec votre femme, avec Mme de Chevreuse et avec Milord duc de Buckingham. 

\speak  En effet, Monseigneur, répondit le mercier, je l'ai entendue prononcer tous ces noms-là. 

\speak  Et à quelle occasion? 

\speak  Elle disait que le cardinal de Richelieu avait attiré le duc de Buckingham à Paris pour le perdre et pour perdre la reine avec lui. 

\speak  Elle disait cela? s'écria le cardinal avec violence. 

\speak  Oui, Monseigneur; mais moi je lui ai dit qu'elle avait tort de tenir de pareils propos, et que Son Éminence était incapable\dots 

\speak  Taisez-vous, vous êtes un imbécile, reprit le cardinal. 

\speak  C'est justement ce que ma femme m'a répondu, Monseigneur. 

\speak  Savez-vous qui a enlevé votre femme? 

\speak  Non, Monseigneur. 

\speak  Vous avez des soupçons, cependant? 

\speak  Oui, Monseigneur; mais ces soupçons ont paru contrarier M. le commissaire, et je ne les ai plus. 

\speak  Votre femme s'est échappée, le saviez-vous? 

\speak  Non, Monseigneur, je l'ai appris depuis que je suis en prison, et toujours par l'entremise de M. le commissaire, un homme bien aimable!» 

Le cardinal réprima un second sourire. 

«Alors vous ignorez ce que votre femme est devenue depuis sa fuite? 

\speak  Absolument, Monseigneur; mais elle a dû rentrer au Louvre. 

\speak  À une heure du matin elle n'y était pas rentrée encore. 

\speak  Ah! mon Dieu! mais qu'est-elle devenue alors? 

\speak  On le saura, soyez tranquille; on ne cache rien au cardinal; le cardinal sait tout. 

\speak  En ce cas, Monseigneur, est-ce que vous croyez que le cardinal consentira à me dire ce qu'est devenue ma femme? 

\speak  Peut-être; mais il faut d'abord que vous avouiez tout ce que vous savez relativement aux relations de votre femme avec Mme de Chevreuse. 

\speak  Mais, Monseigneur, je n'en sais rien; je ne l'ai jamais vue. 

\speak  Quand vous alliez chercher votre femme au Louvre, revenait-elle directement chez vous? 

\speak  Presque jamais: elle avait affaire à des marchands de toile, chez lesquels je la conduisais. 

\speak  Et combien y en avait-il de marchands de toile? 

\speak  Deux, Monseigneur. 

\speak  Où demeurent-ils? 

\speak  Un, rue de Vaugirard; l'autre, rue de La Harpe. 

\speak  Entriez-vous chez eux avec elle? 

\speak  Jamais, Monseigneur; je l'attendais à la porte. 

\speak  Et quel prétexte vous donnait-elle pour entrer ainsi toute seule? 

\speak  Elle ne m'en donnait pas; elle me disait d'attendre, et j'attendais. 

\speak  Vous êtes un mari complaisant, mon cher monsieur Bonacieux!» dit le cardinal. 

«Il m'appelle son cher monsieur! dit en lui-même le mercier. Peste! les affaires vont bien!» 

«Reconnaîtriez-vous ces portes? 

\speak  Oui. 

\speak  Savez-vous les numéros? 

\speak  Oui. 

\speak  Quels sont-ils? 

\speak  N° 25, dans la rue de Vaugirard; n° 75, dans la rue de La Harpe. 

\speak  C'est bien», dit le cardinal. 

À ces mots, il prit une sonnette d'argent, et sonna; l'officier rentra. 

«Allez, dit-il à demi-voix, me chercher Rochefort; et qu'il vienne à l'instant même, s'il est rentré. 

\speak  Le comte est là, dit l'officier, il demande instamment à parler à Votre Éminence!» 

«À Votre Éminence! murmura Bonacieux, qui savait que tel était le titre qu'on donnait d'ordinaire à M. le cardinal;\dots à Votre Éminence!» 

«Qu'il vienne alors, qu'il vienne!» dit vivement Richelieu. 

L'officier s'élança hors de l'appartement, avec cette rapidité que mettaient d'ordinaire tous les serviteurs du cardinal à lui obéir. 

«À Votre Éminence!» murmurait Bonacieux en roulant des yeux égarés. 

Cinq secondes ne s'étaient pas écoulées depuis la disparition de l'officier, que la porte s'ouvrit et qu'un nouveau personnage entra. 

«C'est lui, s'écria Bonacieux. 

\speak  Qui lui? demanda le cardinal. 

\speak  Celui qui m'a enlevé ma femme.» 

Le cardinal sonna une seconde fois. L'officier reparut. 

«Remettez cet homme aux mains de ses deux gardes, et qu'il attende que je le rappelle devant moi. 

\speak  Non, Monseigneur! non, ce n'est pas lui! s'écria Bonacieux; non, je m'étais trompé: c'est un autre qui ne lui ressemble pas du tout! Monsieur est un honnête homme. 

\speak  Emmenez cet imbécile!» dit le cardinal. 

L'officier prit Bonacieux sous le bras, et le reconduisit dans l'antichambre où il trouva ses deux gardes. 

Le nouveau personnage qu'on venait d'introduire suivit des yeux avec impatience Bonacieux jusqu'à ce qu'il fût sorti, et dès que la porte se fut refermée sur lui: 

«Ils se sont vus, dit-il en s'approchant vivement du cardinal. 

\speak  Qui? demanda Son Éminence. 

\speak  Elle et lui. 

\speak  La reine et le duc? s'écria Richelieu. 

\speak  Oui. 

\speak  Et où cela? 

\speak  Au Louvre. 

\speak  Vous en êtes sûr? 

\speak  Parfaitement sûr. 

\speak  Qui vous l'a dit? 

\speak  Mme de Lannoy, qui est toute à Votre Éminence, comme vous le savez. 

\speak  Pourquoi ne l'a-t-elle pas dit plus tôt? 

\speak  Soit hasard, soit défiance, la reine a fait coucher Mme de Fargis dans sa chambre, et l'a gardée toute la journée. 

\speak  C'est bien, nous sommes battus. Tâchons de prendre notre revanche. 

\speak  Je vous y aiderai de toute mon âme, Monseigneur, soyez tranquille. 

\speak  Comment cela s'est-il passé? 

\speak  À minuit et demi, la reine était avec ses femmes\dots 

\speak  Où cela? 

\speak  Dans sa chambre à coucher\dots 

\speak  Bien. 

\speak  Lorsqu'on est venu lui remettre un mouchoir de la part de sa dame de lingerie\dots 

\speak  Après? 

\speak  Aussitôt la reine a manifesté une grande émotion, et, malgré le rouge dont elle avait le visage couvert, elle a pâli. 

\speak  Après! après! 

\speak  Cependant, elle s'est levée, et d'une voix altérée: «Mesdames, a-t-elle dit, attendez-moi dix minutes, puis je reviens.» Et elle a ouvert la porte de son alcôve, puis elle est sortie. 

\speak  Pourquoi Mme de Lannoy n'est-elle pas venue vous prévenir à l'instant même? 

\speak  Rien n'était bien certain encore; d'ailleurs, la reine avait dit: «Mesdames, attendez-moi»; et elle n'osait désobéir à la reine. 

\speak  Et combien de temps la reine est-elle restée hors de la chambre? 

\speak  Trois quarts d'heure. 

\speak  Aucune de ses femmes ne l'accompagnait? 

\speak  Doña Estefania seulement. 

\speak  Et elle est rentrée ensuite? 

\speak  Oui, mais pour prendre un petit coffret de bois de rose à son chiffre, et sortir aussitôt. 

\speak  Et quand elle est rentrée, plus tard, a-t-elle rapporté le coffret? 

\speak  Non. 

\speak  Mme de Lannoy savait-elle ce qu'il y avait dans ce coffret? 

\speak  Oui: les ferrets en diamants que Sa Majesté a donnés à la reine. 

\speak  Et elle est rentrée sans ce coffret? 

\speak  Oui. 

\speak  L'opinion de Mme de Lannoy est qu'elle les a remis alors à Buckingham? 

\speak  Elle en est sûre. 

\speak  Comment cela? 

\speak  Pendant la journée, Mme de Lannoy, en sa qualité de dame d'atour de la reine, a cherché ce coffret, a paru inquiète de ne pas le trouver et a fini par en demander des nouvelles à la reine. 

\speak  Et alors, la reine\dots? 

\speak  La reine est devenue fort rouge et a répondu qu'ayant brisé la veille un de ses ferrets, elle l'avait envoyé raccommoder chez son orfèvre. 

\speak  Il faut y passer et s'assurer si la chose est vraie ou non. 

\speak  J'y suis passé. 

\speak  Eh bien, l'orfèvre? 

\speak  L'orfèvre n'a entendu parler de rien. 

\speak  Bien! bien! Rochefort, tout n'est pas perdu, et peut-être\dots peut-être tout est-il pour le mieux! 

\speak  Le fait est que je ne doute pas que le génie de Votre Éminence\dots 

\speak  Ne répare les bêtises de mon agent, n'est-ce pas? 

\speak  C'est justement ce que j'allais dire, si Votre Éminence m'avait laissé achever ma phrase. 

\speak  Maintenant, savez-vous où se cachaient la duchesse de Chevreuse et le duc de Buckingham? 

\speak  Non, Monseigneur, mes gens n'ont pu rien me dire de positif là-dessus. 

\speak  Je le sais, moi. 

\speak  Vous, Monseigneur? 

\speak  Oui, ou du moins je m'en doute. Ils se tenaient, l'un rue de Vaugirard, n° 25, et l'autre rue de La Harpe, n° 75. 

\speak  Votre Éminence veut-elle que je les fasse arrêter tous deux? 

\speak  Il sera trop tard, ils seront partis. 

\speak  N'importe, on peut s'en assurer. 

\speak  Prenez dix hommes de mes gardes, et fouillez les deux maisons. 

\speak  J'y vais, Monseigneur.» 

Et Rochefort s'élança hors de l'appartement. 

Le cardinal, resté seul, réfléchit un instant et sonna une troisième fois. 

Le même officier reparut. 

«Faites entrer le prisonnier», dit le cardinal. 

Maître Bonacieux fut introduit de nouveau, et, sur un signe du cardinal, l'officier se retira. 

«Vous m'avez trompé, dit sévèrement le cardinal. 

\speak  Moi, s'écria Bonacieux, moi, tromper Votre Éminence! 

\speak  Votre femme, en allant rue de Vaugirard et rue de La Harpe, n'allait pas chez des marchands de toile. 

\speak  Et où allait-elle, juste Dieu? 

\speak  Elle allait chez la duchesse de Chevreuse et chez le duc de Buckingham. 

\speak  Oui, dit Bonacieux rappelant tous ses souvenirs; oui, c'est cela, Votre Éminence a raison. J'ai dit plusieurs fois à ma femme qu'il était étonnant que des marchands de toile demeurassent dans des maisons pareilles, dans des maisons qui n'avaient pas d'enseignes, et chaque fois ma femme s'est mise à rire. Ah! Monseigneur, continua Bonacieux en se jetant aux pieds de l'Éminence, ah! que vous êtes bien le cardinal, le grand cardinal, l'homme de génie que tout le monde révère.» 

Le cardinal, tout médiocre qu'était le triomphe remporté sur un être aussi vulgaire que l'était Bonacieux, n'en jouit pas moins un instant; puis, presque aussitôt, comme si une nouvelle pensée se présentait à son esprit, un sourire plissa ses lèvres, et tendant la main au mercier: 

«Relevez-vous, mon ami, lui dit-il, vous êtes un brave homme. 

\speak  Le cardinal m'a touché la main! j'ai touché la main du grand homme! s'écria Bonacieux; le grand homme m'a appelé son ami! 

\speak  Oui, mon ami; oui! dit le cardinal avec ce ton paterne qu'il savait prendre quelquefois, mais qui ne trompait que les gens qui ne le connaissaient pas; et comme on vous a soupçonné injustement, eh bien, il vous faut une indemnité: tenez! prenez ce sac de cent pistoles, et pardonnez-moi. 

\speak  Que je vous pardonne, Monseigneur! dit Bonacieux hésitant à prendre le sac, craignant sans doute que ce prétendu don ne fût qu'une plaisanterie. Mais vous étiez bien libre de me faire arrêter, vous êtes bien libre de me faire torturer, vous êtes bien libre de me faire pendre: vous êtes le maître, et je n'aurais pas eu le plus petit mot à dire. Vous pardonner, Monseigneur! Allons donc, vous n'y pensez pas! 

\speak  Ah! mon cher monsieur Bonacieux! vous y mettez de la générosité, je le vois, et je vous en remercie. Ainsi donc, vous prenez ce sac, et vous vous en allez sans être trop mécontent? 

\speak  Je m'en vais enchanté, Monseigneur. 

\speak  Adieu donc, ou plutôt à revoir, car j'espère que nous nous reverrons. 

\speak  Tant que Monseigneur voudra, et je suis bien aux ordres de Son Éminence. 

\speak  Ce sera souvent, soyez tranquille, car j'ai trouvé un charme extrême à votre conversation. 

\speak  Oh! Monseigneur! 

\speak  Au revoir, monsieur Bonacieux, au revoir. 

Et le cardinal lui fit un signe de la main, auquel Bonacieux répondit en s'inclinant jusqu'à terre; puis il sortit à reculons, et quand il fut dans l'antichambre, le cardinal l'entendit qui, dans son enthousiasme, criait à tue-tête: «Vive Monseigneur! vive Son Éminence! vive le grand cardinal!» Le cardinal écouta en souriant cette brillante manifestation des sentiments enthousiastes de maître Bonacieux; puis, quand les cris de Bonacieux se furent perdus dans l'éloignement: 

«Bien, dit-il, voici désormais un homme qui se fera tuer pour moi.» 

Et le cardinal se mit à examiner avec la plus grande attention la carte de La Rochelle qui, ainsi que nous l'avons dit, était étendue sur son bureau, traçant avec un crayon la ligne où devait passer la fameuse digue qui, dix-huit mois plus tard, fermait le port de la cité assiégée. 

Comme il en était au plus profond de ses méditations stratégiques, la porte se rouvrit, et Rochefort rentra. 

«Eh bien? dit vivement le cardinal en se levant avec une promptitude qui prouvait le degré d'importance qu'il attachait à la commission dont il avait chargé le comte. 

\speak  Eh bien, dit celui-ci, une jeune femme de vingt-six à vingt-huit ans et un homme de trente-cinq à quarante ans ont logé effectivement, l'un quatre jours et l'autre cinq, dans les maisons indiquées par Votre Éminence: mais la femme est partie cette nuit, et l'homme ce matin. 

\speak  C'étaient eux! s'écria le cardinal, qui regardait à la pendule; et maintenant, continua-t-il, il est trop tard pour faire courir après: la duchesse est à Tours, et le duc à Boulogne. C'est à Londres qu'il faut les rejoindre. 

\speak  Quels sont les ordres de Votre Éminence? 

\speak  Pas un mot de ce qui s'est passé; que la reine reste dans une sécurité parfaite; qu'elle ignore que nous savons son secret; qu'elle croie que nous sommes à la recherche d'une conspiration quelconque. Envoyez-moi le garde des sceaux Séguier. 

\speak  Et cet homme, qu'en a fait Votre Éminence? 

\speak  Quel homme? demanda le cardinal. 

\speak  Ce Bonacieux? 

\speak  J'en ai fait tout ce qu'on pouvait en faire. J'en ai fait l'espion de sa femme.» 

Le comte de Rochefort s'inclina en homme qui reconnaît la grande supériorité du maître, et se retira. 

Resté seul, le cardinal s'assit de nouveau, écrivit une lettre qu'il cacheta de son sceau particulier, puis il sonna. L'officier entra pour la quatrième fois. 

«Faites-moi venir Vitray, dit-il, et dites-lui de s'apprêter pour un voyage.» 

Un instant après, l'homme qu'il avait demandé était debout devant lui, tout botté et tout éperonné. 

«Vitray, dit-il, vous allez partir tout courant pour Londres. Vous ne vous arrêterez pas un instant en route. Vous remettrez cette lettre à Milady. Voici un bon de deux cents pistoles, passez chez mon trésorier et faites-vous payer. Il y en a autant à toucher si vous êtes ici de retour dans six jours et si vous avez bien fait ma commission.» 

Le messager, sans répondre un seul mot, s'inclina, prit la lettre, le bon de deux cents pistoles, et sortit. 

Voici ce que contenait la lettre:  «Milady, 

«Trouvez-vous au premier bal où se trouvera le duc de Buckingham. Il aura à son pourpoint douze ferrets de diamants, approchez-vous de lui et coupez-en deux. 

«Aussitôt que ces ferrets seront en votre possession, prévenez-moi.» 
\include{chapters/15.tex}
%!TeX root=../musketeersfr.tex 

\chapter[Où M. Séguier Chercha La Cloche]{Où M. Le Garde Des Sceaux Séguier Chercha Plus D'Une Fois La Cloche Pour La Sonner, Comme Il Le Faisait Autrefois}
	
\chaptermark{Où M. Séguier Chercha La Cloche}
	
\lettrine{I}{l} est impossible de se faire une idée de l'impression que ces quelques mots produisirent sur Louis XIII. Il rougit et pâlit successivement; et le cardinal vit tout d'abord qu'il venait de conquérir d'un seul coup tout le terrain qu'il avait perdu. 

«M. de Buckingham à Paris! s'écria-t-il, et qu'y vient-il faire? 

\speak  Sans doute conspirer avec nos ennemis les huguenots et les Espagnols. 

\speak  Non, pardieu, non! conspirer contre mon honneur avec Mme de Chevreuse, Mme de Longueville et les Condé! 

\speak  Oh! Sire, quelle idée! La reine est trop sage, et surtout aime trop Votre Majesté. 

\speak  La femme est faible, monsieur le cardinal, dit le roi; et quant à m'aimer beaucoup, j'ai mon opinion faite sur cet amour. 

\speak  Je n'en maintiens pas moins, dit le cardinal, que le duc de Buckingham est venu à Paris pour un projet tout politique. 

\speak  Et moi je suis sûr qu'il est venu pour autre chose, monsieur le cardinal; mais si la reine est coupable, qu'elle tremble! 

\speak  Au fait, dit le cardinal, quelque répugnance que j'aie à arrêter mon esprit sur une pareille trahison, Votre Majesté m'y fait penser: Mme de Lannoy, que, d'après l'ordre de Votre Majesté, j'ai interrogée plusieurs fois, m'a dit ce matin que la nuit avant celle-ci Sa Majesté avait veillé fort tard, que ce matin elle avait beaucoup pleuré et que toute la journée elle avait écrit. 

\speak  C'est cela, dit le roi; à lui sans doute, Cardinal, il me faut les papiers de la reine. 

\speak  Mais comment les prendre, Sire? Il me semble que ce n'est ni moi, ni Votre Majesté qui pouvons nous charger d'une pareille mission. 

\speak  Comment s'y est-on pris pour la maréchale d'Ancre? s'écria le roi au plus haut degré de la colère; on a fouillé ses armoires, et enfin on l'a fouillée elle-même. 

\speak  La maréchale d'Ancre n'était que la maréchale d'Ancre, une aventurière florentine, Sire, voilà tout; tandis que l'auguste épouse de Votre Majesté est Anne d'Autriche, reine de France, c'est-à-dire une des plus grandes princesses du monde. 

\speak  Elle n'en est que plus coupable, monsieur le duc! Plus elle a oublié la haute position où elle était placée, plus elle est bas descendue. Il y a longtemps d'ailleurs que je suis décidé à en finir avec toutes ces petites intrigues de politique et d'amour. Elle a aussi près d'elle un certain La Porte\dots 

\speak  Que je crois la cheville ouvrière de tout cela, je l'avoue, dit le cardinal. 

\speak  Vous pensez donc, comme moi, qu'elle me trompe? dit le roi. 

\speak  Je crois, et je le répète à Votre Majesté, que la reine conspire contre la puissance de son roi, mais je n'ai point dit contre son honneur. 

\speak  Et moi je vous dis contre tous deux; moi je vous dis que la reine ne m'aime pas; je vous dis qu'elle en aime un autre; je vous dis qu'elle aime cet infâme duc de Buckingham! Pourquoi ne l'avez-vous pas fait arrêter pendant qu'il était à Paris? 

\speak  Arrêter le duc! arrêter le premier ministre du roi Charles I\ier\! Y pensez-vous, Sire? Quel éclat! et si alors les soupçons de Votre Majesté, ce dont je continue à douter, avaient quelque consistance, quel éclat terrible! quel scandale désespérant! 

\speak  Mais puisqu'il s'exposait comme un vagabond et un larronneur, il fallait\dots» 

Louis XIII s'arrêta lui-même, effrayé de ce qu'il allait dire, tandis que Richelieu, allongeant le cou, attendait inutilement la parole qui était restée sur les lèvres du roi. 

«Il fallait? 

\speak  Rien, dit le roi, rien. Mais, pendant tout le temps qu'il a été à Paris, vous ne l'avez pas perdu de vue? 

\speak  Non, Sire. 

\speak  Où logeait-il? 

\speak  Rue de La Harpe, n° 75. 

\speak  Où est-ce, cela? 

\speak  Du côté du Luxembourg. 

\speak  Et vous êtes sûr que la reine et lui ne se sont pas vus? 

\speak  Je crois la reine trop attachée à ses devoirs, Sire. 

\speak  Mais ils ont correspondu, c'est à lui que la reine a écrit toute la journée; monsieur le duc, il me faut ces lettres! 

\speak  Sire, cependant\dots 

\speak  Monsieur le duc, à quelque prix que ce soit, je les veux. 

\speak  Je ferai pourtant observer à Votre Majesté\dots 

\speak  Me trahissez-vous donc aussi, monsieur le cardinal, pour vous opposer toujours ainsi à mes volontés? êtes-vous aussi d'accord avec l'Espagnol et avec l'Anglais, avec Mme de Chevreuse et avec la reine? 

\speak  Sire, répondit en soupirant le cardinal, je croyais être à l'abri d'un pareil soupçon. 

\speak  Monsieur le cardinal, vous m'avez entendu; je veux ces lettres. 

\speak  Il n'y aurait qu'un moyen. 

\speak  Lequel? 

\speak  Ce serait de charger de cette mission M. le garde des sceaux Séguier. La chose rentre complètement dans les devoirs de sa charge. 

\speak  Qu'on l'envoie chercher à l'instant même! 

\speak  Il doit être chez moi, Sire; je l'avais fait prier de passer, et lorsque je suis venu au Louvre, j'ai laissé l'ordre, s'il se présentait, de le faire attendre. 

\speak  Qu'on aille le chercher à l'instant même! 

\speak  Les ordres de Votre Majesté seront exécutés; mais\dots 

\speak  Mais quoi? 

\speak  Mais la reine se refusera peut-être à obéir. 

\speak  À mes ordres? 

\speak  Oui, si elle ignore que ces ordres viennent du roi. 

\speak  Eh bien, pour qu'elle n'en doute pas, je vais la prévenir moi-même. 

\speak  Votre Majesté n'oubliera pas que j'ai fait tout ce que j'ai pu pour prévenir une rupture. 

\speak  Oui, duc, je sais que vous êtes fort indulgent pour la reine, trop indulgent peut-être; et nous aurons, je vous en préviens, à parler plus tard de cela. 

\speak  Quand il plaira à Votre Majesté; mais je serai toujours heureux et fier, Sire, de me sacrifier à la bonne harmonie que je désire voir régner entre vous et la reine de France. 

\speak  Bien, cardinal, bien; mais en attendant envoyez chercher M. le garde des sceaux; moi, j'entre chez la reine. 

Et Louis XIII, ouvrant la porte de communication, s'engagea dans le corridor qui conduisait de chez lui chez Anne d'Autriche. 

La reine était au milieu de ses femmes, Mme de Guitaut, Mme de Sablé, Mme de Montbazon et Mme de Guéménée. Dans un coin était cette camériste espagnole doña Estefania, qui l'avait suivie de Madrid. Mme de Guéménée faisait la lecture, et tout le monde écoutait avec attention la lectrice, à l'exception de la reine, qui, au contraire, avait provoqué cette lecture afin de pouvoir, tout en feignant d'écouter, suivre le fil de ses propres pensées. 

Ces pensées, toutes dorées qu'elles étaient par un dernier reflet d'amour, n'en étaient pas moins tristes. Anne d'Autriche, privée de la confiance de son mari, poursuivie par la haine du cardinal, qui ne pouvait lui pardonner d'avoir repoussé un sentiment plus doux, ayant sous les yeux l'exemple de la reine mère, que cette haine avait tourmentée toute sa vie --- quoique Marie de Médicis, s'il faut en croire les mémoires du temps, eût commencé par accorder au cardinal le sentiment qu'Anne d'Autriche finit toujours par lui refuser ---, Anne d'Autriche avait vu tomber autour d'elle ses serviteurs les plus dévoués, ses confidents les plus intimes, ses favoris les plus chers. Comme ces malheureux doués d'un don funeste, elle portait malheur à tout ce qu'elle touchait, son amitié était un signe fatal qui appelait la persécution. Mme de Chevreuse et Mme de Vernel étaient exilées; enfin La Porte ne cachait pas à sa maîtresse qu'il s'attendait à être arrêté d'un instant à l'autre. 

C'est au moment où elle était plongée au plus profond et au plus sombre de ces réflexions, que la porte de la chambre s'ouvrit et que le roi entra. 

La lectrice se tut à l'instant même, toutes les dames se levèrent, et il se fit un profond silence. 

Quant au roi, il ne fit aucune démonstration de politesse; seulement, s'arrêtant devant la reine: 

«Madame, dit-il d'une voix altérée, vous allez recevoir la visite de M. le chancelier, qui vous communiquera certaines affaires dont je l'ai chargé.» 

La malheureuse reine, qu'on menaçait sans cesse de divorce, d'exil et de jugement même, pâlit sous son rouge et ne put s'empêcher de dire: 

«Mais pourquoi cette visite, Sire? Que me dira M. le chancelier que Votre Majesté ne puisse me dire elle-même?» 

Le roi tourna sur ses talons sans répondre, et presque au même instant le capitaine des gardes, M. de Guitaut, annonça la visite de M. le chancelier. 

Lorsque le chancelier parut, le roi était déjà sorti par une autre porte. 

Le chancelier entra demi-souriant, demi-rougissant. Comme nous le retrouverons probablement dans le cours de cette histoire, il n'y a pas de mal à ce que nos lecteurs fassent dès à présent connaissance avec lui. 

Ce chancelier était un plaisant homme. Ce fut Des Roches le Masle, chanoine à Notre-Dame, et qui avait été autrefois valet de chambre du cardinal, qui le proposa à Son Éminence comme un homme tout dévoué. Le cardinal s'y fia et s'en trouva bien. 

On racontait de lui certaines histoires, entre autres celle-ci: 

Après une jeunesse orageuse, il s'était retiré dans un couvent pour y expier au moins pendant quelque temps les folies de l'adolescence. 

Mais, en entrant dans ce saint lieu, le pauvre pénitent n'avait pu refermer si vite la porte, que les passions qu'il fuyait n'y entrassent avec lui. Il en était obsédé sans relâche, et le supérieur, auquel il avait confié cette disgrâce, voulant autant qu'il était en lui l'en garantir, lui avait recommandé pour conjurer le démon tentateur de recourir à la corde de la cloche et de sonner à toute volée. Au bruit dénonciateur, les moines seraient prévenus que la tentation assiégeait un frère, et toute la communauté se mettrait en prières. 

Le conseil parut bon au futur chancelier. Il conjura l'esprit malin à grand renfort de prières faites par les moines; mais le diable ne se laisse pas déposséder facilement d'une place où il a mis garnison; à mesure qu'on redoublait les exorcismes, il redoublait les tentations, de sorte que jour et nuit la cloche sonnait à toute volée, annonçant l'extrême désir de mortification qu'éprouvait le pénitent. 

Les moines n'avaient plus un instant de repos. Le jour, ils ne faisaient que monter et descendre les escaliers qui conduisaient à la chapelle; la nuit, outre complies et matines, ils étaient encore obligés de sauter vingt fois à bas de leurs lits et de se prosterner sur le carreau de leurs cellules. 

On ignore si ce fut le diable qui lâcha prise ou les moines qui se lassèrent; mais, au bout de trois mois, le pénitent reparut dans le monde avec la réputation du plus terrible possédé qui eût jamais existé. 

En sortant du couvent, il entra dans la magistrature, devint président à mortier à la place de son oncle, embrassa le parti du cardinal, ce qui ne prouvait pas peu de sagacité; devint chancelier, servit Son Éminence avec zèle dans sa haine contre la reine mère et sa vengeance contre Anne d'Autriche; stimula les juges dans l'affaire de Chalais, encouragea les essais de M. de Laffemas, grand gibecier de France; puis enfin, investi de toute la confiance du cardinal, confiance qu'il avait si bien gagnée, il en vint à recevoir la singulière commission pour l'exécution de laquelle il se présentait chez la reine. 

La reine était encore debout quand il entra, mais à peine l'eut-elle aperçu, qu'elle se rassit sur son fauteuil et fit signe à ses femmes de se rasseoir sur leurs coussins et leurs tabourets, et, d'un ton de suprême hauteur: 

«Que désirez-vous, monsieur, demanda Anne d'Autriche, et dans quel but vous présentez-vous ici? 

\speak  Pour y faire au nom du roi, madame, et sauf tout le respect que j'ai l'honneur de devoir à Votre Majesté, une perquisition exacte dans vos papiers. 

\speak  Comment, monsieur! une perquisition dans mes papiers\dots à moi! mais voilà une chose indigne! 

\speak  Veuillez me le pardonner, madame, mais, dans cette circonstance, je ne suis que l'instrument dont le roi se sert. Sa Majesté ne sort-elle pas d'ici, et ne vous a-t-elle pas invitée elle-même à vous préparer à cette visite? 

\speak  Fouillez donc, monsieur; je suis une criminelle, à ce qu'il paraît: Estefania, donnez les clefs de mes tables et de mes secrétaires.» 

Le chancelier fit pour la forme une visite dans les meubles, mais il savait bien que ce n'était pas dans un meuble que la reine avait dû serrer la lettre importante qu'elle avait écrite dans la journée. 

Quand le chancelier eut rouvert et refermé vingt fois les tiroirs du secrétaire, il fallut bien, quelque hésitation qu'il éprouvât, il fallut bien, dis-je, en venir à la conclusion de l'affaire, c'est-à-dire à fouiller la reine elle-même. Le chancelier s'avança donc vers Anne d'Autriche, et d'un ton très perplexe et d'un air fort embarrassé: 

«Et maintenant, dit-il, il me reste à faire la perquisition principale. 

\speak  Laquelle? demanda la reine, qui ne comprenait pas ou plutôt qui ne voulait pas comprendre. 

\speak  Sa Majesté est certaine qu'une lettre a été écrite par vous dans la journée; elle sait qu'elle n'a pas encore été envoyée à son adresse. Cette lettre ne se trouve ni dans votre table, ni dans votre secrétaire, et cependant cette lettre est quelque part. 

\speak  Oserez-vous porter la main sur votre reine? dit Anne d'Autriche en se dressant de toute sa hauteur et en fixant sur le chancelier ses yeux, dont l'expression était devenue presque menaçante. 

\speak  Je suis un fidèle sujet du roi, madame; et tout ce que Sa Majesté ordonnera, je le ferai. 

\speak  Eh bien, c'est vrai, dit Anne d'Autriche, et les espions de M. le cardinal l'ont bien servi. J'ai écrit aujourd'hui une lettre, cette lettre n'est point partie. La lettre est là.» 

Et la reine ramena sa belle main à son corsage. 

«Alors donnez-moi cette lettre, madame, dit le chancelier. 

\speak  Je ne la donnerai qu'au roi, monsieur, dit Anne. 

\speak  Si le roi eût voulu que cette lettre lui fût remise, madame, il vous l'eût demandée lui-même. Mais, je vous le répète, c'est moi qu'il a chargé de vous la réclamer, et si vous ne la rendiez pas\dots 

\speak  Eh bien? 

\speak  C'est encore moi qu'il a chargé de vous la prendre. 

\speak  Comment, que voulez-vous dire? 

\speak  Que mes ordres vont loin, madame, et que je suis autorisé à chercher le papier suspect sur la personne même de Votre Majesté. 

\speak  Quelle horreur! s'écria la reine. 

\speak  Veuillez donc, madame, agir plus facilement. 

\speak  Cette conduite est d'une violence infâme; savez-vous cela, monsieur? 

\speak  Le roi commande, madame, excusez-moi. 

\speak  Je ne le souffrirai pas; non, non, plutôt mourir!» s'écria la reine, chez laquelle se révoltait le sang impérieux de l'Espagnole et de l'Autrichienne. 

Le chancelier fit une profonde révérence, puis avec l'intention bien patente de ne pas reculer d'une semelle dans l'accomplissement de la commission dont il s'était chargé, et comme eût pu le faire un valet de bourreau dans la chambre de la question, il s'approcha d'Anne d'Autriche des yeux de laquelle on vit à l'instant même jaillir des pleurs de rage. 

La reine était, comme nous l'avons dit, d'une grande beauté. 

La commission pouvait donc passer pour délicate, et le roi en était arrivé, à force de jalousie contre Buckingham, à n'être plus jaloux de personne. 

Sans doute le chancelier Séguier chercha des yeux à ce moment le cordon de la fameuse cloche; mais, ne le trouvant pas, il en prit son parti et tendit la main vers l'endroit où la reine avait avoué que se trouvait le papier. 

Anne d'Autriche fit un pas en arrière, si pâle qu'on eût dit qu'elle allait mourir; et, s'appuyant de la main gauche, pour ne pas tomber, à une table qui se trouvait derrière elle, elle tira de la droite un papier de sa poitrine et le tendit au garde des sceaux. 

«Tenez, monsieur, la voilà, cette lettre, s'écria la reine d'une voix entrecoupée et frémissante, prenez-la, et me délivrez de votre odieuse présence.» 

Le chancelier, qui de son côté tremblait d'une émotion facile à concevoir, prit la lettre, salua jusqu'à terre et se retira. 

À peine la porte se fut-elle refermée sur lui, que la reine tomba à demi évanouie dans les bras de ses femmes. 

Le chancelier alla porter la lettre au roi sans en avoir lu un seul mot. Le roi la prit d'une main tremblante, chercha l'adresse, qui manquait, devint très pâle, l'ouvrit lentement, puis, voyant par les premiers mots qu'elle était adressée au roi d'Espagne, il lut très rapidement. 

C'était tout un plan d'attaque contre le cardinal. La reine invitait son frère et l'empereur d'Autriche à faire semblant, blessés qu'ils étaient par la politique de Richelieu, dont l'éternelle préoccupation fut l'abaissement de la maison d'Autriche, de déclarer la guerre à la France et d'imposer comme condition de la paix le renvoi du cardinal: mais d'amour, il n'y en avait pas un seul mot dans toute cette lettre. 

Le roi, tout joyeux, s'informa si le cardinal était encore au Louvre. On lui dit que Son Éminence attendait, dans le cabinet de travail, les ordres de Sa Majesté. 

Le roi se rendit aussitôt près de lui. 

«Tenez, duc, lui dit-il, vous aviez raison, et c'est moi qui avais tort; toute l'intrigue est politique, et il n'était aucunement question d'amour dans cette lettre, que voici. En échange, il y est fort question de vous.» 

Le cardinal prit la lettre et la lut avec la plus grande attention; puis, lorsqu'il fut arrivé au bout, il la relut une seconde fois. 

«Eh bien, Votre Majesté, dit-il, vous voyez jusqu'où vont mes ennemis: on vous menace de deux guerres, si vous ne me renvoyez pas. À votre place, en vérité, Sire, je céderais à de si puissantes instances, et ce serait de mon côté avec un véritable bonheur que je me retirerais des affaires. 

\speak  Que dites-vous là, duc? 

\speak  Je dis, Sire, que ma santé se perd dans ces luttes excessives et dans ces travaux éternels. Je dis que, selon toute probabilité, je ne pourrai pas soutenir les fatigues du siège de La Rochelle, et que mieux vaut que vous nommiez là ou M. de Condé, ou M. de Bassompierre, ou enfin quelque vaillant homme dont c'est l'état de mener la guerre, et non pas moi qui suis homme d'Église et qu'on détourne sans cesse de ma vocation pour m'appliquer à des choses auxquelles je n'ai aucune aptitude. Vous en serez plus heureux à l'intérieur, Sire, et je ne doute pas que vous n'en soyez plus grand à l'étranger. 

\speak  Monsieur le duc, dit le roi, je comprends, soyez tranquille; tous ceux qui sont nommés dans cette lettre seront punis comme ils le méritent, et la reine elle-même. 

\speak  Que dites-vous là, Sire? Dieu me garde que, pour moi, la reine éprouve la moindre contrariété! elle m'a toujours cru son ennemi, Sire, quoique Votre Majesté puisse attester que j'ai toujours pris chaudement son parti, même contre vous. Oh! si elle trahissait Votre Majesté à l'endroit de son honneur, ce serait autre chose, et je serais le premier à dire: «Pas de grâce, Sire, pas de grâce pour la coupable!» Heureusement il n'en est rien, et Votre Majesté vient d'en acquérir une nouvelle preuve. 

\speak  C'est vrai, monsieur le cardinal, dit le roi, et vous aviez raison, comme toujours; mais la reine n'en mérite pas moins toute ma colère. 

\speak  C'est vous, Sire, qui avez encouru la sienne; et véritablement, quand elle bouderait sérieusement Votre Majesté, je le comprendrais; Votre Majesté l'a traitée avec une sévérité!\dots 

\speak  C'est ainsi que je traiterai toujours mes ennemis et les vôtres, duc, si haut placés qu'ils soient et quelque péril que je coure à agir sévèrement avec eux. 

\speak  La reine est mon ennemie, mais n'est pas la vôtre, Sire; au contraire, elle est épouse dévouée, soumise et irréprochable; laissez-moi donc, Sire, intercéder pour elle près de Votre Majesté. 

\speak  Qu'elle s'humilie alors, et qu'elle revienne à moi la première! 

\speak  Au contraire, Sire, donnez l'exemple; vous avez eu le premier tort, puisque c'est vous qui avez soupçonné la reine. 

\speak  Moi, revenir le premier? dit le roi; jamais! 

\speak  Sire, je vous en supplie. 

\speak  D'ailleurs, comment reviendrais-je le premier? 

\speak  En faisant une chose que vous sauriez lui être agréable. 

\speak  Laquelle? 

\speak  Donnez un bal; vous savez combien la reine aime la danse; je vous réponds que sa rancune ne tiendra point à une pareille attention. 

\speak  Monsieur le cardinal, vous savez que je n'aime pas tous les plaisirs mondains. 

\speak  La reine ne vous en sera que plus reconnaissante, puisqu'elle sait votre antipathie pour ce plaisir; d'ailleurs ce sera une occasion pour elle de mettre ces beaux ferrets de diamants que vous lui avez donnés l'autre jour à sa fête, et dont elle n'a pas encore eu le temps de se parer. 

\speak  Nous verrons, monsieur le cardinal, nous verrons, dit le roi, qui, dans sa joie de trouver la reine coupable d'un crime dont il se souciait peu, et innocente d'une faute qu'il redoutait fort, était tout prêt à se raccommoder avec elle; nous verrons, mais, sur mon honneur, vous êtes trop indulgent. 

\speak  Sire, dit le cardinal, laissez la sévérité aux ministres, l'indulgence est la vertu royale; usez-en, et vous verrez que vous vous en trouverez bien.» 

Sur quoi le cardinal, entendant la pendule sonner onze heures, s'inclina profondément, demandant congé au roi pour se retirer, et le suppliant de se raccommoder avec la reine. 

Anne d'Autriche, qui, à la suite de la saisie de sa lettre, s'attendait à quelque reproche, fut fort étonnée de voir le lendemain le roi faire près d'elle des tentatives de rapprochement. Son premier mouvement fut répulsif, son orgueil de femme et sa dignité de reine avaient été tous deux si cruellement offensés, qu'elle ne pouvait revenir ainsi du premier coup; mais, vaincue par le conseil de ses femmes, elle eut enfin l'air de commencer à oublier. Le roi profita de ce premier moment de retour pour lui dire qu'incessamment il comptait donner une fête. 

C'était une chose si rare qu'une fête pour la pauvre Anne d'Autriche, qu'à cette annonce, ainsi que l'avait pensé le cardinal, la dernière trace de ses ressentiments disparut sinon dans son cœur, du moins sur son visage. Elle demanda quel jour cette fête devait avoir lieu, mais le roi répondit qu'il fallait qu'il s'entendît sur ce point avec le cardinal. 

En effet, chaque jour le roi demandait au cardinal à quelle époque cette fête aurait lieu, et chaque jour le cardinal, sous un prétexte quelconque, différait de la fixer. 

Dix jours s'écoulèrent ainsi. 

Le huitième jour après la scène que nous avons racontée, le cardinal reçut une lettre, au timbre de Londres, qui contenait seulement ces quelques lignes: 

«Je les ai; mais je ne puis quitter Londres, attendu que je manque d'argent; envoyez-moi cinq cents pistoles, et quatre ou cinq jours après les avoir reçues, je serai à Paris.» 

Le jour même où le cardinal avait reçu cette lettre, le roi lui adressa sa question habituelle. 

Richelieu compta sur ses doigts et se dit tout bas: 

«Elle arrivera, dit-elle, quatre ou cinq jours après avoir reçu l'argent; il faut quatre ou cinq jours à l'argent pour aller, quatre ou cinq jours à elle pour revenir, cela fait dix jours; maintenant faisons la part des vents contraires, des mauvais hasards, des faiblesses de femme, et mettons cela à douze jours. 

\speak  Eh bien, monsieur le duc, dit le roi, vous avez calculé? 

\speak  Oui, Sire: nous sommes aujourd'hui le 20 septembre; les échevins de la ville donnent une fête le 3 octobre. Cela s'arrangera à merveille, car vous n'aurez pas l'air de faire un retour vers la reine.» 

Puis le cardinal ajouta: 

«À propos, Sire, n'oubliez pas de dire à Sa Majesté, la veille de cette fête, que vous désirez voir comment lui vont ses ferrets de diamants.» 
%!TeX root=../musketeersfr.tex 

\chapter{Le Ménage Bonacieux} 
	
	\lettrine{C}{'était} la seconde fois que le cardinal revenait sur ce point des ferrets de diamants avec le roi. Louis XIII fut donc frappé de cette insistance, et pensa que cette recommandation cachait un mystère. 

Plus d'une fois le roi avait été humilié que le cardinal, dont la police, sans avoir atteint encore la perfection de la police moderne, était excellente, fût mieux instruit que lui-même de ce qui se passait dans son propre ménage. Il espéra donc, dans une conversation avec Anne d'Autriche, tirer quelque lumière de cette conversation et revenir ensuite près de Son Éminence avec quelque secret que le cardinal sût ou ne sût pas, ce qui, dans l'un ou l'autre cas, le rehaussait infiniment aux yeux de son ministre. 

Il alla donc trouver la reine, et, selon son habitude, l'aborda avec de nouvelles menaces contre ceux qui l'entouraient. Anne d'Autriche baissa la tête, laissa s'écouler le torrent sans répondre et espérant qu'il finirait par s'arrêter; mais ce n'était pas cela que voulait Louis XIII; Louis XIII voulait une discussion de laquelle jaillît une lumière quelconque, convaincu qu'il était que le cardinal avait quelque arrière-pensée et lui machinait une surprise terrible comme en savait faire Son Éminence. Il arriva à ce but par sa persistance à accuser. 

«Mais, s'écria Anne d'Autriche, lassée de ces vagues attaques; mais, Sire, vous ne me dites pas tout ce que vous avez dans le cœur. Qu'ai-je donc fait? Voyons, quel crime ai-je donc commis? Il est impossible que Votre Majesté fasse tout ce bruit pour une lettre écrite à mon frère.» 

Le roi, attaqué à son tour d'une manière si directe, ne sut que répondre; il pensa que c'était là le moment de placer la recommandation qu'il ne devait faire que la veille de la fête. 

«Madame, dit-il avec majesté, il y aura incessamment bal à l'hôtel de ville; j'entends que, pour faire honneur à nos braves échevins, vous y paraissiez en habit de cérémonie, et surtout parée des ferrets de diamants que je vous ai donnés pour votre fête. Voici ma réponse.» 

La réponse était terrible. Anne d'Autriche crut que Louis XIII savait tout, et que le cardinal avait obtenu de lui cette longue dissimulation de sept ou huit jours, qui était au reste dans son caractère. Elle devint excessivement pâle, appuya sur une console sa main d'une admirable beauté, et qui semblait alors une main de cire, et regardant le roi avec des yeux épouvantés, elle ne répondit pas une seule syllabe. 

«Vous entendez, madame, dit le roi, qui jouissait de cet embarras dans toute son étendue, mais sans en deviner la cause, vous entendez? 

\speak  Oui, Sire, j'entends, balbutia la reine. 

\speak  Vous paraîtrez à ce bal? 

\speak  Oui. 

\speak  Avec vos ferrets? 

\speak  Oui.» 

La pâleur de la reine augmenta encore, s'il était possible; le roi s'en aperçut, et en jouit avec cette froide cruauté qui était un des mauvais côtés de son caractère. 

«Alors, c'est convenu, dit le roi, et voilà tout ce que j'avais à vous dire. 

\speak  Mais quel jour ce bal aura-t-il lieu?» demanda Anne d'Autriche. 

Louis XIII sentit instinctivement qu'il ne devait pas répondre à cette question, la reine l'ayant faite d'une voix presque mourante. 

«Mais très incessamment, madame, dit-il; mais je ne me rappelle plus précisément la date du jour, je la demanderai au cardinal. 

\speak  C'est donc le cardinal qui vous a annoncé cette fête? s'écria la reine. 

\speak  Oui, madame, répondit le roi étonné; mais pourquoi cela? 

\speak  C'est lui, qui vous a dit de m'inviter à y paraître avec ces ferrets? 

\speak  C'est-à-dire, madame\dots 

\speak  C'est lui, Sire, c'est lui! 

\speak  Eh bien qu'importe que ce soit lui ou moi? y a-t-il un crime à cette invitation? 

\speak  Non, Sire. 

\speak  Alors vous paraîtrez? 

\speak  Oui, Sire. 

\speak  C'est bien, dit le roi en se retirant, c'est bien, j'y compte.» 

La reine fit une révérence, moins par étiquette que parce que ses genoux se dérobaient sous elle. 

Le roi partit enchanté. 

«Je suis perdue, murmura la reine, perdue, car le cardinal sait tout, et c'est lui qui pousse le roi, qui ne sait rien encore, mais qui saura tout bientôt. Je suis perdue! Mon Dieu! mon Dieu! mon Dieu!» 

Elle s'agenouilla sur un coussin et pria, la tête enfoncée entre ses bras palpitants. 

En effet, la position était terrible. Buckingham était retourné à Londres, Mme de Chevreuse était à Tours. Plus surveillée que jamais, la reine sentait sourdement qu'une de ses femmes la trahissait, sans savoir dire laquelle. La Porte ne pouvait pas quitter le Louvre. Elle n'avait pas une âme au monde à qui se fier. 

Aussi, en présence du malheur qui la menaçait et de l'abandon qui était le sien, éclata-t-elle en sanglots. 

«Ne puis-je donc être bonne à rien à Votre Majesté?» dit tout à coup une voix pleine de douceur et de pitié. 

La reine se retourna vivement, car il n'y avait pas à se tromper à l'expression de cette voix: c'était une amie qui parlait ainsi. 

En effet, à l'une des portes qui donnaient dans l'appartement de la reine apparut la jolie Mme Bonacieux; elle était occupée à ranger les robes et le linge dans un cabinet, lorsque le roi était entré; elle n'avait pas pu sortir, et avait tout entendu. 

La reine poussa un cri perçant en se voyant surprise, car dans son trouble elle ne reconnut pas d'abord la jeune femme qui lui avait été donnée par La Porte. 

«Oh! ne craignez rien, madame, dit la jeune femme en joignant les mains et en pleurant elle-même des angoisses de la reine; je suis à Votre Majesté corps et âme, et si loin que je sois d'elle, si inférieure que soit ma position, je crois que j'ai trouvé un moyen de tirer Votre Majesté de peine. 

\speak  Vous! ô Ciel! vous! s'écria la reine; mais voyons regardez-moi en face. Je suis trahie de tous côtés, puis-je me fier à vous? 

\speak  Oh! madame! s'écria la jeune femme en tombant à genoux: sur mon âme, je suis prête à mourir pour Votre Majesté!» 

Ce cri était sorti du plus profond du cœur, et, comme le premier, il n'y avait pas à se tromper. 

«Oui, continua Mme Bonacieux, oui, il y a des traîtres ici; mais, par le saint nom de la Vierge, je vous jure que personne n'est plus dévoué que moi à Votre Majesté. Ces ferrets que le roi redemande, vous les avez donnés au duc de Buckingham, n'est-ce pas? Ces ferrets étaient enfermés dans une petite boîte en bois de rose qu'il tenait sous son bras? Est-ce que je me trompe? Est-ce que ce n'est pas cela? 

\speak  Oh! mon Dieu! mon Dieu! murmura la reine dont les dents claquaient d'effroi. 

\speak  Eh bien, ces ferrets, continua Mme Bonacieux, il faut les ravoir. 

\speak  Oui, sans doute, il le faut, s'écria la reine; mais comment faire, comment y arriver? 

\speak  Il faut envoyer quelqu'un au duc. 

\speak  Mais qui?\dots qui?\dots à qui me fier? 

\speak  Ayez confiance en moi, madame; faites-moi cet honneur, ma reine, et je trouverai le messager, moi! 

\speak  Mais il faudra écrire! 

\speak  Oh! oui. C'est indispensable. Deux mots de la main de Votre Majesté et votre cachet particulier. 

\speak  Mais ces deux mots, c'est ma condamnation. C'est le divorce, l'exil! 

\speak  Oui, s'ils tombent entre des mains infâmes! Mais je réponds que ces deux mots seront remis à leur adresse. 

\speak  Oh! mon Dieu! il faut donc que je remette ma vie, mon honneur, ma réputation entre vos mains! 

\speak  Oui! oui, madame, il le faut, et je sauverai tout cela, moi! 

\speak  Mais comment? dites-le-moi au moins. 

\speak  Mon mari a été remis en liberté il y a deux ou trois jours; je n'ai pas encore eu le temps de le revoir. C'est un brave et honnête homme qui n'a ni haine, ni amour pour personne. Il fera ce que je voudrai: il partira sur un ordre de moi, sans savoir ce qu'il porte, et il remettra la lettre de Votre Majesté, sans même savoir qu'elle est de Votre Majesté, à l'adresse qu'elle indiquera.» 

La reine prit les deux mains de la jeune femme avec un élan passionné, la regarda comme pour lire au fond de son cœur, et ne voyant que sincérité dans ses beaux yeux, elle l'embrassa tendrement. 

«Fais cela, s'écria-t-elle, et tu m'auras sauvé la vie, tu m'auras sauvé l'honneur! 

\speak  Oh! n'exagérez pas le service que j'ai le bonheur de vous rendre; je n'ai rien à sauver à Votre Majesté, qui est seulement victime de perfides complots. 

\speak  C'est vrai, c'est vrai, mon enfant, dit la reine, et tu as raison. 

\speak  Donnez-moi donc cette lettre, madame, le temps presse.» 

La reine courut à une petite table sur laquelle se trouvaient encre, papier et plumes: elle écrivit deux lignes, cacheta la lettre de son cachet et la remit à Mme Bonacieux. 

«Et maintenant, dit la reine, nous oublions une chose nécessaire. 

\speak  Laquelle? 

\speak  L'argent.» 

Mme Bonacieux rougit. 

«Oui, c'est vrai, dit-elle, et j'avouerai à Votre Majesté que mon mari\dots 

\speak  Ton mari n'en a pas, c'est cela que tu veux dire. 

\speak  Si fait, il en a, mais il est fort avare, c'est là son défaut. Cependant, que Votre Majesté ne s'inquiète pas, nous trouverons moyen\dots 

\speak  C'est que je n'en ai pas non plus, dit la reine (ceux qui liront les Mémoires de Mme de Motteville ne s'étonneront pas de cette réponse); mais, attends.» 

Anne d'Autriche courut à son écrin. 

«Tiens, dit-elle, voici une bague d'un grand prix à ce qu'on assure; elle vient de mon frère le roi d'Espagne, elle est à moi et j'en puis disposer. Prends cette bague et fais-en de l'argent, et que ton mari parte. 

\speak  Dans une heure vous serez obéie. 

\speak  Tu vois l'adresse, ajouta la reine, parlant si bas qu'à peine pouvait-on entendre ce qu'elle disait: à Milord duc de Buckingham, à Londres. 

\speak  La lettre sera remise à lui-même. 

\speak  Généreuse enfant!» s'écria Anne d'Autriche. 

Mme Bonacieux baisa les mains de la reine, cacha le papier dans son corsage et disparut avec la légèreté d'un oiseau. 

Dix minutes après, elle était chez elle; comme elle l'avait dit à la reine, elle n'avait pas revu son mari depuis sa mise en liberté; elle ignorait donc le changement qui s'était fait en lui à l'endroit du cardinal, changement qu'avaient opéré la flatterie et l'argent de Son Éminence et qu'avaient corroboré, depuis, deux ou trois visites du comte de Rochefort, devenu le meilleur ami de Bonacieux, auquel il avait fait croire sans beaucoup de peine qu'aucun sentiment coupable n'avait amené l'enlèvement de sa femme, mais que c'était seulement une précaution politique. 

Elle trouva M. Bonacieux seul: le pauvre homme remettait à grand-peine de l'ordre dans la maison, dont il avait trouvé les meubles à peu près brisés et les armoires à peu près vides, la justice n'étant pas une des trois choses que le roi Salomon indique comme ne laissant point de traces de leur passage. Quant à la servante, elle s'était enfuie lors de l'arrestation de son maître. La terreur avait gagné la pauvre fille au point qu'elle n'avait cessé de marcher de Paris jusqu'en Bourgogne, son pays natal. 

Le digne mercier avait, aussitôt sa rentrée dans sa maison, fait part à sa femme de son heureux retour, et sa femme lui avait répondu pour le féliciter et pour lui dire que le premier moment qu'elle pourrait dérober à ses devoirs serait consacré tout entier à lui rendre visite. 

Ce premier moment s'était fait attendre cinq jours, ce qui, dans toute autre circonstance, eût paru un peu bien long à maître Bonacieux; mais il avait, dans la visite qu'il avait faite au cardinal et dans les visites que lui faisait Rochefort, ample sujet à réflexion, et, comme on sait, rien ne fait passer le temps comme de réfléchir. 

D'autant plus que les réflexions de Bonacieux étaient toutes couleur de rose. Rochefort l'appelait son ami, son cher Bonacieux, et ne cessait de lui dire que le cardinal faisait le plus grand cas de lui. Le mercier se voyait déjà sur le chemin des honneurs et de la fortune. 

De son côté, Mme Bonacieux avait réfléchi, mais, il faut le dire, à tout autre chose que l'ambition; malgré elle, ses pensées avaient eu pour mobile constant ce beau jeune homme si brave et qui paraissait si amoureux. Mariée à dix-huit ans à M. Bonacieux, ayant toujours vécu au milieu des amis de son mari, peu susceptibles d'inspirer un sentiment quelconque à une jeune femme dont le cœur était plus élevé que sa position, Mme Bonacieux était restée insensible aux séductions vulgaires; mais, à cette époque surtout, le titre de gentilhomme avait une grande influence sur la bourgeoisie, et d'Artagnan était gentilhomme; de plus, il portait l'uniforme des gardes, qui, après l'uniforme des mousquetaires, était le plus apprécié des dames. Il était, nous le répétons, beau, jeune, aventureux; il parlait d'amour en homme qui aime et qui a soif d'être aimé; il y en avait là plus qu'il n'en fallait pour tourner une tête de vingt-trois ans, et Mme Bonacieux en était arrivée juste à cet âge heureux de la vie. 

Les deux époux, quoiqu'ils ne se fussent pas vus depuis plus de huit jours, et que pendant cette semaine de graves événements eussent passé entre eux, s'abordèrent donc avec une certaine préoccupation; néanmoins, M. Bonacieux manifesta une joie réelle et s'avança vers sa femme à bras ouverts. 

Mme Bonacieux lui présenta le front. 

«Causons un peu, dit-elle. 

\speak  Comment? dit Bonacieux étonné. 

\speak  Oui, sans doute, j'ai une chose de la plus haute importance à vous dire. 

\speak  Au fait, et moi aussi, j'ai quelques questions assez sérieuses à vous adresser. Expliquez-moi un peu votre enlèvement, je vous prie. 

\speak  Il ne s'agit point de cela pour le moment, dit Mme Bonacieux. 

\speak  Et de quoi s'agit-il donc? de ma captivité? 

\speak  Je l'ai apprise le jour même; mais comme vous n'étiez coupable d'aucun crime, comme vous n'étiez complice d'aucune intrigue, comme vous ne saviez rien enfin qui pût vous compromettre, ni vous, ni personne, je n'ai attaché à cet événement que l'importance qu'il méritait. 

\speak  Vous en parlez bien à votre aise, madame! reprit Bonacieux blessé du peu d'intérêt que lui témoignait sa femme; savez-vous que j'ai été plongé un jour et une nuit dans un cachot de la Bastille? 

\speak  Un jour et une nuit sont bientôt passés; laissons donc votre captivité, et revenons à ce qui m'amène près de vous. 

\speak  Comment? ce qui vous amène près de moi! N'est-ce donc pas le désir de revoir un mari dont vous êtes séparée depuis huit jours? demanda le mercier piqué au vif. 

\speak  C'est cela d'abord, et autre chose ensuite. 

\speak  Parlez! 

\speak  Une chose du plus haut intérêt et de laquelle dépend notre fortune à venir peut-être. 

\speak  Notre fortune a fort changé de face depuis que je vous ai vue, madame Bonacieux, et je ne serais pas étonné que d'ici à quelques mois elle ne fît envie à beaucoup de gens. 

\speak  Oui, surtout si vous voulez suivre les instructions que je vais vous donner. 

\speak  À moi? 

\speak  Oui, à vous. Il y a une bonne et sainte action à faire, monsieur, et beaucoup d'argent à gagner en même temps.» 

Mme Bonacieux savait qu'en parlant d'argent à son mari, elle le prenait par son faible. 

Mais un homme, fût-ce un mercier, lorsqu'il a causé dix minutes avec le cardinal de Richelieu, n'est plus le même homme. 

«Beaucoup d'argent à gagner! dit Bonacieux en allongeant les lèvres. 

\speak  Oui, beaucoup. 

\speak  Combien, à peu près? 

\speak  Mille pistoles peut-être. 

\speak  Ce que vous avez à me demander est donc bien grave? 

\speak  Oui. 

\speak  Que faut-il faire? 

\speak  Vous partirez sur-le-champ, je vous remettrai un papier dont vous ne vous dessaisirez sous aucun prétexte, et que vous remettrez en main propre. 

\speak  Et pour où partirai-je? 

\speak  Pour Londres. 

\speak  Moi, pour Londres! Allons donc, vous raillez, je n'ai pas affaire à Londres. 

\speak  Mais d'autres ont besoin que vous y alliez. 

\speak  Quels sont ces autres? Je vous avertis, je ne fais plus rien en aveugle, et je veux savoir non seulement à quoi je m'expose, mais encore pour qui je m'expose. 

\speak  Une personne illustre vous envoie, une personne illustre vous attend: la récompense dépassera vos désirs, voilà tout ce que je puis vous promettre. 

\speak  Des intrigues encore, toujours des intrigues! merci, je m'en défie maintenant, et M. le cardinal m'a éclairé là-dessus. 

\speak  Le cardinal! s'écria Mme Bonacieux, vous avez vu le cardinal? 

\speak  Il m'a fait appeler, répondit fièrement le mercier. 

\speak  Et vous vous êtes rendu à son invitation, imprudent que vous êtes. 

\speak  Je dois dire que je n'avais pas le choix de m'y rendre ou de ne pas m'y rendre, car j'étais entre deux gardes. Il est vrai encore de dire que, comme alors je ne connaissais pas Son Éminence, si j'avais pu me dispenser de cette visite, j'en eusse été fort enchanté. 

\speak  Il vous a donc maltraité? il vous a donc fait des menaces? 

\speak  Il m'a tendu la main et m'a appelé son ami, --- son ami! entendez-vous, madame? --- je suis l'ami du grand cardinal! 

\speak  Du grand cardinal! 

\speak  Lui contesteriez-vous ce titre, par hasard, madame? 

\speak  Je ne lui conteste rien, mais je vous dis que la faveur d'un ministre est éphémère, et qu'il faut être fou pour s'attacher à un ministre; il est des pouvoirs au-dessus du sien, qui ne reposent pas sur le caprice d'un homme ou l'issue d'un événement; c'est à ces pouvoirs qu'il faut se rallier. 

\speak  J'en suis fâché, madame, mais je ne connais pas d'autre pouvoir que celui du grand homme que j'ai l'honneur de servir. 

\speak  Vous servez le cardinal? 

\speak  Oui, madame, et comme son serviteur je ne permettrai pas que vous vous livriez à des complots contre la sûreté de l'État, et que vous serviez, vous, les intrigues d'une femme qui n'est pas française et qui a le cœur espagnol. Heureusement, le grand cardinal est là, son regard vigilant surveille et pénètre jusqu'au fond du cœur.» 

Bonacieux répétait mot pour mot une phrase qu'il avait entendu dire au comte de Rochefort; mais la pauvre femme, qui avait compté sur son mari et qui, dans cet espoir, avait répondu de lui à la reine, n'en frémit pas moins, et du danger dans lequel elle avait failli se jeter, et de l'impuissance dans laquelle elle se trouvait. Cependant connaissant la faiblesse et surtout la cupidité de son mari elle ne désespérait pas de l'amener à ses fins. 

«Ah! vous êtes cardinaliste, monsieur, s'écria-t-elle ah! vous servez le parti de ceux qui maltraitent votre femme et qui insultent votre reine! 

\speak  Les intérêts particuliers ne sont rien devant les intérêts de tous. Je suis pour ceux qui sauvent l'État», dit avec emphase Bonacieux. 

C'était une autre phrase du comte de Rochefort, qu'il avait retenue et qu'il trouvait l'occasion de placer. 

«Et savez-vous ce que c'est que l'État dont vous parlez? dit Mme Bonacieux en haussant les épaules. Contentez-vous d'être un bourgeois sans finesse aucune, et tournez-vous du côté qui vous offre le plus d'avantages. 

\speak  Eh! eh! dit Bonacieux en frappant sur un sac à la panse arrondie et qui rendit un son argentin; que dites-vous de ceci, madame la prêcheuse? 

\speak  D'où vient cet argent? 

\speak  Vous ne devinez pas? 

\speak  Du cardinal? 

\speak  De lui et de mon ami le comte de Rochefort. 

\speak  Le comte de Rochefort! mais c'est lui qui m'a enlevée! 

\speak  Cela se peut, madame. 

\speak  Et vous recevez de l'argent de cet homme? 

\speak  Ne m'avez-vous pas dit que cet enlèvement était tout politique? 

\speak  Oui; mais cet enlèvement avait pour but de me faire trahir ma maîtresse, de m'arracher par des tortures des aveux qui pussent compromettre l'honneur et peut-être la vie de mon auguste maîtresse. 

\speak  Madame, reprit Bonacieux, votre auguste maîtresse est une perfide Espagnole, et ce que le cardinal fait est bien fait. 

\speak  Monsieur, dit la jeune femme, je vous savais lâche, avare et imbécile, mais je ne vous savais pas infâme! 

\speak  Madame, dit Bonacieux, qui n'avait jamais vu sa femme en colère, et qui reculait devant le courroux conjugal; madame, que dites-vous donc? 

\speak  Je dis que vous êtes un misérable! continua Mme Bonacieux, qui vit qu'elle reprenait quelque influence sur son mari. Ah! vous faites de la politique, vous! et de la politique cardinaliste encore! Ah! vous vous vendez, corps et âme, au démon pour de l'argent. 

\speak  Non, mais au cardinal. 

\speak  C'est la même chose! s'écria la jeune femme. Qui dit Richelieu, dit Satan. 

\speak  Taisez-vous, madame, taisez-vous, on pourrait vous entendre! 

\speak  Oui, vous avez raison, et je serais honteuse pour vous de votre lâcheté. 

\speak  Mais qu'exigez-vous donc de moi? voyons! 

\speak  Je vous l'ai dit: que vous partiez à l'instant même, monsieur, que vous accomplissiez loyalement la commission dont je daigne vous charger, et à cette condition j'oublie tout, je pardonne, et il y a plus --- elle lui tendit la main --- je vous rends mon amitié.» 

Bonacieux était poltron et avare; mais il aimait sa femme: il fut attendri. Un homme de cinquante ans ne tient pas longtemps rancune à une femme de vingt-trois. Mme Bonacieux vit qu'il hésitait: 

«Allons, êtes-vous décidé? dit-elle. 

\speak  Mais, ma chère amie, réfléchissez donc un peu à ce que vous exigez de moi; Londres est loin de Paris, fort loin, et peut-être la commission dont vous me chargez n'est-elle pas sans dangers. 

\speak  Qu'importe, si vous les évitez! 

\speak  Tenez, madame Bonacieux, dit le mercier, tenez, décidément, je refuse: les intrigues me font peur. J'ai vu la Bastille, moi. Brrrrou! c'est affreux, la Bastille! Rien que d'y penser, j'en ai la chair de poule. On m'a menacé de la torture. Savez-vous ce que c'est que la torture? Des coins de bois qu'on vous enfonce entre les jambes jusqu'à ce que les os éclatent! Non, décidément, je n'irai pas. Et morbleu! que n'y allez-vous vous-même? car, en vérité, je crois que je me suis trompé sur votre compte jusqu'à présent: je crois que vous êtes un homme, et des plus enragés encore! 

\speak  Et vous, vous êtes une femme, une misérable femme, stupide et abrutie. Ah! vous avez peur! Eh bien, si vous ne partez pas à l'instant même, je vous fais arrêter par l'ordre de la reine, et je vous fais mettre à cette Bastille que vous craignez tant.» 

Bonacieux tomba dans une réflexion profonde, il pesa mûrement les deux colères dans son cerveau, celle du cardinal et celle de la reine: celle du cardinal l'emporta énormément. 

«Faites-moi arrêter de la part de la reine, dit-il, et moi je me réclamerai de Son Éminence.» 

Pour le coup, Mme Bonacieux vit qu'elle avait été trop loin, et elle fut épouvantée de s'être si fort avancée. Elle contempla un instant avec effroi cette figure stupide, d'une résolution invincible, comme celle des sots qui ont peur. 

«Eh bien, soit! dit-elle. Peut-être, au bout du compte, avez-vous raison: un homme en sait plus long que les femmes en politique, et vous surtout, monsieur Bonacieux, qui avez causé avec le cardinal. Et cependant, il est bien dur, ajouta-t-elle, que mon mari, un homme sur l'affection duquel je croyais pouvoir compter, me traite aussi disgracieusement et ne satisfasse point à ma fantaisie. 

\speak  C'est que vos fantaisies peuvent mener trop loin, reprit Bonacieux triomphant, et je m'en défie. 

\speak  J'y renoncerai donc, dit la jeune femme en soupirant; c'est bien, n'en parlons plus. 

\speak  Si, au moins, vous me disiez quelle chose je vais faire à Londres, reprit Bonacieux, qui se rappelait un peu tard que Rochefort lui avait recommandé d'essayer de surprendre les secrets de sa femme. 

\speak  Il est inutile que vous le sachiez, dit la jeune femme, qu'une défiance instinctive repoussait maintenant en arrière: il s'agissait d'une bagatelle comme en désirent les femmes, d'une emplette sur laquelle il y avait beaucoup à gagner.» 

Mais plus la jeune femme se défendait, plus au contraire Bonacieux pensa que le secret qu'elle refusait de lui confier était important. Il résolut donc de courir à l'instant même chez le comte de Rochefort, et de lui dire que la reine cherchait un messager pour l'envoyer à Londres. 

«Pardon, si je vous quitte, ma chère madame Bonacieux, dit-il; mais, ne sachant pas que vous me viendriez voir, j'avais pris rendez-vous avec un de mes amis, je reviens à l'instant même, et si vous voulez m'attendre seulement une demi-minute, aussitôt que j'en aurai fini avec cet ami, je reviens vous prendre, et, comme il commence à se faire tard, je vous reconduis au Louvre. 

\speak  Merci, monsieur, répondit Mme Bonacieux: vous n'êtes point assez brave pour m'être d'une utilité quelconque, et je m'en retournerai bien au Louvre toute seule. 

\speak  Comme il vous plaira, madame Bonacieux, reprit l'ex-mercier. Vous reverrai-je bientôt? 

\speak  Sans doute; la semaine prochaine, je l'espère, mon service me laissera quelque liberté, et j'en profiterai pour revenir mettre de l'ordre dans nos affaires, qui doivent être quelque peu dérangées. 

\speak  C'est bien; je vous attendrai. Vous ne m'en voulez pas? 

\speak  Moi! pas le moins du monde. 

\speak  À bientôt, alors? 

\speak  À bientôt.» 

Bonacieux baisa la main de sa femme, et s'éloigna rapidement. 

«Allons, dit Mme Bonacieux, lorsque son mari eut refermé la porte de la rue, et qu'elle se trouva seule, il ne manquait plus à cet imbécile que d'être cardinaliste! Et moi qui avais répondu à la reine, moi qui avais promis à ma pauvre maîtresse\dots Ah! mon Dieu, mon Dieu! elle va me prendre pour quelqu'une de ces misérables dont fourmille le palais, et qu'on a placées près d'elle pour l'espionner! Ah! monsieur Bonacieux! je ne vous ai jamais beaucoup aimé; maintenant, c'est bien pis: je vous hais! et, sur ma parole, vous me le paierez!» 

Au moment où elle disait ces mots, un coup frappé au plafond lui fit lever la tête, et une voix, qui parvint à elle à travers le plancher, lui cria: 

«Chère madame Bonacieux, ouvrez-moi la petite porte de l'allée, et je vais descendre près de vous.» 
%!TeX root=../musketeersfr.tex 

\chapter{L'Amant Et Le Mari} 

\lettrine[ante=«]{A}{h!} madame, dit d'Artagnan en entrant par la porte que lui ouvrait la jeune femme, permettez-moi de vous le dire, vous avez là un triste mari. 

\zz
\noindent --- Vous avez donc entendu notre conversation? demanda vivement Mme Bonacieux en regardant d'Artagnan avec inquiétude. 

\zz
\speak  Tout entière. 

\speak  Mais comment cela? mon Dieu! 

\speak  Par un procédé à moi connu, et par lequel j'ai entendu aussi la conversation plus animée que vous avez eue avec les sbires du cardinal. 

\speak  Et qu'avez-vous compris dans ce que nous disions? 

\speak  Mille choses: d'abord, que votre mari est un niais et un sot, heureusement; puis, que vous étiez embarrassée, ce dont j'ai été fort aise, et que cela me donne une occasion de me mettre à votre service, et Dieu sait si je suis prêt à me jeter dans le feu pour vous; enfin que la reine a besoin qu'un homme brave, intelligent et dévoué fasse pour elle un voyage à Londres. J'ai au moins deux des trois qualités qu'il vous faut, et me voilà.» 

Mme Bonacieux ne répondit pas, mais son cœur battait de joie, et une secrète espérance brilla à ses yeux. 

«Et quelle garantie me donnerez-vous, demanda-t-elle, si je consens à vous confier cette mission? 

\speak  Mon amour pour vous. Voyons, dites, ordonnez: que faut-il faire? 

\speak  Mon Dieu! mon Dieu! murmura la jeune femme, dois-je vous confier un pareil secret, monsieur? Vous êtes presque un enfant! 

\speak  Allons, je vois qu'il vous faut quelqu'un qui vous réponde de moi. 

\speak  J'avoue que cela me rassurerait fort. 

\speak  Connaissez-vous Athos? 

\speak  Non. 

\speak  Porthos? 

\speak  Non. 

\speak  Aramis? 

\speak  Non. Quels sont ces messieurs? 

\speak  Des mousquetaires du roi. Connaissez-vous M. de Tréville, leur capitaine? 

\speak  Oh! oui, celui-là, je le connais, non pas personnellement, mais pour en avoir entendu plus d'une fois parler à la reine comme d'un brave et loyal gentilhomme. 

\speak  Vous ne craignez pas que lui vous trahisse pour le cardinal, n'est-ce pas? 

\speak  Oh! non, certainement. 

\speak  Eh bien, révélez-lui votre secret, et demandez-lui, si important, si précieux, si terrible qu'il soit, si vous pouvez me le confier. 

\speak  Mais ce secret ne m'appartient pas, et je ne puis le révéler ainsi. 

\speak  Vous l'alliez bien confier à M. Bonacieux, dit d'Artagnan avec dépit. 

\speak  Comme on confie une lettre au creux d'un arbre, à l'aile d'un pigeon, au collier d'un chien. 

\speak  Et cependant, moi, vous voyez bien que je vous aime. 

\speak  Vous le dites. 

\speak  Je suis un galant homme! 

\speak  Je le crois. 

\speak  Je suis brave! 

\speak  Oh! cela, j'en suis sûre. 

\speak  Alors, mettez-moi donc à l'épreuve.» 

Mme Bonacieux regarda le jeune homme, retenue par une dernière hésitation. Mais il y avait une telle ardeur dans ses yeux, une telle persuasion dans sa voix, qu'elle se sentit entraînée à se fier à lui. D'ailleurs elle se trouvait dans une de ces circonstances où il faut risquer le tout pour le tout. La reine était aussi bien perdue par une trop grande retenue que par une trop grande confiance. Puis, avouons-le, le sentiment involontaire qu'elle éprouvait pour ce jeune protecteur la décida à parler. 

«Écoutez, lui dit-elle, je me rends à vos protestations et je cède à vos assurances. Mais je vous jure devant Dieu qui nous entend, que si vous me trahissez et que mes ennemis me pardonnent, je me tuerai en vous accusant de ma mort. 

\speak  Et moi, je vous jure devant Dieu, madame, dit d'Artagnan, que si je suis pris en accomplissant les ordres que vous me donnez, je mourrai avant de rien faire ou dire qui compromette quelqu'un.» 

Alors la jeune femme lui confia le terrible secret dont le hasard lui avait déjà révélé une partie en face de la Samaritaine. Ce fut leur mutuelle déclaration d'amour. 

D'Artagnan rayonnait de joie et d'orgueil. Ce secret qu'il possédait, cette femme qu'il aimait, la confiance et l'amour, faisaient de lui un géant. 

«Je pars, dit-il, je pars sur-le-champ. 

\speak  Comment! vous partez! s'écria Mme Bonacieux, et votre régiment, votre capitaine? 

\speak  Sur mon âme, vous m'aviez fait oublier tout cela, chère Constance! oui, vous avez raison, il me faut un congé. 

\speak  Encore un obstacle, murmura Mme Bonacieux avec douleur. 

\speak  Oh! celui-là, s'écria d'Artagnan après un moment de réflexion, je le surmonterai, soyez tranquille. 

\speak  Comment cela? 

\speak  J'irai trouver ce soir même M. de Tréville, que je chargerai de demander pour moi cette faveur à son beau-frère, M. des Essarts. 

\speak  Maintenant, autre chose. 

\speak  Quoi? demanda d'Artagnan, voyant que Mme Bonacieux hésitait à continuer. 

\speak  Vous n'avez peut-être pas d'argent? 

\speak  Peut-être est de trop, dit d'Artagnan en souriant. 

\speak  Alors, reprit Mme Bonacieux en ouvrant une armoire et en tirant de cette armoire le sac qu'une demi-heure auparavant caressait si amoureusement son mari, prenez ce sac. 

\speak  Celui du cardinal! s'écria en éclatant de rire d'Artagnan qui, comme on s'en souvient, grâce à ses carreaux enlevés, n'avait pas perdu une syllabe de la conversation du mercier et de sa femme. 

\speak  Celui du cardinal, répondit Mme Bonacieux; vous voyez qu'il se présente sous un aspect assez respectable. 

\speak  Pardieu! s'écria d'Artagnan, ce sera une chose doublement divertissante que de sauver la reine avec l'argent de Son Éminence! 

\speak  Vous êtes un aimable et charmant jeune homme, dit Mme Bonacieux. Croyez que Sa Majesté ne sera point ingrate. 

\speak  Oh! je suis déjà grandement récompensé! s'écria d'Artagnan. Je vous aime, vous me permettez de vous le dire; c'est déjà plus de bonheur que je n'en osais espérer. 

\speak  Silence! dit Mme Bonacieux en tressaillant. 

\speak  Quoi? 

\speak  On parle dans la rue. 

\speak  C'est la voix\dots 

\speak  De mon mari. Oui, je l'ai reconnue!» 

D'Artagnan courut à la porte et poussa le verrou. 

«Il n'entrera pas que je ne sois parti, dit-il, et quand je serai parti, vous lui ouvrirez. 

\speak  Mais je devrais être partie aussi, moi. Et la disparition de cet argent, comment la justifier si je suis là? 

\speak  Vous avez raison, il faut sortir. 

\speak  Sortir, comment? On nous verra si nous sortons. 

\speak  Alors il faut monter chez moi. 

\speak  Ah! s'écria Mme Bonacieux, vous me dites cela d'un ton qui me fait peur.» 

Mme Bonacieux prononça ces paroles avec une larme dans les yeux. D'Artagnan vit cette larme, et, troublé, attendri, il se jeta à ses genoux. 

«Chez moi, dit-il, vous serez en sûreté comme dans un temple, je vous en donne ma parole de gentilhomme. 

\speak  Partons, dit-elle, je me fie à vous, mon ami.» 

D'Artagnan rouvrit avec précaution le verrou, et tous deux, légers comme des ombres, se glissèrent par la porte intérieure dans l'allée, montèrent sans bruit l'escalier et rentrèrent dans la chambre de d'Artagnan. 

Une fois chez lui, pour plus de sûreté, le jeune homme barricada la porte; ils s'approchèrent tous deux de la fenêtre, et par une fente du volet ils virent M. Bonacieux qui causait avec un homme en manteau. 

À la vue de l'homme en manteau, d'Artagnan bondit, et, tirant son épée à demi, s'élança vers la porte. 

C'était l'homme de Meung. 

«Qu'allez-vous faire? s'écria Mme Bonacieux; vous nous perdez. 

\speak  Mais j'ai juré de tuer cet homme! dit d'Artagnan. 

\speak  Votre vie est vouée en ce moment et ne vous appartient pas. Au nom de la reine, je vous défends de vous jeter dans aucun péril étranger à celui du voyage. 

\speak  Et en votre nom, n'ordonnez-vous rien? 

\speak  En mon nom, dit Mme Bonacieux avec une vive émotion; en mon nom, je vous en prie. Mais écoutons, il me semble qu'ils parlent de moi.» 

D'Artagnan se rapprocha de la fenêtre et prêta l'oreille. 

M. Bonacieux avait rouvert sa porte, et voyant l'appartement vide, il était revenu à l'homme au manteau qu'un instant il avait laissé seul. 

«Elle est partie, dit-il, elle sera retournée au Louvre. 

\speak  Vous êtes sûr, répondit l'étranger, qu'elle ne s'est pas doutée dans quelles intentions vous êtes sorti? 

\speak  Non, répondit Bonacieux avec suffisance; c'est une femme trop superficielle. 

\speak  Le cadet aux gardes est-il chez lui? 

\speak  Je ne le crois pas; comme vous le voyez, son volet est fermé, et l'on ne voit aucune lumière briller à travers les fentes. 

\speak  C'est égal, il faudrait s'en assurer. 

\speak  Comment cela? 

\speak  En allant frapper à sa porte. 

\speak  Je demanderai à son valet. 

\speak  Allez.» 

Bonacieux rentra chez lui, passa par la même porte qui venait de donner passage aux deux fugitifs, monta jusqu'au palier de d'Artagnan et frappa. 

Personne ne répondit. Porthos, pour faire plus grande figure, avait emprunté ce soir-là Planchet. Quant à d'Artagnan, il n'avait garde de donner signe d'existence. 

Au moment où le doigt de Bonacieux résonna sur la porte, les deux jeunes gens sentirent bondir leurs cœurs. 

«Il n'y a personne chez lui, dit Bonacieux. 

\speak  N'importe, rentrons toujours chez vous, nous serons plus en sûreté que sur le seuil d'une porte. 

\speak  Ah! mon Dieu! murmura Mme Bonacieux, nous n'allons plus rien entendre. 

\speak  Au contraire, dit d'Artagnan, nous n'entendrons que mieux.» 

D'Artagnan enleva les trois ou quatre carreaux qui faisaient de sa chambre une autre oreille de Denys, étendit un tapis à terre, se mit à genoux, et fit signe à Mme Bonacieux de se pencher, comme il le faisait vers l'ouverture. 

«Vous êtes sûr qu'il n'y a personne? dit l'inconnu. 

\speak  J'en réponds, dit Bonacieux. 

\speak  Et vous pensez que votre femme?\dots 

\speak  Est retournée au Louvre. 

\speak  Sans parler à aucune personne qu'à vous? 

\speak  J'en suis sûr. 

\speak  C'est un point important, comprenez-vous? 

\speak  Ainsi, la nouvelle que je vous ai apportée a donc une valeur\dots? 

\speak  Très grande, mon cher Bonacieux, je ne vous le cache pas. 

\speak  Alors le cardinal sera content de moi? 

\speak  Je n'en doute pas. 

\speak  Le grand cardinal! 

\speak  Vous êtes sûr que, dans sa conversation avec vous, votre femme n'a pas prononcé de noms propres? 

\speak  Je ne crois pas. 

\speak  Elle n'a nommé ni Mme de Chevreuse, ni M. de Buckingham, ni Mme de Vernet? 

\speak  Non, elle m'a dit seulement qu'elle voulait m'envoyer à Londres pour servir les intérêts d'une personne illustre.» 

«Le traître! murmura Mme Bonacieux. 

\speak  Silence!» dit d'Artagnan en lui prenant une main qu'elle lui abandonna sans y penser. 

«N'importe, continua l'homme au manteau, vous êtes un niais de n'avoir pas feint d'accepter la commission, vous auriez la lettre à présent; État qu'on menace était sauvé, et vous\dots 

\speak  Et moi? 

\speak  Eh bien, vous! le cardinal vous donnait des lettres de noblesse\dots 

\speak  Il vous l'a dit? 

\speak  Oui, je sais qu'il voulait vous faire cette surprise. 

\speak  Soyez tranquille, reprit Bonacieux; ma femme m'adore, et il est encore temps.» 

«Le niais! murmura Mme Bonacieux. 

\speak  Silence!» dit d'Artagnan en lui serrant plus fortement la main. 

«Comment est-il encore temps? reprit l'homme au manteau. 

\speak  Je retourne au Louvre, je demande Mme Bonacieux, je dis que j'ai réfléchi, je renoue l'affaire, j'obtiens la lettre, et je cours chez le cardinal. 

\speak  Eh bien, allez vite; je reviendrai bientôt savoir le résultat de votre démarche.» 

L'inconnu sortit. 

«L'infâme! dit Mme Bonacieux en adressant encore cette épithète à son mari. 

\speak  Silence!» répéta d'Artagnan en lui serrant la main plus fortement encore. 

Un hurlement terrible interrompit alors les réflexions de d'Artagnan et de Mme Bonacieux. C'était son mari, qui s'était aperçu de la disparition de son sac et qui criait au voleur. 

«Oh! mon Dieu! s'écria Mme Bonacieux, il va ameuter tout le quartier.» 

Bonacieux cria longtemps; mais comme de pareils cris, attendu leur fréquence, n'attiraient personne dans la rue des Fossoyeurs, et que d'ailleurs la maison du mercier était depuis quelque temps assez mal famée, voyant que personne ne venait, il sortit en continuant de crier, et l'on entendit sa voix qui s'éloignait dans la direction de la rue du Bac. 

«Et maintenant qu'il est parti, à votre tour de vous éloigner, dit Mme Bonacieux; du courage, mais surtout de la prudence, et songez que vous vous devez à la reine. 

\speak  À elle et à vous! s'écria d'Artagnan. Soyez tranquille, belle Constance, je reviendrai digne de sa reconnaissance; mais reviendrai-je aussi digne de votre amour?» 

La jeune femme ne répondit que par la vive rougeur qui colora ses joues. Quelques instants après, d'Artagnan sortit à son tour, enveloppé, lui aussi, d'un grand manteau que retroussait cavalièrement le fourreau d'une longue épée. 

Mme Bonacieux le suivit des yeux avec ce long regard d'amour dont la femme accompagne l'homme qu'elle se sent aimer; mais lorsqu'il eut disparu à l'angle de la rue, elle tomba à genoux, et joignant les mains: 

«O mon Dieu! s'écria-t-elle, protégez la reine, protégez-moi!»
%!TeX root=../musketeersfr.tex 

\chapter{Plan De Campagne} 
	
\lettrine{D}{'Artagnan} se rendit droit chez M. de Tréville. Il avait réfléchi que, dans quelques minutes, le cardinal serait averti par ce damné inconnu, qui paraissait être son agent, et il pensait avec raison qu'il n'y avait pas un instant à perdre. 

Le cœur du jeune homme débordait de joie. Une occasion où il y avait à la fois gloire à acquérir et argent à gagner se présentait à lui, et, comme premier encouragement, venait de le rapprocher d'une femme qu'il adorait. Ce hasard faisait donc presque du premier coup, pour lui, plus qu'il n'eût osé demander à la Providence. 

M. de Tréville était dans son salon avec sa cour habituelle de gentilshommes. D'Artagnan, que l'on connaissait comme un familier de la maison, alla droit à son cabinet et le fit prévenir qu'il l'attendait pour chose d'importance. 

D'Artagnan était là depuis cinq minutes à peine, lorsque M. de Tréville entra. Au premier coup d'œil et à la joie qui se peignait sur son visage, le digne capitaine comprit qu'il se passait effectivement quelque chose de nouveau. 

Tout le long de la route, d'Artagnan s'était demandé s'il se confierait à M. de Tréville, ou si seulement il lui demanderait de lui accorder carte blanche pour une affaire secrète. Mais M. de Tréville avait toujours été si parfait pour lui, il était si fort dévoué au roi et à la reine, il haïssait si cordialement le cardinal, que le jeune homme résolut de tout lui dire. 

«Vous m'avez fait demander, mon jeune ami? dit M. de Tréville. 

\speak  Oui, monsieur, dit d'Artagnan, et vous me pardonnerez, je l'espère, de vous avoir dérangé, quand vous saurez de quelle chose importante il est question. 

\speak  Dites alors, je vous écoute. 

\speak  Il ne s'agit de rien de moins, dit d'Artagnan, en baissant la voix, que de l'honneur et peut-être de la vie de la reine. 

\speak  Que dites-vous là? demanda M. de Tréville en regardant tout autour de lui s'ils étaient bien seuls, et en ramenant son regard interrogateur sur d'Artagnan. 

\speak  Je dis, monsieur, que le hasard m'a rendu maître d'un secret\dots 

\speak  Que vous garderez, j'espère, jeune homme, sur votre vie. 

\speak  Mais que je dois vous confier, à vous, Monsieur, car vous seul pouvez m'aider dans la mission que je viens de recevoir de Sa Majesté. 

\speak  Ce secret est-il à vous? 

\speak  Non, monsieur, c'est celui de la reine. 

\speak  Êtes-vous autorisé par Sa Majesté à me le confier? 

\speak  Non, monsieur, car au contraire le plus profond mystère m'est recommandé. 

\speak  Et pourquoi donc allez-vous le trahir vis-à-vis de moi? 

\speak  Parce que, je vous le dis, sans vous je ne puis rien, et que j'ai peur que vous ne me refusiez la grâce que je viens vous demander, si vous ne savez pas dans quel but je vous la demande. 

\speak  Gardez votre secret, jeune homme, et dites-moi ce que vous désirez. 

\speak  Je désire que vous obteniez pour moi, de M. des Essarts, un congé de quinze jours. 

\speak  Quand cela? 

\speak  Cette nuit même. 

\speak  Vous quittez Paris? 

\speak  Je vais en mission. 

\speak  Pouvez-vous me dire où? 

\speak  À Londres. 

\speak  Quelqu'un a-t-il intérêt à ce que vous n'arriviez pas à votre but? 

\speak  Le cardinal, je le crois, donnerait tout au monde pour m'empêcher de réussir. 

\speak  Et vous partez seul? 

\speak  Je pars seul. 

\speak  En ce cas, vous ne passerez pas Bondy; c'est moi qui vous le dis, foi de Tréville. 

\speak  Comment cela? 

\speak  On vous fera assassiner. 

\speak  Je serai mort en faisant mon devoir. 

\speak  Mais votre mission ne sera pas remplie. 

\speak  C'est vrai, dit d'Artagnan. 

\speak  Croyez-moi, continua Tréville, dans les entreprises de ce genre, il faut être quatre pour arriver un. 

\speak  Ah! vous avez raison, Monsieur, dit d'Artagnan; mais vous connaissez Athos, Porthos et Aramis, et vous savez si je puis disposer d'eux. 

\speak  Sans leur confier le secret que je n'ai pas voulu savoir? 

\speak  Nous nous sommes juré, une fois pour toutes, confiance aveugle et dévouement à toute épreuve; d'ailleurs vous pouvez leur dire que vous avez toute confiance en moi, et ils ne seront pas plus incrédules que vous. 

\speak  Je puis leur envoyer à chacun un congé de quinze jours, voilà tout: à Athos, que sa blessure fait toujours souffrir, pour aller aux eaux de Forges! à Porthos et à Aramis, pour suivre leur ami, qu'ils ne veulent pas abandonner dans une si douloureuse position. L'envoi de leur congé sera la preuve que j'autorise leur voyage. 

\speak  Merci, monsieur, et vous êtes cent fois bon. 

\speak  Allez donc les trouver à l'instant même, et que tout s'exécute cette nuit. Ah! et d'abord écrivez-moi votre requête à M. des Essarts. Peut-être aviez-vous un espion à vos trousses, et votre visite, qui dans ce cas est déjà connue du cardinal, sera légitimée ainsi.» 

D'Artagnan formula cette demande, et M. de Tréville, en la recevant de ses mains, assura qu'avant deux heures du matin les quatre congés seraient au domicile respectif des voyageurs. 

«Ayez la bonté d'envoyer le mien chez Athos, dit d'Artagnan. Je craindrais, en rentrant chez moi, d'y faire quelque mauvaise rencontre. 

\speak  Soyez tranquille. Adieu et bon voyage! À propos!» dit M. de Tréville en le rappelant. 

D'Artagnan revint sur ses pas. 

«Avez-vous de l'argent?» 

D'Artagnan fit sonner le sac qu'il avait dans sa poche. 

«Assez? demanda M. de Tréville. 

\speak  Trois cents pistoles. 

\speak  C'est bien, on va au bout du monde avec cela; allez donc.» 

D'Artagnan salua M. de Tréville, qui lui tendit la main; d'Artagnan la lui serra avec un respect mêlé de reconnaissance. Depuis qu'il était arrivé à Paris, il n'avait eu qu'à se louer de cet excellent homme, qu'il avait toujours trouvé digne, loyal et grand. 

Sa première visite fut pour Aramis; il n'était pas revenu chez son ami depuis la fameuse soirée où il avait suivi Mme Bonacieux. Il y a plus: à peine avait-il vu le jeune mousquetaire, et à chaque fois qu'il l'avait revu, il avait cru remarquer une profonde tristesse empreinte sur son visage. 

Ce soir encore, Aramis veillait sombre et rêveur; d'Artagnan lui fit quelques questions sur cette mélancolie profonde; Aramis s'excusa sur un commentaire du dix-huitième chapitre de saint Augustin qu'il était forcé d'écrire en latin pour la semaine suivante, et qui le préoccupait beaucoup. 

Comme les deux amis causaient depuis quelques instants, un serviteur de M. de Tréville entra porteur d'un paquet cacheté. 

«Qu'est-ce là? demanda Aramis. 

\speak  Le congé que monsieur a demandé, répondit le laquais. 

\speak  Moi, je n'ai pas demandé de congé. 

\speak  Taisez-vous et prenez, dit d'Artagnan. Et vous, mon ami, voici une demi-pistole pour votre peine; vous direz à M. de Tréville que M. Aramis le remercie bien sincèrement. Allez.» 

Le laquais salua jusqu'à terre et sortit. 

«Que signifie cela? demanda Aramis. 

\speak  Prenez ce qu'il vous faut pour un voyage de quinze jours, et suivez-moi. 

\speak  Mais je ne puis quitter Paris en ce moment, sans savoir\dots» 

Aramis s'arrêta. 

«Ce qu'elle est devenue, n'est-ce pas? continua d'Artagnan. 

\speak  Qui? reprit Aramis. 

\speak  La femme qui était ici, la femme au mouchoir brodé. 

\speak  Qui vous a dit qu'il y avait une femme ici? répliqua Aramis en devenant pâle comme la mort. 

\speak  Je l'ai vue. 

\speak  Et vous savez qui elle est? 

\speak  Je crois m'en douter, du moins. 

\speak  Écoutez, dit Aramis, puisque vous savez tant de choses, savez-vous ce qu'est devenue cette femme? 

\speak  Je présume qu'elle est retournée à Tours. 

\speak  À Tours? oui, c'est bien cela, vous la connaissez. Mais comment est-elle retournée à Tours sans me rien dire? 

\speak  Parce qu'elle a craint d'être arrêtée. 

\speak  Comment ne m'a-t-elle pas écrit? 

\speak  Parce qu'elle craint de vous compromettre. 

\speak  D'Artagnan, vous me rendez la vie! s'écria Aramis. Je me croyais méprisé, trahi. J'étais si heureux de la revoir! Je ne pouvais croire qu'elle risquât sa liberté pour moi, et cependant pour quelle cause serait-elle revenue à Paris? 

\speak  Pour la cause qui aujourd'hui nous fait aller en Angleterre. 

\speak  Et quelle est cette cause? demanda Aramis. 

\speak  Vous le saurez un jour, Aramis; mais, pour le moment, j'imiterai la retenue de la \textit{nièce du docteur}.» 

Aramis sourit, car il se rappelait le conte qu'il avait fait certain soir à ses amis. 

«Eh bien, donc, puisqu'elle a quitté Paris et que vous en êtes sûr, d'Artagnan, rien ne m'y arrête plus, et je suis prêt à vous suivre. Vous dites que nous allons?\dots 

\speak  Chez Athos, pour le moment, et si vous voulez venir, je vous invite même à vous hâter, car nous avons déjà perdu beaucoup de temps. À propos, prévenez Bazin. 

\speak  Bazin vient avec nous? demanda Aramis. 

\speak  Peut-être. En tout cas, il est bon qu'il nous suive pour le moment chez Athos.» 

Aramis appela Bazin, et après lui avoir ordonné de le venir joindre chez Athos: 

«Partons donc», dit-il en prenant son manteau, son épée et ses trois pistolets, et en ouvrant inutilement trois ou quatre tiroirs pour voir s'il n'y trouverait pas quelque pistole égarée. Puis, quand il se fut bien assuré que cette recherche était superflue, il suivit d'Artagnan en se demandant comment il se faisait que le jeune cadet aux gardes sût aussi bien que lui quelle était la femme à laquelle il avait donné l'hospitalité, et sût mieux que lui ce qu'elle était devenue. 

Seulement, en sortant, Aramis posa sa main sur le bras de d'Artagnan, et le regardant fixement: 

«Vous n'avez parlé de cette femme à personne? dit-il. 

\speak  À personne au monde. 

\speak  Pas même à Athos et à Porthos? 

\speak  Je ne leur en ai pas soufflé le moindre mot. 

\speak  À la bonne heure.» 

Et, tranquille sur ce point important, Aramis continua son chemin avec d'Artagnan, et tous deux arrivèrent bien tôt chez Athos. 

Ils le trouvèrent tenant son congé d'une main et la lettre de M. de Tréville de l'autre. 

«Pouvez-vous m'expliquer ce que signifient ce congé et cette lettre que je viens de recevoir?» dit Athos étonné. 

\begin{mail}{}{Mon cher Athos,}
Je veux bien, puisque votre santé l'exige absolument, que vous vous reposiez quinze jours. Allez donc prendre les eaux de Forges ou telles autres qui vous conviendront, et rétablissez-vous promptement 

\closeletter[Votre affectionné]{Tréville}
\end{mail}


«Eh bien, ce congé et cette lettre signifient qu'il faut me suivre, Athos. 

\speak  Aux eaux de Forges? 

\speak  Là ou ailleurs. 

\speak  Pour le service du roi? 

\speak  Du roi ou de la reine: ne sommes-nous pas serviteurs de Leurs Majestés?» 

En ce moment, Porthos entra. 

«Pardieu, dit-il, voici une chose étrange: depuis quand, dans les mousquetaires, accorde-t-on aux gens des congés sans qu'ils les demandent? 

\speak  Depuis, dit d'Artagnan, qu'ils ont des amis qui les demandent pour eux. 

\speak  Ah! ah! dit Porthos, il paraît qu'il y a du nouveau ici? 

\speak  Oui, nous partons, dit Aramis. 

\speak  Pour quel pays? demanda Porthos. 

\speak  Ma foi, je n'en sais trop rien, dit Athos; demande cela à d'Artagnan. 

\speak  Pour Londres, messieurs, dit d'Artagnan. 

\speak  Pour Londres! s'écria Porthos; et qu'allons-nous faire à Londres? 

\speak  Voilà ce que je ne puis vous dire, messieurs, et il faut vous fier à moi. 

\speak  Mais pour aller à Londres, ajouta Porthos, il faut de l'argent, et je n'en ai pas. 

\speak  Ni moi, dit Aramis. 

\speak  Ni moi, dit Athos. 

\speak  J'en ai, moi, reprit d'Artagnan en tirant son trésor de sa poche et en le posant sur la table. Il y a dans ce sac trois cents pistoles; prenons-en chacun soixante-quinze; c'est autant qu'il en faut pour aller à Londres et pour en revenir. D'ailleurs, soyez tranquilles, nous n'y arriverons pas tous, à Londres. 

\speak  Et pourquoi cela? 

\speak  Parce que, selon toute probabilité, il y en aura quelques-uns d'entre nous qui resteront en route. 

\speak  Mais est-ce donc une campagne que nous entreprenons? 

\speak  Et des plus dangereuses, je vous en avertis. 

\speak  Ah çà, mais, puisque nous risquons de nous faire tuer, dit Porthos, je voudrais bien savoir pourquoi, au moins? 

\speak  Tu en seras bien plus avancé! dit Athos. 

\speak  Cependant, dit Aramis, je suis de l'avis de Porthos. 

\speak  Le roi a-t-il l'habitude de vous rendre des comptes? Non; il vous dit tout bonnement: “Messieurs, on se bat en Gascogne ou dans les Flandres; allez vous battre”, et vous y allez. Pourquoi? vous ne vous en inquiétez même pas. 

\speak  D'Artagnan a raison, dit Athos, voilà nos trois congés qui viennent de M. de Tréville, et voilà trois cents pistoles qui viennent je ne sais d'où. Allons nous faire tuer où l'on nous dit d'aller. La vie vaut-elle la peine de faire autant de questions? D'Artagnan, je suis prêt à te suivre. 

\speak  Et moi aussi, dit Porthos. 

\speak  Et moi aussi, dit Aramis. Aussi bien, je ne suis pas fâché de quitter Paris. J'ai besoin de distractions. 

\speak  Eh bien, vous en aurez, des distractions, messieurs, soyez tranquilles, dit d'Artagnan. 

\speak  Et maintenant, quand partons-nous? dit Athos. 

\speak  Tout de suite, répondit d'Artagnan, il n'y a pas une minute à perdre. 

\speak  Holà! Grimaud, Planchet, Mousqueton, Bazin! crièrent les quatre jeunes gens appelant leurs laquais, graissez nos bottes et ramenez les chevaux de l'hôtel.» 

En effet, chaque mousquetaire laissait à l'hôtel général comme à une caserne son cheval et celui de son laquais. 

Planchet, Grimaud, Mousqueton et Bazin partirent en toute hâte. 

«Maintenant, dressons le plan de campagne, dit Porthos. Où allons-nous d'abord? 

\speak  À Calais, dit d'Artagnan; c'est la ligne la plus directe pour arriver à Londres. 

\speak  Eh bien, dit Porthos, voici mon avis. 

\speak  Parle. 

\speak  Quatre hommes voyageant ensemble seraient suspects: d'Artagnan nous donnera à chacun ses instructions, je partirai en avant par la route de Boulogne pour éclairer le chemin; Athos partira deux heures après par celle d'Amiens; Aramis nous suivra par celle de Noyon; quant à d'Artagnan, il partira par celle qu'il voudra, avec les habits de Planchet, tandis que Planchet nous suivra en d'Artagnan et avec l'uniforme des gardes. 

\speak  Messieurs, dit Athos, mon avis est qu'il ne convient pas de mettre en rien des laquais dans une pareille affaire: un secret peut par hasard être trahi par des gentilshommes, mais il est presque toujours vendu par des laquais. 

\speak  Le plan de Porthos me semble impraticable, dit d'Artagnan, en ce que j'ignore moi-même quelles instructions je puis vous donner. Je suis porteur d'une lettre, voilà tout. Je n'ai pas et ne puis faire trois copies de cette lettre, puisqu'elle est scellée; il faut donc, à mon avis, voyager de compagnie. Cette lettre est là, dans cette poche. Et il montra la poche où était la lettre. Si je suis tué, l'un de vous la prendra et vous continuerez la route; s'il est tué, ce sera le tour d'un autre, et ainsi de suite; pourvu qu'un seul arrive, c'est tout ce qu'il faut. 

\speak  Bravo, d'Artagnan! ton avis est le mien, dit Athos. Il faut être conséquent, d'ailleurs: je vais prendre les eaux, vous m'accompagnerez; au lieu des eaux de Forges, je vais prendre les eaux de mer; je suis libre. On veut nous arrêter, je montre la lettre de M. de Tréville, et vous montrez vos congés; on nous attaque, nous nous défendons; on nous juge, nous soutenons mordicus que nous n'avions d'autre intention que de nous tremper un certain nombre de fois dans la mer; on aurait trop bon marché de quatre hommes isolés, tandis que quatre hommes réunis font une troupe. Nous armerons les quatre laquais de pistolets et de mousquetons; si l'on envoie une armée contre nous, nous livrerons bataille, et le survivant, comme l'a dit d'Artagnan, portera la lettre. 

\speak  Bien dit, s'écria Aramis; tu ne parles pas souvent, Athos, mais quand tu parles, c'est comme saint Jean Bouche d'or. J'adopte le plan d'Athos. Et toi, Porthos? 

\speak  Moi aussi, dit Porthos, s'il convient à d'Artagnan. D'Artagnan, porteur de la lettre, est naturellement le chef de l'entreprise; qu'il décide, et nous exécuterons. 

\speak  Eh bien, dit d'Artagnan, je décide que nous adoptions le plan d'Athos et que nous partions dans une demi-heure. 

\speak  Adopté!» reprirent en chœur les trois mousquetaires. 

Et chacun, allongeant la main vers le sac, prit soixante-quinze pistoles et fit ses préparatifs pour partir à l'heure convenue. 
%!TeX root=../musketeersfr.tex 

\chapter{Voyage} 
	
	\lettrine{\accentletter[\gravebox]{A}}{} deux heures du matin, nos quatre aventuriers sortirent de Paris par la barrière Saint-Denis; tant qu'il fit nuit, ils restèrent muets; malgré eux, ils subissaient l'influence de l'obscurité et voyaient des embûches partout. 

Aux premiers rayons du jour, leurs langues se délièrent; avec le soleil, la gaieté revint: c'était comme à la veille d'un combat, le cœur battait, les yeux riaient; on sentait que la vie qu'on allait peut-être quitter était, au bout du compte, une bonne chose. 

L'aspect de la caravane, au reste, était des plus formidables: les chevaux noirs des mousquetaires, leur tournure martiale, cette habitude de l'escadron qui fait marcher régulièrement ces nobles compagnons du soldat, eussent trahi le plus strict incognito. 

Les valets suivaient, armés jusqu'aux dents. 

Tout alla bien jusqu'à Chantilly, où l'on arriva vers les huit heures du matin. Il fallait déjeuner. On descendit devant une auberge que recommandait une enseigne représentant saint Martin donnant la moitié de son manteau à un pauvre. On enjoignit aux laquais de ne pas desseller les chevaux et de se tenir prêts à repartir immédiatement. 

On entra dans la salle commune, et l'on se mit à table. Un gentilhomme, qui venait d'arriver par la route de Dammartin, était assis à cette même table et déjeunait. Il entama la conversation sur la pluie et le beau temps; les voyageurs répondirent: il but à leur santé; les voyageurs lui rendirent sa politesse. 

Mais au moment où Mousqueton venait annoncer que les chevaux étaient prêts et où l'on se levait de table l'étranger proposa à Porthos la santé du cardinal. Porthos répondit qu'il ne demandait pas mieux, si l'étranger à son tour voulait boire à la santé du roi. L'étranger s'écria qu'il ne connaissait d'autre roi que Son Éminence. Porthos l'appela ivrogne; l'étranger tira son épée. 

«Vous avez fait une sottise, dit Athos; n'importe, il n'y a plus à reculer maintenant: tuez cet homme et venez nous rejoindre le plus vite que vous pourrez.» 

Et tous trois remontèrent à cheval et repartirent à toute bride, tandis que Porthos promettait à son adversaire de le perforer de tous les coups connus dans l'escrime. 

«Et d'un! dit Athos au bout de cinq cents pas. 

\speak  Mais pourquoi cet homme s'est-il attaqué à Porthos plutôt qu'à tout autre? demanda Aramis. 

\speak  Parce que, Porthos parlant plus haut que nous tous il l'a pris pour le chef, dit d'Artagnan. 

\speak  J'ai toujours dit que ce cadet de Gascogne était un puits de sagesse», murmura Athos. 

Et les voyageurs continuèrent leur route. 

À Beauvais, on s'arrêta deux heures, tant pour faire souffler les chevaux que pour attendre Porthos. Au bout de deux heures, comme Porthos n'arrivait pas, ni aucune nouvelle de lui, on se remit en chemin. 

À une lieue de Beauvais, à un endroit où le chemin se trouvait resserré entre deux talus, on rencontra huit ou dix hommes qui, profitant de ce que la route était dépavée en cet endroit, avaient l'air d'y travailler en y creusant des trous et en pratiquant des ornières boueuses. 

Aramis, craignant de salir ses bottes dans ce mortier artificiel, les apostropha durement. Athos voulut le retenir, il était trop tard. Les ouvriers se mirent à railler les voyageurs, et firent perdre par leur insolence la tête même au froid Athos qui poussa son cheval contre l'un d'eux. 

Alors chacun de ces hommes recula jusqu'au fossé et y prit un mousquet caché; il en résulta que nos sept voyageurs furent littéralement passés par les armes. Aramis reçut une balle qui lui traversa l'épaule, et Mousqueton une autre balle qui se logea dans les parties charnues qui prolongent le bas des reins. Cependant Mousqueton seul tomba de cheval, non pas qu'il fût grièvement blessé, mais, comme il ne pouvait voir sa blessure, sans doute il crut être plus dangereusement blessé qu'il ne l'était. 

«C'est une embuscade, dit d'Artagnan, ne brûlons pas une amorce, et en route.» 

Aramis, tout blessé qu'il était, saisit la crinière de son cheval, qui l'emporta avec les autres. Celui de Mousqueton les avait rejoints, et galopait tout seul à son rang. 

«Cela nous fera un cheval de rechange, dit Athos. 

\speak  J'aimerais mieux un chapeau, dit d'Artagnan, le mien a été emporté par une balle. C'est bien heureux, ma foi, que la lettre que je porte n'ait pas été dedans. 

\speak  Ah çà, mais ils vont tuer le pauvre Porthos quand il passera, dit Aramis. 

\speak  Si Porthos était sur ses jambes, il nous aurait rejoints maintenant, dit Athos. M'est avis que, sur le terrain, l'ivrogne se sera dégrisé.» 

Et l'on galopa encore pendant deux heures, quoique les chevaux fussent si fatigués, qu'il était à craindre qu'ils ne refusassent bientôt le service. 

Les voyageurs avaient pris la traverse, espérant de cette façon être moins inquiétés, mais, à Crève-cœur, Aramis déclara qu'il ne pouvait aller plus loin. En effet, il avait fallu tout le courage qu'il cachait sous sa forme élégante et sous ses façons polies pour arriver jusque-là. À tout moment il pâlissait, et l'on était obligé de le soutenir sur son cheval; on le descendit à la porte d'un cabaret, on lui laissa Bazin qui, au reste, dans une escarmouche, était plus embarrassant qu'utile, et l'on repartit dans l'espérance d'aller coucher à Amiens. 

«Morbleu! dit Athos, quand ils se retrouvèrent en route, réduits à deux maîtres et à Grimaud et Planchet, morbleu! je ne serai plus leur dupe, et je vous réponds qu'ils ne me feront pas ouvrir la bouche ni tirer l'épée d'ici à Calais. J'en jure\dots 

\speak  Ne jurons pas, dit d'Artagnan, galopons, si toutefois nos chevaux y consentent.» 

Et les voyageurs enfoncèrent leurs éperons dans le ventre de leurs chevaux, qui, vigoureusement stimulés, retrouvèrent des forces. On arriva à Amiens à minuit, et l'on descendit à l'auberge du Lis d'Or. 

L'hôtelier avait l'air du plus honnête homme de la terre, il reçut les voyageurs son bougeoir d'une main et son bonnet de coton de l'autre; il voulut loger les deux voyageurs chacun dans une charmante chambre, malheureusement chacune de ces chambres était à l'extrémité de l'hôtel. D'Artagnan et Athos refusèrent; l'hôte répondit qu'il n'y en avait cependant pas d'autres dignes de Leurs Excellences; mais les voyageurs déclarèrent qu'ils coucheraient dans la chambre commune, chacun sur un matelas qu'on leur jetterait à terre. L'hôte insista, les voyageurs tinrent bon; il fallut faire ce qu'ils voulurent. 

Ils venaient de disposer leur lit et de barricader leur porte en dedans, lorsqu'on frappa au volet de la cour; ils demandèrent qui était là, reconnurent la voix de leurs valets et ouvrirent. 

En effet, c'étaient Planchet et Grimaud. 

«Grimaud suffira pour garder les chevaux, dit Planchet; si ces messieurs veulent, je coucherai en travers de leur porte; de cette façon-là, ils seront sûrs qu'on n'arrivera pas jusqu'à eux. 

\speak  Et sur quoi coucheras-tu? dit d'Artagnan. 

\speak  Voici mon lit», répondit Planchet. 

Et il montra une botte de paille. 

«Viens donc, dit d'Artagnan, tu as raison: la figure de l'hôte ne me convient pas, elle est trop gracieuse. 

\speak  Ni à moi non plus», dit Athos. 

Planchet monta par la fenêtre, s'installa en travers de la porte, tandis que Grimaud allait s'enfermer dans l'écurie, répondant qu'à cinq heures du matin lui et les quatre chevaux seraient prêts. 

La nuit fut assez tranquille, on essaya bien vers les deux heures du matin d'ouvrir la porte, mais comme Planchet se réveilla en sursaut et cria: Qui va là? on répondit qu'on se trompait, et on s'éloigna. 

À quatre heures du matin, on entendit un grand bruit dans les écuries. Grimaud avait voulu réveiller les garçons d'écurie, et les garçons d'écurie le battaient. Quand on ouvrit la fenêtre, on vit le pauvre garçon sans connaissance, la tête fendue d'un coup de manche à fourche. 

Planchet descendit dans la cour et voulut seller les chevaux; les chevaux étaient fourbus. Celui de Mousqueton seul, qui avait voyagé sans maître pendant cinq ou six heures la veille, aurait pu continuer la route; mais, par une erreur inconcevable, le chirurgien vétérinaire qu'on avait envoyé chercher, à ce qu'il paraît, pour saigner le cheval de l'hôte, avait saigné celui de Mousqueton. 

Cela commençait à devenir inquiétant: tous ces accidents successifs étaient peut-être le résultat du hasard, mais ils pouvaient tout aussi bien être le fruit d'un complot. Athos et d'Artagnan sortirent, tandis que Planchet allait s'informer s'il n'y avait pas trois chevaux à vendre dans les environs. À la porte étaient deux chevaux tout équipés, frais et vigoureux. Cela faisait bien l'affaire. Il demanda où étaient les maîtres; on lui dit que les maîtres avaient passé la nuit dans l'auberge et réglaient leur compte à cette heure avec le maître. 

Athos descendit pour payer la dépense, tandis que d'Artagnan et Planchet se tenaient sur la porte de la rue; l'hôtelier était dans une chambre basse et reculée, on pria Athos d'y passer. 

Athos entra sans défiance et tira deux pistoles pour payer: l'hôte était seul et assis devant son bureau, dont un des tiroirs était entrouvert. Il prit l'argent que lui présenta Athos, le tourna et le retourna dans ses mains, et tout à coup, s'écriant que la pièce était fausse, il déclara qu'il allait le faire arrêter, lui et son compagnon, comme faux-monnayeurs. 

«Drôle! dit Athos, en marchant sur lui, je vais te couper les oreilles!» 

Au même moment, quatre hommes armés jusqu'aux dents entrèrent par les portes latérales et se jetèrent sur Athos. 

«Je suis pris, cria Athos de toutes les forces de ses poumons; au large, d'Artagnan! pique, pique!» et il lâcha deux coups de pistolet. 

D'Artagnan et Planchet ne se le firent pas répéter à deux fois, ils détachèrent les deux chevaux qui attendaient à la porte, sautèrent dessus, leur enfoncèrent leurs éperons dans le ventre et partirent au triple galop. 

«Sais-tu ce qu'est devenu Athos? demanda d'Artagnan à Planchet en courant. 

\speak  Ah! monsieur, dit Planchet, j'en ai vu tomber deux à ses deux coups, et il m'a semblé, à travers la porte vitrée, qu'il ferraillait avec les autres. 

\speak  Brave Athos! murmura d'Artagnan. Et quand on pense qu'il faut l'abandonner! Au reste, autant nous attend peut-être à deux pas d'ici. En avant, Planchet, en avant! tu es un brave homme. 

\speak  Je vous l'ai dit, monsieur, répondit Planchet, les Picards, ça se reconnaît à l'user; d'ailleurs je suis ici dans mon pays, ça m'excite.» 

Et tous deux, piquant de plus belle, arrivèrent à Saint-Omer d'une seule traite. À Saint-Omer, ils firent souffler les chevaux la bride passée à leurs bras, de peur d'accident, et mangèrent un morceau sur le pouce tout debout dans la rue; après quoi ils repartirent. 

À cent pas des portes de Calais, le cheval de d'Artagnan s'abattit, et il n'y eut pas moyen de le faire se relever: le sang lui sortait par le nez et par les yeux, restait celui de Planchet, mais celui-là s'était arrêté, et il n'y eut plus moyen de le faire repartir. 

Heureusement, comme nous l'avons dit, ils étaient à cent pas de la ville; ils laissèrent les deux montures sur le grand chemin et coururent au port. Planchet fit remarquer à son maître un gentilhomme qui arrivait avec son valet et qui ne les précédait que d'une cinquantaine de pas. 

Ils s'approchèrent vivement de ce gentilhomme, qui paraissait fort affairé. Il avait ses bottes couvertes de poussière, et s'informait s'il ne pourrait point passer à l'instant même en Angleterre. 

«Rien ne serait plus facile, répondit le patron d'un bâtiment prêt à mettre à la voile; mais, ce matin, est arrivé l'ordre de ne laisser partir personne sans une permission expresse de M. le cardinal. 

\speak  J'ai cette permission, dit le gentilhomme en tirant un papier de sa poche; la voici. 

\speak  Faites-la viser par le gouverneur du port, dit le patron, et donnez-moi la préférence. 

\speak  Où trouverai-je le gouverneur? 

\speak  À sa campagne. 

\speak  Et cette campagne est située? 

\speak  À un quart de lieue de la ville; tenez, vous la voyez d'ici, au pied de cette petite éminence, ce toit en ardoises. 

\speak  Très bien!» dit le gentilhomme. 

Et, suivi de son laquais, il prit le chemin de la maison de campagne du gouverneur. 

D'Artagnan et Planchet suivirent le gentilhomme à cinq cents pas de distance. 

Une fois hors de la ville, d'Artagnan pressa le pas et rejoignit le gentilhomme comme il entrait dans un petit bois. 

«Monsieur, lui dit d'Artagnan, vous me paraissez fort pressé? 

\speak  On ne peut plus pressé, monsieur. 

\speak  J'en suis désespéré, dit d'Artagnan, car, comme je suis très pressé aussi, je voulais vous prier de me rendre un service. 

\speak  Lequel? 

\speak  De me laisser passer le premier. 

\speak  Impossible, dit le gentilhomme, j'ai fait soixante lieues en quarante-quatre heures, et il faut que demain à midi je sois à Londres. 

\speak  J'ai fait le même chemin en quarante heures, et il faut que demain à dix heures du matin je sois à Londres. 

\speak  Désespéré, monsieur; mais je suis arrivé le premier et je ne passerai pas le second. 

\speak  Désespéré, monsieur; mais je suis arrivé le second et je passerai le premier. 

\speak  Service du roi! dit le gentilhomme. 

\speak  Service de moi! dit d'Artagnan. 

\speak  Mais c'est une mauvaise querelle que vous me cherchez là, ce me semble. 

\speak  Parbleu! que voulez-vous que ce soit? 

\speak  Que désirez-vous? 

\speak  Vous voulez le savoir? 

\speak  Certainement. 

\speak  Eh bien, je veux l'ordre dont vous êtes porteur, attendu que je n'en ai pas, moi, et qu'il m'en faut un. 

\speak  Vous plaisantez, je présume. 

\speak  Je ne plaisante jamais. 

\speak  Laissez-moi passer! 

\speak  Vous ne passerez pas. 

\speak  Mon brave jeune homme, je vais vous casser la tête. Holà, Lubin! mes pistolets. 

\speak  Planchet, dit d'Artagnan, charge-toi du valet, je me charge du maître.» 

Planchet, enhardi par le premier exploit, sauta sur Lubin, et comme il était fort et vigoureux, il le renversa les reins contre terre et lui mit le genou sur la poitrine. 

«Faites votre affaire, monsieur, dit Planchet; moi, j'ai fait la mienne.» 

Voyant cela, le gentilhomme tira son épée et fondit sur d'Artagnan; mais il avait affaire à forte partie. 

En trois secondes d'Artagnan lui fournit trois coups d'épée en disant à chaque coup: 

«Un pour Athos, un pour Porthos, un pour Aramis.» 

Au troisième coup, le gentilhomme tomba comme une masse. 

D'Artagnan le crut mort, ou tout au moins évanoui, et s'approcha pour lui prendre l'ordre; mais au moment où il étendait le bras afin de le fouiller, le blessé qui n'avait pas lâché son épée, lui porta un coup de pointe dans la poitrine en disant: 

«Un pour vous. 

\speak  Et un pour moi! au dernier les bons!» s'écria d'Artagnan furieux, en le clouant par terre d'un quatrième coup d'épée dans le ventre. 

Cette fois, le gentilhomme ferma les yeux et s'évanouit. 

D'Artagnan fouilla dans la poche où il l'avait vu remettre l'ordre de passage, et le prit. Il était au nom du comte de Wardes. 

Puis, jetant un dernier coup d'œil sur le beau jeune homme, qui avait vingt-cinq ans à peine et qu'il laissait là, gisant, privé de sentiment et peut-être mort, il poussa un soupir sur cette étrange destinée qui porte les hommes à se détruire les uns les autres pour les intérêts de gens qui leur sont étrangers et qui souvent ne savent pas même qu'ils existent. 

Mais il fut bientôt tiré de ces réflexions par Lubin, qui poussait des hurlements et criait de toutes ses forces au secours. 

Planchet lui appliqua la main sur la gorge et serra de toutes ses forces. 

«Monsieur, dit-il, tant que je le tiendrai ainsi, il ne criera pas, j'en suis bien sûr; mais aussitôt que je le lâcherai, il va se remettre à crier. Je le reconnais pour un Normand et les Normands sont entêtés.» 

En effet, tout comprimé qu'il était, Lubin essayait encore de filer des sons. 

«Attends!» dit d'Artagnan. 

Et prenant son mouchoir, il le bâillonna. 

«Maintenant, dit Planchet, lions-le à un arbre.» 

La chose fut faite en conscience, puis on tira le comte de Wardes près de son domestique; et comme la nuit commençait à tomber et que le garrotté et le blessé étaient tous deux à quelques pas dans le bois, il était évident qu'ils devaient rester jusqu'au lendemain. 

«Et maintenant, dit d'Artagnan, chez le gouverneur! 

\speak  Mais vous êtes blessé, ce me semble? dit Planchet. 

\speak  Ce n'est rien, occupons-nous du plus pressé; puis nous reviendrons à ma blessure, qui, au reste, ne me paraît pas très dangereuse.» 

Et tous deux s'acheminèrent à grands pas vers la campagne du digne fonctionnaire. 

On annonça M. le comte de Wardes. 

D'Artagnan fut introduit. 

«Vous avez un ordre signé du cardinal? dit le gouverneur. 

\speak  Oui, monsieur, répondit d'Artagnan, le voici. 

\speak  Ah! ah! il est en règle et bien recommandé, dit le gouverneur. 

\speak  C'est tout simple, répondit d'Artagnan, je suis de ses plus fidèles. 

\speak  Il paraît que Son Éminence veut empêcher quelqu'un de parvenir en Angleterre. 

\speak  Oui, un certain d'Artagnan, un gentilhomme béarnais qui est parti de Paris avec trois de ses amis dans l'intention de gagner Londres. 

\speak  Le connaissez-vous personnellement? demanda le gouverneur. 

\speak  Qui cela? 

\speak  Ce d'Artagnan? 

\speak  À merveille. 

\speak  Donnez-moi son signalement alors. 

\speak  Rien de plus facile.» 

Et d'Artagnan donna trait pour trait le signalement du comte de Wardes. 

«Est-il accompagné? demanda le gouverneur. 

\speak  Oui, d'un valet nommé Lubin. 

\speak  On veillera sur eux, et si on leur met la main dessus, Son Éminence peut être tranquille, ils seront reconduits à Paris sous bonne escorte. 

\speak  Et ce faisant, monsieur le gouverneur, dit d'Artagnan, vous aurez bien mérité du cardinal. 

\speak  Vous le reverrez à votre retour, monsieur le comte? 

\speak  Sans aucun doute. 

\speak  Dites-lui, je vous prie, que je suis bien son serviteur. 

\speak  Je n'y manquerai pas.» 

Et joyeux de cette assurance, le gouverneur visa le laissez-passer et le remit à d'Artagnan. 

D'Artagnan ne perdit pas son temps en compliments inutiles, il salua le gouverneur, le remercia et partit. 

Une fois dehors, lui et Planchet prirent leur course, et faisant un long détour, ils évitèrent le bois et rentrèrent par une autre porte. 

Le bâtiment était toujours prêt à partir, le patron attendait sur le port. 

«Eh bien? dit-il en apercevant d'Artagnan. 

\speak  Voici ma passe visée, dit celui-ci. 

\speak  Et cet autre gentilhomme? 

\speak  Il ne partira pas aujourd'hui, dit d'Artagnan, mais soyez tranquille, je paierai le passage pour nous deux. 

\speak  En ce cas, partons, dit le patron. 

\speak  Partons!» répéta d'Artagnan. 

Et il sauta avec Planchet dans le canot; cinq minutes après, ils étaient à bord. 

Il était temps: à une demi-lieue en mer, d'Artagnan vit briller une lumière et entendit une détonation. 

C'était le coup de canon qui annonçait la fermeture du port. 

Il était temps de s'occuper de sa blessure; heureusement, comme l'avait pensé d'Artagnan, elle n'était pas des plus dangereuses: la pointe de l'épée avait rencontré une côte et avait glissé le long de l'os; de plus, la chemise s'était collée aussitôt à la plaie, et à peine avait-elle répandu quelques gouttes de sang. 

D'Artagnan était brisé de fatigue: on lui étendit un matelas sur le pont, il se jeta dessus et s'endormit. 

Le lendemain, au point du jour, il se trouva à trois ou quatre lieues seulement des côtes d'Angleterre; la brise avait été faible toute la nuit, et l'on avait peu marché. 

À dix heures, le bâtiment jetait l'ancre dans le port de Douvres. 

À dix heures et demie, d'Artagnan mettait le pied sur la terre d'Angleterre, en s'écriant: 

«Enfin, m'y voilà!» 

Mais ce n'était pas tout: il fallait gagner Londres. En Angleterre, la poste était assez bien servie. D'Artagnan et Planchet prirent chacun un bidet, un postillon courut devant eux; en quatre heures ils arrivèrent aux portes de la capitale. 

D'Artagnan ne connaissait pas Londres, d'Artagnan ne savait pas un mot d'anglais; mais il écrivit le nom de Buckingham sur un papier, et chacun lui indiqua l'hôtel du duc. 

Le duc était à la chasse à Windsor, avec le roi. 

D'Artagnan demanda le valet de chambre de confiance du duc, qui, l'ayant accompagné dans tous ses voyages, parlait parfaitement français; il lui dit qu'il arrivait de Paris pour affaire de vie et de mort, et qu'il fallait qu'il parlât à son maître à l'instant même. 

La confiance avec laquelle parlait d'Artagnan convainquit Patrice; c'était le nom de ce ministre du ministre. Il fit seller deux chevaux et se chargea de conduire le jeune garde. Quant à Planchet, on l'avait descendu de sa monture, raide comme un jonc: le pauvre garçon était au bout de ses forces; d'Artagnan semblait de fer. 

On arriva au château; là on se renseigna: le roi et Buckingham chassaient à l'oiseau dans des marais situés à deux ou trois lieues de là. 

En vingt minutes on fut au lieu indiqué. Bientôt Patrice entendit la voix de son maître, qui appelait son faucon. 

«Qui faut-il que j'annonce à Milord duc? demanda Patrice. 

\speak  Le jeune homme qui, un soir, lui a cherché une querelle sur le Pont-Neuf, en face de la Samaritaine. 

\speak  Singulière recommandation! 

\speak  Vous verrez qu'elle en vaut bien une autre.» 

Patrice mit son cheval au galop, atteignit le duc et lui annonça dans les termes que nous avons dits qu'un messager l'attendait. 

Buckingham reconnut d'Artagnan à l'instant même, et se doutant que quelque chose se passait en France dont on lui faisait parvenir la nouvelle, il ne prit que le temps de demander où était celui qui la lui apportait; et ayant reconnu de loin l'uniforme des gardes, il mit son cheval au galop et vint droit à d'Artagnan. Patrice, par discrétion, se tint à l'écart. 

«Il n'est point arrivé malheur à la reine? s'écria Buckingham, répandant toute sa pensée et tout son amour dans cette interrogation. 

\speak  Je ne crois pas; cependant je crois qu'elle court quelque grand péril dont Votre Grâce seule peut la tirer. 

\speak  Moi? s'écria Buckingham. Eh quoi! je serais assez heureux pour lui être bon à quelque chose! Parlez! parlez! 

\speak  Prenez cette lettre, dit d'Artagnan. 

\speak  Cette lettre! de qui vient cette lettre? 

\speak  De Sa Majesté, à ce que je pense. 

\speak  De Sa Majesté!» dit Buckingham, pâlissant si fort que d'Artagnan crut qu'il allait se trouver mal. 

Et il brisa le cachet. 

«Quelle est cette déchirure? dit-il en montrant à d'Artagnan un endroit où elle était percée à jour. 

\speak  Ah! ah! dit d'Artagnan, je n'avais pas vu cela; c'est l'épée du comte de Wardes qui aura fait ce beau coup en me trouant la poitrine. 

\speak  Vous êtes blessé? demanda Buckingham en rompant le cachet. 

\speak  Oh! rien! dit d'Artagnan, une égratignure. 

\speak  Juste Ciel! qu'ai-je lu! s'écria le duc. Patrice, reste ici, ou plutôt rejoins le roi partout où il sera, et dis à Sa Majesté que je la supplie bien humblement de m'excuser, mais qu'une affaire de la plus haute importance me rappelle à Londres. Venez, monsieur, venez.» 

Et tous deux reprirent au galop le chemin de la capitale.
%!TeX root=../musketeersfr.tex 

\chapter{La Comtesse De Winter}

\lettrine{T}{out} le long de la route, le duc se fit mettre au courant par d'Artagnan non pas de tout ce qui s'était passé, mais de ce que d'Artagnan savait. En rapprochant ce qu'il avait entendu sortir de la bouche du jeune homme de ses souvenirs à lui, il put donc se faire une idée assez exacte d'une position de la gravité de laquelle, au reste, la lettre de la reine, si courte et si peu explicite qu'elle fût, lui donnait la mesure. Mais ce qui l'étonnait surtout, c'est que le cardinal, intéressé comme il l'était à ce que le jeune homme ne mît pas le pied en Angleterre, ne fût point parvenu à l'arrêter en route. Ce fut alors, et sur la manifestation de cet étonnement, que d'Artagnan lui raconta les précautions prises, et comment, grâce au dévouement de ses trois amis qu'il avait éparpillés tout sanglants sur la route, il était arrivé à en être quitte pour le coup d'épée qui avait traversé le billet de la reine, et qu'il avait rendu à M. de Wardes en si terrible monnaie. Tout en écoutant ce récit, fait avec la plus grande simplicité, le duc regardait de temps en temps le jeune homme d'un air étonné, comme s'il n'eût pas pu comprendre que tant de prudence, de courage et de dévouement s'alliât avec un visage qui n'indiquait pas encore vingt ans. 

Les chevaux allaient comme le vent, et en quelques minutes ils furent aux portes de Londres. D'Artagnan avait cru qu'en arrivant dans la ville le duc allait ralentir l'allure du sien, mais il n'en fut pas ainsi: il continua sa route à fond de train, s'inquiétant peu de renverser ceux qui étaient sur son chemin. En effet, en traversant la Cité deux ou trois accidents de ce genre arrivèrent; mais Buckingham ne détourna pas même la tête pour regarder ce qu'étaient devenus ceux qu'il avait culbutés. D'Artagnan le suivait au milieu de cris qui ressemblaient fort à des malédictions. 

En entrant dans la cour de l'hôtel, Buckingham sauta à bas de son cheval, et, sans s'inquiéter de ce qu'il deviendrait, il lui jeta la bride sur le cou et s'élança vers le perron. D'Artagnan en fit autant, avec un peu plus d'inquiétude, cependant, pour ces nobles animaux dont il avait pu apprécier le mérite; mais il eut la consolation de voir que trois ou quatre valets s'étaient déjà élancés des cuisines et des écuries, et s'emparaient aussitôt de leurs montures. 

Le duc marchait si rapidement, que d'Artagnan avait peine à le suivre. Il traversa successivement plusieurs salons d'une élégance dont les plus grands seigneurs de France n'avaient pas même l'idée, et il parvint enfin dans une chambre à coucher qui était à la fois un miracle de goût et de richesse. Dans l'alcôve de cette chambre était une porte, prise dans la tapisserie, que le duc ouvrit avec une petite clef d'or qu'il portait suspendue à son cou par une chaîne du même métal. Par discrétion, d'Artagnan était resté en arrière; mais au moment où Buckingham franchissait le seuil de cette porte, il se retourna, et voyant l'hésitation du jeune homme: 

«Venez, lui dit-il, et si vous avez le bonheur d'être admis en la présence de Sa Majesté, dites-lui ce que vous avez vu.» 

Encouragé par cette invitation, d'Artagnan suivit le duc, qui referma la porte derrière lui. 

Tous deux se trouvèrent alors dans une petite chapelle toute tapissée de soie de Perse et brochée d'or, ardemment éclairée par un grand nombre de bougies. Au-dessus d'une espèce d'autel, et au-dessous d'un dais de velours bleu surmonté de plumes blanches et rouges, était un portrait de grandeur naturelle représentant Anne d'Autriche, si parfaitement ressemblant, que d'Artagnan poussa un cri de surprise: on eût cru que la reine allait parler. 

Sur l'autel, et au-dessous du portrait, était le coffret qui renfermait les ferrets de diamants. 

Le duc s'approcha de l'autel, s'agenouilla comme eût pu faire un prêtre devant le Christ; puis il ouvrit le coffret. 

«Tenez, lui dit-il en tirant du coffre un gros noeud de ruban bleu tout étincelant de diamants; tenez, voici ces précieux ferrets avec lesquels j'avais fait le serment d'être enterré. La reine me les avait donnés, la reine me les reprend: sa volonté, comme celle de Dieu, soit faite en toutes choses.» 

Puis il se mit à baiser les uns après les autres ces ferrets dont il fallait se séparer. Tout à coup, il poussa un cri terrible. 

«Qu'y a-t-il? demanda d'Artagnan avec inquiétude, et que vous arrive-t-il, Milord? 

\speak  Il y a que tout est perdu, s'écria Buckingham en devenant pâle comme un trépassé; deux de ces ferrets manquent, il n'y en a plus que dix. 

\speak  Milord les a-t-il perdus, ou croit-il qu'on les lui ait volés? 

\speak  On me les a volés, reprit le duc, et c'est le cardinal qui a fait le coup. Tenez, voyez, les rubans qui les soutenaient ont été coupés avec des ciseaux. 

\speak  Si Milord pouvait se douter qui a commis le vol\dots Peut-être la personne les a-t-elle encore entre les mains. 

\speak  Attendez, attendez! s'écria le duc. La seule fois que j'ai mis ces ferrets, c'était au bal du roi, il y a huit jours, à Windsor. La comtesse de Winter, avec laquelle j'étais brouillé, s'est rapprochée de moi à ce bal. Ce raccommodement, c'était une vengeance de femme jalouse. Depuis ce jour, je ne l'ai pas revue. Cette femme est un agent du cardinal. 

\speak  Mais il en a donc dans le monde entier! s'écria d'Artagnan. 

\speak  Oh! oui, oui, dit Buckingham en serrant les dents de colère; oui, c'est un terrible lutteur. Mais cependant, quand doit avoir lieu ce bal? 

\speak  Lundi prochain. 

\speak  Lundi prochain! cinq jours encore, c'est plus de temps qu'il ne nous en faut. Patrice! s'écria le duc en ouvrant la porte de la chapelle, Patrice!» 

Son valet de chambre de confiance parut. 

«Mon joaillier et mon secrétaire!» 

Le valet de chambre sortit avec une promptitude et un mutisme qui prouvaient l'habitude qu'il avait contractée d'obéir aveuglément et sans réplique. 

Mais, quoique ce fût le joaillier qui eût été appelé le premier, ce fut le secrétaire qui parut d'abord. C'était tout simple, il habitait l'hôtel. Il trouva Buckingham assis devant une table dans sa chambre à coucher, et écrivant quelques ordres de sa propre main. 

«Monsieur Jackson, lui dit-il, vous allez vous rendre de ce pas chez le lord-chancelier, et lui dire que je le charge de l'exécution de ces ordres. Je désire qu'ils soient promulgués à l'instant même. 

\speak  Mais, Monseigneur, si le lord-chancelier m'interroge sur les motifs qui ont pu porter Votre Grâce à une mesure si extraordinaire, que répondrai-je? 

\speak  Que tel a été mon bon plaisir, et que je n'ai de compte à rendre à personne de ma volonté. 

\speak  Sera-ce la réponse qu'il devra transmettre à Sa Majesté, reprit en souriant le secrétaire, si par hasard Sa Majesté avait la curiosité de savoir pourquoi aucun vaisseau ne peut sortir des ports de la Grande-Bretagne? 

\speak  Vous avez raison, monsieur, répondit Buckingham; il dirait en ce cas au roi que j'ai décidé la guerre, et que cette mesure est mon premier acte d'hostilité contre la France.» 

Le secrétaire s'inclina et sortit. 

«Nous voilà tranquilles de ce côté, dit Buckingham en se retournant vers d'Artagnan. Si les ferrets ne sont point déjà partis pour la France, ils n'y arriveront qu'après vous. 

\speak  Comment cela? 

\speak  Je viens de mettre un embargo sur tous les bâtiments qui se trouvent à cette heure dans les ports de Sa Majesté, et, à moins de permission particulière, pas un seul n'osera lever l'ancre.» 

D'Artagnan regarda avec stupéfaction cet homme qui mettait le pouvoir illimité dont il était revêtu par la confiance d'un roi au service de ses amours. Buckingham vit, à l'expression du visage du jeune homme, ce qui se passait dans sa pensée, et il sourit. 

«Oui, dit-il, oui, c'est qu'Anne d'Autriche est ma véritable reine; sur un mot d'elle, je trahirais mon pays, je trahirais mon roi, je trahirais mon Dieu. Elle m'a demandé de ne point envoyer aux protestants de La Rochelle le secours que je leur avais promis, et je l'ai fait. Je manquais à ma parole, mais qu'importe! j'obéissais à son désir; n'ai-je point été grandement payé de mon obéissance, dites? car c'est à cette obéissance que je dois son portrait.» 

D'Artagnan admira à quels fils fragiles et inconnus sont parfois suspendues les destinées d'un peuple et la vie des hommes. 

Il en était au plus profond de ses réflexions, lorsque l'orfèvre entra: c'était un Irlandais des plus habiles dans son art, et qui avouait lui-même qu'il gagnait cent mille livres par an avec le duc de Buckingham. 

«Monsieur O'Reilly, lui dit le duc en le conduisant dans la chapelle, voyez ces ferrets de diamants, et dites-moi ce qu'ils valent la pièce.» 

L'orfèvre jeta un seul coup d'œil sur la façon élégante dont ils étaient montés, calcula l'un dans l'autre la valeur des diamants, et sans hésitation aucune: 

«Quinze cents pistoles la pièce, Milord, répondit-il. 

\speak  Combien faudrait-il de jours pour faire deux ferrets comme ceux-là? Vous voyez qu'il en manque deux. 

\speak  Huit jours, Milord. 

\speak  Je les paierai trois mille pistoles la pièce, il me les faut après-demain. 

\speak  Milord les aura. 

\speak  Vous êtes un homme précieux, monsieur O'Reilly, mais ce n'est pas le tout: ces ferrets ne peuvent être confiés à personne, il faut qu'ils soient faits dans ce palais. 

\speak  Impossible, Milord, il n'y a que moi qui puisse les exécuter pour qu'on ne voie pas la différence entre les nouveaux et les anciens. 

\speak  Aussi, mon cher monsieur O'Reilly, vous êtes mon prisonnier, et vous voudriez sortir à cette heure de mon palais que vous ne le pourriez pas; prenez-en donc votre parti. Nommez-moi ceux de vos garçons dont vous aurez besoin, et désignez-moi les ustensiles qu'ils doivent apporter.» 

L'orfèvre connaissait le duc, il savait que toute observation était inutile, il en prit donc à l'instant même son parti. 

«Il me sera permis de prévenir ma femme? demanda-t-il. 

\speak  Oh! il vous sera même permis de la voir, mon cher monsieur O'Reilly: votre captivité sera douce, soyez tranquille; et comme tout dérangement vaut un dédommagement, voici, en dehors du prix des deux ferrets, un bon de mille pistoles pour vous faire oublier l'ennui que je vous cause.» 

D'Artagnan ne revenait pas de la surprise que lui causait ce ministre, qui remuait à pleines mains les hommes et les millions. 

Quant à l'orfèvre, il écrivit à sa femme en lui envoyant le bon de mille pistoles, et en la chargeant de lui retourner en échange son plus habile apprenti, un assortiment de diamants dont il lui donnait le poids et le titre, et une liste des outils qui lui étaient nécessaires. 

Buckingham conduisit l'orfèvre dans la chambre qui lui était destinée, et qui, au bout d'une demi-heure, fut transformée en atelier. Puis il mit une sentinelle à chaque porte, avec défense de laisser entrer qui que ce fût, à l'exception de son valet de chambre Patrice. Il est inutile d'ajouter qu'il était absolument défendu à l'orfèvre O'Reilly et à son aide de sortir sous quelque prétexte que ce fût. Ce point réglé, le duc revint à d'Artagnan. 

«Maintenant, mon jeune ami, dit-il, l'Angleterre est à nous deux; que voulez-vous, que désirez-vous? 

\speak  Un lit, répondit d'Artagnan; c'est, pour le moment, je l'avoue, la chose dont j'ai le plus besoin.» 

Buckingham donna à d'Artagnan une chambre qui touchait à la sienne. Il voulait garder le jeune homme sous sa main, non pas qu'il se défiât de lui, mais pour avoir quelqu'un à qui parler constamment de la reine. 

Une heure après fut promulguée dans Londres l'ordonnance de ne laisser sortir des ports aucun bâtiment chargé pour la France, pas même le paquebot des lettres. Aux yeux de tous, c'était une déclaration de guerre entre les deux royaumes. 

Le surlendemain, à onze heures, les deux ferrets en diamants étaient achevés, mais si exactement imités, mais si parfaitement pareils, que Buckingham ne put reconnaître les nouveaux des anciens, et que les plus exercés en pareille matière y auraient été trompés comme lui. 

Aussitôt il fit appeler d'Artagnan. 

«Tenez, lui dit-il, voici les ferrets de diamants que vous êtes venu chercher, et soyez mon témoin que tout ce que la puissance humaine pouvait faire, je l'ai fait. 

\speak  Soyez tranquille, Milord: je dirai ce que j'ai vu; mais Votre Grâce me remet les ferrets sans la boîte? 

\speak  La boîte vous embarrasserait. D'ailleurs la boîte m'est d'autant plus précieuse, qu'elle me reste seule. Vous direz que je la garde. 

\speak  Je ferai votre commission mot à mot, Milord. 

\speak  Et maintenant, reprit Buckingham en regardant fixement le jeune homme, comment m'acquitterai-je jamais envers vous?» 

D'Artagnan rougit jusqu'au blanc des yeux. Il vit que le duc cherchait un moyen de lui faire accepter quelque chose, et cette idée que le sang de ses compagnons et le sien lui allait être payé par de l'or anglais lui répugnait étrangement. 

«Entendons-nous, Milord, répondit d'Artagnan, et pesons bien les faits d'avance, afin qu'il n'y ait point de méprise. Je suis au service du roi et de la reine de France, et fais partie de la compagnie des gardes de M. des Essarts, lequel, ainsi que son beau-frère M. de Tréville, est tout particulièrement attaché à Leurs Majestés. J'ai donc tout fait pour la reine et rien pour Votre Grâce. Il y a plus, c'est que peut-être n'eussé-je rien fait de tout cela, s'il ne se fût agi d'être agréable à quelqu'un qui est ma dame à moi, comme la reine est la vôtre. 

\speak  Oui, dit le duc en souriant, et je crois même connaître cette autre personne, c'est\dots 

\speak  Milord, je ne l'ai point nommée, interrompit vivement le jeune homme. 

\speak  C'est juste, dit le duc; c'est donc à cette personne que je dois être reconnaissant de votre dévouement. 

\speak  Vous l'avez dit, Milord, car justement à cette heure qu'il est question de guerre, je vous avoue que je ne vois dans votre Grâce qu'un Anglais, et par conséquent qu'un ennemi que je serais encore plus enchanté de rencontrer sur le champ de bataille que dans le parc de Windsor ou dans les corridors du Louvre; ce qui, au reste, ne m'empêchera pas d'exécuter de point en point ma mission et de me faire tuer, si besoin est, pour l'accomplir; mais, je le répète à Votre Grâce, sans qu'elle ait personnellement pour cela plus à me remercier de ce que je fais pour moi dans cette seconde entrevue, que de ce que j'ai déjà fait pour elle dans la première. 

\speak  Nous disons, nous: “Fier comme un Écossais”, murmura Buckingham. 

\speak  Et nous disons, nous: “Fier comme un Gascon”, répondit d'Artagnan. Les Gascons sont les Écossais de la France.» 

D'Artagnan salua le duc et s'apprêta à partir. 

«Eh bien, vous vous en allez comme cela? Par où? Comment? 

\speak  C'est vrai. 

\speak  Dieu me damne! les Français ne doutent de rien! 

\speak  J'avais oublié que l'Angleterre était une île, et que vous en étiez le roi. 

\speak  Allez au port, demandez le brick \textit{le Sund}, remettez cette lettre au capitaine; il vous conduira à un petit port où certes on ne vous attend pas, et où n'abordent ordinairement que des bâtiments pêcheurs. 

\speak  Ce port s'appelle? 

\speak  Saint-Valery; mais, attendez donc: arrivé là, vous entrerez dans une mauvaise auberge sans nom et sans enseigne, un véritable bouge à matelots; il n'y a pas à vous tromper, il n'y en a qu'une. 

\speak  Après? 

\speak  Vous demanderez l'hôte, et vous lui direz: \textit{Forward}. 

\speak  Ce qui veut dire? 

\speak  En avant: c'est le mot d'ordre. Il vous donnera un cheval tout sellé et vous indiquera le chemin que vous devez suivre; vous trouverez ainsi quatre relais sur votre route. Si vous voulez, à chacun d'eux, donner votre adresse à Paris, les quatre chevaux vous y suivront; vous en connaissez déjà deux, et vous m'avez paru les apprécier en amateur: ce sont ceux que nous montions; rapportez-vous en à moi, les autres ne leur sont point inférieurs. Ces quatre chevaux sont équipés pour la campagne. Si fier que vous soyez, vous ne refuserez pas d'en accepter un et de faire accepter les trois autres à vos compagnons: c'est pour nous faire la guerre, d'ailleurs. La fin excuse les moyens, comme vous dites, vous autres Français, n'est-ce pas? 

\speak  Oui, Milord, j'accepte, dit d'Artagnan; et s'il plaît à Dieu, nous ferons bon usage de vos présents. 

\speak  Maintenant, votre main, jeune homme; peut-être nous rencontrerons-nous bientôt sur le champ de bataille; mais, en attendant, nous nous quitterons bons amis, je l'espère. 

\speak  Oui, Milord, mais avec l'espérance de devenir ennemis bientôt. 

\speak  Soyez tranquille, je vous le promets. 

\speak  Je compte sur votre parole, Milord.» 

D'Artagnan salua le duc et s'avança vivement vers le port. 

En face la Tour de Londres, il trouva le bâtiment désigné, remit sa lettre au capitaine, qui la fit viser par le gouverneur du port, et appareilla aussitôt. 

Cinquante bâtiments étaient en partance et attendaient. 

En passant bord à bord de l'un d'eux, d'Artagnan crut reconnaître la femme de Meung, la même que le gentilhomme inconnu avait appelée «Milady», et que lui, d'Artagnan, avait trouvée si belle; mais grâce au courant du fleuve et au bon vent qui soufflait, son navire allait si vite qu'au bout d'un instant on fut hors de vue. 

Le lendemain, vers neuf heures du matin, on aborda à Saint-Valery. 

D'Artagnan se dirigea à l'instant même vers l'auberge indiquée, et la reconnut aux cris qui s'en échappaient: on parlait de guerre entre l'Angleterre et la France comme de chose prochaine et indubitable, et les matelots joyeux faisaient bombance. 

D'Artagnan fendit la foule, s'avança vers l'hôte, et prononça le mot \textit{Forward}. À l'instant même, l'hôte lui fit signe de le suivre, sortit avec lui par une porte qui donnait dans la cour, le conduisit à l'écurie où l'attendait un cheval tout sellé, et lui demanda s'il avait besoin de quelque autre chose. 

«J'ai besoin de connaître la route que je dois suivre, dit d'Artagnan. 

\speak  Allez d'ici à Blangy, et de Blangy à Neufchâtel. À Neufchâtel, entrez à l'auberge de la \textit{Herse d'Or}, donnez le mot d'ordre à l'hôtelier, et vous trouverez comme ici un cheval tout sellé. 

\speak  Dois-je quelque chose? demanda d'Artagnan. 

\speak  Tout est payé, dit l'hôte, et largement. Allez donc, et que Dieu vous conduise! 

\speak  Amen!» répondit le jeune homme en partant au galop. 

Quatre heures après, il était à Neufchâtel. 

Il suivit strictement les instructions reçues; à Neufchâtel, comme à Saint-Valery, il trouva une monture toute sellée et qui l'attendait; il voulut transporter les pistolets de la selle qu'il venait de quitter à la selle qu'il allait prendre: les fontes étaient garnies de pistolets pareils. 

«Votre adresse à Paris? 

\speak  Hôtel des Gardes, compagnie des Essarts. 

\speak  Bien, répondit celui-ci. 

\speak  Quelle route faut-il prendre? demanda à son tour d'Artagnan. 

\speak  Celle de Rouen; mais vous laisserez la ville à votre droite. Au petit village d'Écouis, vous vous arrêterez, il n'y a qu'une auberge, l'\textit{Écu de France}. Ne la jugez pas d'après son apparence; elle aura dans ses écuries un cheval qui vaudra celui-ci. 

\speak  Même mot d'ordre? 

\speak  Exactement. 

\speak  Adieu, maître! 

\speak  Bon voyage, gentilhomme! avez-vous besoin de quelque chose?» 

D'Artagnan fit signe de la tête que non, et repartit à fond de train. À Écouis, la même scène se répéta: il trouva un hôte aussi prévenant, un cheval frais et reposé; il laissa son adresse comme il l'avait fait, et repartit du même train pour Pontoise. À Pontoise, il changea une dernière fois de monture, et à neuf heures il entrait au grand galop dans la cour de l'hôtel de M. de Tréville. 

Il avait fait près de soixante lieues en douze heures. 

M. de Tréville le reçut comme s'il l'avait vu le matin même; seulement, en lui serrant la main un peu plus vivement que de coutume, il lui annonça que la compagnie de M. des Essarts était de garde au Louvre et qu'il pouvait se rendre à son poste.
\include{chapters/22.tex}
%!TeX root=../musketeersfr.tex 

\chapter{Le Rendez-Vous} 
	
\lettrine{D}{'Artagnan} revint chez lui tout courant, et quoiqu'il fût plus de trois heures du matin, et qu'il eût les plus méchants quartiers de Paris à traverser, il ne fit aucune mauvaise rencontre. On sait qu'il y a un dieu pour les ivrognes et les amoureux. 

Il trouva la porte de son allée entrouverte, monta son escalier, et frappa doucement et d'une façon convenue entre lui et son laquais. Planchet, qu'il avait renvoyé deux heures auparavant de l'Hôtel de Ville en lui recommandant de l'attendre, vint lui ouvrir la porte. 

«Quelqu'un a-t-il apporté une lettre pour moi? demanda vivement d'Artagnan. 

\speak  Personne n'a apporté de lettre, monsieur, répondit Planchet; mais il y en a une qui est venue toute seule. 

\speak  Que veux-tu dire, imbécile? 

\speak  Je veux dire qu'en rentrant, quoique j'eusse la clef de votre appartement dans ma poche et que cette clef ne m'eût point quitté, j'ai trouvé une lettre sur le tapis vert de la table, dans votre chambre à coucher. 

\speak  Et où est cette lettre? 

\speak  Je l'ai laissée où elle était, monsieur. Il n'est pas naturel que les lettres entrent ainsi chez les gens. Si la fenêtre était ouverte encore, ou seulement entrebâillée je ne dis pas; mais non, tout était hermétiquement fermé. Monsieur, prenez garde, car il y a très certainement quelque magie là-dessous.» 

Pendant ce temps, le jeune homme s'élançait dans la chambre et ouvrait la lettre; elle était de Mme Bonacieux, et conçue en ces termes: «On a de vifs remerciements à vous faire et à vous transmettre. Trouvez-vous ce soir vers dix heures à Saint-Cloud, en face du pavillon qui s'élève à l'angle de la maison de M. d'Estrées. --- «C. B.» 

En lisant cette lettre, d'Artagnan sentait son cœur se dilater et s'étreindre de ce doux spasme qui torture et caresse le cœur des amants. 

C'était le premier billet qu'il recevait, c'était le premier rendez-vous qui lui était accordé. Son cœur, gonflé par l'ivresse de la joie, se sentait prêt à défaillir sur le seuil de ce paradis terrestre qu'on appelait l'amour. 

«Eh bien! monsieur, dit Planchet, qui avait vu son maître rougir et pâlir successivement; eh bien! n'est-ce pas que j'avais deviné juste et que c'est quelque méchante affaire? 

\speak  Tu te trompes, Planchet, répondit d'Artagnan, et la preuve, c'est que voici un écu pour que tu boives à ma santé. 

\speak  Je remercie monsieur de l'écu qu'il me donne, et je lui promets de suivre exactement ses instructions; mais il n'en est pas moins vrai que les lettres qui entrent ainsi dans les maisons fermées\dots 

\speak  Tombent du ciel, mon ami, tombent du ciel. 

\speak  Alors, monsieur est content? demanda Planchet. 

\speak  Mon cher Planchet, je suis le plus heureux des hommes! 

\speak  Et je puis profiter du bonheur de monsieur pour aller me coucher? 

\speak  Oui, va. 

\speak  Que toutes les bénédictions du Ciel tombent sur monsieur, mais il n'en est pas moins vrai que cette lettre\dots» 

Et Planchet se retira en secouant la tête avec un air de doute que n'était point parvenu à effacer entièrement la libéralité de d'Artagnan. 

Resté seul, d'Artagnan lut et relut son billet, puis il baisa et rebaisa vingt fois ces lignes tracées par la main de sa belle maîtresse. Enfin il se coucha, s'endormit et fit des rêves d'or. 

À sept heures du matin, il se leva et appela Planchet, qui, au second appel, ouvrit la porte, le visage encore mal nettoyé des inquiétudes de la veille. 

«Planchet, lui dit d'Artagnan, je sors pour toute la journée peut-être; tu es donc libre jusqu'à sept heures du soir; mais, à sept heures du soir, tiens-toi prêt avec deux chevaux. 

\speak  Allons! dit Planchet, il paraît que nous allons encore nous faire traverser la peau en plusieurs endroits. 

\speak  Tu prendras ton mousqueton et tes pistolets. 

\speak  Eh bien, que disais-je? s'écria Planchet. Là, j'en étais sûr, maudite lettre! 

\speak  Mais rassure-toi donc, imbécile, il s'agit tout simplement d'une partie de plaisir. 

\speak  Oui! comme les voyages d'agrément de l'autre jour, où il pleuvait des balles et où il poussait des chausse-trapes. 

\speak  Au reste, si vous avez peur, monsieur Planchet, reprit d'Artagnan, j'irai sans vous; j'aime mieux voyager seul que d'avoir un compagnon qui tremble. 

\speak  Monsieur me fait injure, dit Planchet; il me semblait cependant qu'il m'avait vu à l'oeuvre. 

\speak  Oui, mais j'ai cru que tu avais usé tout ton courage d'une seule fois. 

\speak  Monsieur verra que dans l'occasion il m'en reste encore; seulement je prie monsieur de ne pas trop le prodiguer, s'il veut qu'il m'en reste longtemps. 

\speak  Crois-tu en avoir encore une certaine somme à dépenser ce soir? 

\speak  Je l'espère. 

\speak  Eh bien, je compte sur toi. 

\speak  À l'heure dite, je serai prêt; seulement je croyais que monsieur n'avait qu'un cheval à l'écurie des gardes. 

\speak  Peut-être n'y en a-t-il qu'un encore dans ce moment-ci, mais ce soir il y en aura quatre. 

\speak  Il paraît que notre voyage était un voyage de remonte? 

\speak  Justement», dit d'Artagnan. 

Et ayant fait à Planchet un dernier geste de recommandation, il sortit. 

M. Bonacieux était sur sa porte. L'intention de d'Artagnan était de passer outre, sans parler au digne mercier; mais celui-ci fit un salut si doux et si bénin, que force fut à son locataire non seulement de le lui rendre, mais encore de lier conversation avec lui. 

Comment d'ailleurs ne pas avoir un peu de condescendance pour un mari dont la femme vous a donné un rendez-vous le soir même à Saint-Cloud, en face du pavillon de M. d'Estrées! D'Artagnan s'approcha de l'air le plus aimable qu'il put prendre. 

La conversation tomba tout naturellement sur l'incarcération du pauvre homme. M. Bonacieux, qui ignorait que d'Artagnan eût entendu sa conversation avec l'inconnu de Meung, raconta à son jeune locataire les persécutions de ce monstre de M. de Laffemas, qu'il ne cessa de qualifier pendant tout son récit du titre de bourreau du cardinal et s'étendit longuement sur la Bastille, les verrous, les guichets, les soupiraux, les grilles et les instruments de torture. 

D'Artagnan l'écouta avec une complaisance exemplaire puis, lorsqu'il eut fini: 

«Et Mme Bonacieux, dit-il enfin, savez-vous qui l'avait enlevée? car je n'oublie pas que c'est à cette circonstance fâcheuse que je dois le bonheur d'avoir fait votre connaissance. 

\speak  Ah! dit M. Bonacieux, ils se sont bien gardés de me le dire, et ma femme de son côté m'a juré ses grands dieux qu'elle ne le savait pas. Mais vous-même, continua M. Bonacieux d'un ton de bonhomie parfaite, qu'êtes-vous devenu tous ces jours passés? je ne vous ai vu, ni vous ni vos amis, et ce n'est pas sur le pavé de Paris, je pense, que vous avez ramassé toute la poussière que Planchet époussetait hier sur vos bottes. 

\speak  Vous avez raison, mon cher monsieur Bonacieux, mes amis et moi nous avons fait un petit voyage. 

\speak  Loin d'ici? 

\speak  Oh! mon Dieu non, à une quarantaine de lieues seulement; nous avons été conduire M. Athos aux eaux de Forges, où mes amis sont restés. 

\speak  Et vous êtes revenu, vous, n'est-ce pas? reprit M. Bonacieux en donnant à sa physionomie son air le plus malin. Un beau garçon comme vous n'obtient pas de longs congés de sa maîtresse, et nous étions impatiemment attendu à Paris, n'est-ce pas? 

\speak  Ma foi, dit en riant le jeune homme, je vous l'avoue, d'autant mieux, mon cher monsieur Bonacieux, que je vois qu'on ne peut rien vous cacher. Oui, j'étais attendu, et bien impatiemment, je vous en réponds.» 

Un léger nuage passa sur le front de Bonacieux, mais si léger, que d'Artagnan ne s'en aperçut pas. 

«Et nous allons être récompensé de notre diligence? continua le mercier avec une légère altération dans la voix, altération que d'Artagnan ne remarqua pas plus qu'il n'avait fait du nuage momentané qui, un instant auparavant, avait assombri la figure du digne homme. 

\speak  Ah! faites donc le bon apôtre! dit en riant d'Artagnan. 

\speak  Non, ce que je vous en dis, reprit Bonacieux, c'est seulement pour savoir si nous rentrons tard. 

\speak  Pourquoi cette question, mon cher hôte? demanda d'Artagnan; est-ce que vous comptez m'attendre? 

\speak  Non, c'est que depuis mon arrestation et le vol qui a été commis chez moi, je m'effraie chaque fois que j'entends ouvrir une porte, et surtout la nuit. Dame, que voulez-vous! je ne suis point homme d'épée, moi! 

\speak  Eh bien, ne vous effrayez pas si je rentre à une heure, à deux ou trois heures du matin; si je ne rentre pas du tout, ne vous effrayez pas encore.» 

Cette fois, Bonacieux devint si pâle, que d'Artagnan ne put faire autrement que de s'en apercevoir, et lui demanda ce qu'il avait. 

«Rien, répondit Bonacieux, rien. Depuis mes malheurs seulement, je suis sujet à des faiblesses qui me prennent tout à coup, et je viens de me sentir passer un frisson. Ne faites pas attention à cela, vous qui n'avez à vous occuper que d'être heureux. 

\speak  Alors j'ai de l'occupation, car je le suis. 

\speak  Pas encore, attendez donc, vous avez dit: à ce soir. 

\speak  Eh bien, ce soir arrivera, Dieu merci! et peut-être l'attendez-vous avec autant d'impatience que moi. Peut-être, ce soir, Mme Bonacieux visitera-t-elle le domicile conjugal. 

\speak  Mme Bonacieux n'est pas libre ce soir, répondit gravement le mari; elle est retenue au Louvre par son service. 

\speak  Tant pis pour vous, mon cher hôte, tant pis; quand je suis heureux, moi, je voudrais que tout le monde le fût; mais il paraît que ce n'est pas possible.» 

Et le jeune homme s'éloigna en riant aux éclats de la plaisanterie que lui seul, pensait-il, pouvait comprendre. 

«Amusez-vous bien!» répondit Bonacieux d'un air sépulcral. 

Mais d'Artagnan était déjà trop loin pour l'entendre, et l'eut-il entendu, dans la disposition d'esprit où il était, il ne l'eût certes pas remarqué. 

Il se dirigea vers l'hôtel de M. de Tréville; sa visite de la veille avait été, on se le rappelle, très courte et très peu explicative. 

Il trouva M. de Tréville dans la joie de son âme. Le roi et la reine avaient été charmants pour lui au bal. Il est vrai que le cardinal avait été parfaitement maussade. 

À une heure du matin, il s'était retiré sous prétexte qu'il était indisposé. Quant à Leurs Majestés, elles n'étaient rentrées au Louvre qu'à six heures du matin. 

«Maintenant, dit M. de Tréville en baissant la voix et en interrogeant du regard tous les angles de l'appartement pour voir s'ils étaient bien seuls, maintenant parlons de vous, mon jeune ami, car il est évident que votre heureux retour est pour quelque chose dans la joie du roi, dans le triomphe de la reine et dans l'humiliation de Son Éminence. Il s'agit de bien vous tenir. 

\speak  Qu'ai-je à craindre, répondit d'Artagnan, tant que j'aurai le bonheur de jouir de la faveur de Leurs Majestés? 

\speak  Tout, croyez-moi. Le cardinal n'est point homme à oublier une mystification tant qu'il n'aura pas réglé ses comptes avec le mystificateur, et le mystificateur m'a bien l'air d'être certain Gascon de ma connaissance. 

\speak  Croyez-vous que le cardinal soit aussi avancé que vous et sache que c'est moi qui ai été à Londres? 

\speak  Diable! vous avez été à Londres. Est-ce de Londres que vous avez rapporté ce beau diamant qui brille à votre doigt? Prenez garde, mon cher d'Artagnan, ce n'est pas une bonne chose que le présent d'un ennemi; n'y a-t-il pas là-dessus certain vers latin\dots Attendez donc\dots 

\speak  Oui, sans doute, reprit d'Artagnan, qui n'avait jamais pu se fourrer la première règle du rudiment dans la tête, et qui, par ignorance, avait fait le désespoir de son précepteur; oui, sans doute, il doit y en avoir un. 

\speak  Il y en a un certainement, dit M. de Tréville, qui avait une teinte de lettres, et M. de Benserade me le citait l'autre jour\dots Attendez donc\dots Ah! m'y voici: \textit{Timeo Danaos et dona ferentes.} 

«Ce qui veut dire: “Défiez-vous de l'ennemi qui vous fait des présents.” 

\speak  Ce diamant ne vient pas d'un ennemi, monsieur, reprit d'Artagnan, il vient de la reine. 

\speak  De la reine! oh! oh! dit M. de Tréville. Effectivement, c'est un véritable bijou royal, qui vaut mille pistoles comme un denier. Par qui la reine vous a-t-elle fait remettre ce cadeau? 

\speak  Elle me l'a remis elle-même. 

\speak  Où cela? 

\speak  Dans le cabinet attenant à la chambre où elle a changé de toilette. 

\speak  Comment? 

\speak  En me donnant sa main à baiser. 

\speak  Vous avez baisé la main de la reine! s'écria M. de Tréville en regardant d'Artagnan. 

\speak  Sa Majesté m'a fait l'honneur de m'accorder cette grâce! 

\speak  Et cela en présence de témoins? Imprudente, trois fois imprudente! 

\speak  Non, monsieur, rassurez-vous, personne ne l'a vue», reprit d'Artagnan. Et il raconta à M. de Tréville comment les choses s'étaient passées. 

«Oh! les femmes, les femmes! s'écria le vieux soldat, je les reconnais bien à leur imagination romanesque; tout ce qui sent le mystérieux les charme; ainsi vous avez vu le bras, voilà tout; vous rencontreriez la reine, que vous ne la reconnaîtriez pas; elle vous rencontrerait, qu'elle ne saurait pas qui vous êtes. 

\speak  Non, mais grâce à ce diamant\dots, reprit le jeune homme. 

\speak  Écoutez, dit M. de Tréville, voulez-vous que je vous donne un conseil, un bon conseil, un conseil d'ami? 

\speak  Vous me ferez honneur, monsieur, dit d'Artagnan. 

\speak  Eh bien, allez chez le premier orfèvre venu et vendez-lui ce diamant pour le prix qu'il vous en donnera; si juif qu'il soit, vous en trouverez toujours bien huit cents pistoles. Les pistoles n'ont pas de nom, jeune homme, et cette bague en a un terrible, ce qui peut trahir celui qui la porte. 

\speak  Vendre cette bague! une bague qui vient de ma souveraine! jamais, dit d'Artagnan. 

\speak  Alors tournez-en le chaton en dedans, pauvre fou, car on sait qu'un cadet de Gascogne ne trouve pas de pareils bijoux dans l'écrin de sa mère. 

\speak  Vous croyez donc que j'ai quelque chose à craindre? demanda d'Artagnan. 

\speak  C'est-à-dire, jeune homme, que celui qui s'endort sur une mine dont la mèche est allumée doit se regarder comme en sûreté en comparaison de vous. 

\speak  Diable! dit d'Artagnan, que le ton d'assurance de M. de Tréville commençait à inquiéter: diable, que faut-il faire? 

\speak  Vous tenir sur vos gardes toujours et avant toute chose. Le cardinal a la mémoire tenace et la main longue; croyez-moi, il vous jouera quelque tour. 

\speak  Mais lequel? 

\speak  Eh! le sais-je, moi! est-ce qu'il n'a pas à son service toutes les ruses du démon? Le moins qui puisse vous arriver est qu'on vous arrête. 

\speak  Comment! on oserait arrêter un homme au service de Sa Majesté? 

\speak  Pardieu! on s'est bien gêné pour Athos! En tout cas, jeune homme, croyez-en un homme qui est depuis trente ans à la cour: ne vous endormez pas dans votre sécurité, ou vous êtes perdu. Bien au contraire, et c'est moi qui vous le dis, voyez des ennemis partout. Si l'on vous cherche querelle, évitez-la, fût-ce un enfant de dix ans qui vous la cherche; si l'on vous attaque de nuit ou de jour, battez en retraite et sans honte; si vous traversez un pont, tâtez les planches, de peur qu'une planche ne vous manque sous le pied; si vous passez devant une maison qu'on bâtit, regardez en l'air de peur qu'une pierre ne vous tombe sur la tête; si vous rentrez tard, faites-vous suivre par votre laquais, et que votre laquais soit armé, si toutefois vous êtes sûr de votre laquais. Défiez-vous de tout le monde, de votre ami, de votre frère, de votre maîtresse, de votre maîtresse surtout.» 

D'Artagnan rougit. 

«De ma maîtresse, répéta-t-il machinalement; et pourquoi plutôt d'elle que d'un autre? 

\speak  C'est que la maîtresse est un des moyens favoris du cardinal, il n'en a pas de plus expéditif: une femme vous vend pour dix pistoles, témoin Dalila. Vous savez les Écritures, hein?» 

D'Artagnan pensa au rendez-vous que lui avait donné Mme Bonacieux pour le soir même; mais nous devons dire, à la louange de notre héros, que la mauvaise opinion que M. de Tréville avait des femmes en général ne lui inspira pas le moindre petit soupçon contre sa jolie hôtesse. 

«Mais, à propos, reprit M. de Tréville, que sont devenus vos trois compagnons? 

\speak  J'allais vous demander si vous n'en aviez pas appris quelques nouvelles. 

\speak  Aucune, monsieur. 

\speak  Eh bien, je les ai laissés sur ma route: Porthos à Chantilly, avec un duel sur les bras; Aramis à Crèvecœur, avec une balle dans l'épaule; et Athos à Amiens, avec une accusation de faux-monnayeur sur le corps. 

\speak  Voyez-vous! dit M. de Tréville; et comment vous êtes-vous échappé, vous? 

\speak  Par miracle, monsieur, je dois le dire, avec un coup d'épée dans la poitrine, et en clouant M. le comte de Wardes sur le revers de la route de Calais, comme un papillon à une tapisserie. 

\speak  Voyez-vous encore! de Wardes, un homme au cardinal, un cousin de Rochefort. Tenez, mon cher ami, il me vient une idée. 

\speak  Dites, monsieur. 

\speak  À votre place, je ferais une chose. 

\speak  Laquelle? 

\speak  Tandis que Son Éminence me ferait chercher à Paris, je reprendrais, moi, sans tambour ni trompette, la route de Picardie, et je m'en irais savoir des nouvelles de mes trois compagnons. Que diable! ils méritent bien cette petite attention de votre part. 

\speak  Le conseil est bon, monsieur, et demain je partirai. 

\speak  Demain! et pourquoi pas ce soir? 

\speak  Ce soir, monsieur, je suis retenu à Paris par une affaire indispensable. 

\speak  Ah! jeune homme! jeune homme! quelque amourette? Prenez garde, je vous le répète: c'est la femme qui nous a perdus, tous tant que nous sommes. Croyez-moi, partez ce soir. 

\speak  Impossible! monsieur. 

\speak  Vous avez donc donné votre parole? 

\speak  Oui, monsieur. 

\speak  Alors c'est autre chose; mais promettez-moi que si vous n'êtes pas tué cette nuit, vous partirez demain. 

\speak  Je vous le promets. 

\speak  Avez-vous besoin d'argent? 

\speak  J'ai encore cinquante pistoles. C'est autant qu'il m'en faut, je le pense. 

\speak  Mais vos compagnons? 

\speak  Je pense qu'ils ne doivent pas en manquer. Nous sommes sortis de Paris chacun avec soixante-quinze pistoles dans nos poches. 

\speak  Vous reverrai-je avant votre départ? 

\speak  Non, pas que je pense, monsieur, à moins qu'il n'y ait du nouveau. 

\speak  Allons, bon voyage! 

\speak  Merci, monsieur.» 

Et d'Artagnan prit congé de M. de Tréville, touché plus que jamais de sa sollicitude toute paternelle pour ses mousquetaires. 

Il passa successivement chez Athos, chez Porthos et chez Aramis. Aucun d'eux n'était rentré. Leurs laquais aussi étaient absents, et l'on n'avait des nouvelles ni des uns, ni des autres. 

Il se serait bien informé d'eux à leurs maîtresses, mais il ne connaissait ni celle de Porthos, ni celle d'Aramis; quant à Athos, il n'en avait pas. 

En passant devant l'hôtel des Gardes, il jeta un coup d'œil dans l'écurie: trois chevaux étaient déjà rentrés sur quatre. Planchet, tout ébahi, était en train de les étriller, et avait déjà fini avec deux d'entre eux. 

«Ah! monsieur, dit Planchet en apercevant d'Artagnan, que je suis aise de vous voir! 

\speak  Et pourquoi cela, Planchet? demanda le jeune homme. 

\speak  Auriez-vous confiance en M. Bonacieux, notre hôte? 

\speak  Moi? pas le moins du monde. 

\speak  Oh! que vous faites bien, monsieur. 

\speak  Mais d'où vient cette question? 

\speak  De ce que, tandis que vous causiez avec lui, je vous observais sans vous écouter; monsieur, sa figure a changé deux ou trois fois de couleur. 

\speak  Bah! 

\speak  Monsieur n'a pas remarqué cela, préoccupé qu'il était de la lettre qu'il venait de recevoir; mais moi, au contraire, que l'étrange façon dont cette lettre était parvenue à la maison avait mis sur mes gardes, je n'ai pas perdu un mouvement de sa physionomie. 

\speak  Et tu l'as trouvée\dots? 

\speak  Traîtreuse, monsieur. 

\speak  Vraiment! 

\speak  De plus, aussitôt que monsieur l'a eu quitté et qu'il a disparu au coin de la rue, M. Bonacieux a pris son chapeau, a fermé sa porte et s'est mis à courir par la rue opposée. 

\speak  En effet, tu as raison, Planchet, tout cela me paraît fort louche, et, sois tranquille, nous ne lui paierons pas notre loyer que la chose ne nous ait été catégoriquement expliquée. 

\speak  Monsieur plaisante, mais monsieur verra. 

\speak  Que veux-tu, Planchet, ce qui doit arriver est écrit! 

\speak  Monsieur ne renonce donc pas à sa promenade de ce soir? 

\speak  Bien au contraire, Planchet, plus j'en voudrai à M. Bonacieux, et plus j'irai au rendez-vous que m'a donné cette lettre qui t'inquiète tant. 

\speak  Alors, si c'est la résolution de monsieur\dots 

\speak  Inébranlable, mon ami; ainsi donc, à neuf heures tiens-toi prêt ici, à l'hôtel; je viendrai te prendre.» 

Planchet, voyant qu'il n'y avait plus aucun espoir de faire renoncer son maître à son projet, poussa un profond soupir, et se mit à étriller le troisième cheval. 

Quant à d'Artagnan, comme c'était au fond un garçon plein de prudence, au lieu de rentrer chez lui, il s'en alla dîner chez ce prêtre gascon qui, au moment de la détresse des quatre amis, leur avait donné un déjeuner de chocolat. 
\include{chapters/24.tex}
%!TeX root=../musketeersfr.tex 

\chapter{Porthos}

\lettrine{A}{u} lieu de rentrer chez lui directement, d'Artagnan mit pied à terre à la porte de M. de Tréville, et monta rapidement l'escalier. Cette fois, il était décidé à lui raconter tout ce qui venait de se passer. Sans doute il lui donnerait de bons conseils dans toute cette affaire; puis, comme M. de Tréville voyait presque journellement la reine, il pourrait peut-être tirer de Sa Majesté quelque renseignement sur la pauvre femme à qui l'on faisait sans doute payer son dévouement à sa maîtresse. 

M. de Tréville écouta le récit du jeune homme avec une gravité qui prouvait qu'il voyait autre chose, dans toute cette aventure, qu'une intrigue d'amour; puis, quand d'Artagnan eut achevé: 

«Hum! dit-il, tout ceci sent Son Éminence d'une lieue. 

\speak  Mais, que faire? dit d'Artagnan. 

\speak  Rien, absolument rien, à cette heure, que quitter Paris, comme je vous l'ai dit, le plus tôt possible. Je verrai la reine, je lui raconterai les détails de la disparition de cette pauvre femme, qu'elle ignore sans doute; ces détails la guideront de son côté, et, à votre retour, peut-être aurai-je quelque bonne nouvelle à vous dire. Reposez vous en sur moi.» 

D'Artagnan savait que, quoique Gascon, M. de Tréville n'avait pas l'habitude de promettre, et que lorsque par hasard il promettait, il tenait plus qu'il n'avait promis. Il le salua donc, plein de reconnaissance pour le passé et pour l'avenir, et le digne capitaine, qui de son côté éprouvait un vif intérêt pour ce jeune homme si brave et si résolu, lui serra affectueusement la main en lui souhaitant un bon voyage. 

Décidé à mettre les conseils de M. de Tréville en pratique à l'instant même, d'Artagnan s'achemina vers la rue des Fossoyeurs, afin de veiller à la confection de son portemanteau. En s'approchant de sa maison, il reconnut M. Bonacieux en costume du matin, debout sur le seuil de sa porte. Tout ce que lui avait dit, la veille, le prudent Planchet sur le caractère sinistre de son hôte revint alors à l'esprit de d'Artagnan, qui le regarda plus attentivement qu'il n'avait fait encore. En effet, outre cette pâleur jaunâtre et maladive qui indique l'infiltration de la bile dans le sang et qui pouvait d'ailleurs n'être qu'accidentelle, d'Artagnan remarqua quelque chose de sournoisement perfide dans l'habitude des rides de sa face. Un fripon ne rit pas de la même façon qu'un honnête homme, un hypocrite ne pleure pas les mêmes larmes qu'un homme de bonne foi. Toute fausseté est un masque, et si bien fait que soit le masque, on arrive toujours, avec un peu d'attention, à le distinguer du visage. 

Il sembla donc à d'Artagnan que M. Bonacieux portait un masque, et même que ce masque était des plus désagréables à voir. 

En conséquence il allait, vaincu par sa répugnance pour cet homme, passer devant lui sans lui parler, quand, ainsi que la veille, M. Bonacieux l'interpella. 

«Eh bien, jeune homme, lui dit-il, il paraît que nous faisons de grasses nuits? Sept heures du matin, peste! Il me semble que vous retournez tant soit peu les habitudes reçues, et que vous rentrez à l'heure où les autres sortent. 

\speak  On ne vous fera pas le même reproche, maître Bonacieux, dit le jeune homme, et vous êtes le modèle des gens rangés. Il est vrai que lorsque l'on possède une jeune et jolie femme, on n'a pas besoin de courir après le bonheur: c'est le bonheur qui vient vous trouver; n'est-ce pas, monsieur Bonacieux?» 

Bonacieux devint pâle comme la mort et grimaça un sourire. 

«Ah! ah! dit Bonacieux, vous êtes un plaisant compagnon. Mais où diable avez-vous été courir cette nuit, mon jeune maître? Il paraît qu'il ne faisait pas bon dans les chemins de traverse.» 

D'Artagnan baissa les yeux vers ses bottes toutes couvertes de boue; mais dans ce mouvement ses regards se portèrent en même temps sur les souliers et les bas du mercier; on eût dit qu'on les avait trempés dans le même bourbier; les uns et les autres étaient maculés de taches absolument pareilles. 

Alors une idée subite traversa l'esprit de d'Artagnan. Ce petit homme gros, court, grisonnant, cette espèce de laquais vêtu d'un habit sombre, traité sans considération par les gens d'épée qui composaient l'escorte, c'était Bonacieux lui-même. Le mari avait présidé à l'enlèvement de sa femme. 

Il prit à d'Artagnan une terrible envie de sauter à la gorge du mercier et de l'étrangler; mais, nous l'avons dit, c'était un garçon fort prudent, et il se contint. Cependant la révolution qui s'était faite sur son visage était si visible, que Bonacieux en fut effrayé et essaya de reculer d'un pas; mais justement il se trouvait devant le battant de la porte, qui était fermée, et l'obstacle qu'il rencontra le força de se tenir à la même place. 

«Ah çà! mais vous qui plaisantez, mon brave homme, dit d'Artagnan, il me semble que si mes bottes ont besoin d'un coup d'éponge, vos bas et vos souliers réclament aussi un coup de brosse. Est-ce que de votre côté vous auriez couru la prétantaine, maître Bonacieux? Ah! diable, ceci ne serait point pardonnable à un homme de votre âge et qui, de plus, a une jeune et jolie femme comme la vôtre. 

\speak  Oh! mon Dieu, non, dit Bonacieux; mais hier j'ai été à Saint-Mandé pour prendre des renseignements sur une servante dont je ne puis absolument me passer, et comme les chemins étaient mauvais, j'en ai rapporté toute cette fange, que je n'ai pas encore eu le temps de faire disparaître.» 

Le lieu que désignait Bonacieux comme celui qui avait été le but de sa course fut une nouvelle preuve à l'appui des soupçons qu'avait conçus d'Artagnan. Bonacieux avait dit Saint-Mandé, parce que Saint-Mandé est le point absolument opposé à Saint-Cloud. 

Cette probabilité lui fut une première consolation. Si Bonacieux savait où était sa femme, on pourrait toujours, en employant des moyens extrêmes, forcer le mercier à desserrer les dents et à laisser échapper son secret. Il s'agissait seulement de changer cette probabilité en certitude. 

«Pardon, mon cher monsieur Bonacieux, si j'en use avec vous sans façon, dit d'Artagnan; mais rien n'altère comme de ne pas dormir, j'ai donc une soif d'enragé; permettez-moi de prendre un verre d'eau chez vous; vous le savez, cela ne se refuse pas entre voisins.» 

Et sans attendre la permission de son hôte, d'Artagnan entra vivement dans la maison, et jeta un coup d'œil rapide sur le lit. Le lit n'était pas défait. Bonacieux ne s'était pas couché. Il rentrait donc seulement il y avait une heure ou deux; il avait accompagné sa femme jusqu'à l'endroit où on l'avait conduite, ou tout au moins jusqu'au premier relais. 

«Merci, maître Bonacieux, dit d'Artagnan en vidant son verre, voilà tout ce que je voulais de vous. Maintenant je rentre chez moi, je vais faire brosser mes bottes par Planchet, et quand il aura fini, je vous l'enverrai si vous voulez pour brosser vos souliers.» 

Et il quitta le mercier tout ébahi de ce singulier adieu et se demandant s'il ne s'était pas enferré lui-même. 

Sur le haut de l'escalier il trouva Planchet tout effaré. 

«Ah! monsieur, s'écria Planchet dès qu'il eut aperçu son maître, en voilà bien d'une autre, et il me tardait bien que vous rentrassiez. 

\speak  Qu'y a-t-il donc? demanda d'Artagnan. 

\speak  Oh! je vous le donne en cent, monsieur, je vous le donne en mille de deviner la visite que j'ai reçue pour vous en votre absence. 

\speak  Quand cela? 

\speak  Il y a une demi-heure, tandis que vous étiez chez M. de Tréville. 

\speak  Et qui donc est venu? Voyons, parle. 

\speak  M. de Cavois. 

\speak  M. de Cavois? 

\speak  En personne. 

\speak  Le capitaine des gardes de Son Éminence? 

\speak  Lui-même. 

\speak  Il venait m'arrêter? 

\speak  Je m'en suis douté, monsieur, et cela malgré son air patelin. 

\speak  Il avait l'air patelin, dis-tu? 

\speak  C'est-à-dire qu'il était tout miel, monsieur. 

\speak  Vraiment? 

\speak  Il venait, disait-il, de la part de Son Éminence, qui vous voulait beaucoup de bien, vous prier de le suivre au Palais-Royal. 

\speak  Et tu lui as répondu? 

\speak  Que la chose était impossible, attendu que vous étiez hors de la maison, comme il le pouvait voir. 

\speak  Alors qu'a-t-il dit? 

\speak  Que vous ne manquiez pas de passer chez lui dans la journée; puis il a ajouté tout bas: «Dis à ton maître que Son Éminence est parfaitement disposée pour lui, et que sa fortune dépend peut-être de cette entrevue.» 

\speak  Le piège est assez maladroit pour le cardinal, reprit en souriant le jeune homme. 

\speak  Aussi, je l'ai vu, le piège, et j'ai répondu que vous seriez désespéré à votre retour. 

\speak  Où est-il allé? a demandé M. de Cavois. À Troyes en Champagne, ai-je répondu. Et quand est-il parti? 

\speak  Hier soir.» 

\speak  Planchet, mon ami, interrompit d'Artagnan, tu es véritablement un homme précieux. 

\speak  Vous comprenez, monsieur, j'ai pensé qu'il serait toujours temps, si vous désirez voir M. de Cavois, de me démentir en disant que vous n'étiez point parti; ce serait moi, dans ce cas, qui aurais fait le mensonge, et comme je ne suis pas gentilhomme, moi, je puis mentir. 

\speak  Rassure-toi, Planchet, tu conserveras ta réputation d'homme véridique: dans un quart d'heure nous partons. 

\speak  C'est le conseil que j'allais donner à monsieur; et où allons-nous, sans être trop curieux? 

\speak  Pardieu! du côté opposé à celui vers lequel tu as dit que j'étais allé. D'ailleurs, n'as-tu pas autant de hâte d'avoir des nouvelles de Grimaud, de Mousqueton et de Bazin que j'en ai, moi, de savoir ce que sont devenus Athos, Porthos et Aramis? 

\speak  Si fait, monsieur, dit Planchet, et je partirai quand vous voudrez; l'air de la province vaut mieux pour nous, à ce que je crois, en ce moment, que l'air de Paris. Ainsi donc\dots 

\speak  Ainsi donc, fais notre paquet, Planchet, et partons; moi, je m'en vais devant, les mains dans mes poches, pour qu'on ne se doute de rien. Tu me rejoindras à l'hôtel des Gardes. À propos, Planchet, je crois que tu as raison à l'endroit de notre hôte, et que c'est décidément une affreuse canaille. 

\speak  Ah! croyez-moi, monsieur, quand je vous dis quelque chose; je suis physionomiste, moi, allez!» 

D'Artagnan descendit le premier, comme la chose avait été convenue; puis, pour n'avoir rien à se reprocher, il se dirigea une dernière fois vers la demeure de ses trois amis: on n'avait reçu aucune nouvelle d'eux, seulement une lettre toute parfumée et d'une écriture élégante et menue était arrivée pour Aramis. D'Artagnan s'en chargea. Dix minutes après, Planchet le rejoignait dans les écuries de l'hôtel des Gardes. D'Artagnan, pour qu'il n'y eût pas de temps perdu, avait déjà sellé son cheval lui-même. 

«C'est bien, dit-il à Planchet, lorsque celui-ci eut joint le portemanteau à l'équipement; maintenant selle les trois autres, et partons. 

\speak  Croyez-vous que nous irons plus vite avec chacun deux chevaux? demanda Planchet avec son air narquois. 

\speak  Non, monsieur le mauvais plaisant, répondit d'Artagnan, mais avec nos quatre chevaux nous pourrons ramener nos trois amis, si toutefois nous les retrouvons vivants. 

\speak  Ce qui serait une grande chance, répondit Planchet, mais enfin il ne faut pas désespérer de la miséricorde de Dieu. 

\speak  Amen», dit d'Artagnan en enfourchant son cheval. 

Et tous deux sortirent de l'hôtel des Gardes, s'éloignèrent chacun par un bout de la rue, l'un devant quitter Paris par la barrière de la Villette et l'autre par la barrière de Montmartre, pour se rejoindre au-delà de Saint-Denis, manoeuvre stratégique qui, ayant été exécutée avec une égale ponctualité, fut couronnée des plus heureux résultats. D'Artagnan et Planchet entrèrent ensemble à Pierrefitte. 

Planchet était plus courageux, il faut le dire, le jour que la nuit. 

Cependant sa prudence naturelle ne l'abandonnait pas un seul instant; il n'avait oublié aucun des incidents du premier voyage, et il tenait pour ennemis tous ceux qu'il rencontrait sur la route. Il en résultait qu'il avait sans cesse le chapeau à la main, ce qui lui valait de sévères mercuriales de la part de d'Artagnan, qui craignait que, grâce à cet excès de politesse, on ne le prît pour le valet d'un homme de peu. 

Cependant, soit qu'effectivement les passants fussent touchés de l'urbanité de Planchet, soit que cette fois personne ne fût aposté sur la route du jeune homme, nos deux voyageurs arrivèrent à Chantilly sans accident aucun et descendirent à l'hôtel du Grand Saint Martin, le même dans lequel ils s'étaient arrêtés lors de leur premier voyage. 

L'hôte, en voyant un jeune homme suivi d'un laquais et de deux chevaux de main, s'avança respectueusement sur le seuil de la porte. Or, comme il avait déjà fait onze lieues, d'Artagnan jugea à propos de s'arrêter, que Porthos fût ou ne fût pas dans l'hôtel. Puis peut-être n'était-il pas prudent de s'informer du premier coup de ce qu'était devenu le mousquetaire. Il résulta de ces réflexions que d'Artagnan, sans demander aucune nouvelle de qui que ce fût, descendit, recommanda les chevaux à son laquais, entra dans une petite chambre destinée à recevoir ceux qui désiraient être seuls, et demanda à son hôte une bouteille de son meilleur vin et un déjeuner aussi bon que possible, demande qui corrobora encore la bonne opinion que l'aubergiste avait prise de son voyageur à la première vue. 

Aussi d'Artagnan fut-il servi avec une célérité miraculeuse. 

Le régiment des gardes se recrutait parmi les premiers gentilshommes du royaume, et d'Artagnan, suivi d'un laquais et voyageant avec quatre chevaux magnifiques, ne pouvait, malgré la simplicité de son uniforme, manquer de faire sensation. L'hôte voulut le servir lui-même; ce que voyant, d'Artagnan fit apporter deux verres et entama la conversation suivante: 

«Ma foi, mon cher hôte, dit d'Artagnan en remplissant les deux verres, je vous ai demandé de votre meilleur vin et si vous m'avez trompé, vous allez être puni par où vous avez péché, attendu que, comme je déteste boire seul, vous allez boire avec moi. Prenez donc ce verre, et buvons. À quoi boirons-nous, voyons, pour ne blesser aucune susceptibilité? Buvons à la prospérité de votre établissement! 

\speak  Votre Seigneurie me fait honneur, dit l'hôte, et je la remercie bien sincèrement de son bon souhait. 

\speak  Mais ne vous y trompez pas, dit d'Artagnan, il y a plus d'égoïsme peut-être que vous ne le pensez dans mon toast: il n'y a que les établissements qui prospèrent dans lesquels on soit bien reçu; dans les hôtels qui périclitent, tout va à la débandade, et le voyageur est victime des embarras de son hôte; or, moi qui voyage beaucoup et surtout sur cette route, je voudrais voir tous les aubergistes faire fortune. 

\speak  En effet, dit l'hôte, il me semble que ce n'est pas la première fois que j'ai l'honneur de voir monsieur. 

\speak  Bah? je suis passé dix fois peut-être à Chantilly, et sur les dix fois je me suis arrêté au moins trois ou quatre fois chez vous. Tenez, j'y étais encore il y a dix ou douze jours à peu près; je faisais la conduite à des amis, à des mousquetaires, à telle enseigne que l'un d'eux s'est pris de dispute avec un étranger, un inconnu, un homme qui lui a cherché je ne sais quelle querelle. 

\speak  Ah! oui vraiment! dit l'hôte, et je me le rappelle parfaitement. N'est-ce pas de M. Porthos que Votre Seigneurie veut me parler? 

\speak  C'est justement le nom de mon compagnon de voyage. 

«Mon Dieu! mon cher hôte, dites-moi, lui serait-il arrivé malheur? 

\speak  Mais Votre Seigneurie a dû remarquer qu'il n'a pas pu continuer sa route. 

\speak  En effet, il nous avait promis de nous rejoindre, et nous ne l'avons pas revu. 

\speak  Il nous a fait l'honneur de rester ici. 

\speak  Comment! il vous a fait l'honneur de rester ici? 

\speak  Oui, monsieur, dans cet hôtel; nous sommes même bien inquiets. 

\speak  Et de quoi? 

\speak  De certaines dépenses qu'il a faites. 

\speak  Eh bien, mais les dépenses qu'il a faites, il les paiera. 

\speak  Ah! monsieur, vous me mettez véritablement du baume dans le sang! Nous avons fait de fort grandes avances, et ce matin encore le chirurgien nous déclarait que si M. Porthos ne le payait pas, c'était à moi qu'il s'en prendrait, attendu que c'était moi qui l'avais envoyé chercher. 

\speak  Mais Porthos est donc blessé? 

\speak  Je ne saurais vous le dire, monsieur. 

\speak  Comment, vous ne sauriez me le dire? vous devriez cependant être mieux informé que personne. 

\speak  Oui, mais dans notre état nous ne disons pas tout ce que nous savons, monsieur, surtout quand on nous a prévenus que nos oreilles répondraient pour notre langue. 

\speak  Eh bien, puis-je voir Porthos? 

\speak  Certainement, monsieur. Prenez l'escalier, montez au premier et frappez au n° 1. Seulement, prévenez que c'est vous. 

\speak  Comment! que je prévienne que c'est moi? 

\speak  Oui, car il pourrait vous arriver malheur. 

\speak  Et quel malheur voulez-vous qu'il m'arrive? 

\speak  M. Porthos peut vous prendre pour quelqu'un de la maison et, dans un mouvement de colère, vous passer son épée à travers le corps ou vous brûler la cervelle. 

\speak  Que lui avez-vous donc fait? 

\speak  Nous lui avons demandé de l'argent. 

\speak  Ah! diable, je comprends cela; c'est une demande que Porthos reçoit très mal quand il n'est pas en fonds; mais je sais qu'il devait y être. 

\speak  C'est ce que nous avions pensé aussi, monsieur; comme la maison est fort régulière et que nous faisons nos comptes toutes les semaines, au bout de huit jours nous lui avons présenté notre note; mais il paraît que nous sommes tombés dans un mauvais moment, car, au premier mot que nous avons prononcé sur la chose, il nous a envoyés à tous les diables; il est vrai qu'il avait joué la veille. 

\speak  Comment, il avait joué la veille! et avec qui? 

\speak  Oh! mon Dieu, qui sait cela? avec un seigneur qui passait et auquel il avait fait proposer une partie de lansquenet. 

\speak  C'est cela, le malheureux aura tout perdu. 

\speak  Jusqu'à son cheval, monsieur, car lorsque l'étranger a été pour partir, nous nous sommes aperçus que son laquais sellait le cheval de M. Porthos. Alors nous lui en avons fait l'observation, mais il nous a répondu que nous nous mêlions de ce qui ne nous regardait pas et que ce cheval était à lui. Nous avons aussitôt fait prévenir M. Porthos de ce qui se passait, mais il nous à fait dire que nous étions des faquins de douter de la parole d'un gentilhomme, et que, puisque celui-là avait dit que le cheval était à lui, il fallait bien que cela fût. 

\speak  Je le reconnais bien là, murmura d'Artagnan. 

\speak  Alors, continua l'hôte, je lui fis répondre que du moment où nous paraissions destinés à ne pas nous entendre à l'endroit du paiement, j'espérais qu'il aurait au moins la bonté d'accorder la faveur de sa pratique à mon confrère le maître de l'Aigle d'Or; mais M. Porthos me répondit que mon hôtel étant le meilleur, il désirait y rester. 

«Cette réponse était trop flatteuse pour que j'insistasse sur son départ. Je me bornai donc à le prier de me rendre sa chambre, qui est la plus belle de l'hôtel, et de se contenter d'un joli petit cabinet au troisième. Mais à ceci M. Porthos répondit que, comme il attendait d'un moment à l'autre sa maîtresse, qui était une des plus grandes dames de la cour, je devais comprendre que la chambre qu'il me faisait l'honneur d'habiter chez moi était encore bien médiocre pour une pareille personne. 

«Cependant, tout en reconnaissant la vérité de ce qu'il disait, je crus devoir insister; mais, sans même se donner la peine d'entrer en discussion avec moi, il prit son pistolet, le mit sur sa table de nuit et déclara qu'au premier mot qu'on lui dirait d'un déménagement quelconque à l'extérieur ou à l'intérieur, il brûlerait la cervelle à celui qui serait assez imprudent pour se mêler d'une chose qui ne regardait que lui. Aussi, depuis ce temps-là, monsieur, personne n'entre plus dans sa chambre, si ce n'est son domestique. 

\speak  Mousqueton est donc ici? 

\speak  Oui, monsieur; cinq jours après son départ, il est revenu de fort mauvaise humeur de son côté; il paraît que lui aussi a eu du désagrément dans son voyage. Malheureusement, il est plus ingambe que son maître, ce qui fait que pour son maître il met tout sens dessus dessous, attendu que, comme il pense qu'on pourrait lui refuser ce qu'il demande, il prend tout ce dont il a besoin sans demander. 

\speak  Le fait est, répondit d'Artagnan, que j'ai toujours remarqué dans Mousqueton un dévouement et une intelligence très supérieurs. 

\speak  Cela est possible, monsieur; mais supposez qu'il m'arrive seulement quatre fois par an de me trouver en contact avec une intelligence et un dévouement semblables, et je suis un homme ruiné. 

\speak  Non, car Porthos vous paiera. 

\speak  Hum! fit l'hôtelier d'un ton de doute. 

\speak  C'est le favori d'une très grande dame qui ne le laissera pas dans l'embarras pour une misère comme celle qu'il vous doit. 

\speak  Si j'ose dire ce que je crois là-dessus\dots 

\speak  Ce que vous croyez? 

\speak  Je dirai plus: ce que je sais. 

\speak  Ce que vous savez? 

\speak  Et même ce dont je suis sûr. 

\speak  Et de quoi êtes-vous sûr, voyons? 

\speak  Je dirai que je connais cette grande dame. 

\speak  Vous? 

\speak  Oui, moi. 

\speak  Et comment la connaissez-vous? 

\speak  Oh! monsieur, si je croyais pouvoir me fier à votre discrétion\dots 

\speak  Parlez, et foi de gentilhomme, vous n'aurez pas à vous repentir de votre confiance. 

\speak  Eh bien, monsieur, vous concevez, l'inquiétude fait faire bien des choses. 

\speak  Qu'avez-vous fait? 

\speak  Oh! d'ailleurs, rien qui ne soit dans le droit d'un créancier. 

\speak  Enfin? 

\speak  M. Porthos nous a remis un billet pour cette duchesse, en nous recommandant de le jeter à la poste. Son domestique n'était pas encore arrivé. Comme il ne pouvait pas quitter sa chambre, il fallait bien qu'il nous chargeât de ses commissions. 

\speak  Ensuite? 

\speak  Au lieu de mettre la lettre à la poste, ce qui n'est jamais bien sûr, j'ai profité de l'occasion de l'un de mes garçons qui allait à Paris, et je lui ai ordonné de la remettre à cette duchesse elle-même. C'était remplir les intentions de M. Porthos, qui nous avait si fort recommandé cette lettre, n'est-ce pas? 

\speak  À peu près. 

\speak  Eh bien, monsieur, savez-vous ce que c'est que cette grande dame? 

\speak  Non; j'en ai entendu parler à Porthos, voilà tout. 

\speak  Savez-vous ce que c'est que cette prétendue duchesse? 

\speak  Je vous le répète, je ne la connais pas. 

\speak  C'est une vieille procureuse au Châtelet, monsieur, nommée Mme Coquenard, laquelle a au moins cinquante ans, et se donne encore des airs d'être jalouse. Cela me paraissait aussi fort singulier, une princesse qui demeure rue aux Ours. 

\speak  Comment savez-vous cela? 

\speak  Parce qu'elle s'est mise dans une grande colère en recevant la lettre, disant que M. Porthos était un volage, et que c'était encore pour quelque femme qu'il avait reçu ce coup d'épée. 

\speak  Mais il a donc reçu un coup d'épée? 

\speak  Ah! mon Dieu! qu'ai-je dit là? 

\speak  Vous avez dit que Porthos avait reçu un coup d'épée. 

\speak  Oui; mais il m'avait si fort défendu de le dire! 

\speak  Pourquoi cela? 

\speak  Dame! monsieur, parce qu'il s'était vanté de perforer cet étranger avec lequel vous l'avez laisse en dispute, et que c'est cet étranger, au contraire, qui, malgré toutes ses rodomontades, l'a couché sur le carreau. Or, comme M. Porthos est un homme fort glorieux, excepté envers la duchesse, qu'il avait cru intéresser en lui faisant le récit de son aventure, il ne veut avouer à personne que c'est un coup d'épée qu'il a reçu. 

\speak  Ainsi c'est donc un coup d'épée qui le retient dans son lit? 

\speak  Et un maître coup d'épée, je vous l'assure. Il faut que votre ami ait l'âme chevillée dans le corps. 

\speak  Vous étiez donc là? 

\speak  Monsieur, je les avais suivis par curiosité, de sorte que j'ai vu le combat sans que les combattants me vissent. 

\speak  Et comment cela s'est-il passé? 

\speak  Oh! la chose n'a pas été longue, je vous en réponds. Ils se sont mis en garde; l'étranger a fait une feinte et s'est fendu; tout cela si rapidement, que lorsque M. Porthos est arrivé à la parade, il avait déjà trois pouces de fer dans la poitrine. Il est tombé en arrière. L'étranger lui a mis aussitôt la pointe de son épée à la gorge; et M. Porthos, se voyant à la merci de son adversaire, s'est avoué vaincu. Sur quoi, l'étranger lui a demandé son nom et apprenant qu'il s'appelait M. Porthos, et non M. d'Artagnan, lui a offert son bras, l'a ramené à l'hôtel, est monté à cheval et a disparu. 

\speak  Ainsi c'est à M. d'Artagnan qu'en voulait cet étranger? 

\speak  Il paraît que oui. 

\speak  Et savez-vous ce qu'il est devenu? 

\speak  Non; je ne l'avais jamais vu jusqu'à ce moment et nous ne l'avons pas revu depuis. 

\speak  Très bien; je sais ce que je voulais savoir. Maintenant, vous dites que la chambre de Porthos est au premier, n° 1? 

\speak  Oui, monsieur, la plus belle de l'auberge; une chambre que j'aurais déjà eu dix fois l'occasion de louer. 

\speak  Bah! tranquillisez vous, dit d'Artagnan en riant; Porthos vous paiera avec l'argent de la duchesse Coquenard. 

\speak  Oh! monsieur, procureuse ou duchesse, si elle lâchait les cordons de sa bourse, ce ne serait rien; mais elle a positivement répondu qu'elle était lasse des exigences et des infidélités de M. Porthos, et qu'elle ne lui enverrait pas un denier. 

\speak  Et avez-vous rendu cette réponse à votre hôte? 

\speak  Nous nous en sommes bien gardés: il aurait vu de quelle manière nous avions fait la commission. 

\speak  Si bien qu'il attend toujours son argent? 

\speak  Oh! mon Dieu, oui! Hier encore, il a écrit; mais, cette fois, c'est son domestique qui a mis la lettre à la poste. 

\speak  Et vous dites que la procureuse est vieille et laide. 

\speak  Cinquante ans au moins, monsieur, et pas belle du tout, à ce qu'a dit Pathaud. 

\speak  En ce cas, soyez tranquille, elle se laissera attendrir; d'ailleurs Porthos ne peut pas vous devoir grand-chose. 

\speak  Comment, pas grand-chose! Une vingtaine de pistoles déjà, sans compter le médecin. Oh! il ne se refuse rien, allez! on voit qu'il est habitué à bien vivre. 

\speak  Eh bien, si sa maîtresse l'abandonne, il trouvera des amis, je vous le certifie. Ainsi, mon cher hôte, n'ayez aucune inquiétude, et continuez d'avoir pour lui tous les soins qu'exige son état. 

\speak  Monsieur m'a promis de ne pas parler de la procureuse et de ne pas dire un mot de la blessure. 

\speak  C'est chose convenue; vous avez ma parole. 

\speak  Oh! c'est qu'il me tuerait, voyez-vous! 

\speak  N'ayez pas peur; il n'est pas si diable qu'il en a l'air. 

En disant ces mots, d'Artagnan monta l'escalier, laissant son hôte un peu plus rassuré à l'endroit de deux choses auxquelles il paraissait beaucoup tenir: sa créance et sa vie. 

Au haut de l'escalier, sur la porte la plus apparente du corridor était tracé, à l'encre noire, un n° 1 gigantesque; d'Artagnan frappa un coup, et, sur l'invitation de passer outre qui lui vint de l'intérieur, il entra. 

Porthos était couché, et faisait une partie de lansquenet avec Mousqueton, pour s'entretenir la main, tandis qu'une broche chargée de perdrix tournait devant le feu, et qu'à chaque coin d'une grande cheminée bouillaient sur deux réchauds deux casseroles, d'où s'exhalait une double odeur de gibelotte et de matelote qui réjouissait l'odorat. En outre, le haut d'un secrétaire et le marbre d'une commode étaient couverts de bouteilles vides. 

À la vue de son ami, Porthos jeta un grand cri de joie; et Mousqueton, se levant respectueusement, lui céda la place et s'en alla donner un coup d'œil aux deux casseroles, dont il paraissait avoir l'inspection particulière. 

«Ah! pardieu! c'est vous, dit Porthos à d'Artagnan, soyez le bienvenu, et excusez-moi si je ne vais pas au-devant de vous. Mais, ajouta-t-il en regardant d'Artagnan avec une certaine inquiétude, vous savez ce qui m'est arrivé? 

\speak  Non. 

\speak  L'hôte ne vous a rien dit? 

\speak  J'ai demandé après vous, et je suis monté tout droit.» 

\speak  Porthos parut respirer plus librement. 

«Et que vous est-il donc arrivé, mon cher Porthos? continua d'Artagnan. 

\speak  Il m'est arrivé qu'en me fendant sur mon adversaire, à qui j'avais déjà allongé trois coups d'épée, et avec lequel je voulais en finir d'un quatrième, mon pied a porté sur une pierre, et je me suis foulé le genou. 

\speak  Vraiment? 

\speak  D'honneur! Heureusement pour le maraud, car je ne l'aurais laissé que mort sur la place, je vous en réponds. 

\speak  Et qu'est-il devenu? 

\speak  Oh! je n'en sais rien; il en a eu assez, et il est parti sans demander son reste; mais vous, mon cher d'Artagnan, que vous est-il arrivé? 

\speak  De sorte, continua d'Artagnan, que cette foulure, mon cher Porthos, vous retient au lit? 

\speak  Ah! mon Dieu, oui, voilà tout; du reste, dans quelques jours je serai sur pied. 

\speak  Pourquoi alors ne vous êtes-vous pas fait transporter à Paris? Vous devez vous ennuyer cruellement ici. 

\speak  C'était mon intention; mais, mon cher ami, il faut que je vous avoue une chose. 

\speak  Laquelle? 

\speak  C'est que, comme je m'ennuyais cruellement, ainsi que vous le dites, et que j'avais dans ma poche les soixante-quinze pistoles que vous m'aviez distribuées j'ai, pour me distraire, fait monter près de moi un gentilhomme qui était de passage, et auquel j'ai proposé de faire une partie de dés. Il a accepté, et, ma foi, mes soixante-quinze pistoles sont passées de ma poche dans la sienne, sans compter mon cheval, qu'il a encore emporté par dessus le marché. Mais vous, mon cher d'Artagnan? 

\speak  Que voulez-vous, mon cher Porthos, on ne peut pas être privilégié de toutes façons, dit d'Artagnan; vous savez le proverbe: “Malheureux au jeu, heureux en amour.” Vous êtes trop heureux en amour pour que le jeu ne se venge pas; mais que vous importent, à vous, les revers de la fortune! n'avez-vous pas, heureux coquin que vous êtes, n'avez-vous pas votre duchesse, qui ne peut manquer de vous venir en aide? 

\speak  Eh bien, voyez, mon cher d'Artagnan, comme je joue de guignon, répondit Porthos de l'air le plus dégagé du monde! je lui ai écrit de m'envoyer quelque cinquante louis dont j'avais absolument besoin, vu la position où je me trouvais\dots 

\speak  Eh bien? 

\speak  Eh bien, il faut qu'elle soit dans ses terres, car elle ne m'a pas répondu. 

\speak  Vraiment? 

\speak  Non. Aussi je lui ai adressé hier une seconde épître plus pressante encore que la première; mais vous voilà, mon très cher, parlons de vous. Je commençais, je vous l'avoue, à être dans une certaine inquiétude sur votre compte. 

\speak  Mais votre hôte se conduit bien envers vous, à ce qu'il paraît, mon cher Porthos, dit d'Artagnan, montrant au malade les casseroles pleines et les bouteilles vides. 

\speak  Couci-couci! répondit Porthos. Il y a déjà trois ou quatre jours que l'impertinent m'a monté son compte, et que je les ai mis à la porte, son compte et lui; de sorte que je suis ici comme une façon de vainqueur, comme une manière de conquérant. Aussi, vous le voyez, craignant toujours d'être forcé dans la position, je suis armé jusqu'aux dents. 

\speak  Cependant, dit en riant d'Artagnan, il me semble que de temps en temps vous faites des sorties.» 

Et il montrait du doigt les bouteilles et les casseroles. 

«Non, pas moi, malheureusement! dit Porthos. Cette misérable foulure me retient au lit, mais Mousqueton bat la campagne, et il rapporte des vivres. Mousqueton, mon ami, continua Porthos, vous voyez qu'il nous arrive du renfort, il nous faudra un supplément de victuailles. 

\speak  Mousqueton, dit d'Artagnan, il faudra que vous me rendiez un service. 

\speak  Lequel, monsieur? 

\speak  C'est de donner votre recette à Planchet; je pourrais me trouver assiégé à mon tour, et je ne serais pas fâché qu'il me fît jouir des mêmes avantages dont vous gratifiez votre maître. 

\speak  Eh! mon Dieu! monsieur, dit Mousqueton d'un air modeste, rien de plus facile. Il s'agit d'être adroit, voilà tout. J'ai été élevé à la campagne, et mon père, dans ses moments perdus, était quelque peu braconnier. 

\speak  Et le reste du temps, que faisait-il? 

\speak  Monsieur, il pratiquait une industrie que j'ai toujours trouvée assez heureuse. 

\speak  Laquelle? 

\speak  Comme c'était au temps des guerres des catholiques et des huguenots, et qu'il voyait les catholiques exterminer les huguenots, et les huguenots exterminer les catholiques, le tout au nom de la religion, il s'était fait une croyance mixte, ce qui lui permettait d'être tantôt catholique, tantôt huguenot. Or il se promenait habituellement, son escopette sur l'épaule, derrière les haies qui bordent les chemins, et quand il voyait venir un catholique seul, la religion protestante l'emportait aussitôt dans son esprit. Il abaissait son escopette dans la direction du voyageur; puis, lorsqu'il était à dix pas de lui, il entamait un dialogue qui finissait presque toujours par l'abandon que le voyageur faisait de sa bourse pour sauver sa vie. Il va sans dire que lorsqu'il voyait venir un huguenot, il se sentait pris d'un zèle catholique si ardent, qu'il ne comprenait pas comment, un quart d'heure auparavant, il avait pu avoir des doutes sur la supériorité de notre sainte religion. Car, moi, monsieur, je suis catholique, mon père, fidèle à ses principes, ayant fait mon frère aîné huguenot. 

\speak  Et comment a fini ce digne homme? demanda d'Artagnan. 

\speak  Oh! de la façon la plus malheureuse, monsieur. Un jour, il s'était trouvé pris dans un chemin creux entre un huguenot et un catholique à qui il avait déjà eu affaire, et qui le reconnurent tous deux; de sorte qu'ils se réunirent contre lui et le pendirent à un arbre; puis ils vinrent se vanter de la belle équipée qu'ils avaient faite dans le cabaret du premier village, où nous étions à boire, mon frère et moi. 

\speak  Et que fîtes-vous? dit d'Artagnan. 

\speak  Nous les laissâmes dire, reprit Mousqueton. Puis comme, en sortant de ce cabaret, ils prenaient chacun une route opposée, mon frère alla s'embusquer sur le chemin du catholique, et moi sur celui du protestant. Deux heures après, tout était fini, nous leur avions fait à chacun son affaire, tout en admirant la prévoyance de notre pauvre père qui avait pris la précaution de nous élever chacun dans une religion différente. 

\speak  En effet, comme vous le dites, Mousqueton, votre père me paraît avoir été un gaillard fort intelligent. Et vous dites donc que, dans ses moments perdus, le brave homme était braconnier? 

\speak  Oui, monsieur, et c'est lui qui m'a appris à nouer un collet et à placer une ligne de fond. Il en résulte que lorsque j'ai vu que notre gredin d'hôte nous nourrissait d'un tas de grosses viandes bonnes pour des manants, et qui n'allaient point à deux estomacs aussi débilités que les nôtres, je me suis remis quelque peu à mon ancien métier. Tout en me promenant dans le bois de M. le Prince, j'ai tendu des collets dans les passées; tout en me couchant au bord des pièces d'eau de Son Altesse, j'ai glissé des lignes dans les étangs. De sorte que maintenant, grâce à Dieu, nous ne manquons pas, comme monsieur peut s'en assurer, de perdrix et de lapins, de carpes et d'anguilles, tous aliments légers et sains, convenables pour des malades. 

\speak  Mais le vin, dit d'Artagnan, qui fournit le vin? c'est votre hôte? 

\speak  C'est-à-dire, oui et non. 

\speak  Comment, oui et non? 

\speak  Il le fournit, il est vrai, mais il ignore qu'il a cet honneur. 

\speak  Expliquez-vous, Mousqueton, votre conversation est pleine de choses instructives. 

\speak  Voici, monsieur. Le hasard a fait que j'ai rencontré dans mes pérégrinations un Espagnol qui avait vu beaucoup de pays, et entre autres le Nouveau Monde. 

\speak  Quel rapport le Nouveau Monde peut-il avoir avec les bouteilles qui sont sur ce secrétaire et sur cette commode? 

\speak  Patience, monsieur, chaque chose viendra à son tour. 

\speak  C'est juste, Mousqueton; je m'en rapporte à vous, et j'écoute. 

\speak  Cet Espagnol avait à son service un laquais qui l'avait accompagné dans son voyage au Mexique. Ce laquais était mon compatriote, de sorte que nous nous liâmes d'autant plus rapidement qu'il y avait entre nous de grands rapports de caractère. Nous aimions tous deux la chasse par-dessus tout, de sorte qu'il me racontait comment, dans les plaines de pampas, les naturels du pays chassent le tigre et les taureaux avec de simples noeuds coulants qu'ils jettent au cou de ces terribles animaux. D'abord, je ne voulais pas croire qu'on pût en arriver à ce degré d'adresse, de jeter à vingt ou trente pas l'extrémité d'une corde où l'on veut; mais devant la preuve il fallait bien reconnaître la vérité du récit. Mon ami plaçait une bouteille à trente pas, et à chaque coup il lui prenait le goulot dans un noeud coulant. Je me livrai à cet exercice, et comme la nature m'a doué de quelques facultés, aujourd'hui je jette le lasso aussi bien qu'aucun homme du monde. Eh bien, comprenez-vous? Notre hôte a une cave très bien garnie, mais dont la clef ne le quitte pas; seulement, cette cave a un soupirail. Or, par ce soupirail, je jette le lasso; et comme je sais maintenant où est le bon coin, j'y puise. Voici, monsieur, comment le Nouveau Monde se trouve être en rapport avec les bouteilles qui sont sur cette commode et sur ce secrétaire. Maintenant, voulez-vous goûter notre vin, et, sans prévention, vous nous direz ce que vous en pensez. 

\speak  Merci, mon ami, merci; malheureusement, je viens de déjeuner. 

\speak  Eh bien, dit Porthos, mets la table, Mousqueton, et tandis que nous déjeunerons, nous, d'Artagnan nous racontera ce qu'il est devenu lui-même, depuis dix jours qu'il nous a quittés. 

\speak  Volontiers», dit d'Artagnan. 

Tandis que Porthos et Mousqueton déjeunaient avec des appétits de convalescents et cette cordialité de frères qui rapproche les hommes dans le malheur, d'Artagnan raconta comment Aramis blessé avait été forcé de s'arrêter à Crèvecœur, comment il avait laissé Athos se débattre à Amiens entre les mains de quatre hommes qui l'accusaient d'être un faux-monnayeur, et comment, lui, d'Artagnan, avait été forcé de passer sur le ventre du comte de Wardes pour arriver jusqu'en Angleterre. 

Mais là s'arrêta la confidence de d'Artagnan; il annonça seulement qu'à son retour de la Grande-Bretagne il avait ramené quatre chevaux magnifiques, dont un pour lui et un autre pour chacun de ses compagnons, puis il termina en annonçant à Porthos que celui qui lui était destiné était déjà installé dans l'écurie de l'hôtel. 

En ce moment Planchet entra; il prévenait son maître que les chevaux étaient suffisamment reposés, et qu'il serait possible d'aller coucher à Clermont. 

Comme d'Artagnan était à peu près rassuré sur Porthos, et qu'il lui tardait d'avoir des nouvelles de ses deux autres amis, il tendit la main au malade, et le prévint qu'il allait se mettre en route pour continuer ses recherches. Au reste, comme il comptait revenir par la même route, si, dans sept à huit jours, Porthos était encore à l'hôtel du Grand Saint Martin, il le reprendrait en passant. 

Porthos répondit que, selon toute probabilité, sa foulure ne lui permettrait pas de s'éloigner d'ici là. D'ailleurs il fallait qu'il restât à Chantilly pour attendre une réponse de sa duchesse. 

D'Artagnan lui souhaita cette réponse prompte et bonne; et après avoir recommandé de nouveau Porthos à Mousqueton, et payé sa dépense à l'hôte, il se remit en route avec Planchet, déjà débarrassé d'un de ses chevaux de main.
%!TeX root=../musketeersfr.tex 

\chapter{La Thèse D'Aramis} 
	
\lettrine{D}{'Artagnan} n'avait rien dit à Porthos de sa blessure ni de sa procureuse. C'était un garçon fort sage que notre Béarnais, si jeune qu'il fût. En conséquence, il avait fait semblant de croire tout ce que lui avait raconté le glorieux mousquetaire, convaincu qu'il n'y a pas d'amitié qui tienne à un secret surpris, surtout quand ce secret intéresse l'orgueil; puis on a toujours une certaine supériorité morale sur ceux dont on sait la vie. 

Or d'Artagnan, dans ses projets d'intrigue à venir, et décidé qu'il était à faire de ses trois compagnons les instruments de sa fortune, d'Artagnan n'était pas fâché de réunir d'avance dans sa main les fils invisibles à l'aide desquels il comptait les mener. 

Cependant, tout le long de la route, une profonde tristesse lui serrait le cœur: il pensait à cette jeune et jolie Mme Bonacieux qui devait lui donner le prix de son dévouement; mais, hâtons-nous de le dire, cette tristesse venait moins chez le jeune homme du regret de son bonheur perdu que de la crainte qu'il éprouvait qu'il n'arrivât malheur à cette pauvre femme. Pour lui, il n'y avait pas de doute, elle était victime d'une vengeance du cardinal et comme on le sait, les vengeances de Son Éminence étaient terribles. Comment avait-il trouvé grâce devant les yeux du ministre, c'est ce qu'il ignorait lui-même et sans doute ce que lui eût révélé M. de Cavois, si le capitaine des gardes l'eût trouvé chez lui. 

Rien ne fait marcher le temps et n'abrège la route comme une pensée qui absorbe en elle-même toutes les facultés de l'organisation de celui qui pense. L'existence extérieure ressemble alors à un sommeil dont cette pensée est le rêve. Par son influence, le temps n'a plus de mesure, l'espace n'a plus de distance. On part d'un lieu, et l'on arrive à un autre, voilà tout. De l'intervalle parcouru, rien ne reste présent à votre souvenir qu'un brouillard vague dans lequel s'effacent mille images confuses d'arbres, de montagnes et de paysages. Ce fut en proie à cette hallucination que d'Artagnan franchit, à l'allure que voulut prendre son cheval, les six ou huit lieues qui séparent Chantilly de Crèvecœur, sans qu'en arrivant dans ce village il se souvînt d'aucune des choses qu'il avait rencontrées sur sa route. 

Là seulement la mémoire lui revint, il secoua la tête aperçut le cabaret où il avait laissé Aramis, et, mettant son cheval au trot, il s'arrêta à la porte. 

Cette fois ce ne fut pas un hôte, mais une hôtesse qui le reçut; d'Artagnan était physionomiste, il enveloppa d'un coup d'œil la grosse figure réjouie de la maîtresse du lieu, et comprit qu'il n'avait pas besoin de dissimuler avec elle et qu'il n'avait rien à craindre de la part d'une si joyeuse physionomie. 

«Ma bonne dame, lui demanda d'Artagnan, pourriez-vous me dire ce qu'est devenu un de mes amis, que nous avons été forcés de laisser ici il y a une douzaine de jours? 

\speak  Un beau jeune homme de vingt-trois à vingt-quatre ans, doux, aimable, bien fait? 

\speak  De plus, blessé à l'épaule. 

\speak  C'est cela! 

\speak  Justement. 

\speak  Eh bien, monsieur, il est toujours ici. 

\speak  Ah! pardieu, ma chère dame, dit d'Artagnan en mettant pied à terre et en jetant la bride de son cheval au bras de Planchet, vous me rendez la vie; où est-il, ce cher Aramis, que je l'embrasse? car, je l'avoue, j'ai hâte de le revoir. 

\speak  Pardon, monsieur, mais je doute qu'il puisse vous recevoir en ce moment. 

\speak  Pourquoi cela? est-ce qu'il est avec une femme? 

\speak  Jésus! que dites-vous là! le pauvre garçon! Non, monsieur, il n'est pas avec une femme. 

\speak  Et avec qui est-il donc? 

\speak  Avec le curé de Montdidier et le supérieur des jésuites d'Amiens. 

\speak  Mon Dieu! s'écria d'Artagnan, le pauvre garçon irait-il plus mal? 

\speak  Non, monsieur, au contraire; mais, à la suite de sa maladie, la grâce l'a touché et il s'est décidé à entrer dans les ordres. 

\speak  C'est juste, dit d'Artagnan, j'avais oublié qu'il n'était mousquetaire que par intérim. 

\speak  Monsieur insiste-t-il toujours pour le voir? 

\speak  Plus que jamais. 

\speak  Eh bien, monsieur n'a qu'à prendre l'escalier à droite dans la cour, au second, n° 5.» 

D'Artagnan s'élança dans la direction indiquée et trouva un de ces escaliers extérieurs comme nous en voyons encore aujourd'hui dans les cours des anciennes auberges. Mais on n'arrivait pas ainsi chez le futur abbé; les défilés de la chambre d'Aramis étaient gardés ni plus ni moins que les jardins d'Aramis; Bazin stationnait dans le corridor et lui barra le passage avec d'autant plus d'intrépidité qu'après bien des années d'épreuve, Bazin se voyait enfin près d'arriver au résultat qu'il avait éternellement ambitionné. 

En effet, le rêve du pauvre Bazin avait toujours été de servir un homme d'Église, et il attendait avec impatience le moment sans cesse entrevu dans l'avenir où Aramis jetterait enfin la casaque aux orties pour prendre la soutane. La promesse renouvelée chaque jour par le jeune homme que le moment ne pouvait tarder l'avait seule retenu au service d'un mousquetaire, service dans lequel, disait-il, il ne pouvait manquer de perdre son âme. 

Bazin était donc au comble de la joie. Selon toute probabilité, cette fois son maître ne se dédirait pas. La réunion de la douleur physique à la douleur morale avait produit l'effet si longtemps désiré: Aramis, souffrant à la fois du corps et de l'âme, avait enfin arrêté sur la religion ses yeux et sa pensée, et il avait regardé comme un avertissement du Ciel le double accident qui lui était arrivé, c'est-à-dire la disparition subite de sa maîtresse et sa blessure à l'épaule. 

On comprend que rien ne pouvait, dans la disposition où il se trouvait, être plus désagréable à Bazin que l'arrivée de d'Artagnan, laquelle pouvait rejeter son maître dans le tourbillon des idées mondaines qui l'avaient si longtemps entraîné. Il résolut donc de défendre bravement la porte; et comme, trahi par la maîtresse de l'auberge, il ne pouvait dire qu'Aramis était absent, il essaya de prouver au nouvel arrivant que ce serait le comble de l'indiscrétion que de déranger son maître dans la pieuse conférence qu'il avait entamée depuis le matin, et qui, au dire de Bazin, ne pouvait être terminée avant le soir. 

Mais d'Artagnan ne tint aucun compte de l'éloquent discours de maître Bazin, et comme il ne se souciait pas d'entamer une polémique avec le valet de son ami, il l'écarta tout simplement d'une main, et de l'autre il tourna le bouton de la porte n° 5. 

La porte s'ouvrit, et d'Artagnan pénétra dans la chambre. 

Aramis, en surtout noir, le chef accommodé d'une espèce de coiffure ronde et plate qui ne ressemblait pas mal à une calotte, était assis devant une table oblongue couverte de rouleaux de papier et d'énormes in-folio; à sa droite était assis le supérieur des jésuites, et à sa gauche le curé de Montdidier. Les rideaux étaient à demi clos et ne laissaient pénétrer qu'un jour mystérieux, ménagé pour une béate rêverie. Tous les objets mondains qui peuvent frapper l'œil quand on entre dans la chambre d'un jeune homme, et surtout lorsque ce jeune homme est mousquetaire, avaient disparu comme par enchantement; et, de peur sans doute que leur vue ne ramenât son maître aux idées de ce monde, Bazin avait fait main basse sur l'épée, les pistolets, le chapeau à plume, les broderies et les dentelles de tout genre et de toute espèce. 

Mais, en leur lieu et place, d'Artagnan crut apercevoir dans un coin obscur comme une forme de discipline suspendue par un clou à la muraille. 

Au bruit que fit d'Artagnan en ouvrant la porte, Aramis leva la tête et reconnut son ami. Mais, au grand étonnement du jeune homme, sa vue ne parut pas produire une grande impression sur le mousquetaire, tant son esprit était détaché des choses de la terre. 

«Bonjour, cher d'Artagnan, dit Aramis; croyez que je suis heureux de vous voir. 

\speak  Et moi aussi, dit d'Artagnan, quoique je ne sois pas encore bien sûr que ce soit à Aramis que je parle. 

\speak  À lui-même, mon ami, à lui-même; mais qui a pu vous faire douter? 

\speak  J'avais peur de me tromper de chambre, et j'ai cru d'abord entrer dans l'appartement de quelque homme Église; puis une autre erreur m'a pris en vous trouvant en compagnie de ces messieurs: c'est que vous ne fussiez gravement malade.» 

Les deux hommes noirs lancèrent sur d'Artagnan, dont ils comprirent l'intention, un regard presque menaçant; mais d'Artagnan ne s'en inquiéta pas. 

«Je vous trouble peut-être, mon cher Aramis, continua d'Artagnan; car, d'après ce que je vois, je suis porté à croire que vous vous confessez à ces messieurs.» 

Aramis rougit imperceptiblement. 

«Vous, me troubler? oh! bien au contraire, cher ami, je vous le jure; et comme preuve de ce que je dis, permettez-moi de me réjouir en vous voyant sain et sauf. 

\speak  Ah! il y vient enfin! pensa d'Artagnan, ce n'est pas malheureux. 

\speak  Car, monsieur, qui est mon ami, vient d'échapper à un rude danger, continua Aramis avec onction, en montrant de la main d'Artagnan aux deux ecclésiastiques. 

\speak  Louez Dieu, monsieur, répondirent ceux-ci en s'inclinant à l'unisson. 

\speak  Je n'y ai pas manqué, mes révérends, répondit le jeune homme en leur rendant leur salut à son tour. 

\speak  Vous arrivez à propos, cher d'Artagnan, dit Aramis, et vous allez, en prenant part à la discussion, l'éclairer de vos lumières. M. le principal d'Amiens, M. le curé de Montdidier et moi, nous argumentons sur certaines questions théologiques dont l'intérêt nous captive depuis longtemps; je serais charmé d'avoir votre avis. 

\speak  L'avis d'un homme d'épée est bien dénué de poids, répondit d'Artagnan, qui commençait à s'inquiéter de la tournure que prenaient les choses, et vous pouvez vous en tenir, croyez-moi, à la science de ces messieurs.» 

Les deux hommes noirs saluèrent à leur tour. 

«Au contraire, reprit Aramis, et votre avis nous sera précieux; voici de quoi il s'agit: M. le principal croit que ma thèse doit être surtout dogmatique et didactique. 

\speak  Votre thèse! vous faites donc une thèse? 

\speak  Sans doute, répondit le jésuite; pour l'examen qui précède l'ordination, une thèse est de rigueur. 

\speak  L'ordination! s'écria d'Artagnan, qui ne pouvait croire à ce que lui avaient dit successivement l'hôtesse et Bazin,\dots l'ordination!» 

Et il promenait ses yeux stupéfaits sur les trois personnages qu'il avait devant lui. 

«Or», continua Aramis en prenant sur son fauteuil la même pose gracieuse que s'il eût été dans une ruelle et en examinant avec complaisance sa main blanche et potelée comme une main de femme, qu'il tenait en l'air pour en faire descendre le sang: «or, comme vous l'avez entendu, d'Artagnan, M. le principal voudrait que ma thèse fût dogmatique, tandis que je voudrais, moi, qu'elle fût idéale. C'est donc pourquoi M. le principal me proposait ce sujet qui n'a point encore été traité, dans lequel je reconnais qu'il y a matière à de magnifiques développements. 

\textit{«Utraque manus in benedicendo clericis inferioribus necessaria est.»} 

D'Artagnan, dont nous connaissons l'érudition, ne sourcilla pas plus à cette citation qu'à celle que lui avait faite M. de Tréville à propos des présents qu'il prétendait que d'Artagnan avait reçus de M. de Buckingham. 

«Ce qui veut dire, reprit Aramis pour lui donner toute facilité: les deux mains sont indispensables aux prêtres des ordres inférieurs, quand ils donnent la bénédiction. 

\speak  Admirable sujet! s'écria le jésuite. 

\speak  Admirable et dogmatique!» répéta le curé qui, de la force de d'Artagnan à peu près sur le latin, surveillait soigneusement le jésuite pour emboîter le pas avec lui et répéter ses paroles comme un écho. 

Quant à d'Artagnan, il demeura parfaitement indifférent à l'enthousiasme des deux hommes noirs. 

«Oui, admirable! \textit{prorsus admirabile}! continua Aramis, mais qui exige une étude approfondie des Pères et des Écritures. Or j'ai avoué à ces savants ecclésiastiques, et cela en toute humilité, que les veilles des corps de garde et le service du roi m'avaient fait un peu négliger l'étude. Je me trouverai donc plus à mon aise, \textit{facilius natans}, dans un sujet de mon choix, qui serait à ces rudes questions théologiques ce que la morale est à la métaphysique en philosophie.» 

D'Artagnan s'ennuyait profondément, le curé aussi. 

«Voyez quel exorde! s'écria le jésuite. 

\speak  \textit{Exordium}, répéta le curé pour dire quelque chose. 

\speak  \textit{Quemadmodum minter cœlorum immensitatem.}» 

Aramis jeta un coup d'œil de côté sur d'Artagnan, et il vit que son ami bâillait à se démonter la mâchoire. 

«Parlons français, mon père, dit-il au jésuite, M. d'Artagnan goûtera plus vivement nos paroles. 

\speak  Oui, je suis fatigué de la route, dit d'Artagnan, et tout ce latin m'échappe. 

\speak  D'accord, dit le jésuite un peu dépité, tandis que le curé, transporté d'aise, tournait sur d'Artagnan un regard plein de reconnaissance; eh bien, voyez le parti qu'on tirerait de cette glose. 

\speak  Moïse, serviteur de Dieu\dots il n'est que serviteur, entendez-vous bien! Moïse bénit avec les mains; il se fait tenir les deux bras, tandis que les Hébreux battent leurs ennemis; donc il bénit avec les deux mains. D'ailleurs, que dit l'Évangile: \textit{imponite manus}, et non pas \textit{manum}. Imposez les mains, et non pas la main. 

\speak  Imposez les mains, répéta le curé en faisant un geste. 

\speak  À saint Pierre, au contraire, de qui les papes sont successeurs, continua le jésuite: \textit{Porrige digitos}. Présentez les doigts; y êtes-vous maintenant? 

\speak  Certes, répondit Aramis en se délectant, mais la chose est subtile. 

\speak  Les doigts! reprit le jésuite; saint Pierre bénit avec les doigts. Le pape bénit donc aussi avec les doigts. Et avec combien de doigts bénit-il? Avec trois doigts, un pour le Père, un pour le Fils, et un pour le Saint-Esprit.» 

Tout le monde se signa; d'Artagnan crut devoir imiter cet exemple. 

«Le pape est successeur de saint Pierre et représente les trois pouvoirs divins; le reste, \textit{ordines inferiores} de la hiérarchie ecclésiastique, bénit par le nom des saints archanges et des anges. Les plus humbles clercs, tels que nos diacres et sacristains, bénissent avec les goupillons, qui simulent un nombre indéfini de doigts bénissants. Voilà le sujet simplifié, \textit{Argumentum omni denudatum ornamento}. Je ferais avec cela, continua le jésuite, deux volumes de la taille de celui-ci.» 

Et, dans son enthousiasme, il frappait sur le saint Chrysostome in-folio qui faisait plier la table sous son poids. 

D'Artagnan frémit. 

«Certes, dit Aramis, je rends justice aux beautés de cette thèse, mais en même temps je la reconnais écrasante pour moi. J'avais choisi ce texte; dites-moi, cher d'Artagnan, s'il n'est point de votre goût: \textit{Non inutile est desiderium in oblatione}, ou mieux encore: un peu de regret ne messied pas dans une offrande au Seigneur. 

\speak  Halte-là! s'écria le jésuite, car cette thèse frise l'hérésie; il y a une proposition presque semblable dans l'\textit{Augustinus} de l'hérésiarque Jansénius, dont tôt ou tard le livre sera brûlé par les mains du bourreau. Prenez garde! mon jeune ami; vous penchez vers les fausses doctrines, mon jeune ami; vous vous perdrez! 

\speak  Vous vous perdrez, dit le curé en secouant douloureusement la tête. 

\speak  Vous touchez à ce fameux point du libre arbitre, qui est un écueil mortel. Vous abordez de front les insinuations des pélagiens et des demi-pélagiens. 

\speak  Mais, mon révérend\dots, reprit Aramis quelque peu abasourdi de la grêle d'arguments qui lui tombait sur la tête. 

\speak  Comment prouverez-vous, continua le jésuite sans lui donner le temps de parler, que l'on doit regretter le monde lorsqu'on s'offre à Dieu? écoutez ce dilemme: Dieu est Dieu, et le monde est le diable. Regretter le monde, c'est regretter le diable: voilà ma conclusion. 

\speak  C'est la mienne aussi, dit le curé. 

\speak  Mais de grâce!\dots dit Aramis. 

\speak  \textit{Desideras diabolum}, infortuné! s'écria le jésuite. 

\speak  Il regrette le diable! Ah! mon jeune ami, reprit le curé en gémissant, ne regrettez pas le diable, c'est moi qui vous en supplie.» 

D'Artagnan tournait à l'idiotisme; il lui semblait être dans une maison de fous, et qu'il allait devenir fou comme ceux qu'il voyait. Seulement il était forcé de se taire, ne comprenant point la langue qui se parlait devant lui. 

«Mais écoutez-moi donc, reprit Aramis avec une politesse sous laquelle commençait à percer un peu d'impatience, je ne dis pas que je regrette; non, je ne prononcerai jamais cette phrase qui ne serait pas orthodoxe\dots» 

Le jésuite leva les bras au ciel, et le curé en fit autant. 

«Non, mais convenez au moins qu'on a mauvaise grâce de n'offrir au Seigneur que ce dont on est parfaitement dégoûté. Ai-je raison, d'Artagnan? 

\speak  Je le crois pardieu bien!» s'écria celui-ci. 

Le curé et le jésuite firent un bond sur leur chaise. 

«Voici mon point de départ, c'est un syllogisme: le monde ne manque pas d'attraits, je quitte le monde, donc je fais un sacrifice; or l'Écriture dit positivement: Faites un sacrifice au Seigneur. 

\speak  Cela est vrai, dirent les antagonistes. 

\speak  Et puis, continua Aramis en se pinçant l'oreille pour la rendre rouge, comme il se secouait les mains pour les rendre blanches, et puis j'ai fait certain rondeau là-dessus que je communiquai à M. Voiture l'an passé, et duquel ce grand homme m'a fait mille compliments. 

\speak  Un rondeau! fit dédaigneusement le jésuite. 

\speak  Un rondeau! dit machinalement le curé. 

\speak  Dites, dites, s'écria d'Artagnan, cela nous changera quelque peu. 

\speak  Non, car il est religieux, répondit Aramis, et c'est de la théologie en vers. 

\speak  Diable! fit d'Artagnan. 

\speak  Le voici, dit Aramis d'un petit air modeste qui n'était pas exempt d'une certaine teinte d'hypocrisie: 

\begin{verse}
<Vous qui pleurez un passé plein de charmes,\\
Et qui trainez des jours infortunés,\\
Tous vos malheurs se verront terminés,\\
Quand à Dieu seul vous offrirez vos larmes,\\
Vous qui pleurez!>
\end{verse}

D'Artagnan et le curé parurent flattés. Le jésuite persista dans son opinion. 

\begin{conversation}
Gardez-vous du goût profane dans le style théologique. Que dit en effet saint Augustin? \textit{Severus sit clericorum sermo}. 

\speak Oui, que le sermon soit clair! dit le curé. 

\speak Or, se hâta d'interrompre le jésuite en voyant que son acolyte se fourvoyait, or votre thèse plaira aux dames, voilà tout; elle aura le succès d'une plaidoirie de maître Patru. 

\speak Plaise à Dieu! s'écria Aramis transporté. 

\speak Vous le voyez, s'écria le jésuite, le monde parle encore en vous à haute voix, \textit{altissima voce}. Vous suivez le monde, mon jeune ami, et je tremble que la grâce ne soit point efficace. 

\speak Rassurez-vous, mon révérend, je réponds de moi. 

\speak Présomption mondaine! 

\speak Je me connais, mon père, ma résolution est irrévocable. 

\speak Alors vous vous obstinez à poursuivre cette thèse? 

\speak Je me sens appelé à traiter celle-là, et non pas une autre; je vais donc la continuer, et demain j'espère que vous serez satisfait des corrections que j'y aurai faites d'après vos avis. 

\speak Travaillez lentement, dit le curé, nous vous laissons dans des dispositions excellentes. 

\speak Oui, le terrain est tout ensemencé, dit le jésuite, et nous n'avons pas à craindre qu'une partie du grain soit tombée sur la pierre, l'autre le long du chemin, et que les oiseaux du ciel aient mangé le reste, \textit{aves cœli comederunt illam}. 

\speak Que la peste t'étouffe avec ton latin! dit d'Artagnan, qui se sentait au bout de ses forces. 

\speak Adieu, mon fils, dit le curé, à demain. 

\speak À demain, jeune téméraire, dit le jésuite; vous promettez d'être une des lumières de l'Église; veuille le Ciel que cette lumière ne soit pas un feu dévorant.
\end{conversation}

D'Artagnan, qui pendant une heure s'était rongé les ongles d'impatience, commençait à attaquer la chair. 

Les deux hommes noirs se levèrent, saluèrent Aramis et d'Artagnan, et s'avancèrent vers la porte. Bazin, qui s'était tenu debout et qui avait écouté toute cette controverse avec une pieuse jubilation, s'élança vers eux, prit le bréviaire du curé, le missel du jésuite, et marcha respectueusement devant eux pour leur frayer le chemin. 

Aramis les conduisit jusqu'au bas de l'escalier et remonta aussitôt près de d'Artagnan qui rêvait encore. 

Restés seuls, les deux amis gardèrent d'abord un silence embarrassé; cependant il fallait que l'un des deux le rompît le premier, et comme d'Artagnan paraissait décidé à laisser cet honneur à son ami: 

«Vous le voyez, dit Aramis, vous me trouvez revenu à mes idées fondamentales. 

\speak  Oui, la grâce efficace vous a touché, comme disait ce monsieur tout à l'heure. 

\speak  Oh! ces plans de retraite sont formés depuis longtemps; et vous m'en avez déjà ouï parler, n'est-ce pas, mon ami? 

\speak  Sans doute, mais je vous avoue que j'ai cru que vous plaisantiez. 

\speak  Avec ces sortes de choses! Oh! d'Artagnan! 

\speak  Dame! on plaisante bien avec la mort. 

\speak  Et l'on a tort, d'Artagnan: car la mort, c'est la porte qui conduit à la perdition ou au salut. 

\speak  D'accord; mais, s'il vous plaît, ne théologisons pas, Aramis; vous devez en avoir assez pour le reste de la journée: quant à moi, j'ai à peu près oublié le peu de latin que je n'ai jamais su; puis, je vous l'avouerai, je n'ai rien mangé depuis ce matin dix heures, et j'ai une faim de tous les diables. 

\speak  Nous dînerons tout à l'heure, cher ami; seulement, vous vous rappellerez que c'est aujourd'hui vendredi; or, dans un pareil jour, je ne puis ni voir, ni manger de la chair. Si vous voulez vous contenter de mon dîner, il se compose de tétragones cuits et de fruits. 

\speak  Qu'entendez-vous par tétragones? demanda d'Artagnan avec inquiétude. 

\speak  J'entends des épinards, reprit Aramis, mais pour vous j'ajouterai des oeufs, et c'est une grave infraction à la règle, car les oeufs sont viande, puisqu'ils engendrent le poulet. 

\speak  Ce festin n'est pas succulent, mais n'importe; pour rester avec vous, je le subirai. 

\speak  Je vous suis reconnaissant du sacrifice, dit Aramis; mais s'il ne profite pas à votre corps, il profitera, soyez-en certain, à votre âme. 

\speak  Ainsi, décidément, Aramis, vous entrez en religion. Que vont dire nos amis, que va dire M. de Tréville? Ils vous traiteront de déserteur, je vous en préviens. 

\speak  Je n'entre pas en religion, j'y rentre. C'est Église que j'avais désertée pour le monde, car vous savez que je me suis fait violence pour prendre la casaque de mousquetaire. 

\speak  Moi, je n'en sais rien. 

\speak  Vous ignorez comment j'ai quitté le séminaire? 

\speak  Tout à fait. 

\speak  Voici mon histoire; d'ailleurs les Écritures disent: «Confessez-vous les uns aux autres», et je me confesse à vous, d'Artagnan. 

\speak  Et moi, je vous donne l'absolution d'avance, vous voyez que je suis bon homme. 

\speak  Ne plaisantez pas avec les choses saintes, mon ami. 

\speak  Alors, dites, je vous écoute. 

\speak  J'étais donc au séminaire depuis l'âge de neuf ans, j'en avais vingt dans trois jours, j'allais être abbé, et tout était dit. Un soir que je me rendais, selon mon habitude, dans une maison que je fréquentais avec plaisir --- on est jeune, que voulez-vous! on est faible, --- un officier qui me voyait d'un œil jaloux lire les vies des saints à la maîtresse de la maison, entra tout à coup et sans être annoncé. Justement, ce soir-là, j'avais traduit un épisode de Judith, et je venais de communiquer mes vers à la dame qui me faisait toutes sortes de compliments, et, penchée sur mon épaule, les relisait avec moi. La pose, qui était quelque peu abandonnée, je l'avoue, blessa cet officier; il ne dit rien, mais lorsque je sortis, il sortit derrière moi, et me rejoignant: 

«--- Monsieur l'abbé, dit-il, aimez-vous les coups de canne? 

«--- Je ne puis le dire, monsieur, répondis-je, personne n'ayant jamais osé m'en donner. 

«--- Eh bien, écoutez-moi, monsieur l'abbé, si vous retournez dans la maison où je vous ai rencontré ce soir, j'oserai, moi.» 

«Je crois que j'eus peur, je devins fort pâle, je sentis les jambes qui me manquaient, je cherchai une réponse que je ne trouvai pas, je me tus. 

«L'officier attendait cette réponse, et voyant qu'elle tardait, il se mit à rire, me tourna le dos et rentra dans la maison. Je rentrai au séminaire. 

«Je suis bon gentilhomme et j'ai le sang vif, comme vous avez pu le remarquer, mon cher d'Artagnan; l'insulte était terrible, et, tout inconnue qu'elle était restée au monde, je la sentais vivre et remuer au fond de mon cœur. Je déclarai à mes supérieurs que je ne me sentais pas suffisamment préparé pour l'ordination, et, sur ma demande, on remit la cérémonie à un an. 

«J'allai trouver le meilleur maître d'armes de Paris, je fis condition avec lui pour prendre une leçon d'escrime chaque jour, et chaque jour, pendant une année, je pris cette leçon. Puis, le jour anniversaire de celui où j'avais été insulté, j'accrochai ma soutane à un clou, je pris un costume complet de cavalier, et je me rendis à un bal que donnait une dame de mes amies, et où je savais que devait se trouver mon homme. C'était rue des Francs-Bourgeois, tout près de la Force. 

«En effet, mon officier y était; je m'approchai de lui, comme il chantait un lai d'amour en regardant tendrement une femme, et je l'interrompis au beau milieu du second couplet. 

«--- Monsieur, lui dis-je, vous déplaît-il toujours que je retourne dans certaine maison de la rue Payenne, et me donnerez-vous encore des coups de canne, s'il me prend fantaisie de vous désobéir?» 

«L'officier me regarda avec étonnement, puis il dit: 

«--- Que me voulez-vous, monsieur? Je ne vous connais pas. 

«--- Je suis, répondis-je, le petit abbé qui lit les vies des saints et qui traduit Judith en vers. 

«--- Ah! ah! je me rappelle, dit l'officier en goguenardant; que me voulez-vous? 

«--- Je voudrais que vous eussiez le loisir de venir faire un tour de promenade avec moi. 

«--- Demain matin, si vous le voulez bien, et ce sera avec le plus grand plaisir. 

«--- Non, pas demain matin, s'il vous plaît, tout de suite. 

«--- Si vous l'exigez absolument\dots 

«--- Mais oui, je l'exige. 

«--- Alors, sortons. Mesdames, dit l'officier, ne vous dérangez pas. Le temps de tuer monsieur seulement, et je reviens vous achever le dernier couplet.» 

«Nous sortîmes. 

«Je le menai rue Payenne, juste à l'endroit où un an auparavant, heure pour heure, il m'avait fait le compliment que je vous ai rapporté. Il faisait un clair de lune superbe. Nous mîmes l'épée à la main, et à la première passe, je le tuai roide. 

\speak  Diable! fit d'Artagnan. 

\speak  Or, continua Aramis, comme les dames ne virent pas revenir leur chanteur, et qu'on le trouva rue Payenne avec un grand coup d'épée au travers du corps, on pensa que c'était moi qui l'avait accommodé ainsi, et la chose fit scandale. Je fus donc pour quelque temps forcé de renoncer à la soutane. Athos, dont je fis la connaissance à cette époque, et Porthos, qui m'avait, en dehors de mes leçons d'escrime, appris quelques bottes gaillardes, me décidèrent à demander une casaque de mousquetaire. Le roi avait fort aimé mon père, tué au siège d'Arras, et l'on m'accorda cette casaque. Vous comprenez donc qu'aujourd'hui le moment est venu pour moi de rentrer dans le sein de l'église 

\speak  Et pourquoi aujourd'hui plutôt qu'hier et que demain? Que vous est-il donc arrivé aujourd'hui, qui vous donne de si méchantes idées? 

\speak  Cette blessure, mon cher d'Artagnan, m'a été un avertissement du Ciel. 

\speak  Cette blessure? bah! elle est à peu près guérie, et je suis sûr qu'aujourd'hui ce n'est pas celle-là qui vous fait le plus souffrir. 

\speak  Et laquelle? demanda Aramis en rougissant. 

\speak  Vous en avez une au cœur, Aramis, une plus vive et plus sanglante, une blessure faite par une femme.» 

L'œil d'Aramis étincela malgré lui. 

«Ah! dit-il en dissimulant son émotion sous une feinte négligence, ne parlez pas de ces choses-là; moi, penser à ces choses-là! avoir des chagrins d'amour? \textit{Vanitas vanitatum}! Me serais-je donc, à votre avis, retourné la cervelle, et pour qui? pour quelque grisette, pour quelque fille de chambre, à qui j'aurais fait la cour dans une garnison, fi! 

\speak  Pardon, mon cher Aramis, mais je croyais que vous portiez vos visées plus haut. 

\speak  Plus haut? et que suis-je pour avoir tant d'ambition? un pauvre mousquetaire fort gueux et fort obscur, qui hait les servitudes et se trouve grandement déplacé dans le monde! 

\speak  Aramis, Aramis! s'écria d'Artagnan en regardant son ami avec un air de doute. 

\speak  Poussière, je rentre dans la poussière. La vie est pleine d'humiliations et de douleurs, continua-t-il en s'assombrissant; tous les fils qui la rattachent au bonheur se rompent tour à tour dans la main de l'homme, surtout les fils d'or. O mon cher d'Artagnan! reprit Aramis en donnant à sa voix une légère teinte d'amertume, croyez-moi, cachez bien vos plaies quand vous en aurez. Le silence est la dernière joie des malheureux; gardez-vous de mettre qui que ce soit sur la trace de vos douleurs, les curieux pompent nos larmes comme les mouches font du sang d'un daim blessé. 

\speak  Hélas, mon cher Aramis, dit d'Artagnan en poussant à son tour un profond soupir, c'est mon histoire à moi-même que vous faites là. 

\speak  Comment? 

\speak  Oui, une femme que j'aimais, que j'adorais, vient de m'être enlevée de force. Je ne sais pas où elle est, où on l'a conduite; elle est peut-être prisonnière, elle est peut-être morte. 

\speak  Mais vous avez au moins la consolation de vous dire qu'elle ne vous a pas quitté volontairement; que si vous n'avez point de ses nouvelles, c'est que toute communication avec vous lui est interdite, tandis que\dots 

\speak  Tandis que\dots 

\speak  Rien, reprit Aramis, rien. 

\speak  Ainsi, vous renoncez à jamais au monde, c'est un parti pris, une résolution arrêtée? 

\speak  À tout jamais. Vous êtes mon ami aujourd'hui, demain vous ne serez plus pour moi qu'une ombre; où plutôt même, vous n'existerez plus. Quant au monde, c'est un sépulcre et pas autre chose. 

\speak  Diable! c'est fort triste ce que vous me dites là. 

\speak  Que voulez-vous! ma vocation m'attire, elle m'enlève. 

D'Artagnan sourit et ne répondit point. Aramis continua: 

«Et cependant, tandis que je tiens encore à la terre j'eusse voulu vous parler de vous, de nos amis. 

\speak  Et moi, dit d'Artagnan, j'eusse voulu vous parler de vous-même, mais je vous vois si détaché de tout; les amours, vous en faites fi; les amis sont des ombres, le monde est un sépulcre. 

\speak  Hélas! vous le verrez par vous-même, dit Aramis avec un soupir. 

\speak  N'en parlons donc plus, dit d'Artagnan, et brûlons cette lettre qui, sans doute, vous annonçait quelque nouvelle infidélité de votre grisette ou de votre fille de chambre. 

\speak  Quelle lettre? s'écria vivement Aramis. 

\speak  Une lettre qui était venue chez vous en votre absence et qu'on m'a remise pour vous. 

\speak  Mais de qui cette lettre? 

\speak  Ah! de quelque suivante éplorée, de quelque grisette au désespoir; la fille de chambre de Mme de Chevreuse peut-être, qui aura été obligée de retourner à Tours avec sa maîtresse, et qui, pour se faire pimpante, aura pris du papier parfumé et aura cacheté sa lettre avec une couronne de duchesse. 

\speak  Que dites-vous là? 

\speak  Tiens, je l'aurai perdue! dit sournoisement le jeune homme en faisant semblant de chercher. Heureusement que le monde est un sépulcre, que les hommes et par conséquent les femmes sont des ombres, que l'amour est un sentiment dont vous faites fi! 

\speak  Ah! d'Artagnan, d'Artagnan! s'écria Aramis, tu me fais mourir! 

\speak  Enfin, la voici!» dit d'Artagnan. 

Et il tira la lettre de sa poche. 

Aramis fit un bond, saisit la lettre, la lut ou plutôt la dévora, son visage rayonnait. 

«Il paraît que la suivante à un beau style, dit nonchalamment le messager. 

\speak  Merci, d'Artagnan! s'écria Aramis presque en délire. Elle a été forcée de retourner à Tours; elle ne m'est pas infidèle, elle m'aime toujours. Viens, mon ami, viens que je t'embrasse, le bonheur m'étouffe!» 

Et les deux amis se mirent à danser autour du vénérable saint Chrysostome, piétinant bravement les feuillets de la thèse qui avaient roulé sur le parquet. 

En ce moment, Bazin entrait avec les épinards et l'omelette. 

«Fuis, malheureux! s'écria Aramis en lui jetant sa calotte au visage; retourne d'où tu viens, remporte ces horribles légumes et cet affreux entremets! demande un lièvre piqué, un chapon gras, un gigot à l'ail et quatre bouteilles de vieux bourgogne.» 

Bazin, qui regardait son maître et qui ne comprenait rien à ce changement, laissa mélancoliquement glisser l'omelette dans les épinards, et les épinards sur le parquet. 

«Voilà le moment de consacrer votre existence au Roi des Rois, dit d'Artagnan, si vous tenez à lui faire une politesse: \textit{Non inutile desiderium in oblatione}. 

\speak  Allez-vous-en au diable avec votre latin! Mon cher d'Artagnan, buvons, morbleu, buvons frais, buvons beaucoup, et racontez-moi un peu ce qu'on fait là-bas.»
%!TeX root=../musketeersfr.tex 

\chapter{La Femme D'Athos} 
	\lettrine[ante=«]{I}{l} reste maintenant à savoir des nouvelles d'Athos, dit d'Artagnan au fringant Aramis, quand il l'eut mis au courant de ce qui s'était passé dans la capitale depuis leur départ, et qu'un excellent dîner leur eut fait oublier à l'un sa thèse, à l'autre sa fatigue. 

\speak  Croyez-vous donc qu'il lui soit arrivé malheur? demanda Aramis. Athos est si froid, si brave et manie si habilement son épée. 

\speak  Oui, sans doute, et personne ne reconnaît mieux que moi le courage et l'adresse d'Athos, mais j'aime mieux sur mon épée le choc des lances que celui des bâtons, je crains qu'Athos n'ait été étrillé par de la valetaille, les valets sont gens qui frappent fort et ne finissent pas tôt. Voilà pourquoi, je vous l'avoue, je voudrais repartir le plus tôt possible. 

\speak  Je tâcherai de vous accompagner, dit Aramis, quoique je ne me sente guère en état de monter à cheval. Hier, j'essayai de la discipline que vous voyez sur ce mur et la douleur m'empêcha de continuer ce pieux exercice. 

\speak  C'est qu'aussi, mon cher ami, on n'a jamais vu essayer de guérir un coup d'escopette avec des coups de martinet; mais vous étiez malade, et la maladie rend la tête faible, ce qui fait que je vous excuse. 

\speak  Et quand partez-vous? 

\speak  Demain, au point du jour; reposez-vous de votre mieux cette nuit, et demain, si vous le pouvez, nous partirons ensemble. 

\speak  À demain donc, dit Aramis; car tout de fer que vous êtes, vous devez avoir besoin de repos.» 

Le lendemain, lorsque d'Artagnan entra chez Aramis, il le trouva à sa fenêtre. 

«Que regardez-vous donc là? demanda d'Artagnan. 

\speak  Ma foi! J'admire ces trois magnifiques chevaux que les garçons d'écurie tiennent en bride; c'est un plaisir de prince que de voyager sur de pareilles montures. 

\speak  Eh bien, mon cher Aramis, vous vous donnerez ce plaisir-là, car l'un de ces chevaux est à vous. 

\speak  Ah! bah, et lequel? 

\speak  Celui des trois que vous voudrez: je n'ai pas de préférence. 

\speak  Et le riche caparaçon qui le couvre est à moi aussi? 

\speak  Sans doute. 

\speak  Vous voulez rire, d'Artagnan. 

\speak  Je ne ris plus depuis que vous parlez français. 

\speak  C'est pour moi, ces fontes dorées, cette housse de velours, cette selle chevillée d'argent? 

\speak  À vous-même, comme le cheval qui piaffe est à moi, comme cet autre cheval qui caracole est à Athos. 

\speak  Peste! ce sont trois bêtes superbes. 

\speak  Je suis flatté qu'elles soient de votre goût. 

\speak  C'est donc le roi qui vous a fait ce cadeau-là? 

\speak  À coup sûr, ce n'est point le cardinal, mais ne vous inquiétez pas d'où ils viennent, et songez seulement qu'un des trois est votre propriété. 

\speak  Je prends celui que tient le valet roux. 

\speak  À merveille! 

\speak  Vive Dieu! s'écria Aramis, voilà qui me fait passer le reste de ma douleur; je monterais là-dessus avec trente balles dans le corps. Ah! sur mon âme, les beaux étriers! Holà! Bazin, venez çà, et à l'instant même.» 

Bazin apparut, morne et languissant, sur le seuil de la porte. 

«Fourbissez mon épée, redressez mon feutre, brossez mon manteau, et chargez mes pistolets! dit Aramis. 

\speak  Cette dernière recommandation est inutile, interrompit d'Artagnan: il y a des pistolets chargés dans vos fontes.» 

Bazin soupira. 

«Allons, maître Bazin, tranquillisez-vous, dit d'Artagnan; on gagne le royaume des cieux dans toutes les conditions. 

\speak  Monsieur était déjà si bon théologien! dit Bazin presque larmoyant; il fût devenu évêque et peut-être cardinal. 

\speak  Eh bien, mon pauvre Bazin, voyons, réfléchis un peu; à quoi sert d'être homme d'Église, je te prie? on n'évite pas pour cela d'aller faire la guerre; tu vois bien que le cardinal va faire la première campagne avec le pot en tête et la pertuisane au poing; et M. de Nogaret de La Valette, qu'en dis-tu? il est cardinal aussi, demande à son laquais combien de fois il lui a fait de la charpie. 

\speak  Hélas! soupira Bazin, je le sais, monsieur, tout est bouleversé dans le monde aujourd'hui.» 

Pendant ce temps, les deux jeunes gens et le pauvre laquais étaient descendus. 

«Tiens-moi l'étrier, Bazin», dit Aramis. 

Et Aramis s'élança en selle avec sa grâce et sa légèreté ordinaire; mais après quelques voltes et quelques courbettes du noble animal, son cavalier ressentit des douleurs tellement insupportables, qu'il pâlit et chancela. D'Artagnan qui, dans la prévision de cet accident, ne l'avait pas perdu des yeux, s'élança vers lui, le retint dans ses bras et le conduisit à sa chambre. 

«C'est bien, mon cher Aramis, soignez-vous, dit-il, j'irai seul à la recherche d'Athos. 

\speak  Vous êtes un homme d'airain, lui dit Aramis. 

\speak  Non, j'ai du bonheur, voilà tout, mais comment allez-vous vivre en m'attendant? plus de thèse, plus de glose sur les doigts et les bénédictions, hein?» 

Aramis sourit. 

«Je ferai des vers, dit-il. 

\speak  Oui, des vers parfumés à l'odeur du billet de la suivante de Mme de Chevreuse. Enseignez donc la prosodie à Bazin, cela le consolera. Quant au cheval, montez-le tous les jours un peu, et cela vous habituera aux manoeuvres. 

\speak  Oh! pour cela, soyez tranquille, dit Aramis, vous me retrouverez prêt à vous suivre.» 

Ils se dirent adieu et, dix minutes après, d'Artagnan, après avoir recommandé son ami à Bazin et à l'hôtesse, trottait dans la direction d'Amiens. 

Comment allait-il retrouver Athos, et même le retrouverait-il? 

La position dans laquelle il l'avait laissé était critique; il pouvait bien avoir succombé. Cette idée, en assombrissant son front, lui arracha quelques soupirs et lui fit formuler tout bas quelques serments de vengeance. De tous ses amis, Athos était le plus âgé, et partant le moins rapproché en apparence de ses goûts et de ses sympathies. 

Cependant il avait pour ce gentilhomme une préférence marquée. L'air noble et distingué d'Athos, ces éclairs de grandeur qui jaillissaient de temps en temps de l'ombre où il se tenait volontairement enfermé, cette inaltérable égalité d'humeur qui en faisait le plus facile compagnon de la terre, cette gaieté forcée et mordante, cette bravoure qu'on eût appelée aveugle si elle n'eût été le résultat du plus rare sang-froid, tant de qualités attiraient plus que l'estime, plus que l'amitié de d'Artagnan, elles attiraient son admiration. 

En effet, considéré même auprès de M. de Tréville, l'élégant et noble courtisan, Athos, dans ses jours de belle humeur, pouvait soutenir avantageusement la comparaison; il était de taille moyenne, mais cette taille était si admirablement prise et si bien proportionnée, que, plus d'une fois, dans ses luttes avec Porthos, il avait fait plier le géant dont la force physique était devenue proverbiale parmi les mousquetaires; sa tête, aux yeux perçants, au nez droit, au menton dessiné comme celui de Brutus, avait un caractère indéfinissable de grandeur et de grâce; ses mains, dont il ne prenait aucun soin, faisaient le désespoir d'Aramis, qui cultivait les siennes à grand renfort de pâte d'amandes et d'huile parfumée; le son de sa voix était pénétrant et mélodieux tout à la fois, et puis, ce qu'il y avait d'indéfinissable dans Athos, qui se faisait toujours obscur et petit, c'était cette science délicate du monde et des usages de la plus brillante société, cette habitude de bonne maison qui perçait comme à son insu dans ses moindres actions. 

S'agissait-il d'un repas, Athos l'ordonnait mieux qu'aucun homme du monde, plaçant chaque convive à la place et au rang que lui avaient faits ses ancêtres ou qu'il s'était faits lui-même. S'agissait-il de science héraldique, Athos connaissait toutes les familles nobles du royaume, leur généalogie, leurs alliances, leurs armes et l'origine de leurs armes. L'étiquette n'avait pas de minuties qui lui fussent étrangères, il savait quels étaient les droits des grands propriétaires, il connaissait à fond la vénerie et la fauconnerie, et un jour il avait, en causant de ce grand art, étonné le roi Louis XIII lui-même, qui cependant y était passé maître. 

Comme tous les grands seigneurs de cette époque, il montait à cheval et faisait des armes dans la perfection. Il y a plus: son éducation avait été si peu négligée, même sous le rapport des études scolastiques, si rares à cette époque chez les gentilshommes, qu'il souriait aux bribes de latin que détachait Aramis, et qu'avait l'air de comprendre Porthos; deux ou trois fois même, au grand étonnement de ses amis, il lui était arrivé, lorsque Aramis laissait échapper quelque erreur de rudiment, de remettre un verbe à son temps et un nom à son cas. En outre, sa probité était inattaquable, dans ce siècle où les hommes de guerre transigeaient si facilement avec leur religion et leur conscience, les amants avec la délicatesse rigoureuse de nos jours, et les pauvres avec le septième commandement de Dieu. C'était donc un homme fort extraordinaire qu'Athos. 

Et cependant, on voyait cette nature si distinguée, cette créature si belle, cette essence si fine, tourner insensiblement vers la vie matérielle, comme les vieillards tournent vers l'imbécillité physique et morale. Athos, dans ses heures de privation, et ces heures étaient fréquentes, s'éteignait dans toute sa partie lumineuse, et son côté brillant disparaissait comme dans une profonde nuit. 

Alors, le demi-dieu évanoui, il restait à peine un homme. La tête basse, l'œil terne, la parole lourde et pénible, Athos regardait pendant de longues heures soit sa bouteille et son verre, soit Grimaud, qui, habitué à lui obéir par signes, lisait dans le regard atone de son maître jusqu'à son moindre désir, qu'il satisfaisait aussitôt. La réunion des quatre amis avait-elle lieu dans un de ces moments-là, un mot, échappé avec un violent effort, était tout le contingent qu'Athos fournissait à la conversation. En échange, Athos à lui seul buvait comme quatre, et cela sans qu'il y parût autrement que par un froncement de sourcil plus indiqué et par une tristesse plus profonde. 

D'Artagnan, dont nous connaissons l'esprit investigateur et pénétrant, n'avait, quelque intérêt qu'il eût à satisfaire sa curiosité sur ce sujet, pu encore assigner aucune cause à ce marasme, ni en noter les occurrences. Jamais Athos ne recevait de lettres, jamais Athos ne faisait aucune démarche qui ne fût connue de tous ses amis. 

On ne pouvait dire que ce fût le vin qui lui donnât cette tristesse, car au contraire il ne buvait que pour combattre cette tristesse, que ce remède, comme nous l'avons dit, rendait plus sombre encore. On ne pouvait attribuer cet excès d'humeur noire au jeu, car, au contraire de Porthos, qui accompagnait de ses chants ou de ses jurons toutes les variations de la chance, Athos, lorsqu'il avait gagné, demeurait aussi impassible que lorsqu'il avait perdu. On l'avait vu, au cercle des mousquetaires, gagner un soir trois mille pistoles, les perdre jusqu'au ceinturon brodé d'or des jours de gala; regagner tout cela, plus cent louis, sans que son beau sourcil noir eût haussé ou baissé d'une demi-ligne, sans que ses mains eussent perdu leur nuance nacrée, sans que sa conversation, qui était agréable ce soir-là, eût cessé d'être calme et agréable. 

Ce n'était pas non plus, comme chez nos voisins les Anglais, une influence atmosphérique qui assombrissait son visage, car cette tristesse devenait plus intense en général vers les beaux jours de l'année; juin et juillet étaient les mois terribles d'Athos. 

Pour le présent, il n'avait pas de chagrin, il haussait les épaules quand on lui parlait de l'avenir; son secret était donc dans le passé, comme on l'avait dit vaguement à d'Artagnan. 

Cette teinte mystérieuse répandue sur toute sa personne rendait encore plus intéressant l'homme dont jamais les yeux ni la bouche, dans l'ivresse la plus complète, n'avaient rien révélé, quelle que fût l'adresse des questions dirigées contre lui. 

«Eh bien, pensait d'Artagnan, le pauvre Athos est peut-être mort à cette heure, et mort par ma faute, car c'est moi qui l'ai entraîné dans cette affaire, dont il ignorait l'origine, dont il ignorera le résultat et dont il ne devait tirer aucun profit. 

\speak  Sans compter, monsieur, répondait Planchet, que nous lui devons probablement la vie. Vous rappelez-vous comme il a crié: “Au large, d'Artagnan! je suis pris.” Et après avoir déchargé ses deux pistolets, quel bruit terrible il faisait avec son épée! On eût dit vingt hommes, ou plutôt vingt diables enragés!» 

Et ces mots redoublaient l'ardeur de d'Artagnan, qui excitait son cheval, lequel n'ayant pas besoin d'être excité emportait son cavalier au galop. 

Vers onze heures du matin, on aperçut Amiens; à onze heures et demie, on était à la porte de l'auberge maudite. 

D'Artagnan avait souvent médité contre l'hôte perfide une de ces bonnes vengeances qui consolent, rien qu'en espérance. Il entra donc dans l'hôtellerie, le feutre sur les yeux, la main gauche sur le pommeau de l'épée et faisant siffler sa cravache de la main droite. 

«Me reconnaissez-vous? dit-il à l'hôte, qui s'avançait pour le saluer. 

\speak  Je n'ai pas cet honneur, Monseigneur, répondit celui-ci les yeux encore éblouis du brillant équipage avec lequel d'Artagnan se présentait. 

\speak  Ah! vous ne me connaissez pas! 

\speak  Non, Monseigneur. 

\speak  Eh bien, deux mots vont vous rendre la mémoire. Qu'avez-vous fait de ce gentilhomme à qui vous eûtes l'audace, voici quinze jours passés à peu près, d'intenter une accusation de fausse monnaie?» 

L'hôte pâlit, car d'Artagnan avait pris l'attitude la plus menaçante, et Planchet se modelait sur son maître. 

«Ah! Monseigneur, ne m'en parlez pas, s'écria l'hôte de son ton de voix le plus larmoyant; ah! Seigneur, combien j'ai payé cette faute! Ah! malheureux que je suis! 

\speak  Ce gentilhomme, vous dis-je, qu'est-il devenu? 

\speak  Daignez m'écouter, Monseigneur, et soyez clément. Voyons, asseyez-vous, par grâce!» 

D'Artagnan, muet de colère et d'inquiétude, s'assit, menaçant comme un juge. Planchet s'adossa fièrement à son fauteuil. 

«Voici l'histoire, Monseigneur, reprit l'hôte tout tremblant, car je vous reconnais à cette heure; c'est vous qui êtes parti quand j'eus ce malheureux démêlé avec ce gentilhomme dont vous parlez. 

\speak  Oui, c'est moi; ainsi vous voyez bien que vous n'avez pas de grâce à attendre si vous ne dites pas toute la vérité. 

\speak  Aussi veuillez m'écouter, et vous la saurez tout entière. 

\speak  J'écoute. 

\speak  J'avais été prévenu par les autorités qu'un faux-monnayeur célèbre arriverait à mon auberge avec plusieurs de ses compagnons, tous déguisés sous le costume de gardes ou de mousquetaires. Vos chevaux, vos laquais, votre figure, Messeigneurs, tout m'avait été dépeint. 

\speak  Après, après? dit d'Artagnan, qui reconnut bien vite d'où venait le signalement si exactement donné. 

\speak  Je pris donc, d'après les ordres de l'autorité, qui m'envoya un renfort de six hommes, telles mesures que je crus urgentes afin de m'assurer de la personne des prétendus faux-monnayeurs. 

\speak  Encore! dit d'Artagnan, à qui ce mot de faux-monnayeur échauffait terriblement les oreilles. 

\speak  Pardonnez-moi, Monseigneur, de dire de telles choses, mais elles sont justement mon excuse. L'autorité m'avait fait peur, et vous savez qu'un aubergiste doit ménager l'autorité. 

\speak  Mais encore une fois, ce gentilhomme, où est-il? qu'est-il devenu? Est-il mort? est-il vivant? 

\speak  Patience, Monseigneur, nous y voici. Il arriva donc ce que vous savez, et dont votre départ précipité, ajouta l'hôte avec une finesse qui n'échappa point à d'Artagnan, semblait autoriser l'issue. Ce gentilhomme votre ami se défendit en désespéré. Son valet, qui, par un malheur imprévu, avait cherché querelle aux gens de l'autorité, déguisés en garçons d'écurie\dots 

\speak  Ah! misérable! s'écria d'Artagnan, vous étiez tous d'accord, et je ne sais à quoi tient que je ne vous extermine tous! 

\speak  Hélas! non, Monseigneur, nous n'étions pas tous d'accord, et vous l'allez bien voir. Monsieur votre ami (pardon de ne point l'appeler par le nom honorable qu'il porte sans doute, mais nous ignorons ce nom), monsieur votre ami, après avoir mis hors de combat deux hommes de ses deux coups de pistolet, battit en retraite en se défendant avec son épée dont il estropia encore un de mes hommes, et d'un coup du plat de laquelle il m'étourdit. 

\speak  Mais, bourreau, finiras-tu? dit d'Artagnan. Athos, que devient Athos? 

\speak  En battant en retraite, comme j'ai dit à Monseigneur, il trouva derrière lui l'escalier de la cave, et comme la porte était ouverte, il tira la clef à lui et se barricada en dedans. Comme on était sûr de le retrouver là, on le laissa libre. 

\speak  Oui, dit d'Artagnan, on ne tenait pas tout à fait à le tuer, on ne cherchait qu'à l'emprisonner. 

\speak  Juste Dieu! à l'emprisonner, Monseigneur? il s'emprisonna bien lui-même, je vous le jure. D'abord il avait fait de rude besogne, un homme était tué sur le coup et deux autres étaient blessés grièvement. Le mort et les deux blessés furent emportés par leurs camarades, et jamais je n'ai plus entendu parler ni des uns, ni des autres. Moi-même, quand je repris mes sens, j'allai trouver M. le gouverneur, auquel je racontai tout ce qui s'était passé, et auquel je demandai ce que je devais faire du prisonnier. Mais M. le gouverneur eut l'air de tomber des nues; il me dit qu'il ignorait complètement ce que je voulais dire, que les ordres qui m'étaient parvenus n'émanaient pas de lui et que si j'avais le malheur de dire à qui que ce fût qu'il était pour quelque chose dans toute cette échauffourée, il me ferait pendre. Il paraît que je m'étais trompé, monsieur, que j'avais arrêté l'un pour l'autre, et que celui qu'on devait arrêter était sauvé. 

\speak  Mais Athos? s'écria d'Artagnan, dont l'impatience se doublait de l'abandon où l'autorité laissait la chose; Athos, qu'est-il devenu? 

\speak  Comme j'avais hâte de réparer mes torts envers le prisonnier, reprit l'aubergiste, je m'acheminai vers la cave afin de lui rendre sa liberté. Ah! monsieur, ce n'était plus un homme, c'était un diable. À cette proposition de liberté, il déclara que c'était un piège qu'on lui tendait et qu'avant de sortir il entendait imposer ses conditions. Je lui dis bien humblement, car je ne me dissimulais pas la mauvaise position où je m'étais mis en portant la main sur un mousquetaire de Sa Majesté, je lui dis que j'étais prêt à me soumettre à ses conditions. 

«--- D'abord, dit-il, je veux qu'on me rende mon valet tout armé.» 

«On s'empressa d'obéir à cet ordre; car vous comprenez bien, monsieur, que nous étions disposés à faire tout ce que voudrait votre ami. M. Grimaud (il a dit ce nom, celui-là, quoiqu'il ne parle pas beaucoup), M. Grimaud fut donc descendu à la cave, tout blessé qu'il était; alors, son maître l'ayant reçu, rebarricada la porte et nous ordonna de rester dans notre boutique. 

\speak  Mais enfin, s'écria d'Artagnan, où est-il? où est Athos? 

\speak  Dans la cave, monsieur. 

\speak  Comment, malheureux, vous le retenez dans la cave depuis ce temps-là? 

\speak  Bonté divine! Non, monsieur. Nous, le retenir dans la cave! vous ne savez donc pas ce qu'il y fait, dans la cave! Ah! si vous pouviez l'en faire sortir, monsieur, je vous en serais reconnaissant toute ma vie, vous adorerais comme mon patron. 

\speak  Alors il est là, je le retrouverai là? 

\speak  Sans doute, monsieur, il s'est obstiné à y rester. Tous les jours, on lui passe par le soupirail du pain au bout d'une fourche, et de la viande quand il en demande; mais, hélas! ce n'est pas de pain et de viande qu'il fait la plus grande consommation. Une fois, j'ai essayé de descendre avec deux de mes garçons, mais il est entré dans une terrible fureur. J'ai entendu le bruit de ses pistolets qu'il armait et de son mousqueton qu'armait son domestique. Puis, comme nous leur demandions quelles étaient leurs intentions, le maître a répondu qu'ils avaient quarante coups à tirer lui et son laquais, et qu'ils les tireraient jusqu'au dernier plutôt que de permettre qu'un seul de nous mît le pied dans la cave. Alors, monsieur, j'ai été me plaindre au gouverneur, lequel m'a répondu que je n'avais que ce que je méritais, et que cela m'apprendrait à insulter les honorables seigneurs qui prenaient gîte chez moi. 

\speak  De sorte que, depuis ce temps?\dots reprit d'Artagnan ne pouvant s'empêcher de rire de la figure piteuse de son hôte. 

\speak  De sorte que, depuis ce temps, monsieur, continua celui-ci, nous menons la vie la plus triste qui se puisse voir; car, monsieur, il faut que vous sachiez que toutes nos provisions sont dans la cave; il y a notre vin en bouteilles et notre vin en pièce, la bière, l'huile et les épices, le lard et les saucissons; et comme il nous est défendu d'y descendre, nous sommes forcés de refuser le boire et le manger aux voyageurs qui nous arrivent, de sorte que tous les jours notre hôtellerie se perd. Encore une semaine avec votre ami dans ma cave, et nous sommes ruinés. 

\speak  Et ce sera justice, drôle. Ne voyait-on pas bien, à notre mine, que nous étions gens de qualité et non faussaires, dites? 

\speak  Oui, monsieur, oui, vous avez raison, dit l'hôte. Mais tenez, tenez, le voilà qui s'emporte. 

\speak  Sans doute qu'on l'aura troublé, dit d'Artagnan. 

\speak  Mais il faut bien qu'on le trouble, s'écria l'hôte; il vient de nous arriver deux gentilshommes anglais. 

\speak  Eh bien? 

\speak  Eh bien, les Anglais aiment le bon vin, comme vous savez, monsieur; ceux-ci ont demandé du meilleur. Ma femme alors aura sollicité de M. Athos la permission d'entrer pour satisfaire ces messieurs; et il aura refusé comme de coutume. Ah! bonté divine! voilà le sabbat qui redouble!» 

D'Artagnan, en effet, entendit mener un grand bruit du côté de la cave; il se leva et, précédé de l'hôte qui se tordait les mains, et suivi de Planchet qui tenait son mousqueton tout armé, il s'approcha du lieu de la scène. 

Les deux gentilshommes étaient exaspérés, ils avaient fait une longue course et mouraient de faim et de soif. 

«Mais c'est une tyrannie, s'écriaient-ils en très bon français, quoique avec un accent étranger, que ce maître fou ne veuille pas laisser à ces bonnes gens l'usage de leur vin. Ça, nous allons enfoncer la porte, et s'il est trop enragé, eh bien! nous le tuerons. 

\speak  Tout beau, messieurs! dit d'Artagnan en tirant ses pistolets de sa ceinture; vous ne tuerez personne, s'il vous plaît. 

\speak  Bon, bon, disait derrière la porte la voix calme d'Athos, qu'on les laisse un peu entrer, ces mangeurs de petits enfants, et nous allons voir.» 

Tout braves qu'ils paraissaient être, les deux gentilshommes anglais se regardèrent en hésitant; on eût dit qu'il y avait dans cette cave un de ces ogres faméliques, gigantesques héros des légendes populaires, et dont nul ne force impunément la caverne. 

Il y eut un moment de silence; mais enfin les deux Anglais eurent honte de reculer, et le plus hargneux des deux descendit les cinq ou six marches dont se composait l'escalier et donna dans la porte un coup de pied à fendre une muraille. 

«Planchet, dit d'Artagnan en armant ses pistolets, je me charge de celui qui est en haut, charge-toi de celui qui est en bas. Ah! messieurs! vous voulez de la bataille! eh bien! on va vous en donner! 

\speak  Mon Dieu, s'écria la voix creuse d'Athos, j'entends d'Artagnan, ce me semble. 

\speak  En effet, dit d'Artagnan en haussant la voix à son tour, c'est moi-même, mon ami. 

\speak  Ah! bon! alors, dit Athos, nous allons les travailler, ces enfonceurs de portes.» 

Les gentilshommes avaient mis l'épée à la main, mais ils se trouvaient pris entre deux feux; ils hésitèrent un instant encore; mais, comme la première fois, l'orgueil l'emporta, et un second coup de pied fit craquer la porte dans toute sa hauteur. 

«Range-toi, d'Artagnan, range-toi, cria Athos, range-toi, je vais tirer. 

\speak  Messieurs, dit d'Artagnan, que la réflexion n'abandonnait jamais, messieurs, songez-y! De la patience, Athos. Vous vous engagez là dans une mauvaise affaire, et vous allez être criblés. Voici mon valet et moi qui vous lâcherons trois coups de feu, autant vous arriveront de la cave; puis nous aurons encore nos épées, dont, je vous assure, mon ami et moi nous jouons passablement. Laissez-moi faire vos affaires et les miennes. Tout à l'heure vous aurez à boire, je vous en donne ma parole. 

\speak  S'il en reste», grogna la voix railleuse d'Athos. 

L'hôtelier sentit une sueur froide couler le long de son échine. 

«Comment, s'il en reste! murmura-t-il. 

\speak  Que diable! il en restera, reprit d'Artagnan; soyez donc tranquille, à eux deux ils n'auront pas bu toute la cave. Messieurs, remettez vos épées au fourreau. 

\speak  Eh bien, vous, remettez vos pistolets à votre ceinture. 

\speak  Volontiers.» 

Et d'Artagnan donna l'exemple. Puis, se retournant vers Planchet, il lui fit signe de désarmer son mousqueton. 

Les Anglais, convaincus, remirent en grommelant leurs épées au fourreau. On leur raconta l'histoire de l'emprisonnement d'Athos. Et comme ils étaient bons gentilshommes, ils donnèrent tort à l'hôtelier. 

«Maintenant, messieurs, dit d'Artagnan, remontez chez vous, et, dans dix minutes, je vous réponds qu'on vous y portera tout ce que vous pourrez désirer.» 

Les Anglais saluèrent et sortirent. 

«Maintenant que je suis seul, mon cher Athos, dit d'Artagnan, ouvrez-moi la porte, je vous en prie. 

\speak  À l'instant même», dit Athos. 

Alors on entendit un grand bruit de fagots entrechoqués et de poutres gémissantes: c'étaient les contrescarpes et les bastions d'Athos, que l'assiégé démolissait lui-même. 

Un instant après, la porte s'ébranla, et l'on vit paraître la tête pâle d'Athos qui, d'un coup d'œil rapide, explorait les environs. 

D'Artagnan se jeta à son cou et l'embrassa tendrement puis il voulut l'entraîner hors de ce séjour humide, alors il s'aperçut qu'Athos chancelait. 

«Vous êtes blessé? lui dit-il. 

\speak  Moi! pas le moins du monde; je suis ivre mort, voilà tout, et jamais homme n'a mieux fait ce qu'il fallait pour cela. Vive Dieu! mon hôte, il faut que j'en aie bu au moins pour ma part cent cinquante bouteilles. 

\speak  Miséricorde! s'écria l'hôte, si le valet en a bu la moitié du maître seulement, je suis ruiné. 

\speak  Grimaud est un laquais de bonne maison, qui ne se serait pas permis le même ordinaire que moi; il a bu à la pièce seulement; tenez, je crois qu'il a oublié de remettre le fosset. Entendez-vous? cela coule.» 

D'Artagnan partit d'un éclat de rire qui changea le frisson de l'hôte en fièvre chaude. 

En même temps, Grimaud parut à son tour derrière son maître, le mousqueton sur l'épaule, la tête tremblante, comme ces satyres ivres des tableaux de Rubens. Il était arrosé par-devant et par-derrière d'une liqueur grasse que l'hôte reconnut pour être sa meilleure huile d'olive. 

Le cortège traversa la grande salle et alla s'installer dans la meilleure chambre de l'auberge, que d'Artagnan occupa d'autorité. 

Pendant ce temps, l'hôte et sa femme se précipitèrent avec des lampes dans la cave, qui leur avait été si longtemps interdite et où un affreux spectacle les attendait. 

Au-delà des fortifications auxquelles Athos avait fait brèche pour sortir et qui se composaient de fagots, de planches et de futailles vides entassées selon toutes les règles de l'art stratégique, on voyait çà et là, nageant dans les mares d'huile et de vin, les ossements de tous les jambons mangés, tandis qu'un amas de bouteilles cassées jonchait tout l'angle gauche de la cave et qu'un tonneau, dont le robinet était resté ouvert, perdait par cette ouverture les dernières gouttes de son sang. L'image de la dévastation et de la mort, comme dit le poète de l'Antiquité, régnait là comme sur un champ de bataille. 

Sur cinquante saucissons, pendus aux solives, dix restaient à peine. 

Alors les hurlements de l'hôte et de l'hôtesse percèrent la voûte de la cave, d'Artagnan lui-même en fut ému. Athos ne tourna pas même la tête. 

Mais à la douleur succéda la rage. L'hôte s'arma d'une broche et, dans son désespoir, s'élança dans la chambre où les deux amis s'étaient retirés. 

«Du vin! dit Athos en apercevant l'hôte. 

\speak  Du vin! s'écria l'hôte stupéfait, du vin! mais vous m'en avez bu pour plus de cent pistoles; mais je suis un homme ruiné, perdu, anéanti! 

\speak  Bah! dit Athos, nous sommes constamment restés sur notre soif. 

\speak  Si vous vous étiez contentés de boire, encore; mais vous avez cassé toutes les bouteilles. 

\speak  Vous m'avez poussé sur un tas qui a dégringolé. C'est votre faute. 

\speak  Toute mon huile est perdue! 

\speak  L'huile est un baume souverain pour les blessures, et il fallait bien que ce pauvre Grimaud pansât celles que vous lui avez faites. 

\speak  Tous mes saucissons rongés! 

\speak  Il y a énormément de rats dans cette cave. 

\speak  Vous allez me payer tout cela, cria l'hôte exaspéré. 

\speak  Triple drôle!» dit Athos en se soulevant. Mais il retomba aussitôt; il venait de donner la mesure de ses forces. D'Artagnan vint à son secours en levant sa cravache. 

L'hôte recula d'un pas et se mit à fondre en larmes. 

«Cela vous apprendra, dit d'Artagnan, à traiter d'une façon plus courtoise les hôtes que Dieu vous envoie. 

\speak  Dieu\dots, dites le diable! 

\speak  Mon cher ami, dit d'Artagnan, si vous nous rompez encore les oreilles, nous allons nous renfermer tous les quatre dans votre cave, et nous verrons si véritablement le dégât est aussi grand que vous le dites. 

\speak  Eh bien, oui, messieurs, dit l'hôte, j'ai tort, je l'avoue; mais à tout péché miséricorde; vous êtes des seigneurs et je suis un pauvre aubergiste, vous aurez pitié de moi. 

\speak  Ah! si tu parles comme cela, dit Athos, tu vas me fendre le cœur, et les larmes vont couler de mes yeux comme le vin coulait de tes futailles. On n'est pas si diable qu'on en a l'air. Voyons, viens ici et causons.» 

L'hôte s'approcha avec inquiétude. 

«Viens, te dis-je, et n'aie pas peur, continua Athos. Au moment où j'allais te payer, j'avais posé ma bourse sur la table. 

\speak  Oui, Monseigneur. 

\speak  Cette bourse contenait soixante pistoles, où est-elle? 

\speak  Déposée au greffe, Monseigneur: on avait dit que c'était de la fausse monnaie. 

\speak  Eh bien, fais-toi rendre ma bourse, et garde les soixante pistoles. 

\speak  Mais Monseigneur sait bien que le greffe ne lâche pas ce qu'il tient. Si c'était de la fausse monnaie, il y aurait encore de l'espoir; mais malheureusement ce sont de bonnes pièces. 

\speak  Arrange-toi avec lui, mon brave homme, cela ne me regarde pas, d'autant plus qu'il ne me reste pas une livre. 

\speak  Voyons, dit d'Artagnan, l'ancien cheval d'Athos, où est-il? 

\speak  À l'écurie. 

\speak  Combien vaut-il? 

\speak  Cinquante pistoles tout au plus. 

\speak  Il en vaut quatre-vingts; prends-le, et que tout soit dit. 

\speak  Comment! tu vends mon cheval, dit Athos, tu vends mon Bajazet? et sur quoi ferai-je la campagne? sur Grimaud? 

\speak  Je t'en amène un autre, dit d'Artagnan. 

\speak  Un autre? 

\speak  Et magnifique! s'écria l'hôte. 

\speak  Alors, s'il y en a un autre plus beau et plus jeune, prends le vieux, et à boire! 

\speak  Duquel? demanda l'hôte tout à fait rasséréné. 

\speak  De celui qui est au fond, près des lattes; il en reste encore vingt-cinq bouteilles, toutes les autres ont été cassées dans ma chute. Montez-en six. 

\speak  Mais c'est un foudre que cet homme! dit l'hôte à part lui; s'il reste seulement quinze jours ici, et qu'il paie ce qu'il boira, je rétablirai mes affaires. 

\speak  Et n'oublie pas, continua d'Artagnan, de monter quatre bouteilles du pareil aux deux seigneurs anglais. 

\speak  Maintenant, dit Athos, en attendant qu'on nous apporte du vin, conte-moi, d'Artagnan, ce que sont devenus les autres; voyons.» 

D'Artagnan lui raconta comment il avait trouvé Porthos dans son lit avec une foulure, et Aramis à une table entre les deux théologiens. Comme il achevait, l'hôte rentra avec les bouteilles demandées et un jambon qui, heureusement pour lui, était resté hors de la cave. 

«C'est bien, dit Athos en remplissant son verre et celui de d'Artagnan, voilà pour Porthos et pour Aramis; mais vous, mon ami, qu'avez-vous et que vous est-il arrivé personnellement? Je vous trouve un air sinistre. 

\speak  Hélas! dit d'Artagnan, c'est que je suis le plus malheureux de nous tous, moi! 

\speak  Toi malheureux, d'Artagnan! dit Athos. Voyons, comment es-tu malheureux? Dis-moi cela. 

\speak  Plus tard, dit d'Artagnan. 

\speak  Plus tard! et pourquoi plus tard? parce que tu crois que je suis ivre, d'Artagnan? Retiens bien ceci: je n'ai jamais les idées plus nettes que dans le vin. Parle donc, je suis tout oreilles.» 

D'Artagnan raconta son aventure avec Mme Bonacieux. 

Athos l'écouta sans sourciller; puis, lorsqu'il eut fini: 

«Misères que tout cela, dit Athos, misères!» 

C'était le mot d'Athos. 

«Vous dites toujours misères! mon cher Athos, dit d'Artagnan; cela vous sied bien mal, à vous qui n'avez jamais aimé.» 

L'œil mort d'Athos s'enflamma soudain, mais ce ne fut qu'un éclair, il redevint terne et vague comme auparavant. 

«C'est vrai, dit-il tranquillement, je n'ai jamais aimé, moi. 

\speak  Vous voyez bien alors, cœur de pierre, dit d'Artagnan, que vous avez tort d'être dur pour nous autres cœurs tendres. 

\speak  cœurs tendres, cœurs percés, dit Athos. 

\speak  Que dites-vous? 

\speak  Je dis que l'amour est une loterie où celui qui gagne, gagne la mort! Vous êtes bien heureux d'avoir perdu, croyez-moi, mon cher d'Artagnan. Et si j'ai un conseil à vous donner, c'est de perdre toujours. 

\speak  Elle avait l'air de si bien m'aimer! 

\speak  Elle en avait l'air. 

\speak  Oh! elle m'aimait. 

\speak  Enfant! il n'y a pas un homme qui n'ait cru comme vous que sa maîtresse l'aimait, et il n'y a pas un homme qui n'ait été trompé par sa maîtresse. 

\speak  Excepté vous, Athos, qui n'en avez jamais eu. 

\speak  C'est vrai, dit Athos après un moment de silence, je n'en ai jamais eu, moi. Buvons! 

\speak  Mais alors, philosophe que vous êtes, dit d'Artagnan, instruisez-moi, soutenez-moi; j'ai besoin de savoir et d'être consolé. 

\speak  Consolé de quoi? 

\speak  De mon malheur. 

\speak  Votre malheur fait rire, dit Athos en haussant les épaules; je serais curieux de savoir ce que vous diriez si je vous racontais une histoire d'amour. 

\speak  Arrivée à vous? 

\speak  Ou à un de mes amis, qu'importe! 

\speak  Dites, Athos, dites. 

\speak  Buvons, nous ferons mieux. 

\speak  Buvez et racontez. 

\speak  Au fait, cela se peut, dit Athos en vidant et remplissant son verre, les deux choses vont à merveille ensemble. 

\speak  J'écoute», dit d'Artagnan. 

Athos se recueillit, et, à mesure qu'il se recueillait, d'Artagnan le voyait pâlir; il en était à cette période de l'ivresse où les buveurs vulgaires tombent et dorment. Lui, il rêvait tout haut sans dormir. Ce somnambulisme de l'ivresse avait quelque chose d'effrayant. 

«Vous le voulez absolument? demanda-t-il. 

\speak  Je vous en prie, dit d'Artagnan. 

\speak  Qu'il soit fait donc comme vous le désirez. Un de mes amis, un de mes amis, entendez-vous bien! pas moi, dit Athos en s'interrompant avec un sourire sombre; un des comtes de ma province, c'est-à-dire du Berry, noble comme un Dandolo ou un Montmorency, devint amoureux à vingt-cinq ans d'une jeune fille de seize, belle comme les amours. À travers la naïveté de son âge perçait un esprit ardent, un esprit non pas de femme, mais de poète; elle ne plaisait pas, elle enivrait; elle vivait dans un petit bourg, près de son frère qui était curé. Tous deux étaient arrivés dans le pays: ils venaient on ne savait d'où; mais en la voyant si belle et en voyant son frère si pieux, on ne songeait pas à leur demander d'où ils venaient. Du reste, on les disait de bonne extraction. Mon ami, qui était le seigneur du pays, aurait pu la séduire ou la prendre de force, à son gré, il était le maître; qui serait venu à l'aide de deux étrangers, de deux inconnus? Malheureusement il était honnête homme, il l'épousa. Le sot, le niais, l'imbécile! 

\speak  Mais pourquoi cela, puisqu'il l'aimait? demanda d'Artagnan. 

\speak  Attendez donc, dit Athos. Il l'emmena dans son château, et en fit la première dame de sa province; et il faut lui rendre justice, elle tenait parfaitement son rang. 

\speak  Eh bien? demanda d'Artagnan. 

\speak  Eh bien, un jour qu'elle était à la chasse avec son mari, continua Athos à voix basse et en parlant fort vite, elle tomba de cheval et s'évanouit; le comte s'élança à son secours, et comme elle étouffait dans ses habits, il les fendit avec son poignard et lui découvrit l'épaule. Devinez ce qu'elle avait sur l'épaule, d'Artagnan? dit Athos avec un grand éclat de rire. 

\speak  Puis-je le savoir? demanda d'Artagnan. 

\speak  Une fleur de lis, dit Athos. Elle était marquée!» 

Et Athos vida d'un seul trait le verre qu'il tenait à la main. 

«Horreur! s'écria d'Artagnan, que me dites-vous là? 

\speak  La vérité. Mon cher, l'ange était un démon. La pauvre fille avait volé. 

\speak  Et que fit le comte? 

\speak  Le comte était un grand seigneur, il avait sur ses terres droit de justice basse et haute: il acheva de déchirer les habits de la comtesse, il lui lia les mains derrière le dos et la pendit à un arbre. 

\speak  Ciel! Athos! un meurtre! s'écria d'Artagnan. 

\speak  Oui, un meurtre, pas davantage, dit Athos pâle comme la mort. Mais on me laisse manquer de vin, ce me semble.» 

Et Athos saisit au goulot la dernière bouteille qui restait, l'approcha de sa bouche et la vida d'un seul trait, comme il eût fait d'un verre ordinaire. 

Puis il laissa tomber sa tête sur ses deux mains; d'Artagnan demeura devant lui, saisi d'épouvante. 

«Cela m'a guéri des femmes belles, poétiques et amoureuses, dit Athos en se relevant et sans songer à continuer l'apologue du comte. Dieu vous en accorde autant! Buvons! 

\speak  Ainsi elle est morte? balbutia d'Artagnan. 

\speak  Parbleu! dit Athos. Mais tendez votre verre. Du jambon, drôle, cria Athos, nous ne pouvons plus boire! 

\speak  Et son frère? ajouta timidement d'Artagnan. 

\speak  Son frère? reprit Athos. 

\speak  Oui, le prêtre? 

\speak  Ah! je m'en informai pour le faire pendre à son tour; mais il avait pris les devants, il avait quitté sa cure depuis la veille. 

\speak  A-t-on su au moins ce que c'était que ce misérable? 

\speak  C'était sans doute le premier amant et le complice de la belle, un digne homme qui avait fait semblant d'être curé peut-être pour marier sa maîtresse et lui assurer un sort. Il aura été écartelé, je l'espère. 

\speak  Oh! mon Dieu! mon Dieu! fit d'Artagnan, tout étourdi de cette horrible aventure. 

\speak  Mangez donc de ce jambon, d'Artagnan, il est exquis, dit Athos en coupant une tranche qu'il mit sur l'assiette du jeune homme. Quel malheur qu'il n'y en ait pas eu seulement quatre comme celui-là dans la cave! j'aurais bu cinquante bouteilles de plus.» 

D'Artagnan ne pouvait plus supporter cette conversation, qui l'eût rendu fou; il laissa tomber sa tête sur ses deux mains et fit semblant de s'endormir. 

«Les jeunes gens ne savent plus boire, dit Athos en le regardant en pitié, et pourtant celui-là est des meilleurs!\dots»
%!TeX root=../musketeersfr.tex 

\chapter{Retour} 
	
\lettrine{D}{'Artagnan} était resté étourdi de la terrible confidence d'Athos; cependant bien des choses lui paraissaient encore obscures dans cette demi-révélation; d'abord elle avait été faite par un homme tout à fait ivre à un homme qui l'était à moitié, et cependant, malgré ce vague que fait monter au cerveau la fumée de deux ou trois bouteilles de bourgogne, d'Artagnan, en se réveillant le lendemain matin, avait chaque parole d'Athos aussi présente à son esprit que si, à mesure qu'elles étaient tombées de sa bouche, elles s'étaient imprimées dans son esprit. Tout ce doute ne lui donna qu'un plus vif désir d'arriver à une certitude, et il passa chez son ami avec l'intention bien arrêtée de renouer sa conversation de la veille mais il trouva Athos de sens tout à fait rassis, c'est-à-dire le plus fin et le plus impénétrable des hommes. 

Au reste, le mousquetaire, après avoir échangé avec lui une poignée de main, alla le premier au-devant de sa pensée. 

«J'étais bien ivre hier, mon cher d'Artagnan, dit-il, j'ai senti cela ce matin à ma langue, qui était encore fort épaisse, et à mon pouls qui était encore fort agité; je parie que j'ai dit mille extravagances.» 

Et, en disant ces mots, il regarda son ami avec une fixité qui l'embarrassa. 

«Mais non pas, répliqua d'Artagnan, et, si je me le rappelle bien, vous n'avez rien dit que de fort ordinaire. 

\speak  Ah! vous m'étonnez! Je croyais vous avoir raconté une histoire des plus lamentables.» 

Et il regardait le jeune homme comme s'il eût voulu lire au plus profond de son cœur. 

«Ma foi! dit d'Artagnan, il paraît que j'étais encore plus ivre que vous, puisque je ne me souviens de rien.» 

Athos ne se paya point de cette parole, et il reprit: 

«Vous n'êtes pas sans avoir remarqué, mon cher ami, que chacun a son genre d'ivresse, triste ou gaie, moi j'ai l'ivresse triste, et, quand une fois je suis gris, ma manière est de raconter toutes les histoires lugubres que ma sotte nourrice m'a inculquées dans le cerveau. C'est mon défaut; défaut capital, j'en conviens; mais, à cela près, je suis bon buveur.» 

Athos disait cela d'une façon si naturelle, que d'Artagnan fut ébranlé dans sa conviction. 

«Oh! c'est donc cela, en effet, reprit le jeune homme en essayant de ressaisir la vérité, c'est donc cela que je me souviens, comme, au reste, on se souvient d'un rêve, que nous avons parlé de pendus. 

\speak  Ah! vous voyez bien, dit Athos en pâlissant et cependant en essayant de rire, j'en étais sûr, les pendus sont mon cauchemar, à moi. 

\speak  Oui, oui, reprit d'Artagnan, et voilà la mémoire qui me revient; oui, il s'agissait\dots attendez donc\dots il s'agissait d'une femme. 

\speak  Voyez, répondit Athos en devenant presque livide, c'est ma grande histoire de la femme blonde, et quand je raconte celle-là, c'est que je suis ivre mort. 

\speak  Oui, c'est cela, dit d'Artagnan, l'histoire de la femme blonde, grande et belle, aux yeux bleus. 

\speak  Oui, et pendue. 

\speak  Par son mari, qui était un seigneur de votre connaissance, continua d'Artagnan en regardant fixement Athos. 

\speak  Eh bien, voyez cependant comme on compromettrait un homme quand on ne sait plus ce que l'on dit, reprit Athos en haussant les épaules, comme s'il se fût pris lui-même en pitié. Décidément, je ne veux plus me griser, d'Artagnan, c'est une trop mauvaise habitude.» 

D'Artagnan garda le silence. 

Puis Athos, changeant tout à coup de conversation: 

«À propos, dit-il, je vous remercie du cheval que vous m'avez amené. 

\speak  Est-il de votre goût? demanda d'Artagnan. 

\speak  Oui, mais ce n'était pas un cheval de fatigue. 

\speak  Vous vous trompez; j'ai fait avec lui dix lieues en moins d'une heure et demie, et il n'y paraissait pas plus que s'il eût fait le tour de la place Saint-Sulpice. 

\speak  Ah çà, vous allez me donner des regrets. 

\speak  Des regrets? 

\speak  Oui, je m'en suis défait. 

\speak  Comment cela? 

\speak  Voici le fait: ce matin, je me suis réveillé à six heures, vous dormiez comme un sourd, et je ne savais que faire; j'étais encore tout hébété de notre débauche d'hier; je descendis dans la grande salle, et j'avisai un de nos Anglais qui marchandait un cheval à un maquignon, le sien étant mort hier d'un coup de sang. Je m'approchai de lui, et comme je vis qu'il offrait cent pistoles d'un alezan brûlé: «Par Dieu, lui dis-je, mon gentilhomme, moi aussi j'ai un cheval à vendre. 

«--- Et très beau même, dit-il, je l'ai vu hier, le valet de votre ami le tenait en main. 

«--- Trouvez-vous qu'il vaille cent pistoles? 

«--- Oui, et voulez-vous me le donner pour ce prix-là? 

«--- Non, mais je vous le joue. 

«--- Vous me le jouez? 

«--- Oui. 

«--- À quoi? 

«--- Aux dés.» 

«Ce qui fut dit fut fait; et j'ai perdu le cheval. Ah! mais par exemple, continua Athos, j'ai regagné le caparaçon.» 

D'Artagnan fit une mine assez maussade. 

«Cela vous contrarie? dit Athos. 

\speak  Mais oui, je vous l'avoue, reprit d'Artagnan; ce cheval devait servir à nous faire reconnaître un jour de bataille; c'était un gage, un souvenir. Athos, vous avez eu tort. 

\speak  Eh! mon cher ami, mettez-vous à ma place, reprit le mousquetaire; je m'ennuyais à périr, moi, et puis, d'honneur, je n'aime pas les chevaux anglais. Voyons, s'il ne s'agit que d'être reconnu par quelqu'un, eh bien, la selle suffira; elle est assez remarquable. Quant au cheval, nous trouverons quelque excuse pour motiver sa disparition. Que diable! un cheval est mortel; mettons que le mien a eu la morve ou le farcin.» 

D'Artagnan ne se déridait pas. 

«Cela me contrarie, continua Athos, que vous paraissiez tant tenir à ces animaux, car je ne suis pas au bout de mon histoire. 

\speak  Qu'avez-vous donc fait encore? 

\speak  Après avoir perdu mon cheval, neuf contre dix, voyez le coup, l'idée me vint de jouer le vôtre. 

\speak  Oui, mais vous vous en tîntes, j'espère, à l'idée? 

\speak  Non pas, je la mis à exécution à l'instant même. 

\speak  Ah! par exemple! s'écria d'Artagnan inquiet. 

\speak  Je jouai, et je perdis. 

\speak  Mon cheval? 

\speak  Votre cheval; sept contre huit; faute d'un point\dots, vous connaissez le proverbe. 

\speak  Athos, vous n'êtes pas dans votre bon sens, je vous jure! 

\speak  Mon cher, c'était hier, quand je vous contais mes sottes histoires, qu'il fallait me dire cela, et non pas ce matin. Je le perdis donc avec tous les équipages et harnais possibles. 

\speak  Mais c'est affreux! 

\speak  Attendez donc, vous n'y êtes point, je ferais un joueur excellent, si je ne m'entêtais pas; mais je m'entête, c'est comme quand je bois; je m'entêtai donc\dots 

\speak  Mais que pûtes-vous jouer, il ne vous restait plus rien? 

\speak  Si fait, si fait, mon ami; il nous restait ce diamant qui brille à votre doigt, et que j'avais remarqué hier. 

\speak  Ce diamant! s'écria d'Artagnan, en portant vivement la main à sa bague. 

\speak  Et comme je suis connaisseur, en ayant eu quelques-uns pour mon propre compte, je l'avais estimé mille pistoles. 

\speak  J'espère, dit sérieusement d'Artagnan à demi mort de frayeur, que vous n'avez aucunement fait mention de mon diamant? 

\speak  Au contraire, cher ami; vous comprenez, ce diamant devenait notre seule ressource; avec lui, je pouvais regagner nos harnais et nos chevaux, et, de plus, l'argent pour faire la route. 

\speak  Athos, vous me faites frémir! s'écria d'Artagnan. 

\speak  Je parlai donc de votre diamant à mon partenaire, lequel l'avait aussi remarqué. Que diable aussi, mon cher, vous portez à votre doigt une étoile du ciel, et vous ne voulez pas qu'on y fasse attention! Impossible! 

\speak  Achevez, mon cher; achevez! dit d'Artagnan, car, d'honneur! avec votre sang-froid, vous me faites mourir! 

\speak  Nous divisâmes donc ce diamant en dix parties de cent pistoles chacune. 

\speak  Ah! vous voulez rire et m'éprouver? dit d'Artagnan que la colère commençait à prendre aux cheveux comme Minerve prend Achille, dans \textit{l'Iliade}. 

\speak  Non, je ne plaisante pas, mordieu! j'aurais bien voulu vous y voir, vous! il y avait quinze jours que je n'avais envisagé face humaine et que j'étais là à m'abrutir en m'abouchant avec des bouteilles. 

\speak  Ce n'est point une raison pour jouer mon diamant, cela? répondit d'Artagnan en serrant sa main avec une crispation nerveuse. 

\speak  Écoutez donc la fin; dix parties de cent pistoles chacune en dix coups sans revanche. En treize coups je perdis tout. En treize coups! Le nombre 13 m'a toujours été fatal, c'était le 13 du mois de juillet que\dots 

\speak  Ventrebleu! s'écria d'Artagnan en se levant de table, l'histoire du jour lui faisant oublier celle de la veille. 

\speak  Patience, dit Athos, j'avais un plan. L'Anglais était un original, je l'avais vu le matin causer avec Grimaud, et Grimaud m'avait averti qu'il lui avait fait des propositions pour entrer à son service. Je lui joue Grimaud, le silencieux Grimaud, divisé en dix portions. 

\speak  Ah! pour le coup! dit d'Artagnan éclatant de rire malgré lui. 

\speak  Grimaud lui-même, entendez-vous cela! et avec les dix parts de Grimaud, qui ne vaut pas en tout un ducaton, je regagne le diamant. Dites maintenant que la persistance n'est pas une vertu. 

\speak  Ma foi, c'est très drôle! s'écria d'Artagnan consolé et se tenant les côtes de rire. 

\speak  Vous comprenez que, me sentant en veine, je me remis aussitôt à jouer sur le diamant. 

\speak  Ah! diable, dit d'Artagnan assombri de nouveau. 

\speak  J'ai regagné vos harnais, puis votre cheval, puis mes harnais, puis mon cheval, puis reperdu. Bref, j'ai rattrapé votre harnais, puis le mien. Voilà où nous en sommes. C'est un coup superbe; aussi je m'en suis tenu là.» 

D'Artagnan respira comme si on lui eût enlevé l'hôtellerie de dessus la poitrine. 

«Enfin, le diamant me reste? dit-il timidement. 

\speak  Intact! cher ami; plus les harnais de votre Bucéphale et du mien. 

\speak  Mais que ferons-nous de nos harnais sans chevaux? 

\speak  J'ai une idée sur eux. 

\speak  Athos, vous me faites frémir. 

\speak  Écoutez, vous n'avez pas joué depuis longtemps, vous, d'Artagnan? 

\speak  Et je n'ai point l'envie de jouer. 

\speak  Ne jurons de rien. Vous n'avez pas joué depuis longtemps, disais-je, vous devez donc avoir la main bonne. 

\speak  Eh bien, après? 

\speak  Eh bien, l'Anglais et son compagnon sont encore là. J'ai remarqué qu'ils regrettaient beaucoup les harnais. Vous, vous paraissez tenir à votre cheval. A votre place, je jouerais vos harnais contre votre cheval. 

\speak  Mais il ne voudra pas un seul harnais. 

\speak  Jouez les deux, pardieu! je ne suis point un égoïste comme vous, moi. 

\speak  Vous feriez cela? dit d'Artagnan indécis, tant la confiance d'Athos commençait à le gagner à son insu. 

\speak  Parole d'honneur, en un seul coup. 

\speak  Mais c'est qu'ayant perdu les chevaux, je tenais énormément à conserver les harnais. 

\speak  Jouez votre diamant, alors. 

\speak  Oh! ceci, c'est autre chose; jamais, jamais. 

\speak  Diable! dit Athos, je vous proposerais bien de jouer Planchet; mais comme cela a déjà été fait, l'Anglais ne voudrait peut-être plus. 

\speak  Décidément, mon cher Athos, dit d'Artagnan, j'aime mieux ne rien risquer. 

\speak  C'est dommage, dit froidement Athos, l'Anglais est cousu de pistoles. Eh! mon Dieu, essayez un coup, un coup est bientôt joué. 

\speak  Et si je perds? 

\speak  Vous gagnerez. 

\speak  Mais si je perds? 

\speak  Eh bien, vous donnerez les harnais. 

\speak  Va pour un coup», dit d'Artagnan. 

Athos se mit en quête de l'Anglais et le trouva dans l'écurie, où il examinait les harnais d'un œil de convoitise. L'occasion était bonne. Il fit ses conditions: les deux harnais contre un cheval ou cent pistoles, à choisir. L'Anglais calcula vite: les deux harnais valaient trois cents pistoles à eux deux; il topa. 

D'Artagnan jeta les dés en tremblant et amena le nombre trois; sa pâleur effraya Athos, qui se contenta de dire: 

«Voilà un triste coup, compagnon; vous aurez les chevaux tout harnachés, monsieur.» 

L'Anglais, triomphant, ne se donna même la peine de rouler les dés, il les jeta sur la table sans regarder, tant il était sûr de la victoire; d'Artagnan s'était détourné pour cacher sa mauvaise humeur. 

«Tiens, tiens, tiens, dit Athos avec sa voix tranquille, ce coup de dés est extraordinaire, et je ne l'ai vu que quatre fois dans ma vie: deux as!» 

L'Anglais regarda et fut saisi d'étonnement, d'Artagnan regarda et fut saisi de plaisir. 

«Oui, continua Athos, quatre fois seulement: une fois chez M. de Créquy; une autre fois chez moi, à la campagne, dans mon château de\dots quand j'avais un château; une troisième fois chez M. de Tréville, où il nous surprit tous; enfin une quatrième fois au cabaret, où il échut à moi et où je perdis sur lui cent louis et un souper. 

\speak  Alors, monsieur reprend son cheval, dit l'Anglais. 

\speak  Certes, dit d'Artagnan. 

\speak  Alors il n'y a pas de revanche? 

\speak  Nos conditions disaient: pas de revanche, vous vous le rappelez? 

\speak  C'est vrai; le cheval va être rendu à votre valet, monsieur. 

\speak  Un moment, dit Athos; avec votre permission, monsieur, je demande à dire un mot à mon ami. 

\speak  Dites.» 

Athos tira d'Artagnan à part. 

«Eh bien, lui dit d'Artagnan, que me veux-tu encore, tentateur, tu veux que je joue, n'est-ce pas? 

\speak  Non, je veux que vous réfléchissiez. 

\speak  À quoi? 

\speak  Vous allez reprendre le cheval, n'est-ce pas? 

\speak  Sans doute. 

\speak  Vous avez tort, je prendrais les cent pistoles; vous savez que vous avez joué les harnais contre le cheval ou cent pistoles, à votre choix. 

\speak  Oui. 

\speak  Je prendrais les cent pistoles. 

\speak  Eh bien, moi, je prends le cheval. 

\speak  Et vous avez tort, je vous le répète; que ferons-nous d'un cheval pour nous deux, je ne puis pas monter en croupe; nous aurions l'air des deux fils Aymon qui ont perdu leurs frères; vous ne pouvez pas m'humilier en chevauchant près de moi, en chevauchant sur ce magnifique destrier. Moi, sans balancer un seul instant, je prendrais les cent pistoles, nous avons besoin d'argent pour revenir à Paris. 

\speak  Je tiens à ce cheval, Athos. 

\speak  Et vous avez tort, mon ami; un cheval prend un écart, un cheval bute et se couronne, un cheval mange dans un râtelier où a mangé un cheval morveux: voilà un cheval ou plutôt cent pistoles perdues; il faut que le maître nourrisse son cheval, tandis qu'au contraire cent pistoles nourrissent leur maître. 

\speak  Mais comment reviendrons-nous? 

\speak  Sur les chevaux de nos laquais, pardieu! on verra toujours bien à l'air de nos figures que nous sommes gens de condition. 

\speak  La belle mine que nous aurons sur des bidets, tandis qu'Aramis et Porthos caracoleront sur leurs chevaux! 

\speak  Aramis! Porthos! s'écria Athos, et il se mit à rire. 

\speak  Quoi? demanda d'Artagnan, qui ne comprenait rien à l'hilarité de son ami. 

\speak  Bien, bien, continuons, dit Athos. 

\speak  Ainsi, votre avis\dots? 

\speak  Est de prendre les cent pistoles, d'Artagnan; avec les cent pistoles nous allons festiner jusqu'à la fin du mois; nous avons essuyé des fatigues, voyez-vous, et il sera bon de nous reposer un peu. 

\speak  Me reposer! oh! non, Athos, aussitôt à Paris je me mets à la recherche de cette pauvre femme. 

\speak  Eh bien, croyez-vous que votre cheval vous sera aussi utile pour cela que de bons louis d'or? Prenez les cent pistoles, mon ami, prenez les cent pistoles.» 

D'Artagnan n'avait besoin que d'une raison pour se rendre. Celle-là lui parut excellente. D'ailleurs, en résistant plus longtemps, il craignait de paraître égoïste aux yeux d'Athos; il acquiesça donc et choisit les cent pistoles, que l'Anglais lui compta sur-le-champ. 

Puis l'on ne songea plus qu'à partir. La paix signée avec l'aubergiste, outre le vieux cheval d'Athos, coûta six pistoles; d'Artagnan et Athos prirent les chevaux de Planchet et de Grimaud, les deux valets se mirent en route à pied, portant les selles sur leurs têtes. 

Si mal montés que fussent les deux amis, ils prirent bientôt les devants sur leurs valets et arrivèrent à Crèvecœur. De loin ils aperçurent Aramis mélancoliquement appuyé sur sa fenêtre et regardant, comme \textit{ma soeur Anne}, poudroyer l'horizon. 

«Holà, eh! Aramis! que diable faites-vous donc là? crièrent les deux amis. 

\speak  Ah! c'est vous, d'Artagnan, c'est vous Athos, dit le jeune homme; je songeais avec quelle rapidité s'en vont les biens de ce monde, et mon cheval anglais, qui s'éloignait et qui vient de disparaître au milieu d'un tourbillon de poussière, m'était une vivante image de la fragilité des choses de la terre. La vie elle-même peut se résoudre en trois mots: \textit{Erat, est, fuit}. 

\speak  Cela veut dire au fond? demanda d'Artagnan, qui commençait à se douter de la vérité. 

\speak  Cela veut dire que je viens de faire un marché de dupe: soixante louis, un cheval qui, à la manière dont il file, peut faire au trot cinq lieues à l'heure.» 

D'Artagnan et Athos éclatèrent de rire. 

«Mon cher d'Artagnan, dit Aramis, ne m'en veuillez pas trop, je vous prie: nécessité n'a pas de loi; d'ailleurs je suis le premier puni, puisque cet infâme maquignon m'a volé cinquante louis au moins. Ah! vous êtes bons ménagers, vous autres! vous venez sur les chevaux de vos laquais et vous faites mener vos chevaux de luxe en main, doucement et à petites journées.» 

Au même instant un fourgon, qui depuis quelques instants pointait sur la route d'Amiens, s'arrêta, et l'on vit sortir Grimaud et Planchet leurs selles sur la tête. Le fourgon retournait à vide vers Paris, et les deux laquais s'étaient engagés, moyennant leur transport, à désaltérer le voiturier tout le long de la route. 

«Qu'est-ce que cela? dit Aramis en voyant ce qui se passait; rien que les selles? 

\speak  Comprenez-vous maintenant? dit Athos. 

\speak  Mes amis, c'est exactement comme moi. J'ai conservé le harnais, par instinct. Holà, Bazin! portez mon harnais neuf auprès de celui de ces messieurs. 

\speak  Et qu'avez-vous fait de vos curés? demanda d'Artagnan. 

\speak  Mon cher, je les ai invités à dîner le lendemain, dit Aramis: il y a ici du vin exquis, cela soit dit en passant; je les ai grisés de mon mieux; alors le curé m'a défendu de quitter la casaque, et le jésuite m'a prié de le faire recevoir mousquetaire. 

\speak  Sans thèse! cria d'Artagnan, sans thèse! je demande la suppression de la thèse, moi! 

\speak  Depuis lors, continua Aramis, je vis agréablement. J'ai commencé un poème en vers d'une syllabe; c'est assez difficile, mais le mérite en toutes choses est dans la difficulté. La matière est galante, je vous lirai le premier chant, il a quatre cents vers et dure une minute. 

\speak  Ma foi, mon cher Aramis, dit d'Artagnan, qui détestait presque autant les vers que le latin, ajoutez au mérite de la difficulté celui de la brièveté, et vous êtes sûr au moins que votre poème aura deux mérites. 

\speak  Puis, continua Aramis, il respire des passions honnêtes, vous verrez. Ah çà, mes amis, nous retournons donc à Paris? Bravo, je suis prêt; nous allons donc revoir ce bon Porthos, tant mieux. Vous ne croyez pas qu'il me manquait, ce grand niais-là? Ce n'est pas lui qui aurait vendu son cheval, fût-ce contre un royaume. Je voudrais déjà le voir sur sa bête et sur sa selle. Il aura, j'en suis sûr, l'air du grand mogol.» 

On fit une halte d'une heure pour faire souffler les chevaux; Aramis solda son compte, plaça Bazin dans le fourgon avec ses camarades, et l'on se mit en route pour aller retrouver Porthos. 

On le trouva debout, moins pâle que ne l'avait vu d'Artagnan à sa première visite, et assis à une table où, quoiqu'il fût seul, figurait un dîner de quatre personnes; ce dîner se composait de viandes galamment troussées, de vins choisis et de fruits superbes. 

«Ah! pardieu! dit-il en se levant, vous arrivez à merveille, messieurs, j'en étais justement au potage, et vous allez dîner avec moi. 

\speak  Oh! oh! fit d'Artagnan, ce n'est pas Mousqueton qui a pris au lasso de pareilles bouteilles, puis voilà un fricandeau piqué et un filet de boeuf\dots 

\speak  Je me refais, dit Porthos, je me refais, rien n'affaiblit comme ces diables de foulures; avez-vous eu des foulures, Athos? 

\speak  Jamais; seulement je me rappelle que dans notre échauffourée de la rue Férou je reçus un coup d'épée qui, au bout de quinze ou dix-huit jours, m'avait produit exactement le même effet. 

\speak  Mais ce dîner n'était pas pour vous seul, mon cher Porthos? dit Aramis. 

\speak  Non, dit Porthos; j'attendais quelques gentilshommes du voisinage qui viennent de me faire dire qu'ils ne viendraient pas; vous les remplacerez et je ne perdrai pas au change. Holà, Mousqueton! des sièges, et que l'on double les bouteilles! 

\speak  Savez-vous ce que nous mangeons ici? dit Athos au bout de dix minutes. 

\speak  Pardieu! répondit d'Artagnan, moi je mange du veau piqué aux cardons et à la moelle. 

\speak  Et moi des filets d'agneau, dit Porthos. 

\speak  Et moi un blanc de volaille, dit Aramis. 

\speak  Vous vous trompez tous, messieurs, répondit Athos, vous mangez du cheval. 

\speak  Allons donc! dit d'Artagnan. 

\speak  Du cheval!» fit Aramis avec une grimace de dégoût. 

Porthos seul ne répondit pas. 

«Oui, du cheval; n'est-ce pas, Porthos, que nous mangeons du cheval? Peut-être même les caparaçons avec! 

\speak  Non, messieurs, j'ai gardé le harnais, dit Porthos. 

\speak  Ma foi, nous nous valons tous, dit Aramis: on dirait que nous nous sommes donné le mot. 

\speak  Que voulez-vous, dit Porthos, ce cheval faisait honte à mes visiteurs, et je n'ai pas voulu les humilier! 

\speak  Puis, votre duchesse est toujours aux eaux, n'est-ce pas? reprit d'Artagnan. 

\speak  Toujours, répondit Porthos. Or, ma foi, le gouverneur de la province, un des gentilshommes que j'attendais aujourd'hui à dîner, m'a paru le désirer si fort que je le lui ai donné. 

\speak  Donné! s'écria d'Artagnan. 

\speak  Oh! mon Dieu! oui, donné! c'est le mot, dit Porthos; car il valait certainement cent cinquante louis, et le ladre n'a voulu me le payer que quatre-vingts. 

\speak  Sans la selle? dit Aramis. 

\speak  Oui, sans la selle. 

\speak  Vous remarquerez, messieurs, dit Athos, que c'est encore Porthos qui a fait le meilleur marché de nous tous.» 

Ce fut alors un hourra de rires dont le pauvre Porthos fut tout saisi; mais on lui expliqua bientôt la raison de cette hilarité, qu'il partagea bruyamment selon sa coutume. 

«De sorte que nous sommes tous en fonds? dit d'Artagnan. 

\speak  Mais pas pour mon compte, dit Athos; j'ai trouvé le vin d'Espagne d'Aramis si bon, que j'en ai fait charger une soixantaine de bouteilles dans le fourgon des laquais: ce qui m'a fort désargenté. 

\speak  Et moi, dit Aramis, imaginez donc que j'avais donné jusqu'à mon dernier sou à l'église de Montdidier et aux jésuites d'Amiens; que j'avais pris en outre des engagements qu'il m'a fallu tenir, des messes commandées pour moi et pour vous, messieurs, que l'on dira, messieurs, et dont je ne doute pas que nous ne nous trouvions à merveille. 

\speak  Et moi, dit Porthos, ma foulure, croyez-vous qu'elle ne m'a rien coûté? sans compter la blessure de Mousqueton, pour laquelle j'ai été obligé de faire venir le chirurgien deux fois par jour, lequel m'a fait payer ses visites double sous prétexte que cet imbécile de Mousqueton avait été se faire donner une balle dans un endroit qu'on ne montre ordinairement qu'aux apothicaires; aussi je lui ai bien recommandé de ne plus se faire blesser là. 

\speak  Allons, allons, dit Athos, en échangeant un sourire avec d'Artagnan et Aramis, je vois que vous vous êtes conduit grandement à l'égard du pauvre garçon: c'est d'un bon maître. 

\speak  Bref, continua Porthos, ma dépense payée, il me restera bien une trentaine d'écus. 

\speak  Et à moi une dizaine de pistoles, dit Aramis. 

\speak  Allons, allons, dit Athos, il paraît que nous sommes les Crésus de la société. Combien vous reste-t-il sur vos cent pistoles, d'Artagnan? 

\speak  Sur mes cent pistoles? D'abord, je vous en ai donné cinquante. 

\speak  Vous croyez? 

\speak  Pardieu! --- Ah! c'est vrai, je me rappelle. 

\speak  Puis, j'en ai payé six à l'hôte. 

\speak  Quel animal que cet hôte! pourquoi lui avez-vous donné six pistoles? 

\speak  C'est vous qui m'avez dit de les lui donner. 

\speak  C'est vrai que je suis trop bon. Bref, en reliquat? 

\speak  Vingt-cinq pistoles, dit d'Artagnan. 

\speak  Et moi, dit Athos en tirant quelque menue monnaie de sa poche, moi\dots 

\speak  Vous, rien. 

\speak  Ma foi, ou si peu de chose, que ce n'est pas la peine de rapporter à la masse. 

\speak  Maintenant, calculons combien nous possédons en tout: Porthos? 

\speak  Trente écus. 

\speak  Aramis? 

\speak  Dix pistoles. 

\speak  Et vous, d'Artagnan? 

\speak  Vingt-cinq. 

\speak  Cela fait en tout? dit Athos. 

\speak  Quatre cent soixante-quinze livres! dit d'Artagnan, qui comptait comme Archimède. 

\speak  Arrivés à Paris, nous en aurons bien encore quatre cents, dit Porthos, plus les harnais. 

\speak  Mais nos chevaux d'escadron? dit Aramis. 

\speak  Eh bien, des quatre chevaux des laquais nous en ferons deux de maître que nous tirerons au sort; avec les quatre cents livres, on en fera un demi pour un des démontés, puis nous donnerons les grattures de nos poches à d'Artagnan, qui a la main bonne, et qui ira les jouer dans le premier tripot venu, voilà. 

\speak  Dînons donc, dit Porthos, cela refroidit.» 

Les quatre amis, plus tranquilles désormais sur leur avenir, firent honneur au repas, dont les restes furent abandonnés à MM. Mousqueton, Bazin, Planchet et Grimaud. 

En arrivant à Paris, d'Artagnan trouva une lettre de M. de Tréville qui le prévenait que, sur sa demande, le roi venait de lui accorder la faveur d'entrer dans les mousquetaires. 

Comme c'était tout ce que d'Artagnan ambitionnait au monde, à part bien entendu le désir de retrouver Mme Bonacieux, il courut tout joyeux chez ses camarades, qu'il venait de quitter il y avait une demi-heure, et qu'il trouva fort tristes et fort préoccupés. Ils étaient réunis en conseil chez Athos: ce qui indiquait toujours des circonstances d'une certaine gravité. 

M. de Tréville venait de les faire prévenir que l'intention bien arrêtée de Sa Majesté étant d'ouvrir la campagne le 1er mai, ils eussent à préparer incontinent leurs équipages. 

Les quatre philosophes se regardèrent tout ébahis: M. de Tréville ne plaisantait pas sous le rapport de la discipline. 

«Et à combien estimez-vous ces équipages? dit d'Artagnan. 

\speak  Oh! il n'y a pas à dire, reprit Aramis, nous venons de faire nos comptes avec une lésinerie de Spartiates, et il nous faut à chacun quinze cents livres. 

\speak  Quatre fois quinze font soixante, soit six mille livres, dit Athos. 

\speak  Moi, dit d'Artagnan, il me semble qu'avec mille livres chacun, il est vrai que je ne parle pas en Spartiate, mais en procureur\dots» 

Ce mot de procureur réveilla Porthos. 

«Tiens, j'ai une idée! dit-il. 

\speak  C'est déjà quelque chose: moi, je n'en ai pas même l'ombre, fit froidement Athos, mais quant à d'Artagnan, messieurs, le bonheur d'être désormais des nôtres l'a rendu fou; mille livres! je déclare que pour moi seul il m'en faut deux mille. 

\speak  Quatre fois deux font huit, dit alors Aramis: c'est donc huit mille livres qu'il nous faut pour nos équipages, sur lesquels équipages, il est vrai, nous avons déjà les selles. 

\speak  Plus, dit Athos, en attendant que d'Artagnan qui allait remercier M. de Tréville eût fermé la porte, plus ce beau diamant qui brille au doigt de notre ami. Que diable! d'Artagnan est trop bon camarade pour laisser des frères dans l'embarras, quand il porte à son médius la rançon d'un roi.»
\include{chapters/29.tex}
\include{chapters/30.tex}
%!TeX root=../musketeersfr.tex 

\chapter{Anglais Et Français} 

\lettrine{L}{'heure} venue, on se rendit avec les quatre laquais, derrière le Luxembourg, dans un enclos abandonné aux chèvres. Athos donna une pièce de monnaie au chevrier pour qu'il s'écartât. Les laquais furent chargés de faire sentinelle. 

Bientôt une troupe silencieuse s'approcha du même enclos, y pénétra et joignit les mousquetaires; puis, selon les habitudes d'outre-mer, les présentations eurent lieu. 

Les Anglais étaient tous gens de la plus haute qualité, les noms bizarres de leurs adversaires furent donc pour eux un sujet non seulement de surprise, mais encore d'inquiétude. 

«Mais, avec tout cela, dit Lord de Winter quand les trois amis eurent été nommés, nous ne savons pas qui vous êtes, et nous ne nous battrons pas avec des noms pareils; ce sont des noms de bergers, cela. 

\speak  Aussi, comme vous le supposez bien, Milord, ce sont de faux noms, dit Athos. 

\speak  Ce qui ne nous donne qu'un plus grand désir de connaître les noms véritables, répondit l'Anglais. 

\speak  Vous avez bien joué contre nous sans les connaître, dit Athos, à telles enseignes que vous nous avez gagné nos deux chevaux? 

\speak  C'est vrai, mais nous ne risquions que nos pistoles; cette fois nous risquons notre sang: on joue avec tout le monde, on ne se bat qu'avec ses égaux. 

\speak  C'est juste», dit Athos. Et il prit à l'écart celui des quatre Anglais avec lequel il devait se battre, et lui dit son nom tout bas. 

Porthos et Aramis en firent autant de leur côté. 

«Cela vous suffit-il, dit Athos à son adversaire, et me trouvez-vous assez grand seigneur pour me faire la grâce de croiser l'épée avec moi? 

\speak  Oui, monsieur, dit l'Anglais en s'inclinant. 

\speak  Eh bien, maintenant, voulez-vous que je vous dise une chose? reprit froidement Athos. 

\speak  Laquelle? demanda l'Anglais. 

\speak  C'est que vous auriez aussi bien fait de ne pas exiger que je me fisse connaître. 

\speak  Pourquoi cela? 

\speak  Parce qu'on me croit mort, que j'ai des raisons pour désirer qu'on ne sache pas que je vis, et que je vais être obligé de vous tuer, pour que mon secret ne coure pas les champs.» 

L'Anglais regarda Athos, croyant que celui-ci plaisantait; mais Athos ne plaisantait pas le moins du monde. 

«Messieurs, dit-il en s'adressant à la fois à ses compagnons et à leurs adversaires, y sommes-nous? 

\speak  Oui, répondirent tout d'une voix Anglais et Français. 

\speak  Alors, en garde», dit Athos. 

Et aussitôt huit épées brillèrent aux rayons du soleil couchant, et le combat commença avec un acharnement bien naturel entre gens deux fois ennemis. 

Athos s'escrimait avec autant de calme et de méthode que s'il eût été dans une salle d'armes. 

Porthos, corrigé sans doute de sa trop grande confiance par son aventure de Chantilly, jouait un jeu plein de finesse et de prudence. 

Aramis, qui avait le troisième chant de son poème à finir, se dépêchait en homme très pressé. 

Athos, le premier, tua son adversaire: il ne lui avait porté qu'un coup, mais, comme il l'en avait prévenu, le coup avait été mortel. L'épée lui traversa le cœur. 

Porthos, le second, étendit le sien sur l'herbe: il lui avait percé la cuisse. Alors, comme l'Anglais, sans faire plus longue résistance, lui avait rendu son épée, Porthos le prit dans ses bras et le porta dans son carrosse. 

Aramis poussa le sien si vigoureusement, qu'après avoir rompu une cinquantaine de pas, il finit par prendre la fuite à toutes jambes et disparut aux huées des laquais. 

Quant à d'Artagnan, il avait joué purement et simplement un jeu défensif; puis, lorsqu'il avait vu son adversaire bien fatigué, il lui avait, d'une vigoureuse flanconade, fait sauter son épée. Le baron, se voyant désarmé, fit deux ou trois pas en arrière; mais, dans ce mouvement, son pied glissa, et il tomba à la renverse. 

D'Artagnan fut sur lui d'un seul bond, et lui portant l'épée à la gorge: 

«Je pourrais vous tuer, monsieur, dit-il à l'Anglais, et vous êtes bien entre mes mains, mais je vous donne la vie pour l'amour de votre soeur.» 

D'Artagnan était au comble de la joie; il venait de réaliser le plan qu'il avait arrêté d'avance, et dont le développement avait fait éclore sur son visage les sourires dont nous avons parlé. 

L'Anglais, enchanté d'avoir affaire à un gentilhomme d'aussi bonne composition, serra d'Artagnan entre ses bras, fit mille caresses aux trois mousquetaires, et, comme l'adversaire de Porthos était déjà installé dans la voiture et que celui d'Aramis avait pris la poudre d'escampette, on ne songea plus qu'au défunt. 

Comme Porthos et Aramis le déshabillaient dans l'espérance que sa blessure n'était pas mortelle, une grosse bourse s'échappa de sa ceinture. D'Artagnan la ramassa et la tendit à Lord de Winter. 

«Et que diable voulez-vous que je fasse de cela? dit l'Anglais. 

\speak  Vous la rendrez à sa famille, dit d'Artagnan. 

\speak  Sa famille se soucie bien de cette misère: elle hérite de quinze mille louis de rente: gardez cette bourse pour vos laquais.» 

D'Artagnan mit la bourse dans sa poche. 

«Et maintenant, mon jeune ami, car vous me permettrez, je l'espère, de vous donner ce nom, dit Lord de Winter, dès ce soir, si vous le voulez bien, je vous présenterai à ma soeur, Lady Clarick; car je veux qu'elle vous prenne à son tour dans ses bonnes grâces, et, comme elle n'est point tout à fait mal en cour, peut-être dans l'avenir un mot dit par elle ne vous serait-il point inutile.» 

D'Artagnan rougit de plaisir, et s'inclina en signe d'assentiment. 

Pendant ce temps, Athos s'était approché de d'Artagnan. 

«Que voulez-vous faire de cette bourse? lui dit-il tout bas à l'oreille. 

\speak  Mais je comptais vous la remettre, mon cher Athos. 

\speak  À moi? et pourquoi cela? 

\speak  Dame, vous l'avez tué: ce sont les dépouilles opimes. 

\speak  Moi, héritier d'un ennemi! dit Athos, pour qui donc me prenez-vous? 

\speak  C'est l'habitude à la guerre, dit d'Artagnan; pourquoi ne serait-ce pas l'habitude dans un duel? 

\speak  Même sur le champ de bataille, dit Athos, je n'ai jamais fait cela.» 

Porthos leva les épaules. Aramis, d'un mouvement de lèvres, approuva Athos. 

«Alors, dit d'Artagnan, donnons cet argent aux laquais, comme Lord de Winter nous a dit de le faire. 

\speak  Oui, dit Athos, donnons cette bourse, non à nos laquais, mais aux laquais anglais.» 

Athos prit la bourse, et la jeta dans la main du cocher: 

«Pour vous et vos camarades.» 

Cette grandeur de manières dans un homme entièrement dénué frappa Porthos lui-même, et cette générosité française, redite par Lord de Winter et son ami, eut partout un grand succès, excepté auprès de MM. Grimaud, Mousqueton, Planchet et Bazin. 

Lord de Winter, en quittant d'Artagnan, lui donna l'adresse de sa soeur; elle demeurait place Royale, qui était alors le quartier à la mode, au n° 6. D'ailleurs, il s'engageait à le venir prendre pour le présenter. D'Artagnan lui donna rendez-vous à huit heures, chez Athos. 

Cette présentation à Milady occupait fort la tête de notre Gascon. Il se rappelait de quelle façon étrange cette femme avait été mêlée jusque-là dans sa destinée. Selon sa conviction, c'était quelque créature du cardinal, et cependant il se sentait invinciblement entraîné vers elle, par un de ces sentiments dont on ne se rend pas compte. Sa seule crainte était que Milady ne reconnût en lui l'homme de Meung et de Douvres. Alors, elle saurait qu'il était des amis de M. de Tréville, et par conséquent qu'il appartenait corps et âme au roi, ce qui, dès lors, lui ferait perdre une partie de ses avantages, puisque, connu de Milady comme il la connaissait, il jouerait avec elle à jeu égal. Quant à ce commencement d'intrigue entre elle et le comte de Wardes, notre présomptueux ne s'en préoccupait que médiocrement, bien que le marquis fût jeune, beau, riche et fort avant dans la faveur du cardinal. Ce n'est pas pour rien que l'on a vingt ans, et surtout que l'on est né à Tarbes. 

D'Artagnan commença par aller faire chez lui une toilette flamboyante; puis, il s'en revint chez Athos, et, selon son habitude, lui raconta tout. Athos écouta ses projets; puis il secoua la tête, et lui recommanda la prudence avec une sorte d'amertume. 

«Quoi! lui dit-il, vous venez de perdre une femme que vous disiez bonne, charmante, parfaite, et voilà que vous courez déjà après une autre!» 

D'Artagnan sentit la vérité de ce reproche. 

«J'aimais Mme Bonacieux avec le cœur, tandis que j'aime Milady avec la tête, dit-il; en me faisant conduire chez elle, je cherche surtout à m'éclairer sur le rôle qu'elle joue à la cour. 

\speak  Le rôle qu'elle joue, pardieu! il n'est pas difficile à deviner d'après tout ce que vous m'avez dit. C'est quelque émissaire du cardinal: une femme qui vous attirera dans un piège, où vous laisserez votre tête tout bonnement. 

\speak  Diable! mon cher Athos, vous voyez les choses bien en noir, ce me semble. 

\speak  Mon cher, je me défie des femmes; que voulez-vous! je suis payé pour cela, et surtout des femmes blondes. Milady est blonde, m'avez-vous dit? 

\speak  Elle a les cheveux du plus beau blond qui se puisse voir. 

\speak  Ah! mon pauvre d'Artagnan, fit Athos. 

\speak  Écoutez, je veux m'éclairer; puis, quand je saurai ce que je désire savoir, je m'éloignerai. 

\speak  Éclairez-vous», dit flegmatiquement Athos. 

Lord de Winter arriva à l'heure dite, mais Athos, prévenu à temps, passa dans la seconde pièce. Il trouva donc d'Artagnan seul, et, comme il était près de huit heures, il emmena le jeune homme. 

Un élégant carrosse attendait en bas, et comme il était attelé de deux excellents chevaux, en un instant on fut place Royale. 

Milady Clarick reçut gracieusement d'Artagnan. Son hôtel était d'une somptuosité remarquable; et, bien que la plupart des Anglais, chassés par la guerre, quittassent la France, ou fussent sur le point de la quitter, Milady venait de faire faire chez elle de nouvelles dépenses: ce qui prouvait que la mesure générale qui renvoyait les Anglais ne la regardait pas. 

«Vous voyez, dit Lord de Winter en présentant d'Artagnan à sa soeur, un jeune gentilhomme qui a tenu ma vie entre ses mains, et qui n'a point voulu abuser de ses avantages, quoique nous fussions deux fois ennemis, puisque c'est moi qui l'ai insulté, et que je suis anglais. Remerciez-le donc, madame, si vous avez quelque amitié pour moi.» 

Milady fronça légèrement le sourcil; un nuage à peine visible passa sur son front, et un sourire tellement étrange apparut sur ses lèvres, que le jeune homme, qui vit cette triple nuance, en eut comme un frisson. 

Le frère ne vit rien; il s'était retourné pour jouer avec le singe favori de Milady, qui l'avait tiré par son pourpoint. 

«Soyez le bienvenu, monsieur, dit Milady d'une voix dont la douceur singulière contrastait avec les symptômes de mauvaise humeur que venait de remarquer d'Artagnan, vous avez acquis aujourd'hui des droits éternels à ma reconnaissance.» 

L'Anglais alors se retourna et raconta le combat sans omettre un détail. Milady l'écouta avec la plus grande attention; cependant on voyait facilement, quelque effort qu'elle fît pour cacher ses impressions, que ce récit ne lui était point agréable. Le sang lui montait à la tête, et son petit pied s'agitait impatiemment sous sa robe. 

Lord de Winter ne s'aperçut de rien. Puis, lorsqu'il eut fini, il s'approcha d'une table où étaient servis sur un plateau une bouteille de vin d'Espagne et des verres. Il emplit deux verres et d'un signe invita d'Artagnan à boire. 

D'Artagnan savait que c'était fort désobliger un Anglais que de refuser de toaster avec lui. Il s'approcha donc de la table, et prit le second verre. Cependant il n'avait point perdu de vue Milady, et dans la glace il s'aperçut du changement qui venait de s'opérer sur son visage. Maintenant qu'elle croyait n'être plus regardée, un sentiment qui ressemblait à de la férocité animait sa physionomie. Elle mordait son mouchoir à belles dents. 

Cette jolie petite soubrette, que d'Artagnan avait déjà remarquée, entra alors; elle dit en anglais quelques mots à Lord de Winter, qui demanda aussitôt à d'Artagnan la permission de se retirer, s'excusant sur l'urgence de l'affaire qui l'appelait, et chargeant sa soeur d'obtenir son pardon. 

D'Artagnan échangea une poignée de main avec Lord de Winter et revint près de Milady. Le visage de cette femme, avec une mobilité surprenante, avait repris son expression gracieuse, seulement quelques petites taches rouges disséminées sur son mouchoir indiquaient qu'elle s'était mordu les lèvres jusqu'au sang. 

Ses lèvres étaient magnifiques, on eût dit du corail. 

La conversation prit une tournure enjouée. Milady paraissait s'être entièrement remise. Elle raconta que Lord de Winter n'était que son beau-frère et non son frère: elle avait épousé un cadet de famille qui l'avait laissée veuve avec un enfant. Cet enfant était le seul héritier de Lord de Winter, si Lord de Winter ne se mariait point. Tout cela laissait voir à d'Artagnan un voile qui enveloppait quelque chose, mais il ne distinguait pas encore sous ce voile. 

Au reste, au bout d'une demi-heure de conversation, d'Artagnan était convaincu que Milady était sa compatriote: elle parlait le français avec une pureté et une élégance qui ne laissaient aucun doute à cet égard. 

D'Artagnan se répandit en propos galants et en protestations de dévouement. À toutes les fadaises qui échappèrent à notre Gascon, Milady sourit avec bienveillance. L'heure de se retirer arriva. D'Artagnan prit congé de Milady et sortit du salon le plus heureux des hommes. 

Sur l'escalier il rencontra la jolie soubrette, laquelle le frôla doucement en passant, et, tout en rougissant jusqu'aux yeux, lui demanda pardon de l'avoir touché, d'une voix si douce, que le pardon lui fut accordé à l'instant même. 

D'Artagnan revint le lendemain et fut reçu encore mieux que la veille. Lord de Winter n'y était point, et ce fut Milady qui lui fit cette fois tous les honneurs de la soirée. Elle parut prendre un grand intérêt à lui, lui demanda d'où il était, quels étaient ses amis, et s'il n'avait pas pensé quelquefois à s'attacher au service de M. le cardinal. 

D'Artagnan, qui, comme on le sait, était fort prudent pour un garçon de vingt ans, se souvint alors de ses soupçons sur Milady; il lui fit un grand éloge de Son Éminence, lui dit qu'il n'eût point manqué d'entrer dans les gardes du cardinal au lieu d'entrer dans les gardes du roi, s'il eût connu par exemple M. de Cavois au lieu de connaître M. de Tréville. 

Milady changea de conversation sans affectation aucune, et demanda à d'Artagnan de la façon la plus négligée du monde s'il n'avait jamais été en Angleterre. 

D'Artagnan répondit qu'il y avait été envoyé par M. de Tréville pour traiter d'une remonte de chevaux et qu'il en avait même ramené quatre comme échantillon. 

Milady, dans le cours de la conversation, se pinça deux ou trois fois les lèvres: elle avait affaire à un Gascon qui jouait serré. 

À la même heure que la veille d'Artagnan se retira. Dans le corridor il rencontra encore la jolie Ketty; c'était le nom de la soubrette. Celle-ci le regarda avec une expression de mystérieuse bienveillance à laquelle il n'y avait point à se tromper. Mais d'Artagnan était si préoccupé de la maîtresse, qu'il ne remarquait absolument que ce qui venait d'elle. 

D'Artagnan revint chez Milady le lendemain et le surlendemain, et chaque fois Milady lui fit un accueil plus gracieux. 

Chaque fois aussi, soit dans l'antichambre, soit dans le corridor, soit sur l'escalier, il rencontrait la jolie soubrette. 

Mais, comme nous l'avons dit, d'Artagnan ne faisait aucune attention à cette persistance de la pauvre Ketty.
\include{chapters/32.tex}
%!TeX root=../musketeersfr.tex 

\chapter{Soubrette Et Maîtresse}

\lettrine{C}{ependant,} comme nous l'avons dit, malgré les cris de sa conscience et les sages conseils d'Athos, d'Artagnan devenait d'heure en heure plus amoureux de Milady; aussi ne manquait-il pas tous les jours d'aller lui faire une cour à laquelle l'aventureux Gascon était convaincu qu'elle ne pouvait, tôt ou tard, manquer de répondre. 

Un soir qu'il arrivait le nez au vent, léger comme un homme qui attend une pluie d'or, il rencontra la soubrette sous la porte cochère; mais cette fois la jolie Ketty ne se contenta point de lui sourire en passant, elle lui prit doucement la main. 

«Bon! fit d'Artagnan, elle est chargée de quelque message pour moi de la part de sa maîtresse; elle va m'assigner quelque rendez-vous qu'on n'aura pas osé me donner de vive voix.» 

Et il regarda la belle enfant de l'air le plus vainqueur qu'il put prendre. 

«Je voudrais bien vous dire deux mots, monsieur le chevalier\dots, balbutia la soubrette. 

\speak  Parle, mon enfant, parle, dit d'Artagnan, j'écoute. 

\speak  Ici, impossible: ce que j'ai à vous dire est trop long et surtout trop secret. 

\speak  Eh bien, mais comment faire alors? 

\speak  Si monsieur le chevalier voulait me suivre, dit timidement Ketty. 

\speak  Où tu voudras, ma belle enfant. 

\speak  Alors, venez.» 

Et Ketty, qui n'avait point lâché la main de d'Artagnan, l'entraîna par un petit escalier sombre et tournant, et, après lui avoir fait monter une quinzaine de marches, ouvrit une porte. 

«Entrez, monsieur le chevalier, dit-elle, ici nous serons seuls et nous pourrons causer. 

\speak  Et quelle est donc cette chambre, ma belle enfant? demanda d'Artagnan. 

\speak  C'est la mienne, monsieur le chevalier; elle communique avec celle de ma maîtresse par cette porte. Mais soyez tranquille, elle ne pourra entendre ce que nous dirons, jamais elle ne se couche qu'à minuit.» 

D'Artagnan jeta un coup d'œil autour de lui. La petite chambre était charmante de goût et de propreté; mais, malgré lui, ses yeux se fixèrent sur cette porte que Ketty lui avait dit conduire à la chambre de Milady. 

Ketty devina ce qui se passait dans l'âme du jeune homme et poussa un soupir. 

«Vous aimez donc bien ma maîtresse, monsieur le chevalier, dit-elle. 

\speak  Oh! plus que je ne puis dire! j'en suis fou!» 

Ketty poussa un second soupir. 

«Hélas! monsieur, dit-elle, c'est bien dommage! 

\speak  Et que diable vois-tu donc là de si fâcheux? demanda d'Artagnan. 

\speak  C'est que, monsieur, reprit Ketty, ma maîtresse ne vous aime pas du tout. 

\speak  Hein! fit d'Artagnan, t'aurait-elle chargée de me le dire? 

\speak  Oh! non pas, monsieur! mais c'est moi qui, par intérêt pour vous, ai pris la résolution de vous en prévenir. 

\speak  Merci, ma bonne Ketty, mais de l'intention seulement, car la confidence, tu en conviendras, n'est point agréable. 

\speak  C'est-à-dire que vous ne croyez point à ce que je vous ai dit, n'est-ce pas? 

\speak  On a toujours peine à croire de pareilles choses, ma belle enfant, ne fût-ce que par amour-propre. 

\speak  Donc vous ne me croyez pas? 

\speak  J'avoue que jusqu'à ce que tu daignes me donner quelques preuves de ce que tu avances\dots 

\speak  Que dites-vous de celle-ci?» 

Et Ketty tira de sa poitrine un petit billet. 

«Pour moi? dit d'Artagnan en s'emparant vivement de la lettre. 

\speak  Non, pour un autre. 

\speak  Pour un autre? 

\speak  Oui. 

\speak  Son nom, son nom! s'écria d'Artagnan. 

\speak  Voyez l'adresse. 

\speak  M. le comte de Wardes.» 

Le souvenir de la scène de Saint-Germain se présenta aussitôt à l'esprit du présomptueux Gascon; par un mouvement rapide comme la pensée, il déchira l'enveloppe malgré le cri que poussa Ketty en voyant ce qu'il allait faire, ou plutôt ce qu'il faisait. 

«Oh! mon Dieu! monsieur le chevalier, dit-elle, que faites-vous? 

\speak  Moi, rien!» dit d'Artagnan, et il lut: 

«Vous n'avez pas répondu à mon premier billet; êtes-vous donc souffrant, ou bien auriez-vous oublié quels yeux vous me fîtes au bal de Mme de Guise? Voici l'occasion, comte! ne la laissez pas échapper.» 

D'Artagnan pâlit; il était blessé dans son amour-propre, il se crut blessé dans son amour. 

«Pauvre cher monsieur d'Artagnan! dit Ketty d'une voix pleine de compassion et en serrant de nouveau la main du jeune homme. 

\speak  Tu me plains, bonne petite! dit d'Artagnan. 

\speak  Oh! oui, de tout mon cœur! car je sais ce que c'est que l'amour, moi! 

\speak  Tu sais ce que c'est que l'amour? dit d'Artagnan la regardant pour la première fois avec une certaine attention. 

\speak  Hélas! oui. 

\speak  Eh bien, au lieu de me plaindre, alors, tu ferais bien mieux de m'aider à me venger de ta maîtresse. 

\speak  Et quelle sorte de vengeance voudriez-vous en tirer? 

\speak  Je voudrais triompher d'elle, supplanter mon rival. 

\speak  Je ne vous aiderai jamais à cela, monsieur le chevalier! dit vivement Ketty. 

\speak  Et pourquoi cela? demanda d'Artagnan. 

\speak  Pour deux raisons. 

\speak  Lesquelles? 

\speak  La première, c'est que jamais ma maîtresse ne vous a aimé. 

\speak  Qu'en sais-tu? 

\speak  Vous l'avez blessée au cœur. 

\speak  Moi! en quoi puis-je l'avoir blessée, moi qui, depuis que je la connais, vis à ses pieds comme un esclave! parle, je t'en prie. 

\speak  Je n'avouerais jamais cela qu'à l'homme\dots qui lirait jusqu'au fond de mon âme!» 

D'Artagnan regarda Ketty pour la seconde fois. La jeune fille était d'une fraîcheur et d'une beauté que bien des duchesses eussent achetées de leur couronne. 

«Ketty, dit-il, je lirai jusqu'au fond de ton âme quand tu voudras; qu'à cela ne tienne, ma chère enfant.» 

Et il lui donna un baiser sous lequel la pauvre enfant devint rouge comme une cerise. 

«Oh! non, s'écria Ketty, vous ne m'aimez pas! C'est ma maîtresse que vous aimez, vous me l'avez dit tout à l'heure. 

\speak  Et cela t'empêche-t-il de me faire connaître la seconde raison? 

\speak  La seconde raison, monsieur le chevalier, reprit Ketty enhardie par le baiser d'abord et ensuite par l'expression des yeux du jeune homme, c'est qu'en amour chacun pour soi.» 

Alors seulement d'Artagnan se rappela les coups d'œil languissants de Ketty, ses rencontres dans l'antichambre, sur l'escalier, dans le corridor, ses frôlements de main chaque fois qu'elle le rencontrait, et ses soupirs étouffés; mais, absorbé par le désir de plaire à la grande dame, il avait dédaigné la soubrette: qui chasse l'aigle ne s'inquiète pas du passereau. 

Mais cette fois notre Gascon vit d'un seul coup d'œil tout le parti qu'on pouvait tirer de cet amour que Ketty venait d'avouer d'une façon si naïve ou si effrontée: interception des lettres adressées au comte de Wardes, intelligences dans la place, entrée à toute heure dans la chambre de Ketty, contiguë à celle de sa maîtresse. Le perfide, comme on le voit, sacrifiait déjà en idée la pauvre fille pour obtenir Milady de gré ou de force. 

«Eh bien, dit-il à la jeune fille, veux-tu, ma chère Ketty, que je te donne une preuve de cet amour dont tu doutes? 

\speak  De quel amour? demanda la jeune fille. 

\speak  De celui que je suis tout prêt à ressentir pour toi. 

\speak  Et quelle est cette preuve? 

\speak  Veux-tu que ce soir je passe avec toi le temps que je passe ordinairement avec ta maîtresse? 

\speak  Oh! oui, dit Ketty en battant des mains, bien volontiers. 

\speak  Eh bien, ma chère enfant, dit d'Artagnan en s'établissant dans un fauteuil, viens çà que je te dise que tu es la plus jolie soubrette que j'aie jamais vue!» 

Et il le lui dit tant et si bien, que la pauvre enfant, qui ne demandait pas mieux que de le croire, le crut\dots Cependant, au grand étonnement de d'Artagnan, la jolie Ketty se défendait avec une certaine résolution. 

Le temps passe vite, lorsqu'il se passe en attaques et en défenses. 

Minuit sonna, et l'on entendit presque en même temps retentir la sonnette dans la chambre de Milady. 

«Grand Dieu! s'écria Ketty, voici ma maîtresse qui m'appelle! Partez, partez vite!» 

D'Artagnan se leva, prit son chapeau comme s'il avait l'intention d'obéir; puis, ouvrant vivement la porte d'une grande armoire au lieu d'ouvrir celle de l'escalier, il se blottit dedans au milieu des robes et des peignoirs de Milady. 

«Que faites-vous donc?» s'écria Ketty. 

D'Artagnan, qui d'avance avait pris la clef, s'enferma dans son armoire sans répondre. 

«Eh bien, cria Milady d'une voix aigre, dormez-vous donc que vous ne venez pas quand je sonne?» 

Et d'Artagnan entendit qu'on ouvrit violemment la porte de communication. 

«Me voici, Milady, me voici», s'écria Ketty en s'élançant à la rencontre de sa maîtresse. 

Toutes deux rentrèrent dans la chambre à coucher et comme la porte de communication resta ouverte, d'Artagnan put entendre quelque temps encore Milady gronder sa suivante, puis enfin elle s'apaisa, et la conversation tomba sur lui tandis que Ketty accommodait sa maîtresse. 

«Eh bien, dit Milady, je n'ai pas vu notre Gascon ce soir? 

\speak  Comment, madame, dit Ketty, il n'est pas venu! Serait-il volage avant d'être heureux? 

\speak  Oh non! il faut qu'il ait été empêché par M. de Tréville ou par M. des Essarts. Je m'y connais, Ketty, et je le tiens, celui-là. 

\speak  Qu'en fera madame? 

\speak  Ce que j'en ferai!\dots Sois tranquille, Ketty, il y a entre cet homme et moi une chose qu'il ignore\dots il a manqué me faire perdre mon crédit près de Son Éminence\dots Oh! je me vengerai! 

\speak  Je croyais que madame l'aimait? 

\speak  Moi, l'aimer! je le déteste! Un niais, qui tient la vie de Lord de Winter entre ses mains et qui ne le tue pas, et qui me fait perdre trois cent mille livres de rente! 

\speak  C'est vrai, dit Ketty, votre fils était le seul héritier de son oncle, et jusqu'à sa majorité vous auriez eu la jouissance de sa fortune.» 

D'Artagnan frissonna jusqu'à la moelle des os en entendant cette suave créature lui reprocher, avec cette voix stridente qu'elle avait tant de peine à cacher dans la conversation, de n'avoir pas tué un homme qu'il l'avait vue combler d'amitié. 

«Aussi, continua Milady, je me serais déjà vengée sur lui-même, si, je ne sais pourquoi, le cardinal ne m'avait recommandé de le ménager. 

\speak  Oh! oui, mais madame n'a point ménagé cette petite femme qu'il aimait. 

\speak  Oh! la mercière de la rue des Fossoyeurs: est-ce qu'il n'a pas déjà oublié qu'elle existait? La belle vengeance, ma foi!» 

Une sueur froide coulait sur le front de d'Artagnan: c'était donc un monstre que cette femme. 

Il se remit à écouter, mais malheureusement la toilette était finie. 

«C'est bien, dit Milady, rentrez chez vous et demain tâchez enfin d'avoir une réponse à cette lettre que je vous ai donnée. 

\speak  Pour M. de Wardes? dit Ketty. 

\speak  Sans doute, pour M. de Wardes. 

\speak  En voilà un, dit Ketty, qui m'a bien l'air d'être tout le contraire de ce pauvre M. d'Artagnan. 

\speak  Sortez, mademoiselle, dit Milady, je n'aime pas les commentaires.» 

D'Artagnan entendit la porte qui se refermait, puis le bruit de deux verrous que mettait Milady afin de s'enfermer chez elle; de son côté, mais le plus doucement qu'elle put, Ketty donna à la serrure un tour de clef; d'Artagnan alors poussa la porte de l'armoire. 

«O mon Dieu! dit tout bas Ketty, qu'avez-vous? et comme vous êtes pâle! 

\speak  L'abominable créature! murmura d'Artagnan. 

\speak  Silence! silence! sortez, dit Ketty; il n'y a qu'une cloison entre ma chambre et celle de Milady, on entend de l'une tout ce qui se dit dans l'autre! 

\speak  C'est justement pour cela que je ne sortirai pas, dit d'Artagnan. 

\speak  Comment? fit Ketty en rougissant. 

\speak  Ou du moins que je sortirai\dots plus tard.» 

Et il attira Ketty à lui; il n'y avait plus moyen de résister, la résistance fait tant de bruit! aussi Ketty céda. 

C'était un mouvement de vengeance contre Milady. D'Artagnan trouva qu'on avait raison de dire que la vengeance est le plaisir des dieux. Aussi, avec un peu de cœur, se serait-il contenté de cette nouvelle conquête; mais d'Artagnan n'avait que de l'ambition et de l'orgueil. 

Cependant, il faut le dire à sa louange, le premier emploi qu'il avait fait de son influence sur Ketty avait été d'essayer de savoir d'elle ce qu'était devenue Mme Bonacieux, mais la pauvre fille jura sur le crucifix à d'Artagnan qu'elle l'ignorait complètement, sa maîtresse ne laissant jamais pénétrer que la moitié de ses secrets; seulement, elle croyait pouvoir répondre qu'elle n'était pas morte. 

Quant à la cause qui avait manqué faire perdre à Milady son crédit près du cardinal, Ketty n'en savait pas davantage; mais cette fois, d'Artagnan était plus avancé qu'elle: comme il avait aperçu Milady sur un bâtiment consigné au moment où lui-même quittait l'Angleterre, il se douta qu'il était question cette fois des ferrets de diamants. 

Mais ce qu'il y avait de plus clair dans tout cela, c'est que la haine véritable, la haine profonde, la haine invétérée de Milady lui venait de ce qu'il n'avait pas tué son beau-frère. 

D'Artagnan retourna le lendemain chez Milady. Elle était de fort méchante humeur, d'Artagnan se douta que c'était le défaut de réponse de M. de Wardes qui l'agaçait ainsi. Ketty entra; mais Milady la reçut fort durement. Un coup d'œil qu'elle lança à d'Artagnan voulait dire: Vous voyez ce que je souffre pour vous. 

Cependant vers la fin de la soirée, la belle lionne s'adoucit, elle écouta en souriant les doux propos de d'Artagnan, elle lui donna même sa main à baiser. 

D'Artagnan sortit ne sachant plus que penser: mais comme c'était un garçon à qui on ne faisait pas facilement perdre la tête, tout en faisant sa cour à Milady il avait bâti dans son esprit un petit plan. 

Il trouva Ketty à la porte, et comme la veille il monta chez elle pour avoir des nouvelles. Ketty avait été fort grondée, on l'avait accusée de négligence. Milady ne comprenait rien au silence du comte de Wardes, et elle lui avait ordonné d'entrer chez elle à neuf heures du matin pour y prendre une troisième lettre. 

D'Artagnan fit promettre à Ketty de lui apporter chez lui cette lettre le lendemain matin; la pauvre fille promit tout ce que voulut son amant: elle était folle. 

Les choses se passèrent comme la veille: d'Artagnan s'enferma dans son armoire, Milady appela, fit sa toilette, renvoya Ketty et referma sa porte. Comme la veille d'Artagnan ne rentra chez lui qu'à cinq heures du matin. 

À onze heures, il vit arriver Ketty; elle tenait à la main un nouveau billet de Milady. Cette fois, la pauvre enfant n'essaya pas même de le disputer à d'Artagnan; elle le laissa faire; elle appartenait corps et âme à son beau soldat. 

D'Artagnan ouvrit le billet et lut ce qui suit: 

«Voilà la troisième fois que je vous écris pour vous dire que je vous aime. Prenez garde que je ne vous écrive une quatrième pour vous dire que je vous déteste. 

«Si vous vous repentez de la façon dont vous avez agi avec moi, la jeune fille qui vous remettra ce billet vous dira de quelle manière un galant homme peut obtenir son pardon.» 

D'Artagnan rougit et pâlit plusieurs fois en lisant ce billet. 

«Oh! vous l'aimez toujours! dit Ketty, qui n'avait pas détourné un instant les yeux du visage du jeune homme. 

\speak  Non, Ketty, tu te trompes, je ne l'aime plus; mais je veux me venger de ses mépris. 

\speak  Oui, je connais votre vengeance; vous me l'avez dite. 

\speak  Que t'importe, Ketty! tu sais bien que c'est toi seule que j'aime. 

\speak  Comment peut-on savoir cela? 

\speak  Par le mépris que je ferai d'elle.» 

Ketty soupira. 

D'Artagnan prit une plume et écrivit: 

\begin{mail}{}{Madame,}
	
Jusqu'ici j'avais douté que ce fût bien à moi que vos deux premiers billets eussent été adressés, tant je me croyais indigne d'un pareil honneur; d'ailleurs j'étais si souffrant, que j'eusse en tout cas hésité à y répondre. 

Mais aujourd'hui il faut bien que je croie à l'excès de vos bontés, puisque non seulement votre lettre, mais encore votre suivante, m'affirme que j'ai le bonheur d'être aimé de vous. 

Elle n'a pas besoin de me dire de quelle manière un galant homme peut obtenir son pardon. J'irai donc vous demander le mien ce soir à onze heures. Tarder d'un jour serait à mes yeux, maintenant, vous faire une nouvelle offense. 

\closeletter[Celui que vous avez rendu le plus heureux des hommes.]{Comte de Wardes.}
\end{mail}

Ce billet était d'abord un faux, c'était ensuite une indélicatesse; c'était même, au point de vue de nos mœurs actuelles, quelque chose comme une infamie; mais on se ménageait moins à cette époque qu'on ne le fait aujourd'hui. D'ailleurs d'Artagnan, par ses propres aveux, savait Milady coupable de trahison à des chefs plus importants, et il n'avait pour elle qu'une estime fort mince. Et cependant malgré ce peu d'estime, il sentait qu'une passion insensée le brûlait pour cette femme. Passion ivre de mépris, mais passion ou soif, comme on voudra. 

L'intention de d'Artagnan était bien simple: par la chambre de Ketty il arrivait à celle de sa maîtresse; il profitait du premier moment de surprise, de honte, de terreur pour triompher d'elle; peut-être aussi échouerait-il, mais il fallait bien donner quelque chose au hasard. Dans huit jours la campagne s'ouvrait, et il fallait partir; d'Artagnan n'avait pas le temps de filer le parfait amour. 

«Tiens, dit le jeune homme en remettant à Ketty le billet tout cacheté, donne cette lettre à Milady; c'est la réponse de M. de Wardes.» 

La pauvre Ketty devint pâle comme la mort, elle se doutait de ce que contenait le billet. 

«Écoute, ma chère enfant, lui dit d'Artagnan, tu comprends qu'il faut que tout cela finisse d'une façon ou de l'autre; Milady peut découvrir que tu as remis le premier billet à mon valet, au lieu de le remettre au valet du comte; que c'est moi qui ai décacheté les autres qui devaient être décachetés par M. de Wardes; alors Milady te chasse, et, tu la connais, ce n'est pas une femme à borner là sa vengeance. 

\speak  Hélas! dit Ketty, pour qui me suis-je exposée à tout cela? 

\speak  Pour moi, je le sais bien, ma toute belle, dit le jeune homme, aussi je t'en suis bien reconnaissant, je te le jure. 

\speak  Mais enfin, que contient votre billet? 

\speak  Milady te le dira. 

\speak  Ah! vous ne m'aimez pas! s'écria Ketty, et je suis bien malheureuse!» 

À ce reproche il y a une réponse à laquelle les femmes se trompent toujours; d'Artagnan répondit de manière que Ketty demeurât dans la plus grande erreur. 

Cependant elle pleura beaucoup avant de se décider à remettre cette lettre à Milady, mais enfin elle se décida, c'est tout ce que voulait d'Artagnan. 

D'ailleurs il lui promit que le soir il sortirait de bonne heure de chez sa maîtresse, et qu'en sortant de chez sa maîtresse il monterait chez elle. 

Cette promesse acheva de consoler la pauvre Ketty. 
\include{chapters/34.tex}
%!TeX root=../musketeersfr.tex 

\chapter{La Nuit Tous Les Chats Sont Gris}

\lettrine{C}{e} soir, attendu si impatiemment par Porthos et par d'Artagnan, arriva enfin. 

\zz
D'Artagnan, comme d'habitude, se présenta vers les neuf heures chez Milady. Il la trouva d'une humeur charmante; jamais elle ne l'avait si bien reçu. Notre Gascon vit du premier coup d'œil que son billet avait été remis, et ce billet faisait son effet. 

Ketty entra pour apporter des sorbets. Sa maîtresse lui fit une mine charmante, lui sourit de son plus gracieux sourire; mais, hélas! la pauvre fille était si triste, qu'elle ne s'aperçut même pas de la bienveillance de Milady. 

D'Artagnan regardait l'une après l'autre ces deux femmes, et il était forcé de s'avouer que la nature s'était trompée en les formant; à la grande dame elle avait donné une âme vénale et vile, à la soubrette elle avait donné le cœur d'une duchesse. 

À dix heures Milady commença à paraître inquiète, d'Artagnan comprit ce que cela voulait dire; elle regardait la pendule, se levait, se rasseyait, souriait à d'Artagnan d'un air qui voulait dire: Vous êtes fort aimable sans doute, mais vous seriez charmant si vous partiez! 

D'Artagnan se leva et prit son chapeau; Milady lui donna sa main à baiser; le jeune homme sentit qu'elle la lui serrait et comprit que c'était par un sentiment non pas de coquetterie, mais de reconnaissance à cause de son départ. 

«Elle l'aime diablement», murmura-t-il. Puis il sortit. 

Cette fois Ketty ne l'attendait aucunement, ni dans l'antichambre, ni dans le corridor, ni sous la grande porte. Il fallut que d'Artagnan trouvât tout seul l'escalier et la petite chambre. 

Ketty était assise la tête cachée dans ses mains, et pleurait. 

Elle entendit entrer d'Artagnan, mais elle ne releva point la tête; le jeune homme alla à elle et lui prit les mains, alors elle éclata en sanglots. 

Comme l'avait présumé d'Artagnan, Milady, en recevant la lettre, avait, dans le délire de sa joie, tout dit à sa suivante; puis, en récompense de la manière dont cette fois elle avait fait la commission, elle lui avait donné une bourse. Ketty, en rentrant chez elle, avait jeté la bourse dans un coin, où elle était restée tout ouverte, dégorgeant trois ou quatre pièces d'or sur le tapis. 

La pauvre fille, à la voix de d'Artagnan, releva la tête. D'Artagnan lui-même fut effrayé du bouleversement de son visage; elle joignit les mains d'un air suppliant, mais sans oser dire une parole. 

Si peu sensible que fût le cœur de d'Artagnan, il se sentit attendri par cette douleur muette; mais il tenait trop à ses projets et surtout à celui-ci, pour rien changer au programme qu'il avait fait d'avance. Il ne laissa donc à Ketty aucun espoir de le fléchir, seulement il lui présenta son action comme une simple vengeance. 

Cette vengeance, au reste, devenait d'autant plus facile, que Milady, sans doute pour cacher sa rougeur à son amant, avait recommandé à Ketty d'éteindre toutes les lumières dans l'appartement, et même dans sa chambre, à elle. Avant le jour, M. de Wardes devait sortir, toujours dans l'obscurité. 

Au bout d'un instant on entendit Milady qui rentrait dans sa chambre. D'Artagnan s'élança aussitôt dans son armoire. À peine y était-il blotti que la sonnette se fit entendre. 

Ketty entra chez sa maîtresse, et ne laissa point la porte ouverte; mais la cloison était si mince, que l'on entendait à peu près tout ce qui se disait entre les deux femmes. 

Milady semblait ivre de joie, elle se faisait répéter par Ketty les moindres détails de la prétendue entrevue de la soubrette avec de Wardes, comment il avait reçu sa lettre, comment il avait répondu, quelle était l'expression de son visage, s'il paraissait bien amoureux; et à toutes ces questions la pauvre Ketty, forcée de faire bonne contenance, répondait d'une voix étouffée dont sa maîtresse ne remarquait même pas l'accent douloureux, tant le bonheur est égoïste. 

Enfin, comme l'heure de son entretien avec le comte approchait, Milady fit en effet tout éteindre chez elle, et ordonna à Ketty de rentrer dans sa chambre, et d'introduire de Wardes aussitôt qu'il se présenterait. 

L'attente de Ketty ne fut pas longue. À peine d'Artagnan eut-il vu par le trou de la serrure de son armoire que tout l'appartement était dans l'obscurité, qu'il s'élança de sa cachette au moment même où Ketty refermait la porte de communication. 

«Qu'est-ce que ce bruit? demanda Milady. 

\speak  C'est moi, dit d'Artagnan à demi-voix; moi, le comte de Wardes. 

\speak  Oh! mon Dieu, mon Dieu! murmura Ketty, il n'a pas même pu attendre l'heure qu'il avait fixée lui-même! 

\speak  Eh bien, dit Milady d'une voix tremblante, pourquoi n'entre-t-il pas? Comte, comte, ajouta-t-elle, vous savez bien que je vous attends!» 

À cet appel, d'Artagnan éloigna doucement Ketty et s'élança dans la chambre de Milady. 

Si la rage et la douleur doivent torturer une âme, c'est celle de l'amant qui reçoit sous un nom qui n'est pas le sien des protestations d'amour qui s'adressent à son heureux rival. 

D'Artagnan était dans une situation douloureuse qu'il n'avait pas prévue, la jalousie le mordait au cœur, et il souffrait presque autant que la pauvre Ketty, qui pleurait en ce même moment dans la chambre voisine. 

«Oui, comte, disait Milady de sa plus douce voix en lui serrant tendrement la main dans les siennes; oui, je suis heureuse de l'amour que vos regards et vos paroles m'ont exprimé chaque fois que nous nous sommes rencontrés. Moi aussi, je vous aime. Oh! demain, demain, je veux quelque gage de vous qui me prouve que vous pensez à moi, et comme vous pourriez m'oublier, tenez.» 

Et elle passa une bague de son doigt à celui de d'Artagnan. 

D'Artagnan se rappela avoir vu cette bague à la main de Milady: c'était un magnifique saphir entouré de brillants. 

Le premier mouvement de d'Artagnan fut de le lui rendre, mais Milady ajouta: 

«Non, non; gardez cette bague pour l'amour de moi. Vous me rendez d'ailleurs, en l'acceptant, ajouta-t-elle d'une voix émue, un service bien plus grand que vous ne sauriez l'imaginer.» 

«Cette femme est pleine de mystères», murmura en lui-même d'Artagnan. 

En ce moment il se sentit prêt à tout révéler. Il ouvrit la bouche pour dire à Milady qui il était, et dans quel but de vengeance il était venu, mais elle ajouta: 

«Pauvre ange, que ce monstre de Gascon a failli tuer!» 

Le monstre, c'était lui. 

«Oh! continua Milady, est-ce que vos blessures vous font encore souffrir? 

\speak  Oui, beaucoup, dit d'Artagnan, qui ne savait trop que répondre. 

\speak  Soyez tranquille, murmura Milady, je vous vengerai, moi, et cruellement!» 

«Peste! se dit d'Artagnan, le moment des confidences n'est pas encore venu.» 

Il fallut quelque temps à d'Artagnan pour se remettre de ce petit dialogue: mais toutes les idées de vengeance qu'il avait apportées s'étaient complètement évanouies. Cette femme exerçait sur lui une incroyable puissance, il la haïssait et l'adorait à la fois, il n'avait jamais cru que deux sentiments si contraires pussent habiter dans le même cœur, et en se réunissant, former un amour étrange et en quelque sorte diabolique. 

Cependant une heure venait de sonner; il fallut se séparer; d'Artagnan, au moment de quitter Milady, ne sentit plus qu'un vif regret de s'éloigner, et, dans l'adieu passionné qu'ils s'adressèrent réciproquement, une nouvelle entrevue fut convenue pour la semaine suivante. La pauvre Ketty espérait pouvoir adresser quelques mots à d'Artagnan lorsqu'il passerait dans sa chambre; mais Milady le reconduisit elle-même dans l'obscurité et ne le quitta que sur l'escalier. 

Le lendemain au matin, d'Artagnan courut chez Athos. Il était engagé dans une si singulière aventure qu'il voulait lui demander conseil. Il lui raconta tout: Athos fronça plusieurs fois le sourcil. 

«Votre Milady, lui dit-il, me paraît une créature infâme, mais vous n'en avez pas moins eu tort de la tromper: vous voilà d'une façon ou d'une autre une ennemie terrible sur les bras.» 

Et tout en lui parlant, Athos regardait avec attention le saphir entouré de diamants qui avait pris au doigt de d'Artagnan la place de la bague de la reine, soigneusement remise dans un écrin. 

«Vous regardez cette bague? dit le Gascon tout glorieux d'étaler aux regards de ses amis un si riche présent. 

\speak  Oui, dit Athos, elle me rappelle un bijou de famille. 

\speak  Elle est belle, n'est-ce pas? dit d'Artagnan. 

\speak  Magnifique! répondit Athos; je ne croyais pas qu'il existât deux saphirs d'une si belle eau. L'avez-vous donc troquée contre votre diamant? 

\speak  Non, dit d'Artagnan; c'est un cadeau de ma belle Anglaise, ou plutôt de ma belle Française: car, quoique je ne le lui aie point demandé, je suis convaincu qu'elle est née en France. 

\speak  Cette bague vous vient de Milady? s'écria Athos avec une voix dans laquelle il était facile de distinguer une grande émotion. 

\speak  D'elle-même; elle me l'a donnée cette nuit. 

\speak  Montrez-moi donc cette bague, dit Athos. 

\speak  La voici», répondit d'Artagnan en la tirant de son doigt. 

Athos l'examina et devint très pâle, puis il l'essaya à l'annulaire de sa main gauche; elle allait à ce doigt comme si elle eût été faite pour lui. Un nuage de colère et de vengeance passa sur le front ordinairement calme du gentilhomme. 

«Il est impossible que ce soit la même, dit-il; comment cette bague se trouverait-elle entre les mains de Milady Clarick? Et cependant il est bien difficile qu'il y ait entre deux bijoux une pareille ressemblance. 

\speak  Connaissez-vous cette bague? demanda d'Artagnan. 

\speak  J'avais cru la reconnaître, dit Athos, mais sans doute que je me trompais.» 

Et il la rendit à d'Artagnan, sans cesser cependant de la regarder. 

«Tenez, dit-il au bout d'un instant, d'Artagnan, ôtez cette bague de votre doigt ou tournez-en le chaton en dedans; elle me rappelle de si cruels souvenirs, que je n'aurais pas ma tête pour causer avec vous. Ne veniez-vous pas me demander des conseils, ne me disiez-vous point que vous étiez embarrassé sur ce que vous deviez faire?\dots Mais attendez\dots rendez-moi ce saphir: celui dont je voulais parler doit avoir une de ses faces éraillée par suite d'un accident.» 

D'Artagnan tira de nouveau la bague de son doigt et la rendit à Athos. 

Athos tressaillit: 

«Tenez, dit-il, voyez, n'est-ce pas étrange?» 

Et il montrait à d'Artagnan cette égratignure qu'il se rappelait devoir exister. 

«Mais de qui vous venait ce saphir, Athos? 

\speak  De ma mère, qui le tenait de sa mère à elle. Comme je vous le dis, c'est un vieux bijou\dots qui ne devait jamais sortir de la famille. 

\speak  Et vous l'avez\dots vendu? demanda avec hésitation d'Artagnan. 

\speak  Non, reprit Athos avec un singulier sourire; je l'ai donné pendant une nuit d'amour, comme il vous a été donné à vous.» 

D'Artagnan resta pensif à son tour, il lui semblait voir dans l'âme de Milady des abîmes dont les profondeurs étaient sombres et inconnues. 

Il remit la bague non pas à son doigt, mais dans sa poche. 

«Écoutez, lui dit Athos en lui prenant la main, vous savez si je vous aime, d'Artagnan; j'aurais un fils que je ne l'aimerais pas plus que vous. Eh bien, croyez-moi, renoncez à cette femme. Je ne la connais pas, mais une espèce d'intuition me dit que c'est une créature perdue, et qu'il y a quelque chose de fatal en elle. 

\speak  Et vous avez raison, dit d'Artagnan. Aussi, je m'en sépare; je vous avoue que cette femme m'effraie moi-même. 

\speak  Aurez-vous ce courage? dit Athos. 

\speak  Je l'aurai, répondit d'Artagnan, et à l'instant même. 

\speak  Eh bien, vrai, mon enfant, vous avez raison, dit le gentilhomme en serrant la main du Gascon avec une affection presque paternelle; que Dieu veuille que cette femme, qui est à peine entrée dans votre vie, n'y laisse pas une trace funeste!» 

Et Athos salua d'Artagnan de la tête, en homme qui veut faire comprendre qu'il n'est pas fâché de rester seul avec ses pensées. 

En rentrant chez lui d'Artagnan trouva Ketty, qui l'attendait. Un mois de fièvre n'eût pas plus changé la pauvre enfant qu'elle ne l'était pour cette nuit d'insomnie et de douleur. 

Elle était envoyée par sa maîtresse au faux de Wardes. Sa maîtresse était folle d'amour, ivre de joie: elle voulait savoir quand le comte lui donnerait une seconde entrevue. 

Et la pauvre Ketty, pâle et tremblante, attendait la réponse de d'Artagnan. 

Athos avait une grande influence sur le jeune homme: les conseils de son ami joints aux cris de son propre cœur l'avaient déterminé, maintenant que son orgueil était sauvé et sa vengeance satisfaite, à ne plus revoir Milady. Pour toute réponse il prit donc une plume et écrivit la lettre suivante: 

\begin{mail}{}{}
Ne comptez pas sur moi, madame, pour le prochain rendez-vous: depuis ma convalescence j'ai tant d'occupations de ce genre qu'il m'a fallu y mettre un certain ordre. Quand votre tour viendra, j'aurai l'honneur de vous en faire part. 
\closeletter[Je vous baise les mains.]{Comte de Wardes }
\end{mail}

Du saphir pas un mot: le Gascon voulait-il garder une arme contre Milady? ou bien, soyons franc, ne conservait-il pas ce saphir comme une dernière ressource pour l'équipement? 

On aurait tort au reste de juger les actions d'une époque au point de vue d'une autre époque. Ce qui aujourd'hui serait regardé comme une honte pour un galant homme était dans ce temps une chose toute simple et toute naturelle, et les cadets des meilleures familles se faisaient en général entretenir par leurs maîtresses. 

D'Artagnan passa sa lettre tout ouverte à Ketty, qui la lut d'abord sans la comprendre et qui faillit devenir folle de joie en la relisant une seconde fois. 

Ketty ne pouvait croire à ce bonheur: d'Artagnan fut forcé de lui renouveler de vive voix les assurances que la lettre lui donnait par écrit; et quel que fût, avec le caractère emporté de Milady, le danger que courût la pauvre enfant à remettre ce billet à sa maîtresse, elle n'en revint pas moins place Royale de toute la vitesse de ses jambes. 

Le cœur de la meilleure femme est impitoyable pour les douleurs d'une rivale. 

Milady ouvrit la lettre avec un empressement égal à celui que Ketty avait mis à l'apporter, mais au premier mot qu'elle lut, elle devint livide; puis elle froissa le papier; puis elle se retourna avec un éclair dans les yeux du côté de Ketty. 

«Qu'est-ce que cette lettre? dit-elle. 

\speak  Mais c'est la réponse à celle de madame, répondit Ketty toute tremblante. 

\speak  Impossible! s'écria Milady; impossible qu'un gentilhomme ait écrit à une femme une pareille lettre!» 

Puis tout à coup tressaillant: 

«Mon Dieu! dit-elle, saurait-il\dots» Et elle s'arrêta. 

Ses dents grinçaient, elle était couleur de cendre: elle voulut faire un pas vers la fenêtre pour aller chercher de l'air; mais elle ne put qu'étendre les bras, les jambes lui manquèrent, et elle tomba sur un fauteuil. 

Ketty crut qu'elle se trouvait mal et se précipita pour ouvrir son corsage. Mais Milady se releva vivement: 

«Que me voulez-vous? dit-elle, et pourquoi portez-vous la main sur moi? 

\speak  J'ai pensé que madame se trouvait mal et j'ai voulu lui porter secours, répondit la suivante tout épouvantée de l'expression terrible qu'avait prise la figure de sa maîtresse. 

\speak  Me trouver mal, moi? moi? me prenez-vous pour une femmelette? Quand on m'insulte, je ne me trouve pas mal, je me venge, entendez-vous!» 

Et de la main elle fit signe à Ketty de sortir.
%!TeX root=../musketeersfr.tex 

\chapter{Rêve De Vengeance}

\lettrine{L}{e} soir Milady donna l'ordre d'introduire M. d'Artagnan aussitôt qu'il viendrait, selon son habitude. Mais il ne vint pas. 

\zz
Le lendemain Ketty vint voir de nouveau le jeune homme et lui raconta tout ce qui s'était passé la veille: d'Artagnan sourit; cette jalouse colère de Milady, c'était sa vengeance. 

Le soir Milady fut plus impatiente encore que la veille, elle renouvela l'ordre relatif au Gascon; mais comme la veille elle l'attendit inutilement. 

Le lendemain Ketty se présenta chez d'Artagnan, non plus joyeuse et alerte comme les deux jours précédents, mais au contraire triste à mourir. 

D'Artagnan demanda à la pauvre fille ce qu'elle avait; mais celle-ci, pour toute réponse, tira une lettre de sa poche et la lui remit. 

Cette lettre était de l'écriture de Milady: seulement cette fois elle était bien à l'adresse de d'Artagnan et non à celle de M. de Wardes. 

Il l'ouvrit et lut ce qui suit: 

\begin{mail}{}{Cher monsieur d'Artagnan,}
C'est mal de négliger ainsi ses amis, surtout au moment où l'on va les quitter pour si longtemps. Mon beau-frère et moi nous avons attendu hier et avant-hier inutilement. En sera-t-il de même ce soir?

\closeletter[Votre bien reconnaissante,]{Lady Clarick}
\end{mail}

«C'est tout simple, dit d'Artagnan, et je m'attendais à cette lettre. Mon crédit hausse de la baisse du comte de Wardes. 

\speak  Est-ce que vous irez? demanda Ketty. 

\speak  Écoute, ma chère enfant, dit le Gascon, qui cherchait à s'excuser à ses propres yeux de manquer à la promesse qu'il avait faite à Athos, tu comprends qu'il serait impolitique de ne pas se rendre à une invitation si positive. Milady, en ne me voyant pas revenir, ne comprendrait rien à l'interruption de mes visites, elle pourrait se douter de quelque chose, et qui peut dire jusqu'où irait la vengeance d'une femme de cette trempe? 

\speak  Oh! mon Dieu! dit Ketty, vous savez présenter les choses de façon que vous avez toujours raison. Mais vous allez encore lui faire la cour; et si cette fois vous alliez lui plaire sous votre véritable nom et votre vrai visage, ce serait bien pis que la première fois!» 

L'instinct faisait deviner à la pauvre fille une partie de ce qui allait arriver. 

D'Artagnan la rassura du mieux qu'il put et lui promit de rester insensible aux séductions de Milady. 

Il lui fit répondre qu'il était on ne peut plus reconnaissant de ses bontés et qu'il se rendrait à ses ordres; mais il n'osa lui écrire de peur de ne pouvoir, à des yeux aussi exercés que ceux de Milady, déguiser suffisamment son écriture. 

À neuf heures sonnant, d'Artagnan était place Royale. Il était évident que les domestiques qui attendaient dans l'antichambre étaient prévenus, car aussitôt que d'Artagnan parut, avant même qu'il eût demandé si Milady était visible, un d'eux courut l'annoncer. 

«Faites entrer», dit Milady d'une voix brève, mais si perçante que d'Artagnan l'entendit de l'antichambre. 

On l'introduisit. 

«Je n'y suis pour personne, dit Milady; entendez-vous, pour personne.» 

Le laquais sortit. 

D'Artagnan jeta un regard curieux sur Milady: elle était pâle et avait les yeux fatigués, soit par les larmes, soit par l'insomnie. On avait avec intention diminué le nombre habituel des lumières, et cependant la jeune femme ne pouvait arriver à cacher les traces de la fièvre qui l'avait dévorée depuis deux jours. 

D'Artagnan s'approcha d'elle avec sa galanterie ordinaire; elle fit alors un effort suprême pour le recevoir, mais jamais physionomie plus bouleversée ne démentit sourire plus aimable. 

Aux questions que d'Artagnan lui fit sur sa santé: 

«Mauvaise, répondit-elle, très mauvaise. 

\speak  Mais alors, dit d'Artagnan, je suis indiscret, vous avez besoin de repos sans doute et je vais me retirer. 

\speak  Non pas, dit Milady; au contraire, restez, monsieur d'Artagnan, votre aimable compagnie me distraira.» 

«Oh! oh! pensa d'Artagnan, elle n'a jamais été si charmante, défions-nous.» 

Milady prit l'air le plus affectueux qu'elle put prendre, et donna tout l'éclat possible à sa conversation. En même temps cette fièvre qui l'avait abandonnée un instant revenait rendre l'éclat à ses yeux, le coloris à ses joues, le carmin à ses lèvres. D'Artagnan retrouva la Circé qui l'avait déjà enveloppé de ses enchantements. Son amour, qu'il croyait éteint et qui n'était qu'assoupi, se réveilla dans son cœur. Milady souriait et d'Artagnan sentait qu'il se damnerait pour ce sourire. 

Il y eut un moment où il sentit quelque chose comme un remords de ce qu'il avait fait contre elle. 

Peu à peu Milady devint plus communicative. Elle demanda à d'Artagnan s'il avait une maîtresse. 

«Hélas! dit d'Artagnan de l'air le plus sentimental qu'il put prendre, pouvez-vous être assez cruelle pour me faire une pareille question, à moi qui, depuis que je vous ai vue, ne respire et ne soupire que par vous et pour vous!» 

Milady sourit d'un étrange sourire. 

«Ainsi vous m'aimez? dit-elle. 

\speak  Ai-je besoin de vous le dire, et ne vous en êtes-vous point aperçue? 

\speak  Si fait; mais, vous le savez, plus les cœurs sont fiers, plus ils sont difficiles à prendre. 

\speak  Oh! les difficultés ne m'effraient pas, dit d'Artagnan; il n'y a que les impossibilités qui m'épouvantent. 

\speak  Rien n'est impossible, dit Milady, à un véritable amour. 

\speak  Rien, madame? 

\speak  Rien», reprit Milady. 

«Diable! reprit d'Artagnan à part lui, la note est changée. Deviendrait-elle amoureuse de moi, par hasard, la capricieuse, et serait-elle disposée à me donner à moi-même quelque autre saphir pareil à celui qu'elle m'a donné me prenant pour de Wardes?» 

D'Artagnan rapprocha vivement son siège de celui de Milady. 

«Voyons, dit-elle, que feriez-vous bien pour prouver cet amour dont vous parlez? 

\speak  Tout ce qu'on exigerait de moi. Qu'on ordonne, et je suis prêt. 

\speak  À tout? 

\speak  À tout! s'écria d'Artagnan qui savait d'avance qu'il n'avait pas grand-chose à risquer en s'engageant ainsi. 

\speak  Eh bien, causons un peu, dit à son tour Milady en rapprochant son fauteuil de la chaise de d'Artagnan. 

\speak  Je vous écoute, madame», dit celui-ci. 

Milady resta un instant soucieuse et comme indécise puis paraissant prendre une résolution: 

«J'ai un ennemi, dit-elle. 

\speak  Vous, madame! s'écria d'Artagnan jouant la surprise, est-ce possible, mon Dieu? belle et bonne comme vous l'êtes! 

\speak  Un ennemi mortel. 

\speak  En vérité? 

\speak  Un ennemi qui m'a insultée si cruellement que c'est entre lui et moi une guerre à mort. Puis-je compter sur vous comme auxiliaire?» 

D'Artagnan comprit sur-le-champ où la vindicative créature en voulait venir. 

«Vous le pouvez, madame, dit-il avec emphase, mon bras et ma vie vous appartiennent comme mon amour. 

\speak  Alors, dit Milady, puisque vous êtes aussi généreux qu'amoureux\dots» 

Elle s'arrêta. 

«Eh bien? demanda d'Artagnan. 

\speak  Eh bien, reprit Milady après un moment de silence, cessez dès aujourd'hui de parler d'impossibilités. 

\speak  Ne m'accablez pas de mon bonheur», s'écria d'Artagnan en se précipitant à genoux et en couvrant de baisers les mains qu'on lui abandonnait. 

\speak  Venge-moi de cet infâme de Wardes, murmura Milady entre ses dents, et je saurai bien me débarrasser de toi ensuite, double sot, lame d'épée vivante! 

\speak  Tombe volontairement entre mes bras après m'avoir raillé si effrontément, hypocrite et dangereuse femme, pensait d'Artagnan de son côté, et ensuite je rirai de toi avec celui que tu veux tuer par ma main.» 

D'Artagnan releva la tête. 

«Je suis prêt, dit-il. 

\speak  Vous m'avez donc comprise, cher monsieur d'Artagnan! dit Milady. 

\speak  Je devinerais un de vos regards. 

\speak  Ainsi vous emploieriez pour moi votre bras, qui s'est déjà acquis tant de renommée? 

\speak  À l'instant même. 

Mais moi, dit Milady, comment paierai-je un pareil service; je connais les amoureux, ce sont des gens qui ne font rien pour rien? 

\speak  Vous savez la seule réponse que je désire, dit d'Artagnan, la seule qui soit digne de vous et de moi!» 

Et il l'attira doucement vers lui. 

Elle résista à peine. 

«Intéressé! dit-elle en souriant. 

\speak  Ah! s'écria d'Artagnan véritablement emporté par la passion que cette femme avait le don d'allumer dans son cœur, ah! c'est que mon bonheur me paraît invraisemblable, et qu'ayant toujours peur de le voir s'envoler comme un rêve, j'ai hâte d'en faire une réalité. 

\speak  Eh bien, méritez donc ce prétendu bonheur. 

\speak  Je suis à vos ordres, dit d'Artagnan. 

\speak  Bien sûr? fit Milady avec un dernier doute. 

\speak  Nommez-moi l'infâme qui a pu faire pleurer vos beaux yeux. 

\speak  Qui vous dit que j'ai pleuré? dit-elle. 

\speak  Il me semblait\dots 

\speak  Les femmes comme moi ne pleurent pas, dit Milady. 

\speak  Tant mieux! Voyons, dites-moi comment il s'appelle. 

\speak  Songez que son nom c'est tout mon secret. 

\speak  Il faut cependant que je sache son nom. 

\speak  Oui, il le faut; voyez si j'ai confiance en vous! 

\speak  Vous me comblez de joie. Comment s'appelle-t-il? 

\speak  Vous le connaissez. 

\speak  Vraiment? 

\speak  Oui. 

\speak  Ce n'est pas un de mes amis? reprit d'Artagnan en jouant l'hésitation pour faire croire à son ignorance. 

\speak  Si c'était un de vos amis, vous hésiteriez donc?» s'écria Milady. Et un éclair de menace passa dans ses yeux. 

«Non, fût-ce mon frère!» s'écria d'Artagnan comme emporté par l'enthousiasme. 

Notre Gascon s'avançait sans risque; car il savait où il allait. 

«J'aime votre dévouement, dit Milady. 

\speak  Hélas! n'aimez-vous que cela en moi? demanda d'Artagnan. 

\speak  Je vous aime aussi, vous», dit-elle en lui prenant la main. 

Et l'ardente pression fit frissonner d'Artagnan, comme si, par le toucher, cette fièvre qui brûlait Milady le gagnait lui-même. 

«Vous m'aimez, vous! s'écria-t-il. Oh! si cela était, ce serait à en perdre la raison.» 

Et il l'enveloppa de ses deux bras. Elle n'essaya point d'écarter ses lèvres de son baiser, seulement elle ne le lui rendit pas. 

Ses lèvres étaient froides: il sembla à d'Artagnan qu'il venait d'embrasser une statue. 

Il n'en était pas moins ivre de joie, électrisé d'amour, il croyait presque à la tendresse de Milady; il croyait presque au crime de de Wardes. Si de Wardes eût été en ce moment sous sa main, il l'eût tué. 

Milady saisit l'occasion. 

«Il s'appelle\dots, dit-elle à son tour. 

\speak  De Wardes, je le sais, s'écria d'Artagnan. 

\speak  Et comment le savez-vous?» demanda Milady en lui saisissant les deux mains et en essayant de lire par ses yeux jusqu'au fond de son âme. 

D'Artagnan sentit qu'il s'était laissé emporter, et qu'il avait fait une faute. 

«Dites, dites, mais dites donc! répétait Milady, comment le savez-vous? 

\speak  Comment je le sais? dit d'Artagnan. 

\speak  Oui. 

\speak  Je le sais, parce que, hier, de Wardes, dans un salon où j'étais, a montré une bague qu'il a dit tenir de vous. 

\speak  Le misérable!» s'écria Milady. 

L'épithète, comme on le comprend bien, retentit jusqu'au fond du cœur de d'Artagnan. 

«Eh bien? continua-t-elle. 

\speak  Eh bien, je vous vengerai de ce misérable, reprit d'Artagnan en se donnant des airs de don Japhet d'Arménie. 

\speak  Merci, mon brave ami! s'écria Milady; et quand serai-je vengée? 

\speak  Demain, tout de suite, quand vous voudrez.» 

Milady allait s'écrier: «Tout de suite»; mais elle réfléchit qu'une pareille précipitation serait peu gracieuse pour d'Artagnan. 

D'ailleurs, elle avait mille précautions à prendre, mille conseils à donner à son défenseur, pour qu'il évitât les explications devant témoins avec le comte. Tout cela se trouva prévu par un mot de d'Artagnan. 

«Demain, dit-il, vous serez vengée ou je serai mort. 

\speak  Non! dit-elle, vous me vengerez; mais vous ne mourrez pas. C'est un lâche. 

\speak  Avec les femmes peut-être, mais pas avec les hommes. J'en sais quelque chose, moi. 

\speak  Mais il me semble que dans votre lutte avec lui, vous n'avez pas eu à vous plaindre de la fortune. 

\speak  La fortune est une courtisane: favorable hier, elle peut me trahir demain. 

\speak  Ce qui veut dire que vous hésitez maintenant. 

\speak  Non, je n'hésite pas, Dieu m'en garde; mais serait-il juste de me laisser aller à une mort possible sans m'avoir donné au moins un peu plus que de l'espoir?» 

Milady répondit par un coup d'œil qui voulait dire: 

«N'est-ce que cela? parlez donc.» 

Puis, accompagnant le coup d'œil de paroles explicatives. 

«C'est trop juste, dit-elle tendrement. 

\speak  Oh! vous êtes un ange, dit le jeune homme. 

\speak  Ainsi, tout est convenu? dit-elle. 

\speak  Sauf ce que je vous demande, chère âme! 

\speak  Mais, lorsque je vous dis que vous pouvez vous fier à ma tendresse? 

\speak  Je n'ai pas de lendemain pour attendre. 

\speak  Silence; j'entends mon frère: il est inutile qu'il vous trouve ici.» 

Elle sonna; Ketty parut. 

«Sortez par cette porte, dit-elle en poussant une petit porte dérobée, et revenez à onze heures; nous achèverons cet entretien: Ketty vous introduira chez moi.» 

La pauvre enfant pensa tomber à la renverse en entendant ces paroles. 

«Eh bien, que faites-vous, mademoiselle, à demeurer immobile comme une statue? Allons, reconduisez le chevalier; et ce soir, à onze heures, vous avez entendu!» 

«Il paraît que ses rendez-vous sont à onze heures, pensa d'Artagnan: c'est une habitude prise.» 

Milady lui tendit une main qu'il baisa tendrement. 

«Voyons, dit-il en se retirant et en répondant à peine aux reproches de Ketty, voyons, ne soyons pas un sot; décidément cette femme est une grande scélérate: prenons garde.» 
%!TeX root=../musketeersfr.tex 

\chapter{Le Secret De Milady} 
	
\lettrine{D}{'Artagnan} était sorti de l'hôtel au lieu de monter tout de suite chez Ketty, malgré les instances que lui avait faites la jeune fille, et cela pour deux raisons: la première parce que de cette façon il évitait les reproches, les récriminations, les prières; la seconde, parce qu'il n'était pas fâché de lire un peu dans sa pensée, et, s'il était possible, dans celle de cette femme. 

Tout ce qu'il y avait de plus clair là-dedans, c'est que d'Artagnan aimait Milady comme un fou et qu'elle ne l'aimait pas le moins du monde. Un instant d'Artagnan comprit que ce qu'il aurait de mieux à faire serait de rentrer chez lui et d'écrire à Milady une longue lettre dans laquelle il lui avouerait que lui et de Wardes étaient jusqu'à présent absolument le même, que par conséquent il ne pouvait s'engager, sous peine de suicide, à tuer de Wardes. Mais lui aussi était éperonné d'un féroce désir de vengeance; il voulait posséder à son tour cette femme sous son propre nom; et comme cette vengeance lui paraissait avoir une certaine douceur, il ne voulait point y renoncer. 

Il fit cinq ou six fois le tour de la place Royale, se retournant de dix pas en dix pas pour regarder la lumière de l'appartement de Milady, qu'on apercevait à travers les jalousies; il était évident que cette fois la jeune femme était moins pressée que la première de rentrer dans sa chambre. 

Enfin la lumière disparut. 

Avec cette lueur s'éteignit la dernière irrésolution dans le cœur de d'Artagnan; il se rappela les détails de la première nuit, et, le cœur bondissant, la tête en feu, il rentra dans l'hôtel et se précipita dans la chambre de Ketty. 

La jeune fille, pâle comme la mort, tremblant de tous ses membres, voulut arrêter son amant; mais Milady, l'oreille au guet, avait entendu le bruit qu'avait fait d'Artagnan: elle ouvrit la porte. 

«Venez», dit-elle. 

Tout cela était d'une si incroyable imprudence, d'une si monstrueuse effronterie, qu'à peine si d'Artagnan pouvait croire à ce qu'il voyait et à ce qu'il entendait. Il croyait être entraîné dans quelqu'une de ces intrigues fantastiques comme on en accomplit en rêve. 

Il ne s'élança pas moins vers Milady, cédant à cette attraction que l'aimant exerce sur le fer. La porte se referma derrière eux. 

Ketty s'élança à son tour contre la porte. 

La jalousie, la fureur, l'orgueil offensé, toutes les passions enfin qui se disputent le cœur d'une femme amoureuse la poussaient à une révélation; mais elle était perdue si elle avouait avoir donné les mains à une pareille machination; et, par-dessus tout, d'Artagnan était perdu pour elle. Cette dernière pensée d'amour lui conseilla encore ce dernier sacrifice. 

D'Artagnan, de son côté, était arrivé au comble de tous ses voeux: ce n'était plus un rival qu'on aimait en lui, c'était lui-même qu'on avait l'air d'aimer. Une voix secrète lui disait bien au fond du cœur qu'il n'était qu'un instrument de vengeance que l'on caressait en attendant qu'il donnât la mort, mais l'orgueil, mais l'amour-propre, mais la folie faisaient taire cette voix, étouffaient ce murmure. Puis notre Gascon, avec la dose de confiance que nous lui connaissons, se comparait à de Wardes et se demandait pourquoi, au bout du compte, on ne l'aimerait pas, lui aussi, pour lui-même. 

Il s'abandonna donc tout entier aux sensations du moment. Milady ne fut plus pour lui cette femme aux intentions fatales qui l'avait un instant épouvanté, ce fut une maîtresse ardente et passionnée s'abandonnant tout entière à un amour qu'elle semblait éprouver elle-même. Deux heures à peu près s'écoulèrent ainsi. 

Cependant les transports des deux amants se calmèrent; Milady, qui n'avait point les mêmes motifs que d'Artagnan pour oublier, revint la première à la réalité et demanda au jeune homme si les mesures qui devaient amener le lendemain entre lui et de Wardes une rencontre étaient bien arrêtées d'avance dans son esprit. 

Mais d'Artagnan, dont les idées avaient pris un tout autre cours, s'oublia comme un sot et répondit galamment qu'il était bien tard pour s'occuper de duels à coups d'épée. 

Cette froideur pour les seuls intérêts qui l'occupassent effraya Milady, dont les questions devinrent plus pressantes. 

Alors d'Artagnan, qui n'avait jamais sérieusement pensé à ce duel impossible, voulut détourner la conversation, mais il n'était plus de force. 

Milady le contint dans les limites qu'elle avait tracées d'avance avec son esprit irrésistible et sa volonté de fer. 

D'Artagnan se crut fort spirituel en conseillant à Milady de renoncer, en pardonnant à de Wardes, aux projets furieux qu'elle avait formés. 

Mais aux premiers mots qu'il dit, la jeune femme tressaillit et s'éloigna. 

«Auriez-vous peur, cher d'Artagnan? dit-elle d'une voix aiguë et railleuse qui résonna étrangement dans l'obscurité. 

\speak  Vous ne le pensez pas, chère âme! répondit d'Artagnan; mais enfin, si ce pauvre comte de Wardes était moins coupable que vous ne le pensez? 

\speak  En tout cas dit gravement Milady, il m'a trompée, et du moment où il m'a trompée il a mérité la mort. 

\speak  Il mourra donc, puisque vous le condamnez!» dit d'Artagnan d'un ton si ferme, qu'il parut à Milady l'expression d'un dévouement à toute épreuve. 

Aussitôt elle se rapprocha de lui. 

Nous ne pourrions dire le temps que dura la nuit pour Milady; mais d'Artagnan croyait être près d'elle depuis deux heures à peine lorsque le jour parut aux fentes des jalousies et bientôt envahit la chambre de sa lueur blafarde. 

Alors Milady, voyant que d'Artagnan allait la quitter, lui rappela la promesse qu'il lui avait faite de la venger de de Wardes. 

«Je suis tout prêt, dit d'Artagnan, mais auparavant je voudrais être certain d'une chose. 

\speak  De laquelle? demanda Milady. 

\speak  C'est que vous m'aimez. 

\speak  Je vous en ai donné la preuve, ce me semble. 

\speak  Oui, aussi je suis à vous corps et âme. 

\speak  Merci, mon brave amant! mais de même que je vous ai prouvé mon amour, vous me prouverez le vôtre à votre tour, n'est-ce pas? 

\speak  Certainement. Mais si vous m'aimez comme vous me le dites, reprit d'Artagnan, ne craignez-vous pas un peu pour moi? 

\speak  Que puis-je craindre? 

\speak  Mais enfin, que je sois blessé dangereusement, tué même. 

\speak  Impossible, dit Milady, vous êtes un homme si vaillant et une si fine épée. 

\speak  Vous ne préféreriez donc point, reprit d'Artagnan, un moyen qui vous vengerait de même tout en rendant inutile le combat.» 

Milady regarda son amant en silence: cette lueur blafarde des premiers rayons du jour donnait à ses yeux clairs une expression étrangement funeste. 

«Vraiment, dit-elle, je crois que voilà que vous hésitez maintenant. 

\speak  Non, je n'hésite pas; mais c'est que ce pauvre comte de Wardes me fait vraiment peine depuis que vous ne l'aimez plus, et il me semble qu'un homme doit être si cruellement puni par la perte seule de votre amour, qu'il n'a pas besoin d'autre châtiment. 

\speak  Qui vous dit que je l'aie aimé? demanda Milady. 

\speak  Au moins puis-je croire maintenant sans trop de fatuité que vous en aimez un autre, dit le jeune homme d'un ton caressant, et je vous le répète, je m'intéresse au comte. 

\speak  Vous? demanda Milady. 

\speak  Oui moi. 

\speak  Et pourquoi vous? 

\speak  Parce que seul je sais\dots 

\speak  Quoi? 

\speak  Qu'il est loin d'être ou plutôt d'avoir été aussi coupable envers vous qu'il le paraît. 

\speak  En vérité! dit Milady d'un air inquiet; expliquez-vous, car je ne sais vraiment ce que vous voulez dire.» 

Et elle regardait d'Artagnan, qui la tenait embrassée avec des yeux qui semblaient s'enflammer peu à peu. 

«Oui, je suis galant homme, moi! dit d'Artagnan décidé à en finir; et depuis que votre amour est à moi, que je suis bien sûr de le posséder, car je le possède, n'est-ce pas?\dots 

\speak  Tout entier, continuez. 

\speak  Eh bien, je me sens comme transporté, un aveu me pèse. 

\speak  Un aveu? 

\speak  Si j'eusse douté de votre amour je ne l'eusse pas fait; mais vous m'aimez, ma belle maîtresse? n'est-ce pas, vous m'aimez? 

\speak  Sans doute. 

\speak  Alors si par excès d'amour je me suis rendu coupable envers vous, vous me pardonnerez? 

\speak  Peut-être!» 

D'Artagnan essaya, avec le plus doux sourire qu'il pût prendre, de rapprocher ses lèvres des lèvres de Milady, mais celle-ci l'écarta. 

«Cet aveu, dit-elle en pâlissant, quel est cet aveu? 

\speak  Vous aviez donné rendez-vous à de Wardes, jeudi dernier, dans cette même chambre, n'est-ce pas? 

\speak  Moi, non! cela n'est pas, dit Milady d'un ton de voix si ferme et d'un visage si impassible, que si d'Artagnan n'eût pas eu une certitude si parfaite, il eût douté. 

\speak  Ne mentez pas, mon bel ange, dit d'Artagnan en souriant, ce serait inutile. 

\speak  Comment cela? parlez donc! vous me faites mourir! 

\speak  Oh! rassurez-vous, vous n'êtes point coupable envers moi, et je vous ai déjà pardonné! 

\speak  Après, après? 

\speak  De Wardes ne peut se glorifier de rien. 

\speak  Pourquoi? Vous m'avez dit vous-même que cette bague\dots 

\speak  Cette bague, mon amour, c'est moi qui l'ai. Le comte de Wardes de jeudi et le d'Artagnan d'aujourd'hui sont la même personne.» 

L'imprudent s'attendait à une surprise mêlée de pudeur, à un petit orage qui se résoudrait en larmes; mais il se trompait étrangement, et son erreur ne fut pas longue. 

Pâle et terrible, Milady se redressa, et, repoussant d'Artagnan d'un violent coup dans la poitrine, elle s'élança hors du lit. 

Il faisait alors presque grand jour. 

D'Artagnan la retint par son peignoir de fine toile des Indes pour implorer son pardon; mais elle, d'un mouvement puissant et résolu, elle essaya de fuir. Alors la batiste se déchira en laissant à nu les épaules et sur l'une de ces belles épaules rondes et blanches, d'Artagnan avec un saisissement inexprimable, reconnut la fleur de lis, cette marque indélébile qu'imprime la main infamante du bourreau. 

«Grand Dieu!» s'écria d'Artagnan en lâchant le peignoir. 

Et il demeura muet, immobile et glacé sur le lit. 

Mais Milady se sentait dénoncée par l'effroi même de d'Artagnan. Sans doute il avait tout vu: le jeune homme maintenant savait son secret, secret terrible, que tout le monde ignorait, excepté lui. 

Elle se retourna, non plus comme une femme furieuse mais comme une panthère blessée. 

«Ah! misérable, dit-elle, tu m'as lâchement trahie, et de plus tu as mon secret! Tu mourras!» 

Et elle courut à un coffret de marqueterie posé sur la toilette, l'ouvrit d'une main fiévreuse et tremblante, en tira un petit poignard à manche d'or, à la lame aiguë et mince et revint d'un bond sur d'Artagnan à demi nu. 

Quoique le jeune homme fût brave, on le sait, il fut épouvanté de cette figure bouleversée, de ces pupilles dilatées horriblement, de ces joues pâles et de ces lèvres sanglantes; il recula jusqu'à la ruelle, comme il eût fait à l'approche d'un serpent qui eût rampé vers lui, et son épée se rencontrant sous sa main souillée de sueur, il la tira du fourreau. 

Mais sans s'inquiéter de l'épée, Milady essaya de remonter sur le lit pour le frapper, et elle ne s'arrêta que lorsqu'elle sentit la pointe aiguë sur sa gorge. 

Alors elle essaya de saisir cette épée avec les mains mais d'Artagnan l'écarta toujours de ses étreintes et, la lui présentant tantôt aux yeux, tantôt à la poitrine, il se laissa glisser à bas du lit, cherchant pour faire retraite la porte qui conduisait chez Ketty. 

Milady, pendant ce temps, se ruait sur lui avec d'horribles transports, rugissant d'une façon formidable. 

Cependant cela ressemblait à un duel, aussi d'Artagnan se remettait petit à petit. 

«Bien, belle dame, bien! disait-il, mais, de par Dieu, calmez-vous, ou je vous dessine une seconde fleur de lis sur l'autre épaule. 

\speak  Infâme! infâme!» hurlait Milady. 

Mais d'Artagnan, cherchant toujours la porte, se tenait sur la défensive. 

Au bruit qu'ils faisaient, elle renversant les meubles pour aller à lui, lui s'abritant derrière les meubles pour se garantir d'elle, Ketty ouvrit la porte. D'Artagnan, qui avait sans cesse manoeuvré pour se rapprocher de cette porte, n'en était plus qu'à trois pas. D'un seul élan il s'élança de la chambre de Milady dans celle de la suivante, et, rapide comme l'éclair, il referma la porte, contre laquelle il s'appuya de tout son poids tandis que Ketty poussait les verrous. 

Alors Milady essaya de renverser l'arc-boutant qui l'enfermait dans sa chambre, avec des forces bien au-dessus de celles d'une femme; puis, lorsqu'elle sentit que c'était chose impossible, elle cribla la porte de coups de poignard, dont quelques-uns traversèrent l'épaisseur du bois. 

Chaque coup était accompagné d'une imprécation terrible. 

«Vite, vite, Ketty, dit d'Artagnan à demi-voix lorsque les verrous furent mis, fais-moi sortir de l'hôtel, ou si nous lui laissons le temps de se retourner, elle me fera tuer par les laquais. 

\speak  Mais vous ne pouvez pas sortir ainsi, dit Ketty, vous êtes tout nu. 

\speak  C'est vrai, dit d'Artagnan, qui s'aperçut alors seulement du costume dans lequel il se trouvait, c'est vrai; habille-moi comme tu pourras, mais hâtons-nous; comprends-tu, il y va de la vie et de la mort!» 

Ketty ne comprenait que trop; en un tour de main elle l'affubla d'une robe à fleurs, d'une large coiffe et d'un mantelet; elle lui donna des pantoufles, dans lesquelles il passa ses pieds nus, puis elle l'entraîna par les degrés. Il était temps, Milady avait déjà sonné et réveillé tout l'hôtel. Le portier tira le cordon à la voix de Ketty au moment même où Milady, à demi nue de son côté, criait par la fenêtre: 

«N'ouvrez pas!» 
%!TeX root=../musketeersfr.tex 

\chapter[Comment Athos Trouva Son Équipement]{Comment, Sans Se Déranger, Athos Trouva Son Équipement}

\lettrine{L}{e} jeune homme s'enfuit tandis qu'elle le menaçait encore d'un geste impuissant. Au moment où elle le perdit de vue, Milady tomba évanouie dans sa chambre. 

\zz
D'Artagnan était tellement bouleversé, que, sans s'inquiéter de ce que deviendrait Ketty, il traversa la moitié de Paris tout en courant, et ne s'arrêta que devant la porte d'Athos. L'égarement de son esprit, la terreur qui l'éperonnait, les cris de quelques patrouilles qui se mirent à sa poursuite, et les huées de quelques passants qui, malgré l'heure peu avancée, se rendaient à leurs affaires, ne firent que précipiter sa course. 

Il traversa la cour, monta les deux étages d'Athos et frappa à la porte à tout rompre. 

Grimaud vint ouvrir les yeux bouffis de sommeil. D'Artagnan s'élança avec tant de force dans l'antichambre qu'il faillit le culbuter en entrant. 

Malgré le mutisme habituel du pauvre garçon, cette fois la parole lui revint. 

«Hé, là, là! s'écria-t-il, que voulez-vous, coureuse? que demandez-vous, drôlesse?» 

D'Artagnan releva ses coiffes et dégagea ses mains de dessous son mantelet; à la vue de ses moustaches et de son épée nue, le pauvre diable s'aperçut qu'il avait affaire à un homme. 

Il crut alors que c'était quelque assassin. 

«Au secours! à l'aide! au secours! s'écria-t-il. 

\speak  Tais-toi, malheureux! dit le jeune homme, je suis d'Artagnan, ne me reconnais-tu pas? Où est ton maître? 

\speak  Vous, monsieur d'Artagnan! s'écria Grimaud épouvanté. Impossible. 

\speak  Grimaud, dit Athos sortant de son appartement en robe de chambre, je crois que vous vous permettez de parler. 

\speak  Ah! monsieur! c'est que\dots 

\speak  Silence.» 

Grimaud se contenta de montrer du doigt d'Artagnan à son maître. 

Athos reconnut son camarade, et, tout flegmatique qu'il était, il partit d'un éclat de rire que motivait bien la mascarade étrange qu'il avait sous les yeux: coiffes de travers, jupes tombantes sur les souliers; manches retroussées et moustaches raides d'émotion. 

«Ne riez pas, mon ami, s'écria d'Artagnan; de par le Ciel ne riez pas, car, sur mon âme, je vous le dis, il n'y a point de quoi rire.» 

Et il prononça ces mots d'un air si solennel et avec une épouvante si vraie qu'Athos lui prit aussitôt les mains en s'écriant: 

«Seriez-vous blessé, mon ami? vous êtes bien pâle! 

\speak  Non, mais il vient de m'arriver un terrible événement. Êtes-vous seul, Athos? 

\speak  Pardieu! qui voulez-vous donc qui soit chez moi à cette heure? 

\speak  Bien, bien.» 

Et d'Artagnan se précipita dans la chambre d'Athos. 

«Hé, parlez! dit celui-ci en refermant la porte et en poussant les verrous pour n'être pas dérangés. Le roi est-il mort? avez-vous tué M. le cardinal? vous êtes tout renversé; voyons, voyons, dites, car je meurs véritablement d'inquiétude. 

\speak  Athos, dit d'Artagnan se débarrassant de ses vêtements de femme et apparaissant en chemise, préparez-vous à entendre une histoire incroyable, inouïe. 

\speak  Prenez d'abord cette robe de chambre», dit le mousquetaire à son ami. 

D'Artagnan passa la robe de chambre, prenant une manche pour une autre tant il était encore ému. 

«Eh bien? dit Athos. 

\speak  Eh bien, répondit d'Artagnan en se courbant vers l'oreille d'Athos et en baissant la voix, Milady est marquée d'une fleur de lis à l'épaule. 

\speak  Ah! cria le mousquetaire comme s'il eût reçu une balle dans le cœur. 

\speak  Voyons, dit d'Artagnan, êtes-vous sûr que l'\textit{autre} soit bien morte? 

\speak  L'\textit{autre?} dit Athos d'une voix si sourde, qu'à peine si d'Artagnan l'entendit. 

\speak  Oui, celle dont vous m'avez parlé un jour à Amiens.» 

Athos poussa un gémissement et laissa tomber sa tête dans ses mains. 

«Celle-ci, continua d'Artagnan, est une femme de vingt-six à vingt-huit ans. 

\speak  Blonde, dit Athos, n'est-ce pas? 

\speak  Oui. 

\speak  Des yeux clairs, d'une clarté étrange, avec des cils et sourcils noirs? 

\speak  Oui. 

\speak  Grande, bien faite? Il lui manque une dent près de l'œillère gauche. 

\speak  Oui. 

\speak  La fleur de lis est petite, rousse de couleur et comme effacée par les couches de pâte qu'on y applique. 

\speak  Oui. 

\speak  Cependant vous dites qu'elle est anglaise! 

\speak  On l'appelle Milady, mais elle peut être française. Malgré cela, Lord de Winter n'est que son beau-frère. 

\speak  Je veux la voir, d'Artagnan. 

\speak  Prenez garde, Athos, prenez garde; vous avez voulu la tuer, elle est femme à vous rendre la pareille et à ne pas vous manquer. 

\speak  Elle n'osera rien dire, car ce serait se dénoncer elle-même. 

\speak  Elle est capable de tout! L'avez-vous jamais vue furieuse? 

\speak  Non, dit Athos. 

\speak  Une tigresse, une panthère! Ah! mon cher Athos! j'ai bien peur d'avoir attiré sur nous deux une vengeance terrible!» 

D'Artagnan raconta tout alors: la colère insensée de Milady et ses menaces de mort. 

«Vous avez raison, et, sur mon âme, je donnerais ma vie pour un cheveu, dit Athos. Heureusement, c'est après-demain que nous quittons Paris; nous allons, selon toute probabilité, à La Rochelle, et une fois partis\dots 

\speak  Elle vous suivra jusqu'au bout du monde, Athos, si elle vous reconnaît; laissez donc sa haine s'exercer sur moi seul. 

\speak  Ah! mon cher! que m'importe qu'elle me tue! dit Athos; est-ce que par hasard vous croyez que je tiens à la vie? 

\speak  Il y a quelque horrible mystère sous tout cela, Athos! cette femme est l'espion du cardinal, j'en suis sûr! 

\speak  En ce cas, prenez garde à vous. Si le cardinal ne vous a pas dans une haute admiration pour l'affaire de Londres, il vous a en grande haine; mais comme, au bout du compte, il ne peut rien vous reprocher ostensiblement, et qu'il faut que haine se satisfasse, surtout quand c'est une haine de cardinal, prenez garde à vous! Si vous sortez, ne sortez pas seul; si vous mangez, prenez vos précautions: méfiez-vous de tout enfin, même de votre ombre. 

\speak  Heureusement, dit d'Artagnan, qu'il s'agit seulement d'aller jusqu'à après-demain soir sans encombre, car une fois à l'armée nous n'aurons plus, je l'espère, que des hommes à craindre. 

\speak  En attendant, dit Athos, je renonce à mes projets de réclusion, et je vais partout avec vous: il faut que vous retourniez rue des Fossoyeurs, je vous accompagne. 

\speak  Mais si près que ce soit d'ici, reprit d'Artagnan, je ne puis y retourner comme cela. 

\speak  C'est juste», dit Athos. Et il tira la sonnette. 

Grimaud entra. 

Athos lui fit signe d'aller chez d'Artagnan, et d'en rapporter des habits. 

Grimaud répondit par un autre signe qu'il comprenait parfaitement et partit. 

«Ah çà! mais voilà qui ne nous avance pas pour l'équipement, cher ami, dit Athos; car, si je ne m'abuse, vous avez laissé toute votre défroque chez Milady, qui n'aura sans doute pas l'attention de vous la retourner. Heureusement que vous avez le saphir. 

\speak  Le saphir est à vous, mon cher Athos! ne m'avez-vous pas dit que c'était une bague de famille? 

\speak  Oui, mon père l'acheta deux mille écus, à ce qu'il me dit autrefois; il faisait partie des cadeaux de noces qu'il fit à ma mère; et il est magnifique. Ma mère me le donna, et moi, fou que j'étais, plutôt que de garder cette bague comme une relique sainte, je la donnai à mon tour à cette misérable. 

\speak  Alors, mon cher, reprenez cette bague, à laquelle je comprends que vous devez tenir. 

\speak  Moi, reprendre cette bague, après qu'elle a passé par les mains de l'infâme! jamais: cette bague est souillée, d'Artagnan. 

\speak  Vendez-la donc. 

\speak  Vendre un diamant qui vient de ma mère! je vous avoue que je regarderais cela comme une profanation. 

\speak  Alors engagez-la, on vous prêtera bien dessus un millier d'écus. Avec cette somme vous serez au-dessus de vos affaires, puis, au premier argent qui vous rentrera, vous la dégagerez, et vous la reprendrez lavée de ses anciennes taches, car elle aura passé par les mains des usuriers.» 

Athos sourit. 

«Vous êtes un charmant compagnon, dit-il, mon cher d'Artagnan; vous relevez par votre éternelle gaieté les pauvres esprits dans l'affliction. Eh bien, oui, engageons cette bague, mais à une condition! 

\speak  Laquelle? 

\speak  C'est qu'il y aura cinq cents écus pour vous et cinq cents écus pour moi. 

\speak  Y songez-vous, Athos? je n'ai pas besoin du quart de cette somme, moi qui suis dans les gardes, et en vendant ma selle je me la procurerai. Que me faut-il? Un cheval pour Planchet, voilà tout. Puis vous oubliez que j'ai une bague aussi. 

\speak  À laquelle vous tenez encore plus, ce me semble, que je ne tiens, moi, à la mienne; du moins j'ai cru m'en apercevoir. 

\speak  Oui, car dans une circonstance extrême elle peut nous tirer non seulement de quelque grand embarras mais encore de quelque grand danger; c'est non seulement un diamant précieux, mais c'est encore un talisman enchanté. 

\speak  Je ne vous comprends pas, mais je crois à ce que vous me dites. Revenons donc à ma bague, ou plutôt à la vôtre, vous toucherez la moitié de la somme qu'on nous donnera sur elle ou je la jette dans la Seine, et je doute que, comme à Polycrate, quelque poisson soit assez complaisant pour nous la rapporter. 

\speak  Eh bien, donc, j'accepte!» dit d'Artagnan. 

En ce moment Grimaud rentra accompagné de Planchet; celui-ci, inquiet de son maître et curieux de savoir ce qui lui était arrivé, avait profité de la circonstance et apportait les habits lui-même. 

D'Artagnan s'habilla, Athos en fit autant: puis quand tous deux furent prêts à sortir, ce dernier fit à Grimaud le signe d'un homme qui met en joue; celui-ci décrocha aussitôt son mousqueton et s'apprêta à accompagner son maître. 

Athos et d'Artagnan suivis de leurs valets arrivèrent sans incident à la rue des Fossoyeurs. Bonacieux était sur la porte, il regarda d'Artagnan d'un air goguenard. 

«Eh, mon cher locataire! dit-il, hâtez-vous donc, vous avez une belle jeune fille qui vous attend chez vous, et les femmes, vous le savez, n'aiment pas qu'on les fasse attendre! 

\speak  C'est Ketty!» s'écria d'Artagnan. 

Et il s'élança dans l'allée. 

Effectivement, sur le carré conduisant à sa chambre, et tapie contre sa porte, il trouva la pauvre enfant toute tremblante. Dès qu'elle l'aperçut: 

«Vous m'avez promis votre protection, vous m'avez promis de me sauver de sa colère, dit-elle; souvenez-vous que c'est vous qui m'avez perdue! 

\speak  Oui, sans doute, dit d'Artagnan, sois tranquille, Ketty. Mais qu'est-il arrivé après mon départ? 

\speak  Le sais-je? dit Ketty. Aux cris qu'elle a poussés, les laquais sont accourus; elle était folle de colère; tout ce qu'il existe d'imprécations elle les a vomies contre vous. Alors j'ai pensé qu'elle se rappellerait que c'était par ma chambre que vous aviez pénétré dans la sienne, et qu'alors elle songerait que j'étais votre complice; j'ai pris le peu d'argent que j'avais, mes hardes les plus précieuses, et je me suis sauvée. 

\speak  Pauvre enfant! Mais que vais-je faire de toi? Je pars après-demain. 

\speak  Tout ce que vous voudrez, Monsieur le chevalier, faites-moi quitter Paris, faites-moi quitter la France. 

\speak  Je ne puis cependant pas t'emmener avec moi au siège de La Rochelle, dit d'Artagnan. 

\speak  Non; mais vous pouvez me placer en province, chez quelque dame de votre connaissance: dans votre pays, par exemple. 

\speak  Ah! ma chère amie! dans mon pays les dames n'ont point de femmes de chambre. Mais, attends, j'ai ton affaire. Planchet, va me chercher Aramis: qu'il vienne tout de suite. Nous avons quelque chose de très important à lui dire. 

\speak  Je comprends, dit Athos; mais pourquoi pas Porthos? Il me semble que sa marquise\dots 

\speak  La marquise de Porthos se fait habiller par les clercs de son mari, dit d'Artagnan en riant. D'ailleurs Ketty ne voudrait pas demeurer rue aux Ours, n'est-ce pas, Ketty? 

\speak  Je demeurerai où l'on voudra, dit Ketty, pourvu que je sois bien cachée et que l'on ne sache pas où je suis. 

\speak  Maintenant, Ketty, que nous allons nous séparer, et par conséquent que tu n'es plus jalouse de moi\dots 

\speak  Monsieur le chevalier, de loin ou de près, dit Ketty, je vous aimerai toujours.» 

«Où diable la constance va-t-elle se nicher?» murmura Athos. 

«Moi aussi, dit d'Artagnan, moi aussi, je t'aimerai toujours, sois tranquille. Mais voyons, réponds-moi. Maintenant j'attache une grande importance à la question que je te fais: n'aurais-tu jamais entendu parler d'une jeune dame qu'on aurait enlevée pendant une nuit. 

\speak  Attendez donc\dots Oh! mon Dieu! monsieur le chevalier, est-ce que vous aimez encore cette femme? 

\speak  Non, c'est un de mes amis qui l'aime. Tiens, c'est Athos que voilà. 

\speak  Moi! s'écria Athos avec un accent pareil à celui d'un homme qui s'aperçoit qu'il va marcher sur une couleuvre. 

\speak  Sans doute, vous! fit d'Artagnan en serrant la main d'Athos. Vous savez bien l'intérêt que nous prenons tous à cette pauvre petite Mme Bonacieux. D'ailleurs Ketty ne dira rien: n'est-ce pas, Ketty? Tu comprends, mon enfant, continua d'Artagnan, c'est la femme de cet affreux magot que tu as vu sur le pas de la porte en entrant ici. 

\speak  Oh! mon Dieu! s'écria Ketty, vous me rappelez ma peur; pourvu qu'il ne m'ait pas reconnue! 

\speak  Comment, reconnue! tu as donc déjà vu cet homme? 

\speak  Il est venu deux fois chez Milady. 

\speak  C'est cela. Vers quelle époque? 

\speak  Mais il y a quinze ou dix-huit jours à peu près. 

\speak  Justement. 

\speak  Et hier soir il est revenu. 

\speak  Hier soir. 

\speak  Oui, un instant avant que vous vinssiez vous-même. 

\speak  Mon cher Athos, nous sommes enveloppés dans un réseau d'espions! Et tu crois qu'il t'a reconnue, Ketty? 

\speak  J'ai baissé ma coiffe en l'apercevant, mais peut-être était-il trop tard. 

\speak  Descendez, Athos, vous dont il se méfie moins que de moi, et voyez s'il est toujours sur sa porte.» 

Athos descendit et remonta bientôt. 

«Il est parti, dit-il, et la maison est fermée. 

\speak  Il est allé faire son rapport, et dire que tous les pigeons sont en ce moment au colombier. 

\speak  Eh bien, mais, envolons-nous, dit Athos, et ne laissons ici que Planchet pour nous rapporter les nouvelles. 

\speak  Un instant! Et Aramis que nous avons envoyé chercher! 

\speak  C'est juste, dit Athos, attendons Aramis. 

En ce moment Aramis entra. 

On lui exposa l'affaire, et on lui dit comment il était urgent que parmi toutes ses hautes connaissances il trouvât une place à Ketty. 

Aramis réfléchit un instant, et dit en rougissant: 

«Cela vous rendra-t-il bien réellement service, d'Artagnan? 

\speak  Je vous en serai reconnaissant toute ma vie. 

\speak  Eh bien, Mme de Bois-Tracy m'a demandé, pour une de ses amies qui habite la province, je crois, une femme de chambre sûre; et si vous pouvez, mon cher d'Artagnan, me répondre de mademoiselle\dots 

\speak  Oh! monsieur, s'écria Ketty, je serai toute dévouée, soyez-en certain, à la personne qui me donnera les moyens de quitter Paris. 

\speak  Alors, dit Aramis, cela va pour le mieux.» 

Il se mit à une table et écrivit un petit mot qu'il cacheta avec une bague, et donna le billet à Ketty. 

«Maintenant, mon enfant, dit d'Artagnan, tu sais qu'il ne fait pas meilleur ici pour nous que pour toi. Ainsi séparons-nous. Nous nous retrouverons dans des jours meilleurs. 

\speak  Et dans quelque temps que nous nous retrouvions et dans quelque lieu que ce soit, dit Ketty, vous me retrouverez vous aimant encore comme je vous aime aujourd'hui.» 

«Serment de joueur», dit Athos pendant que d'Artagnan allait reconduire Ketty sur l'escalier. 

Un instant après, les trois jeunes gens se séparèrent en prenant rendez-vous à quatre heures chez Athos et en laissant Planchet pour garder la maison. 

Aramis rentra chez lui, et Athos et d'Artagnan s'inquiétèrent du placement du saphir. 

Comme l'avait prévu notre Gascon, on trouva facilement trois cents pistoles sur la bague. De plus, le juif annonça que si on voulait la lui vendre, comme elle lui ferait un pendant magnifique pour des boucles d'oreilles, il en donnerait jusqu'à cinq cents pistoles. 

Athos et d'Artagnan, avec l'activité de deux soldats et la science de deux connaisseurs, mirent trois heures à peine à acheter tout l'équipement du mousquetaire. D'ailleurs Athos était de bonne composition et grand seigneur jusqu'au bout des ongles. Chaque fois qu'une chose lui convenait, il payait le prix demandé sans essayer même d'en rabattre. D'Artagnan voulait bien là-dessus faire ses observations, mais Athos lui posait la main sur l'épaule en souriant, et d'Artagnan comprenait que c'était bon pour lui, petit gentilhomme gascon, de marchander, mais non pour un homme qui avait les airs d'un prince. 

Le mousquetaire trouva un superbe cheval andalou, noir comme du jais, aux narines de feu, aux jambes fines et élégantes, qui prenait six ans. Il l'examina et le trouva sans défaut. On le lui fit mille livres. 

Peut-être l'eût-il eu pour moins; mais tandis que d'Artagnan discutait sur le prix avec le maquignon, Athos comptait les cent pistoles sur la table. 

Grimaud eut un cheval picard, trapu et fort, qui coûta trois cents livres. 

Mais la selle de ce dernier cheval et les armes de Grimaud achetées, il ne restait plus un sou des cent cinquante pistoles d'Athos. D'Artagnan offrit à son ami de mordre une bouchée dans la part qui lui revenait, quitte à lui rendre plus tard ce qu'il lui aurait emprunté. 

Mais Athos, pour toute réponse, se contenta de hausser les épaules. 

«Combien le juif donnait-il du saphir pour l'avoir en toute propriété? demanda Athos. 

\speak  Cinq cents pistoles. 

\speak  C'est-à-dire, deux cents pistoles de plus; cent pistoles pour vous, cent pistoles pour moi. Mais c'est une véritable fortune, cela, mon ami, retournez chez le juif. 

\speak  Comment, vous voulez\dots 

\speak  Cette bague, décidément, me rappellerait de trop tristes souvenirs; puis nous n'aurons jamais trois cents pistoles à lui rendre, de sorte que nous perdrions deux mille livres à ce marché. Allez lui dire que la bague est à lui, d'Artagnan, et revenez avec les deux cents pistoles. 

\speak  Réfléchissez, Athos. 

\speak  L'argent comptant est cher par le temps qui court, et il faut savoir faire des sacrifices. Allez, d'Artagnan, allez; Grimaud vous accompagnera avec son mousqueton.» 

Une demi-heure après, d'Artagnan revint avec les deux mille livres et sans qu'il lui fût arrivé aucun accident. 

Ce fut ainsi qu'Athos trouva dans son ménage des ressources auxquelles il ne s'attendait pas.
%!TeX root=../musketeersfr.tex 

\chapter{Une Vision} 
	
\lettrine{\accentletter[\gravebox]{A}}{} quatre heures, les quatre amis étaient donc réunis chez Athos. Leurs préoccupations sur l'équipement avaient tout à fait disparu, et chaque visage ne conservait plus l'expression que de ses propres et secrètes inquiétudes; car derrière tout bonheur présent est cachée une crainte à venir. 

Tout à coup Planchet entra apportant deux lettres à l'adresse de d'Artagnan. 

L'une était un petit billet gentiment plié en long avec un joli cachet de cire verte sur lequel était empreinte une colombe rapportant un rameau vert. 

L'autre était une grande épître carrée et resplendissante des armes terribles de Son Éminence le cardinal-duc. 

À la vue de la petite lettre, le cœur de d'Artagnan bondit, car il avait cru reconnaître l'écriture; et quoiqu'il n'eût vu cette écriture qu'une fois, la mémoire en était restée au plus profond de son cœur. 

Il prit donc la petite épître et la décacheta vivement. 

«Promenez-vous, lui disait-on, mercredi prochain, de six heures à sept heures du soir, sur la route de Chaillot, et regardez avec soin dans les carrosses qui passeront, mais si vous tenez à votre vie et à celle des gens qui vous aiment, ne dites pas un mot, ne faites pas un mouvement qui puisse faire croire que vous avez reconnu celle qui s'expose à tout pour vous apercevoir un instant.» 

Pas de signature. 

«C'est un piège, dit Athos, n'y allez pas, d'Artagnan. 

\speak  Cependant, dit d'Artagnan, il me semble bien reconnaître l'écriture. 

\speak  Elle est peut-être contrefaite, reprit Athos; à six ou sept heures, dans ce temps-ci, la route de Chaillot est tout à fait déserte: autant que vous alliez vous promener dans la forêt de Bondy. 

\speak  Mais si nous y allions tous! dit d'Artagnan; que diable! on ne nous dévorera point tous les quatre; plus, quatre laquais; plus, les chevaux; plus, les armes. 

\speak  Puis ce sera une occasion de montrer nos équipages, dit Porthos. 

\speak  Mais si c'est une femme qui écrit, dit Aramis, et que cette femme désire ne pas être vue, songez que vous la compromettez, d'Artagnan: ce qui est mal de la part d'un gentilhomme. 

\speak  Nous resterons en arrière, dit Porthos, et lui seul s'avancera. 

\speak  Oui, mais un coup de pistolet est bientôt tiré d'un carrosse qui marche au galop. 

\speak  Bah! dit d'Artagnan, on me manquera. Nous rejoindrons alors le carrosse, et nous exterminerons ceux qui se trouvent dedans. Ce sera toujours autant d'ennemis de moins. 

\speak  Il a raison, dit Porthos; bataille; il faut bien essayer nos armes d'ailleurs. 

\speak  Bah! donnons-nous ce plaisir, dit Aramis de son air doux et nonchalant. 

\speak  Comme vous voudrez, dit Athos. 

\speak  Messieurs, dit d'Artagnan, il est quatre heures et demie, et nous avons le temps à peine d'être à six heures sur la route de Chaillot. 

\speak  Puis, si nous sortions trop tard, dit Porthos, on ne nous verrait pas, ce qui serait dommage. Allons donc nous apprêter, messieurs. 

\speak  Mais cette seconde lettre, dit Athos, vous l'oubliez; il me semble que le cachet indique cependant qu'elle mérite bien d'être ouverte: quant à moi, je vous déclare, mon cher d'Artagnan, que je m'en soucie bien plus que du petit brimborion que vous venez tout doucement de glisser sur votre cœur.» 

D'Artagnan rougit. 

«Eh bien, dit le jeune homme, voyons, messieurs, ce que me veut Son Éminence.» 

Et d'Artagnan décacheta la lettre et lut:

\begin{a4}
\vspace{-0.5cm}
\end{a4}

\begin{mail}{}{}
M. d'Artagnan, garde du roi, compagnie des Essarts, est attendu au Palais-Cardinal ce soir à huit heures. \closeletter{La Houdinière,\\ \textit{Capitaine des gardes.}}
\end{mail}

«Diable! dit Athos, voici un rendez-vous bien autrement inquiétant que l'autre. 

\speak  J'irai au second en sortant du premier, dit d'Artagnan: l'un est pour sept heures, l'autre pour huit; il y aura temps pour tout. 

\speak  Hum! je n'irais pas, dit Aramis: un galant chevalier ne peut manquer à un rendez-vous donné par une dame; mais un gentilhomme prudent peut s'excuser de ne pas se rendre chez Son Éminence, surtout lorsqu'il a quelque raison de croire que ce n'est pas pour y recevoir des compliments. 

\speak  Je suis de l'avis d'Aramis, dit Porthos. 

\speak  Messieurs, répondit d'Artagnan, j'ai déjà reçu par M. de Cavois pareille invitation de Son Éminence, je l'ai négligée, et le lendemain il m'est arrivé un grand malheur! Constance a disparu; quelque chose qui puisse advenir, j'irai. 

\speak  Si c'est un parti pris, dit Athos, faites. 

\speak  Mais la Bastille? dit Aramis. 

\speak  Bah! vous m'en tirerez, reprit d'Artagnan. 

\speak  Sans doute, reprirent Aramis et Porthos avec un aplomb admirable et comme si c'était la chose la plus simple, sans doute nous vous en tirerons; mais, en attendant, comme nous devons partir après-demain, vous feriez mieux de ne pas risquer cette Bastille. 

\speak  Faisons mieux, dit Athos, ne le quittons pas de la soirée, attendons-le chacun à une porte du palais avec trois mousquetaires derrière nous; si nous voyons sortir quelque voiture à portière fermée et à demi suspecte, nous tomberons dessus. Il y a longtemps que nous n'avons eu maille à partir avec les gardes de M. le cardinal, et M. de Tréville doit nous croire morts. 

\speak  Décidément, Athos, dit Aramis, vous étiez fait pour être général d'armée; que dites-vous du plan, messieurs? 

\speak  Admirable! répétèrent en choeur les jeunes gens. 

\speak  Eh bien, dit Porthos, je cours à l'hôtel, je préviens nos camarades de se tenir prêts pour huit heures, le rendez-vous sera sur la place du Palais-Cardinal; vous, pendant ce temps, faites seller les chevaux par les laquais. 

\speak  Mais moi, je n'ai pas de cheval, dit d'Artagnan; mais je vais en faire prendre un chez M. de Tréville. 

\speak  C'est inutile, dit Aramis, vous prendrez un des miens. 

\speak  Combien en avez-vous donc? demanda d'Artagnan. 

\speak  Trois, répondit en souriant Aramis. 

\speak  Mon cher! dit Athos, vous êtes certainement le poète le mieux monté de France et de Navarre. 

\speak  Écoutez, mon cher Aramis, vous ne saurez que faire de trois chevaux, n'est-ce pas? je ne comprends pas même que vous ayez acheté trois chevaux. 

\speak  Aussi, je n'en ai acheté que deux, dit Aramis. 

\speak  Le troisième vous est donc tombé du ciel? 

\speak  Non, le troisième m'a été amené ce matin même par un domestique sans livrée qui n'a pas voulu me dire à qui il appartenait et qui m'a affirmé avoir reçu l'ordre de son maître\dots 

\speak  Ou de sa maîtresse, interrompit d'Artagnan. 

\speak  La chose n'y fait rien, dit Aramis en rougissant\dots et qui m'a affirmé, dis-je, avoir reçu l'ordre de sa maîtresse de mettre ce cheval dans mon écurie sans me dire de quelle part il venait. 

\speak  Il n'y a qu'aux poètes que ces choses-là arrivent, reprit gravement Athos. 

\speak  Eh bien, en ce cas, faisons mieux, dit d'Artagnan; lequel des deux chevaux monterez-vous: celui que vous avez acheté, ou celui qu'on vous a donné? 

\speak  Celui que l'on m'a donné sans contredit; vous comprenez, d'Artagnan, que je ne puis faire cette injure\dots 

\speak  Au donateur inconnu, reprit d'Artagnan. 

\speak  Ou à la donatrice mystérieuse, dit Athos. 

\speak  Celui que vous avez acheté vous devient donc inutile? 

\speak  À peu près. 

\speak  Et vous l'avez choisi vous-même? 

\speak  Et avec le plus grand soin; la sûreté du cavalier, vous le savez, dépend presque toujours de son cheval! 

\speak  Eh bien, cédez-le-moi pour le prix qu'il vous a coûté! 

\speak  J'allais vous l'offrir, mon cher d'Artagnan, en vous donnant tout le temps qui vous sera nécessaire pour me rendre cette bagatelle. 

\speak  Et combien vous coûte-t-il? 

\speak  Huit cents livres. 

\speak  Voici quarante doubles pistoles, mon cher ami, dit d'Artagnan en tirant la somme de sa poche; je sais que c'est la monnaie avec laquelle on vous paie vos poèmes. 

\speak  Vous êtes donc en fonds? dit Aramis. 

\speak  Riche, richissime, mon cher!» 

Et d'Artagnan fit sonner dans sa poche le reste de ses pistoles. 

«Envoyez votre selle à l'Hôtel des Mousquetaires, et l'on vous amènera votre cheval ici avec les nôtres. 

\speak  Très bien; mais il est bientôt cinq heures, hâtons-nous.» 

Un quart d'heure après, Porthos apparut à un bout de la rue Férou sur un genet magnifique; Mousqueton le suivait sur un cheval d'Auvergne, petit, mais solide. Porthos resplendissait de joie et d'orgueil. 

En même temps Aramis apparut à l'autre bout de la rue monté sur un superbe coursier anglais; Bazin le suivait sur un cheval rouan, tenant en laisse un vigoureux mecklembourgeois: c'était la monture de d'Artagnan. 

Les deux mousquetaires se rencontrèrent à la porte: Athos et d'Artagnan les regardaient par la fenêtre. 

«Diable! dit Aramis, vous avez là un superbe cheval, mon cher Porthos. 

\speak  Oui, répondit Porthos; c'est celui qu'on devait m'envoyer tout d'abord: une mauvaise plaisanterie du mari lui a substitué l'autre; mais le mari a été puni depuis et j'ai obtenu toute satisfaction.» 

Planchet et Grimaud parurent alors à leur tour, tenant en main les montures de leurs maîtres; d'Artagnan et Athos descendirent, se mirent en selle près de leurs compagnons, et tous quatre se mirent en marche: Athos sur le cheval qu'il devait à sa femme, Aramis sur le cheval qu'il devait à sa maîtresse, Porthos sur le cheval qu'il devait à sa procureuse, et d'Artagnan sur le cheval qu'il devait à sa bonne fortune, la meilleure maîtresse qui soit. 

Les valets suivirent. 

Comme l'avait pensé Porthos, la cavalcade fit bon effet; et si Mme Coquenard s'était trouvée sur le chemin de Porthos et eût pu voir quel grand air il avait sur son beau genet d'Espagne, elle n'aurait pas regretté la saignée qu'elle avait faite au coffre-fort de son mari. 

Près du Louvre les quatre amis rencontrèrent M. de Tréville qui revenait de Saint-Germain; il les arrêta pour leur faire compliment sur leur équipage, ce qui en un instant amena autour d'eux quelques centaines de badauds. 

D'Artagnan profita de la circonstance pour parler à M. de Tréville de la lettre au grand cachet rouge et aux armes ducales; il est bien entendu que de l'autre il n'en souffla point mot. 

M. de Tréville approuva la résolution qu'il avait prise, et l'assura que, si le lendemain il n'avait pas reparu, il saurait bien le retrouver, lui, partout où il serait. 

En ce moment, l'horloge de la Samaritaine sonna six heures; les quatre amis s'excusèrent sur un rendez-vous, et prirent congé de M. de Tréville. 

Un temps de galop les conduisit sur la route de Chaillot; le jour commençait à baisser, les voitures passaient et repassaient; d'Artagnan, gardé à quelques pas par ses amis, plongeait ses regards jusqu'au fond des carrosses, et n'y apercevait aucune figure de connaissance. 

Enfin, après un quart d'heure d'attente et comme le crépuscule tombait tout à fait, une voiture apparut, arrivant au grand galop par la route de Sèvres; un pressentiment dit d'avance à d'Artagnan que cette voiture renfermait la personne qui lui avait donné rendez-vous: le jeune homme fut tout étonné lui-même de sentir son cœur battre si violemment. Presque aussitôt une tête de femme sortit par la portière, deux doigts sur la bouche, comme pour recommander le silence, ou comme pour envoyer un baiser; d'Artagnan poussa un léger cri de joie, cette femme, ou plutôt cette apparition, car la voiture était passée avec la rapidité d'une vision, était Mme Bonacieux. 

Par un mouvement involontaire, et malgré la recommandation faite, d'Artagnan lança son cheval au galop et en quelques bonds rejoignit la voiture; mais la glace de la portière était hermétiquement fermée: la vision avait disparu. 

D'Artagnan se rappela alors cette recommandation: «Si vous tenez à votre vie et à celle des personnes qui vous aiment, demeurez immobile et comme si vous n'aviez rien vu.» 

Il s'arrêta donc, tremblant non pour lui, mais pour la pauvre femme qui évidemment s'était exposée à un grand péril en lui donnant ce rendez-vous. 

La voiture continua sa route toujours marchant à fond de train, s'enfonça dans Paris et disparut. 

D'Artagnan était resté interdit à la même place et ne sachant que penser. Si c'était Mme Bonacieux et si elle revenait à Paris, pourquoi ce rendez-vous fugitif, pourquoi ce simple échange d'un coup d'œil, pourquoi ce baiser perdu? Si d'un autre côté ce n'était pas elle, ce qui était encore bien possible, car le peu de jour qui restait rendait une erreur facile, si ce n'était pas elle, ne serait-ce pas le commencement d'un coup de main monté contre lui avec l'appât de cette femme pour laquelle on connaissait son amour? 

Les trois compagnons se rapprochèrent de lui. Tous trois avaient parfaitement vu une tête de femme apparaître à la portière, mais aucun d'eux, excepté Athos, ne connaissait Mme Bonacieux. L'avis d'Athos, au reste, fut que c'était bien elle; mais moins préoccupé que d'Artagnan de ce joli visage, il avait cru voir une seconde tête, une tête d'homme au fond de la voiture. 

«S'il en est ainsi, dit d'Artagnan, ils la transportent sans doute d'une prison dans une autre. Mais que veulent-ils donc faire de cette pauvre créature, et comment la rejoindrai-je jamais? 

\speak  Ami, dit gravement Athos, rappelez-vous que les morts sont les seuls qu'on ne soit pas exposé à rencontrer sur la terre. Vous en savez quelque chose ainsi que moi, n'est-ce pas? Or, si votre maîtresse n'est pas morte, si c'est elle que nous venons de voir, vous la retrouverez un jour ou l'autre. Et peut-être, mon Dieu, ajouta-t-il avec un accent misanthropique qui lui était propre, peut être plus tôt que vous ne voudrez.» 

Sept heures et demie sonnèrent, la voiture était en retard d'une vingtaine de minutes sur le rendez-vous donné. Les amis de d'Artagnan lui rappelèrent qu'il avait une visite à faire, tout en lui faisant observer qu'il était encore temps de s'en dédire. 

Mais d'Artagnan était à la fois entêté et curieux. Il avait mis dans sa tête qu'il irait au Palais-Cardinal, et qu'il saurait ce que voulait lui dire Son Éminence. Rien ne put le faire changer de résolution. 

On arriva rue Saint-Honoré, et place du Palais-Cardinal on trouva les douze mousquetaires convoqués qui se promenaient en attendant leurs camarades. Là seulement, on leur expliqua ce dont il était question. 

D'Artagnan était fort connu dans l'honorable corps des mousquetaires du roi, où l'on savait qu'il prendrait un jour sa place; on le regardait donc d'avance comme un camarade. Il résulta de ces antécédents que chacun accepta de grand cœur la mission pour laquelle il était convié; d'ailleurs il s'agissait, selon toute probabilité, de jouer un mauvais tour à M. le cardinal et à ses gens, et pour de pareilles expéditions, ces dignes gentilshommes étaient toujours prêts. 

Athos les partagea donc en trois groupes, prit le commandement de l'un, donna le second à Aramis et le troisième à Porthos, puis chaque groupe alla s'embusquer en face d'une sortie. 

D'Artagnan, de son côté, entra bravement par la porte principale. 

Quoiqu'il se sentît vigoureusement appuyé, le jeune homme n'était pas sans inquiétude en montant pas à pas le grand escalier. Sa conduite avec Milady ressemblait tant soit peu à une trahison, et il se doutait des relations politiques qui existaient entre cette femme et le cardinal; de plus, de Wardes, qu'il avait si mal accommodé, était des fidèles de Son Éminence, et d'Artagnan savait que si Son Éminence était terrible à ses ennemis, elle était fort attachée à ses amis. 

«Si de Wardes a raconté toute notre affaire au cardinal, ce qui n'est pas douteux, et s'il m'a reconnu, ce qui est probable, je dois me regarder à peu près comme un homme condamné, disait d'Artagnan en secouant la tête. Mais pourquoi a-t-il attendu jusqu'aujourd'hui? C'est tout simple, Milady aura porté plainte contre moi avec cette hypocrite douleur qui la rend si intéressante, et ce dernier crime aura fait déborder le vase. 

«Heureusement, ajouta-t-il, mes bons amis sont en bas, et ils ne me laisseront pas emmener sans me défendre. Cependant la compagnie des mousquetaires de M. de Tréville ne peut pas faire à elle seule la guerre au cardinal, qui dispose des forces de toute la France, et devant lequel la reine est sans pouvoir et le roi sans volonté. D'Artagnan, mon ami, tu es brave, tu as d'excellentes qualités, mais les femmes te perdront!» 

Il en était à cette triste conclusion lorsqu'il entra dans l'antichambre. Il remit sa lettre à l'huissier de service qui le fit passer dans la salle d'attente et s'enfonça dans l'intérieur du palais. 

Dans cette salle d'attente étaient cinq ou six gardes de M. le cardinal, qui, reconnaissant d'Artagnan et sachant que c'était lui qui avait blessé Jussac, le regardèrent en souriant d'un singulier sourire. 

Ce sourire parut à d'Artagnan d'un mauvais augure; seulement, comme notre Gascon n'était pas facile à intimider, ou que plutôt, grâce à un grand orgueil naturel aux gens de son pays, il ne laissait pas voir facilement ce qui se passait dans son âme, quand ce qui s'y passait ressemblait à de la crainte, il se campa fièrement devant MM. les gardes et attendit la main sur la hanche, dans une attitude qui ne manquait pas de majesté. 

L'huissier rentra et fit signe à d'Artagnan de le suivre. Il sembla au jeune homme que les gardes, en le regardant s'éloigner, chuchotaient entre eux. 

Il suivit un corridor, traversa un grand salon, entra dans une bibliothèque, et se trouva en face d'un homme assis devant un bureau et qui écrivait. 

L'huissier l'introduisit et se retira sans dire une parole. D'Artagnan resta debout et examina cet homme. 

D'Artagnan crut d'abord qu'il avait affaire à quelque juge examinant son dossier, mais il s'aperçut que l'homme de bureau écrivait ou plutôt corrigeait des lignes d'inégales longueurs, en scandant des mots sur ses doigts; il vit qu'il était en face d'un poète. Au bout d'un instant, le poète ferma son manuscrit sur la couverture duquel était écrit: Mirame, \textit{tragédie en cinq actes}, et leva la tête. 

D'Artagnan reconnut le cardinal.
%!TeX root=../musketeersfr.tex 

\chapter{Le Cardinal}

\lettrine{L}{e} cardinal appuya son coude sur son manuscrit, sa joue sur sa main, et regarda un instant le jeune homme. Nul n'avait l'œil plus profondément scrutateur que le cardinal de Richelieu, et d'Artagnan sentit ce regard courir par ses veines comme une fièvre. 

Cependant il fit bonne contenance, tenant son feutre à la main, et attendant le bon plaisir de Son Éminence, sans trop d'orgueil, mais aussi sans trop d'humilité. 

«Monsieur, lui dit le cardinal, êtes-vous un d'Artagnan du Béarn? 

\speak  Oui, Monseigneur, répondit le jeune homme. 

\speak  Il y a plusieurs branches de d'Artagnan à Tarbes et dans les environs, dit le cardinal, à laquelle appartenez-vous? 

\speak  Je suis le fils de celui qui a fait les guerres de religion avec le grand roi Henri, père de Sa Gracieuse Majesté. 

\speak  C'est bien cela. C'est vous qui êtes parti, il y a sept à huit mois à peu près, de votre pays, pour venir chercher fortune dans la capitale? 

\speak  Oui, Monseigneur. 

\speak  Vous êtes venu par Meung, où il vous est arrivé quelque chose, je ne sais plus trop quoi, mais enfin quelque chose. 

Monseigneur, dit d'Artagnan, voici ce qui m'est arrivé\dots 

\speak  Inutile, inutile, reprit le cardinal avec un sourire qui indiquait qu'il connaissait l'histoire aussi bien que celui qui voulait la lui raconter; vous étiez recommandé à M. de Tréville, n'est-ce pas? 

\speak  Oui, Monseigneur; mais justement, dans cette malheureuse affaire de Meung\dots 

\speak  La lettre avait été perdue, reprit l'Éminence; oui, je sais cela; mais M. de Tréville est un habile physionomiste qui connaît les hommes à la première vue, et il vous a placé dans la compagnie de son beau-frère, M. des Essarts, en vous laissant espérer qu'un jour ou l'autre vous entreriez dans les mousquetaires. 

\speak  Monseigneur est parfaitement renseigné, dit d'Artagnan. 

Depuis ce temps-là, il vous est arrivé bien des choses: vous vous êtes promené derrière les Chartreux, un jour qu'il eût mieux valu que vous fussiez ailleurs; puis, vous avez fait avec vos amis un voyage aux eaux de Forges; eux se sont arrêtés en route; mais vous, vous avez continué votre chemin. C'est tout simple, vous aviez des affaires en Angleterre. 

\speak  Monseigneur, dit d'Artagnan tout interdit, j'allais\dots 

\speak  À la chasse, à Windsor, ou ailleurs, cela ne regarde personne. Je sais cela, moi, parce que mon état est de tout savoir. À votre retour, vous avez été reçu par une auguste personne, et je vois avec plaisir que vous avez conservé le souvenir qu'elle vous a donné.» 

\speak  D'Artagnan porta la main au diamant qu'il tenait de la reine, et en tourna vivement le chaton en dedans; mais il était trop tard. 

«Le lendemain de ce jour vous avez reçu la visite de Cavois, reprit le cardinal; il allait vous prier de passer au palais; cette visite vous ne la lui avez pas rendue, et vous avez eu tort. 

\speak  Monseigneur, je craignais d'avoir encouru la disgrâce de Votre Éminence. 

\speak  Eh! pourquoi cela, monsieur? pour avoir suivi les ordres de vos supérieurs avec plus d'intelligence et de courage que ne l'eût fait un autre, encourir ma disgrâce quand vous méritiez des éloges! Ce sont les gens qui n'obéissent pas que je punis, et non pas ceux qui, comme vous, obéissent\dots trop bien\dots Et, la preuve, rappelez-vous la date du jour où je vous avais fait dire de me venir voir, et cherchez dans votre mémoire ce qui est arrivé le soir même.» 

C'était le soir même qu'avait eu lieu l'enlèvement de Mme Bonacieux. D'Artagnan frissonna; et il se rappela qu'une demi-heure auparavant la pauvre femme était passée près de lui, sans doute encore emportée par la même puissance qui l'avait fait disparaître. 

«Enfin, continua le cardinal, comme je n'entendais pas parler de vous depuis quelque temps, j'ai voulu savoir ce que vous faisiez. D'ailleurs, vous me devez bien quelque remerciement: vous avez remarqué vous-même combien vous avez été ménagé dans toutes les circonstances. 

D'Artagnan s'inclina avec respect. 

«Cela, continua le cardinal, partait non seulement d'un sentiment d'équité naturelle, mais encore d'un plan que je m'étais tracé à votre égard. 

D'Artagnan était de plus en plus étonné. 

«Je voulais vous exposer ce plan le jour où vous reçûtes ma première invitation; mais vous n'êtes pas venu. Heureusement, rien n'est perdu pour ce retard, et aujourd'hui vous allez l'entendre. Asseyez-vous là, devant moi, monsieur d'Artagnan: vous êtes assez bon gentilhomme pour ne pas écouter debout.» 

Et le cardinal indiqua du doigt une chaise au jeune homme, qui était si étonné de ce qui se passait, que, pour obéir, il attendit un second signe de son interlocuteur. 

«Vous êtes brave, monsieur d'Artagnan, continua l'Éminence; vous êtes prudent, ce qui vaut mieux. J'aime les hommes de tête et de cœur, moi; ne vous effrayez pas, dit-il en souriant, par les hommes de cœur, j'entends les hommes de courage; mais, tout jeune que vous êtes, et à peine entrant dans le monde, vous avez des ennemis puissants: si vous n'y prenez garde, ils vous perdront! 

\speak  Hélas! Monseigneur, répondit le jeune homme, ils le feront bien facilement, sans doute; car ils sont forts et bien appuyés, tandis que moi je suis seul! 

\speak  Oui, c'est vrai; mais, tout seul que vous êtes, vous avez déjà fait beaucoup, et vous ferez encore plus, je n'en doute pas. Cependant, vous avez, je le crois, besoin d'être guidé dans l'aventureuse carrière que vous avez entreprise; car, si je ne me trompe, vous êtes venu à Paris avec l'ambitieuse idée de faire fortune. 

\speak  Je suis dans l'âge des folles espérances, Monseigneur, dit d'Artagnan. 

\speak  Il n'y a de folles espérances que pour les sots, monsieur, et vous êtes homme d'esprit. Voyons, que diriez-vous d'une enseigne dans mes gardes, et d'une compagnie après la campagne? 

\speak  Ah! Monseigneur! 

\speak  Vous acceptez, n'est-ce pas? 

\speak  Monseigneur, reprit d'Artagnan d'un air embarrassé. 

\speak  Comment, vous refusez? s'écria le cardinal avec étonnement. 

\speak  Je suis dans les gardes de Sa Majesté, Monseigneur, et je n'ai point de raisons d'être mécontent. 

\speak  Mais il me semble, dit l'Éminence, que mes gardes, à moi, sont aussi les gardes de Sa Majesté, et que, pourvu qu'on serve dans un corps français, on sert le roi. 

\speak  Monseigneur, Votre Éminence a mal compris mes paroles. 

\speak  Vous voulez un prétexte, n'est-ce pas? Je comprends. Eh bien, ce prétexte, vous l'avez. L'avancement, la campagne qui s'ouvre, l'occasion que je vous offre, voilà pour le monde; pour vous, le besoin de protections sûres; car il est bon que vous sachiez, monsieur d'Artagnan, que j'ai reçu des plaintes graves contre vous, vous ne consacrez pas exclusivement vos jours et vos nuits au service du roi.» 

D'Artagnan rougit. 

«Au reste, continua le cardinal en posant la main sur une liasse de papiers, j'ai là tout un dossier qui vous concerne; mais avant de le lire, j'ai voulu causer avec vous. Je vous sais homme de résolution et vos services bien dirigés, au lieu de vous mener à mal pourraient vous rapporter beaucoup. Allons, réfléchissez, et décidez-vous. 

\speak  Votre bonté me confond, Monseigneur, répondit d'Artagnan, et je reconnais dans Votre Éminence une grandeur d'âme qui me fait petit comme un ver de terre; mais enfin, puisque Monseigneur me permet de lui parler franchement\dots» 

D'Artagnan s'arrêta. 

«Oui, parlez. 

\speak  Eh bien, je dirai à Votre Éminence que tous mes amis sont aux mousquetaires et aux gardes du roi, et que mes ennemis, par une fatalité inconcevable, sont à Votre Éminence; je serais donc mal venu ici et mal regardé là-bas, si j'acceptais ce que m'offre Monseigneur. 

\speak  Auriez-vous déjà cette orgueilleuse idée que je ne vous offre pas ce que vous valez, monsieur? dit le cardinal avec un sourire de dédain. 

\speak  Monseigneur, Votre Éminence est cent fois trop bonne pour moi, et au contraire je pense n'avoir point encore fait assez pour être digne de ses bontés. Le siège de La Rochelle va s'ouvrir, Monseigneur; je servirai sous les yeux de Votre Éminence, et si j'ai le bonheur de me conduire à ce siège de telle façon que je mérite d'attirer ses regards, eh bien, après j'aurai au moins derrière moi quelque action d'éclat pour justifier la protection dont elle voudra bien m'honorer. Toute chose doit se faire à son temps, Monseigneur; peut-être plus tard aurai-je le droit de me donner, à cette heure j'aurais l'air de me vendre. 

\speak  C'est-à-dire que vous refusez de me servir, monsieur, dit le cardinal avec un ton de dépit dans lequel perçait cependant une sorte d'estime; demeurez donc libre et gardez vos haines et vos sympathies. 

\speak  Monseigneur\dots 

\speak  Bien, bien, dit le cardinal, je ne vous en veux pas, mais vous comprenez, on a assez de défendre ses amis et de les récompenser, on ne doit rien à ses ennemis, et cependant je vous donnerai un conseil: tenez-vous bien, monsieur d'Artagnan, car, du moment que j'aurai retiré ma main de dessus vous, je n'achèterai pas votre vie pour une obole. 

\speak  J'y tâcherai, Monseigneur, répondit le Gascon avec une noble assurance. 

\speak  Songez plus tard, et à un certain moment, s'il vous arrive malheur, dit Richelieu avec intention, que c'est moi qui ai été vous chercher, et que j'ai fait ce que j'ai pu pour que ce malheur ne vous arrivât pas. 

\speak  J'aurai, quoi qu'il arrive, dit d'Artagnan en mettant la main sur sa poitrine et en s'inclinant, une éternelle reconnaissance à Votre Éminence de ce qu'elle fait pour moi en ce moment. 

\speak  Eh bien donc! comme vous l'avez dit, monsieur d'Artagnan, nous nous reverrons après la campagne; je vous suivrai des yeux; car je serai là-bas, reprit le cardinal en montrant du doigt à d'Artagnan une magnifique armure qu'il devait endosser, et à notre retour, eh bien, nous compterons! 

\speak  Ah! Monseigneur, s'écria d'Artagnan, épargnez-moi le poids de votre disgrâce; restez neutre, Monseigneur, si vous trouvez que j'agis en galant homme. 

\speak  Jeune homme, dit Richelieu, si je puis vous dire encore une fois ce que je vous ai dit aujourd'hui, je vous promets de vous le dire.» 

Cette dernière parole de Richelieu exprimait un doute terrible; elle consterna d'Artagnan plus que n'eût fait une menace, car c'était un avertissement. Le cardinal cherchait donc à le préserver de quelque malheur qui le menaçait. Il ouvrit la bouche pour répondre, mais d'un geste hautain, le cardinal le congédia. 

D'Artagnan sortit; mais à la porte le cœur fut prêt à lui manquer, et peu s'en fallut qu'il ne rentrât. Cependant la figure grave et sévère d'Athos lui apparut: s'il faisait avec le cardinal le pacte que celui-ci lui proposait, Athos ne lui donnerait plus la main, Athos le renierait. 

Ce fut cette crainte qui le retint, tant est puissante l'influence d'un caractère vraiment grand sur tout ce qui l'entoure. 

D'Artagnan descendit par le même escalier qu'il était entré, et trouva devant la porte Athos et les quatre mousquetaires qui attendaient son retour et qui commençaient à s'inquiéter. D'un mot d'Artagnan les rassura, et Planchet courut prévenir les autres postes qu'il était inutile de monter une plus longue garde, attendu que son maître était sorti sain et sauf du Palais-Cardinal. 

Rentrés chez Athos, Aramis et Porthos s'informèrent des causes de cet étrange rendez-vous; mais d'Artagnan se contenta de leur dire que M. de Richelieu l'avait fait venir pour lui proposer d'entrer dans ses gardes avec le grade d'enseigne, et qu'il avait refusé. 

«Et vous avez eu raison», s'écrièrent d'une seule voix Porthos et Aramis. 

Athos tomba dans une profonde rêverie et ne répondit rien. Mais lorsqu'il fut seul avec d'Artagnan: 

«Vous avez fait ce que vous deviez faire, d'Artagnan, dit Athos, mais peut-être avez-vous eu tort.» 

D'Artagnan poussa un soupir; car cette voix répondait à une voix secrète de son âme, qui lui disait que de grands malheurs l'attendaient. 

La journée du lendemain se passa en préparatifs de départ; d'Artagnan alla faire ses adieux à M. de Tréville. À cette heure on croyait encore que la séparation des gardes et des mousquetaires serait momentanée, le roi tenant son parlement le jour même et devant partir le lendemain. M. de Tréville se contenta donc de demander à d'Artagnan s'il avait besoin de lui, mais d'Artagnan répondit fièrement qu'il avait tout ce qu'il lui fallait. 

La nuit réunit tous les camarades de la compagnie des gardes de M. des Essarts et de la compagnie des mousquetaires de M. de Tréville, qui avaient fait amitié ensemble. On se quittait pour se revoir quand il plairait à Dieu et s'il plaisait à Dieu. La nuit fut donc des plus bruyantes, comme on peut le penser, car, en pareil cas, on ne peut combattre l'extrême préoccupation que par l'extrême insouciance. 

Le lendemain, au premier son des trompettes, les amis se quittèrent: les mousquetaires coururent à l'hôtel de M. de Tréville, les gardes à celui de M. des Essarts. Chacun des capitaines conduisit aussitôt sa compagnie au Louvre, où le roi passait sa revue. 

Le roi était triste et paraissait malade, ce qui lui ôtait un peu de sa haute mine. En effet, la veille, la fièvre l'avait pris au milieu du parlement et tandis qu'il tenait son lit de justice. Il n'en était pas moins décidé à partir le soir même; et, malgré les observations qu'on lui avait faites, il avait voulu passer sa revue, espérant, par le premier coup de vigueur, vaincre la maladie qui commençait à s'emparer de lui. 

La revue passée, les gardes se mirent seuls en marche, les mousquetaires ne devant partir qu'avec le roi, ce qui permit à Porthos d'aller faire, dans son superbe équipage, un tour dans la rue aux Ours. 

La procureuse le vit passer dans son uniforme neuf et sur son beau cheval. Elle aimait trop Porthos pour le laisser partir ainsi; elle lui fit signe de descendre et de venir auprès d'elle. Porthos était magnifique; ses éperons résonnaient, sa cuirasse brillait, son épée lui battait fièrement les jambes. Cette fois les clercs n'eurent aucune envie de rire, tant Porthos avait l'air d'un coupeur d'oreilles. 

Le mousquetaire fut introduit près de M. Coquenard, dont le petit œil gris brilla de colère en voyant son cousin tout flambant neuf. Cependant une chose le consola intérieurement; c'est qu'on disait partout que la campagne serait rude: il espérait tout doucement, au fond du cœur, que Porthos y serait tué. 

Porthos présenta ses compliments à maître Coquenard et lui fit ses adieux; maître Coquenard lui souhaita toutes sortes de prospérités. Quant à Mme Coquenard, elle ne pouvait retenir ses larmes; mais on ne tira aucune mauvaise conséquence de sa douleur, on la savait fort attachée à ses parents, pour lesquels elle avait toujours eu de cruelles disputes avec son mari. 

Mais les véritables adieux se firent dans la chambre de Mme Coquenard: ils furent déchirants. 

Tant que la procureuse put suivre des yeux son amant, elle agita un mouchoir en se penchant hors de la fenêtre, à croire qu'elle voulait se précipiter. Porthos reçut toutes ces marques de tendresse en homme habitué à de pareilles démonstrations. Seulement, en tournant le coin de la rue, il souleva son feutre et l'agita en signe d'adieu. 

De son côté, Aramis écrivait une longue lettre. À qui? Personne n'en savait rien. Dans la chambre voisine, Ketty, qui devait partir le soir même pour Tours, attendait cette lettre mystérieuse. 

Athos buvait à petits coups la dernière bouteille de son vin d'Espagne. 

Pendant ce temps, d'Artagnan défilait avec sa compagnie. 

En arrivant au faubourg Saint-Antoine, il se retourna pour regarder gaiement la Bastille; mais, comme c'était la Bastille seulement qu'il regardait, il ne vit point Milady, qui, montée sur un cheval isabelle, le désignait du doigt à deux hommes de mauvaise mine qui s'approchèrent aussitôt des rangs pour le reconnaître. Sur une interrogation qu'ils firent du regard, Milady répondit par un signe que c'était bien lui. Puis, certaine qu'il ne pouvait plus y avoir de méprise dans l'exécution de ses ordres, elle piqua son cheval et disparut. 

Les deux hommes suivirent alors la compagnie, et, à la sortie du faubourg Saint-Antoine, montèrent sur des chevaux tout préparés qu'un domestique sans livrée tenait en les attendant.
%!TeX root=../musketeersfr.tex 

\chapter{Le Siège De La Rochelle}

\lettrine{L}{e} siège de La Rochelle fut un des grands événements politiques du règne de Louis XIII, et une des grandes entreprises militaires du cardinal. Il est donc intéressant, et même nécessaire, que nous en disions quelques mots; plusieurs détails de ce siège se liant d'ailleurs d'une manière trop importante à l'histoire que nous avons entrepris de raconter, pour que nous les passions sous silence. 

Les vues politiques du cardinal, lorsqu'il entreprit ce siège, étaient considérables. Exposons-les d'abord, puis nous passerons aux vues particulières qui n'eurent peut-être pas sur Son Éminence moins d'influence que les premières. 

Des villes importantes données par Henri IV aux huguenots comme places de sûreté, il ne restait plus que La Rochelle. Il s'agissait donc de détruire ce dernier boulevard du calvinisme, levain dangereux, auquel se venaient incessamment mêler des ferments de révolte civile ou de guerre étrangère. 

Espagnols, Anglais, Italiens mécontents, aventuriers de toute nation, soldats de fortune de toute secte accouraient au premier appel sous les drapeaux des protestants et s'organisaient comme une vaste association dont les branches divergeaient à loisir sur tous les points de l'Europe. 

La Rochelle, qui avait pris une nouvelle importance de la ruine des autres villes calvinistes, était donc le foyer des dissensions et des ambitions. Il y avait plus, son port était la dernière porte ouverte aux Anglais dans le royaume de France; et en la fermant à l'Angleterre, notre éternelle ennemie, le cardinal achevait l'oeuvre de Jeanne d'Arc et du duc de Guise. 

Aussi Bassompierre, qui était à la fois protestant et catholique, protestant de conviction et catholique comme commandeur du Saint-Esprit; Bassompierre, qui était allemand de naissance et français de cœur; Bassompierre, enfin, qui avait un commandement particulier au siège de La Rochelle, disait-il, en chargeant à la tête de plusieurs autres seigneurs protestants comme lui: 

«Vous verrez, messieurs, que nous serons assez bêtes pour prendre La Rochelle!» 

Et Bassompierre avait raison: la canonnade de l'île de Ré lui présageait les dragonnades des Cévennes; la prise de La Rochelle était la préface de la révocation de l'édit de Nantes. 

Mais nous l'avons dit, à côté de ces vues du ministre niveleur et simplificateur, et qui appartiennent à l'histoire, le chroniqueur est bien forcé de reconnaître les petites visées de l'homme amoureux et du rival jaloux. 

Richelieu, comme chacun sait, avait été amoureux de la reine; cet amour avait-il chez lui un simple but politique ou était-ce tout naturellement une de ces profondes passions comme en inspira Anne d'Autriche à ceux qui l'entouraient, c'est ce que nous ne saurions dire; mais en tout cas on a vu, par les développements antérieurs de cette histoire, que Buckingham l'avait emporté sur lui, et que, dans deux ou trois circonstances et particulièrement dans celles des ferrets, il l'avait, grâce au dévouement des trois mousquetaires et au courage de d'Artagnan, cruellement mystifié. 

Il s'agissait donc pour Richelieu, non seulement de débarrasser la France d'un ennemi, mais de se venger d'un rival; au reste, la vengeance devait être grande et éclatante, et digne en tout d'un homme qui tient dans sa main, pour épée de combat, les forces de tout un royaume. 

Richelieu savait qu'en combattant l'Angleterre il combattait Buckingham, qu'en triomphant de l'Angleterre il triomphait de Buckingham, enfin qu'en humiliant l'Angleterre aux yeux de l'Europe il humiliait Buckingham aux yeux de la reine. 

De son côté Buckingham, tout en mettant en avant l'honneur de l'Angleterre, était mû par des intérêts absolument semblables à ceux du cardinal; Buckingham aussi poursuivait une vengeance particulière: sous aucun prétexte, Buckingham n'avait pu rentrer en France comme ambassadeur, il voulait y rentrer comme conquérant. 

Il en résulte que le véritable enjeu de cette partie, que les deux plus puissants royaumes jouaient pour le bon plaisir de deux hommes amoureux, était un simple regard d'Anne d'Autriche. 

Le premier avantage avait été au duc de Buckingham: arrivé inopinément en vue de l'île de Ré avec quatre-vingt-dix vaisseaux et vingt mille hommes à peu près, il avait surpris le comte de Toiras, qui commandait pour le roi dans l'île; il avait, après un combat sanglant, opéré son débarquement. 

Relatons en passant que dans ce combat avait péri le baron de Chantal; le baron de Chantal laissait orpheline une petite fille de dix-huit mois. 

Cette petite fille fut depuis Mme de Sévigné. 

Le comte de Toiras se retira dans la citadelle Saint-Martin avec la garnison, et jeta une centaine d'hommes dans un petit fort qu'on appelait le fort de La Prée. 

Cet événement avait hâté les résolutions du cardinal; et en attendant que le roi et lui pussent aller prendre le commandement du siège de La Rochelle, qui était résolu, il avait fait partir Monsieur pour diriger les premières opérations, et avait fait filer vers le théâtre de la guerre toutes les troupes dont il avait pu disposer. 

C'était de ce détachement envoyé en avant-garde que faisait partie notre ami d'Artagnan. 

Le roi, comme nous l'avons dit, devait suivre, aussitôt son lit de justice tenu, mais en se levant de ce lit de justice, le 28 juin, il s'était senti pris par la fièvre; il n'en avait pas moins voulu partir, mais, son état empirant, il avait été forcé de s'arrêter à Villeroi. 

Or, où s'arrêtait le roi s'arrêtaient les mousquetaires; il en résultait que d'Artagnan, qui était purement et simplement dans les gardes, se trouvait séparé, momentanément du moins, de ses bons amis Athos, Porthos et Aramis; cette séparation, qui n'était pour lui qu'une contrariété, fût certes devenue une inquiétude sérieuse s'il eût pu deviner de quels dangers inconnus il était entouré. 

Il n'en arriva pas moins sans accident au camp établi devant La Rochelle, vers le 10 du mois de septembre de l'année 1627. 

Tout était dans le même état: le duc de Buckingham et ses Anglais, maîtres de l'île de Ré, continuaient d'assiéger mais sans succès, la citadelle de Saint-Martin et le fort de La Prée, et les hostilités avec La Rochelle étaient commencées depuis deux ou trois jours à propos d'un fort que le duc d'Angoulême venait de faire construire près de la ville. 

Les gardes, sous le commandement de M. des Essarts, avaient leur logement aux Minimes. 

Mais nous le savons, d'Artagnan, préoccupé de l'ambition de passer aux mousquetaires, avait rarement fait amitié avec ses camarades; il se trouvait donc isolé et livré à ses propres réflexions. 

Ses réflexions n'étaient pas riantes: depuis un an qu'il était arrivé à Paris, il s'était mêlé aux affaires publiques; ses affaires privées n'avaient pas fait grand chemin comme amour et comme fortune. 

Comme amour, la seule femme qu'il eût aimée était Mme Bonacieux, et Mme Bonacieux avait disparu sans qu'il pût découvrir encore ce qu'elle était devenue. 

Comme fortunes il s'était fait, lui chétif, ennemi du cardinal, c'est-à-dire d'un homme devant lequel tremblaient les plus grands du royaume, à commencer par le roi. 

Cet homme pouvait l'écraser, et cependant il ne l'avait pas fait: pour un esprit aussi perspicace que l'était d'Artagnan, cette indulgence était un jour par lequel il voyait dans un meilleur avenir. 

Puis, il s'était fait encore un autre ennemi moins à craindre, pensait-il, mais que cependant il sentait instinctivement n'être pas à mépriser: cet ennemi, c'était Milady. 

En échange de tout cela il avait acquis la protection et la bienveillance de la reine, mais la bienveillance de la reine était, par le temps qui courait, une cause de plus de persécution; et sa protection, on le sait, protégeait fort mal: témoins Chalais et Mme Bonacieux. 

Ce qu'il avait donc gagné de plus clair dans tout cela c'était le diamant de cinq ou six mille livres qu'il portait au doigt; et encore ce diamant, en supposant que d'Artagnan dans ses projets d'ambition, voulût le garder pour s'en faire un jour un signe de reconnaissance près de la reine n'avait en attendant, puisqu'il ne pouvait s'en défaire, pas plus de valeur que les cailloux qu'il foulait à ses pieds. 

Nous disons «que les cailloux qu'il foulait à ses pieds», car d'Artagnan faisait ces réflexions en se promenant solitairement sur un joli petit chemin qui conduisait du camp au village d'Angoutin; or ces réflexions l'avaient conduit plus loin qu'il ne croyait, et le jour commençait à baisser, lorsqu'au dernier rayon du soleil couchant il lui sembla voir briller derrière une haie le canon d'un mousquet. 

D'Artagnan avait l'œil vif et l'esprit prompt, il comprit que le mousquet n'était pas venu là tout seul et que celui qui le portait ne s'était pas caché derrière une haie dans des intentions amicales. Il résolut donc de gagner au large, lorsque de l'autre côté de la route, derrière un rocher, il aperçut l'extrémité d'un second mousquet. 

C'était évidemment une embuscade. 

Le jeune homme jeta un coup d'œil sur le premier mousquet et vit avec une certaine inquiétude qu'il s'abaissait dans sa direction, mais aussitôt qu'il vit l'orifice du canon immobile il se jeta ventre à terre. En même temps le coup partit, il entendit le sifflement d'une balle qui passait au-dessus de sa tête. 

Il n'y avait pas de temps à perdre, d'Artagnan se redressa d'un bond, et au même moment la balle de l'autre mousquet fit voler les cailloux à l'endroit même du chemin où il s'était jeté la face contre terre. 

D'Artagnan n'était pas un de ces hommes inutilement braves qui cherchent une mort ridicule pour qu'on dise d'eux qu'ils n'ont pas reculé d'un pas, d'ailleurs il ne s'agissait plus de courage ici, d'Artagnan était tombé dans un guet-apens. 

«S'il y a un troisième coup, se dit-il, je suis un homme perdu!» 

Et aussitôt prenant ses jambes à son cou, il s'enfuit dans la direction du camp, avec la vitesse des gens de son pays si renommés pour leur agilité; mais, quelle que fût la rapidité de sa course, le premier qui avait tiré, ayant eu le temps de recharger son arme, lui tira un second coup si bien ajusté, cette fois, que la balle traversa son feutre et le fit voler à dix pas de lui. 

Cependant, comme d'Artagnan n'avait pas d'autre chapeau, il ramassa le sien tout en courant, arriva fort essoufflé et fort pâle, dans son logis, s'assit sans rien dire à personne et se mit à réfléchir. 

Cet événement pouvait avoir trois causes: 

La première et la plus naturelle pouvait être une embuscade des Rochelois, qui n'eussent pas été fâchés de tuer un des gardes de Sa Majesté, d'abord parce que c'était un ennemi de moins, et que cet ennemi pouvait avoir une bourse bien garnie dans sa poche. 

D'Artagnan prit son chapeau, examina le trou de la balle, et secoua la tête. La balle n'était pas une balle de mousquet, c'était une balle d'arquebuse; la justesse du coup lui avait déjà donné l'idée qu'il avait été tiré par une arme particulière: ce n'était donc pas une embuscade militaire, puisque la balle n'était pas de calibre. 

Ce pouvait être un bon souvenir de M. le cardinal. On se rappelle qu'au moment même où il avait, grâce à ce bienheureux rayon de soleil, aperçu le canon du fusil, il s'étonnait de la longanimité de Son Éminence à son égard. 

Mais d'Artagnan secoua la tête. Pour les gens vers lesquels elle n'avait qu'à étendre la main, Son Éminence recourait rarement à de pareils moyens. 

Ce pouvait être une vengeance de Milady. 

Ceci, c'était plus probable. 

Il chercha inutilement à se rappeler ou les traits ou le costume des assassins; il s'était éloigné d'eux si rapidement, qu'il n'avait eu le loisir de rien remarquer. 

«Ah! mes pauvres amis, murmura d'Artagnan, où êtes-vous? et que vous me faites faute!» 

D'Artagnan passa une fort mauvaise nuit. Trois ou quatre fois il se réveilla en sursaut, se figurant qu'un homme s'approchait de son lit pour le poignarder. Cependant le jour parut sans que l'obscurité eût amené aucun incident. 

Mais d'Artagnan se douta bien que ce qui était différé n'était pas perdu. 

D'Artagnan resta toute la journée dans son logis; il se donna pour excuse, vis-à-vis de lui-même, que le temps était mauvais. 

Le surlendemain, à neuf heures, on battit aux champs. Le duc d'Orléans visitait les postes. Les gardes coururent aux armes, d'Artagnan prit son rang au milieu de ses camarades. 

Monsieur passa sur le front de bataille; puis tous les officiers supérieurs s'approchèrent de lui pour lui faire leur cour, M. des Essarts, le capitaine des gardes, comme les autres. 

Au bout d'un instant il parut à d'Artagnan que M. des Essarts lui faisait signe de s'approcher de lui: il attendit un nouveau geste de son supérieur, craignant de se tromper, mais ce geste s'étant renouvelé, il quitta les rangs et s'avança pour prendre l'ordre. 

«Monsieur va demander des hommes de bonne volonté pour une mission dangereuse, mais qui fera honneur à ceux qui l'auront accomplie, et je vous ai fait signe afin que vous vous tinssiez prêt. 

\speak  Merci, mon capitaine!» répondit d'Artagnan, qui ne demandait pas mieux que de se distinguer sous les yeux du lieutenant général. 

En effet, les Rochelois avaient fait une sortie pendant la nuit et avaient repris un bastion dont l'armée royaliste s'était emparée deux jours auparavant; il s'agissait de pousser une reconnaissance perdue pour voir comment l'armée gardait ce bastion. 

Effectivement, au bout de quelques instants, Monsieur éleva la voix et dit: 

«Il me faudrait, pour cette mission, trois ou quatre volontaires conduits par un homme sûr. 

\speak  Quant à l'homme sûr, je l'ai sous la main, Monseigneur, dit M. des Essarts en montrant d'Artagnan; et quant aux quatre ou cinq volontaires, Monseigneur n'a qu'à faire connaître ses intentions, et les hommes ne lui manqueront pas. 

\speak  Quatre hommes de bonne volonté pour venir se faire tuer avec moi!» dit d'Artagnan en levant son épée. 

Deux de ses camarades aux gardes s'élancèrent aussitôt, et deux soldats s'étant joints à eux, il se trouva que le nombre demandé était suffisant; d'Artagnan refusa donc tous les autres, ne voulant pas faire de passe-droit à ceux qui avaient la priorité. 

On ignorait si, après la prise du bastion, les Rochelois l'avaient évacué ou s'ils y avaient laissé garnison; il fallait donc examiner le lieu indiqué d'assez près pour vérifier la chose. 

D'Artagnan partit avec ses quatre compagnons et suivit la tranchée: les deux gardes marchaient au même rang que lui et les soldats venaient par-derrière. 

Ils arrivèrent ainsi, en se couvrant de revêtements, jusqu'à une centaine de pas du bastion! Là, d'Artagnan, en se retournant, s'aperçut que les deux soldats avaient disparu. 

Il crut qu'ayant eu peur ils étaient restés en arrière et continua d'avancer. 

Au détour de la contrescarpe, ils se trouvèrent à soixante pas à peu près du bastion. 

On ne voyait personne, et le bastion semblait abandonné. 

Les trois enfants perdus délibéraient s'ils iraient plus avant, lorsque tout à coup une ceinture de fumée ceignit le géant de pierre, et une douzaine de balles vinrent siffler autour de d'Artagnan et de ses deux compagnons. 

Ils savaient ce qu'ils voulaient savoir: le bastion était gardé. Une plus longue station dans cet endroit dangereux eût donc été une imprudence inutile; d'Artagnan et les deux gardes tournèrent le dos et commencèrent une retraite qui ressemblait à une fuite. 

En arrivant à l'angle de la tranchée qui allait leur servir de rempart, un des gardes tomba: une balle lui avait traversé la poitrine. L'autre, qui était sain et sauf, continua sa course vers le camp. 

D'Artagnan ne voulut pas abandonner ainsi son compagnon, et s'inclina vers lui pour le relever et l'aider à rejoindre les lignes; mais en ce moment deux coups de fusil partirent: une balle cassa la tête du garde déjà blessé, et l'autre vint s'aplatir sur le roc après avoir passé à deux pouces de d'Artagnan. 

Le jeune homme se retourna vivement, car cette attaque ne pouvait venir du bastion, qui était masqué par l'angle de la tranchée. L'idée des deux soldats qui l'avaient abandonné lui revint à l'esprit et lui rappela ses assassins de la surveille; il résolut donc cette fois de savoir à quoi s'en tenir, et tomba sur le corps de son camarade comme s'il était mort. 

Il vit aussitôt deux têtes qui s'élevaient au-dessus d'un ouvrage abandonné qui était à trente pas de là: c'étaient celles de nos deux soldats. D'Artagnan ne s'était pas trompé: ces deux hommes ne l'avaient suivi que pour l'assassiner, espérant que la mort du jeune homme serait mise sur le compte de l'ennemi. 

Seulement, comme il pouvait n'être que blessé et dénoncer leur crime, ils s'approchèrent pour l'achever; heureusement, trompés par la ruse de d'Artagnan, ils négligèrent de recharger leurs fusils. 

Lorsqu'ils furent à dix pas de lui, d'Artagnan, qui en tombant avait eu grand soin de ne pas lâcher son épée, se releva tout à coup et d'un bond se trouva près d'eux. 

Les assassins comprirent que s'ils s'enfuyaient du côté du camp sans avoir tué leur homme, ils seraient accusés par lui; aussi leur première idée fut-elle de passer à l'ennemi. L'un d'eux prit son fusil par le canon, et s'en servit comme d'une massue: il en porta un coup terrible à d'Artagnan, qui l'évita en se jetant de côté, mais par ce mouvement il livra passage au bandit, qui s'élança aussitôt vers le bastion. Comme les Rochelois qui le gardaient ignoraient dans quelle intention cet homme venait à eux, ils firent feu sur lui et il tomba frappé d'une balle qui lui brisa l'épaule. 

Pendant ce temps, d'Artagnan s'était jeté sur le second soldat, l'attaquant avec son épée; la lutte ne fut pas longue, ce misérable n'avait pour se défendre que son arquebuse déchargée; l'épée du garde glissa contre le canon de l'arme devenue inutile et alla traverser la cuisse de l'assassin, qui tomba. D'Artagnan lui mit aussitôt la pointe du fer sur la gorge. 

«Oh! ne me tuez pas! s'écria le bandit; grâce, grâce, mon officier! et je vous dirai tout. 

\speak  Ton secret vaut-il la peine que je te garde la vie au moins? demanda le jeune homme en retenant son bras. 

\speak  Oui; si vous estimez que l'existence soit quelque chose quand on a vingt-deux ans comme vous et qu'on peut arriver à tout, étant beau et brave comme vous l'êtes. 

\speak  Misérable! dit d'Artagnan, voyons, parle vite, qui t'a chargé de m'assassiner? 

\speak  Une femme que je ne connais pas, mais qu'on appelle Milady. 

\speak  Mais si tu ne connais pas cette femme, comment sais-tu son nom? 

\speak  Mon camarade la connaissait et l'appelait ainsi, c'est à lui qu'elle a eu affaire et non pas à moi; il a même dans sa poche une lettre de cette personne qui doit avoir pour vous une grande importance, à ce que je lui ai entendu dire. 

\speak  Mais comment te trouves-tu de moitié dans ce guet-apens? 

\speak  Il m'a proposé de faire le coup à nous deux et j'ai accepté. 

\speak  Et combien vous a-t-elle donné pour cette belle expédition? 

\speak  Cent louis. 

\speak  Eh bien, à la bonne heure, dit le jeune homme en riant, elle estime que je vaux quelque chose; cent louis! c'est une somme pour deux misérables comme vous: aussi je comprends que tu aies accepté, et je te fais grâce, mais à une condition! 

\speak  Laquelle? demanda le soldat inquiet en voyant que tout n'était pas fini. 

\speak  C'est que tu vas aller me chercher la lettre que ton camarade a dans sa poche. 

\speak  Mais, s'écria le bandit, c'est une autre manière de me tuer; comment voulez-vous que j'aille chercher cette lettre sous le feu du bastion? 

\speak  Il faut pourtant que tu te décides à l'aller chercher, ou je te jure que tu vas mourir de ma main. 

\speak  Grâce, monsieur, pitié! au nom de cette jeune dame que vous aimez, que vous croyez morte peut-être, et qui ne l'est pas! s'écria le bandit en se mettant à genoux et s'appuyant sur sa main, car il commençait à perdre ses forces avec son sang. 

\speak  Et d'où sais-tu qu'il y a une jeune femme que j'aime, et que j'ai cru cette femme morte? demanda d'Artagnan. 

\speak  Par cette lettre que mon camarade a dans sa poche. 

\speak  Tu vois bien alors qu'il faut que j'aie cette lettre, dit d'Artagnan; ainsi donc plus de retard, plus d'hésitation, ou quelle que soit ma répugnance à tremper une seconde fois mon épée dans le sang d'un misérable comme toi, je le jure par ma foi d'honnête homme\dots» 

Et à ces mots d'Artagnan fit un geste si menaçant, que le blessé se releva. 

«Arrêtez! arrêtez! s'écria-t-il reprenant courage à force de terreur, j'irai\dots j'irai!\dots» 

D'Artagnan prit l'arquebuse du soldat, le fit passer devant lui et le poussa vers son compagnon en lui piquant les reins de la pointe de son épée. 

C'était une chose affreuse que de voir ce malheureux, laissant sur le chemin qu'il parcourait une longue trace de sang, pâle de sa mort prochaine, essayant de se traîner sans être vu jusqu'au corps de son complice qui gisait à vingt pas de là! 

La terreur était tellement peinte sur son visage couvert d'une froide sueur, que d'Artagnan en eut pitié; et que, le regardant avec mépris: 

«Eh bien, lui dit-il, je vais te montrer la différence qu'il y a entre un homme de cœur et un lâche comme toi; reste, j'irai.» 

Et d'un pas agile, l'œil au guet, observant les mouvements de l'ennemi, s'aidant de tous les accidents de terrain, d'Artagnan parvint jusqu'au second soldat. 

Il y avait deux moyens d'arriver à son but: le fouiller sur la place, ou l'emporter en se faisant un bouclier de son corps, et le fouiller dans la tranchée. 

D'Artagnan préféra le second moyen et chargea l'assassin sur ses épaules au moment même où l'ennemi faisait feu. 

Une légère secousse, le bruit mat de trois balles qui trouaient les chairs, un dernier cri, un frémissement d'agonie prouvèrent à d'Artagnan que celui qui avait voulu l'assassiner venait de lui sauver la vie. 

D'Artagnan regagna la tranchée et jeta le cadavre auprès du blessé aussi pâle qu'un mort. 

Aussitôt il commença l'inventaire: un portefeuille de cuir, une bourse où se trouvait évidemment une partie de la somme que le bandit avait reçue, un cornet et des dés formaient l'héritage du mort. 

Il laissa le cornet et les dés où ils étaient tombés, jeta la bourse au blessé et ouvrit avidement le portefeuille. 

Au milieu de quelques papiers sans importance, il trouva la lettre suivante: c'était celle qu'il était allé chercher au risque de sa vie: «Puisque vous avez perdu la trace de cette femme et qu'elle est maintenant en sûreté dans ce couvent où vous n'auriez jamais dû la laisser arriver, tâchez au moins de ne pas manquer l'homme; sinon, vous savez que j'ai la main longue et que vous payeriez cher les cent louis que vous avez à moi.» 

Pas de signature. Néanmoins il était évident que la lettre venait de Milady. En conséquence, il la garda comme pièce à conviction, et, en sûreté derrière l'angle de la tranchée, il se mit à interroger le blessé. Celui-ci confessa qu'il s'était chargé avec son camarade, le même qui venait d'être tué, d'enlever une jeune femme qui devait sortir de Paris par la barrière de La Villette, mais que, s'étant arrêtés à boire dans un cabaret, ils avaient manqué la voiture de dix minutes. 

«Mais qu'eussiez-vous fait de cette femme? demanda d'Artagnan avec angoisse. 

\speak  Nous devions la remettre dans un hôtel de la place Royale, dit le blessé. 

\speak  Oui! oui! murmura d'Artagnan, c'est bien cela, chez Milady elle-même.» 

Alors le jeune homme comprit en frémissant quelle terrible soif de vengeance poussait cette femme à le perdre, ainsi que ceux qui l'aimaient, et combien elle en savait sur les affaires de la cour, puisqu'elle avait tout découvert. Sans doute elle devait ces renseignements au cardinal. 

Mais, au milieu de tout cela, il comprit, avec un sentiment de joie bien réel, que la reine avait fini par découvrir la prison où la pauvre Mme Bonacieux expiait son dévouement, et qu'elle l'avait tirée de cette prison. Alors la lettre qu'il avait reçue de la jeune femme et son passage sur la route de Chaillot, passage pareil à une apparition, lui furent expliqués. 

Dès lors, ainsi qu'Athos l'avait prédit, il était possible de retrouver Mme Bonacieux, et un couvent n'était pas imprenable. 

Cette idée acheva de lui remettre la clémence au cœur. Il se retourna vers le blessé qui suivait avec anxiété toutes les expressions diverses de son visage, et lui tendant le bras: 

«Allons, lui dit-il, je ne veux pas t'abandonner ainsi. Appuie-toi sur moi et retournons au camp. 

\speak  Oui, dit le blessé, qui avait peine à croire à tant de magnanimité, mais n'est-ce point pour me faire pendre? 

\speak  Tu as ma parole, dit-il, et pour la seconde fois je te donne la vie.» 

Le blessé se laissa glisser à genoux et baisa de nouveau les pieds de son sauveur; mais d'Artagnan, qui n'avait plus aucun motif de rester si près de l'ennemi, abrégea lui-même les témoignages de sa reconnaissance. 

Le garde qui était revenu à la première décharge des Rochelois avait annoncé la mort de ses quatre compagnons. On fut donc à la fois fort étonné et fort joyeux dans le régiment, quand on vit reparaître le jeune homme sain et sauf. 

D'Artagnan expliqua le coup d'épée de son compagnon par une sortie qu'il improvisa. Il raconta la mort de l'autre soldat et les périls qu'ils avaient courus. Ce récit fut pour lui l'occasion d'un véritable triomphe. Toute l'armée parla de cette expédition pendant un jour, et Monsieur lui en fit faire ses compliments. 

Au reste, comme toute belle action porte avec elle sa récompense, la belle action de d'Artagnan eut pour résultat de lui rendre la tranquillité qu'il avait perdue. En effet, d'Artagnan croyait pouvoir être tranquille, puisque, de ses deux ennemis, l'un était tué et l'autre dévoué à ses intérêts. 

Cette tranquillité prouvait une chose, c'est que d'Artagnan ne connaissait pas encore Milady.
%!TeX root=../musketeersfr.tex 

\chapter{Le Vin D'Anjou}

\lettrine{A}{près} des nouvelles presque désespérées du roi, le bruit de sa convalescence commençait à se répandre dans le camp; et comme il avait grande hâte d'arriver en personne au siège, on disait qu'aussitôt qu'il pourrait remonter à cheval, il se remettrait en route. 

Pendant ce temps, Monsieur, qui savait que, d'un jour à l'autre, il allait être remplacé dans son commandement, soit par le duc d'Angoulême, soit par Bassompierre ou par Schomberg, qui se disputaient le commandement, faisait peu de choses, perdait ses journées en tâtonnements, et n'osait risquer quelque grande entreprise pour chasser les Anglais de l'île de Ré, où ils assiégeaient toujours la citadelle Saint-Martin et le fort de La Prée, tandis que, de leur côté, les Français assiégeaient La Rochelle. 

D'Artagnan, comme nous l'avons dit, était redevenu plus tranquille, comme il arrive toujours après un danger passé, et quand le danger semble évanoui; il ne lui restait qu'une inquiétude, c'était de n'apprendre aucune nouvelle de ses amis. 

Mais, un matin du commencement du mois de novembre, tout lui fut expliqué par cette lettre, datée de Villeroi: 
\begin{mail}{}{Monsieur d'Artagnan,} 
	MM. Athos, Porthos et Aramis, après avoir fait une bonne partie chez moi, et s'être égayés beaucoup, ont mené si grand bruit, que le prévôt du château, homme très rigide, les a consignés pour quelques jours; mais j'accomplis les ordres qu'ils m'ont donnés, de vous envoyer douze bouteilles de mon vin d'Anjou, dont ils ont fait grand cas: ils veulent que vous buviez à leur santé avec leur vin favori.
	
	Je l'ai fait, et suis, monsieur, avec un grand respect,
	
	\closeletter[Votre serviteur très humble et très obéissant,]{Godeau,\\ \textit{Hôtelier de messieurs les mousquetaires.}}
	\end{mail}

«À la bonne heure! s'écria d'Artagnan, ils pensent à moi dans leurs plaisirs comme je pensais à eux dans mon ennui; bien certainement que je boirai à leur santé et de grand cœur; mais je n'y boirai pas seul.» 

Et d'Artagnan courut chez deux gardes, avec lesquels il avait fait plus amitié qu'avec les autres, afin de les inviter à boire avec lui le délicieux petit vin d'Anjou qui venait d'arriver de Villeroi. L'un des deux gardes était invité pour le soir même, et l'autre invité pour le lendemain; la réunion fut donc fixée au surlendemain. 

D'Artagnan, en rentrant, envoya les douze bouteilles de vin à la buvette des gardes, en recommandant qu'on les lui gardât avec soin; puis, le jour de la solennité, comme le dîner était fixé pour l'heure de midi, d'Artagnan envoya, dès neuf heures, Planchet pour tout préparer. 

Planchet, tout fier d'être élevé à la dignité de maître d'hôtel, songea à tout apprêter en homme intelligent; à cet effet il s'adjoignit le valet d'un des convives de son maître, nommé Fourreau, et ce faux soldat qui avait voulu tuer d'Artagnan, et qui, n'appartenant à aucun corps, était entré à son service ou plutôt à celui de Planchet, depuis que d'Artagnan lui avait sauvé la vie. 

L'heure du festin venue, les deux convives arrivèrent, prirent place et les mets s'alignèrent sur la table. Planchet servait la serviette au bras, Fourreau débouchait les bouteilles, et Brisemont, c'était le nom du convalescent, transvasait dans des carafons de verre le vin qui paraissait avoir déposé par effet des secousses de la route. De ce vin, la première bouteille était un peu trouble vers la fin, Brisemont versa cette lie dans un verre, et d'Artagnan lui permit de la boire; car le pauvre diable n'avait pas encore beaucoup de forces. 

Les convives, après avoir mangé le potage, allaient porter le premier verre à leurs lèvres, lorsque tout à coup le canon retentit au fort Louis et au fort Neuf; aussitôt les gardes, croyant qu'il s'agissait de quelque attaque imprévue, soit des assiégés, soit des Anglais, sautèrent sur leurs épées; d'Artagnan, non moins leste, fit comme eux, et tous trois sortirent en courant, afin de se rendre à leurs postes. 

Mais à peine furent-ils hors de la buvette, qu'ils se trouvèrent fixés sur la cause de ce grand bruit; les cris de Vive le roi! Vive M. le cardinal! retentissaient de tous côtés, et les tambours battaient dans toutes les directions. 

En effet, le roi, impatient comme on l'avait dit, venait de doubler deux étapes, et arrivait à l'instant même avec toute sa maison et un renfort de dix mille hommes de troupe; ses mousquetaires le précédaient et le suivaient. D'Artagnan, placé en haie avec sa compagnie, salua d'un geste expressif ses amis, qui lui répondirent des yeux, et M. de Tréville, qui le reconnut tout d'abord. 

La cérémonie de réception achevée, les quatre amis furent bientôt dans les bras l'un de l'autre. 

«Pardieu! s'écria d'Artagnan, il n'est pas possible de mieux arriver, et les viandes n'auront pas encore eu le temps de refroidir! n'est-ce pas, messieurs? ajouta le jeune homme en se tournant vers les deux gardes, qu'il présenta à ses amis. 

\speak  Ah! ah! il paraît que nous banquetions, dit Porthos. 

\speak  J'espère, dit Aramis, qu'il n'y a pas de femmes à votre dîner! 

\speak  Est-ce qu'il y a du vin potable dans votre bicoque? demanda Athos. 

\speak  Mais, pardieu! il y a le vôtre, cher ami, répondit d'Artagnan. 

\speak  Notre vin? fit Athos étonné. 

\speak  Oui, celui que vous m'avez envoyé. 

\speak  Nous vous avons envoyé du vin? 

\speak  Mais vous savez bien, de ce petit vin des coteaux d'Anjou? 

\speak  Oui, je sais bien de quel vin vous voulez parler. 

\speak  Le vin que vous préférez. 

\speak  Sans doute, quand je n'ai ni champagne ni chambertin. 

\speak  Eh bien, à défaut de champagne et de chambertin, vous vous contenterez de celui-là. 

\speak  Nous avons donc fait venir du vin d'Anjou, gourmet que nous sommes? dit Porthos. 

\speak  Mais non, c'est le vin qu'on m'a envoyé de votre part. 

\speak  De notre part? firent les trois mousquetaires. 

\speak  Est-ce vous, Aramis, dit Athos, qui avez envoyé du vin? 

\speak  Non, et vous, Porthos? 

\speak  Non, et vous, Athos? 

\speak  Non. 

\speak  Si ce n'est pas vous, dit d'Artagnan, c'est votre hôtelier. 

\speak  Notre hôtelier? 

\speak  Eh oui! votre hôtelier, Godeau, hôtelier des mousquetaires. 

\speak  Ma foi, qu'il vienne d'où il voudra, n'importe, dit Porthos, goûtons-le, et, s'il est bon, buvons-le. 

\speak  Non pas, dit Athos, ne buvons pas le vin qui a une source inconnue. 

\speak  Vous avez raison, Athos, dit d'Artagnan. Personne de vous n'a chargé l'hôtelier Godeau de m'envoyer du vin? 

\speak  Non! et cependant il vous en a envoyé de notre part? 

\speak  Voici la lettre!» dit d'Artagnan. 

Et il présenta le billet à ses camarades. 

«Ce n'est pas son écriture! s'écria Athos, je la connais, c'est moi qui, avant de partir, ai réglé les comptes de la communauté. 

\speak  Fausse lettre, dit Porthos; nous n'avons pas été consignés. 

\speak  D'Artagnan, demanda Aramis d'un ton de reproche, comment avez-vous pu croire que nous avions fait du bruit?\dots» 

D'Artagnan pâlit, et un tremblement convulsif secoua tous ses membres. 

«Tu m'effraies, dit Athos, qui ne le tutoyait que dans les grandes occasions, qu'est-il donc arrivé? 

\speak  Courons, courons, mes amis! s'écria d'Artagnan, un horrible soupçon me traverse l'esprit! serait-ce encore une vengeance de cette femme?» 

Ce fut Athos qui pâlit à son tour. 

D'Artagnan s'élança vers la buvette, les trois mousquetaires et les deux gardes l'y suivirent. 

Le premier objet qui frappa la vue de d'Artagnan en entrant dans la salle à manger, fut Brisemont étendu par terre et se roulant dans d'atroces convulsions. 

Planchet et Fourreau, pâles comme des morts, essayaient de lui porter secours; mais il était évident que tout secours était inutile: tous les traits du moribond étaient crispés par l'agonie. 

«Ah! s'écria-t-il en apercevant d'Artagnan, ah! c'est affreux, vous avez l'air de me faire grâce et vous m'empoisonnez! 

\speak  Moi! s'écria d'Artagnan, moi, malheureux! moi! que dis-tu donc là? 

\speak  Je dis que c'est vous qui m'avez donné ce vin, je dis que c'est vous qui m'avez dit de le boire, je dis que vous avez voulu vous venger de moi, je dis que c'est affreux! 

\speak  N'en croyez rien, Brisemont, dit d'Artagnan, n'en croyez rien; je vous jure, je vous proteste\dots 

\speak  Oh! mais Dieu est là! Dieu vous punira! Mon Dieu! qu'il souffre un jour ce que je souffre! 

\speak  Sur l'évangile, s'écria d'Artagnan en se précipitant vers le moribond, je vous jure que j'ignorais que ce vin fût empoisonné et que j'allais en boire comme vous. 

\speak  Je ne vous crois pas», dit le soldat. 

Et il expira dans un redoublement de tortures. 

«Affreux! affreux! murmurait Athos, tandis que Porthos brisait les bouteilles et qu'Aramis donnait des ordres un peu tardifs pour qu'on allât chercher un confesseur. 

\speak  O mes amis! dit d'Artagnan, vous venez encore une fois de me sauver la vie, non seulement à moi, mais à ces messieurs. Messieurs, continua-t-il en s'adressant aux gardes, je vous demanderai le silence sur toute cette aventure; de grands personnages pourraient avoir trempé dans ce que vous avez vu, et le mal de tout cela retomberait sur nous. 

\speak  Ah! monsieur! balbutiait Planchet plus mort que vif; ah! monsieur! que je l'ai échappé belle! 

\speak  Comment, drôle, s'écria d'Artagnan, tu allais donc boire mon vin? 

\speak  À la santé du roi, monsieur, j'allais en boire un pauvre verre, si Fourreau ne m'avait pas dit qu'on m'appelait. 

\speak  Hélas! dit Fourreau, dont les dents claquaient de terreur, je voulais l'éloigner pour boire tout seul! 

\speak  Messieurs, dit d'Artagnan en s'adressant aux gardes, vous comprenez qu'un pareil festin ne pourrait être que fort triste après ce qui vient de se passer; ainsi recevez toutes mes excuses et remettez la partie à un autre jour, je vous prie.» 

Les deux gardes acceptèrent courtoisement les excuses de d'Artagnan, et, comprenant que les quatre amis désiraient demeurer seuls, ils se retirèrent. 

Lorsque le jeune garde et les trois mousquetaires furent sans témoins, ils se regardèrent d'un air qui voulait dire que chacun comprenait la gravité de la situation. 

«D'abord, dit Athos, sortons de cette chambre; c'est une mauvaise compagnie qu'un mort, mort de mort violente. 

\speak  Planchet, dit d'Artagnan, je vous recommande le cadavre de ce pauvre diable. Qu'il soit enterré en terre sainte. Il avait commis un crime, c'est vrai, mais il s'en était repenti.» 

Et les quatre amis sortirent de la chambre, laissant à Planchet et à Fourreau le soin de rendre les honneurs mortuaires à Brisemont. 

L'hôte leur donna une autre chambre dans laquelle il leur servit des oeufs à la coque et de l'eau, qu'Athos alla puiser lui-même à la fontaine. En quelques paroles Porthos et Aramis furent mis au courant de la situation. 

«Eh bien, dit d'Artagnan à Athos, vous le voyez, cher ami, c'est une guerre à mort.» 

Athos secoua la tête. 

«Oui, oui, dit-il, je le vois bien; mais croyez-vous que ce soit elle? 

\speak  J'en suis sûr. 

\speak  Cependant je vous avoue que je doute encore. 

\speak  Mais cette fleur de lis sur l'épaule? 

\speak  C'est une Anglaise qui aura commis quelque méfait en France, et qu'on aura flétrie à la suite de son crime. 

\speak  Athos, c'est votre femme, vous dis-je, répétait d'Artagnan, ne vous rappelez-vous donc pas comme les deux signalements se ressemblent? 

\speak  J'aurais cependant cru que l'autre était morte, je l'avais si bien pendue.» 

Ce fut d'Artagnan qui secoua la tête à son tour. 

«Mais enfin, que faire? dit le jeune homme. 

\speak  Le fait est qu'on ne peut rester ainsi avec une épée éternellement suspendue au-dessus de sa tête, dit Athos, et qu'il faut sortir de cette situation. 

\speak  Mais comment? 

\speak  Écoutez, tâchez de la rejoindre et d'avoir une explication avec elle; dites-lui: La paix ou la guerre! ma parole de gentilhomme de ne jamais rien dire de vous, de ne jamais rien faire contre vous; de votre côté serment solennel de rester neutre à mon égard: sinon, je vais trouver le chancelier, je vais trouver le roi, je vais trouver le bourreau, j'ameute la cour contre vous, je vous dénonce comme flétrie, je vous fais mettre en jugement, et si l'on vous absout, eh bien, je vous tue, foi de gentilhomme! au coin de quelque borne, comme je tuerais un chien enragé. 

\speak  J'aime assez ce moyen, dit d'Artagnan, mais comment la joindre? 

\speak  Le temps, cher ami, le temps amène l'occasion, l'occasion c'est la martingale de l'homme: plus on a engagé, plus l'on gagne quand on sait attendre. 

\speak  Oui, mais attendre entouré d'assassins et d'empoisonneurs\dots 

\speak  Bah! dit Athos, Dieu nous a gardés jusqu'à présent, Dieu nous gardera encore. 

\speak  Oui, nous; nous d'ailleurs, nous sommes des hommes, et, à tout prendre, c'est notre état de risquer notre vie: mais elle! ajouta-t-il à demi-voix. 

\speak  Qui elle? demanda Athos. 

\speak  Constance. 

\speak  Mme Bonacieux! ah! c'est juste, fit Athos; pauvre ami! j'oubliais que vous étiez amoureux. 

\speak  Eh bien, mais, dit Aramis, n'avez-vous pas vu par la lettre même que vous avez trouvée sur le misérable mort qu'elle était dans un couvent? On est très bien dans un couvent, et aussitôt le siège de La Rochelle terminé, je vous promets que pour mon compte\dots 

\speak  Bon! dit Athos, bon! oui, mon cher Aramis! nous savons que vos voeux tendent à la religion. 

\speak  Je ne suis mousquetaire que par intérim, dit humblement Aramis. 

\speak  Il paraît qu'il y a longtemps qu'il n'a reçu des nouvelles de sa maîtresse, dit tout bas Athos; mais ne faites pas attention, nous connaissons cela. 

\speak  Eh bien, dit Porthos, il me semble qu'il y aurait un moyen bien simple. 

\speak  Lequel? demanda d'Artagnan. 

\speak  Elle est dans un couvent, dites-vous? reprit Porthos. 

\speak  Oui. 

\speak  Eh bien, aussitôt le siège fini, nous l'enlevons de ce couvent. 

\speak  Mais encore faut-il savoir dans quel couvent elle est. 

\speak  C'est juste, dit Porthos. 

\speak  Mais, j'y pense, dit Athos, ne prétendez-vous pas, cher d'Artagnan, que c'est la reine qui a fait choix de ce couvent pour elle? 

\speak  Oui, je le crois du moins. 

\speak  Eh bien, mais Porthos nous aidera là-dedans. 

\speak  Et comment cela, s'il vous plaît? 

\speak  Mais par votre marquise, votre duchesse, votre princesse; elle doit avoir le bras long. 

\speak  Chut! dit Porthos en mettant un doigt sur ses lèvres, je la crois cardinaliste et elle ne doit rien savoir. 

\speak  Alors, dit Aramis, je me charge, moi, d'en avoir des nouvelles. 

\speak  Vous, Aramis, s'écrièrent les trois amis, vous, et comment cela? 

\speak  Par l'aumônier de la reine, avec lequel je suis fort lié\dots», dit Aramis en rougissant. 

Et sur cette assurance, les quatre amis, qui avaient achevé leur modeste repas, se séparèrent avec promesse de se revoir le soir même: d'Artagnan retourna aux Minimes, et les trois mousquetaires rejoignirent le quartier du roi, où ils avaient à faire préparer leur logis. 
%!TeX root=../musketeersfr.tex 

\chapter{L'Auberge Du Colombier-Rouge} 
	
\lettrine{\accentletter[\gravebox]{A}}{} peine arrivé au camp, le roi, qui avait si grande hâte de se trouver en face de l'ennemi, et qui, à meilleur droit que le cardinal, partageait sa haine contre Buckingham, voulut faire toutes les dispositions, d'abord pour chasser les Anglais de l'île de Ré, ensuite pour presser le siège de La Rochelle; mais, malgré lui, il fut retardé par les dissensions qui éclatèrent entre MM. de Bassompierre et Schomberg, contre le duc d'Angoulême. 

MM. de Bassompierre et Schomberg étaient maréchaux de France, et réclamaient leur droit de commander l'armée sous les ordres du roi; mais le cardinal, qui craignait que Bassompierre, huguenot au fond du cœur, ne pressât faiblement les Anglais et les Rochelois, ses frères en religion, poussait au contraire le duc d'Angoulême, que le roi, à son instigation, avait nommé lieutenant général. Il en résulta que, sous peine de voir MM. de Bassompierre et Schomberg déserter l'armée, on fut obligé de faire à chacun un commandement particulier: Bassompierre prit ses quartiers au nord de la ville, depuis La Leu jusqu'à Dompierre; le duc d'Angoulême à l'est, depuis Dompierre jusqu'à Périgny; et M. de Schomberg au midi, depuis Périgny jusqu'à Angoutin. 

Le logis de Monsieur était à Dompierre. 

Le logis du roi était tantôt à Étré, tantôt à La Jarrie. 

Enfin le logis du cardinal était sur les dunes, au pont de La Pierre, dans une simple maison sans aucun retranchement. 

De cette façon, Monsieur surveillait Bassompierre; le roi, le duc d'Angoulême, et le cardinal, M. de Schomberg. 

Aussitôt cette organisation établie, on s'était occupé de chasser les Anglais de l'île. 

La conjoncture était favorable: les Anglais, qui ont, avant toute chose, besoin de bons vivres pour être de bons soldats, ne mangeant que des viandes salées et de mauvais biscuits, avaient force malades dans leur camp; de plus, la mer, fort mauvaise à cette époque de l'année sur toutes les côtes de l'océan, mettait tous les jours quelque petit bâtiment à mal; et la plage, depuis la pointe de l'Aiguillon jusqu'à la tranchée, était littéralement, à chaque marée, couverte des débris de pinasses, de roberges et de felouques; il en résultait que, même les gens du roi se tinssent-ils dans leur camp, il était évident qu'un jour ou l'autre Buckingham, qui ne demeurait dans l'île de Ré que par entêtement, serait obligé de lever le siège. 

Mais, comme M. de Toiras fit dire que tout se préparait dans le camp ennemi pour un nouvel assaut, le roi jugea qu'il fallait en finir et donna les ordres nécessaires pour une affaire décisive. 

Notre intention n'étant pas de faire un journal de siège, mais au contraire de n'en rapporter que les événements qui ont trait à l'histoire que nous racontons, nous nous contenterons de dire en deux mots que l'entreprise réussit au grand étonnement du roi et à la grande gloire de M. le cardinal. Les Anglais, repoussés pied à pied, battus dans toutes les rencontres, écrasés au passage de l'île de Loix, furent obligés de se rembarquer, laissant sur le champ de bataille deux mille hommes parmi lesquels cinq colonels, trois lieutenant-colonels, deux cent cinquante capitaines et vingt gentilshommes de qualité, quatre pièces de canon et soixante drapeaux qui furent apportés à Paris par Claude de Saint-Simon, et suspendus en grande pompe aux voûtes de Notre-Dame. 

Des Te Deum furent chantés au camp, et de là se répandirent par toute la France. 

Le cardinal resta donc maître de poursuivre le siège sans avoir, du moins momentanément, rien à craindre de la part des Anglais. 

Mais, comme nous venons de le dire, le repos n'était que momentané. 

Un envoyé du duc de Buckingham, nommé Montaigu, avait été pris, et l'on avait acquis la preuve d'une ligue entre l'Empire, l'Espagne, l'Angleterre et la Lorraine. 

Cette ligue était dirigée contre la France. 

De plus, dans le logis de Buckingham, qu'il avait été forcé d'abandonner plus précipitamment qu'il ne l'avait cru, on avait trouvé des papiers qui confirmaient cette ligue, et qui, à ce qu'assure M. le cardinal dans ses mémoires, compromettaient fort Mme de Chevreuse, et par conséquent la reine. 

C'était sur le cardinal que pesait toute la responsabilité, car on n'est pas ministre absolu sans être responsable; aussi toutes les ressources de son vaste génie étaient-elles tendues nuit et jour, et occupées à écouter le moindre bruit qui s'élevait dans un des grands royaumes de l'Europe. 

Le cardinal connaissait l'activité et surtout la haine de Buckingham; si la ligue qui menaçait la France triomphait, toute son influence était perdue: la politique espagnole et la politique autrichienne avaient leurs représentants dans le cabinet du Louvre, où elles n'avaient encore que des partisans; lui Richelieu, le ministre français, le ministre national par excellence, était perdu. Le roi, qui, tout en lui obéissant comme un enfant, le haïssait comme un enfant hait son maître, l'abandonnait aux vengeances réunies de Monsieur et de la reine; il était donc perdu, et peut-être la France avec lui. Il fallait parer à tout cela. 

Aussi vit-on les courriers, devenus à chaque instant plus nombreux, se succéder nuit et jour dans cette petite maison du pont de La Pierre, où le cardinal avait établi sa résidence. 

C'étaient des moines qui portaient si mal le froc, qu'il était facile de reconnaître qu'ils appartenaient surtout à l'église militante; des femmes un peu gênées dans leurs costumes de pages, et dont les larges trousses ne pouvaient entièrement dissimuler les formes arrondies; enfin des paysans aux mains noircies, mais à la jambe fine, et qui sentaient l'homme de qualité à une lieue à la ronde. 

Puis encore d'autres visites moins agréables, car deux ou trois fois le bruit se répandit que le cardinal avait failli être assassiné. 

Il est vrai que les ennemis de Son Éminence disaient que c'était elle-même qui mettait en campagne les assassins maladroits, afin d'avoir le cas échéant le droit d'user de représailles; mais il ne faut croire ni à ce que disent les ministres, ni à ce que disent leurs ennemis. 

Ce qui n'empêchait pas, au reste, le cardinal, à qui ses plus acharnés détracteurs n'ont jamais contesté la bravoure personnelle, de faire force courses nocturnes tantôt pour communiquer au duc d'Angoulême des ordres importants, tantôt pour aller se concerter avec le roi, tantôt pour aller conférer avec quelque messager qu'il ne voulait pas qu'on laissât entrer chez lui. 

De leur côté les mousquetaires qui n'avaient pas grand-chose à faire au siège n'étaient pas tenus sévèrement et menaient joyeuse vie. Cela leur était d'autant plus facile, à nos trois compagnons surtout, qu'étant des amis de M. de Tréville, ils obtenaient facilement de lui de s'attarder et de rester après la fermeture du camp avec des permissions particulières. 

Or, un soir que d'Artagnan, qui était de tranchée, n'avait pu les accompagner, Athos, Porthos et Aramis, montés sur leurs chevaux de bataille, enveloppés de manteaux de guerre, une main sur la crosse de leurs pistolets, revenaient tous trois d'une buvette qu'Athos avait découverte deux jours auparavant sur la route de La Jarrie, et qu'on appelait le Colombier-Rouge, suivant le chemin qui conduisait au camp, tout en se tenant sur leurs gardes, comme nous l'avons dit, de peur d'embuscade, lorsqu'à un quart de lieue à peu près du village de Boisnar ils crurent entendre le pas d'une cavalcade qui venait à eux; aussitôt tous trois s'arrêtèrent, serrés l'un contre l'autre, et attendirent, tenant le milieu de la route: au bout d'un instant, et comme la lune sortait justement d'un nuage, ils virent apparaître au détour d'un chemin deux cavaliers qui, en les apercevant, s'arrêtèrent à leur tour, paraissant délibérer s'ils devaient continuer leur route ou retourner en arrière. Cette hésitation donna quelques soupçons aux trois amis, et Athos, faisant quelques pas en avant, cria de sa voix ferme: 

«Qui vive? 

\speak  Qui vive vous-même? répondit un de ces deux cavaliers. 

\speak  Ce n'est pas répondre, cela! dit Athos. Qui vive? Répondez, ou nous chargeons. 

\speak  Prenez garde à ce que vous allez faire, messieurs! dit alors une voix vibrante qui paraissait avoir l'habitude du commandement. 

\speak  C'est quelque officier supérieur qui fait sa ronde de nuit, dit Athos, que voulez-vous faire, messieurs? 

\speak  Qui êtes-vous? dit la même voix du même ton de commandement; répondez à votre tour, ou vous pourriez vous mal trouver de votre désobéissance. 

\speak  Mousquetaires du roi, dit Athos, de plus en plus convaincu que celui qui les interrogeait en avait le droit. 

\speak  Quelle compagnie? 

\speak  Compagnie de Tréville. 

\speak  Avancez à l'ordre, et venez me rendre compte de ce que vous faites ici, à cette heure.» 

Les trois compagnons s'avancèrent, l'oreille un peu basse, car tous trois maintenant étaient convaincus qu'ils avaient affaire à plus fort qu'eux; on laissa, au reste, à Athos le soin de porter la parole. 

Un des deux cavaliers, celui qui avait pris la parole en second lieu, était à dix pas en avant de son compagnon; Athos fit signe à Porthos et à Aramis de rester de leur côté en arrière, et s'avança seul. 

«Pardon, mon officier! dit Athos; mais nous ignorions à qui nous avions affaire, et vous pouvez voir que nous faisions bonne garde. 

\speak  Votre nom? dit l'officier, qui se couvrait une partie du visage avec son manteau. 

\speak  Mais vous-même, monsieur, dit Athos qui commençait à se révolter contre cette inquisition; donnez-moi, je vous prie, la preuve que vous avez le droit de m'interroger. 

\speak  Votre nom? reprit une seconde fois le cavalier en laissant tomber son manteau de manière à avoir le visage découvert. 

\speak  Monsieur le cardinal! s'écria le mousquetaire stupéfait. 

\speak  Votre nom? reprit pour la troisième fois Son Éminence. 

\speak  Athos», dit le mousquetaire. 

Le cardinal fit un signe à l'écuyer, qui se rapprocha. 

«Ces trois mousquetaires nous suivront, dit-il à voix basse, je ne veux pas qu'on sache que je suis sorti du camp, et, en nous suivant, nous serons sûrs qu'ils ne le diront à personne. 

\speak  Nous sommes gentilshommes, Monseigneur, dit Athos; demandez-nous donc notre parole et ne vous inquiétez de rien. Dieu merci, nous savons garder un secret.» 

Le cardinal fixa ses yeux perçants sur ce hardi interlocuteur. 

«Vous avez l'oreille fine, monsieur Athos, dit le cardinal; mais maintenant, écoutez ceci: ce n'est point par défiance que je vous prie de me suivre, c'est pour ma sûreté: sans doute vos deux compagnons sont MM. Porthos et Aramis? 

\speak  Oui, Votre Éminence, dit Athos, tandis que les deux mousquetaires restés en arrière s'approchaient, le chapeau à la main. 

\speak  Je vous connais, messieurs, dit le cardinal, je vous connais: je sais que vous n'êtes pas tout à fait de mes amis, et j'en suis fâché, mais je sais que vous êtes de braves et loyaux gentilshommes, et qu'on peut se fier à vous. Monsieur Athos, faites-moi donc l'honneur de m'accompagner, vous et vos deux amis, et alors j'aurai une escorte à faire envie à Sa Majesté, si nous la rencontrons.» 

Les trois mousquetaires s'inclinèrent jusque sur le cou de leurs chevaux. 

«Eh bien, sur mon honneur, dit Athos, Votre Éminence a raison de nous emmener avec elle: nous avons rencontré sur la route des visages affreux, et nous avons même eu avec quatre de ces visages une querelle au Colombier-Rouge. 

\speak  Une querelle, et pourquoi, messieurs? dit le cardinal, je n'aime pas les querelleurs, vous le savez! 

\speak  C'est justement pour cela que j'ai l'honneur de prévenir Votre Éminence de ce qui vient d'arriver; car elle pourrait l'apprendre par d'autres que par nous, et, sur un faux rapport, croire que nous sommes en faute. 

\speak  Et quels ont été les résultats de cette querelle? demanda le cardinal en fronçant le sourcil. 

\speak  Mais mon ami Aramis, que voici, a reçu un petit coup d'épée dans le bras, ce qui ne l'empêchera pas, comme Votre Éminence peut le voir, de monter à l'assaut demain, si Votre Éminence ordonne l'escalade. 

\speak  Mais vous n'êtes pas hommes à vous laisser donner des coups d'épée ainsi, dit le cardinal: voyons, soyez francs, messieurs, vous en avez bien rendu quelques-uns; confessez-vous, vous savez que j'ai le droit de donner l'absolution. 

\speak  Moi, Monseigneur, dit Athos, je n'ai pas même mis l'épée à la main, mais j'ai pris celui à qui j'avais affaire à bras-le-corps et je l'ai jeté par la fenêtre; il paraît qu'en tombant, continua Athos avec quelque hésitation, il s'est cassé la cuisse. 

\speak  Ah! ah! fit le cardinal; et vous, monsieur Porthos? 

\speak  Moi, Monseigneur, sachant que le duel est défendu, j'ai saisi un banc, et j'en ai donné à l'un de ces brigands un coup qui, je crois, lui a brisé l'épaule. 

\speak  Bien, dit le cardinal; et vous, monsieur Aramis? 

\speak  Moi, Monseigneur, comme je suis d'un naturel très doux et que, d'ailleurs, ce que Monseigneur ne sait peut-être pas, je suis sur le point de rentrer dans les ordres, je voulais séparer mes camarades, quand un de ces misérables m'a donné traîtreusement un coup d'épée à travers le bras gauche: alors la patience m'a manqué, j'ai tiré mon épée à mon tour, et comme il revenait à la charge, je crois avoir senti qu'en se jetant sur moi il se l'était passée au travers du corps: je sais bien qu'il est tombé seulement, et il m'a semblé qu'on l'emportait avec ses deux compagnons. 

\speak  Diable, messieurs! dit le cardinal, trois hommes hors de combat pour une dispute de cabaret, vous n'y allez pas de main morte; et à propos de quoi était venue la querelle? 

\speak  Ces misérables étaient ivres, dit Athos, et sachant qu'il y avait une femme qui était arrivée le soir dans le cabaret, ils voulaient forcer la porte. 

\speak  Forcer la porte! dit le cardinal, et pour quoi faire? 

\speak  Pour lui faire violence sans doute, dit Athos; j'ai eu l'honneur de dire à Votre Éminence que ces misérables étaient ivres. 

\speak  Et cette femme était jeune et jolie? demanda le cardinal avec une certaine inquiétude. 

\speak  Nous ne l'avons pas vue, Monseigneur, dit Athos. 

\speak  Vous ne l'avez pas vue; ah! très bien, reprit vivement le cardinal; vous avez bien fait de défendre l'honneur d'une femme, et, comme c'est à l'auberge du Colombier-Rouge que je vais moi-même, je saurai si vous m'avez dit la vérité. 

\speak  Monseigneur, dit fièrement Athos, nous sommes gentilshommes, et pour sauver notre tête, nous ne ferions pas un mensonge. 

\speak  Aussi je ne doute pas de ce que vous me dites, monsieur Athos, je n'en doute pas un seul instant; mais, ajouta-t-il pour changer la conversation, cette dame était donc seule? 

\speak  Cette dame avait un cavalier enfermé avec elle, dit Athos; mais, comme malgré le bruit ce cavalier ne s'est pas montré, il est à présumer que c'est un lâche. 

\speak  Ne jugez pas témérairement, dit l'évangile», répliqua le cardinal. 

Athos s'inclina. 

«Et maintenant, messieurs, c'est bien, continua Son Éminence, je sais ce que je voulais savoir; suivez-moi.» 

Les trois mousquetaires passèrent derrière le cardinal, qui s'enveloppa de nouveau le visage de son manteau et remit son cheval en marche, se tenant à huit ou dix pas en avant de ses quatre compagnons. 

On arriva bientôt à l'auberge silencieuse et solitaire; sans doute l'hôte savait quel illustre visiteur il attendait, et en conséquence il avait renvoyé les importuns. 

Dix pas avant d'arriver à la porte, le cardinal fit signe à son écuyer et aux trois mousquetaires de faire halte, un cheval tout sellé était attaché au contrevent, le cardinal frappa trois coups et de certaine façon. 

Un homme enveloppé d'un manteau sortit aussitôt et échangea quelques rapides paroles avec le cardinal; après quoi il remonta à cheval et repartit dans la direction de Surgères, qui était aussi celle de Paris. 

«Avancez, messieurs, dit le cardinal. 

\speak  Vous m'avez dit la vérité, mes gentilshommes, dit-il en s'adressant aux trois mousquetaires, il ne tiendra pas à moi que notre rencontre de ce soir ne vous soit avantageuse; en attendant, suivez-moi.» 

Le cardinal mit pied à terre, les trois mousquetaires en firent autant; le cardinal jeta la bride de son cheval aux mains de son écuyer, les trois mousquetaires attachèrent les brides des leurs aux contrevents. 

L'hôte se tenait sur le seuil de la porte; pour lui, le cardinal n'était qu'un officier venant visiter une dame. 

«Avez-vous quelque chambre au rez-de-chaussée où ces messieurs puissent m'attendre près d'un bon feu?» dit le cardinal. 

L'hôte ouvrit la porte d'une grande salle, dans laquelle justement on venait de remplacer un mauvais poêle par une grande et excellente cheminée. 

«J'ai celle-ci, répondit-il. 

\speak  C'est bien, dit le cardinal; entrez là, messieurs, et veuillez m'attendre; je ne serai pas plus d'une demi-heure.» 

Et tandis que les trois mousquetaires entraient dans la chambre du rez-de-chaussée, le cardinal, sans demander plus amples renseignements, monta l'escalier en homme qui n'a pas besoin qu'on lui indique son chemin.
%!TeX root=../musketeersfr.tex 

\chapter{De L'Utilité Des Tuyaux De Poêle}

\lettrine{I}{l} était évident que, sans s'en douter, et mus seulement par leur caractère chevaleresque et aventureux, nos trois amis venaient de rendre service à quelqu'un que le cardinal honorait de sa protection particulière. 

Maintenant quel était ce quelqu'un? C'est la question que se firent d'abord les trois mousquetaires; puis, voyant qu'aucune des réponses que pouvait leur faire leur intelligence n'était satisfaisante, Porthos appela l'hôte et demanda des dés. 

Porthos et Aramis se placèrent à une table et se mirent à jouer. Athos se promena en réfléchissant. 

En réfléchissant et en se promenant, Athos passait et repassait devant le tuyau du poêle rompu par la moitié et dont l'autre extrémité donnait dans la chambre supérieure, et à chaque fois qu'il passait et repassait, il entendait un murmure de paroles qui finit par fixer son attention. Athos s'approcha, et il distingua quelques mots qui lui parurent sans doute mériter un si grand intérêt qu'il fit signe à ses compagnons de se taire, restant lui-même courbé l'oreille tendue à la hauteur de l'orifice inférieur. 

«Écoutez, Milady, disait le cardinal, l'affaire est importante: asseyez-vous là et causons. 

\speak  Milady! murmura Athos. 

\speak  J'écoute Votre Éminence avec la plus grande attention, répondit une voix de femme qui fit tressaillir le mousquetaire. 

\speak  Un petit bâtiment avec équipage anglais, dont le capitaine est à moi, vous attend à l'embouchure de la Charente, au fort de La Pointe; il mettra à la voile demain matin. 

\speak  Il faut alors que je m'y rende cette nuit? 

\speak  À l'instant même, c'est-à-dire lorsque vous aurez reçu mes instructions. Deux hommes que vous trouverez à la porte en sortant vous serviront d'escorte; vous me laisserez sortir le premier, puis une demi-heure après moi, vous sortirez à votre tour. 

\speak  Oui, Monseigneur. Maintenant revenons à la mission dont vous voulez bien me charger; et comme je tiens à continuer de mériter la confiance de Votre Éminence, daignez me l'exposer en termes clairs et précis, afin que je ne commette aucune erreur.» 

Il y eut un instant de profond silence entre les deux interlocuteurs; il était évident que le cardinal mesurait d'avance les termes dans lesquels il allait parler, et que Milady recueillait toutes ses facultés intellectuelles pour comprendre les choses qu'il allait dire et les graver dans sa mémoire quand elles seraient dites. 

Athos profita de ce moment pour dire à ses deux compagnons de fermer la porte en dedans et pour leur faire signe de venir écouter avec lui. 

Les deux mousquetaires, qui aimaient leurs aises, apportèrent une chaise pour chacun d'eux, et une chaise pour Athos. Tous trois s'assirent alors, leurs têtes rapprochées et l'oreille au guet. 

«Vous allez partir pour Londres, continua le cardinal. Arrivée à Londres, vous irez trouver Buckingham. 

\speak  Je ferai observer à Son Éminence, dit Milady, que depuis l'affaire des ferrets de diamants, pour laquelle le duc m'a toujours soupçonnée, Sa Grâce se défie de moi. 

\speak  Aussi cette fois-ci, dit le cardinal, ne s'agit-il plus de capter sa confiance, mais de se présenter franchement et loyalement à lui comme négociatrice. 

\speak  Franchement et loyalement, répéta Milady avec une indicible expression de duplicité. 

\speak  Oui, franchement et loyalement, reprit le cardinal du même ton; toute cette négociation doit être faite à découvert. 

\speak  Je suivrai à la lettre les instructions de Son Éminence, et j'attends qu'elle me les donne. 

\speak  Vous irez trouver Buckingham de ma part, et vous lui direz que je sais tous les préparatifs qu'il fait mais que je ne m'en inquiète guère, attendu qu'au premier mouvement qu'il risquera, je perds la reine. 

\speak  Croira-t-il que Votre Éminence est en mesure d'accomplir la menace qu'elle lui fait? 

\speak  Oui, car j'ai des preuves. 

\speak  Il faut que je puisse présenter ces preuves à son appréciation. 

\speak  Sans doute, et vous lui direz que je publie le rapport de Bois-Robert et du marquis de Beautru sur l'entrevue que le duc a eu chez Mme la connétable avec la reine, le soir que Mme la connétable a donné une fête masquée; vous lui direz, afin qu'il ne doute de rien, qu'il y est venu sous le costume du grand mogol que devait porter le chevalier de Guise, et qu'il a acheté à ce dernier moyennant la somme de trois mille pistoles. 

\speak  Bien, Monseigneur. 

\speak  Tous les détails de son entrée au Louvre et de sa sortie pendant la nuit où il s'est introduit au palais sous le costume d'un diseur de bonne aventure italien me sont connus; vous lui direz, pour qu'il ne doute pas encore de l'authenticité de mes renseignements, qu'il avait sous son manteau une grande robe blanche semée de larmes noires, de têtes de mort et d'os en sautoir: car, en cas de surprise, il devait se faire passer pour le fantôme de la Dame blanche qui, comme chacun le sait, revient au Louvre chaque fois que quelque grand événement va s'accomplir. 

\speak  Est-ce tout, Monseigneur? 

\speak  Dites-lui que je sais encore tous les détails de l'aventure d'Amiens, que j'en ferai faire un petit roman, spirituellement tourné, avec un plan du jardin et les portraits des principaux acteurs de cette scène nocturne. 

\speak  Je lui dirai cela. 

\speak  Dites-lui encore que je tiens Montaigu, que Montaigu est à la Bastille, qu'on n'a surpris aucune lettre sur lui, c'est vrai, mais que la torture peut lui faire dire ce qu'il sait, et même\dots ce qu'il ne sait pas. 

\speak  À merveille. 

\speak  Enfin ajoutez que Sa Grâce, dans la précipitation qu'elle a mise à quitter l'île de Ré, oublia dans son logis certaine lettre de Mme de Chevreuse qui compromet singulièrement la reine, en ce qu'elle prouve non seulement que Sa Majesté peut aimer les ennemis du roi, mais encore qu'elle conspire avec ceux de la France. Vous avez bien retenu tout ce que je vous ai dit, n'est-ce pas? 

\speak  Votre Éminence va en juger: le bal de Mme la connétable; la nuit du Louvre; la soirée d'Amiens; l'arrestation de Montaigu; la lettre de Mme de Chevreuse. 

\speak  C'est cela, dit le cardinal, c'est cela: vous avez une bien heureuse mémoire, Milady. 

\speak  Mais, reprit celle à qui le cardinal venait d'adresser ce compliment flatteur, si malgré toutes ces raisons le duc ne se rend pas et continue de menacer la France? 

\speak  Le duc est amoureux comme un fou, ou plutôt comme un niais, reprit Richelieu avec une profonde amertume; comme les anciens paladins, il n'a entrepris cette guerre que pour obtenir un regard de sa belle. S'il sait que cette guerre peut coûter l'honneur et peut-être la liberté à la dame de ses pensées, comme il dit, je vous réponds qu'il y regardera à deux fois. 

\speak  Et cependant, dit Milady avec une persistance qui prouvait qu'elle voulait voir clair jusqu'au bout, dans la mission dont elle allait être chargée, cependant s'il persiste? 

\speak  S'il persiste, dit le cardinal\dots, ce n'est pas probable. 

\speak  C'est possible, dit Milady. 

\speak  S'il persiste\dots» 

Son Éminence fit une pause et reprit\dots 

«S'il persiste, eh bien, j'espérerai dans un de ces événements qui changent la face des États. 

\speak  Si Son Éminence voulait me citer dans l'histoire quelques-uns de ces événements, dit Milady, peut-être partagerais-je sa confiance dans l'avenir. 

\speak  Eh bien, tenez! par exemple, dit Richelieu, lorsqu'en 1610, pour une cause à peu près pareille à celle qui fait mouvoir le duc, le roi Henri IV, de glorieuse mémoire, allait à la fois envahir les Flandres et l'Italie pour frapper à la fois l'Autriche des deux côtés, eh bien, n'est-il pas arrivé un événement qui a sauvé l'Autriche? Pourquoi le roi de France n'aurait-il pas la même chance que l'empereur? 

\speak  Votre Éminence veut parler du coup de couteau de la rue de la Ferronnerie? 

\speak  Justement, dit le cardinal. 

\speak  Votre Éminence ne craint-elle pas que le supplice de Ravaillac épouvante ceux qui auraient un instant l'idée de l'imiter? 

\speak  Il y aura en tout temps et dans tous les pays, surtout si ces pays sont divisés de religion, des fanatiques qui ne demanderont pas mieux que de se faire martyrs. Et tenez, justement il me revient à cette heure que les puritains sont furieux contre le duc de Buckingham et que leurs prédicateurs le désignent comme l'Antéchrist. 

\speak  Eh bien? fit Milady. 

\speak  Eh bien, continua le cardinal d'un air indifférent, il ne s'agirait, pour le moment, par exemple, que de trouver une femme, belle, jeune, adroite, qui eût à se venger elle-même du duc. Une pareille femme peut se rencontrer: le duc est homme à bonnes fortunes, et, s'il a semé bien des amours par ses promesses de constance éternelle, il a dû semer bien des haines aussi par ses éternelles infidélités. 

\speak  Sans doute, dit froidement Milady, une pareille femme peut se rencontrer. 

\speak  Eh bien, une pareille femme, qui mettrait le couteau de Jacques Clément ou de Ravaillac aux mains d'un fanatique, sauverait la France. 

\speak  Oui, mais elle serait complice d'un assassinat. 

\speak  A-t-on jamais connu les complices de Ravaillac ou de Jacques Clément? 

\speak  Non, car peut-être étaient-ils placés trop haut pour qu'on osât les aller chercher là où ils étaient: on ne brûlerait pas le Palais de Justice pour tout le monde, Monseigneur. 

\speak  Vous croyez donc que l'incendie du Palais de Justice a une cause autre que celle du hasard? demanda Richelieu du ton dont il eût fait une question sans aucune importance. 

\speak  Moi, Monseigneur, répondit Milady, je ne crois rien, je cite un fait, voilà tout, seulement, je dis que si je m'appelais Mlle de Monpensier ou la reine Marie de Médicis, je prendrais moins de précautions que j'en prends, m'appelant tout simplement Lady Clarick. 

\speak  C'est juste, dit Richelieu, et que voudriez-vous donc? 

\speak  Je voudrais un ordre qui ratifiât d'avance tout ce que je croirai devoir faire pour le plus grand bien de la France. 

\speak  Mais il faudrait d'abord trouver la femme que j'ai dit, et qui aurait à se venger du duc. 

\speak  Elle est trouvée, dit Milady. 

\speak  Puis il faudrait trouver ce misérable fanatique qui servira d'instrument à la justice de Dieu. 

\speak  On le trouvera. 

\speak  Eh bien, dit le duc, alors il sera temps de réclamer l'ordre que vous demandiez tout à l'heure. 

\speak  Votre Éminence a raison, dit Milady, et c'est moi qui ai eu tort de voir dans la mission dont elle m'honore autre chose que ce qui est réellement, c'est-à-dire d'annoncer à Sa Grâce, de la part de Son Éminence, que vous connaissez les différents déguisements à l'aide desquels il est parvenu à se rapprocher de la reine pendant la fête donnée par Mme la connétable; que vous avez les preuves de l'entrevue accordée au Louvre par la reine à certain astrologue italien qui n'est autre que le duc de Buckingham; que vous avez commandé un petit roman, des plus spirituels, sur l'aventure d'Amiens, avec plan du jardin où cette aventure s'est passée et portraits des acteurs qui y ont figuré; que Montaigu est à la Bastille, et que la torture peut lui faire dire des choses dont il se souvient et même des choses qu'il aurait oubliées; enfin, que vous possédez certaine lettre de Mme de Chevreuse, trouvée dans le logis de Sa Grâce, qui compromet singulièrement, non seulement celle qui l'a écrite, mais encore celle au nom de qui elle a été écrite. Puis, s'il persiste malgré tout cela, comme c'est à ce que je viens de dire que se borne ma mission, je n'aurai plus qu'à prier Dieu de faire un miracle pour sauver la France. C'est bien cela, n'est-ce pas, Monseigneur, et je n'ai pas autre chose à faire? 

\speak  C'est bien cela, reprit sèchement le cardinal. 

\speak  Et maintenant, dit Milady sans paraître remarquer le changement de ton du duc à son égard, maintenant que j'ai reçu les instructions de Votre Éminence à propos de ses ennemis, Monseigneur me permettra-t-il de lui dire deux mots des miens? 

\speak  Vous avez donc des ennemis? demanda Richelieu. 

\speak  Oui, Monseigneur; des ennemis contre lesquels vous me devez tout votre appui, car je me les suis faits en servant Votre Éminence. 

\speak  Et lesquels? répliqua le duc. 

\speak  D'abord une petite intrigante du nom de Bonacieux. 

\speak  Elle est dans la prison de Mantes. 

\speak  C'est-à-dire qu'elle y était, reprit Milady, mais la reine a surpris un ordre du roi, à l'aide duquel elle l'a fait transporter dans un couvent. 

\speak  Dans un couvent? dit le duc. 

\speak  Oui, dans un couvent. 

\speak  Et dans lequel? 

\speak  Je l'ignore, le secret a été bien gardé\dots 

\speak  Je le saurai, moi! 

\speak  Et Votre Éminence me dira dans quel couvent est cette femme? 

\speak  Je n'y vois pas d'inconvénient, dit le cardinal. 

\speak  Bien; maintenant j'ai un autre ennemi bien autrement à craindre pour moi que cette petite Mme Bonacieux. 

\speak  Et lequel? 

\speak  Son amant. 

\speak  Comment s'appelle-t-il? 

\speak  Oh! Votre Éminence le connaît bien, s'écria Milady emportée par la colère, c'est notre mauvais génie à tous deux; c'est celui qui, dans une rencontre avec les gardes de Votre Éminence, a décidé la victoire en faveur des mousquetaires du roi; c'est celui qui a donné trois coups d'épée à de Wardes, votre émissaire, et qui a fait échouer l'affaire des ferrets; c'est celui enfin qui, sachant que c'était moi qui lui avais enlevé Mme Bonacieux, a juré ma mort. 

\speak  Ah! ah! dit le cardinal, je sais de qui vous voulez parler. 

\speak  Je veux parler de ce misérable d'Artagnan. 

\speak  C'est un hardi compagnon, dit le cardinal. 

\speak  Et c'est justement parce que c'est un hardi compagnon qu'il n'en est que plus à craindre. 

\speak  Il faudrait, dit le duc, avoir une preuve de ses intelligences avec Buckingham. 

\speak  Une preuve, s'écria Milady, j'en aurai dix. 

\speak  Eh bien, alors! c'est la chose la plus simple du monde, ayez-moi cette preuve et je l'envoie à la Bastille. 

\speak  Bien, Monseigneur! mais ensuite? 

\speak  Quand on est à la Bastille, il n'y a pas d'ensuite, dit le cardinal d'une voix sourde. Ah! pardieu, continua-t-il, s'il m'était aussi facile de me débarrasser de mon ennemi qu'il m'est facile de me débarrasser des vôtres, et si c'était contre de pareilles gens que vous me demandiez l'impunité!\dots 

\speak  Monseigneur, reprit Milady, troc pour troc, existence pour existence, homme pour homme; donnez-moi celui-là, je vous donne l'autre. 

\speak  Je ne sais pas ce que vous voulez dire, reprit le cardinal, et ne veux même pas le savoir, mais j'ai le désir de vous être agréable et ne vois aucun inconvénient à vous donner ce que vous demandez à l'égard d'une si infime créature; d'autant plus, comme vous me le dites, que ce petit d'Artagnan est un libertin, un duelliste, un traître. 

\speak  Un infâme, Monseigneur, un infâme! 

\speak  Donnez-moi donc du papier, une plume et de l'encre, dit le cardinal. 

\speak  En voici, Monseigneur.» 

Il se fit un instant de silence qui prouvait que le cardinal était occupé à chercher les termes dans lesquels devait être écrit le billet, ou même à l'écrire. Athos, qui n'avait pas perdu un mot de la conversation, prit ses deux compagnons chacun par une main et les conduisit à l'autre bout de la chambre. 

«Eh bien, dit Porthos, que veux-tu, et pourquoi ne nous laisses-tu pas écouter la fin de la conversation? 

\speak  Chut! dit Athos parlant à voix basse, nous en avons entendu tout ce qu'il est nécessaire que nous entendions; d'ailleurs je ne vous empêche pas d'écouter le reste, mais il faut que je sorte. 

\speak  Il faut que tu sortes! dit Porthos; mais si le cardinal te demande, que répondrons-nous? 

\speak  Vous n'attendrez pas qu'il me demande, vous lui direz les premiers que je suis parti en éclaireur parce que certaines paroles de notre hôte m'ont donné à penser que le chemin n'était pas sûr; j'en toucherai d'abord deux mots à l'écuyer du cardinal; le reste me regarde, ne vous en inquiétez pas. 

\speak  Soyez prudent, Athos! dit Aramis. 

\speak  Soyez tranquille, répondit Athos, vous le savez, j'ai du sang-froid.» 

Porthos et Aramis allèrent reprendre leur place près du tuyau de poêle. 

Quant à Athos, il sortit sans aucun mystère, alla prendre son cheval attaché avec ceux de ses deux amis aux tourniquets des contrevents, convainquit en quatre mots l'écuyer de la nécessité d'une avant-garde pour le retour, visita avec affectation l'amorce de ses pistolets, mit l'épée aux dents et suivit, en enfant perdu, la route qui conduisait au camp.
%!TeX root=../musketeersfr.tex 

\chapter{Scène Conjugale}

\lettrine{C}{omme} l'avait prévu Athos, le cardinal ne tarda point à descendre; il ouvrit la porte de la chambre où étaient entrés les mousquetaires, et trouva Porthos faisant une partie de dés acharnée avec Aramis. D'un coup d'œil rapide, il fouilla tous les coins de la salle, et vit qu'un de ses hommes lui manquait. 

«Qu'est devenu M. Athos? demanda-t-il. 

\speak  Monseigneur, répondit Porthos, il est parti en éclaireur sur quelques propos de notre hôte, qui lui ont fait croire que la route n'était pas sûre. 

\speak  Et vous, qu'avez-vous fait, monsieur Porthos? 

\speak  J'ai gagné cinq pistoles à Aramis. 

\speak  Et maintenant, vous pouvez revenir avec moi? 

\speak  Nous sommes aux ordres de Votre Éminence. 

\speak  À cheval donc, messieurs, car il se fait tard.» 

L'écuyer était à la porte, et tenait en bride le cheval du cardinal. Un peu plus loin, un groupe de deux hommes et de trois chevaux apparaissait dans l'ombre; ces deux hommes étaient ceux qui devaient conduire Milady au fort de La Pointe, et veiller à son embarquement. 

L'écuyer confirma au cardinal ce que les deux mousquetaires lui avaient déjà dit à propos d'Athos. Le cardinal fit un geste approbateur, et reprit la route, s'entourant au retour des mêmes précautions qu'il avait prises au départ. 

Laissons-le suivre le chemin du camp, protégé par l'écuyer et les deux mousquetaires, et revenons à Athos. 

Pendant une centaine de pas, il avait marché de la même allure; mais, une fois hors de vue, il avait lancé son cheval à droite, avait fait un détour, et était revenu à une vingtaine de pas, dans le taillis, guetter le passage de la petite troupe; ayant reconnu les chapeaux bordés de ses compagnons et la frange dorée du manteau de M. le cardinal, il attendit que les cavaliers eussent tourné l'angle de la route, et, les ayant perdus de vue, il revint au galop à l'auberge, qu'on lui ouvrit sans difficulté. 

L'hôte le reconnut. 

«Mon officier, dit Athos, a oublié de faire à la dame du premier une recommandation importante, il m'envoie pour réparer son oubli. 

\speak  Montez, dit l'hôte, elle est encore dans sa chambre.» 

Athos profita de la permission, monta l'escalier de son pas le plus léger, arriva sur le carré, et, à travers la porte entrouverte, il vit Milady qui attachait son chapeau. 

Il entra dans la chambre, et referma la porte derrière lui. 

Au bruit qu'il fit en repoussant le verrou, Milady se retourna. 

Athos était debout devant la porte, enveloppé dans son manteau, son chapeau rabattu sur ses yeux. 

En voyant cette figure muette et immobile comme une statue, Milady eut peur. 

«Qui êtes-vous? et que demandez-vous?» s'écria-t-elle. «Allons, c'est bien elle!» murmura Athos. 

Et, laissant tomber son manteau, et relevant son feutre, il s'avança vers Milady. 

«Me reconnaissez-vous, madame?» dit-il. 

Milady fit un pas en avant, puis recula comme à la vue d'un serpent. 

«Allons, dit Athos, c'est bien, je vois que vous me reconnaissez. 

\speak  Le comte de La Fère! murmura Milady en pâlissant et en reculant jusqu'à ce que la muraille l'empêchât d'aller plus loin. 

\speak  Oui, Milady, répondit Athos, le comte de La Fère en personne, qui revient tout exprès de l'autre monde pour avoir le plaisir de vous voir. Asseyons-nous donc, et causons, comme dit Monseigneur le cardinal.» 

Milady, dominée par une terreur inexprimable, s'assit sans proférer une seule parole. 

«Vous êtes donc un démon envoyé sur la terre? dit Athos. Votre puissance est grande, je le sais; mais vous savez aussi qu'avec l'aide de Dieu les hommes ont souvent vaincu les démons les plus terribles. Vous vous êtes déjà trouvée sur mon chemin, je croyais vous avoir terrassée, madame; mais, ou je me trompai, ou l'enfer vous a ressuscitée.» 

Milady, à ces paroles qui lui rappelaient des souvenirs effroyables, baissa la tête avec un gémissement sourd. 

«Oui, l'enfer vous a ressuscitée, reprit Athos, l'enfer vous a faite riche, l'enfer vous a donné un autre nom, l'enfer vous a presque refait même un autre visage; mais il n'a effacé ni les souillures de votre âme, ni la flétrissure de votre corps.» 

Milady se leva comme mue par un ressort, et ses yeux lancèrent des éclairs. Athos resta assis. 

«Vous me croyiez mort, n'est-ce pas, comme je vous croyais morte? et ce nom d'Athos avait caché le comte de La Fère, comme le nom de Milady Clarick avait caché Anne de Breuil! N'était-ce pas ainsi que vous vous appeliez quand votre honoré frère nous a mariés? Notre position est vraiment étrange, poursuivit Athos en riant; nous n'avons vécu jusqu'à présent l'un et l'autre que parce que nous nous croyions morts, et qu'un souvenir gêne moins qu'une créature, quoique ce soit chose dévorante parfois qu'un souvenir! 

\speak  Mais enfin, dit Milady d'une voix sourde, qui vous ramène vers moi? et que me voulez-vous? 

\speak  Je veux vous dire que, tout en restant invisible à vos yeux, je ne vous ai pas perdue de vue, moi! 

\speak  Vous savez ce que j'ai fait? 

\speak  Je puis vous raconter jour par jour vos actions, depuis votre entrée au service du cardinal jusqu'à ce soir.» 

Un sourire d'incrédulité passa sur les lèvres pâles de Milady. 

«Écoutez: c'est vous qui avez coupé les deux ferrets de diamants sur l'épaule du duc de Buckingham; c'est vous qui avez fait enlever Mme Bonacieux; c'est vous qui, amoureuse de de Wardes, et croyant passer la nuit avec lui, avez ouvert votre porte à M. d'Artagnan; c'est vous qui, croyant que de Wardes vous avait trompée, avez voulu le faire tuer par son rival; c'est vous qui, lorsque ce rival eut découvert votre infâme secret, avez voulu le faire tuer à son tour par deux assassins que vous avez envoyés à sa poursuite; c'est vous qui, voyant que les balles avaient manqué leur coup, avez envoyé du vin empoisonné avec une fausse lettre, pour faire croire à votre victime que ce vin venait de ses amis; c'est vous, enfin, qui venez là, dans cette chambre, assise sur cette chaise où je suis, de prendre avec le cardinal de Richelieu l'engagement de faire assassiner le duc de Buckingham, en échange de la promesse qu'il vous a faite de vous laisser assassiner d'Artagnan.» 

Milady était livide. 

«Mais vous êtes donc Satan? dit-elle. 

\speak  Peut-être, dit Athos; mais, en tout cas, écoutez bien ceci: Assassinez ou faites assassiner le duc de Buckingham, peu m'importe! je ne le connais pas: d'ailleurs c'est un Anglais; mais ne touchez pas du bout du doigt à un seul cheveu de d'Artagnan, qui est un fidèle ami que j'aime et que je défends, ou, je vous le jure par la tête de mon père, le crime que vous aurez commis sera le dernier. 

\speak  M. d'Artagnan m'a cruellement offensée, dit Milady d'une voix sourde, M. d'Artagnan mourra. 

\speak  En vérité, cela est-il possible qu'on vous offense, madame? dit en riant Athos; il vous a offensée, et il mourra? 

\speak  Il mourra, reprit Milady; elle d'abord, lui ensuite.» 

Athos fut saisi comme d'un vertige: la vue de cette créature, qui n'avait rien d'une femme, lui rappelait des souvenirs terribles; il pensa qu'un jour, dans une situation moins dangereuse que celle où il se trouvait, il avait déjà voulu la sacrifier à son honneur; son désir de meurtre lui revint brûlant et l'envahit comme une fièvre ardente: il se leva à son tour, porta la main à sa ceinture, en tira un pistolet et l'arma. 

Milady, pâle comme un cadavre, voulut crier, mais sa langue glacée ne put proférer qu'un son rauque qui n'avait rien de la parole humaine et qui semblait le râle d'une bête fauve; collée contre la sombre tapisserie, elle apparaissait, les cheveux épars, comme l'image effrayante de la terreur. 

Athos leva lentement son pistolet, étendit le bras de manière que l'arme touchât presque le front de Milady puis, d'une voix d'autant plus terrible qu'elle avait le calme suprême d'une inflexible résolution: 

«Madame, dit-il, vous allez à l'instant même me remettre le papier que vous a signé le cardinal, ou, sur mon âme, je vous fais sauter la cervelle.» 

Avec un autre homme Milady aurait pu conserver quelque doute, mais elle connaissait Athos; cependant elle resta immobile. 

«Vous avez une seconde pour vous décider», dit-il. 

Milady vit à la contraction de son visage que le coup allait partir; elle porta vivement la main à sa poitrine, en tira un papier et le tendit à Athos. 

«Tenez, dit-elle, et soyez maudit!» 

Athos prit le papier, repassa le pistolet à sa ceinture, s'approcha de la lampe pour s'assurer que c'était bien celui-là, le déplia et lut: 

\begin{mail}{3 \textit{décembre} 1627.}

C'est par mon ordre et pour le bien de l'État que le porteur du présent a fait ce qu'il a fait.
\closeletter{Richelieu}
\end{mail}

«Et maintenant, dit Athos en reprenant son manteau et en replaçant son feutre sur sa tête, maintenant que je t'ai arraché les dents, vipère, mords si tu peux.» 

Et il sortit de la chambre sans même regarder en arrière. 

À la porte il trouva les deux hommes et le cheval qu'ils tenaient en main. 

«Messieurs, dit-il, l'ordre de Monseigneur, vous le savez, est de conduire cette femme, sans perdre de temps, au fort de La Pointe et de ne la quitter que lorsqu'elle sera à bord.» 

Comme ces paroles s'accordaient effectivement avec l'ordre qu'ils avaient reçu, ils inclinèrent la tête en signe d'assentiment. 

Quant à Athos, il se mit légèrement en selle et partit au galop; seulement, au lieu de suivre la route, il prit à travers champs, piquant avec vigueur son cheval et de temps en temps s'arrêtant pour écouter. 

Dans une de ces haltes, il entendit sur la route le pas de plusieurs chevaux. Il ne douta point que ce ne fût le cardinal et son escorte. Aussitôt il fit une nouvelle pointe en avant, bouchonna son cheval avec de la bruyère et des feuilles d'arbres, et vint se mettre en travers de la route à deux cents pas du camp à peu près. 

«Qui vive? cria-t-il de loin quand il aperçut les cavaliers. 

\speak  C'est notre brave mousquetaire, je crois, dit le cardinal. 

\speak  Oui, Monseigneur, répondit Athos. C'est lui-même. 

\speak  Monsieur Athos, dit Richelieu, recevez tous mes remerciements pour la bonne garde que vous nous avez faite; messieurs, nous voici arrivés: prenez la porte à gauche, le mot d'ordre est \textit{Roi} et \textit{Ré}.» 

En disant ces mots, le cardinal salua de la tête les trois amis, et prit à droite suivi de son écuyer; car, cette nuit-là, lui-même couchait au camp. 

«Eh bien! dirent ensemble Porthos et Aramis lorsque le cardinal fut hors de la portée de la voix, eh bien il a signé le papier qu'elle demandait. 

\speak  Je le sais, dit tranquillement Athos, puisque le voici.» 

Et les trois amis n'échangèrent plus une seule parole jusqu'à leur quartier, excepté pour donner le mot d'ordre aux sentinelles. 

Seulement, on envoya Mousqueton dire à Planchet que son maître était prié, en relevant de tranchée, de se rendre à l'instant même au logis des mousquetaires. 

D'un autre côté, comme l'avait prévu Athos, Milady, en retrouvant à la porte les hommes qui l'attendaient, ne fit aucune difficulté de les suivre; elle avait bien eu l'envie un instant de se faire reconduire devant le cardinal et de lui tout raconter, mais une révélation de sa part amenait une révélation de la part d'Athos: elle dirait bien qu'Athos l'avait pendue, mais Athos dirait qu'elle était marquée; elle pensa qu'il valait donc encore mieux garder le silence, partir discrètement, accomplir avec son habileté ordinaire la mission difficile dont elle s'était chargée, puis, toutes les choses accomplies à la satisfaction du cardinal, venir lui réclamer sa vengeance. 

En conséquence, après avoir voyagé toute la nuit, à sept heures du matin elle était au fort de La Pointe, à huit heures elle était embarquée, et à neuf heures le bâtiment, qui, avec des lettres de marque du cardinal, était censé être en partance pour Bayonne, levait l'ancre et faisait voile pour l'Angleterre.
\include{chapters/46.tex}
%!TeX root=../musketeersfr.tex 

\chapter{Le Conseil Des Mousquetaires}

\lettrine{C}{omme} l'avait prévu Athos, le bastion n'était occupé que par une douzaine de morts tant Français que Rochelois. 

\zz
«Messieurs, dit Athos, qui avait pris le commandement de l'expédition, tandis que Grimaud va mettre la table, commençons par recueillir les fusils et les cartouches; nous pouvons d'ailleurs causer tout en accomplissant cette besogne. Ces messieurs, ajouta-t-il en montrant les morts, ne nous écoutent pas. 

\speak  Mais nous pourrions toujours les jeter dans le fossé, dit Porthos, après toutefois nous être assurés qu'ils n'ont rien dans leurs poches. 

\speak  Oui, dit Aramis, c'est l'affaire de Grimaud. 

\speak  Ah! bien alors, dit d'Artagnan, que Grimaud les fouille et les jette par-dessus les murailles. 

\speak  Gardons-nous-en bien, dit Athos, ils peuvent nous servir. 

\speak  Ces morts peuvent nous servir? dit Porthos. Ah çà, vous devenez fou, cher ami. 

\speak  Ne jugez pas témérairement, disent l'évangile et M. le cardinal, répondit Athos; combien de fusils, messieurs? 

\speak  Douze, répondit Aramis. 

\speak  Combien de coups à tirer? 

\speak  Une centaine. 

\speak  C'est tout autant qu'il nous en faut; chargeons les armes.» 

Les quatre mousquetaires se mirent à la besogne. Comme ils achevaient de charger le dernier fusil, Grimaud fit signe que le déjeuner était servi. 

Athos répondit, toujours par geste, que c'était bien, et indiqua à Grimaud une espèce de poivrière où celui-ci comprit qu'il se devait tenir en sentinelle. Seulement, pour adoucir l'ennui de la faction, Athos lui permit d'emporter un pain, deux côtelettes et une bouteille de vin. 

«Et maintenant, à table», dit Athos. 

Les quatre amis s'assirent à terre, les jambes croisées, comme les Turcs ou comme les tailleurs. 

«Ah! maintenant, dit d'Artagnan, que tu n'as plus la crainte d'être entendu, j'espère que tu vas nous faire part de ton secret, Athos. 

\speak  J'espère que je vous procure à la fois de l'agrément et de la gloire, messieurs, dit Athos. Je vous ai fait faire une promenade charmante; voici un déjeuner des plus succulents, et cinq cents personnes là-bas, comme vous pouvez les voir à travers les meurtrières, qui nous prennent pour des fous ou pour des héros, deux classes d'imbéciles qui se ressemblent assez. 

\speak  Mais ce secret? demanda d'Artagnan. 

\speak  Le secret, dit Athos, c'est que j'ai vu Milady hier soir.» 

D'Artagnan portait son verre à ses lèvres; mais à ce nom de Milady, la main lui trembla si fort, qu'il le posa à terre pour ne pas en répandre le contenu. 

«Tu as vu ta fem\dots 

\speak  Chut donc! interrompit Athos: vous oubliez, mon cher, que ces messieurs ne sont pas initiés comme vous dans le secret de mes affaires de ménage; j'ai vu Milady. 

\speak  Et où cela? demanda d'Artagnan. 

\speak  À deux lieues d'ici à peu près, à l'auberge du Colombier-Rouge. 

\speak  En ce cas je suis perdu, dit d'Artagnan. 

\speak  Non, pas tout à fait encore, reprit Athos; car, à cette heure, elle doit avoir quitté les côtes de France.» 

D'Artagnan respira. 

«Mais au bout du compte, demanda Porthos, qu'est-ce donc que cette Milady? 

\speak  Une femme charmante, dit Athos en dégustant un verre de vin mousseux. Canaille d'hôtelier! s'écria-t-il, qui nous donne du vin d'Anjou pour du vin de Champagne, et qui croit que nous nous y laisserons prendre! Oui, continua-t-il, une femme charmante qui a eu des bontés pour notre ami d'Artagnan, qui lui a fait je ne sais quelle noirceur dont elle a essayé de se venger, il y a un mois en voulant le faire tuer à coups de mousquet, il y a huit jours en essayant de l'empoisonner, et hier en demandant sa tête au cardinal. 

\speak  Comment! en demandant ma tête au cardinal? s'écria d'Artagnan, pâle de terreur. 

\speak  Ça, dit Porthos, c'est vrai comme l'évangile; je l'ai entendu de mes deux oreilles. 

\speak  Moi aussi, dit Aramis. 

\speak  Alors, dit d'Artagnan en laissant tomber son bras avec découragement, il est inutile de lutter plus longtemps; autant que je me brûle la cervelle et que tout soit fini! 

\speak  C'est la dernière sottise qu'il faut faire, dit Athos, attendu que c'est la seule à laquelle il n'y ait pas de remède. 

\speak  Mais je n'en réchapperai jamais, dit d'Artagnan, avec des ennemis pareils. D'abord mon inconnu de Meung; ensuite de Wardes, à qui j'ai donné trois coups d'épée; puis Milady, dont j'ai surpris le secret; enfin, le cardinal, dont j'ai fait échouer la vengeance. 

\speak  Eh bien, dit Athos, tout cela ne fait que quatre, et nous sommes quatre, un contre un. Pardieu! si nous en croyons les signes que nous fait Grimaud, nous allons avoir affaire à un bien plus grand nombre de gens. Qu'y a-t-il, Grimaud? Considérant la gravité de la circonstance, je vous permets de parler, mon ami, mais soyez laconique je vous prie. Que voyez-vous? 

\speak  Une troupe. 

\speak  De combien de personnes? 

\speak  De vingt hommes. 

\speak  Quels hommes? 

\speak  Seize pionniers, quatre soldats. 

\speak  À combien de pas sont-ils? 

\speak  À cinq cents pas. 

\speak  Bon, nous avons encore le temps d'achever cette volaille et de boire un verre de vin à ta santé, d'Artagnan! 

\speak  À ta santé! répétèrent Porthos et Aramis. 

\speak  Eh bien donc, à ma santé! quoique je ne croie pas que vos souhaits me servent à grand-chose. 

\speak  Bah! dit Athos, Dieu est grand, comme disent les sectateurs de Mahomet, et l'avenir est dans ses mains.» 

Puis, avalant le contenu de son verre, qu'il posa près de lui, Athos se leva nonchalamment, prit le premier fusil venu et s'approcha d'une meurtrière. 

Porthos, Aramis et d'Artagnan en firent autant. Quant à Grimaud, il reçut l'ordre de se placer derrière les quatre amis afin de recharger les armes. 

Au bout d'un instant on vit paraître la troupe; elle suivait une espèce de boyau de tranchée qui établissait une communication entre le bastion et la ville. 

«Pardieu! dit Athos, c'est bien la peine de nous déranger pour une vingtaine de drôles armés de pioches, de hoyaux et de pelles! Grimaud n'aurait eu qu'à leur faire signe de s'en aller, et je suis convaincu qu'ils nous eussent laissés tranquilles. 

\speak  J'en doute, observa d'Artagnan, car ils avancent fort résolument de ce côté. D'ailleurs, il y a avec les travailleurs quatre soldats et un brigadier armés de mousquets. 

\speak  C'est qu'ils ne nous ont pas vus, reprit Athos. 

\speak  Ma foi! dit Aramis, j'avoue que j'ai répugnance à tirer sur ces pauvres diables de bourgeois. 

\speak  Mauvais prêtre, répondit Porthos, qui a pitié des hérétiques! 

\speak  En vérité, dit Athos, Aramis a raison, je vais les prévenir. 

\speak  Que diable faites-vous donc? s'écria d'Artagnan, vous allez vous faire fusiller, mon cher.» 

Mais Athos ne tint aucun compte de l'avis, et, montant sur la brèche, son fusil d'une main et son chapeau de l'autre: 

«Messieurs, dit-il en s'adressant aux soldats et aux travailleurs, qui, étonnés de son apparition, s'arrêtaient à cinquante pas environ du bastion, et en les saluant courtoisement, messieurs, nous sommes, quelques amis et moi, en train de déjeuner dans ce bastion. Or, vous savez que rien n'est désagréable comme d'être dérangé quand on déjeune; nous vous prions donc, si vous avez absolument affaire ici, d'attendre que nous ayons fini notre repas, ou de repasser plus tard, à moins qu'il ne vous prenne la salutaire envie de quitter le parti de la rébellion et de venir boire avec nous à la santé du roi de France. 

\speak  Prends garde, Athos! s'écria d'Artagnan; ne vois-tu pas qu'ils te mettent en joue? 

\speak  Si fait, si fait, dit Athos, mais ce sont des bourgeois qui tirent fort mal, et qui n'ont garde de me toucher.» 

En effet, au même instant quatre coups de fusil partirent, et les balles vinrent s'aplatir autour d'Athos, mais sans qu'une seule le touchât. 

Quatre coups de fusil leur répondirent presque en même temps, mais ils étaient mieux dirigés que ceux des agresseurs, trois soldats tombèrent tués raide, et un des travailleurs fut blessé. 

«Grimaud, un autre mousquet!» dit Athos toujours sur la brèche. 

Grimaud obéit aussitôt. De leur côté, les trois amis avaient chargé leurs armes; une seconde décharge suivit la première: le brigadier et deux pionniers tombèrent morts, le reste de la troupe prit la fuite. 

«Allons, messieurs, une sortie», dit Athos. 

Et les quatre amis, s'élançant hors du fort, parvinrent jusqu'au champ de bataille, ramassèrent les quatre mousquets des soldats et la demi-pique du brigadier; et, convaincus que les fuyards ne s'arrêteraient qu'à la ville, reprirent le chemin du bastion, rapportant les trophées de leur victoire. 

«Rechargez les armes, Grimaud, dit Athos, et nous, messieurs, reprenons notre déjeuner et continuons notre conversation. Où en étions-nous? 

\speak  Je me le rappelle, dit d'Artagnan; vous disiez qu'après avoir demandé ma tête au cardinal, milady avait quitté les côtes de France. 

\speak  C'est vrai. 

\speak  Et où va-t-elle? ajouta d'Artagnan, qui se préoccupait fort de l'itinéraire que devrait suivre milady. 

\speak  Elle va en Angleterre, répondit Athos. 

\speak  Et dans quel but? 

\speak  Dans le but d'assassiner ou de faire assassiner Buckingham.» 

D'Artagnan poussa une exclamation de surprise et d'indignation. 

«Mais c'est infâme! s'écria-t-il. 

\speak  Oh! quant à cela, dit Athos, je vous prie de croire que je m'en inquiète fort peu. Maintenant que vous avez fini, Grimaud, continua Athos, prenez la demi-pique de notre brigadier, attachez-y une serviette et plantez-la au haut de notre bastion, afin que ces rebelles de Rochelois voient qu'ils ont affaire à de braves et loyaux soldats du roi.» 

Grimaud obéit sans répondre. Un instant après le drapeau blanc flottait au-dessus de la tête des quatre amis; un tonnerre d'applaudissements salua son apparition; la moitié du camp était aux barrières. 

«Comment! reprit d'Artagnan, tu t'inquiètes fort peu qu'elle tue ou qu'elle fasse tuer Buckingham? Mais le duc est notre ami. 

\speak  Le duc est Anglais, le duc combat contre nous; qu'elle fasse du duc ce qu'elle voudra, je m'en soucie comme d'une bouteille vide.» 

Et Athos envoya à quinze pas de lui une bouteille qu'il tenait, et dont il venait de transvaser jusqu'à la dernière goutte dans son verre. 

«Un instant, dit d'Artagnan, je n'abandonne pas Buckingham ainsi; il nous avait donné de fort beaux chevaux. 

\speak  Et surtout de fort belles selles, ajouta Porthos, qui, à ce moment même, portait à son manteau le galon de la sienne. 

\speak  Puis, observa Aramis, Dieu veut la conversion et non la mort du pécheur. 

\speak  \textit{Amen}, dit Athos, et nous reviendrons là-dessus plus tard, si tel est votre plaisir; mais ce qui, pour le moment, me préoccupait le plus, et je suis sûr que tu me comprendras, d'Artagnan, c'était de reprendre à cette femme une espèce de blanc-seing qu'elle avait extorqué au cardinal, et à l'aide duquel elle devait impunément se débarrasser de toi et peut-être de nous. 

\speak  Mais c'est donc un démon que cette créature? dit Porthos en tendant son assiette à Aramis, qui découpait une volaille. 

\speak  Et ce blanc-seing, dit d'Artagnan, ce blanc-seing est-il resté entre ses mains? 

\speak  Non, il est passé dans les miennes; je ne dirai pas que ce fut sans peine, par exemple, car je mentirais. 

\speak  Mon cher Athos, dit d'Artagnan, je ne compte plus les fois que je vous dois la vie. 

\speak  Alors c'était donc pour venir près d'elle que vous nous avez quittés? demanda Aramis. 

\speak  Justement. Et tu as cette lettre du cardinal? dit d'Artagnan. 

\speak  La voici», dit Athos. 

Et il tira le précieux papier de la poche de sa casaque. 

D'Artagnan le déplia d'une main dont il n'essayait pas même de dissimuler le tremblement et lut: 

\begin{mail}{5 \textit{décembre} 1627.}

C'est par mon ordre et pour le bien de l'État que le porteur du présent a fait ce qu'il a fait.
\closeletter{Richelieu}
\end{mail}

«En effet, dit Aramis, c'est une absolution dans toutes les règles. 

\speak  Il faut déchirer ce papier, s'écria d'Artagnan, qui semblait lire sa sentence de mort. 

\speak  Bien au contraire, dit Athos, il faut le conserver précieusement, et je ne donnerais pas ce papier quand on le couvrirait de pièces d'or. 

\speak  Et que va-t-elle faire maintenant? demanda le jeune homme. 

\speak  Mais, dit négligemment Athos, elle va probablement écrire au cardinal qu'un damné mousquetaire, nommé Athos, lui a arraché son sauf-conduit; elle lui donnera dans la même lettre le conseil de se débarrasser, en même temps que de lui, de ses deux amis, Porthos et Aramis; le cardinal se rappellera que ce sont les mêmes hommes qu'il rencontre toujours sur son chemin; alors, un beau matin il fera arrêter d'Artagnan, et, pour qu'il ne s'ennuie pas tout seul, il nous enverra lui tenir compagnie à la Bastille. 

\speak  Ah çà, mais, dit Porthos, il me semble que vous faites là de tristes plaisanteries, mon cher. 

\speak  Je ne plaisante pas, répondit Athos. 

\speak  Savez-vous, dit Porthos, que tordre le cou à cette damnée Milady serait un péché moins grand que de le tordre à ces pauvres diables de huguenots, qui n'ont jamais commis d'autres crimes que de chanter en français des psaumes que nous chantons en latin? 

\speak  Qu'en dit l'abbé? demanda tranquillement Athos. 

\speak  Je dis que je suis de l'avis de Porthos, répondit Aramis. 

\speak  Et moi donc! fit d'Artagnan. 

\speak  Heureusement qu'elle est loin, observa Porthos; car j'avoue qu'elle me gênerait fort ici. 

\speak  Elle me gêne en Angleterre aussi bien qu'en France, dit Athos. 

\speak  Elle me gêne partout, continua d'Artagnan. 

\speak  Mais puisque vous la teniez, dit Porthos, que ne l'avez-vous noyée, étranglée, pendue? il n'y a que les morts qui ne reviennent pas. 

\speak  Vous croyez cela, Porthos? répondit le mousquetaire avec un sombre sourire que d'Artagnan comprit seul. 

\speak  J'ai une idée, dit d'Artagnan. 

\speak  Voyons, dirent les mousquetaires. 

\speak  Aux armes!» cria Grimaud. 

Les jeunes gens se levèrent vivement et coururent aux fusils. 

Cette fois, une petite troupe s'avançait composée de vingt ou vingt-cinq hommes; mais ce n'étaient plus des travailleurs, c'étaient des soldats de la garnison. 

«Si nous retournions au camp? dit Porthos, il me semble que la partie n'est pas égale. 

\speak  Impossible pour trois raisons, répondit Athos: la première, c'est que nous n'avons pas fini de déjeuner; la seconde, c'est que nous avons encore des choses d'importance à dire; la troisième, c'est qu'il s'en manque encore de dix minutes que l'heure ne soit écoulée. 

\speak  Voyons, dit Aramis, il faut cependant arrêter un plan de bataille. 

\speak  Il est bien simple, répondit Athos: aussitôt que l'ennemi est à portée de mousquet, nous faisons feu; s'il continue d'avancer, nous faisons feu encore, nous faisons feu tant que nous avons des fusils chargés; si ce qui reste de la troupe veut encore monter à l'assaut, nous laissons les assiégeants descendre jusque dans le fossé, et alors nous leur poussons sur la tête ce pan de mur qui ne tient plus que par un miracle d'équilibre. 

\speak  Bravo! s'écria Porthos; décidément, Athos, vous étiez né pour être général, et le cardinal, qui se croit un grand homme de guerre, est bien peu de chose auprès de vous. 

\speak  Messieurs, dit Athos, pas de double emploi, je vous prie; visez bien chacun votre homme. 

\speak  Je tiens le mien, dit d'Artagnan. 

\speak  Et moi le mien dit Porthos. 

\speak  Et moi idem, dit Aramis. 

\speak  Alors feu!» dit Athos. 

Les quatre coups de fusil ne firent qu'une détonation, et quatre hommes tombèrent. 

Aussitôt le tambour battit, et la petite troupe s'avança au pas de charge. 

Alors les coups de fusil se succédèrent sans régularité, mais toujours envoyés avec la même justesse. Cependant, comme s'ils eussent connu la faiblesse numérique des amis, les Rochelois continuaient d'avancer au pas de course. 

Sur trois autres coups de fusil, deux hommes tombèrent; mais cependant la marche de ceux qui restaient debout ne se ralentissait pas. 

Arrivés au bas du bastion, les ennemis étaient encore douze ou quinze; une dernière décharge les accueillit, mais ne les arrêta point: ils sautèrent dans le fossé et s'apprêtèrent à escalader la brèche. 

«Allons, mes amis, dit Athos, finissons-en d'un coup: à la muraille! à la muraille!» 

Et les quatre amis, secondés par Grimaud, se mirent à pousser avec le canon de leurs fusils un énorme pan de mur, qui s'inclina comme si le vent le poussait, et, se détachant de sa base, tomba avec un bruit horrible dans le fossé: puis on entendit un grand cri, un nuage de poussière monta vers le ciel, et tout fut dit. 

«Les aurions-nous écrasés depuis le premier jusqu'au dernier? demanda Athos. 

\speak  Ma foi, cela m'en a l'air, dit d'Artagnan. 

\speak  Non, dit Porthos, en voilà deux ou trois qui se sauvent tout éclopés.» 

En effet, trois ou quatre de ces malheureux, couverts de boue et de sang, fuyaient dans le chemin creux et regagnaient la ville: c'était tout ce qui restait de la petite troupe. 

Athos regarda à sa montre. 

«Messieurs, dit-il, il y a une heure que nous sommes ici, et maintenant le pari est gagné, mais il faut être beaux joueurs: d'ailleurs d'Artagnan ne nous a pas dit son idée.» 

Et le mousquetaire, avec son sang-froid habituel, alla s'asseoir devant les restes du déjeuner. 

«Mon idée? dit d'Artagnan. 

\speak  Oui, vous disiez que vous aviez une idée, répliqua Athos. 

\speak  Ah! j'y suis, reprit d'Artagnan: je passe en Angleterre une seconde fois, je vais trouver M. de Buckingham et je l'avertis du complot tramé contre sa vie. 

\speak  Vous ne ferez pas cela, d'Artagnan, dit froidement Athos. 

\speak  Et pourquoi cela? ne l'ai-je pas fait déjà? 

\speak  Oui, mais à cette époque nous n'étions pas en guerre; à cette époque, M. de Buckingham était un allié et non un ennemi: ce que vous voulez faire serait taxé de trahison.» 

D'Artagnan comprit la force de ce raisonnement et se tut. 

«Mais, dit Porthos, il me semble que j'ai une idée à mon tour. 

\speak  Silence pour l'idée de M. Porthos! dit Aramis. 

\speak  Je demande un congé à M. de Tréville, sous un prétexte quelconque que vous trouverez: je ne suis pas fort sur les prétextes, moi. Milady ne me connaît pas, je m'approche d'elle sans qu'elle me redoute, et lorsque je trouve ma belle, je l'étrangle. 

\speak  Eh bien, dit Athos, je ne suis pas très éloigné d'adopter l'idée de Porthos. 

\speak  Fi donc! dit Aramis, tuer une femme! Non, tenez, moi, j'ai la véritable idée. 

\speak  Voyons votre idée, Aramis! demanda Athos, qui avait beaucoup de déférence pour le jeune mousquetaire. 

\speak  Il faut prévenir la reine. 

\speak  Ah! ma foi, oui, s'écrièrent ensemble Porthos et d'Artagnan; je crois que nous touchons au moyen. 

\speak  Prévenir la reine! dit Athos, et comment cela? Avons-nous des relations à la cour? Pouvons-nous envoyer quelqu'un à Paris sans qu'on le sache au camp? D'ici à Paris il y a cent quarante lieues; notre lettre ne sera pas à Angers que nous serons au cachot, nous. 

\speak  Quant à ce qui est de faire remettre sûrement une lettre à Sa Majesté, proposa Aramis en rougissant, moi, je m'en charge; je connais à Tours une personne adroite\dots» 

Aramis s'arrêta en voyant sourire Athos. 

«Eh bien, vous n'adoptez pas ce moyen, Athos? dit d'Artagnan. 

\speak  Je ne le repousse pas tout à fait, dit Athos, mais je voulais seulement faire observer à Aramis qu'il ne peut quitter le camp; que tout autre qu'un de nous n'est pas sûr; que, deux heures après que le messager sera parti, tous les capucins, tous les alguazils, tous les bonnets noirs du cardinal sauront votre lettre par cœur, et qu'on arrêtera vous et votre adroite personne. 

\speak  Sans compter, objecta Porthos, que la reine sauvera M. de Buckingham, mais ne nous sauvera pas du tout, nous autres. 

\speak  Messieurs, dit d'Artagnan, ce qu'objecte Porthos est plein de sens. 

\speak  Ah! ah! que se passe-t-il donc dans la ville? dit Athos. 

\speak  On bat la générale.» 

Les quatre amis écoutèrent, et le bruit du tambour parvint effectivement jusqu'à eux. 

«Vous allez voir qu'ils vont nous envoyer un régiment tout entier, dit Athos. 

\speak  Vous ne comptez pas tenir contre un régiment tout entier? dit Porthos. 

\speak  Pourquoi pas? dit le mousquetaire, je me sens en train; et je tiendrais devant une armée, si nous avions seulement eu la précaution de prendre une douzaine de bouteilles en plus. 

\speak  Sur ma parole, le tambour se rapproche, dit d'Artagnan. 

\speak  Laissez-le se rapprocher, dit Athos; il y a pour un quart d'heure de chemin d'ici à la ville, et par conséquent de la ville ici. C'est plus de temps qu'il ne nous en faut pour arrêter notre plan; si nous nous en allons d'ici, nous ne retrouverons jamais un endroit aussi convenable. Et tenez, justement, messieurs, voilà la vraie idée qui me vient. 

\speak  Dites alors. 

\speak  Permettez que je donne à Grimaud quelques ordres indispensables.» 

Athos fit signe à son valet d'approcher. 

«Grimaud, dit Athos, en montrant les morts qui gisaient dans le bastion, vous allez prendre ces messieurs, vous allez les dresser contre la muraille, vous leur mettrez leur chapeau sur la tête et leur fusil à la main. 

\speak  O grand homme! s'écria d'Artagnan, je te comprends. 

\speak  Vous comprenez? dit Porthos. 

\speak  Et toi, comprends-tu, Grimaud?» demanda Aramis. 

Grimaud fit signe que oui. 

«C'est tout ce qu'il faut, dit Athos, revenons à mon idée. 

\speak  Je voudrais pourtant bien comprendre, observa Porthos. 

\speak  C'est inutile. 

\speak  Oui, oui, l'idée d'Athos, dirent en même temps d'Artagnan et Aramis. 

\speak  Cette Milady, cette femme, cette créature, ce démon, a un beau-frère, à ce que vous m'avez dit, je crois, d'Artagnan. 

\speak  Oui, je le connais beaucoup même, et je crois aussi qu'il n'a pas une grande sympathie pour sa belle-soeur. 

\speak  Il n'y a pas de mal à cela, répondit Athos, et il la détesterait que cela n'en vaudrait que mieux. 

\speak  En ce cas nous sommes servis à souhait. 

\speak  Cependant, dit Porthos, je voudrais bien comprendre ce que fait Grimaud. 

\speak  Silence, Porthos! dit Aramis. 

\speak  Comment se nomme ce beau-frère? 

\speak  Lord de Winter. 

\speak  Où est-il maintenant? 

\speak  Il est retourné à Londres au premier bruit de guerre. 

\speak  Eh bien, voilà justement l'homme qu'il nous faut, dit Athos, c'est celui qu'il nous convient de prévenir; nous lui ferons savoir que sa belle-soeur est sur le point d'assassiner quelqu'un, et nous le prierons de ne pas la perdre de vue. Il y a bien à Londres, je l'espère, quelque établissement dans le genre des Madelonnettes ou des Filles repenties; il y fait mettre sa belle-soeur, et nous sommes tranquilles. 

\speak  Oui, dit d'Artagnan, jusqu'à ce qu'elle en sorte. 

\speak  Ah! ma foi, reprit Athos, vous en demandez trop, d'Artagnan, je vous ai donné tout ce que j'avais et je vous préviens que c'est le fond de mon sac. 

\speak  Moi, je trouve que c'est ce qu'il y a de mieux, dit Aramis; nous prévenons à la fois la reine et Lord de Winter. 

\speak  Oui, mais par qui ferons-nous porter la lettre à Tours et la lettre à Londres? 

\speak  Je réponds de Bazin, dit Aramis. 

\speak  Et moi de Planchet, continua d'Artagnan. 

\speak  En effet, dit Porthos, si nous ne pouvons nous absenter du camp, nos laquais peuvent le quitter. 

\speak  Sans doute, dit Aramis, et dès aujourd'hui nous écrivons les lettres, nous leur donnons de l'argent, et ils partent. 

\speak  Nous leur donnons de l'argent? reprit Athos, vous en avez donc, de l'argent?» 

Les quatre amis se regardèrent, et un nuage passa sur les fronts qui s'étaient un instant éclaircis. 

«Alerte! cria d'Artagnan, je vois des points noirs et des points rouges qui s'agitent là-bas; que disiez-vous donc d'un régiment, Athos? c'est une véritable armée. 

\speak  Ma foi, oui, dit Athos, les voilà. Voyez-vous les sournois qui venaient sans tambours ni trompettes. Ah! ah! tu as fini, Grimaud?» 

Grimaud fit signe que oui, et montra une douzaine de morts qu'il avait placés dans les attitudes les plus pittoresques: les uns au port d'armes, les autres ayant l'air de mettre en joue, les autres l'épée à la main. 

«Bravo! reprit Athos, voilà qui fait honneur à ton imagination. 

\speak  C'est égal, dit Porthos, je voudrais cependant bien comprendre. 

\speak  Décampons d'abord, interrompit d'Artagnan, tu comprendras après. 

\speak  Un instant, messieurs, un instant! donnons le temps à Grimaud de desservir. 

\speak  Ah! dit Aramis, voici les points noirs et les points rouges qui grandissent fort visiblement et je suis de l'avis de d'Artagnan; je crois que nous n'avons pas de temps à perdre pour regagner notre camp. 

\speak  Ma foi, dit Athos, je n'ai plus rien contre la retraite: nous avions parié pour une heure, nous sommes restés une heure et demie; il n'y a rien à dire; partons, messieurs, partons.» 

Grimaud avait déjà pris les devants avec le panier et la desserte. 

Les quatre amis sortirent derrière lui et firent une dizaine de pas. 

«Eh! s'écria Athos, que diable faisons-nous, messieurs? 

\speak  Avez-vous oublié quelque chose? demanda Aramis. 

\speak  Et le drapeau, morbleu! Il ne faut pas laisser un drapeau aux mains de l'ennemi, même quand ce drapeau ne serait qu'une serviette.» 

Et Athos s'élança dans le bastion, monta sur la plate-forme, et enleva le drapeau; seulement comme les Rochelois étaient arrivés à portée de mousquet, ils firent un feu terrible sur cet homme, qui, comme par plaisir, allait s'exposer aux coups. 

Mais on eût dit qu'Athos avait un charme attaché à sa personne, les balles passèrent en sifflant tout autour de lui, pas une ne le toucha. 

Athos agita son étendard en tournant le dos aux gens de la ville et en saluant ceux du camp. Des deux côtés de grands cris retentirent, d'un côté des cris de colère, de l'autre des cris d'enthousiasme. 

Une seconde décharge suivit la première, et trois balles, en la trouant, firent réellement de la serviette un drapeau. On entendit les clameurs de tout le camp qui criait: 

\speak  Descendez, descendez!» 

Athos descendit; ses camarades, qui l'attendaient avec anxiété, le virent paraître avec joie. 

\speak  Allons, Athos, allons, dit d'Artagnan, allongeons, allongeons; maintenant que nous avons tout trouvé, excepté l'argent, il serait stupide d'être tués.» 

Mais Athos continua de marcher majestueusement, quelque observation que pussent lui faire ses compagnons, qui, voyant toute observation inutile, réglèrent leur pas sur le sien. 

Grimaud et son panier avaient pris les devants et se trouvaient tous deux hors d'atteinte. 

Au bout d'un instant on entendit le bruit d'une fusillade enragée. 

«Qu'est-ce que cela? demanda Porthos, et sur quoi tirent-ils? je n'entends pas siffler les balles et je ne vois personne. 

\speak  Ils tirent sur nos morts, répondit Athos. 

\speak  Mais nos morts ne répondront pas. 

\speak  Justement; alors ils croiront à une embuscade, ils délibéreront; ils enverront un parlementaire, et quand ils s'apercevront de la plaisanterie, nous serons hors de la portée des balles. Voilà pourquoi il est inutile de gagner une pleurésie en nous pressant. 

\speak  Oh! je comprends, s'écria Porthos émerveillé. 

\speak  C'est bien heureux!» dit Athos en haussant les épaules. 

De leur côté, les Français, en voyant revenir les quatre amis au pas, poussaient des cris d'enthousiasme. 

Enfin une nouvelle mousquetade se fit entendre, et cette fois les balles vinrent s'aplatir sur les cailloux autour des quatre amis et siffler lugubrement à leurs oreilles. Les Rochelois venaient enfin de s'emparer du bastion. 

«Voici des gens bien maladroits, dit Athos; combien en avons-nous tué? douze? 

\speak  Ou quinze. 

\speak  Combien en avons-nous écrasé? 

\speak  Huit ou dix. 

\speak  Et en échange de tout cela pas une égratignure? Ah! si fait! Qu'avez-vous donc là à la main, d'Artagnan? du sang, ce me semble? 

\speak  Ce n'est rien, dit d'Artagnan. 

\speak  Une balle perdue? 

\speak  Pas même. 

\speak  Qu'est-ce donc alors?» 

Nous l'avons dit, Athos aimait d'Artagnan comme son enfant, et ce caractère sombre et inflexible avait parfois pour le jeune homme des sollicitudes de père. 

«Une écorchure, reprit d'Artagnan; mes doigts ont été pris entre deux pierres, celle du mur et celle de ma bague; alors la peau s'est ouverte. 

\speak  Voilà ce que c'est que d'avoir des diamants, mon maître, dit dédaigneusement Athos. 

\speak  Ah çà, mais, s'écria Porthos, il y a un diamant en effet, et pourquoi diable alors, puisqu'il y a un diamant, nous plaignons-nous de ne pas avoir d'argent? 

\speak  Tiens, au fait! dit Aramis. 

\speak  À la bonne heure, Porthos; cette fois-ci voilà une idée. 

\speak  Sans doute, dit Porthos, en se rengorgeant sur le compliment d'Athos, puisqu'il y a un diamant, vendons-le. 

\speak  Mais, dit d'Artagnan, c'est le diamant de la reine. 

\speak  Raison de plus, reprit Athos, la reine sauvant M. de Buckingham son amant, rien de plus juste; la reine nous sauvant, nous ses amis, rien de plus moral: vendons le diamant. Qu'en pense monsieur l'abbé? Je ne demande pas l'avis de Porthos, il est donné. 

\speak  Mais je pense, dit Aramis en rougissant, que sa bague ne venant pas d'une maîtresse, et par conséquent n'étant pas un gage d'amour, d'Artagnan peut la vendre. 

\speak  Mon cher, vous parlez comme la théologie en personne. Ainsi votre avis est?\dots 

\speak  De vendre le diamant, répondit Aramis. 

\speak  Eh bien, dit gaiement d'Artagnan, vendons le diamant et n'en parlons plus.» 

La fusillade continuait, mais les amis étaient hors de portée, et les Rochelois ne tiraient plus que pour l'acquit de leur conscience. 

«Ma foi, dit Athos, il était temps que cette idée vînt à Porthos; nous voici au camp. Ainsi, messieurs, pas un mot de plus sur cette affaire. On nous observe, on vient à notre rencontre, nous allons être portés en triomphe.» 

En effet, comme nous l'avons dit, tout le camp était en émoi; plus de deux mille personnes avaient assisté, comme à un spectacle, à l'heureuse forfanterie des quatre amis, forfanterie dont on était bien loin de soupçonner le véritable motif. On n'entendait que le cri de: Vivent les gardes! Vivent les mousquetaires! M. de Busigny était venu le premier serrer la main à Athos et reconnaître que le pari était perdu. Le dragon et le Suisse l'avaient suivi, tous les camarades avaient suivi le dragon et le Suisse. C'étaient des félicitations, des poignées de main, des embrassades à n'en plus finir, des rires inextinguibles à l'endroit des Rochelois; enfin, un tumulte si grand, que M. le cardinal crut qu'il y avait émeute et envoya La Houdinière, son capitaine des gardes, s'informer de ce qui se passait. 

La chose fut racontée au messager avec toute l'efflorescence de l'enthousiasme. 

«Eh bien? demanda le cardinal en voyant La Houdinière. 

\speak  Eh bien, Monseigneur, dit celui-ci, ce sont trois mousquetaires et un garde qui ont fait le pari avec M. de Busigny d'aller déjeuner au bastion Saint-Gervais, et qui, tout en déjeunant, ont tenu là deux heures contre l'ennemi, et ont tué je ne sais combien de Rochelois. 

\speak  Vous êtes-vous informé du nom de ces trois mousquetaires? 

\speak  Oui, Monseigneur. 

\speak  Comment les appelle-t-on? 

\speak  Ce sont MM. Athos, Porthos et Aramis. 

\speak  Toujours mes trois braves! murmura le cardinal. Et le garde? 

\speak  M. d'Artagnan. 

\speak  Toujours mon jeune drôle! Décidément il faut que ces quatre hommes soient à moi.» 

Le soir même, le cardinal parla à M. de Tréville de l'exploit du matin, qui faisait la conversation de tout le camp. M. de Tréville, qui tenait le récit de l'aventure de la bouche même de ceux qui en étaient les héros, la raconta dans tous ses détails à Son Éminence, sans oublier l'épisode de la serviette. 

«C'est bien, monsieur de Tréville, dit le cardinal, faites-moi tenir cette serviette, je vous prie. J'y ferai broder trois fleurs de lis d'or, et je la donnerai pour guidon à votre compagnie. 

\speak  Monseigneur, dit M. de Tréville, il y aura injustice pour les gardes: M. d'Artagnan n'est pas à moi, mais à M. des Essarts. 

\speak  Eh bien, prenez-le, dit le cardinal; il n'est pas juste que, puisque ces quatre braves militaires s'aiment tant, ils ne servent pas dans la même compagnie.» 

Le même soir, M. de Tréville annonça cette bonne nouvelle aux trois mousquetaires et à d'Artagnan, en les invitant tous les quatre à déjeuner le lendemain. 

D'Artagnan ne se possédait pas de joie. On le sait, le rêve de toute sa vie avait été d'être mousquetaire. 

Les trois amis étaient fort joyeux. 

«Ma foi! dit d'Artagnan à Athos, tu as eu une triomphante idée, et, comme tu l'as dit, nous y avons acquis de la gloire, et nous avons pu lier une conversation de la plus haute importance. 

\speak  Que nous pourrons reprendre maintenant, sans que personne nous soupçonne; car, avec l'aide de Dieu, nous allons passer désormais pour des cardinalistes.» 

Le même soir, d'Artagnan alla présenter ses hommages à M. des Essarts, et lui faire part de l'avancement qu'il avait obtenu. 

M. des Essarts, qui aimait beaucoup d'Artagnan, lui fit alors ses offres de service: ce changement de corps amenant des dépenses d'équipement. 

D'Artagnan refusa; mais, trouvant l'occasion bonne, il le pria de faire estimer le diamant qu'il lui remit, et dont il désirait faire de l'argent. 

Le lendemain à huit heures du matin, le valet de M. des Essarts entra chez d'Artagnan, et lui remit un sac d'or contenant sept mille livres. 

C'était le prix du diamant de la reine. 
%!TeX root=../musketeersfr.tex 

\chapter{Affaire De Famille}

\lettrine{A}{thos} avait trouvé le mot: \textit{affaire de famille}. Une affaire de famille n'était point soumise à l'investigation du cardinal; une affaire de famille ne regardait personne; on pouvait s'occuper devant tout le monde d'une affaire de famille. 

Ainsi, Athos avait trouvé le mot: affaire de famille. 

Aramis avait trouvé l'idée: les laquais. 

Porthos avait trouvé le moyen: le diamant. 

D'Artagnan seul n'avait rien trouvé, lui ordinairement le plus inventif des quatre; mais il faut dire aussi que le nom seul de Milady le paralysait. 

Ah! si; nous nous trompons: il avait trouvé un acheteur pour le diamant. 

Le déjeuner chez M. de Tréville fut d'une gaieté charmante. D'Artagnan avait déjà son uniforme; comme il était à peu près de la même taille qu'Aramis, et qu'Aramis, largement payé, comme on se le rappelle, par le libraire qui lui avait acheté son poème, avait fait tout en double, il avait cédé à son ami un équipement complet. 

D'Artagnan eût été au comble de ses voeux, s'il n'eût point vu pointer Milady, comme un nuage sombre à l'horizon. 

Après déjeuner, on convint qu'on se réunirait le soir au logis d'Athos, et que là on terminerait l'affaire. 

D'Artagnan passa la journée à montrer son habit de mousquetaire dans toutes les rues du camp. 

Le soir, à l'heure dite, les quatre amis se réunirent: il ne restait plus que trois choses à décider: 

Ce qu'on écrirait au frère de Milady; 

Ce qu'on écrirait à la personne adroite de Tours; 

Et quels seraient les laquais qui porteraient les lettres. 

Chacun offrait le sien: Athos parlait de la discrétion de Grimaud, qui ne parlait que lorsque son maître lui décousait la bouche; Porthos vantait la force de Mousqueton, qui était de taille à rosser quatre hommes de complexion ordinaire; Aramis, confiant dans l'adresse de Bazin, faisait un éloge pompeux de son candidat; enfin, d'Artagnan avait foi entière dans la bravoure de Planchet, et rappelait de quelle façon il s'était conduit dans l'affaire épineuse de Boulogne. 

Ces quatre vertus disputèrent longtemps le prix, et donnèrent lieu à de magnifiques discours, que nous ne rapporterons pas ici, de peur qu'ils ne fassent longueur. 

«Malheureusement, dit Athos, il faudrait que celui qu'on enverra possédât en lui seul les quatre qualités réunies. 

\speak  Mais où rencontrer un pareil laquais? 

\speak  Introuvable! dit Athos; je le sais bien: prenez donc Grimaud. 

\speak  Prenez Mousqueton. 

\speak  Prenez Bazin. 

\speak  Prenez Planchet; Planchet est brave et adroit: c'est déjà deux qualités sur quatre. 

\speak  Messieurs, dit Aramis, le principal n'est pas de savoir lequel de nos quatre laquais est le plus discret, le plus fort, le plus adroit ou le plus brave; le principal est de savoir lequel aime le plus l'argent. 

\speak  Ce que dit Aramis est plein de sens, reprit Athos; il faut spéculer sur les défauts des gens et non sur leurs vertus: Monsieur l'abbé, vous êtes un grand moraliste! 

\speak  Sans doute, répliqua Aramis; car non seulement nous avons besoin d'être bien servis pour réussir, mais encore pour ne pas échouer; car, en cas d'échec, il y va de la tête, non pas pour les laquais\dots 

\speak  Plus bas, Aramis! dit Athos. 

\speak  C'est juste, non pas pour les laquais, reprit Aramis, mais pour le maître, et même pour les maîtres! Nos valets nous sont-ils assez dévoués pour risquer leur vie pour nous? Non. 

\speak  Ma foi, dit d'Artagnan, je répondrais presque de Planchet, moi. 

\speak  Eh bien, mon cher ami, ajoutez à son dévouement naturel une bonne somme qui lui donne quelque aisance, et alors, au lieu d'en répondre une fois, répondez-en deux. 

\speak  Eh! bon Dieu! vous serez trompés tout de même, dit Athos, qui était optimiste quand il s'agissait des choses, et pessimiste quand il s'agissait des hommes. Ils promettront tout pour avoir de l'argent, et en chemin la peur les empêchera d'agir. Une fois pris, on les serrera; serrés, ils avoueront. Que diable! nous ne sommes pas des enfants! Pour aller en Angleterre (Athos baissa la voix), il faut traverser toute la France, semée d'espions et de créatures du cardinal; il faut une passe pour s'embarquer; il faut savoir l'anglais pour demander son chemin à Londres. Tenez, je vois la chose bien difficile. 

\speak  Mais point du tout, dit d'Artagnan, qui tenait fort à ce que la chose s'accomplît; je la vois facile, au contraire, moi. Il va sans dire, parbleu! que si l'on écrit à Lord de Winter des choses par-dessus les maisons, des horreurs du cardinal\dots 

\speak  Plus bas! dit Athos. 

\speak  Des intrigues et des secrets d'état, continua d'Artagnan en se conformant à la recommandation, il va sans dire que nous serons tous roués vifs; mais, pour Dieu, n'oubliez pas, comme vous l'avez dit vous-même, Athos, que nous lui écrivons pour affaire de famille; que nous lui écrivons à cette seule fin qu'il mette Milady, dès son arrivée à Londres, hors d'état de nous nuire. Je lui écrirai donc une lettre à peu près en ces termes: 

\speak  Voyons, dit Aramis, en prenant par avance un visage de critique. 

\speak «Monsieur et cher ami\dots» 

\speak  Ah! oui; cher ami, à un Anglais, interrompit Athos; bien commencé! bravo, d'Artagnan! Rien qu'avec ce mot-là vous serez écartelé, au lieu d'être roué vif. 

\speak  Eh bien, soit; je dirai donc, monsieur, tout court. 

\speak  Vous pouvez même dire, Milord, reprit Athos, qui tenait fort aux convenances. 

\speak »Milord, vous souvient-il du petit enclos aux chèvres du Luxembourg?» 

\speak  Bon! le Luxembourg à présent! On croira que c'est une allusion à la reine mère! Voilà qui est ingénieux, dit Athos. 

\speak  Eh bien, nous mettrons tout simplement: «Milord, vous souvient-il de certain petit enclos où l'on vous sauva la vie?» 

\speak  Mon cher d'Artagnan, dit Athos, vous ne serez jamais qu'un fort mauvais rédacteur: «Où l'on vous sauva la vie!» Fi donc! ce n'est pas digne. On ne rappelle pas ces services-là à un galant homme. Bienfait reproché, offense faite. 

\speak  Ah! mon cher, dit d'Artagnan, vous êtes insupportable, et s'il faut écrire sous votre censure, ma foi, j'y renonce. 

\speak  Et vous faites bien. Maniez le mousquet et l'épée, mon cher, vous vous tirez galamment des deux exercices; mais passez la plume à M. l'abbé, cela le regarde. 

\speak  Ah! oui, au fait, dit Porthos, passez la plume à Aramis, qui écrit des thèses en latin, lui. 

\speak  Eh bien, soit dit d'Artagnan, rédigez-nous cette note, Aramis; mais, de par notre Saint-Père le pape! tenez-vous serré, car je vous épluche à mon tour, je vous en préviens. 

\speak  Je ne demande pas mieux, dit Aramis avec cette naïve confiance que tout poète a en lui-même; mais qu'on me mette au courant: j'ai bien ouï dire, de-ci de-là, que cette belle-sœur était une coquine, j'en ai même acquis la preuve en écoutant sa conversation avec le cardinal. 

\speak  Plus bas donc, sacrebleu! dit Athos. 

\speak  Mais, continua Aramis, le détail m'échappe. 

\speak  Et à moi aussi», dit Porthos. 

D'Artagnan et Athos se regardèrent quelque temps en silence. Enfin Athos, après s'être recueilli, et en devenant plus pâle encore qu'il n'était de coutume, fit un signe d'adhésion, d'Artagnan comprit qu'il pouvait parler. 

«Eh bien, voici ce qu'il y a à dire, reprit d'Artagnan: Milord, votre belle-sœur est une scélérate, qui a voulu vous faire tuer pour hériter de vous. Mais elle ne pouvait épouser votre frère, étant déjà mariée en France, et ayant été\dots» 

D'Artagnan s'arrêta comme s'il cherchait le mot, en regardant Athos. 

«Chassée par son mari, dit Athos. 

\speak  Parce qu'elle avait été marquée, continua d'Artagnan. 

\speak  Bah! s'écria Porthos, impossible! elle a voulu faire tuer son beau-frère? 

\speak  Oui. 

\speak  Elle était mariée? demanda Aramis. 

\speak  Oui. 

\speak  Et son mari s'est aperçu qu'elle avait une fleur de lis sur l'épaule? s'écria Porthos. 

\speak  Oui.» 

Ces trois oui avaient été dits par Athos, chacun avec une intonation plus sombre. 

«Et qui l'a vue, cette fleur de lis? demanda Aramis. 

\speak  D'Artagnan et moi, ou plutôt, pour observer l'ordre chronologique, moi et d'Artagnan, répondit Athos. 

\speak  Et le mari de cette affreuse créature vit encore? dit Aramis. 

\speak  Il vit encore. 

\speak  Vous en êtes sûr? 

\speak  J'en suis sûr.» 

Il y eut un instant de froid silence, pendant lequel chacun se sentit impressionné selon sa nature. 

«Cette fois, reprit Athos, interrompant le premier le silence, d'Artagnan nous a donné un excellent programme, et c'est cela qu'il faut écrire d'abord. 

\speak  Diable! vous avez raison, Athos, reprit Aramis, et la rédaction est épineuse. M. le chancelier lui-même serait embarrassé pour rédiger une épître de cette force, et cependant M. le chancelier rédige très agréablement un procès-verbal. N'importe! taisez-vous, j'écris.» 

Aramis en effet prit la plume, réfléchit quelques instants, se mit à écrire huit ou dix lignes d'une charmante petite écriture de femme, puis, d'une voix douce et lente, comme si chaque mot eût été scrupuleusement pesé, il lut ce qui suit: 
\begin{mail}{}{Milord,}
La personne qui vous écrit ces quelques lignes a eu l'honneur de croiser l'épée avec vous dans un petit enclos de la rue d'Enfer. Comme vous avez bien voulu, depuis, vous dire plusieurs fois l'ami de cette personne, elle vous doit de reconnaître cette amitié par un bon avis. Deux fois vous avez failli être victime d'une proche parente que vous croyez votre héritière, parce que vous ignorez qu'avant de contracter mariage en Angleterre, elle était déjà mariée en France. Mais, la troisième fois, qui est celle-ci, vous pouvez y succomber. Votre parente est partie de La Rochelle pour l'Angleterre pendant la nuit. Surveillez son arrivée car elle a de grands et terribles projets. Si vous tenez absolument à savoir ce dont elle est capable, lisez son passé sur son épaule gauche.
\end{mail}

«Eh bien, voilà qui est à merveille, dit Athos, et vous avez une plume de secrétaire d'état, mon cher Aramis. Lord de Winter fera bonne garde maintenant, si toutefois l'avis lui arrive; et tombât-il aux mains de Son Éminence elle-même, nous ne saurions être compromis. Mais comme le valet qui partira pourrait nous faire accroire qu'il a été à Londres et s'arrêter à Châtelleraut, ne lui donnons avec la lettre que la moitié de la somme en lui promettant l'autre moitié en échange de la réponse. Avez-vous le diamant? continua Athos. 

«J'ai mieux que cela, j'ai la somme.» 

Et d'Artagnan jeta le sac sur la table: au son de l'or, Aramis leva les yeux. Porthos tressaillit; quant à Athos, il resta impassible. 

«Combien dans ce petit sac? dit-il. 

\speak  Sept mille livres en louis de douze francs. 

\speak  Sept mille livres! s'écria Porthos, ce mauvais petit diamant valait sept mille livres? 

\speak  Il paraît, dit Athos, puisque les voilà; je ne présume pas que notre ami d'Artagnan y ait mis du sien. 

\speak  Mais, messieurs, dans tout cela, dit d'Artagnan, nous ne pensons pas à la reine. Soignons un peu la santé de son cher Buckingham. C'est le moins que nous lui devions. 

\speak  C'est juste, dit Athos, mais ceci regarde Aramis. 

\speak  Eh bien, répondit celui-ci en rougissant, que faut-il que je fasse? 

\speak  Mais, répliqua Athos, c'est tout simple: rédiger une seconde lettre pour cette adroite personne qui habite Tours.» 

Aramis reprit la plume, se mit à réfléchir de nouveau, et écrivit les lignes suivantes, qu'il soumit à l'instant même à l'approbation de ses amis: 

«Ma chère cousine\dots» 

«Ah! dit Athos, cette personne adroite est votre parente! 

\speak  Cousine germaine, dit Aramis. 

\speak  Va donc pour cousine!» 

Aramis continua: 

«Ma chère cousine, Son Éminence le cardinal, que Dieu conserve pour le bonheur de la France et la confusion des ennemis du royaume, est sur le point d'en finir avec les rebelles hérétiques de La Rochelle: il est probable que le secours de la flotte anglaise n'arrivera pas même en vue de la place; j'oserai même dire que je suis certain que M. de Buckingham sera empêché de partir par quelque grand événement. Son Éminence est le plus illustre politique des temps passés, du temps présent et probablement des temps à venir. Il éteindrait le soleil si le soleil le gênait. Donnez ces heureuses nouvelles à votre sœur, ma chère cousine. J'ai rêvé que cet Anglais maudit était mort. Je ne puis me rappeler si c'était par le fer ou par le poison; seulement ce dont je suis sûr, c'est que j'ai rêvé qu'il était mort, et, vous le savez, mes rêves ne me trompent jamais. Assurez-vous donc de me voir revenir bientôt.» 

«À merveille! s'écria Athos, vous êtes le roi des poètes; mon cher Aramis, vous parlez comme l'Apocalypse et vous êtes vrai comme l'évangile. Il ne vous reste maintenant que l'adresse à mettre sur cette lettre. 

\speak  C'est bien facile», dit Aramis. 

Il plia coquettement la lettre, la reprit et écrivit: 

«À Mademoiselle Marie Michon, lingère à Tours. 

Les trois amis se regardèrent en riant: ils étaient pris. 

«Maintenant, dit Aramis, vous comprenez, messieurs, que Bazin seul peut porter cette lettre à Tours; ma cousine ne connaît que Bazin et n'a confiance qu'en lui: tout autre ferait échouer l'affaire. D'ailleurs Bazin est ambitieux et savant; Bazin a lu l'histoire, messieurs, il sait que Sixte Quint est devenu pape après avoir gardé les pourceaux; eh bien, comme il compte se mettre d'église en même temps que moi, il ne désespère pas à son tour de devenir pape ou tout au moins cardinal: vous comprenez qu'un homme qui a de pareilles visées ne se laissera pas prendre, ou, s'il est pris, subira le martyre plutôt que de parler. 

\speak  Bien, bien, dit d'Artagnan, je vous passe de grand cœur Bazin; mais passez-moi Planchet: Milady l'a fait jeter à la porte, certain jour, avec force coups de bâton; or Planchet a bonne mémoire, et, je vous en réponds, s'il peut supposer une vengeance possible, il se fera plutôt échiner que d'y renoncer. Si vos affaires de Tours sont vos affaires, Aramis, celles de Londres sont les miennes. Je prie donc qu'on choisisse Planchet, lequel d'ailleurs a déjà été à Londres avec moi et sait dire très correctement: London, \textit{sir, if you please} et \textit{my master} lord d'Artagnan; avec cela soyez tranquilles, il fera son chemin en allant et en revenant. 

\speak  En ce cas, dit Athos, il faut que Planchet reçoive sept cents livres pour aller et sept cents livres pour revenir, et Bazin, trois cents livres pour aller et trois cents livres pour revenir; cela réduira la somme à cinq mille livres; nous prendrons mille livres chacun pour les employer comme bon nous semblera, et nous laisserons un fond de mille livres que gardera l'abbé pour les cas extraordinaires ou les besoins communs. Cela vous va-t-il? 

\speak  Mon cher Athos, dit Aramis, vous parlez comme Nestor, qui était, comme chacun sait, le plus sage des Grecs. 

\speak  Eh bien, c'est dit, reprit Athos, Planchet et Bazin partiront; à tout prendre, je ne suis pas fâché de conserver Grimaud: il est accoutumé à mes façons et j'y tiens; la journée d'hier a déjà dû l'ébranler, ce voyage le perdrait.» 

On fit venir Planchet, et on lui donna des instructions; il avait été prévenu déjà par d'Artagnan, qui, du premier coup, lui avait annoncé la gloire, ensuite l'argent, puis le danger. 

«Je porterai la lettre dans le parement de mon habit, dit Planchet, et je l'avalerai si l'on me prend. 

\speak  Mais alors tu ne pourras pas faire la commission, dit d'Artagnan. 

\speak  Vous m'en donnerez ce soir une copie que je saurai par cœur demain.» 

D'Artagnan regarda ses amis comme pour leur dire: 

«Eh bien, que vous avais-je promis?» 

«Maintenant, continua-t-il en s'adressant à Planchet, tu as huit jours pour arriver près de Lord de Winter, tu as huit autres jours pour revenir ici, en tout seize jours; si le seizième jour de ton départ, à huit heures du soir, tu n'es pas arrivé, pas d'argent, fût-il huit heures cinq minutes. 

Alors, monsieur, dit Planchet, achetez-moi une montre. 

Prends celle-ci, dit Athos, en lui donnant la sienne avec une insouciante générosité, et sois brave garçon. Songe que, si tu parles, si tu bavardes, si tu flânes, tu fais couper le cou à ton maître, qui a si grande confiance dans ta fidélité qu'il nous a répondu de toi. Mais songe aussi que s'il arrive, par ta faute, malheur à d'Artagnan, je te retrouverai partout, et ce sera pour t'ouvrir le ventre. 

\speak  Oh! monsieur! dit Planchet, humilié du soupçon et surtout effrayé de l'air calme du mousquetaire. 

\speak  Et moi, dit Porthos en roulant ses gros yeux, songe que je t'écorche vif. 

\speak  Ah! monsieur! 

\speak  Et moi, continua Aramis de sa voix douce et mélodieuse, songe que je te brûle à petit feu comme un sauvage. 

\speak  Ah! monsieur!» 

Et Planchet se mit à pleurer; nous n'oserions dire si ce fut de terreur, à cause des menaces qui lui étaient faites, ou d'attendrissement de voir quatre amis si étroitement unis. 

D'Artagnan lui prit la main, et l'embrassa. 

«Vois-tu, Planchet, lui dit-il, ces messieurs te disent tout cela par tendresse pour moi, mais au fond ils t'aiment. 

\speak  Ah! monsieur! dit Planchet, ou je réussirai, ou l'on me coupera en quatre; me coupât-on en quatre, soyez convaincu qu'il n'y a pas un morceau qui parlera.» 

Il fut décidé que Planchet partirait le lendemain à huit heures du matin, afin, comme il l'avait dit, qu'il pût, pendant la nuit, apprendre la lettre par cœur. Il gagna juste douze heures à cet arrangement; il devait être revenu le seizième jour, à huit heures du soir. 

Le matin, au moment où il allait monter à cheval, d'Artagnan, qui se sentait au fond du cœur un faible pour le duc, prit Planchet à part. 

«Écoute, lui dit-il, quand tu auras remis la lettre à Lord de Winter et qu'il l'aura lue, tu lui diras encore: “Veillez sur Sa Grâce Lord Buckingham, car on veut l'assassiner.” Mais ceci, Planchet, vois-tu, c'est si grave et si important, que je n'ai pas même voulu avouer à mes amis que je te confierais ce secret, et que pour une commission de capitaine je ne voudrais pas te l'écrire. 

\speak  Soyez tranquille, monsieur, dit Planchet, vous verrez si l'on peut compter sur moi. 

Et monté sur un excellent cheval, qu'il devait quitter à vingt lieues de là pour prendre la poste, Planchet partit au galop, le cœur un peu serré par la triple promesse que lui avaient faite les mousquetaires, mais du reste dans les meilleures dispositions du monde. 

Bazin partit le lendemain matin pour Tours, et eut huit jours pour faire sa commission. 

Les quatre amis, pendant toute la durée de ces deux absences, avaient, comme on le comprend bien, plus que jamais l'œil au guet, le nez au vent et l'oreille aux écoutes. Leurs journées se passaient à essayer de surprendre ce qu'on disait, à guetter les allures du cardinal et à flairer les courriers qui arrivaient. Plus d'une fois un tremblement insurmontable les prit, lorsqu'on les appela pour quelque service inattendu. Ils avaient d'ailleurs à se garder pour leur propre sûreté; Milady était un fantôme qui, lorsqu'il était apparu une fois aux gens, ne les laissait pas dormir tranquillement. 

Le matin du huitième jour, Bazin, frais comme toujours et souriant selon son habitude, entra dans le cabaret de Parpaillot, comme les quatre amis étaient en train de déjeuner, en disant, selon la convention arrêtée: 

«Monsieur Aramis, voici la réponse de votre cousine.» 

Les quatre amis échangèrent un coup d'œil joyeux: la moitié de la besogne était faite; il est vrai que c'était la plus courte et la plus facile. 

Aramis prit, en rougissant malgré lui, la lettre, qui était d'une écriture grossière et sans orthographe. 

«Bon Dieu! s'écria-t-il en riant, décidément j'en désespère; jamais cette pauvre Michon n'écrira comme M. de Voiture. 

\speak  Qu'est-ce que cela feut dire, cette baufre Migeon? demanda le Suisse, qui était en train de causer avec les quatre amis quand la lettre était arrivée. 

\speak  Oh! mon Dieu! moins que rien, dit Aramis, une petite lingère charmante que j'aimais fort et à qui j'ai demandé quelques lignes de sa main en manière de souvenir. 

\speak  Dutieu! dit le Suisse; zi zella il être auzi grante tame que son l'égridure, fous l'être en ponne fordune, mon gamarate! 

Aramis lut la lettre et la passa à Athos. 

«Voyez donc ce qu'elle m'écrit, Athos», dit-il. 

Athos jeta un coup d'œil sur l'épître, et, pour faire évanouir tous les soupçons qui auraient pu naître, lut tout haut: 

\begin{mail}{}{Mon cousin,} 
	Ma sœur et moi devinons très bien les rêves, et nous en avons même une peur affreuse; mais du vôtre, on pourra dire, je l'espère, tout songe est mensonge. Adieu! portez-vous bien, et faites que de temps en temps nous entendions parler de vous. 
	
	\closeletter{Aglaé Michon.}
\end{mail}

«Et de quel rêve parle-t-elle? demanda le dragon, qui s'était approché pendant la lecture. 

\speak  Foui, te quel rêfe? dit le Suisse. 

\speak  Eh! pardieu! dit Aramis, c'est tout simple, d'un rêve que j'ai fait et que je lui ai raconté. 

\speak  Oh! foui, par Tieu! c'être tout simple de ragonter son rêfe; mais moi je ne rêfe jamais. 

\speak  Vous êtes fort heureux, dit Athos en se levant, et je voudrais bien pouvoir en dire autant que vous! 

\speak  Chamais! reprit le Suisse, enchanté qu'un homme comme Athos lui enviât quelque chose, chamais! chamais!» 

D'Artagnan, voyant qu'Athos se levait, en fit autant, prit son bras, et sortit. 

Porthos et Aramis restèrent pour faire face aux quolibets du dragon et du Suisse. 

Quant à Bazin, il s'alla coucher sur une botte de paille; et comme il avait plus d'imagination que le Suisse, il rêva que M. Aramis, devenu pape, le coiffait d'un chapeau de cardinal. 

Mais, comme nous l'avons dit, Bazin n'avait, par son heureux retour, enlevé qu'une partie de l'inquiétude qui aiguillonnait les quatre amis. Les jours de l'attente sont longs, et d'Artagnan surtout aurait parié que les jours avaient maintenant quarante-huit heures. Il oubliait les lenteurs obligées de la navigation, il s'exagérait la puissance de Milady. Il prêtait à cette femme, qui lui apparaissait pareille à un démon, des auxiliaires surnaturels comme elle; il s'imaginait, au moindre bruit, qu'on venait l'arrêter, et qu'on ramenait Planchet pour le confronter avec lui et ses amis. Il y a plus: sa confiance autrefois si grande dans le digne Picard, diminuait de jour en jour. Cette inquiétude était si grande, qu'elle gagnait Porthos et Aramis. Il n'y avait qu'Athos qui demeurât impassible, comme si aucun danger ne s'agitait autour de lui, et qu'il respirât son atmosphère quotidienne. 

Le seizième jour surtout, ces signes d'agitation étaient si visibles chez d'Artagnan et ses deux amis, qu'ils ne pouvaient rester en place, et qu'ils erraient comme des ombres sur le chemin par lequel devait revenir Planchet. 

«Vraiment, leur disait Athos, vous n'êtes pas des hommes, mais des enfants, pour qu'une femme vous fasse si grand-peur! Et de quoi s'agit-il, après tout? D'être emprisonnés! Eh bien, mais on nous tirera de prison: on en a bien retiré Mme Bonacieux. D'être décapités? Mais tous les jours, dans la tranchée, nous allons joyeusement nous exposer à pis que cela, car un boulet peut nous casser la jambe, et je suis convaincu qu'un chirurgien nous fait plus souffrir en nous coupant la cuisse qu'un bourreau en nous coupant la tête. Demeurez donc tranquilles; dans deux heures, dans quatre, dans six heures, au plus tard, Planchet sera ici: il a promis d'y être, et moi j'ai très grande foi aux promesses de Planchet, qui m'a l'air d'un fort brave garçon. 

\speak  Mais s'il n'arrive pas? dit d'Artagnan. 

\speak  Eh bien, s'il n'arrive pas, c'est qu'il aura été retardé, voilà tout. Il peut être tombé de cheval, il peut avoir fait une cabriole par-dessus le pont, il peut avoir couru si vite qu'il en ait attrapé une fluxion de poitrine. Eh! messieurs! faisons donc la part des événements. La vie est un chapelet de petites misères que le philosophe égrène en riant. Soyez philosophes comme moi, messieurs, mettez-vous à table et buvons; rien ne fait paraître l'avenir couleur de rose comme de le regarder à travers un verre de chambertin. 

\speak  C'est fort bien, répondit d'Artagnan; mais je suis las d'avoir à craindre, en buvant frais, que le vin ne sorte de la cave de Milady. 

\speak  Vous êtes bien difficile, dit Athos, une si belle femme! 

\speak  Une femme de marque!» dit Porthos avec son gros rire. 

Athos tressaillit, passa la main sur son front pour en essuyer la sueur, et se leva à son tour avec un mouvement nerveux qu'il ne put réprimer. 

Le jour s'écoula cependant, et le soir vint plus lentement, mais enfin il vint; les buvettes s'emplirent de chalands; Athos, qui avait empoché sa part du diamant, ne quittait plus le Parpaillot. Il avait trouvé dans M. de Busigny, qui, au reste, leur avait donné un dîner magnifique, un \textit{partner} digne de lui. Ils jouaient donc ensemble, comme d'habitude, quand sept heures sonnèrent: on entendit passer les patrouilles qui allaient doubler les postes; à sept heures et demie la retraite sonna. 

«Nous sommes perdus, dit d'Artagnan à l'oreille d'Athos. 

\speak  Vous voulez dire que nous avons perdu, dit tranquillement Athos en tirant quatre pistoles de sa poche et en les jetant sur la table. Allons, messieurs, continua-t-il, on bat la retraite, allons nous coucher.» 

Et Athos sortit du Parpaillot suivi de d'Artagnan. Aramis venait derrière donnant le bras à Porthos. Aramis mâchonnait des vers, et Porthos s'arrachait de temps en temps quelques poils de moustache en signe de désespoir. 

Mais voilà que tout à coup, dans l'obscurité, une ombre se dessine, dont la forme est familière à d'Artagnan, et qu'une voix bien connue lui dit: 

«Monsieur, je vous apporte votre manteau, car il fait frais ce soir. 

\speak  Planchet! s'écria d'Artagnan, ivre de joie. 

\speak  Planchet! répétèrent Porthos et Aramis. 

\speak  Eh bien, oui, Planchet, dit Athos, qu'y a-t-il d'étonnant à cela? Il avait promis d'être de retour à huit heures, et voilà les huit heures qui sonnent. Bravo! Planchet, vous êtes un garçon de parole, et si jamais vous quittez votre maître, je vous garde une place à mon service. 

\speak  Oh! non, jamais, dit Planchet, jamais je ne quitterai M. d'Artagnan.» 

En même temps d'Artagnan sentit que Planchet lui glissait un billet dans la main. 

D'Artagnan avait grande envie d'embrasser Planchet au retour comme il l'avait embrassé au départ; mais il eut peur que cette marque d'effusion, donnée à son laquais en pleine rue, ne parût extraordinaire à quelque passant, et il se contint. 

«J'ai le billet, dit-il à Athos et à ses amis. 

\speak  C'est bien, dit Athos, entrons chez nous, et nous le lirons. 

Le billet brûlait la main de d'Artagnan: il voulait hâter le pas; mais Athos lui prit le bras et le passa sous le sien, et force fut au jeune homme de régler sa course sur celle de son ami. 

Enfin on entra dans la tente, on alluma une lampe, et tandis que Planchet se tenait sur la porte pour que les quatre amis ne fussent pas surpris, d'Artagnan, d'une main tremblante, brisa le cachet et ouvrit la lettre tant attendue. 

Elle contenait une demi-ligne, d'une écriture toute britannique et d'une concision toute spartiate: 

«\textit{Thank you, be easy.}» 

Ce qui voulait dire: 

«Merci, soyez tranquille.» 

Athos prit la lettre des mains de d'Artagnan, l'approcha de la lampe, y mit le feu, et ne la lâcha point qu'elle ne fût réduite en cendres. 

Puis appelant Planchet: 

«Maintenant, mon garçon, lui dit-il, tu peux réclamer tes sept cents livres, mais tu ne risquais pas grand-chose avec un billet comme celui-là. 

\speak  Ce n'est pas faute que j'aie inventé bien des moyens de le serrer, dit Planchet. 

\speak  Eh bien, dit d'Artagnan, conte-nous cela. 

\speak  Dame! c'est bien long, monsieur. 

\speak  Tu as raison, Planchet, dit Athos; d'ailleurs la retraite est battue, et nous serions remarqués en gardant de la lumière plus longtemps que les autres. 

\speak  Soit, dit d'Artagnan, couchons-nous. Dors bien, Planchet! 

\speak  Ma foi, monsieur! ce sera la première fois depuis seize jours. 

\speak  Et moi aussi! dit d'Artagnan. 

\speak  Et moi aussi! répéta Porthos. 

\speak  Et moi aussi! répéta Aramis. 

\speak  Eh bien, voulez-vous que je vous avoue la vérité? et moi aussi!» dit Athos.
%!TeX root=../musketeersfr.tex 

\chapter{Fatalité}

\lettrine{C}{ependant} Milady, ivre de colère, rugissant sur le pont du bâtiment comme une lionne qu'on embarque, avait été tentée de se jeter à la mer pour regagner la côte, car elle ne pouvait se faire à l'idée qu'elle avait été insultée par d'Artagnan, menacée par Athos, et qu'elle quittait la France sans se venger d'eux. Bientôt, cette idée était devenue pour elle tellement insupportable, qu'au risque de ce qui pouvait arriver de terrible pour elle-même, elle avait supplié le capitaine de la jeter sur la côte; mais le capitaine, pressé d'échapper à sa fausse position, placé entre les croiseurs français et anglais, comme la chauve-souris entre les rats et les oiseaux, avait grande hâte de regagner l'Angleterre, et refusa obstinément d'obéir à ce qu'il prenait pour un caprice de femme, promettant à sa passagère, qui au reste lui était particulièrement recommandée par le cardinal, de la jeter, si la mer et les Français le permettaient, dans un des ports de la Bretagne, soit à Lorient, soit à Brest; mais en attendant, le vent était contraire, la mer mauvaise, on louvoyait et l'on courait des bordées. Neuf jours après la sortie de la Charente, Milady, toute pâle de ses chagrins et de sa rage, voyait apparaître seulement les côtes bleuâtres du Finistère. 

Elle calcula que pour traverser ce coin de la France et revenir près du cardinal il lui fallait au moins trois jours; ajoutez un jour pour le débarquement et cela faisait quatre; ajoutez ces quatre jours aux neuf autres, c'était treize jours de perdus, treize jours pendant lesquels tant d'événements importants se pouvaient passer à Londres. Elle songea que sans aucun doute le cardinal serait furieux de son retour, et que par conséquent il serait plus disposé à écouter les plaintes qu'on porterait contre elle que les accusations qu'elle porterait contre les autres. Elle laissa donc passer Lorient et Brest sans insister près du capitaine, qui, de son côté, se garda bien de lui donner l'éveil. Milady continua donc sa route, et le jour même où Planchet s'embarquait de Portsmouth pour la France, la messagère de son Éminence entrait triomphante dans le port. 

Toute la ville était agitée d'un mouvement extraordinaire: --- quatre grands vaisseaux récemment achevés venaient d'être lancés à la mer; --- debout sur la jetée, chamarré d'or, éblouissant, selon son habitude de diamants et de pierreries, le feutre orné d'une plume blanche qui retombait sur son épaule, on voyait Buckingham entouré d'un état-major presque aussi brillant que lui. 

C'était une de ces belles et rares journées d'hiver où l'Angleterre se souvient qu'il y a un soleil. L'astre pâli, mais cependant splendide encore, se couchait à l'horizon, empourprant à la fois le ciel et la mer de bandes de feu et jetant sur les tours et les vieilles maisons de la ville un dernier rayon d'or qui faisait étinceler les vitres comme le reflet d'un incendie. Milady, en respirant cet air de l'Océan plus vif et plus balsamique à l'approche de la terre, en contemplant toute la puissance de ces préparatifs qu'elle était chargée de détruire, toute la puissance de cette armée qu'elle devait combattre à elle seule --- elle femme --- avec quelques sacs d'or, se compara mentalement à Judith, la terrible Juive, lorsqu'elle pénétra dans le camp des Assyriens et qu'elle vit la masse énorme de chars, de chevaux, d'hommes et d'armes qu'un geste de sa main devait dissiper comme un nuage de fumée. 

On entra dans la rade; mais comme on s'apprêtait à y jeter l'ancre, un petit cutter formidablement armé s'approcha du bâtiment marchand, se donnant comme garde-côte, et fit mettre à la mer son canot, qui se dirigea vers l'échelle. Ce canot renfermait un officier, un contremaître et huit rameurs; l'officier seul monta à bord, où il fut reçu avec toute la déférence qu'inspire l'uniforme. 

L'officier s'entretint quelques instants avec le patron, lui fit lire un papier dont il était porteur, et, sur l'ordre du capitaine marchand, tout l'équipage du bâtiment, matelots et passagers, fut appelé sur le pont. 

Lorsque cette espèce d'appel fut fait, l'officier s'enquit tout haut du point de départ du brik, de sa route, de ses atterrissements, et à toutes les questions le capitaine satisfit sans hésitation et sans difficulté. Alors l'officier commença de passer la revue de toutes les personnes les unes après les autres, et, s'arrêtant à Milady, la considéra avec un grand soin, mais sans lui adresser une seule parole. 

Puis il revint au capitaine, lui dit encore quelques mots; et, comme si c'eût été à lui désormais que le bâtiment dût obéir, il commanda une manoeuvre que l'équipage exécuta aussitôt. Alors le bâtiment se remit en route, toujours escorté du petit cutter, qui voguait bord à bord avec lui, menaçant son flanc de la bouche de ses six canons tandis que la barque suivait dans le sillage du navire, faible point près de l'énorme masse. 

Pendant l'examen que l'officier avait fait de Milady, Milady, comme on le pense bien, l'avait de son côté dévoré du regard. Mais, quelque habitude que cette femme aux yeux de flamme eût de lire dans le cœur de ceux dont elle avait besoin de deviner les secrets, elle trouva cette fois un visage d'une impassibilité telle qu'aucune découverte ne suivit son investigation. L'officier qui s'était arrêté devant elle et qui l'avait silencieusement étudiée avec tant de soin pouvait être âgé de vingt-cinq à vingt-six ans, était blanc de visage avec des yeux bleu clair un peu enfoncés; sa bouche, fine et bien dessinée, demeurait immobile dans ses lignes correctes; son menton, vigoureusement accusé, dénotait cette force de volonté qui, dans le type vulgaire britannique, n'est ordinairement que de l'entêtement; un front un peu fuyant, comme il convient aux poètes, aux enthousiastes et aux soldats, était à peine ombragé d'une chevelure courte et clairsemée, qui, comme la barbe qui couvrait le bas de son visage, était d'une belle couleur châtain foncé. 

Lorsqu'on entra dans le port, il faisait déjà nuit. La brume épaississait encore l'obscurité et formait autour des fanaux et des lanternes des jetées un cercle pareil à celui qui entoure la lune quand le temps menace de devenir pluvieux. L'air qu'on respirait était triste, humide et froid. 

Milady, cette femme si forte, se sentait frissonner malgré elle. 

L'officier se fit indiquer les paquets de Milady, fit porter son bagage dans le canot; et lorsque cette opération fut faite, il l'invita à y descendre elle-même en lui tendant sa main. 

Milady regarda cet homme et hésita. 

«Qui êtes-vous, monsieur, demanda-t-elle, qui avez la bonté de vous occuper si particulièrement de moi? 

\speak  Vous devez le voir, madame, à mon uniforme; je suis officier de la marine anglaise, répondit le jeune homme. 

\speak  Mais enfin, est-ce l'habitude que les officiers de la marine anglaise se mettent aux ordres de leurs compatriotes lorsqu'ils abordent dans un port de la Grande-Bretagne, et poussent la galanterie jusqu'à les conduire à terre? 

\speak  Oui, Milady, c'est l'habitude, non point par galanterie, mais par prudence, qu'en temps de guerre les étrangers soient conduits à une hôtellerie désignée, afin que jusqu'à parfaite information sur eux ils restent sous la surveillance du gouvernement.» 

Ces mots furent prononcés avec la politesse la plus exacte et le calme le plus parfait. Cependant ils n'eurent point le don de convaincre Milady. 

«Mais je ne suis pas étrangère, monsieur, dit-elle avec l'accent le plus pur qui ait jamais retenti de Portsmouth à Manchester, je me nomme Lady Clarick, et cette mesure\dots 

\speak  Cette mesure est générale, Milady, et vous tenteriez inutilement de vous y soustraire. 

\speak  Je vous suivrai donc, monsieur.» 

Et acceptant la main de l'officier, elle commença de descendre l'échelle au bas de laquelle l'attendait le canot. L'officier la suivit; un grand manteau était étendu à la poupe, l'officier la fit asseoir sur le manteau et s'assit près d'elle. 

«Nagez», dit-il aux matelots. 

Les huit rames retombèrent dans la mer, ne formant qu'un seul bruit, ne frappant qu'un seul coup, et le canot sembla voler sur la surface de l'eau. 

Au bout de cinq minutes on touchait à terre. 

L'officier sauta sur le quai et offrit la main à Milady. 

Une voiture attendait. 

«Cette voiture est-elle pour nous? demanda Milady. 

\speak  Oui, madame, répondit l'officier. 

\speak  L'hôtellerie est donc bien loin? 

\speak  À l'autre bout de la ville. 

\speak  Allons», dit Milady. 

Et elle monta résolument dans la voiture. 

L'officier veilla à ce que les paquets fussent soigneusement attachés derrière la caisse, et cette opération terminée, prit sa place près de Milady et referma la portière. 

Aussitôt, sans qu'aucun ordre fût donné et sans qu'on eût besoin de lui indiquer sa destination, le cocher partit au galop et s'enfonça dans les rues de la ville. 

Une réception si étrange devait être pour Milady une ample matière à réflexion; aussi, voyant que le jeune officier ne paraissait nullement disposé à lier conversation, elle s'accouda dans un angle de la voiture et passa les unes après les autres en revue toutes les suppositions qui se présentaient à son esprit. 

Cependant, au bout d'un quart d'heure, étonnée de la longueur du chemin, elle se pencha vers la portière pour voir où on la conduisait. On n'apercevait plus de maisons; des arbres apparaissaient dans les ténèbres comme de grands fantômes noirs courant les uns après les autres. 

Milady frissonna. 

«Mais nous ne sommes plus dans la ville, monsieur», dit-elle. 

Le jeune officier garda le silence. 

«Je n'irai pas plus loin, si vous ne me dites pas où vous me conduisez; je vous en préviens, monsieur!» 

Cette menace n'obtint aucune réponse. 

«Oh! c'est trop fort! s'écria Milady, au secours! au secours!» 

Pas une voix ne répondit à la sienne, la voiture continua de rouler avec rapidité; l'officier semblait une statue. 

Milady regarda l'officier avec une de ces expressions terribles, particulières à son visage et qui manquaient si rarement leur effet; la colère faisait étinceler ses yeux dans l'ombre. 

Le jeune homme resta impassible. 

Milady voulut ouvrir la portière et se précipiter. 

«Prenez garde, madame, dit froidement le jeune homme, vous vous tuerez en sautant.» 

Milady se rassit écumante; l'officier se pencha, la regarda à son tour et parut surpris de voir cette figure, si belle naguère, bouleversée par la rage et devenue presque hideuse. L'astucieuse créature comprit qu'elle se perdait en laissant voir ainsi dans son âme; elle rasséréna ses traits, et d'une voix gémissante: 

«Au nom du Ciel, monsieur! dites-moi si c'est à vous, si c'est à votre gouvernement, si c'est à un ennemi que je dois attribuer la violence que l'on me fait? 

\speak  On ne vous fait aucune violence, madame, et ce qui vous arrive est le résultat d'une mesure toute simple que nous sommes forcés de prendre avec tous ceux qui débarquent en Angleterre. 

\speak  Alors vous ne me connaissez pas, monsieur? 

\speak  C'est la première fois que j'ai l'honneur de vous voir. 

\speak  Et, sur votre honneur, vous n'avez aucun sujet de haine contre moi? 

\speak  Aucun, je vous le jure.» 

II y avait tant de sérénité, de sang-froid, de douceur même dans la voix du jeune homme, que Milady fut rassurée. 

Enfin, après une heure de marche à peu près, la voiture s'arrêta devant une grille de fer qui fermait un chemin creux conduisant à un château sévère de forme, massif et isolé. Alors, comme les roues tournaient sur un sable fin, Milady entendit un vaste mugissement, qu'elle reconnut pour le bruit de la mer qui vient se briser sur une côte escarpée. 

La voiture passa sous deux voûtes, et enfin s'arrêta dans une cour sombre et carrée; presque aussitôt la portière de la voiture s'ouvrit, le jeune homme sauta légèrement à terre et présenta sa main à Milady, qui s'appuya dessus, et descendit à son tour avec assez de calme. 

«Toujours est-il, dit Milady en regardant autour d'elle et en ramenant ses yeux sur le jeune officier avec le plus gracieux sourire, que je suis prisonnière; mais ce ne sera pas pour longtemps, j'en suis sûre, ajouta-t-elle, ma conscience et votre politesse, monsieur, m'en sont garants.» 

Si flatteur que fût le compliment, l'officier ne répondit rien; mais, tirant de sa ceinture un petit sifflet d'argent pareil à celui dont se servent les contremaîtres sur les bâtiments de guerre, il siffla trois fois, sur trois modulations différentes: alors plusieurs hommes parurent, dételèrent les chevaux fumants et emmenèrent la voiture sous une remise. 

Puis l'officier, toujours avec la même politesse calme, invita sa prisonnière à entrer dans la maison. Celle-ci, toujours avec son même visage souriant, lui prit le bras, et entra avec lui sous une porte basse et cintrée qui, par une voûte éclairée seulement au fond, conduisait à un escalier de pierre tournant autour d'une arête de pierre; puis on s'arrêta devant une porte massive qui, après l'introduction dans la serrure d'une clef que le jeune homme portait sur lui, roula lourdement sur ses gonds et donna ouverture à la chambre destinée à Milady. 

D'un seul regard, la prisonnière embrassa l'appartement dans ses moindres détails. 

C'était une chambre dont l'ameublement était à la fois bien propre pour une prison et bien sévère pour une habitation d'homme libre; cependant, des barreaux aux fenêtres et des verrous extérieurs à la porte décidaient le procès en faveur de la prison. 

Un instant toute la force d'âme de cette créature, trempée cependant aux sources les plus vigoureuses, l'abandonna; elle tomba sur un fauteuil, croisant les bras, baissant la tête, et s'attendant à chaque instant à voir entrer un juge pour l'interroger. 

Mais personne n'entra, que deux ou trois soldats de marine qui apportèrent les malles et les caisses, les déposèrent dans un coin et se retirèrent sans rien dire. 

L'officier présidait à tous ces détails avec le même calme que Milady lui avait constamment vu, ne prononçant pas une parole lui-même, et se faisant obéir d'un geste de sa main ou d'un coup de son sifflet. 

On eût dit qu'entre cet homme et ses inférieurs la langue parlée n'existait pas ou devenait inutile. 

Enfin Milady n'y put tenir plus longtemps, elle rompit le silence: 

«Au nom du Ciel, monsieur! s'écria-t-elle, que veut dire tout ce qui se passe? Fixez mes irrésolutions; j'ai du courage pour tout danger que je prévois, pour tout malheur que je comprends. Où suis-je et que suis-je ici? suis-je libre, pourquoi ces barreaux et ces portes? suis-je prisonnière, quel crime ai-je commis? 

\speak  Vous êtes ici dans l'appartement qui vous est destiné, madame. J'ai reçu l'ordre d'aller vous prendre en mer et de vous conduire en ce château: cet ordre, je l'ai accompli, je crois, avec toute la rigidité d'un soldat, mais aussi avec toute la courtoisie d'un gentilhomme. Là se termine, du moins jusqu'à présent, la charge que j'avais à remplir près de vous, le reste regarde une autre personne. 

\speak  Et cette autre personne, quelle est-elle? demanda Milady; ne pouvez-vous me dire son nom?\dots» 

En ce moment on entendit par les escaliers un grand bruit d'éperons; quelques voix passèrent et s'éteignirent, et le bruit d'un pas isolé se rapprocha de la porte. 

«Cette personne, la voici, madame», dit l'officier en démasquant le passage, et en se rangeant dans l'attitude du respect et de la soumission. 

En même temps, la porte s'ouvrit; un homme parut sur le seuil. 

Il était sans chapeau, portait l'épée au côté, et froissait un mouchoir entre ses doigts. 

Milady crut reconnaître cette ombre dans l'ombre, elle s'appuya d'une main sur le bras de son fauteuil, et avança la tête comme pour aller au-devant d'une certitude. 

Alors l'étranger s'avança lentement; et, à mesure qu'il s'avançait en entrant dans le cercle de lumière projeté par la lampe, Milady se reculait involontairement. 

Puis, lorsqu'elle n'eut plus aucun doute: 

«Eh quoi! mon frère! s'écria-t-elle au comble de la stupeur, c'est vous? 

\speak  Oui, belle dame! répondit Lord de Winter en faisant un salut moitié courtois, moitié ironique, moi-même. 

\speak  Mais alors, ce château? 

\speak  Est à moi. 

\speak  Cette chambre? 

\speak  C'est la vôtre. 

\speak  Je suis donc votre prisonnière? 

\speak  À peu près. 

\speak  Mais c'est un affreux abus de la force! 

\speak  Pas de grands mots; asseyons-nous, et causons tranquillement, comme il convient de faire entre un frère et une soeur.» 

Puis, se retournant vers la porte, et voyant que le jeune officier attendait ses derniers ordres: 

«C'est bien, dit-il, je vous remercie; maintenant, laissez-nous, monsieur Felton.» 
\include{chapters/50.tex}
\include{chapters/51.tex}
%!TeX root=../musketeersfr.tex 

\chapter{Premiere Journée De Captivité}

\lettrine{R}{evenons} à Milady, qu'un regard jeté sur les côtes de France nous a fait perdre de vue un instant. 
	
\zz
Nous la retrouverons dans la position désespérée où nous l'avons laissée, se creusant un abîme de sombres réflexions, sombre enfer à la porte duquel elle a presque laissé l'espérance: car pour la première fois elle doute, pour la première fois elle craint. 

Dans deux occasions sa fortune lui a manqué, dans deux occasions elle s'est vue découverte et trahie, et dans ces deux occasions, c'est contre le génie fatal envoyé sans doute par le Seigneur pour la combattre qu'elle a échoué: d'Artagnan l'a vaincue, elle, cette invincible puissance du mal. 

Il l'a abusée dans son amour, humiliée dans son orgueil, trompée dans son ambition, et maintenant voilà qu'il la perd dans sa fortune, qu'il l'atteint dans sa liberté, qu'il la menace même dans sa vie. Bien plus, il a levé un coin de son masque, cette égide dont elle se couvre et qui la rend si forte. 

D'Artagnan a détourné de Buckingham, qu'elle hait, comme elle hait tout ce qu'elle a aimé, la tempête dont le menaçait Richelieu dans la personne de la reine. D'Artagnan s'est fait passer pour de Wardes, pour lequel elle avait une de ces fantaisies de tigresse, indomptables comme en ont les femmes de ce caractère. D'Artagnan connaît ce terrible secret qu'elle a juré que nul ne connaîtrait sans mourir. Enfin, au moment où elle vient d'obtenir un blanc-seing à l'aide duquel elle va se venger de son ennemi, le blanc-seing lui est arraché des mains, et c'est d'Artagnan qui la tient prisonnière et qui va l'envoyer dans quelque immonde Botany-Bay, dans quelque Tyburn infâme de l'océan Indien. 

Car tout cela lui vient de d'Artagnan sans doute; de qui viendraient tant de hontes amassées sur sa tête, sinon de lui? Lui seul a pu transmettre à Lord de Winter tous ces affreux secrets, qu'il a découverts les uns après les autres par une sorte de fatalité. Il connaît son beau-frère, il lui aura écrit. 

Que de haine elle distille! Là, immobile, et les yeux ardents et fixes dans son appartement désert, comme les éclats de ses rugissements sourds, qui parfois s'échappent avec sa respiration du fond de sa poitrine, accompagnent bien le bruit de la houle qui monte, gronde, mugit et vient se briser, comme un désespoir éternel et impuissant, contre les rochers sur lesquels est bâti ce château sombre et orgueilleux! Comme, à la lueur des éclairs que sa colère orageuse fait briller dans son esprit, elle conçoit contre Mme Bonacieux, contre Buckingham, et surtout contre d'Artagnan, de magnifiques projets de vengeance, perdus dans les lointains de l'avenir! 

Oui, mais pour se venger il faut être libre, et pour être libre, quand on est prisonnier, il faut percer un mur, desceller des barreaux, trouer un plancher; toutes entreprises que peut mener à bout un homme patient et fort mais devant lesquelles doivent échouer les irritations fébriles d'une femme. D'ailleurs, pour faire tout cela il faut avoir le temps, des mois, des années, et elle\dots elle a dix ou douze jours, à ce que lui a dit Lord de Winter, son fraternel et terrible geôlier. 

Et cependant, si elle était un homme, elle tenterait tout cela, et peut-être réussirait-elle: pourquoi donc le Ciel s'est-il ainsi trompé, en mettant cette âme virile dans ce corps frêle et délicat! 

Aussi les premiers moments de la captivité ont été terribles: quelques convulsions de rage qu'elle n'a pu vaincre ont payé sa dette de faiblesse féminine à la nature. Mais peu à peu elle a surmonté les éclats de sa folle colère, les frémissements nerveux qui ont agité son corps ont disparu, et maintenant elle s'est repliée sur elle-même comme un serpent fatigué qui se repose. 

«Allons, allons; j'étais folle de m'emporter ainsi, dit-elle en plongeant dans la glace, qui reflète dans ses yeux son regard brûlant, par lequel elle semble s'interroger elle-même. Pas de violence, la violence est une preuve de faiblesse. D'abord je n'ai jamais réussi par ce moyen: peut-être, si j'usais de ma force contre des femmes, aurais-je chance de les trouver plus faibles encore que moi, et par conséquent de les vaincre; mais c'est contre des hommes que je lutte, et je ne suis qu'une femme pour eux. Luttons en femme, ma force est dans ma faiblesse.» 

Alors, comme pour se rendre compte à elle-même des changements qu'elle pouvait imposer à sa physionomie si expressive et si mobile, elle lui fit prendre à la fois toutes les expressions, depuis celle de la colère qui crispait ses traits, jusqu'à celle du plus doux, du plus affectueux et du plus séduisant sourire. Puis ses cheveux prirent successivement sous ses mains savantes les ondulations qu'elle crut pouvoir aider aux charmes de son visage. Enfin elle murmura, satisfaite d'elle-même: 

«Allons, rien n'est perdu. Je suis toujours belle.» 

Il était huit heures du soir à peu près. Milady aperçut un lit; elle pensa qu'un repos de quelques heures rafraîchirait non seulement sa tête et ses idées, mais encore son teint. Cependant, avant de se coucher, une idée meilleure lui vint. Elle avait entendu parler de souper. Déjà elle était depuis une heure dans cette chambre, on ne pouvait tarder à lui apporter son repas. La prisonnière ne voulut pas perdre de temps, et elle résolut de faire, dès cette même soirée, quelque tentative pour sonder le terrain, en étudiant le caractère des gens auxquels sa garde était confiée. 

Une lumière apparut sous la porte; cette lumière annonçait le retour de ses geôliers. Milady, qui s'était levée, se rejeta vivement sur son fauteuil, la tête renversée en arrière, ses beaux cheveux dénoués et épars, sa gorge demi-nue sous ses dentelles froissées, une main sur son cœur et l'autre pendante. 

On ouvrit les verrous, la porte grinça sur ses gonds, des pas retentirent dans la chambre et s'approchèrent. 

«Posez là cette table», dit une voix que la prisonnière reconnut pour celle de Felton. 

L'ordre fut exécuté. 

«Vous apporterez des flambeaux et ferez relever la sentinelle», continua Felton. 

Ce double ordre que donna aux mêmes individus le jeune lieutenant prouva à Milady que ses serviteurs étaient les mêmes hommes que ses gardiens, c'est-à-dire des soldats. 

Les ordres de Felton étaient, au reste, exécutés avec une silencieuse rapidité qui donnait une bonne idée de l'état florissant dans lequel il maintenait la discipline. 

Enfin, Felton, qui n'avait pas encore regardé Milady, se retourna vers elle. 

«Ah! ah! dit-il, elle dort, c'est bien: à son réveil elle soupera.» 

Et il fit quelques pas pour sortir. 

«Mais, mon lieutenant, dit un soldat moins stoïque que son chef, et qui s'était approché de Milady, cette femme ne dort pas. 

\speak  Comment, elle ne dort pas? dit Felton, que fait-elle donc, alors? 

\speak  Elle est évanouie; son visage est très pâle, et j'ai beau écouter, je n'entends pas sa respiration. 

\speak  Vous avez raison, dit Felton après avoir regardé Milady de la place où il se trouvait, sans faire un pas vers elle, allez prévenir Lord de Winter que sa prisonnière est évanouie, car je ne sais que faire, le cas n'ayant pas été prévu.» 

Le soldat sortit pour obéir aux ordres de son officier; Felton s'assit sur un fauteuil qui se trouvait par hasard près de la porte et attendit sans dire une parole, sans faire un geste. Milady possédait ce grand art, tant étudié par les femmes, de voir à travers ses longs cils sans avoir l'air d'ouvrir les paupières: elle aperçut Felton qui lui tournait le dos, elle continua de le regarder pendant dix minutes à peu près, et pendant ces dix minutes, l'impassible gardien ne se retourna pas une seule fois. 

Elle songea alors que Lord de Winter allait venir et rendre, par sa présence, une nouvelle force à son geôlier: sa première épreuve était perdue, elle en prit son parti en femme qui compte sur ses ressources; en conséquence elle leva la tête, ouvrit les yeux et soupira faiblement. 

À ce soupir, Felton se retourna enfin. 

«Ah! vous voici réveillée, madame! dit-il, je n'ai donc plus affaire ici! Si vous avez besoin de quelque chose, vous appellerez. 

\speak  Oh! mon Dieu, mon Dieu! que j'ai souffert!» murmura Milady avec cette voix harmonieuse qui, pareille à celle des enchanteresses antiques, charmait tous ceux qu'elle voulait perdre. 

Et elle prit en se redressant sur son fauteuil une position plus gracieuse et plus abandonnée encore que celle qu'elle avait lorsqu'elle était couchée. 

Felton se leva. 

«Vous serez servie ainsi trois fois par jour, madame, dit-il: le matin à neuf heures, dans la journée à une heure, et le soir à huit heures. Si cela ne vous convient pas, vous pouvez indiquer vos heures au lieu de celles que je vous propose, et, sur ce point, on se conformera à vos désirs. 

\speak  Mais vais-je donc rester toujours seule dans cette grande et triste chambre? demanda Milady. 

\speak  Une femme des environs a été prévenue, elle sera demain au château, et viendra toutes les fois que vous désirerez sa présence. 

\speak  Je vous rends grâce, monsieur», répondit humblement la prisonnière. 

Felton fit un léger salut et se dirigea vers la porte. Au moment où il allait en franchir le seuil, Lord de Winter parut dans le corridor, suivi du soldat qui était allé lui porter la nouvelle de l'évanouissement de Milady. Il tenait à la main un flacon de sels. «Eh bien! qu'est-ce? et que se passe-t-il donc ici? dit-il d'une voix railleuse en voyant sa prisonnière debout et Felton prêt à sortir. Cette morte est-elle donc déjà ressuscitée? Pardieu, Felton, mon enfant, tu n'as donc pas vu qu'on te prenait pour un novice et qu'on te jouait le premier acte d'une comédie dont nous aurons sans doute le plaisir de suivre tous les développements? 

\speak  Je l'ai bien pensé, Milord, dit Felton; mais, enfin, comme la prisonnière est femme, après tout, j'ai voulu avoir les égards que tout homme bien né doit à une femme, sinon pour elle, du moins pour lui-même.» 

Milady frissonna par tout son corps. Ces paroles de Felton passaient comme une glace par toutes ses veines. 

«Ainsi, reprit de Winter en riant, ces beaux cheveux savamment étalés, cette peau blanche et ce langoureux regard ne t'ont pas encore séduit, cœur de pierre? 

\speak  Non, Milord, répondit l'impassible jeune homme, et croyez-moi bien, il faut plus que des manèges et des coquetteries de femme pour me corrompre. 

\speak  En ce cas, mon brave lieutenant, laissons Milady chercher autre chose et allons souper; ah! sois tranquille, elle a l'imagination féconde et le second acte de la comédie ne tardera pas à suivre le premier.» 

Et à ces mots Lord de Winter passa son bras sous celui de Felton et l'emmena en riant. 

«Oh! je trouverai bien ce qu'il te faut, murmura Milady entre ses dents; sois tranquille, pauvre moine manqué, pauvre soldat converti qui t'es taillé ton uniforme dans un froc.» 

«À propos, reprit de Winter en s'arrêtant sur le seuil de la porte, il ne faut pas, Milady, que cet échec vous ôte l'appétit. Tâtez de ce poulet et de ces poissons que je n'ai pas fait empoisonner, sur l'honneur. Je m'accommode assez de mon cuisinier, et comme il ne doit pas hériter de moi, j'ai en lui pleine et entière confiance. Faites comme moi. Adieu, chère soeur! à votre prochain évanouissement.» 

C'était tout ce que pouvait supporter Milady: ses mains se crispèrent sur son fauteuil, ses dents grincèrent sourdement, ses yeux suivirent le mouvement de la porte qui se fermait derrière Lord de Winter et Felton; et, lorsqu'elle se vit seule, une nouvelle crise de désespoir la prit; elle jeta les yeux sur la table, vit briller un couteau, s'élança et le saisit; mais son désappointement fut cruel: la lame en était ronde et d'argent flexible. 

Un éclat de rire retentit derrière la porte mal fermée, et la porte se rouvrit. 

«Ah! ah! s'écria Lord de Winter; ah! ah! vois-tu bien, mon brave Felton, vois-tu ce que je t'avais dit: ce couteau, c'était pour toi; mon enfant, elle t'aurait tué; vois-tu, c'est un de ses travers, de se débarrasser ainsi, d'une façon ou de l'autre, des gens qui la gênent. Si je t'eusse écouté, le couteau eût été pointu et d'acier: alors plus de Felton, elle t'aurait égorgé et, après toi, tout le monde. Vois donc, John, comme elle sait bien tenir son couteau.» 

En effet, Milady tenait encore l'arme offensive dans sa main crispée, mais ces derniers mots, cette suprême insulte, détendirent ses mains, ses forces et jusqu'à sa volonté. 

Le couteau tomba par terre. 

«Vous avez raison, Milord, dit Felton avec un accent de profond dégoût qui retentit jusqu'au fond du cœur de Milady, vous avez raison et c'est moi qui avais tort.» 

Et tous deux sortirent de nouveau. 

Mais cette fois, Milady prêta une oreille plus attentive que la première fois, et elle entendit leurs pas s'éloigner et s'éteindre dans le fond du corridor. 

«Je suis perdue, murmura-t-elle, me voilà au pouvoir de gens sur lesquels je n'aurai pas plus de prise que sur des statues de bronze ou de granit; ils me savent par cœur et sont cuirassés contre toutes mes armes. 

«Il est cependant impossible que cela finisse comme ils l'ont décidé.» 

En effet, comme l'indiquait cette dernière réflexion, ce retour instinctif à l'espérance, dans cette âme profonde la crainte et les sentiments faibles ne surnageaient pas longtemps. Milady se mit à table, mangea de plusieurs mets, but un peu de vin d'Espagne, et sentit revenir toute sa résolution. 

Avant de se coucher elle avait déjà commenté, analysé, retourné sur toutes leurs faces, examiné sous tous les points, les paroles, les pas, les gestes, les signes et jusqu'au silence de ses geôliers, et de cette étude profonde, habile et savante, il était résulté que Felton était, à tout prendre, le plus vulnérable de ses deux persécuteurs. 

Un mot surtout revenait à l'esprit de la prisonnière: 

«Si je t'eusse écouté», avait dit Lord de Winter à Felton. 

Donc Felton avait parlé en sa faveur, puisque Lord de Winter n'avait pas voulu écouter Felton. 

«Faible ou forte, répétait Milady, cet homme a donc une lueur de pitié dans son âme; de cette lueur je ferai un incendie qui le dévorera. 

«Quant à l'autre, il me connaît, il me craint et sait ce qu'il a à attendre de moi si jamais je m'échappe de ses mains, il est donc inutile de rien tenter sur lui. Mais Felton, c'est autre chose; c'est un jeune homme naïf, pur et qui semble vertueux; celui-là, il y a moyen de le perdre.» 

Et Milady se coucha et s'endormit le sourire sur les lèvres; quelqu'un qui l'eût vue dormant eût dit une jeune fille rêvant à la couronne de fleurs qu'elle devait mettre sur son front à la prochaine fête. 
%!TeX root=../musketeersfr.tex 

\chapter{Deuxième Journée De Captivité}

\lettrine{M}{ilady} rêvait qu'elle tenait enfin d'Artagnan, qu'elle assistait à son supplice, et c'était la vue de son sang odieux, coulant sous la hache du bourreau, qui dessinait ce charmant sourire sur les lèvres. 

Elle dormait comme dort un prisonnier bercé par sa première espérance. 

Le lendemain, lorsqu'on entra dans sa chambre, elle était encore au lit. Felton était dans le corridor: il amenait la femme dont il avait parlé la veille, et qui venait d'arriver; cette femme entra et s'approcha du lit de Milady en lui offrant ses services. 

Milady était habituellement pâle; son teint pouvait donc tromper une personne qui la voyait pour la première fois. 

«J'ai la fièvre, dit-elle; je n'ai pas dormi un seul instant pendant toute cette longue nuit, je souffre horriblement: serez-vous plus humaine qu'on ne l'a été hier avec moi? Tout ce que je demande, au reste, c'est la permission de rester couchée. 

\speak  Voulez-vous qu'on appelle un médecin?» dit la femme. 

Felton écoutait ce dialogue sans dire une parole. 

Milady réfléchissait que plus on l'entourerait de monde, plus elle aurait de monde à apitoyer, et plus la surveillance de Lord de Winter redoublerait; d'ailleurs le médecin pourrait déclarer que la maladie était feinte, et Milady après avoir perdu la première partie ne voulait pas perdre la seconde. 

«Aller chercher un médecin, dit-elle, à quoi bon? ces messieurs ont déclaré hier que mon mal était une comédie, il en serait sans doute de même aujourd'hui; car depuis hier soir, on a eu le temps de prévenir le docteur. 

\speak  Alors, dit Felton impatienté, dites vous-même, madame, quel traitement vous voulez suivre. 

\speak  Eh! le sais-je, moi? mon Dieu! je sens que je souffre, voilà tout, que l'on me donne ce que l'on voudra, peu m'importe. 

\speak  Allez chercher Lord de Winter, dit Felton fatigué de ces plaintes éternelles. 

\speak  Oh! non, non! s'écria Milady, non, monsieur, ne l'appelez pas, je vous en conjure, je suis bien, je n'ai besoin de rien, ne l'appelez pas.» 

Elle mit une véhémence si prodigieuse, une éloquence si entraînante dans cette exclamation, que Felton, entraîné, fit quelques pas dans la chambre. 

«Il est ému», pensa Milady. 

«Cependant, madame, dit Felton, si vous souffrez \textit{réellement}, on enverra chercher un médecin, et si vous nous trompez, eh bien, ce sera tant pis pour vous, mais du moins, de notre côté, nous n'aurons rien à nous reprocher.» 

Milady ne répondit point; mais renversant sa belle tête sur son oreiller, elle fondit en larmes et éclata en sanglots. 

Felton la regarda un instant avec son impassibilité ordinaire; puis voyant que la crise menaçait de se prolonger, il sortit; la femme le suivit. Lord de Winter ne parut pas. 

«Je crois que je commence à voir clair», murmura Milady avec une joie sauvage, en s'ensevelissant sous les draps pour cacher à tous ceux qui pourraient l'épier cet élan de satisfaction intérieure. 

Deux heures s'écoulèrent. 

«Maintenant il est temps que la maladie cesse, dit-elle: levons-nous et obtenons quelque succès dès aujourd'hui; je n'ai que dix jours, et ce soir il y en aura deux d'écoulés. 

En entrant, le matin, dans la chambre de Milady, on lui avait apporté son déjeuner; or elle avait pensé qu'on ne tarderait pas à venir enlever la table, et qu'en ce moment elle reverrait Felton. 

Milady ne se trompait pas. Felton reparut, et, sans faire attention si Milady avait ou non touché au repas, fit un signe pour qu'on emportât hors de la chambre la table, que l'on apportait ordinairement toute servie. 

Felton resta le dernier, il tenait un livre à la main. 

Milady, couchée dans un fauteuil près de la cheminée, belle, pâle et résignée, ressemblait à une vierge sainte attendant le martyre. 

Felton s'approcha d'elle et dit: 

«Lord de Winter, qui est catholique comme vous, madame, a pensé que la privation des rites et des cérémonies de votre religion peut vous être pénible: il consent donc à ce que vous lisiez chaque jour l'ordinaire de \textit{votre messe}, et voici un livre qui en contient le rituel.» 

À l'air dont Felton déposa ce livre sur la petite table près de laquelle était Milady, au ton dont il prononça ces deux mots, \textit{votre messe}, au sourire dédaigneux dont il les accompagna, Milady leva la tête et regarda plus attentivement l'officier. 

Alors, à cette coiffure sévère, à ce costume d'une simplicité exagérée, à ce front poli comme le marbre, mais dur et impénétrable comme lui, elle reconnut un de ces sombres puritains qu'elle avait rencontrés si souvent tant à la cour du roi Jacques qu'à celle du roi de France, où, malgré le souvenir de la Saint-Barthélémy, ils venaient parfois chercher un refuge. 

Elle eut donc une de ces inspirations subites comme les gens de génie seuls en reçoivent dans les grandes crises, dans les moments suprêmes qui doivent décider de leur fortune ou de leur vie. 

Ces deux mots, \textit{votre messe}, et un simple coup d'œil jeté sur Felton, lui avaient en effet révélé toute l'importance de la réponse qu'elle allait faire. 

Mais avec cette rapidité d'intelligence qui lui était particulière, cette réponse toute formulée se présenta sur ses lèvres: 

«Moi! dit-elle avec un accent de dédain monté à l'unisson de celui qu'elle avait remarqué dans la voix du jeune officier, moi, monsieur, \textit{ma messe!} Lord de Winter, le catholique corrompu, sait bien que je ne suis pas de sa religion, et c'est un piège qu'il veut me tendre! 

\speak  Et de quelle religion êtes-vous donc, madame? demanda Felton avec un étonnement que, malgré son empire sur lui-même, il ne put cacher entièrement. 

\speak  Je le dirai, s'écria Milady avec une exaltation feinte, le jour où j'aurai assez souffert pour ma foi.» 

Le regard de Felton découvrit à Milady toute l'étendue de l'espace qu'elle venait de s'ouvrir par cette seule parole. 

Cependant le jeune officier demeura muet et immobile, son regard seul avait parlé. 

«Je suis aux mains de mes ennemis, continua-t-elle avec ce ton d'enthousiasme qu'elle savait familier aux puritains; eh bien, que mon Dieu me sauve ou que je périsse pour mon Dieu! voilà la réponse que je vous prie de faire à Lord de Winter. Et quant à ce livre, ajouta-t-elle en montrant le rituel du bout du doigt, mais sans le toucher, comme si elle eût dû être souillée par cet attouchement, vous pouvez le remporter et vous en servir pour vous-même, car sans doute vous êtes doublement complice de Lord de Winter, complice dans sa persécution, complice dans son hérésie.» 

Felton ne répondit rien, prit le livre avec le même sentiment de répugnance qu'il avait déjà manifesté et se retira pensif. Lord de Winter vint vers les cinq heures du soir; Milady avait eu le temps pendant toute la journée de se tracer son plan de conduite; elle le reçut en femme qui a déjà repris tous ses avantages. 

«Il paraît, dit le baron en s'asseyant dans un fauteuil en face de celui qu'occupait Milady et en étendant nonchalamment ses pieds sur le foyer, il paraît que nous avons fait une petite apostasie! 

\speak  Que voulez-vous dire, monsieur? 

\speak  Je veux dire que depuis la dernière fois que nous nous sommes vus, nous avons changé de religion; auriez-vous épousé un troisième mari protestant, par hasard? 

\speak  Expliquez-vous, Milord, reprit la prisonnière avec majesté, car je vous déclare que j'entends vos paroles, mais que je ne les comprends pas. 

\speak  Alors, c'est que vous n'avez pas de religion du tout; j'aime mieux cela, reprit en ricanant Lord de Winter. 

\speak  Il est certain que cela est plus selon vos principes, reprit froidement Milady. 

\speak  Oh! je vous avoue que cela m'est parfaitement égal. 

\speak  Oh! vous n'avoueriez pas cette indifférence religieuse, Milord, que vos débauches et vos crimes en feraient foi. 

\speak  Hein! vous parlez de débauches, madame Messaline, vous parlez de crimes, Lady Macbeth! Ou j'ai mal entendu, ou vous êtes, pardieu, bien impudente. 

\speak  Vous parlez ainsi parce que vous savez qu'on nous écoute, monsieur, répondit froidement Milady, et que vous voulez intéresser vos geôliers et vos bourreaux contre moi. 

\speak  Mes geôliers! mes bourreaux! Ouais, madame, vous le prenez sur un ton poétique, et la comédie d'hier tourne ce soir à la tragédie. Au reste, dans huit jours vous serez où vous devez être et ma tâche sera achevée. 

\speak  Tâche infâme! tâche impie! reprit Milady avec l'exaltation de la victime qui provoque son juge. 

\speak  Je crois, ma parole d'honneur, dit de Winter en se levant, que la drôlesse devient folle. Allons, allons, calmez-vous, madame la puritaine, ou je vous fais mettre au cachot. Pardieu! c'est mon vin d'Espagne qui vous monte à la tête, n'est-ce pas? mais, soyez tranquille, cette ivresse-là n'est pas dangereuse et n'aura pas de suites.» 

Et Lord de Winter se retira en jurant, ce qui à cette époque était une habitude toute cavalière. 

Felton était en effet derrière la porte et n'avait pas perdu un mot de toute cette scène. 

Milady avait deviné juste. 

«Oui, va! va! dit-elle à son frère, les suites approchent, au contraire, mais tu ne les verras, imbécile, que lorsqu'il ne sera plus temps de les éviter.» 

Le silence se rétablit, deux heures s'écoulèrent; on apporta le souper, et l'on trouva Milady occupée à faire tout haut ses prières, prières qu'elle avait apprises d'un vieux serviteur de son second mari, puritain des plus austères. Elle semblait en extase et ne parut pas même faire attention à ce qui se passait autour d'elle. Felton fit signe qu'on ne la dérangeât point, et lorsque tout fut en état il sortit sans bruit avec les soldats. 

Milady savait qu'elle pouvait être épiée, elle continua donc ses prières jusqu'à la fin, et il lui sembla que le soldat qui était de sentinelle à sa porte ne marchait plus du même pas et paraissait écouter. 

Pour le moment, elle n'en voulait pas davantage, elle se releva, se mit à table, mangea peu et ne but que de l'eau. 

Une heure après on vint enlever la table, mais Milady remarqua que cette fois Felton n'accompagnait point les soldats. 

Il craignait donc de la voir trop souvent. 

Elle se retourna vers le mur pour sourire, car il y avait dans ce sourire une telle expression de triomphe que ce seul sourire l'eût dénoncée. 

Elle laissa encore s'écouler une demi-heure, et comme en ce moment tout faisait silence dans le vieux château, comme on n'entendait que l'éternel murmure de la houle, cette respiration immense de l'océan, de sa voix pure, harmonieuse et vibrante, elle commença le premier couplet de ce psaume alors en entière faveur près des puritains: 

\begin{verse}
Seigneur, si tu nous abandonnes,\\
C'est pour voir si nous sommes forts;\\
Mais ensuite c'est toi qui donnes\\
De ta céleste main la palme à nos efforts. 
\end{verse}

Ces vers n'étaient pas excellents, il s'en fallait même de beaucoup; mais, comme on le sait, les protestants ne se piquaient pas de poésie. 

Tout en chantant, Milady écoutait: le soldat de garde à sa porte s'était arrêté comme s'il eût été changé en pierre. Milady put donc juger de l'effet qu'elle avait produit. 

Alors elle continua son chant avec une ferveur et un sentiment inexprimables; il lui sembla que les sons se répandaient au loin sous les voûtes et allaient comme un charme magique adoucir le cœur de ses geôliers. Cependant il paraît que le soldat en sentinelle, zélé catholique sans doute, secoua le charme, car à travers la porte: 

«Taisez-vous donc madame, dit-il, votre chanson est triste comme un \textit{De profondis}, et si, outre l'agrément d'être en garnison ici, il faut encore y entendre de pareilles choses, ce sera à n'y point tenir. 

\speak  Silence! dit alors une voix grave, que Milady reconnut pour celle de Felton; de quoi vous mêlez-vous, drôle? Vous a-t-on ordonné d'empêcher cette femme de chanter? Non. On vous a dit de la garder, de tirer sur elle si elle essayait de fuir. Gardez-la; si elle fuit, tuez-la, mais ne changez rien à la consigne.» 

Une expression de joie indicible illumina le visage de Milady, mais cette expression fut fugitive comme le reflet d'un éclair, et, sans paraître avoir entendu le dialogue dont elle n'avait pas perdu un mot, elle reprit en donnant à sa voix tout le charme, toute l'étendue et toute la séduction que le démon y avait mis:
\begin{verse}
Pour tant de pleurs et de misère,\\
Pour mon exil et pour mes fers,\\
J'ai ma jeunesse, ma prière,\\
Et Dieu, qui comptera les maux que j'ai soufferts. 
\end{verse}

Cette voix, d'une étendue inouïe et d'une passion sublime, donnait à la poésie rude et inculte de ces psaumes une magie et une expression que les puritains les plus exaltés trouvaient rarement dans les chants de leurs frères et qu'ils étaient forcés d'orner de toutes les ressources de leur imagination: Felton crut entendre chanter l'ange qui consolait les trois Hébreux dans la fournaise. 

Milady continua: 

\begin{verse}
	Mais le jour de la délivrance\\
	Viendra pour nous, Dieu juste et fort;\\
	Et s'il trompe notre espérance,\\
	Il nous reste toujours le martyre et la mort. 
\end{verse}

Ce couplet, dans lequel la terrible enchanteresse s'efforça de mettre toute son âme, acheva de porter le désordre dans le cœur du jeune officier: il ouvrit brusquement la porte, et Milady le vit apparaître pâle comme toujours, mais les yeux ardents et presque égarés. 

«Pourquoi chantez-vous ainsi, dit-il, et avec une pareille voix? 

\speak  Pardon, monsieur, dit Milady avec douceur, j'oubliais que mes chants ne sont pas de mise dans cette maison. Je vous ai sans doute offensé dans vos croyances; mais c'était sans le vouloir, je vous jure; pardonnez-moi donc une faute qui est peut-être grande, mais qui certainement est involontaire.» 

Milady était si belle dans ce moment, l'extase religieuse dans laquelle elle semblait plongée donnait une telle expression à sa physionomie, que Felton, ébloui, crut voir l'ange que tout à l'heure il croyait seulement entendre. 

«Oui, oui, répondit-il, oui: vous troublez, vous agitez les gens qui habitent ce château.» 

Et le pauvre insensé ne s'apercevait pas lui-même de l'incohérence de ses discours, tandis que Milady plongeait son œil de lynx au plus profond de son cœur. 

«Je me tairai, dit Milady en baissant les yeux avec toute la douceur qu'elle put donner à sa voix, avec toute la résignation qu'elle put imprimer à son maintien. 

\speak  Non, non, madame, dit Felton; seulement, chantez moins haut, la nuit surtout.» 

Et à ces mots, Felton, sentant qu'il ne pourrait pas conserver longtemps sa sévérité à l'égard de la prisonnière, s'élança hors de son appartement. 

«Vous avez bien fait, lieutenant, dit le soldat; ces chants bouleversent l'âme; cependant on finit par s'y accoutumer: sa voix est si belle!» 
%!TeX root=../musketeersfr.tex 

\chapter{Troisième Journée De Captivité}

\lettrine{F}{elton} était venu; mais il y avait encore un pas à faire: il fallait le retenir, ou plutôt il fallait qu'il restât tout seul; et Milady ne voyait encore qu'obscurément le moyen qui devait la conduire à ce résultat. 

Il fallait plus encore: il fallait le faire parler, afin de lui parler aussi: car, Milady le savait bien, sa plus grande séduction était dans sa voix, qui parcourait si habilement toute la gamme des tons, depuis la parole humaine jusqu'au langage céleste. 

Et cependant, malgré toute cette séduction, Milady pouvait échouer, car Felton était prévenu, et cela contre le moindre hasard. Dès lors, elle surveilla toutes ses actions, toutes ses paroles, jusqu'au plus simple regard de ses yeux, jusqu'à son geste, jusqu'à sa respiration, qu'on pouvait interpréter comme un soupir. Enfin, elle étudia tout comme fait un habile comédien à qui l'on vient de donner un rôle nouveau dans un emploi qu'il n'a pas l'habitude de tenir. 

Vis-à-vis de Lord de Winter sa conduite était plus facile; aussi avait-elle été arrêtée dès la veille. Rester muette et digne en sa présence, de temps en temps l'irriter par un dédain affecté, par un mot méprisant, le pousser à des menaces et à des violences qui faisaient un contraste avec sa résignation à elle, tel était son projet. Felton verrait: peut-être ne dirait-il rien; mais il verrait. 

Le matin, Felton vint comme d'habitude; mais Milady le laissa présider à tous les apprêts du déjeuner sans lui adresser la parole. Aussi, au moment où il allait se retirer, eut-elle une lueur d'espoir; car elle crut que c'était lui qui allait parler; mais ses lèvres remuèrent sans qu'aucun son sortît de sa bouche, et, faisant un effort sur lui-même, il renferma dans son cœur les paroles qui allaient s'échapper de ses lèvres, et sortit. 

Vers midi, Lord de Winter entra. 

Il faisait une assez belle journée d'hiver, et un rayon de ce pâle soleil d'Angleterre qui éclaire, mais qui n'échauffe pas, passait à travers les barreaux de la prison. 

Milady regardait par la fenêtre, et fit semblant de ne pas entendre la porte qui s'ouvrait. 

«Ah! ah! dit Lord de Winter, après avoir fait de la comédie, après avoir fait de la tragédie, voilà que nous faisons de la mélancolie.» 

La prisonnière ne répondit pas. 

«Oui, oui, continua Lord de Winter, je comprends; vous voudriez bien être en liberté sur ce rivage; vous voudriez bien, sur un bon navire, fendre les flots de cette mer verte comme de l'émeraude; vous voudriez bien, soit sur terre, soit sur l'océan, me dresser une de ces bonnes petites embuscades comme vous savez si bien les combiner. Patience! patience! Dans quatre jours, le rivage vous sera permis, la mer vous sera ouverte, plus ouverte que vous ne le voudrez, car dans quatre jours l'Angleterre sera débarrassée de vous.» 

Milady joignit les mains, et levant ses beaux yeux vers le ciel: 

«Seigneur! Seigneur! dit-elle avec une angélique suavité de geste et d'intonation, pardonnez à cet homme, comme je lui pardonne moi-même. 

\speak  Oui, prie, maudite, s'écria le baron, ta prière est d'autant plus généreuse que tu es, je te le jure, au pouvoir d'un homme qui ne pardonnera pas.» 

Et il sortit. 

Au moment où il sortait, un regard perçant glissa par la porte entrebâillée, et elle aperçut Felton qui se rangeait rapidement pour n'être pas vu d'elle. 

Alors elle se jeta à genoux et se mit à prier. 

«Mon Dieu! mon Dieu! dit-elle, vous savez pour quelle sainte cause je souffre, donnez-moi donc la force de souffrir.» 

La porte s'ouvrit doucement; la belle suppliante fit semblant de n'avoir pas entendu, et d'une voix pleine de larmes, elle continua: 

«Dieu vengeur! Dieu de bonté! laisserez-vous s'accomplir les affreux projets de cet homme!» 

Alors, seulement, elle feignit d'entendre le bruit des pas de Felton et, se relevant rapide comme la pensée, elle rougit comme si elle eût été honteuse d'avoir été surprise à genoux. 

«Je n'aime point à déranger ceux qui prient, madame, dit gravement Felton; ne vous dérangez donc pas pour moi, je vous en conjure. 

\speak  Comment savez-vous que je priais, monsieur? dit Milady d'une voix suffoquée par les sanglots; vous vous trompiez, monsieur, je ne priais pas. 

\speak  Pensez-vous donc, madame, répondit Felton de sa même voix grave, quoique avec un accent plus doux, que je me croie le droit d'empêcher une créature de se prosterner devant son Créateur? À Dieu ne plaise! D'ailleurs le repentir sied bien aux coupables; quelque crime qu'il ait commis, un coupable m'est sacré aux pieds de Dieu. 

\speak  Coupable, moi! dit Milady avec un sourire qui eût désarmé l'ange du jugement dernier. Coupable! mon Dieu, tu sais si je le suis! Dites que je suis condamnée, monsieur, à la bonne heure; mais vous le savez, Dieu qui aime les martyrs, permet que l'on condamne quelquefois les innocents. 

\speak  Fussiez-vous condamnée, fussiez-vous martyre, répondit Felton, raison de plus pour prier, et moi-même je vous aiderai de mes prières. 

\speak  Oh! vous êtes un juste, vous, s'écria Milady en se précipitant à ses pieds; tenez, je n'y puis tenir plus longtemps, car je crains de manquer de force au moment où il me faudra soutenir la lutte et confesser ma foi, écoutez donc la supplication d'une femme au désespoir. On vous abuse, monsieur, mais il n'est pas question de cela, je ne vous demande qu'une grâce, et, si vous me l'accordez, je vous bénirai dans ce monde et dans l'autre. 

\speak  Parlez au maître, madame, dit Felton; je ne suis heureusement chargé, moi, ni de pardonner ni de punir, et c'est à plus haut que moi que Dieu a remis cette responsabilité. 

\speak  À vous, non, à vous seul. Écoutez-moi, plutôt que de contribuer à ma perte, plutôt que de contribuer à mon ignominie. 

\speak  Si vous avez mérité cette honte, madame, si vous avez encouru cette ignominie, il faut la subir en l'offrant à Dieu. 

\speak  Que dites-vous? Oh! vous ne me comprenez pas! Quand je parle d'ignominie, vous croyez que je parle d'un châtiment quelconque, de la prison ou de la mort! Plût au Ciel! que m'importent, à moi, la mort ou la prison! 

\speak  C'est moi qui ne vous comprends plus, madame. 

\speak  Ou qui faites semblant de ne plus me comprendre, monsieur, répondit la prisonnière avec un sourire de doute. 

\speak  Non, madame, sur l'honneur d'un soldat, sur la foi d'un chrétien! 

\speak  Comment! vous ignorez les desseins de Lord de Winter sur moi. 

\speak  Je les ignore. 

\speak  Impossible, vous son confident! 

\speak  Je ne mens jamais, madame. 

\speak  Oh! il se cache trop peu cependant pour qu'on ne les devine pas. 

\speak  Je ne cherche à rien deviner, madame; j'attends qu'on me confie, et à part ce qu'il m'a dit devant vous, Lord de Winter ne m'a rien confié. 

\speak  Mais, s'écria Milady avec un incroyable accent de vérité, vous n'êtes donc pas son complice, vous ne savez donc pas qu'il me destine à une honte que tous les châtiments de la terre ne sauraient égaler en horreur? 

\speak  Vous vous trompez, madame, dit Felton en rougissant, Lord de Winter n'est pas capable d'un tel crime.» 

«Bon, dit Milady en elle-même, sans savoir ce que c'est, il appelle cela un crime!» 

Puis tout haut: 

«L'ami de l'infâme est capable de tout. 

\speak  Qui appelez-vous l'infâme? demanda Felton. 

\speak  Y a-t-il donc en Angleterre deux hommes à qui un semblable nom puisse convenir? 

\speak  Vous voulez parler de Georges Villiers? dit Felton, dont les regards s'enflammèrent. 

\speak  Que les païens, les gentils et les infidèles appellent duc de Buckingham, reprit Milady; je n'aurais pas cru qu'il y aurait eu un Anglais dans toute l'Angleterre qui eût eu besoin d'une si longue explication pour reconnaître celui dont je voulais parler! 

\speak  La main du Seigneur est étendue sur lui, dit Felton, il n'échappera pas au châtiment qu'il mérite.» 

Felton ne faisait qu'exprimer à l'égard du duc le sentiment d'exécration que tous les Anglais avaient voué à celui que les catholiques eux-mêmes appelaient l'exacteur, le concussionnaire, le débauché, et que les puritains appelaient tout simplement Satan. 

«Oh! mon Dieu! mon Dieu! s'écria Milady, quand je vous supplie d'envoyer à cet homme le châtiment qui lui est dû, vous savez que ce n'est pas ma propre vengeance que je poursuis, mais la délivrance de tout un peuple que j'implore. 

\speak  Le connaissez-vous donc?» demanda Felton. 

«Enfin, il m'interroge», se dit en elle-même Milady au comble de la joie d'en être arrivée si vite à un si grand résultat. 

«Oh! si je le connais! oh, oui! pour mon malheur, pour mon malheur éternel.» 

Et Milady se tordit les bras comme arrivée au paroxysme de la douleur. Felton sentit sans doute en lui-même que sa force l'abandonnait, et il fit quelques pas vers la porte; la prisonnière, qui ne le perdait pas de vue, bondit à sa poursuite et l'arrêta. 

«Monsieur! s'écria-t-elle, soyez bon, soyez clément, écoutez ma prière: ce couteau que la fatale prudence du baron m'a enlevé, parce qu'il sait l'usage que j'en veux faire; oh! écoutez-moi jusqu'au bout! ce couteau, rendez-le moi une minute seulement, par grâce, par pitié! J'embrasse vos genoux; voyez, vous fermerez la porte, ce n'est pas à vous que j'en veux: Dieu! vous en vouloir, à vous, le seul être juste, bon et compatissant que j'aie rencontré! à vous, mon sauveur peut-être! une minute, ce couteau, une minute, une seule, et je vous le rends par le guichet de la porte; rien qu'une minute, monsieur Felton, et vous m'aurez sauvé l'honneur! 

\speak  Vous tuer! s'écria Felton avec terreur, oubliant de retirer ses mains des mains de la prisonnière; vous tuer! 

\speak  J'ai dit, monsieur, murmura Milady en baissant la voix et en se laissant tomber affaissée sur le parquet, j'ai dit mon secret! il sait tout! mon Dieu, je suis perdue!» 

Felton demeurait debout, immobile et indécis. 

«Il doute encore, pensa Milady, je n'ai pas été assez vraie.» 

On entendit marcher dans le corridor; Milady reconnut le pas de Lord de Winter. Felton le reconnut aussi et s'avança vers la porte. 

Milady s'élança. 

«Oh! pas un mot, dit-elle d'une voix concentrée, pas un mot de tout ce que je vous ai dit à cet homme, ou je suis perdue, et c'est vous, vous\dots» 

Puis, comme les pas se rapprochaient, elle se tut de peur qu'on n'entendit sa voix, appuyant avec un geste de terreur infinie sa belle main sur la bouche de Felton. Felton repoussa doucement Milady, qui alla tomber sur une chaise longue. 

Lord de Winter passa devant la porte sans s'arrêter, et l'on entendit le bruit des pas qui s'éloignaient. 

Felton, pâle comme la mort, resta quelques instants l'oreille tendue et écoutant, puis quand le bruit se fut éteint tout à fait, il respira comme un homme qui sort d'un songe, et s'élança hors de l'appartement. 

«Ah! dit Milady en écoutant à son tour le bruit des pas de Felton, qui s'éloignaient dans la direction opposée à ceux de Lord de Winter, enfin tu es donc à moi!» 

Puis son front se rembrunit. 

«S'il parle au baron, dit-elle, je suis perdue, car le baron, qui sait bien que je ne me tuerai pas, me mettra devant lui un couteau entre les mains, et il verra bien que tout ce grand désespoir n'était qu'un jeu.» 

Elle alla se placer devant sa glace et se regarda; jamais elle n'avait été si belle. 

«Oh! oui! dit-elle en souriant, mais il ne lui parlera pas.» 

Le soir, Lord de Winter accompagna le souper. 

\speak  Monsieur, lui dit Milady, votre présence est-elle un accessoire obligé de ma captivité, et ne pourriez-vous pas m'épargner ce surcroît de tortures que me causent vos visites? 

\speak  Comment donc, chère soeur! dit de Winter, ne m'avez-vous pas sentimentalement annoncé, de cette jolie bouche si cruelle pour moi aujourd'hui, que vous veniez en Angleterre à cette seule fin de me voir tout à votre aise, jouissance dont, me disiez-vous, vous ressentiez si vivement la privation, que vous avez tout risqué pour cela, mal de mer, tempête, captivité! eh bien, me voilà, soyez satisfaite; d'ailleurs, cette fois ma visite a un motif.» 

Milady frissonna, elle crut que Felton avait parlé; jamais de sa vie, peut-être, cette femme, qui avait éprouvé tant d'émotions puissantes et opposées, n'avait senti battre son cœur si violemment. 

Elle était assise; Lord de Winter prit un fauteuil, le tira à son côté et s'assit auprès d'elle, puis prenant dans sa poche un papier qu'il déploya lentement: 

«Tenez, lui dit-il, je voulais vous montrer cette espèce de passeport que j'ai rédigé moi-même et qui vous servira désormais de numéro d'ordre dans la vie que je consens à vous laisser.» 

Puis ramenant ses yeux de Milady sur le papier, il lut: 

«Ordre de conduire à\dots» Le nom est en blanc, interrompit de Winter: si vous avez quelque préférence, vous me l'indiquerez; et pour peu que ce soit à un millier de lieues de Londres, il sera fait droit à votre requête. Je reprends donc: «Ordre de conduire à\dots la nommée Charlotte Backson, flétrie par la justice du royaume de France, mais libérée après châtiment; elle demeurera dans cette résidence, sans jamais s'en écarter de plus de trois lieues. En cas de tentative d'évasion, la peine de mort lui sera appliquée. Elle touchera cinq shillings par jour pour son logement et sa nourriture.» 

«Cet ordre ne me concerne pas, répondit froidement Milady, puisqu'un autre nom que le mien y est porté. 

\speak  Un nom! Est-ce que vous en avez un? 

\speak  J'ai celui de votre frère. 

\speak  Vous vous trompez, mon frère n'est que votre second mari, et le premier vit encore. Dites-moi son nom et je le mettrai en place du nom de Charlotte Backson. Non?\dots vous ne voulez pas?\dots vous gardez le silence? C'est bien! vous serez écrouée sous le nom de Charlotte Backson.» 

Milady demeura silencieuse; seulement, cette fois ce n'était plus par affectation, mais par terreur: elle crut l'ordre prêt à être exécuté: elle pensa que Lord de Winter avait avancé son départ; elle crut qu'elle était condamnée à partir le soir même. Tout dans son esprit fut donc perdu pendant un instant, quand tout à coup elle s'aperçut que l'ordre n'était revêtu d'aucune signature. 

La joie qu'elle ressentit de cette découverte fut si grande, qu'elle ne put la cacher. 

«Oui, oui, dit Lord de Winter, qui s'aperçut de ce qui se passait en elle, oui, vous cherchez la signature, et vous vous dites: tout n'est pas perdu, puisque cet acte n'est pas signé; on me le montre pour m'effrayer, voilà tout. Vous vous trompez: demain cet ordre sera envoyé à Lord Buckingham; après-demain il reviendra signé de sa main et revêtu de son sceau, et vingt-quatre heures après, c'est moi qui vous en réponds, il recevra son commencement d'exécution. Adieu, madame, voilà tout ce que j'avais à vous dire. 

\speak  Et moi je vous répondrai, monsieur, que cet abus de pouvoir, que cet exil sous un nom supposé sont une infamie. 

\speak  Aimez-vous mieux être pendue sous votre vrai nom, Milady? Vous le savez, les lois anglaises sont inexorables sur l'abus que l'on fait du mariage; expliquez-vous franchement: quoique mon nom ou plutôt le nom de mon frère se trouve mêlé dans tout cela, je risquerai le scandale d'un procès public pour être sûr que du coup je serai débarrassé de vous.» 

Milady ne répondit pas, mais devint pâle comme un cadavre. 

«Oh! je vois que vous aimez mieux la pérégrination. À merveille, madame, et il y a un vieux proverbe qui dit que les voyages forment la jeunesse. Ma foi! vous n'avez pas tort, après tout, et la vie est bonne. C'est pour cela que je ne me soucie pas que vous me l'ôtiez. Reste donc à régler l'affaire des cinq shillings; je me montre un peu parcimonieux, n'est-ce pas? cela tient à ce que je ne me soucie pas que vous corrompiez vos gardiens. D'ailleurs il vous restera toujours vos charmes pour les séduire. Usez-en si votre échec avec Felton ne vous a pas dégoûtée des tentatives de ce genre.» 

«Felton n'a point parlé, se dit Milady à elle-même, rien n'est perdu alors.» 

«Et maintenant, madame, à vous revoir. Demain je viendrai vous annoncer le départ de mon messager.» 

Lord de Winter se leva, salua ironiquement Milady et sortit. 

Milady respira: elle avait encore quatre jours devant elle; quatre jours lui suffiraient pour achever de séduire Felton. 

Une idée terrible lui vint alors, c'est que Lord de Winter enverrait peut-être Felton lui-même pour faire signer l'ordre à Buckingham; de cette façon Felton lui échappait, et pour que la prisonnière réussît il fallait la magie d'une séduction continue. 

Cependant, comme nous l'avons dit, une chose la rassurait: Felton n'avait pas parlé. 

Elle ne voulut point paraître émue par les menaces de Lord de Winter, elle se mit à table et mangea. 

Puis, comme elle avait fait la veille, elle se mit à genoux, et répéta tout haut ses prières. Comme la veille, le soldat cessa de marcher et s'arrêta pour l'écouter. 

Bientôt elle entendit des pas plus légers que ceux de la sentinelle qui venaient du fond du corridor et qui s'arrêtaient devant sa porte. 

«C'est lui», dit-elle. 

Et elle commença le même chant religieux qui la veille avait si violemment exalté Felton. 

Mais, quoique sa voix douce, pleine et sonore eût vibré plus harmonieuse et plus déchirante que jamais, la porte resta close. Il parut bien à Milady, dans un des regards furtifs qu'elle lançait sur le petit guichet, apercevoir à travers le grillage serré les yeux ardents du jeune homme mais, que ce fût une réalité ou une vision, cette fois il eut sur lui-même la puissance de ne pas entrer. 

Seulement, quelques instants après qu'elle eût fini son chant religieux, Milady crut entendre un profond soupir; puis les mêmes pas qu'elle avait entendus s'approcher s'éloignèrent lentement et comme à regret. 
\include{chapters/55.tex}
\include{chapters/56.tex}
%!TeX root=../musketeersfr.tex 

\chapter{Un Moyen De Tragédie Classique}

\lettrine{A}{près} un moment de silence employé par Milady à observer le jeune homme qui l'écoutait, elle continua son récit: 

\zz
«Il y avait près de trois jours que je n'avais ni bu ni mangé, je souffrais des tortures atroces: parfois il me passait comme des nuages qui me serraient le front, qui me voilaient les yeux: c'était le délire. 

«Le soir vint; j'étais si faible, qu'à chaque instant je m'évanouissais et à chaque fois que je m'évanouissais je remerciais Dieu, car je croyais que j'allais mourir. 

«Au milieu de l'un de ces évanouissements, j'entendis la porte s'ouvrir; la terreur me rappela à moi. 

«Mon persécuteur entra suivi d'un homme masqué, il était masqué lui-même; mais je reconnus son pas, je reconnus cet air imposant que l'enfer a donné à sa personne pour le malheur de l'humanité. 

«Eh bien, me dit-il, êtes-vous décidée à me faire le serment que je vous ai demandé? 

«Vous l'avez dit, les puritains n'ont qu'une parole: la mienne, vous l'avez entendue, c'est de vous poursuivre sur la terre au tribunal des hommes, dans le ciel au tribunal de Dieu! 

«Ainsi, vous persistez? 

«Je le jure devant ce Dieu qui m'entend: je prendrai le monde entier à témoin de votre crime, et cela jusqu'à ce que j'aie trouvé un vengeur. 

«Vous êtes une prostituée, dit-il d'une voix tonnante, et vous subirez le supplice des prostituées! Flétrie aux yeux du monde que vous invoquerez, tâchez de prouver à ce monde que vous n'êtes ni coupable ni folle!» 

«Puis s'adressant à l'homme qui l'accompagnait: 

«Bourreau, dit-il, fais ton devoir.» 

\speak  Oh! son nom, son nom! s'écria Felton; son nom, dites-le-moi! 

\speak  Alors, malgré mes cris, malgré ma résistance, car je commençais à comprendre qu'il s'agissait pour moi de quelque chose de pire que la mort, le bourreau me saisit, me renversa sur le parquet, me meurtrit de ses étreintes, et suffoquée par les sanglots, presque sans connaissance invoquant Dieu, qui ne m'écoutait pas, je poussai tout à coup un effroyable cri de douleur et de honte; un fer brûlant, un fer rouge, le fer du bourreau, s'était imprimé sur mon épaule.» 

Felton poussa un rugissement. 

«Tenez, dit Milady, en se levant alors avec une majesté de reine, --- tenez, Felton, voyez comment on a inventé un nouveau martyre pour la jeune fille pure et cependant victime de la brutalité d'un scélérat. Apprenez à connaître le cœur des hommes, et désormais faites-vous moins facilement l'instrument de leurs injustes vengeances.» 

Milady d'un geste rapide ouvrit sa robe, déchira la batiste qui couvrait son sein, et, rouge d'une feinte colère et d'une honte jouée, montra au jeune homme l'empreinte ineffaçable qui déshonorait cette épaule si belle. 

«Mais, s'écria Felton, c'est une fleur de lis que je vois là! 

\speak  Et voilà justement où est l'infamie, répondit Milady. La flétrissure d'Angleterre!\dots il fallait prouver quel tribunal me l'avait imposée, et j'aurais fait un appel public à tous les tribunaux du royaume; mais la flétrissure de France\dots oh! par elle, j'étais bien réellement flétrie.» 

C'en était trop pour Felton. 

Pâle, immobile, écrasé par cette révélation effroyable, ébloui par la beauté surhumaine de cette femme qui se dévoilait à lui avec une impudeur qu'il trouva sublime, il finit par tomber à genoux devant elle comme faisaient les premiers chrétiens devant ces pures et saintes martyres que la persécution des empereurs livrait dans le cirque à la sanguinaire lubricité des populaces. La flétrissure disparut, la beauté seule resta. 

«Pardon, pardon! s'écria Felton, oh! pardon!» 

Milady lut dans ses yeux: Amour, amour. 

«Pardon de quoi? demanda-t-elle. 

\speak  Pardon de m'être joint à vos persécuteurs.» 

Milady lui tendit la main. 

«Si belle, si jeune!» s'écria Felton en couvrant cette main de baisers. 

Milady laissa tomber sur lui un de ces regards qui d'un esclave font un roi. 

Felton était puritain: il quitta la main de cette femme pour baiser ses pieds. 

Il ne l'aimait déjà plus, il l'adorait. 

Quand cette crise fut passée, quand Milady parut avoir recouvré son sang-froid, qu'elle n'avait jamais perdu; lorsque Felton eut vu se refermer sous le voile de la chasteté ces trésors d'amour qu'on ne lui cachait si bien que pour les lui faire désirer plus ardemment: 

«Ah! maintenant, dit-il, je n'ai plus qu'une chose à vous demander, c'est le nom de votre véritable bourreau; car pour moi il n'y en a qu'un; l'autre était l'instrument, voilà tout. 

\speak  Eh quoi, frère! s'écria Milady, il faut encore que je te le nomme, et tu ne l'as pas deviné? 

\speak  Quoi! reprit Felton, lui!\dots encore lui!\dots toujours lui!\dots Quoi! le vrai coupable\dots 

\speak  Le vrai coupable, dit Milady, c'est le ravageur de l'Angleterre, le persécuteur des vrais croyants, le lâche ravisseur de l'honneur de tant de femmes, celui qui pour un caprice de son cœur corrompu va faire verser tant de sang à deux royaumes, qui protège les protestants aujourd'hui et qui les trahira demain\dots 

\speak  Buckingham! c'est donc Buckingham!» s'écria Felton exaspéré. 

Milady cacha son visage dans ses mains, comme si elle n'eût pu supporter la honte que lui rappelait ce nom. 

«Buckingham, le bourreau de cette angélique créature! s'écria Felton. Et tu ne l'as pas foudroyé, mon Dieu! et tu l'as laissé noble, honoré, puissant pour notre perte à tous! 

\speak  Dieu abandonne qui s'abandonne lui-même, dit Milady. 

\speak  Mais il veut donc attirer sur sa tête le châtiment réservé aux maudits! continua Felton avec une exaltation croissante, il veut donc que la vengeance humaine prévienne la justice céleste! 

\speak  Les hommes le craignent et l'épargnent. 

\speak  Oh! moi, dit Felton, je ne le crains pas et je ne l'épargnerai pas!\dots» 

Milady sentit son âme baignée d'une joie infernale. 

«Mais comment Lord de Winter, mon protecteur, mon père, demanda Felton, se trouve-t-il mêlé à tout cela? 

\speak  Écoutez, Felton, reprit Milady, car à côté des hommes lâches et méprisables, il est encore des natures grandes et généreuses. J'avais un fiancé, un homme que j'aimais et qui m'aimait; un cœur comme le vôtre, Felton, un homme comme vous. Je vins à lui et je lui racontai tout, il me connaissait, celui-là, et ne douta point un instant. C'était un grand seigneur, c'était un homme en tout point l'égal de Buckingham. Il ne dit rien, il ceignit seulement son épée, s'enveloppa de son manteau et se rendit à Buckingham Palace. 

\speak  Oui, oui, dit Felton, je comprends; quoique avec de pareils hommes ce ne soit pas l'épée qu'il faille employer, mais le poignard. 

\speak  Buckingham était parti depuis la veille, envoyé comme ambassadeur en Espagne, où il allait demander la main de l'infante pour le roi Charles I\ier\, qui n'était alors que prince de Galles. Mon fiancé revint. 

«Écoutez, me dit-il, cet homme est parti, et pour le moment, par conséquent, il échappe à ma vengeance; mais en attendant soyons unis, comme nous devions l'être, puis rapportez-vous-en à Lord de Winter pour soutenir son honneur et celui de sa femme.» 

\speak  Lord de Winter! s'écria Felton. 

\speak  Oui, dit Milady, Lord de Winter, et maintenant vous devez tout comprendre, n'est-ce pas? Buckingham resta plus d'un an absent. Huit jours avant son arrivée, Lord de Winter mourut subitement, me laissant sa seule héritière. D'où venait le coup? Dieu, qui sait tout, le sait sans doute, moi je n'accuse personne\dots 

\speak  Oh! quel abîme, quel abîme! s'écria Felton. 

\speak  Lord de Winter était mort sans rien dire à son frère. Le secret terrible devait être caché à tous, jusqu'à ce qu'il éclatât comme la foudre sur la tête du coupable. Votre protecteur avait vu avec peine ce mariage de son frère aîné avec une jeune fille sans fortune. Je sentis que je ne pouvais attendre d'un homme trompé dans ses espérances d'héritage aucun appui. Je passai en France résolue à y demeurer pendant tout le reste de ma vie. Mais toute ma fortune est en Angleterre; les communications fermées par la guerre, tout me manqua: force fut alors d'y revenir; il y a six jours j'abordais à Portsmouth. 

\speak  Eh bien? dit Felton. 

\speak  Eh bien, Buckingham apprit sans doute mon retour, il en parla à Lord de Winter, déjà prévenu contre moi, et lui dit que sa belle-soeur était une prostituée, une femme flétrie. La voix pure et noble de mon mari n'était plus là pour me défendre. Lord de Winter crut tout ce qu'on lui dit, avec d'autant plus de facilité qu'il avait intérêt à le croire. Il me fit arrêter, me conduisit ici, me remit sous votre garde. Vous savez le reste: après-demain il me bannit, il me déporte; après-demain il me relègue parmi les infâmes. Oh! la trame est bien ourdie, allez! le complot est habile et mon honneur n'y survivra pas. Vous voyez bien qu'il faut que je meure, Felton; Felton, donnez-moi ce couteau!» 

Et à ces mots, comme si toutes ses forces étaient épuisées, Milady se laissa aller débile et languissante entre les bras du jeune officier, qui, ivre d'amour, de colère et de voluptés inconnues, la reçut avec transport, la serra contre son cœur, tout frissonnant à l'haleine de cette bouche si belle, tout éperdu au contact de ce sein si palpitant. 

«Non, non, dit-il; non, tu vivras honorée et pure, tu vivras pour triompher de tes ennemis.» 

Milady le repoussa lentement de la main en l'attirant du regard; mais Felton, à son tour, s'empara d'elle, l'implorant comme une Divinité. 

«Oh! la mort, la mort! dit-elle en voilant sa voix et ses paupières, oh! la mort plutôt que la honte; Felton, mon frère, mon ami, je t'en conjure! 

\speak  Non, s'écria Felton, non, tu vivras, et tu seras vengée! 

\speak  Felton, je porte malheur à tout ce qui m'entoure! Felton, abandonne-moi! Felton, laisse-moi mourir! 

\speak  Eh bien, nous mourrons donc ensemble!» s'écria-t-il en appuyant ses lèvres sur celles de la prisonnière. 

Plusieurs coups retentirent à la porte; cette fois, Milady le repoussa réellement. 

«Écoutez, dit-elle, on nous a entendus, on vient! c'en est fait, nous sommes perdus! 

\speak  Non, dit Felton, c'est la sentinelle qui me prévient seulement qu'une ronde arrive. 

\speak  Alors, courez à la porte et ouvrez vous-même.» 

Felton obéit; cette femme était déjà toute sa pensée, toute son âme. 

Il se trouva en face d'un sergent commandant une patrouille de surveillance. 

«Eh bien, qu'y a-t-il? demanda le jeune lieutenant. 

\speak  Vous m'aviez dit d'ouvrir la porte si j'entendais crier au secours, dit le soldat, mais vous aviez oublié de me laisser la clef; je vous ai entendu crier sans comprendre ce que vous disiez, j'ai voulu ouvrir la porte, elle était fermée en dedans, alors j'ai appelé le sergent. 

\speak  Et me voilà», dit le sergent. 

Felton, égaré, presque fou, demeurait sans voix. 

Milady comprit que c'était à elle de s'emparer de la situation, elle courut à la table et prit le couteau qu'y avait déposé Felton: 

«Et de quel droit voulez-vous m'empêcher de mourir? dit-elle. 

\speak  Grand Dieu!» s'écria Felton en voyant le couteau luire à sa main. 

En ce moment, un éclat de rire ironique retentit dans le corridor. 

Le baron, attiré par le bruit, en robe de chambre, son épée sous le bras, se tenait debout sur le seuil de la porte. 

«Ah! ah! dit-il, nous voici au dernier acte de la tragédie; vous le voyez, Felton, le drame a suivi toutes les phases que j'avais indiquées; mais soyez tranquille, le sang ne coulera pas.» 

Milady comprit qu'elle était perdue si elle ne donnait pas à Felton une preuve immédiate et terrible de son courage. 

«Vous vous trompez, Milord, le sang coulera, et puisse ce sang retomber sur ceux qui le font couler!» 

Felton jeta un cri et se précipita vers elle; il était trop tard: Milady s'était frappée. Mais le couteau avait rencontré, heureusement, nous devrions dire adroitement, le busc de fer qui, à cette époque, défendait comme une cuirasse la poitrine des femmes; il avait glissé en déchirant la robe, et avait pénétré de biais entre la chair et les côtes. 

La robe de Milady n'en fut pas moins tachée de sang en une seconde. 

Milady était tombée à la renverse et semblait évanouie. 

Felton arracha le couteau. 

«Voyez, Milord, dit-il d'un air sombre, voici une femme qui était sous ma garde et qui s'est tuée! 

\speak  Soyez tranquille, Felton, dit Lord de Winter, elle n'est pas morte, les démons ne meurent pas si facilement, soyez tranquille et allez m'attendre chez moi. 

\speak  Mais, Milord\dots 

\speak  Allez, je vous l'ordonne.» 

À cette injonction de son supérieur, Felton obéit; mais, en sortant, il mit le couteau dans sa poitrine. 

Quant à Lord de Winter, il se contenta d'appeler la femme qui servait Milady et, lorsqu'elle fut venue, lui recommandant la prisonnière toujours évanouie, il la laissa seule avec elle. 

Cependant, comme à tout prendre, malgré ses soupçons, la blessure pouvait être grave, il envoya, à l'instant même, un homme à cheval chercher un médecin.
%!TeX root=../musketeersfr.tex 

\chapter{Évasion}

\lettrine{C}{omme} l'avait pensé Lord de Winter, la blessure de Milady n'était pas dangereuse; aussi dès qu'elle se trouva seule avec la femme que le baron avait fait appeler et qui se hâtait de la déshabiller, rouvrit-elle les yeux. 

Cependant, il fallait jouer la faiblesse et la douleur; ce n'étaient pas choses difficiles pour une comédienne comme Milady; aussi la pauvre femme fut-elle si complètement dupe de sa prisonnière, que, malgré ses instances, elle s'obstina à la veiller toute la nuit. 

Mais la présence de cette femme n'empêchait pas Milady de songer. 

Il n'y avait plus de doute, Felton était convaincu, Felton était à elle: un ange apparût-il au jeune homme pour accuser Milady, il le prendrait certainement, dans la disposition d'esprit où il se trouvait, pour un envoyé du démon. 

Milady souriait à cette pensée, car Felton, c'était désormais sa seule espérance, son seul moyen de salut. 

Mais Lord de Winter pouvait l'avoir soupçonné, mais Felton maintenant pouvait être surveillé lui-même. 

Vers les quatre heures du matin, le médecin arriva; mais depuis le temps où Milady s'était frappée, la blessure s'était déjà refermée: le médecin ne put donc en mesurer ni la direction, ni la profondeur; il reconnut seulement au pouls de la malade que le cas n'était point grave. 

Le matin, Milady, sous prétexte qu'elle n'avait pas dormi de la nuit et qu'elle avait besoin de repos, renvoya la femme qui veillait près d'elle. 

Elle avait une espérance, c'est que Felton arriverait à l'heure du déjeuner, mais Felton ne vint pas. 

Ses craintes s'étaient-elles réalisées? Felton, soupçonné par le baron, allait-il lui manquer au moment décisif? Elle n'avait plus qu'un jour: Lord de Winter lui avait annoncé son embarquement pour le 23 et l'on était arrivé au matin du 22. 

Néanmoins, elle attendit encore assez patiemment jusqu'à l'heure du dîner. 

Quoiqu'elle n'eût pas mangé le matin, le dîner fut apporté à l'heure habituelle; Milady s'aperçut alors avec effroi que l'uniforme des soldats qui la gardaient était changé. 

Alors elle se hasarda à demander ce qu'était devenu Felton. On lui répondit que Felton était monté à cheval il y avait une heure, et était parti. 

Elle s'informa si le baron était toujours au château; le soldat répondit que oui, et qu'il avait ordre de le prévenir si la prisonnière désirait lui parler. 

Milady répondit qu'elle était trop faible pour le moment, et que son seul désir était de demeurer seule. 

Le soldat sortit, laissant le dîner servi. 

Felton était écarté, les soldats de marine étaient changés, on se défiait donc de Felton. 

C'était le dernier coup porté à la prisonnière. 

Restée seule, elle se leva; ce lit où elle se tenait par prudence et pour qu'on la crût gravement blessée, la brûlait comme un brasier ardent. Elle jeta un coup d'œil sur la porte: le baron avait fait clouer une planche sur le guichet; il craignait sans doute que, par cette ouverture, elle ne parvint encore, par quelque moyen diabolique, à séduire les gardes. 

Milady sourit de joie; elle pouvait donc se livrer à ses transports sans être observée: elle parcourait la chambre avec l'exaltation d'une folle furieuse ou d'une tigresse enfermée dans une cage de fer. Certes, si le couteau lui fût resté, elle eût songé, non plus à se tuer elle-même, mais, cette fois, à tuer le baron. 

À six heures, Lord de Winter entra; il était armé jusqu'aux dents. Cet homme, dans lequel, jusque-là, Milady n'avait vu qu'un gentleman assez niais, était devenu un admirable geôlier: il semblait tout prévoir, tout deviner, tout prévenir. 

Un seul regard jeté sur Milady lui apprit ce qui se passait dans son âme. 

«Soit, dit-il, mais vous ne me tuerez point encore aujourd'hui; vous n'avez plus d'armes, et d'ailleurs je suis sur mes gardes. Vous aviez commencé à pervertir mon pauvre Felton: il subissait déjà votre infernale influence, mais je veux le sauver, il ne vous verra plus, tout est fini. Rassemblez vos hardes, demain vous partirez. J'avais fixé l'embarquement au 24, mais j'ai pensé que plus la chose serait rapprochée, plus elle serait sûre. Demain à midi j'aurai l'ordre de votre exil, signé Buckingham. Si vous dites un seul mot à qui que ce soit avant d'être sur le navire, mon sergent vous fera sauter la cervelle, et il en a l'ordre; si, sur le navire, vous dites un mot à qui que ce soit avant que le capitaine vous le permette, le capitaine vous fait jeter à la mer, c'est convenu. Au revoir, voilà ce que pour aujourd'hui j'avais à vous dire. Demain je vous reverrai pour vous faire mes adieux!» 

Et sur ces paroles le baron sortit. 

Milady avait écouté toute cette menaçante tirade le sourire du dédain sur les lèvres, mais la rage dans le cœur. 

On servit le souper; Milady sentit qu'elle avait besoin de forces, elle ne savait pas ce qui pouvait se passer pendant cette nuit qui s'approchait menaçante, car de gros nuages roulaient au ciel, et des éclairs lointains annonçaient un orage. 

L'orage éclata vers les dix heures du soir: Milady sentait une consolation à voir la nature partager le désordre de son cœur; la foudre grondait dans l'air comme la colère dans sa pensée, il lui semblait que la rafale, en passant, échevelait son front comme les arbres dont elle courbait les branches et enlevait les feuilles; elle hurlait comme l'ouragan, et sa voix se perdait dans la grande voix de la nature, qui, elle aussi, semblait gémir et se désespérer. 

Tout à coup elle entendit frapper à une vitre, et, à la lueur d'un éclair, elle vit le visage d'un homme apparaître derrière les barreaux. 

Elle courut à la fenêtre et l'ouvrit. 

«Felton! s'écria-t-elle, je suis sauvée! 

\speak  Oui, dit Felton! mais silence, silence! il me faut le temps de scier vos barreaux. Prenez garde seulement qu'ils ne vous voient par le guichet. 

\speak  Oh! c'est une preuve que le Seigneur est pour nous, Felton, reprit Milady, ils ont fermé le guichet avec une planche. 

\speak  C'est bien, Dieu les a rendus insensés! dit Felton. 

\speak  Mais que faut-il que je fasse? demanda Milady. 

\speak  Rien, rien; refermez la fenêtre seulement. Couchez-vous, ou, du moins, mettez-vous dans votre lit tout habillée; quand j'aurai fini, je frapperai aux carreaux. Mais pourrez-vous me suivre? 

\speak  Oh! oui. 

\speak  Votre blessure? 

\speak  Me fait souffrir, mais ne m'empêche pas de marcher. 

\speak  Tenez-vous donc prête au premier signal.» 

Milady referma la fenêtre, éteignit la lampe, et alla, comme le lui avait recommandé Felton, se blottir dans son lit. Au milieu des plaintes de l'orage, elle entendait le grincement de la lime contre les barreaux, et, à la lueur de chaque éclair, elle apercevait l'ombre de Felton derrière les vitres. 

Elle passa une heure sans respirer, haletante, la sueur sur le front, et le cœur serré par une épouvantable angoisse à chaque mouvement qu'elle entendait dans le corridor. 

Il y a des heures qui durent une année. 

Au bout d'une heure, Felton frappa de nouveau. 

Milady bondit hors de son lit et alla ouvrir. Deux barreaux de moins formaient une ouverture à passer un homme. 

«Êtes-vous prête? demanda Felton. 

\speak  Oui. Faut-il que j'emporte quelque chose? 

\speak  De l'or, si vous en avez. 

\speak  Oui, heureusement on m'a laissé ce que j'en avais. 

\speak  Tant mieux, car j'ai usé tout le mien pour fréter une barque. 

\speak  Prenez», dit Milady en mettant aux mains de Felton un sac plein d'or. 

Felton prit le sac et le jeta au pied du mur. 

«Maintenant, dit-il, voulez-vous venir? 

\speak  Me voici.» 

Milady monta sur un fauteuil et passa tout le haut de son corps par la fenêtre: elle vit le jeune officier suspendu au-dessus de l'abîme par une échelle de corde. 

Pour la première fois, un mouvement de terreur lui rappela qu'elle était femme. 

Le vide l'épouvantait. 

«Je m'en étais douté, dit Felton. 

\speak  Ce n'est rien, ce n'est rien, dit Milady, je descendrai les yeux fermés. 

\speak  Avez-vous confiance en moi? dit Felton. 

\speak  Vous le demandez? 

\speak  Rapprochez vos deux mains; croisez-les, c'est bien.» 

Felton lui lia les deux poignets avec son mouchoir, puis par-dessus le mouchoir, avec une corde. 

«Que faites-vous? demanda Milady avec surprise. 

\speak  Passez vos bras autour de mon cou et ne craignez rien. 

\speak  Mais je vous ferai perdre l'équilibre, et nous nous briserons tous les deux. 

\speak  Soyez tranquille, je suis marin.» 

Il n'y avait pas une seconde à perdre; Milady passa ses deux bras autour du cou de Felton et se laissa glisser hors de la fenêtre. 

Felton se mit à descendre les échelons lentement et un à un. Malgré la pesanteur des deux corps, le souffle de l'ouragan les balançait dans l'air. 

Tout à coup Felton s'arrêta. 

«Qu'y a-t-il? demanda Milady. 

\speak  Silence, dit Felton, j'entends des pas. 

\speak  Nous sommes découverts!» 

Il se fit un silence de quelques instants. 

«Non, dit Felton, ce n'est rien. 

\speak  Mais enfin quel est ce bruit? 

\speak  Celui de la patrouille qui va passer sur le chemin de ronde. 

\speak  Où est le chemin de ronde? 

\speak  Juste au-dessous de nous. 

\speak  Elle va nous découvrir. 

\speak  Non, s'il ne fait pas d'éclairs. 

\speak  Elle heurtera le bas de l'échelle. 

\speak  Heureusement elle est trop courte de six pieds. 

\speak  Les voilà, mon Dieu! 

\speak  Silence!» 

Tous deux restèrent suspendus, immobiles et sans souffle, à vingt pieds du sol; pendant ce temps les soldats passaient au-dessous riant et causant. 

Il y eut pour les fugitifs un moment terrible. 

La patrouille passa; on entendit le bruit des pas qui s'éloignait, et le murmure des voix qui allait s'affaiblissant. 

«Maintenant, dit Felton, nous sommes sauvés.» 

Milady poussa un soupir et s'évanouit. 

Felton continua de descendre. Parvenu au bas de l'échelle, et lorsqu'il ne sentit plus d'appui pour ses pieds, il se cramponna avec ses mains; enfin, arrivé au dernier échelon il se laissa pendre à la force des poignets et toucha la terre. Il se baissa, ramassa le sac d'or et le prit entre ses dents. 

Puis il souleva Milady dans ses bras, et s'éloigna vivement du côté opposé à celui qu'avait pris la patrouille. Bientôt il quitta le chemin de ronde, descendit à travers les rochers, et, arrivé au bord de la mer, fit entendre un coup de sifflet. 

Un signal pareil lui répondit, et, cinq minutes après, il vit apparaître une barque montée par quatre hommes. 

La barque s'approcha aussi près qu'elle put du rivage, mais il n'y avait pas assez de fond pour qu'elle pût toucher le bord; Felton se mit à l'eau jusqu'à la ceinture, ne voulant confier à personne son précieux fardeau. 

Heureusement la tempête commençait à se calmer, et cependant la mer était encore violente; la petite barque bondissait sur les vagues comme une coquille de noix. 

«Au sloop, dit Felton, et nagez vivement.» 

Les quatre hommes se mirent à la rame; mais la mer était trop grosse pour que les avirons eussent grande prise dessus. 

Toutefois on s'éloignait du château; c'était le principal. La nuit était profondément ténébreuse, et il était déjà presque impossible de distinguer le rivage de la barque, à plus forte raison n'eût-on pas pu distinguer la barque du rivage. 

Un point noir se balançait sur la mer. 

C'était le sloop. 

Pendant que la barque s'avançait de son côté de toute la force de ses quatre rameurs, Felton déliait la corde, puis le mouchoir qui liait les mains de Milady. 

Puis, lorsque ses mains furent déliées, il prit de l'eau de la mer et la lui jeta au visage. 

Milady poussa un soupir et ouvrit les yeux. 

«Où suis-je? dit-elle. 

\speak  Sauvée, répondit le jeune officier. 

\speak  Oh! sauvée! sauvée! s'écria-t-elle. Oui, voici le ciel, voici la mer! Cet air que je respire, c'est celui de la liberté. Ah!\dots merci, Felton, merci!» 

Le jeune homme la pressa contre son cœur. 

«Mais qu'ai-je donc aux mains? demanda Milady; il me semble qu'on m'a brisé les poignets dans un étau.» 

En effet, Milady souleva ses bras: elle avait les poignets meurtris. 

«Hélas! dit Felton en regardant ces belles mains et en secouant doucement la tête. 

\speak  Oh! ce n'est rien, ce n'est rien! s'écria Milady: maintenant je me rappelle!» 

Milady chercha des yeux autour d'elle. 

«Il est là», dit Felton en poussant du pied le sac d'or. 

On s'approchait du sloop. Le marin de quart héla la barque, la barque répondit. 

«Quel est ce bâtiment? demanda Milady. 

\speak  Celui que j'ai frété pour vous. 

\speak  Où va-t-il me conduire? 

\speak  Où vous voudrez, pourvu que, moi, vous me jetiez à Portsmouth. 

\speak  Qu'allez-vous faire à Portsmouth? demanda Milady. 

\speak  Accomplir les ordres de Lord de Winter, dit Felton avec un sombre sourire. 

\speak  Quels ordres? demanda Milady. 

\speak  Vous ne comprenez donc pas? dit Felton. 

\speak  Non; expliquez-vous, je vous en prie. 

\speak  Comme il se défiait de moi, il a voulu vous garder lui-même, et m'a envoyé à sa place faire signer à Buckingham l'ordre de votre déportation. 

\speak  Mais s'il se défiait de vous, comment vous a-t-il confié cet ordre? 

\speak  Étais-je censé savoir ce que je portais? 

\speak  C'est juste. Et vous allez à Portsmouth? 

\speak  Je n'ai pas de temps à perdre: c'est demain le 23, et Buckingham part demain avec la flotte. 

\speak  Il part demain, pour où part-il? 

\speak  Pour La Rochelle. 

\speak  Il ne faut pas qu'il parte! s'écria Milady, oubliant sa présence d'esprit accoutumée. 

\speak  Soyez tranquille, répondit Felton, il ne partira pas.» 

Milady tressaillit de joie; elle venait de lire au plus profond du cœur du jeune homme: la mort de Buckingham y était écrite en toutes lettres. 

«Felton\dots, dit-elle, vous êtes grand comme Judas Macchabée! Si vous mourez, je meurs avec vous: voilà tout ce que je puis vous dire. 

\speak  Silence! dit Felton, nous sommes arrivés.» 

En effet, on touchait au sloop. 

Felton monta le premier à l'échelle et donna la main à Milady, tandis que les matelots la soutenaient, car la mer était encore fort agitée. 

Un instant après ils étaient sur le pont. 

«Capitaine, dit Felton, voici la personne dont je vous ai parlé, et qu'il faut conduire saine et sauve en France. 

\speak  Moyennant mille pistoles, dit le capitaine. 

\speak  Je vous en ai donné cinq cents. 

\speak  C'est juste, dit le capitaine. 

\speak  Et voilà les cinq cents autres, reprit Milady, en portant la main au sac d'or. 

\speak  Non, dit le capitaine, je n'ai qu'une parole, et je l'ai donnée à ce jeune homme; les cinq cents autres pistoles ne me sont dues qu'en arrivant à Boulogne. 

\speak  Et nous y arriverons? 

\speak  Sains et saufs, dit le capitaine, aussi vrai que je m'appelle Jack Buttler. 

\speak  Eh bien, dit Milady, si vous tenez votre parole, ce n'est pas cinq cents, mais mille pistoles que je vous donnerai. 

\speak  Hurrah pour vous alors, ma belle dame, cria le capitaine, et puisse Dieu m'envoyer souvent des pratiques comme Votre Seigneurie! 

\speak  En attendant, dit Felton, conduisez-nous dans la petite baie de Chichester, en avant de Portsmouth; vous savez qu'il est convenu que vous nous conduirez là.» 

Le capitaine répondit en commandant la manoeuvre nécessaire, et vers les sept heures du matin le petit bâtiment jetait l'ancre dans la baie désignée. 

Pendant cette traversée, Felton avait tout raconté à Milady: comment, au lieu d'aller à Londres, il avait frété le petit bâtiment, comment il était revenu, comment il avait escaladé la muraille en plaçant dans les interstices des pierres, à mesure qu'il montait, des crampons, pour assurer ses pieds, et comment enfin, arrivé aux barreaux, il avait attaché l'échelle, Milady savait le reste. 

De son côté, Milady essaya d'encourager Felton dans son projet, mais aux premiers mots qui sortirent de sa bouche, elle vit bien que le jeune fanatique avait plutôt besoin d'être modéré que d'être affermi. 

Il fut convenu que Milady attendrait Felton jusqu'à dix heures; si à dix heures il n'était pas de retour, elle partirait. 

Alors, en supposant qu'il fût libre, il la rejoindrait en France, au couvent des Carmélites de Béthune. 
\include{chapters/59.tex}
%!TeX root=../musketeersfr.tex 

\chapter{En France}

\lettrine{L}{a} première crainte du roi d'Angleterre, Charles I\ier\, en apprenant cette mort, fut qu'une si terrible nouvelle ne décourageât les Rochelois; il essaya, dit Richelieu dans ses Mémoires, de la leur cacher le plus longtemps possible, faisant fermer les ports par tout son royaume, et prenant soigneusement garde qu'aucun vaisseau ne sortit jusqu'à ce que l'armée que Buckingham apprêtait fût partie, se chargeant, à défaut de Buckingham, de surveiller lui-même le départ. 

Il poussa même la sévérité de cet ordre jusqu'à retenir en Angleterre l'ambassadeur de Danemark, qui avait pris congé, et l'ambassadeur ordinaire de Hollande, qui devait ramener dans le port de Flessingue les navires des Indes que Charles I\ier\ avait fait restituer aux Provinces-Unies. 

Mais comme il ne songea à donner cet ordre que cinq heures après l'événement, c'est-à-dire à deux heures de l'après-midi, deux navires étaient déjà sortis du port: l'un emmenant, comme nous le savons, Milady, laquelle, se doutant déjà de l'événement, fut encore confirmée dans cette croyance en voyant le pavillon noir se déployer au mât du vaisseau amiral. 

Quant au second bâtiment, nous dirons plus tard qui il portait et comment il partit. 

Pendant ce temps, du reste, rien de nouveau au camp de La Rochelle; seulement le roi, qui s'ennuyait fort, comme toujours, mais peut-être encore un peu plus au camp qu'ailleurs, résolut d'aller incognito passer les fêtes de Saint-Louis à Saint-Germain, et demanda au cardinal de lui faire préparer une escorte de vingt mousquetaires seulement. Le cardinal, que l'ennui du roi gagnait quelquefois, accorda avec grand plaisir ce congé à son royal lieutenant, lequel promit d'être de retour vers le 15 septembre. 

M. de Tréville, prévenu par Son Éminence, fit son portemanteau, et comme, sans en savoir la cause, il savait le vif désir et même l'impérieux besoin que ses amis avaient de revenir à Paris, il va sans dire qu'il les désigna pour faire partie de l'escorte. 

Les quatre jeunes gens surent la nouvelle un quart d'heure après M. de Tréville, car ils furent les premiers à qui il la communiqua. Ce fut alors que d'Artagnan apprécia la faveur que lui avait accordée le cardinal en le faisant enfin passer aux mousquetaires; sans cette circonstance, il était forcé de rester au camp tandis que ses compagnons partaient. 

On verra plus tard que cette impatience de remonter vers Paris avait pour cause le danger que devait courir Mme Bonacieux en se rencontrant au couvent de Béthune avec Milady, son ennemie mortelle. Aussi, comme nous l'avons dit, Aramis avait écrit immédiatement à Marie Michon, cette lingère de Tours qui avait de si belles connaissances, pour qu'elle obtînt que la reine donnât l'autorisation à Mme Bonacieux de sortir du couvent et de se retirer soit en Lorraine, soit en Belgique. La réponse ne s'était pas fait attendre, et, huit ou dix jours après, Aramis avait reçu cette lettre: 


\begin{mail}{}{Mon cher cousin,}
	
Voici l'autorisation de ma sœur à retirer notre petite servante du couvent de Béthune, dont vous pensez que l'air est mauvais pour elle. Ma sœur vous envoie cette autorisation avec grand plaisir, car elle aime fort cette petite fille, à laquelle elle se réserve d'être utile plus tard.
\closeletter[Je vous embrasse.]{Marie Michon}
\end{mail}

À cette lettre était jointe une autorisation ainsi conçue:

\begin{mail}{Au Louvre, le 10 août 1628.}{}
La supérieure du couvent de Béthune remettra aux mains de la personne qui lui remettra ce billet la novice qui était entrée dans son couvent sous ma recommandation et sous mon patronage.\closeletter{Anne}
\end{mail}


On comprend combien ces relations de parenté entre Aramis et une lingère qui appelait la reine sa sœur avaient égayé la verve des jeunes gens; mais Aramis, après avoir rougi deux ou trois fois jusqu'au blanc des yeux aux grosses plaisanteries de Porthos, avait prié ses amis de ne plus revenir sur ce sujet, déclarant que s'il lui en était dit encore un seul mot, il n'emploierait plus sa cousine comme intermédiaire dans ces sortes d'affaires. 

Il ne fut donc plus question de Marie Michon entre les quatre mousquetaires, qui d'ailleurs avaient ce qu'ils voulaient: l'ordre de tirer Mme Bonacieux du couvent des Carmélites de Béthune. Il est vrai que cet ordre ne leur servirait pas à grand-chose tant qu'ils seraient au camp de La Rochelle, c'est-à-dire à l'autre bout de la France; aussi d'Artagnan allait-il demander un congé à M. de Tréville, en lui confiant tout bonnement l'importance de son départ, lorsque cette nouvelle lui fut transmise, ainsi qu'à ses trois compagnons, que le roi allait partir pour Paris avec une escorte de vingt mousquetaires, et qu'ils faisaient partie de l'escorte. 

La joie fut grande. On envoya les valets devant avec les bagages, et l'on partit le 16 au matin. 

Le cardinal reconduisit Sa Majesté de Surgères à Mauzé, et là, le roi et son ministre prirent congé l'un de l'autre avec de grandes démonstrations d'amitié. 

Cependant le roi, qui cherchait de la distraction, tout en cheminant le plus vite qu'il lui était possible, car il désirait être arrivé à Paris pour le 23, s'arrêtait de temps en temps pour voler la pie, passe-temps dont le goût lui avait autrefois été inspiré par de Luynes, et pour lequel il avait toujours conservé une grande prédilection. Sur les vingt mousquetaires, seize, lorsque la chose arrivait, se réjouissaient fort de ce bon temps; mais quatre maugréaient de leur mieux. D'Artagnan surtout avait des bourdonnements perpétuels dans les oreilles, ce que Porthos expliquait ainsi: 

«Une très grande dame m'a appris que cela veut dire que l'on parle de vous quelque part.» 

Enfin l'escorte traversa Paris le 23, dans la nuit; le roi remercia M. de Tréville, et lui permit de distribuer des congés pour quatre jours, à la condition que pas un des favorisés ne paraîtrait dans un lieu public, sous peine de la Bastille. 

Les quatre premiers congés accordés, comme on le pense bien, furent à nos quatre amis. Il y a plus, Athos obtint de M. de Tréville six jours au lieu de quatre et fit mettre dans ces six jours deux nuits de plus, car ils partirent le 24, à cinq heures du soir, et par complaisance encore, M. de Tréville postdata le congé du 25 au matin. 

«Eh, mon Dieu, disait d'Artagnan, qui, comme on le sait, ne doutait jamais de rien, il me semble que nous faisons bien de l'embarras pour une chose bien simple: en deux jours, et en crevant deux ou trois chevaux (peu m'importe: j'ai de l'argent), je suis à Béthune, je remets la lettre de la reine à la supérieure, et je ramène le cher trésor que je vais chercher, non pas en Lorraine, non pas en Belgique, mais à Paris, où il sera mieux caché, surtout tant que M. le cardinal sera à La Rochelle. Puis, une fois de retour de la campagne, eh bien, moitié par la protection de sa cousine, moitié en faveur de ce que nous avons fait personnellement pour elle, nous obtiendrons de la reine ce que nous voudrons. Restez donc ici, ne vous épuisez pas de fatigue inutilement; moi et Planchet, c'est tout ce qu'il faut pour une expédition aussi simple.» 

À ceci Athos répondit tranquillement: 

«Nous aussi, nous avons de l'argent; car je n'ai pas encore bu tout à fait le reste du diamant, et Porthos et Aramis ne l'ont pas tout à fait mangé. Nous crèverons donc aussi bien quatre chevaux qu'un. Mais songez, d'Artagnan, ajouta-t-il d'une voix si sombre que son accent donna le frisson au jeune homme, songez que Béthune est une ville où le cardinal a donné rendez-vous à une femme qui, partout où elle va, mène le malheur après elle. Si vous n'aviez affaire qu'à quatre hommes, d'Artagnan, je vous laisserais aller seul; vous avez affaire à cette femme, allons-y quatre, et plaise à Dieu qu'avec nos quatre valets nous soyons en nombre suffisant! 

\speak  Vous m'épouvantez, Athos, s'écria d'Artagnan; que craignez-vous donc, mon Dieu? 

\speak  Tout!» répondit Athos. 

D'Artagnan examina les visages de ses compagnons, qui, comme celui d'Athos, portaient l'empreinte d'une inquiétude profonde, et l'on continua la route au plus grand pas des chevaux, mais sans ajouter une seule parole. 

Le 25 au soir, comme ils entraient à Arras, et comme d'Artagnan venait de mettre pied à terre à l'auberge de la Herse d'Or pour boire un verre de vin, un cavalier sortit de la cour de la poste, où il venait de relayer, prenant au grand galop, et avec un cheval frais, le chemin de Paris. Au moment où il passait de la grande porte dans la rue, le vent entrouvrit le manteau dont il était enveloppé, quoiqu'on fût au mois d'août, et enleva son chapeau, que le voyageur retint de sa main, au moment où il avait déjà quitté sa tête, et l'enfonça vivement sur ses yeux. 

D'Artagnan, qui avait les yeux fixés sur cet homme, devint fort pâle et laissa tomber son verre. 

«Qu'avez-vous, monsieur? dit Planchet\dots Oh! là, accourez, messieurs, voilà mon maître qui se trouve mal!» 

Les trois amis accoururent et trouvèrent d'Artagnan qui, au lieu de se trouver mal, courait à son cheval. Ils l'arrêtèrent sur le seuil de la porte. 

«Eh bien, où diable vas-tu donc ainsi? lui cria Athos. 

\speak  C'est lui! s'écria d'Artagnan, pâle de colère et la sueur sur le front, c'est lui! laissez-moi le rejoindre! 

\speak  Mais qui, lui? demanda Athos. 

\speak  Lui, cet homme! 

\speak  Quel homme? 

\speak  Cet homme maudit, mon mauvais génie, que j'ai toujours vu lorsque j'étais menacé de quelque malheur: celui qui accompagnait l'horrible femme lorsque je la rencontrai pour la première fois, celui que je cherchais quand j'ai provoqué Athos, celui que j'ai vu le matin du jour où Mme Bonacieux a été enlevée! l'homme de Meung enfin! je l'ai vu, c'est lui! Je l'ai reconnu quand le vent a entrouvert son manteau. 

\speak  Diable! dit Athos rêveur. 

\speak  En selle, messieurs, en selle; poursuivons-le, et nous le rattraperons. 

\speak  Mon cher, dit Aramis, songez qu'il va du côté opposé à celui où nous allons; qu'il a un cheval frais et que nos chevaux sont fatigués; que par conséquent nous crèverons nos chevaux sans même avoir la chance de le rejoindre. Laissons l'homme, d'Artagnan, sauvons la femme. 

\speak  Eh! monsieur! s'écria un garçon d'écurie courant après l'inconnu, eh! monsieur, voilà un papier qui s'est échappé de votre chapeau! Eh! monsieur! eh! 

\speak  Mon ami, dit d'Artagnan, une demi-pistole pour ce papier! 

\speak  Ma foi, monsieur, avec grand plaisir! le voici! 

Le garçon d'écurie, enchanté de la bonne journée qu'il avait faite, rentra dans la cour de l'hôtel: d'Artagnan déplia le papier. 

«Eh bien? demandèrent ses amis en l'entourant. 

\speak  Rien qu'un mot! dit d'Artagnan. 

\speak  Oui, dit Aramis, mais ce nom est un nom de ville ou de village. 

\speak «\textit{Armentières}», lut Porthos. Armentières, je ne connais pas cela! 

\speak  Et ce nom de ville ou de village est écrit de sa main! s'écria Athos. 

\speak  Allons, allons, gardons soigneusement ce papier, dit d'Artagnan, peut-être n'ai-je pas perdu ma dernière pistole. À cheval, mes amis, à cheval!» 

Et les quatre compagnons s'élancèrent au galop sur la route de Béthune.
%!TeX root=../musketeersfr.tex 

\chapter{Le Couvent Des Carmélites De Béthune}

\lettrine{L}{es} grands criminels portent avec eux une espèce de prédestination qui leur fait surmonter tous les obstacles, qui les fait échapper à tous les dangers, jusqu'au moment que la Providence, lassée, a marqué pour l'écueil de leur fortune impie. 

Il en était ainsi de Milady: elle passa au travers des croiseurs des deux nations, et arriva à Boulogne sans aucun accident. 

En débarquant à Portsmouth, Milady était une Anglaise que les persécutions de la France chassaient de La Rochelle; débarquée à Boulogne, après deux jours de traversée, elle se fit passer pour une Française que les Anglais inquiétaient à Portsmouth, dans la haine qu'ils avaient conçue contre la France. 

Milady avait d'ailleurs le plus efficace des passeports: sa beauté, sa grande mine et la générosité avec laquelle elle répandait les pistoles. Affranchie des formalités d'usage par le sourire affable et les manières galantes d'un vieux gouverneur du port, qui lui baisa la main, elle ne resta à Boulogne que le temps de mettre à la poste une lettre ainsi conçue: 

\begin{a4}
	\clearpage
\end{a4}

\begin{mail}{À Son Éminence Monseigneur le cardinal de Richelieu,\\ en son camp devant La Rochelle.}{Monseigneur,} 
Que Votre Éminence se rassure, Sa Grâce le duc de Buckingham ne \textit{partira point} pour la France.	\addPS{Boulogne, 25 au soir. --- Selon les désirs de Votre Éminence, je me rends au couvent des Carmélites de Béthune où j'attendrai ses ordres.}
	\closeletter{Milady de ------} 
\end{mail}

Effectivement, le même soir, Milady se mit en route; la nuit la prit: elle s'arrêta et coucha dans une auberge; puis, le lendemain, à cinq heures du matin, elle partit, et trois heures après, elle entra à Béthune. 

Elle se fit indiquer le couvent des Carmélites et y entra aussitôt. 

La supérieure vint au-devant d'elle; Milady lui montra l'ordre du cardinal, l'abbesse lui fit donner une chambre et servir à déjeuner. 

Tout le passé s'était déjà effacé aux yeux de cette femme, et, le regard fixé vers l'avenir, elle ne voyait que la haute fortune que lui réservait le cardinal, qu'elle avait si heureusement servi, sans que son nom fût mêlé en rien à toute cette sanglante affaire. Les passions toujours nouvelles qui la consumaient donnaient à sa vie l'apparence de ces nuages qui volent dans le ciel, reflétant tantôt l'azur, tantôt le feu, tantôt le noir opaque de la tempête, et qui ne laissent d'autres traces sur la terre que la dévastation et la mort. 

Après le déjeuner, l'abbesse vint lui faire sa visite; il y a peu de distraction au cloître, et la bonne supérieure avait hâte de faire connaissance avec sa nouvelle pensionnaire. 

Milady voulait plaire à l'abbesse; or, c'était chose facile à cette femme si réellement supérieure; elle essaya d'être aimable: elle fut charmante et séduisit la bonne supérieure par sa conversation si variée et par les grâces répandues dans toute sa personne. 

L'abbesse, qui était une fille de noblesse, aimait surtout les histoires de cour, qui parviennent si rarement jusqu'aux extrémités du royaume et qui, surtout, ont tant de peine à franchir les murs des couvents, au seuil desquels viennent expirer les bruits du monde. 

Milady, au contraire, était fort au courant de toutes les intrigues aristocratiques, au milieu desquelles, depuis cinq ou six ans, elle avait constamment vécu, elle se mit donc à entretenir la bonne abbesse des pratiques mondaines de la cour de France, mêlées aux dévotions outrées du roi, elle lui fit la chronique scandaleuse des seigneurs et des dames de la cour, que l'abbesse connaissait parfaitement de nom, toucha légèrement les amours de la reine et de Buckingham, parlant beaucoup pour qu'on parlât un peu. 

Mais l'abbesse se contenta d'écouter et de sourire, le tout sans répondre. Cependant, comme Milady vit que ce genre de récit l'amusait fort, elle continua; seulement, elle fit tomber la conversation sur le cardinal. 

Mais elle était fort embarrassée; elle ignorait si l'abbesse était royaliste ou cardinaliste: elle se tint dans un milieu prudent; mais l'abbesse, de son côté, se tint dans une réserve plus prudente encore, se contentant de faire une profonde inclination de tête toutes les fois que la voyageuse prononçait le nom de Son Éminence. 

Milady commença à croire qu'elle s'ennuierait fort dans le couvent; elle résolut donc de risquer quelque chose pour savoir de suite à quoi s'en tenir. Voulant voir jusqu'où irait la discrétion de cette bonne abbesse, elle se mit à dire un mal, très dissimulé d'abord, puis très circonstancié du cardinal, racontant les amours du ministre avec Mme d'Aiguillon, avec Marion de Lorme et avec quelques autres femmes galantes. 

L'abbesse écouta plus attentivement, s'anima peu à peu et sourit. 

«Bon, dit Milady, elle prend goût à mon discours; si elle est cardinaliste, elle n'y met pas de fanatisme au moins.» 

Alors elle passa aux persécutions exercées par le cardinal sur ses ennemis. L'abbesse se contenta de se signer, sans approuver ni désapprouver. 

Cela confirma Milady dans son opinion que la religieuse était plutôt royaliste que cardinaliste. Milady continua, renchérissant de plus en plus. 

«Je suis fort ignorante de toutes ces matières-là, dit enfin l'abbesse, mais tout éloignées que nous sommes de la cour, tout en dehors des intérêts du monde où nous nous trouvons placées, nous avons des exemples fort tristes de ce que vous nous racontez là; et l'une de nos pensionnaires a bien souffert des vengeances et des persécutions de M. le cardinal. 

\speak  Une de vos pensionnaires, dit Milady; oh! mon Dieu! pauvre femme, je la plains alors. 

\speak  Et vous avez raison, car elle est bien à plaindre: prison, menaces, mauvais traitements, elle a tout souffert. Mais, après tout, reprit l'abbesse, M. le cardinal avait peut-être des motifs plausibles pour agir ainsi, et quoiqu'elle ait l'air d'un ange, il ne faut pas toujours juger les gens sur la mine.» 

«Bon! dit Milady à elle-même, qui sait! je vais peut-être découvrir quelque chose ici, je suis en veine.» 

Et elle s'appliqua à donner à son visage une expression de candeur parfaite. 

«Hélas! dit Milady, je le sais; on dit cela, qu'il ne faut pas croire aux physionomies; mais à quoi croira-t-on cependant, si ce n'est au plus bel ouvrage du Seigneur? Quant à moi, je serai trompée toute ma vie peut-être; mais je me fierai toujours à une personne dont le visage m'inspirera de la sympathie. 

\speak  Vous seriez donc tentée de croire, dit l'abbesse, que cette jeune femme est innocente? 

\speak  M. le cardinal ne punit pas que les crimes, dit-elle; il y a certaines vertus qu'il poursuit plus sévèrement que certains forfaits. 

\speak  Permettez-moi, madame, de vous exprimer ma surprise, dit l'abbesse. 

\speak  Et sur quoi? demanda Milady avec naïveté. 

\speak  Mais sur le langage que vous tenez. 

\speak  Que trouvez-vous d'étonnant à ce langage? demanda en souriant Milady. 

\speak  Vous êtes l'amie du cardinal, puisqu'il vous envoie ici, et cependant\dots 

\speak  Et cependant j'en dis du mal, reprit Milady, achevant la pensée de la supérieure. 

\speak  Au moins n'en dites-vous pas de bien. 

\speak  C'est que je ne suis pas son amie, dit-elle en soupirant, mais sa victime. 

\speak  Mais cependant cette lettre par laquelle il vous recommande à moi?\dots 

\speak  Est un ordre à moi de me tenir dans une espèce de prison dont il me fera tirer par quelques-uns de ses satellites. 

\speak  Mais pourquoi n'avez-vous pas fui? 

\speak  Où irais-je? croyez-vous qu'il y ait un endroit de la terre où ne puisse atteindre le cardinal, s'il veut se donner la peine de tendre la main? Si j'étais un homme, à la rigueur cela serait possible encore; mais une femme, que voulez-vous que fasse une femme? Cette jeune pensionnaire que vous avez ici a-t-elle essayé de fuir, elle? 

\speak  Non, c'est vrai; mais elle, c'est autre chose, je la crois retenue en France par quelque amour. 

\speak  Alors, dit Milady avec un soupir, si elle aime, elle n'est pas tout à fait malheureuse. 

\speak  Ainsi, dit l'abbesse en regardant Milady avec un intérêt croissant, c'est encore une pauvre persécutée que je vois? 

\speak  Hélas, oui, dit Milady. 

L'abbesse regarda un instant Milady avec inquiétude, comme si une nouvelle pensée surgissait dans son esprit. 

«Vous n'êtes pas ennemie de notre sainte foi? dit-elle en balbutiant. 

\speak  Moi, s'écria Milady, moi, protestante! Oh! non, j'atteste le Dieu qui nous entend que je suis au contraire fervente catholique. 

\speak  Alors, madame, dit l'abbesse en souriant, rassurez-vous; la maison où vous êtes ne sera pas une prison bien dure, et nous ferons tout ce qu'il faudra pour vous faire chérir la captivité. Il y a plus, vous trouverez ici cette jeune femme persécutée sans doute par suite de quelque intrigue de cour. Elle est aimable, gracieuse. 

\speak  Comment la nommez-vous? 

\speak  Elle m'a été recommandée par quelqu'un de très haut placé, sous le nom de Ketty. Je n'ai pas cherché à savoir son autre nom. 

\speak  Ketty! s'écria Milady; quoi! vous êtes sûre?\dots 

\speak  Qu'elle se fait appeler ainsi? Oui, madame, la connaîtriez-vous?» 

Milady sourit à elle-même et à l'idée qui lui était venue que cette jeune femme pouvait être son ancienne camérière. Il se mêlait au souvenir de cette jeune fille un souvenir de colère, et un désir de vengeance avait bouleversé les traits de Milady, qui reprirent au reste presque aussitôt l'expression calme et bienveillante que cette femme aux cent visages leur avait momentanément fait perdre. 

«Et quand pourrai-je voir cette jeune dame, pour laquelle je me sens déjà une si grande sympathie? demanda Milady. 

\speak  Mais, ce soir, dit l'abbesse, dans la journée même. Mais vous voyagez depuis quatre jours, m'avez-vous dit vous-même; ce matin vous vous êtes levée à cinq heures, vous devez avoir besoin de repos. Couchez-vous et dormez, à l'heure du dîner nous vous réveillerons.» 

Quoique Milady eût très bien pu se passer de sommeil, soutenue qu'elle était par toutes les excitations qu'une aventure nouvelle faisait éprouver à son cœur avide d'intrigues, elle n'en accepta pas moins l'offre de la supérieure: depuis douze ou quinze jours elle avait passé par tant d'émotions diverses que, si son corps de fer pouvait encore soutenir la fatigue, son âme avait besoin de repos. 

Elle prit donc congé de l'abbesse et se coucha, doucement bercée par les idées de vengeance auxquelles l'avait tout naturellement ramenée le nom de Ketty. Elle se rappelait cette promesse presque illimitée que lui avait faite le cardinal, si elle réussissait dans son entreprise. Elle avait réussi, elle pourrait donc se venger de d'Artagnan. 

Une seule chose épouvantait Milady, c'était le souvenir de son mari! le comte de La Fère, qu'elle avait cru mort ou du moins expatrié, et qu'elle retrouvait dans Athos, le meilleur ami de d'Artagnan. 

Mais aussi, s'il était l'ami de d'Artagnan, il avait dû lui prêter assistance dans toutes les menées à l'aide desquelles la reine avait déjoué les projets de Son Éminence; s'il était l'ami de d'Artagnan, il était l'ennemi du cardinal; et sans doute elle parviendrait à l'envelopper dans la vengeance aux replis de laquelle elle comptait étouffer le jeune mousquetaire. 

Toutes ces espérances étaient de douces pensées pour Milady; aussi, bercée par elles, s'endormit-elle bientôt. 

Elle fut réveillée par une voix douce qui retentit au pied de son lit. Elle ouvrit les yeux, et vit l'abbesse accompagnée d'une jeune femme aux cheveux blonds, au teint délicat, qui fixait sur elle un regard plein d'une bienveillante curiosité. 

La figure de cette jeune femme lui était complètement inconnue; toutes deux s'examinèrent avec une scrupuleuse attention, tout en échangeant les compliments d'usage: toutes deux étaient fort belles, mais de beautés tout à fait différentes. Cependant Milady sourit en reconnaissant qu'elle l'emportait de beaucoup sur la jeune femme en grand air et en façons aristocratiques. Il est vrai que l'habit de novice que portait la jeune femme n'était pas très avantageux pour soutenir une lutte de ce genre. 

L'abbesse les présenta l'une à l'autre; puis, lorsque cette formalité fut remplie, comme ses devoirs l'appelaient à l'église, elle laissa les deux jeunes femmes seules. 

La novice, voyant Milady couchée, voulait suivre la supérieure, mais Milady la retint. 

«Comment, madame, lui dit-elle, à peine vous ai-je aperçue et vous voulez déjà me priver de votre présence, sur laquelle je comptais cependant un peu, je vous l'avoue, pour le temps que j'ai à passer ici? 

\speak  Non, madame, répondit la novice, seulement je craignais d'avoir mal choisi mon temps: vous dormiez, vous êtes fatiguée. 

\speak  Eh bien, dit Milady, que peuvent demander les gens qui dorment? un bon réveil. Ce réveil, vous me l'avez donné; laissez-moi en jouir tout à mon aise.» 

Et lui prenant la main, elle l'attira sur un fauteuil qui était près de son lit. 

La novice s'assit. 

«Mon Dieu! dit-elle, que je suis malheureuse! voilà six mois que je suis ici, sans l'ombre d'une distraction, vous arrivez, votre présence allait être pour moi une compagnie charmante, et voilà que, selon toute probabilité, d'un moment à l'autre je vais quitter le couvent! 

\speak  Comment! dit Milady, vous sortez bientôt? 

\speak  Du moins je l'espère, dit la novice avec une expression de joie qu'elle ne cherchait pas le moins du monde à déguiser. 

\speak  Je crois avoir appris que vous aviez souffert de la part du cardinal, continua Milady; c'eût été un motif de plus de sympathie entre nous. 

\speak  Ce que m'a dit notre bonne mère est donc la vérité, que vous étiez aussi une victime de ce méchant cardinal? 

\speak  Chut! dit Milady, même ici ne parlons pas ainsi de lui; tous mes malheurs viennent d'avoir dit à peu près ce que vous venez de dire, devant une femme que je croyais mon amie et qui m'a trahie. Et vous êtes aussi, vous, la victime d'une trahison? 

\speak  Non, dit la novice, mais de mon dévouement à une femme que j'aimais, pour qui j'eusse donné ma vie, pour qui je la donnerais encore. 

\speak  Et qui vous a abandonnée, c'est cela! 

\speak  J'ai été assez injuste pour le croire, mais depuis deux ou trois jours j'ai acquis la preuve du contraire, et j'en remercie Dieu; il m'aurait coûté de croire qu'elle m'avait oubliée. Mais vous, madame, continua la novice, il me semble que vous êtes libre, et que si vous vouliez fuir, il ne tiendrait qu'à vous. 

\speak  Où voulez-vous que j'aille, sans amis, sans argent, dans une partie de la France que je ne connais pas, où je ne suis jamais venue?\dots 

\speak  Oh! s'écria la novice, quant à des amis, vous en aurez partout où vous vous montrerez, vous paraissez si bonne et vous êtes si belle! 

\speak  Cela n'empêche pas, reprit Milady en adoucissant son sourire de manière à lui donner une expression angélique, que je suis seule et persécutée. 

\speak  Écoutez, dit la novice, il faut avoir bon espoir dans le Ciel, voyez-vous; il vient toujours un moment où le bien que l'on a fait plaide votre cause devant Dieu, et, tenez, peut-être est-ce un bonheur pour vous, tout humble et sans pouvoir que je suis, que vous m'ayez rencontrée: car, si je sors d'ici, eh bien, j'aurai quelques amis puissants, qui, après s'être mis en campagne pour moi, pourront aussi se mettre en campagne pour vous. 

\speak  Oh! quand j'ai dit que j'étais seule, dit Milady, espérant faire parler la novice en parlant d'elle-même, ce n'est pas faute d'avoir aussi quelques connaissances haut placées; mais ces connaissances tremblent elles-mêmes devant le cardinal: la reine elle-même n'ose pas soutenir contre le terrible ministre; j'ai la preuve que Sa Majesté, malgré son excellent cœur, a plus d'une fois été obligée d'abandonner à la colère de Son Éminence les personnes qui l'avaient servie. 

\speak  Croyez-moi, madame, la reine peut avoir l'air d'avoir abandonné ces personnes-là; mais il ne faut pas en croire l'apparence: plus elles sont persécutées, plus elle pense à elles, et souvent, au moment où elles y pensent le moins, elles ont la preuve d'un bon souvenir. 

\speak  Hélas! dit Milady, je le crois: la reine est si bonne. 

\speak  Oh! vous la connaissez donc, cette belle et noble reine, que vous parlez d'elle ainsi! s'écria la novice avec enthousiasme. 

\speak  C'est-à-dire, reprit Milady, poussée dans ses retranchements, qu'elle, personnellement, je n'ai pas l'honneur de la connaître; mais je connais bon nombre de ses amis les plus intimes: je connais M. de Putange; j'ai connu en Angleterre M. Dujart; je connais M. de Tréville. 

\speak  M. de Tréville! s'écria la novice, vous connaissez M. de Tréville? 

\speak  Oui, parfaitement, beaucoup même. 

\speak  Le capitaine des mousquetaires du roi? 

\speak  Le capitaine des mousquetaires du roi. 

\speak  Oh! mais vous allez voir, s'écria la novice, que tout à l'heure nous allons être des connaissances achevées, presque des amies; si vous connaissez M. de Tréville, vous avez dû aller chez lui? 

\speak  Souvent! dit Milady, qui, entrée dans cette voie, et s'apercevant que le mensonge réussissait, voulait le pousser jusqu'au bout. 

\speak  Chez lui, vous avez dû voir quelques-uns de ses mousquetaires? 

\speak  Tous ceux qu'il reçoit habituellement! répondit Milady, pour laquelle cette conversation commençait à prendre un intérêt réel. 

\speak  Nommez-moi quelques-uns de ceux que vous connaissez, et vous verrez qu'ils seront de mes amis. 

\speak  Mais, dit Milady embarrassée, je connais M. de Louvigny, M. de Courtivron, M. de Férussac.» 

La novice la laissa dire; puis, voyant qu'elle s'arrêtait: 

«Vous ne connaissez pas, dit-elle, un gentilhomme nommé Athos?» 

Milady devint aussi pâle que les draps dans lesquels elle était couchée, et, si maîtresse qu'elle fût d'elle-même, ne put s'empêcher de pousser un cri en saisissant la main de son interlocutrice et en la dévorant du regard. 

«Quoi! qu'avez-vous? Oh! mon Dieu! demanda cette pauvre femme, ai-je donc dit quelque chose qui vous ait blessée? 

\speak  Non, mais ce nom m'a frappée, parce que, moi aussi j'ai connu ce gentilhomme, et qu'il me paraît étrange de trouver quelqu'un qui le connaisse beaucoup. 

\speak  Oh! oui! beaucoup! beaucoup! non seulement lui, mais encore ses amis: MM. Porthos et Aramis! 

\speak  En vérité! eux aussi je les connais! s'écria Milady, qui sentit le froid pénétrer jusqu'à son cœur. 

\speak  Eh bien, si vous les connaissez, vous devez savoir qu'ils sont bons et francs compagnons; que ne vous adressez-vous à eux, si vous avez besoin d'appui? 

\speak  C'est-à-dire, balbutia Milady, je ne suis liée réellement avec aucun d'eux; je les connais pour en avoir beaucoup entendu parler par un de leurs amis, M. d'Artagnan. 

\speak  Vous connaissez M. d'Artagnan!» s'écria la novice à son tour, en saisissant la main de Milady et en la dévorant des yeux. 

Puis, remarquant l'étrange expression du regard de Milady: 

«Pardon, madame, dit-elle, vous le connaissez, à quel titre? 

\speak  Mais, reprit Milady embarrassée, mais à titre d'ami. 

\speak  Vous me trompez, madame, dit la novice; vous avez été sa maîtresse. 

\speak  C'est vous qui l'avez été, madame, s'écria Milady à son tour. 

\speak  Moi! dit la novice. 

\speak  Oui, vous; je vous connais maintenant: vous êtes madame Bonacieux.» 

La jeune femme se recula, pleine de surprise et de terreur. 

«Oh! ne niez pas! répondez, reprit Milady. 

\speak  Eh bien, oui, madame! je l'aime, dit la novice; sommes-nous rivales?» 

La figure de Milady s'illumina d'un feu tellement sauvage que, dans toute autre circonstance, Mme Bonacieux se fût enfuie d'épouvante; mais elle était toute à sa jalousie. 

«Voyons, dites, madame, reprit Mme Bonacieux avec une énergie dont on l'eût crue incapable, avez-vous été ou êtes-vous sa maîtresse? 

\speak  Oh! non! s'écria Milady avec un accent qui n'admettait pas le doute sur sa vérité, jamais! jamais! 

\speak  Je vous crois, dit Mme Bonacieux; mais pourquoi donc alors vous êtes-vous écriée ainsi? 

\speak  Comment, vous ne comprenez pas! dit Milady, qui était déjà remise de son trouble, et qui avait retrouvé toute sa présence d'esprit. 

\speak  Comment voulez-vous que je comprenne? je ne sais rien. 

\speak  Vous ne comprenez pas que M. d'Artagnan étant mon ami, il m'avait prise pour confidente? 

\speak  Vraiment! 

\speak  Vous ne comprenez pas que je sais tout, votre enlèvement de la petite maison de Saint-Germain, son désespoir, celui de ses amis, leurs recherches inutiles depuis ce moment! Et comment ne voulez-vous pas que je m'en étonne, quand, sans m'en douter, je me trouve en face de vous, de vous dont nous avons parlé si souvent ensemble, de vous qu'il aime de toute la force de son âme, de vous qu'il m'avait fait aimer avant que je vous eusse vue? Ah! chère Constance, je vous trouve donc, je vous vois donc enfin!» 

Et Milady tendit ses bras à Mme Bonacieux, qui, convaincue par ce qu'elle venait de lui dire, ne vit plus dans cette femme, qu'un instant auparavant elle avait crue sa rivale, qu'une amie sincère et dévouée. 

«Oh! pardonnez-moi! pardonnez-moi! s'écria-t-elle en se laissant aller sur son épaule, je l'aime tant!» 

Ces deux femmes se tinrent un instant embrassées. Certes, si les forces de Milady eussent été à la hauteur de sa haine, Mme Bonacieux ne fût sortie que morte de cet embrassement. Mais, ne pouvant pas l'étouffer, elle lui sourit. 

«O chère belle! chère bonne petite! dit Milady, que je suis heureuse de vous voir! Laissez-moi vous regarder. Et, en disant ces mots, elle la dévorait effectivement du regard. Oui, c'est bien vous. Ah! d'après ce qu'il m'a dit, je vous reconnais à cette heure, je vous reconnais parfaitement.» 

La pauvre jeune femme ne pouvait se douter de ce qui se passait d'affreusement cruel derrière le rempart de ce front pur, derrière ces yeux si brillants où elle ne lisait que de l'intérêt et de la compassion. 

«Alors vous savez ce que j'ai souffert, dit Mme Bonacieux, puisqu'il vous a dit ce qu'il souffrait; mais souffrir pour lui, c'est du bonheur.» 

Milady reprit machinalement: 

«Oui, c'est du bonheur.» 

Elle pensait à autre chose. 

«Et puis, continua Mme Bonacieux, mon supplice touche à son terme; demain, ce soir peut-être, je le reverrai, et alors le passé n'existera plus. 

\speak  Ce soir? demain? s'écria Milady tirée de sa rêverie par ces paroles, que voulez-vous dire? attendez-vous quelque nouvelle de lui? 

\speak  Je l'attends lui-même. 

\speak  Lui-même; d'Artagnan, ici! 

\speak  Lui-même. 

\speak  Mais, c'est impossible! il est au siège de La Rochelle avec le cardinal; il ne reviendra à Paris qu'après la prise de la ville. 

\speak  Vous le croyez ainsi, mais est-ce qu'il y a quelque chose d'impossible à mon d'Artagnan, le noble et loyal gentilhomme! 

\speak  Oh! je ne puis vous croire! 

\speak  Eh bien, lisez donc!» dit, dans l'excès de son orgueil et de sa joie, la malheureuse jeune femme en présentant une lettre à Milady. 

«L'écriture de Mme de Chevreuse! se dit en elle-même Milady. Ah! j'étais bien sûre qu'ils avaient des intelligences de ce côté-là!» 

Et elle lut avidement ces quelques lignes:
\begin{quotation}
	«Ma chère enfant, tenez-vous prête; notre ami vous verra bientôt, et il ne vous verra que pour vous arracher de la prison où votre sûreté exigeait que vous fussiez cachée: préparez-vous donc au départ et ne désespérez jamais de nous.

«Notre charmant Gascon vient de se montrer brave et fidèle comme toujours, dites-lui qu'on lui est bien reconnaissant quelque part de l'avis qu'il a donné.» 
\end{quotation}

«Oui, oui, dit Milady, oui, la lettre est précise. Savez-vous quel est cet avis? 

\speak  Non. Je me doute seulement qu'il aura prévenu la reine de quelque nouvelle machination du cardinal. 

\speak  Oui, c'est cela sans doute!» dit Milady en rendant la lettre à Mme Bonacieux et en laissant retomber sa tête pensive sur sa poitrine. 

En ce moment on entendit le galop d'un cheval. 

«Oh! s'écria Mme Bonacieux en s'élançant à la fenêtre, serait-ce déjà lui?» 

Milady était restée dans son lit, pétrifiée par la surprise; tant de choses inattendues lui arrivaient tout à coup, que pour la première fois la tête lui manquait. 

«Lui! lui! murmura-t-elle, serait-ce lui?» 

Et elle demeurait dans son lit les yeux fixes. 

«Hélas, non! dit Mme Bonacieux, c'est un homme que je ne connais pas, et qui cependant a l'air de venir ici; oui, il ralentit sa course, il s'arrête à la porte, il sonne. 

Milady sauta hors de son lit. 

«Vous êtes bien sûre que ce n'est pas lui? dit-elle. 

\speak  Oh! oui, bien sûre! 

\speak  Vous avez peut-être mal vu. 

\speak  Oh! je verrais la plume de son feutre, le bout de son manteau, que je le reconnaîtrais, lui! 

Milady s'habillait toujours. 

«N'importe! cet homme vient ici, dites-vous? 

\speak  Oui, il est entré. 

\speak  C'est ou pour vous ou pour moi. 

\speak  Oh! mon Dieu, comme vous semblez agitée! 

\speak  Oui, je l'avoue, je n'ai pas votre confiance, je crains tout du cardinal. 

\speak  Chut! dit Mme Bonacieux, on vient!» 

Effectivement, la porte s'ouvrit, et la supérieure entra. 

«Est-ce vous qui arrivez de Boulogne? demanda-t-elle à Milady. 

\speak  Oui, c'est moi, répondit celle-ci, et, tâchant de ressaisir son sang-froid, qui me demande? 

\speak  Un homme qui ne veut pas dire son nom, mais qui vient de la part du cardinal. 

\speak  Et qui veut me parler? demanda Milady. 

\speak  Qui veut parler à une dame arrivant de Boulogne. 

\speak  Alors faites entrer, madame, je vous prie. 

\speak  Oh! mon Dieu! mon Dieu! dit Mme Bonacieux, serait-ce quelque mauvaise nouvelle? 

\speak  J'en ai peur. 

\speak  Je vous laisse avec cet étranger, mais aussitôt son départ, si vous le permettez, je reviendrai. 

\speak  Comment donc! je vous en prie.» 

La supérieure et Mme Bonacieux sortirent. 

Milady resta seule, les yeux fixés sur la porte; un instant après on entendit le bruit d'éperons qui retentissaient sur les escaliers, puis les pas se rapprochèrent, puis la porte s'ouvrit, et un homme parut. 

Milady jeta un cri de joie: cet homme c'était le comte de Rochefort, l'âme damnée de Son Éminence.
%!TeX root=../musketeersfr.tex 

\chapter{Deux Variétés De Démons} 
	
\lettrine[ante=«]{A}{h!} s'écrièrent ensemble Rochefort et Milady, c'est vous! 

\zz
\noindent— Oui, c'est moi. 

\zz
\noindent— Et vous arrivez\dots? demanda Milady. 

\zz
\noindent— De La Rochelle, et vous? 

\speak  D'Angleterre. 

\speak  Buckingham? 

\speak  Mort ou blessé dangereusement; comme je partais sans avoir rien pu obtenir de lui, un fanatique venait de l'assassiner. 

\speak  Ah! fit Rochefort avec un sourire, voilà un hasard bien heureux! et qui satisfera Son Éminence! L'avez-vous prévenue? 

\speak  Je lui ai écrit de Boulogne. Mais comment êtes-vous ici? 

\speak  Son Éminence, inquiète, m'a envoyé à votre recherche. 

\speak  Je suis arrivée d'hier seulement. 

\speak  Et qu'avez-vous fait depuis hier? 

\speak  Je n'ai pas perdu mon temps. 

\speak  Oh! je m'en doute bien! 

\speak  Savez-vous qui j'ai rencontré ici? 

\speak  Non. 

\speak  Devinez. 

\speak  Comment voulez-vous?\dots 

\speak  Cette jeune femme que la reine a tirée de prison. 

\speak  La maîtresse du petit d'Artagnan? 

\speak  Oui, Mme Bonacieux, dont le cardinal ignorait la retraite. 

\speak  Eh bien, dit Rochefort, voilà encore un hasard qui peut aller de pair avec l'autre, M. le cardinal est en vérité un homme privilégié. 

\speak  Comprenez-vous mon étonnement, continua Milady, quand je me suis trouvée face à face avec cette femme? 

\speak  Vous connaît-elle? 

\speak  Non. 

\speak  Alors elle vous regarde comme une étrangère?» 

Milady sourit. 

«Je suis sa meilleure amie! 

\speak  Sur mon honneur, dit Rochefort, il n'y a que vous, ma chère comtesse, pour faire de ces miracles-là. 

\speak  Et bien m'en a pris, chevalier, dit Milady, car savez-vous ce qui se passe? 

\speak  Non. 

\speak  On va la venir chercher demain ou après-demain avec un ordre de la reine. 

\speak  Vraiment? et qui cela? 

\speak  D'Artagnan et ses amis. 

\speak  En vérité ils en feront tant, que nous serons obligés de les envoyer à la Bastille. 

\speak  Pourquoi n'est-ce point déjà fait? 

\speak  Que voulez-vous! parce que M. le cardinal a pour ces hommes une faiblesse que je ne comprends pas. 

\speak  Vraiment? 

\speak  Oui. 

\speak  Eh bien, dites-lui ceci, Rochefort: dites-lui que notre conversation à l'auberge du Colombier-Rouge a été entendue par ces quatre hommes; dites-lui qu'après son départ l'un d'eux est monté et m'a arraché par violence le sauf-conduit qu'il m'avait donné; dites-lui qu'ils avaient fait prévenir Lord de Winter de mon passage en Angleterre; que, cette fois encore, ils ont failli faire échouer ma mission, comme ils ont fait échouer celle des ferrets; dites-lui que parmi ces quatre hommes, deux seulement sont à craindre, d'Artagnan et Athos; dites-lui que le troisième, Aramis, est l'amant de Mme de Chevreuse: il faut laisser vivre celui-là, on sait son secret, il peut être utile; quant au quatrième, Porthos, c'est un sot, un fat et un niais, qu'il ne s'en occupe même pas. 

\speak  Mais ces quatre hommes doivent être à cette heure au siège de La Rochelle. 

\speak  Je le croyais comme vous; mais une lettre que Mme Bonacieux a reçue de Mme de Chevreuse, et qu'elle a eu l'imprudence de me communiquer, me porte à croire que ces quatre hommes au contraire sont en campagne pour la venir enlever. 

\speak  Diable! comment faire? 

\speak  Que vous a dit le cardinal à mon égard? 

\speak  De prendre vos dépêches écrites ou verbales, de revenir en poste, et, quand il saura ce que vous avez fait, il avisera à ce que vous devez faire. 

\speak  Je dois donc rester ici? demanda Milady. 

\speak  Ici ou dans les environs. 

\speak  Vous ne pouvez m'emmener avec vous? 

\speak  Non, l'ordre est formel: aux environs du camp, vous pourriez être reconnue, et votre présence, vous le comprenez, compromettrait Son Éminence, surtout après ce qui vient de se passer là-bas. Seulement, dites-moi d'avance où vous attendrez des nouvelles du cardinal, que je sache toujours où vous retrouver. 

\speak  Écoutez, il est probable que je ne pourrai rester ici. 

\speak  Pourquoi? 

\speak  Vous oubliez que mes ennemis peuvent arriver d'un moment à l'autre. 

\speak  C'est vrai; mais alors cette petite femme va échapper à Son Éminence? 

\speak  Bah! dit Milady avec un sourire qui n'appartenait qu'à elle, vous oubliez que je suis sa meilleure amie. 

\speak  Ah! c'est vrai! je puis donc dire au cardinal, à l'endroit de cette femme\dots 

\speak  Qu'il soit tranquille. 

\speak  Voilà tout? 

\speak  Il saura ce que cela veut dire. 

\speak  Il le devinera. Maintenant, voyons, que dois-je faire? 

\speak  Repartir à l'instant même; il me semble que les nouvelles que vous reportez valent bien la peine que l'on fasse diligence. 

\speak  Ma chaise s'est cassée en entrant à Lillers. 

\speak  À merveille! 

\speak  Comment, à merveille? 

\speak  Oui, j'ai besoin de votre chaise, moi, dit la comtesse. 

\speak  Et comment partirai-je, alors? 

\speak  À franc étrier. 

\speak  Vous en parlez bien à votre aise, cent quatre-vingts lieues. 

\speak  Qu'est-ce que cela? 

\speak  On les fera. Après? 

\speak  Après: en passant à Lillers, vous me renvoyez la chaise avec ordre à votre domestique de se mettre à ma disposition. 

\speak  Bien. 

\speak  Vous avez sans doute sur vous quelque ordre du cardinal? 

\speak  J'ai mon plein pouvoir. 

\speak  Vous le montrez à l'abbesse, et vous dites qu'on viendra me chercher, soit aujourd'hui, soit demain, et que j'aurai à suivre la personne qui se présentera en votre nom. 

\speak  Très bien! 

\speak  N'oubliez pas de me traiter durement en parlant de moi à l'abbesse. 

\speak  À quoi bon? 

\speak  Je suis une victime du cardinal. Il faut bien que j'inspire de la confiance à cette pauvre petite Mme Bonacieux. 

\speak  C'est juste. Maintenant voulez-vous me faire un rapport de tout ce qui est arrivé? 

\speak  Mais je vous ai raconté les événements, vous avez bonne mémoire, répétez les choses comme je vous les ai dites, un papier se perd. 

\speak  Vous avez raison; seulement que je sache où vous retrouver, que je n'aille pas courir inutilement dans les environs. 

\speak  C'est juste, attendez. 

\speak  Voulez-vous une carte? 

\speak  Oh! je connais ce pays à merveille. 

\speak  Vous? quand donc y êtes-vous venue? 

\speak  J'y ai été élevée. 

\speak  Vraiment? 

\speak  C'est bon à quelque chose, vous le voyez, que d'avoir été élevée quelque part. 

\speak  Vous m'attendrez donc\dots? 

\speak  Laissez-moi réfléchir un instant; eh! tenez, à Armentières. 

\speak  Qu'est-ce que cela, Armentières? 

\speak  Une petite ville sur la Lys! je n'aurai qu'à traverser la rivière et je suis en pays étranger. 

\speak  À merveille! mais il est bien entendu que vous ne traverserez la rivière qu'en cas de danger. 

\speak  C'est bien entendu. 

\speak  Et, dans ce cas, comment saurai-je où vous êtes? 

\speak  Vous n'avez pas besoin de votre laquais? 

\speak  Non. 

\speak  C'est un homme sûr? 

\speak  À l'épreuve. 

\speak  Donnez-le-moi; personne ne le connaît, je le laisse à l'endroit que je quitte, et il vous conduit où je suis. 

\speak  Et vous dites que vous m'attendez à Argentières? 

\speak  À Armentières, répondit Milady. 

\speak  Écrivez-moi ce nom-là sur un morceau de papier, de peur que je l'oublie; ce n'est pas compromettant, un nom de ville, n'est-ce pas? 

\speak  Eh, qui sait? N'importe, dit Milady en écrivant le nom sur une demi-feuille de papier, je me compromets. 

\speak  Bien! dit Rochefort en prenant des mains de Milady le papier, qu'il plia et qu'il enfonça dans la coiffe de son feutre; d'ailleurs, soyez tranquille, je vais faire comme les enfants, et, dans le cas où je perdrais ce papier, répéter le nom tout le long de la route. Maintenant est-ce tout? 

\speak  Je le crois. 

\speak  Cherchons bien: Buckingham mort ou grièvement blessé; votre entretien avec le cardinal entendu des quatre mousquetaires; Lord de Winter prévenu de votre arrivée à Portsmouth; d'Artagnan et Athos à la Bastille; Aramis l'amant de Mme de Chevreuse; Porthos un fat; Mme Bonacieux retrouvée; vous envoyer la chaise le plus tôt possible; mettre mon laquais à votre disposition; faire de vous une victime du cardinal, pour que l'abbesse ne prenne aucun soupçon; Armentières sur les bords de la Lys. Est-ce cela? 

\speak  En vérité, mon cher chevalier, vous êtes un miracle de mémoire. À propos, ajoutez une chose\dots 

\speak  Laquelle? 

\speak  J'ai vu de très jolis bois qui doivent toucher au jardin du couvent, dites qu'il m'est permis de me promener dans ces bois; qui sait? j'aurai peut-être besoin de sortir par une porte de derrière. 

\speak  Vous pensez à tout. 

\speak  Et vous, vous oubliez une chose\dots 

\speak  Laquelle? 

\speak  C'est de me demander si j'ai besoin d'argent. 

\speak  C'est juste, combien voulez-vous? 

\speak  Tout ce que vous aurez d'or. 

\speak  J'ai cinq cents pistoles à peu près. 

\speak  J'en ai autant: avec mille pistoles on fait face à tout; videz vos poches. 

\speak  Voilà, comtesse. 

\speak  Bien, mon cher comte! et vous partez\dots? 

\speak  Dans une heure; le temps de manger un morceau, pendant lequel j'enverrai chercher un cheval de poste. 

\speak  À merveille! Adieu, chevalier! 

\speak  Adieu, comtesse! 

\speak  Recommandez-moi au cardinal, dit Milady. 

\speak  Recommandez-moi à Satan», répliqua Rochefort. 

Milady et Rochefort échangèrent un sourire et se séparèrent. 

Une heure après, Rochefort partit au grand galop de son cheval; cinq heures après il passait à Arras. 

Nos lecteurs savent déjà comment il avait été reconnu par d'Artagnan, et comment cette reconnaissance, en inspirant des craintes aux quatre mousquetaires, avait donné une nouvelle activité à leur voyage. 

%!TeX root=../musketeersfr.tex 

\chapter{Une Goutte D'Eau} 

\lettrine{\accentletter[\gravebox]{A}}{} peine Rochefort fut-il sorti, que Mme Bonacieux rentra. Elle trouva Milady le visage riant. 

\zz
«Eh bien, dit la jeune femme, ce que vous craigniez est donc arrivé; ce soir ou demain le cardinal vous envoie prendre? 

\speak  Qui vous a dit cela, mon enfant? demanda Milady. 

\speak  Je l'ai entendu de la bouche même du messager. 

\speak  Venez vous asseoir ici près de moi, dit Milady. 

\speak  Me voici. 

\speak  Attendez que je m'assure si personne ne nous écoute. 

\speak  Pourquoi toutes ces précautions? 

\speak  Vous allez le savoir.» 

Milady se leva et alla à la porte, l'ouvrit, regarda dans le corridor, et revint se rasseoir près de Mme Bonacieux. 

«Alors, dit-elle, il a bien joué son rôle. 

\speak  Qui cela? 

\speak  Celui qui s'est présenté à l'abbesse comme l'envoyé du cardinal. 

\speak  C'était donc un rôle qu'il jouait? 

\speak  Oui, mon enfant. 

\speak  Cet homme n'est donc pas\dots 

\speak  Cet homme, dit Milady en baissant la voix, c'est mon frère. 

\speak  Votre frère! s'écria Mme Bonacieux. 

\speak  Eh bien, il n'y a que vous qui sachiez ce secret, mon enfant; si vous le confiez à qui que ce soit au monde, je serai perdue, et vous aussi peut-être. 

\speak  Oh! mon Dieu! 

\speak  Écoutez, voici ce qui se passe: mon frère, qui venait à mon secours pour m'enlever ici de force, s'il le fallait, a rencontré l'émissaire du cardinal qui venait me chercher; il l'a suivi. Arrivé à un endroit du chemin solitaire et écarté, il a mis l'épée à la main en sommant le messager de lui remettre les papiers dont il était porteur; le messager a voulu se défendre, mon frère l'a tué. 

\speak  Oh! fit Mme Bonacieux en frissonnant. 

\speak  C'était le seul moyen, songez-y. Alors mon frère a résolu de substituer la ruse à la force: il a pris les papiers, il s'est présenté ici comme l'émissaire du cardinal lui-même, et dans une heure ou deux, une voiture doit venir me prendre de la part de Son Éminence. 

\speak  Je comprends; cette voiture, c'est votre frère qui vous l'envoie. 

\speak  Justement; mais ce n'est pas tout: cette lettre que vous avez reçue, et que vous croyez de Mme Chevreuse\dots 

\speak  Eh bien? 

\speak  Elle est fausse. 

\speak  Comment cela? 

\speak  Oui, fausse: c'est un piège pour que vous ne fassiez pas de résistance quand on viendra vous chercher. 

\speak  Mais c'est d'Artagnan qui viendra. 

\speak  Détrompez-vous, d'Artagnan et ses amis sont retenus au siège de La Rochelle. 

\speak  Comment savez-vous cela? 

\speak  Mon frère a rencontré des émissaires du cardinal en habits de mousquetaires. On vous aurait appelée à la porte, vous auriez cru avoir affaire à des amis, on vous enlevait et on vous ramenait à Paris. 

\speak  Oh! mon Dieu! ma tête se perd au milieu de ce chaos d'iniquités. Je sens que si cela durait, continua Mme Bonacieux en portant ses mains à son front, je deviendrais folle! 

\speak  Attendez\dots 

\speak  Quoi? 

\speak  J'entends le pas d'un cheval, c'est celui de mon frère qui repart; je veux lui dire un dernier adieu, venez.» 

Milady ouvrit la fenêtre et fit signe à Mme Bonacieux de l'y rejoindre. La jeune femme y alla. 

Rochefort passait au galop. 

«Adieu, frère», s'écria Milady. 

Le chevalier leva la tête, vit les deux jeunes femmes, et, tout courant, fit à Milady un signe amical de la main. 

«Ce bon Georges!» dit-elle en refermant la fenêtre avec une expression de visage pleine d'affection et de mélancolie. 

Et elle revint s'asseoir à sa place, comme si elle eût été plongée dans des réflexions toutes personnelles. 

«Chère dame! dit Mme Bonacieux, pardon de vous interrompre! mais que me conseillez-vous de faire? mon Dieu! Vous avez plus d'expérience que moi, parlez, je vous écoute. 

\speak  D'abord, dit Milady, il se peut que je me trompe et que d'Artagnan et ses amis viennent véritablement à votre secours. 

\speak  Oh! c'eût été trop beau! s'écria Mme Bonacieux, et tant de bonheur n'est pas fait pour moi! 

\speak  Alors, vous comprenez; ce serait tout simplement une question de temps, une espèce de course à qui arrivera le premier. Si ce sont vos amis qui l'emportent en rapidité, vous êtes sauvée; si ce sont les satellites du cardinal, vous êtes perdue. 

\speak  Oh! oui, oui, perdue sans miséricorde! Que faire donc? que faire? 

\speak  Il y aurait un moyen bien simple, bien naturel\dots 

\speak  Lequel, dites? 

\speak  Ce serait d'attendre, cachée dans les environs, et de s'assurer ainsi quels sont les hommes qui viendront vous demander. 

\speak  Mais où attendre? 

\speak  Oh! ceci n'est point une question: moi-même je m'arrête et je me cache à quelques lieues d'ici en attendant que mon frère vienne me rejoindre; eh bien, je vous emmène avec moi, nous nous cachons et nous attendons ensemble. 

\speak  Mais on ne me laissera pas partir, je suis ici presque prisonnière. 

\speak  Comme on croit que je pars sur un ordre du cardinal, on ne vous croira pas très pressée de me suivre. 

\speak  Eh bien? 

\speak  Eh bien, la voiture est à la porte, vous me dites adieu, vous montez sur le marchepied pour me serrer dans vos bras une dernière fois; le domestique de mon frère qui vient me prendre est prévenu, il fait un signe au postillon, et nous partons au galop. 

\speak  Mais d'Artagnan, d'Artagnan, s'il vient? 

\speak  Ne le saurons-nous pas? 

\speak  Comment? 

\speak  Rien de plus facile. Nous renvoyons à Béthune ce domestique de mon frère, à qui, je vous l'ai dit, nous pouvons nous fier; il prend un déguisement et se loge en face du couvent: si ce sont les émissaires du cardinal qui viennent, il ne bouge pas; si c'est M. d'Artagnan et ses amis, il les amène où nous sommes. 

\speak  Il les connaît donc? 

\speak  Sans doute, n'a-t-il pas vu M. d'Artagnan chez moi! 

\speak  Oh! oui, oui, vous avez raison; ainsi, tout va bien, tout est pour le mieux; mais ne nous éloignons pas d'ici. 

\speak  À sept ou huit lieues tout au plus, nous nous tenons sur la frontière par exemple, et à la première alerte, nous sortons de France. 

\speak  Et d'ici là, que faire? 

\speak  Attendre. 

\speak  Mais s'ils arrivent? 

\speak  La voiture de mon frère arrivera avant eux. 

\speak  Si je suis loin de vous quand on viendra vous prendre; à dîner ou à souper, par exemple? 

\speak  Faites une chose. 

\speak  Laquelle? 

\speak  Dites à votre bonne supérieure que, pour nous quitter le moins possible, vous lui demanderez la permission de partager mon repas. 

\speak  Le permettra-t-elle? 

\speak  Quel inconvénient y a-t-il à cela? 

\speak  Oh! très bien, de cette façon nous ne nous quitterons pas un instant! 

\speak  Eh bien, descendez chez elle pour lui faire votre demande! je me sens la tête lourde, je vais faire un tour au jardin. 

\speak  Allez, et où vous retrouverai-je? 

\speak  Ici dans une heure. 

\speak  Ici dans une heure; oh! vous êtes bonne et je vous remercie. 

\speak  Comment ne m'intéresserais-je pas à vous? Quand vous ne seriez pas belle et charmante, n'êtes-vous pas l'amie d'un de mes meilleurs amis! 

\speak  Cher d'Artagnan, oh! comme il vous remerciera! 

\speak  Je l'espère bien. Allons! tout est convenu, descendons. 

\speak  Vous allez au jardin? 

\speak  Oui. 

\speak  Suivez ce corridor, un petit escalier vous y conduit. 

\speak  À merveille! merci.» 

Et les deux femmes se quittèrent en échangeant un charmant sourire. 

Milady avait dit la vérité, elle avait la tête lourde; car ses projets mal classés s'y heurtaient comme dans un chaos. Elle avait besoin d'être seule pour mettre un peu d'ordre dans ses pensées. Elle voyait vaguement dans l'avenir; mais il lui fallait un peu de silence et de quiétude pour donner à toutes ses idées, encore confuses, une forme distincte, un plan arrêté. 

Ce qu'il y avait de plus pressé, c'était d'enlever Mme Bonacieux, de la mettre en lieu de sûreté, et là, le cas échéant, de s'en faire un otage. Milady commençait à redouter l'issue de ce duel terrible, où ses ennemis mettaient autant de persévérance qu'elle mettait, elle, d'acharnement. 

D'ailleurs elle sentait, comme on sent venir un orage, que cette issue était proche et ne pouvait manquer d'être terrible. 

Le principal pour elle, comme nous l'avons dit, était donc de tenir Mme Bonacieux entre ses mains. Mme Bonacieux, c'était la vie de d'Artagnan; c'était plus que sa vie, c'était celle de la femme qu'il aimait; c'était, en cas de mauvaise fortune, un moyen de traiter et d'obtenir sûrement de bonnes conditions. 

Or, ce point était arrêté: Mme Bonacieux, sans défiance, la suivait; une fois cachée avec elle à Armentières, il était facile de lui faire croire que d'Artagnan n'était pas venu à Béthune. Dans quinze jours au plus, Rochefort serait de retour; pendant ces quinze jours, d'ailleurs, elle aviserait à ce qu'elle aurait à faire pour se venger des quatre amis. Elle ne s'ennuierait pas, Dieu merci, car elle aurait le plus doux passe-temps que les événements pussent accorder à une femme de son caractère: une bonne vengeance à perfectionner. 

Tout en rêvant, elle jetait les yeux autour d'elle et classait dans sa tête la topographie du jardin. Milady était comme un bon général, qui prévoit tout ensemble la victoire et la défaite, et qui est tout près, selon les chances de la bataille, à marcher en avant ou à battre en retraite. 

Au bout d'une heure, elle entendit une douce voix qui l'appelait; c'était celle de Mme Bonacieux. La bonne abbesse avait naturellement consenti à tout, et, pour commencer, elles allaient souper ensemble. 

En arrivant dans la cour, elles entendirent le bruit d'une voiture qui s'arrêtait a la porte. 

«Entendez-vous? dit-elle. 

\speak  Oui, le roulement d'une voiture. 

\speak  C'est celle que mon frère nous envoie. 

\speak  Oh! mon Dieu! 

\speak  Voyons, du courage!» 

On sonna à la porte du couvent, Milady ne s'était pas trompée. 

«Montez dans votre chambre, dit-elle à Mme Bonacieux, vous avez bien quelques bijoux que vous désirez emporter. 

\speak  J'ai ses lettres, dit-elle. 

\speak  Eh bien, allez les chercher et venez me rejoindre chez moi, nous souperons à la hâte, peut-être voyagerons-nous une partie de la nuit, il faut prendre des forces. 

\speak  Grand Dieu! dit Mme Bonacieux en mettant la main sur sa poitrine, le cœur m'étouffe, je ne puis marcher. 

\speak  Du courage, allons, du courage! pensez que dans un quart d'heure vous êtes sauvée, et songez que ce que vous allez faire, c'est pour lui que vous le faites. 

\speak  Oh! oui, tout pour lui. Vous m'avez rendu mon courage par un seul mot; allez, je vous rejoins.» 

Milady monta vivement chez elle, elle y trouva le laquais de Rochefort, et lui donna ses instructions. 

Il devait attendre à la porte; si par hasard les mousquetaires paraissaient, la voiture partait au galop, faisait le tour du couvent, et allait attendre Milady à un petit village qui était situé de l'autre côté du bois. Dans ce cas, Milady traversait le jardin et gagnait le village à pied; nous l'avons dit déjà, Milady connaissait à merveille cette partie de la France. 

Si les mousquetaires ne paraissaient pas, les choses allaient comme il était convenu: Mme Bonacieux montait dans la voiture sous prétexte de lui dire adieu et Milady enlevait Mme Bonacieux. 

Mme Bonacieux entra, et pour lui ôter tout soupçon si elle en avait, Milady répéta devant elle au laquais toute la dernière partie de ses instructions. 

Milady fit quelques questions sur la voiture: c'était une chaise attelée de trois chevaux, conduite par un postillon; le laquais de Rochefort devait la précéder en courrier. 

C'était à tort que Milady craignait que Mme Bonacieux n'eût des soupçons: la pauvre jeune femme était trop pure pour soupçonner dans une autre femme une telle perfidie; d'ailleurs le nom de la comtesse de Winter, qu'elle avait entendu prononcer par l'abbesse, lui était parfaitement inconnu, et elle ignorait même qu'une femme eût eu une part si grande et si fatale aux malheurs de sa vie. 

«Vous le voyez, dit Milady, lorsque le laquais fut sorti, tout est prêt. L'abbesse ne se doute de rien et croit qu'on me vient chercher de la part du cardinal. Cet homme va donner les derniers ordres; prenez la moindre chose, buvez un doigt de vin et partons. 

\speak  Oui, dit machinalement Mme Bonacieux, oui, partons.» 

Milady lui fit signe de s'asseoir devant elle, lui versa un petit verre de vin d'Espagne et lui servit un blanc de poulet. 

«Voyez, lui dit-elle, si tout ne nous seconde pas: voici la nuit qui vient; au point du jour nous serons arrivées dans notre retraite, et nul ne pourra se douter où nous sommes. Voyons, du courage, prenez quelque chose.» 

Mme Bonacieux mangea machinalement quelques bouchées et trempa ses lèvres dans son verre. 

«Allons donc, allons donc, dit Milady portant le sien à ses lèvres, faites comme moi.» 

Mais au moment où elle l'approchait de sa bouche, sa main resta suspendue: elle venait d'entendre sur la route comme le roulement lointain d'un galop qui allait s'approchant; puis, presque en même temps, il lui sembla entendre des hennissements de chevaux. 

Ce bruit la tira de sa joie comme un bruit d'orage réveille au milieu d'un beau rêve; elle pâlit et courut à la fenêtre, tandis que Mme Bonacieux, se levant toute tremblante, s'appuyait sur sa chaise pour ne point tomber. 

On ne voyait rien encore, seulement on entendait le galop qui allait toujours se rapprochant. 

«Oh! mon Dieu, dit Mme Bonacieux, qu'est-ce que ce bruit? 

\speak  Celui de nos amis ou de nos ennemis, dit Milady avec son sang-froid terrible; restez où vous êtes, je vais vous le dire.» 

Mme Bonacieux demeura debout, muette, immobile et pâle comme une statue. 

Le bruit devenait plus fort, les chevaux ne devaient pas être à plus de cent cinquante pas; si on ne les apercevait point encore, c'est que la route faisait un coude. Toutefois, le bruit devenait si distinct qu'on eût pu compter les chevaux par le bruit saccadé de leurs fers. 

Milady regardait de toute la puissance de son attention; il faisait juste assez clair pour qu'elle pût reconnaître ceux qui venaient. 

Tout à coup, au détour du chemin, elle vit reluire des chapeaux galonnés et flotter des plumes; elle compta deux, puis cinq puis huit cavaliers; l'un d'eux précédait tous les autres de deux longueurs de cheval. 

Milady poussa un rugissement étouffé. Dans celui qui tenait la tête elle reconnut d'Artagnan. 

«Oh! mon Dieu! mon Dieu! s'écria Mme Bonacieux, qu'y a-t-il donc? 

\speak  C'est l'uniforme des gardes de M. le cardinal; pas un instant à perdre! s'écria Milady. Fuyons, fuyons! 

\speak  Oui, oui, fuyons», répéta Mme Bonacieux, mais sans pouvoir faire un pas, clouée qu'elle était à sa place par la terreur. 

On entendit les cavaliers qui passaient sous la fenêtre. 

«Venez donc! mais venez donc! s'écriait Milady en essayant de traîner la jeune femme par le bras. Grâce au jardin, nous pouvons fuir encore, j'ai la clef, mais hâtons-nous, dans cinq minutes il serait trop tard.» 

Mme Bonacieux essaya de marcher, fit deux pas et tomba sur ses genoux. 

Milady essaya de la soulever et de l'emporter, mais elle ne put en venir à bout. 

En ce moment on entendit le roulement de la voiture, qui à la vue des mousquetaires partait au galop. Puis, trois ou quatre coups de feu retentirent. 

«Une dernière fois, voulez-vous venir? s'écria Milady. 

\speak  Oh! mon Dieu! mon Dieu! vous voyez bien que les forces me manquent; vous voyez bien que je ne puis marcher: fuyez seule. 

\speak  Fuir seule! vous laisser ici! non, non, jamais», s'écria Milady. 

Tout à coup, un éclair livide jaillit de ses yeux; d'un bond, éperdue, elle courut à la table, versa dans le verre de Mme Bonacieux le contenu d'un chaton de bague qu'elle ouvrit avec une promptitude singulière. 

C'était un grain rougeâtre qui se fondit aussitôt. 

Puis, prenant le verre d'une main ferme: 

«Buvez, dit-elle, ce vin vous donnera des forces, buvez.» 

Et elle approcha le verre des lèvres de la jeune femme qui but machinalement. 

«Ah! ce n'est pas ainsi que je voulais me venger, dit Milady en reposant avec un sourire infernal le verre sur la table, mais, ma foi! on fait ce qu'on peut.» 

Et elle s'élança hors de l'appartement. 

Mme Bonacieux la regarda fuir, sans pouvoir la suivre; elle était comme ces gens qui rêvent qu'on les poursuit et qui essayent vainement de marcher. 

Quelques minutes se passèrent, un bruit affreux retentissait à la porte; à chaque instant Mme Bonacieux s'attendait à voir reparaître Milady, qui ne reparaissait pas. 

Plusieurs fois, de terreur sans doute, la sueur monta froide à son front brûlant. 

Enfin elle entendit le grincement des grilles qu'on ouvrait, le bruit des bottes et des éperons retentit par les escaliers; il se faisait un grand murmure de voix qui allaient se rapprochant, et au milieu desquelles il lui semblait entendre prononcer son nom. 

Tout à coup elle jeta un grand cri de joie et s'élança vers la porte, elle avait reconnu la voix de d'Artagnan. 

«D'Artagnan! d'Artagnan! s'écria-t-elle, est-ce vous? Par ici, par ici. 

\speak  Constance! Constance! répondit le jeune homme, où êtes-vous? mon Dieu!» 

Au même moment, la porte de la cellule céda au choc plutôt qu'elle ne s'ouvrit; plusieurs hommes se précipitèrent dans la chambre; Mme Bonacieux était tombée dans un fauteuil sans pouvoir faire un mouvement. 

D'Artagnan jeta un pistolet encore fumant qu'il tenait à la main, et tomba à genoux devant sa maîtresse, Athos repassa le sien à sa ceinture; Porthos et Aramis, qui tenaient leurs épées nues, les remirent au fourreau. 

«Oh! d'Artagnan! mon bien-aimé d'Artagnan! tu viens donc enfin, tu ne m'avais pas trompée, c'est bien toi! 

\speak  Oui, oui, Constance! réunis! 

\speak  Oh! \textit{elle} avait beau dire que tu ne viendrais pas, j'espérais sourdement; je n'ai pas voulu fuir; oh! comme j'ai bien fait, comme je suis heureuse!» 

À ce mot, \textit{elle}, Athos, qui s'était assis tranquillement, se leva tout à coup. 

«\textit{Elle!} qui, \textit{elle?} demanda d'Artagnan. 

\speak  Mais ma compagne; celle qui, par amitié pour moi, voulait me soustraire à mes persécuteurs; celle qui, vous prenant pour des gardes du cardinal, vient de s'enfuir. 

\speak  Votre compagne, s'écria d'Artagnan, devenant plus pâle que le voile blanc de sa maîtresse, de quelle compagne voulez-vous donc parler? 

\speak  De celle dont la voiture était à la porte, d'une femme qui se dit votre amie, d'Artagnan; d'une femme à qui vous avez tout raconté. 

\speak  Son nom, son nom! s'écria d'Artagnan; mon Dieu! ne savez-vous donc pas son nom? 

\speak  Si fait, on l'a prononcé devant moi, attendez\dots mais c'est étrange\dots oh! mon Dieu! ma tête se trouble, je n'y vois plus. 

\speak  À moi, mes amis, à moi! ses mains sont glacées, s'écria d'Artagnan, elle se trouve mal; grand Dieu! elle perd connaissance!» 

Tandis que Porthos appelait au secours de toute la puissance de sa voix, Aramis courut à la table pour prendre un verre d'eau; mais il s'arrêta en voyant l'horrible altération du visage d'Athos, qui, debout devant la table, les cheveux hérissés, les yeux glacés de stupeur, regardait l'un des verres et semblait en proie au doute le plus horrible. 

«Oh! disait Athos, oh! non, c'est impossible! Dieu ne permettrait pas un pareil crime. 

\speak  De l'eau, de l'eau, criait d'Artagnan, de l'eau! 

«Pauvre femme, pauvre femme!» murmurait Athos d'une voix brisée. 

Mme Bonacieux rouvrit les yeux sous les baisers de d'Artagnan. 

«Elle revient à elle! s'écria le jeune homme. Oh! mon Dieu, mon Dieu! je te remercie! 

\speak  Madame, dit Athos, madame, au nom du Ciel! à qui ce verre vide? 

\speak  À moi, monsieur\dots, répondit la jeune femme d'une voix mourante. 

\speak  Mais qui vous a versé ce vin qui était dans ce verre? 

\speak  \textit{Elle}. 

\speak  Mais, qui donc, \textit{elle?} 

\speak  Ah! je me souviens, dit Mme Bonacieux, la comtesse de Winter\dots» 

Les quatre amis poussèrent un seul et même cri, mais celui d'Athos domina tous les autres. 

En ce moment, le visage de Mme Bonacieux devint livide, une douleur sourde la terrassa, elle tomba haletante dans les bras de Porthos et d'Aramis. 

D'Artagnan saisit les mains d'Athos avec une angoisse difficile à décrire. 

«Et quoi! dit-il, tu crois\dots» 

Sa voix s'éteignit dans un sanglot. 

«Je crois tout, dit Athos en se mordant les lèvres jusqu'au sang. 

\speak  D'Artagnan, d'Artagnan! s'écria Mme Bonacieux, où es-tu? ne me quitte pas, tu vois bien que je vais mourir.» 

D'Artagnan lâcha les mains d'Athos, qu'il tenait encore entre ses mains crispées, et courut à elle. 

Son visage si beau était tout bouleversé, ses yeux vitreux n'avaient déjà plus de regard, un tremblement convulsif agitait son corps, la sueur coulait sur son front. 

«Au nom du Ciel! courez appeler Porthos, Aramis; demandez du secours! 

\speak  Inutile, dit Athos, inutile, au poison qu'elle verse il n'y a pas de contrepoison. 

\speak  Oui, oui, du secours, du secours! murmura Mme Bonacieux; du secours!» 

Puis, rassemblant toutes ses forces, elle prit la tête du jeune homme entre ses deux mains, le regarda un instant comme si toute son âme était passée dans son regard, et, avec un cri sanglotant, elle appuya ses lèvres sur les siennes. 

«Constance! Constance!» s'écria d'Artagnan. 

Un soupir s'échappa de la bouche de Mme Bonacieux, effleurant celle de d'Artagnan; ce soupir, c'était cette âme si chaste et si aimante qui remontait au ciel. 

D'Artagnan ne serrait plus qu'un cadavre entre ses bras. 

Le jeune homme poussa un cri et tomba près de sa maîtresse, aussi pâle et aussi glacé qu'elle. 

Porthos pleura, Aramis montra le poing au ciel, Athos fit le signe de la croix. 

En ce moment un homme parut sur la porte, presque aussi pâle que ceux qui étaient dans la chambre, et regarda tout autour de lui, vit Mme Bonacieux morte et d'Artagnan évanoui. 

Il apparaissait juste à cet instant de stupeur qui suit les grandes catastrophes. 

«Je ne m'étais pas trompé, dit-il, voilà M. d'Artagnan, et vous êtes ses trois amis, MM. Athos, Porthos et Aramis.» 

Ceux dont les noms venaient d'être prononcés regardaient l'étranger avec étonnement, il leur semblait à tous trois le reconnaître. 

«Messieurs, reprit le nouveau venu, vous êtes comme moi à la recherche d'une femme qui, ajouta-t-il avec un sourire terrible, a dû passer par ici, car j'y vois un cadavre!» 

Les trois amis restèrent muets; seulement la voix comme le visage leur rappelait un homme qu'ils avaient déjà vu; cependant, ils ne pouvaient se souvenir dans quelles circonstances. 

«Messieurs, continua l'étranger, puisque vous ne voulez pas reconnaître un homme qui probablement vous doit la vie deux fois, il faut bien que je me nomme; je suis Lord de Winter, le beau-frère de cette femme.» 

Les trois amis jetèrent un cri de surprise. 

Athos se leva et lui tendit la main. 

«Soyez le bienvenu, Milord, dit-il, vous êtes des nôtres. 

\speak  Je suis parti cinq heures après elle de Portsmouth, dit Lord de Winter, je suis arrivé trois heures après elle à Boulogne, je l'ai manquée de vingt minutes à Saint-Omer; enfin, à Lillers, j'ai perdu sa trace. J'allais au hasard, m'informant à tout le monde, quand je vous ai vus passer au galop; j'ai reconnu M. d'Artagnan. Je vous ai appelés, vous ne m'avez pas répondu; j'ai voulu vous suivre, mais mon cheval était trop fatigué pour aller du même train que les vôtres. Et cependant il paraît que malgré la diligence que vous avez faite, vous êtes encore arrivés trop tard! 

\speak  Vous voyez, dit Athos en montrant à Lord de Winter Mme Bonacieux morte et d'Artagnan que Porthos et Aramis essayaient de rappeler à la vie. 

\speak  Sont-ils donc morts tous deux? demanda froidement Lord de Winter. 

\speak  Non, heureusement, répondit Athos, M. d'Artagnan n'est qu'évanoui. 

\speak  Ah! tant mieux!» dit Lord de Winter. 

En effet, en ce moment d'Artagnan rouvrit les yeux. 

Il s'arracha des bras de Porthos et d'Aramis et se jeta comme un insensé sur le corps de sa maîtresse. 

Athos se leva, marcha vers son ami d'un pas lent et solennel, l'embrassa tendrement, et, comme il éclatait en sanglots, il lui dit de sa voix si noble et si persuasive: 

«Ami, sois homme: les femmes pleurent les morts, les hommes les vengent! 

\speak  Oh! oui, dit d'Artagnan, oui! si c'est pour la venger, je suis prêt à te suivre!» 

Athos profita de ce moment de force que l'espoir de la vengeance rendait à son malheureux ami pour faire signe à Porthos et à Aramis d'aller chercher la supérieure. 

Les deux amis la rencontrèrent dans le corridor, encore toute troublée et tout éperdue de tant d'événements; elle appela quelques religieuses, qui, contre toutes les habitudes monastiques, se trouvèrent en présence de cinq hommes. 

«Madame, dit Athos en passant le bras de d'Artagnan sous le sien, nous abandonnons à vos soins pieux le corps de cette malheureuse femme. Ce fut un ange sur la terre avant d'être un ange au ciel. Traitez-la comme une de vos sœurs; nous reviendrons un jour prier sur sa tombe.» 

D'Artagnan cacha sa figure dans la poitrine d'Athos et éclata en sanglots. 

«Pleure, dit Athos, pleure, cœur plein d'amour, de jeunesse et de vie! Hélas! je voudrais bien pouvoir pleurer comme toi!» 

Et il entraîna son ami, affectueux comme un père, consolant comme un prêtre, grand comme l'homme qui a beaucoup souffert. 

Tous cinq, suivis de leurs valets, tenant leurs chevaux par la bride, s'avancèrent vers la ville de Béthune, dont on apercevait le faubourg, et ils s'arrêtèrent devant la première auberge qu'ils rencontrèrent. 

«Mais, dit d'Artagnan, ne poursuivons-nous pas cette femme? 

\speak  Plus tard, dit Athos, j'ai des mesures à prendre. 

\speak  Elle nous échappera, reprit le jeune homme, elle nous échappera, Athos, et ce sera ta faute. 

\speak  Je réponds d'elle», dit Athos. 

D'Artagnan avait une telle confiance dans la parole de son ami, qu'il baissa la tête et entra dans l'auberge sans rien répondre. 

Porthos et Aramis se regardaient, ne comprenant rien à l'assurance d'Athos. 

Lord de Winter croyait qu'il parlait ainsi pour engourdir la douleur de d'Artagnan. 

«Maintenant, messieurs, dit Athos lorsqu'il se fut assuré qu'il y avait cinq chambres de libres dans l'hôtel, retirons-nous chacun chez soi; d'Artagnan a besoin d'être seul pour pleurer et vous pour dormir. Je me charge de tout, soyez tranquilles. 

\speak  Il me semble cependant, dit Lord de Winter, que s'il y a quelque mesure à prendre contre la comtesse, cela me regarde: c'est ma belle-sœur. 

\speak  Et moi, dit Athos, c'est ma femme. 

D'Artagnan tressaillit, car il comprit qu'Athos était sûr de sa vengeance, puisqu'il révélait un pareil secret; Porthos et Aramis se regardèrent en pâlissant. Lord de Winter pensa qu'Athos était fou. 

«Retirez-vous donc, dit Athos, et laissez-moi faire. Vous voyez bien qu'en ma qualité de mari cela me regarde. Seulement, d'Artagnan, si vous ne l'avez pas perdu, remettez-moi ce papier qui s'est échappé du chapeau de cet homme et sur lequel est écrit le nom de la ville\dots 

\speak  Ah! dit d'Artagnan, je comprends, ce nom écrit de sa main\dots 

\speak  Tu vois bien, dit Athos, qu'il y a un Dieu dans le ciel!» 

%!TeX root=../musketeersfr.tex 

\chapter{L'Homme Au Manteau Rouge}

\lettrine{L}{e} désespoir d'Athos avait fait place à une douleur concentrée, qui rendait plus lucides encore les brillantes facultés d'esprit de cet homme. 

\zz
Tout entier à une seule pensée, celle de la promesse qu'il avait faite et de la responsabilité qu'il avait prise, il se retira le dernier dans sa chambre, pria l'hôte de lui procurer une carte de la province, se courba dessus, interrogea les lignes tracées, reconnut que quatre chemins différents se rendaient de Béthune à Armentières, et fit appeler les valets. 

Planchet, Grimaud, Mousqueton et Bazin se présentèrent et reçurent les ordres clairs, ponctuels et graves d'Athos. 

Ils devaient partir au point du jour, le lendemain, et se rendre à Armentières, chacun par une route différente. Planchet, le plus intelligent des quatre, devait suivre celle par laquelle avait disparu la voiture sur laquelle les quatre amis avaient tiré, et qui était accompagnée, on se le rappelle, du domestique de Rochefort. 

Athos mit les valets en campagne d'abord, parce que, depuis que ces hommes étaient à son service et à celui de ses amis, il avait reconnu en chacun d'eux des qualités différentes et essentielles. 

Puis, des valets qui interrogent inspirent aux passants moins de défiance que leurs maîtres, et trouvent plus de sympathie chez ceux auxquels ils s'adressent. 

Enfin, Milady connaissait les maîtres, tandis qu'elle ne connaissait pas les valets; au contraire, les valets connaissaient parfaitement Milady. 

Tous quatre devaient se trouver réunis le lendemain à onze heures à l'endroit indiqué; s'ils avaient découvert la retraite de Milady, trois resteraient à la garder, le quatrième reviendrait à Béthune pour prévenir Athos et servir de guide aux quatre amis. 

Ces dispositions prises, les valets se retirèrent à leur tour. 

Athos alors se leva de sa chaise, ceignit son épée, s'enveloppa dans son manteau et sortit de l'hôtel; il était dix heures à peu près. À dix heures du soir, on le sait, en province les rues sont peu fréquentées. Athos cependant cherchait visiblement quelqu'un à qui il pût adresser une question. Enfin il rencontra un passant attardé, s'approcha de lui, lui dit quelques paroles; l'homme auquel il s'adressait recula avec terreur, cependant il répondit aux paroles du mousquetaire par une indication. Athos offrit à cet homme une demi-pistole pour l'accompagner, mais l'homme refusa. 

Athos s'enfonça dans la rue que l'indicateur avait désignée du doigt; mais, arrivé à un carrefour, il s'arrêta de nouveau, visiblement embarrassé. Cependant, comme, plus qu'aucun autre lieu, le carrefour lui offrait la chance de rencontrer quelqu'un, il s'y arrêta. En effet, au bout d'un instant, un veilleur de nuit passa. Athos lui répéta la même question qu'il avait déjà faite à la première personne qu'il avait rencontrée, le veilleur de nuit laissa apercevoir la même terreur, refusa à son tour d'accompagner Athos, et lui montra de la main le chemin qu'il devait suivre. 

Athos marcha dans la direction indiquée et atteignit le faubourg situé à l'extrémité de la ville opposée à celle par laquelle lui et ses compagnons étaient entrés. Là il parut de nouveau inquiet et embarrassé, et s'arrêta pour la troisième fois. 

Heureusement un mendiant passa, qui s'approcha d'Athos pour lui demander l'aumône. Athos lui proposa un écu pour l'accompagner où il allait. Le mendiant hésita un instant, mais à la vue de la pièce d'argent qui brillait dans l'obscurité, il se décida et marcha devant Athos. 

Arrivé à l'angle d'une rue, il lui montra de loin une petite maison isolée, solitaire, triste; Athos s'en approcha, tandis que le mendiant, qui avait reçu son salaire, s'en éloignait à toutes jambes. 

Athos en fit le tour, avant de distinguer la porte au milieu de la couleur rougeâtre dont cette maison était peinte; aucune lumière ne paraissait à travers les gerçures des contrevents, aucun bruit ne pouvait faire supposer qu'elle fût habitée, elle était sombre et muette comme un tombeau. 

Trois fois Athos frappa sans qu'on lui répondît. Au troisième coup cependant des pas intérieurs se rapprochèrent; enfin la porte s'entrebâilla, et un homme de haute taille, au teint pâle, aux cheveux et à la barbe noire, parut. 

Athos et lui échangèrent quelques mots à voix basse, puis l'homme à la haute taille fit signe au mousquetaire qu'il pouvait entrer. Athos profita à l'instant même de la permission, et la porte se referma derrière lui. 

L'homme qu'Athos était venu chercher si loin et qu'il avait trouvé avec tant de peine, le fit entrer dans son laboratoire, où il était occupé à retenir avec des fils de fer les os cliquetants d'un squelette. Tout le corps était déjà rajusté: la tête seule était posée sur une table. 

Tout le reste de l'ameublement indiquait que celui chez lequel on se trouvait s'occupait de sciences naturelles: il y avait des bocaux pleins de serpents, étiquetés selon les espèces; des lézards desséchés reluisaient comme des émeraudes taillées dans de grands cadres de bois noir; enfin, des bottes d'herbes sauvages, odoriférantes et sans doute douées de vertus inconnues au vulgaire des hommes, étaient attachées au plafond et descendaient dans les angles de l'appartement. 

Du reste, pas de famille, pas de serviteurs; l'homme à la haute taille habitait seul cette maison. 

Athos jeta un coup d'œil froid et indifférent sur tous les objets que nous venons de décrire, et, sur l'invitation de celui qu'il venait chercher, il s'assit près de lui. 

Alors il lui expliqua la cause de sa visite et le service qu'il réclamait de lui; mais à peine eut-il exposé sa demande, que l'inconnu, qui était resté debout devant le mousquetaire, recula de terreur et refusa. Alors Athos tira de sa poche un petit papier sur lequel étaient écrites deux lignes accompagnées d'une signature et d'un sceau, et le présenta à celui qui donnait trop prématurément ces signes de répugnance. L'homme à la grande taille eut à peine lu ces deux lignes, vu la signature et reconnu le sceau, qu'il s'inclina en signe qu'il n'avait plus aucune objection à faire, et qu'il était prêt à obéir. 

Athos n'en demanda pas davantage; il se leva, salua, sortit, reprit en s'en allant le chemin qu'il avait suivi pour venir, rentra dans l'hôtel et s'enferma chez lui. 

Au point du jour, d'Artagnan entra dans sa chambre et demanda ce qu'il fallait faire. 

«Attendre», répondit Athos. 

Quelques instants après, la supérieure du couvent fit prévenir les mousquetaires que l'enterrement de la victime de Milady aurait lieu à midi. Quant à l'empoisonneuse, on n'en avait pas eu de nouvelles; seulement elle avait dû fuir par le jardin, sur le sable duquel on avait reconnu la trace de ses pas et dont on avait retrouvé la porte fermée; quant à la clé, elle avait disparu. 

À l'heure indiquée, Lord de Winter et les quatre amis se rendirent au couvent: les cloches sonnaient à toute volée, la chapelle était ouverte, la grille du chœur était fermée. Au milieu du chœur, le corps de la victime, revêtue de ses habits de novice, était exposé. De chaque côté du chœur et derrière des grilles s'ouvrant sur le couvent était toute la communauté des Carmélites, qui écoutait de là le service divin et mêlait son chant au chant des prêtres, sans voir les profanes et sans être vue d'eux. 

À la porte de la chapelle, d'Artagnan sentit son courage qui fuyait de nouveau; il se retourna pour chercher Athos, mais Athos avait disparu. 

Fidèle à sa mission de vengeance, Athos s'était fait conduire au jardin; et là, sur le sable, suivant les pas légers de cette femme qui avait laissé une trace sanglante partout où elle avait passé, il s'avança jusqu'à la porte qui donnait sur le bois, se la fit ouvrir, et s'enfonça dans la forêt. 

Alors tous ses doutes se confirmèrent: le chemin par lequel la voiture avait disparu contournait la forêt. Athos suivit le chemin quelque temps les yeux fixés sur le sol; de légères taches de sang, qui provenaient d'une blessure faite ou à l'homme qui accompagnait la voiture en courrier, ou à l'un des chevaux, piquetaient le chemin. Au bout de trois quarts de lieue à peu près, à cinquante pas de Festubert, une tache de sang plus large apparaissait; le sol était piétiné par les chevaux. Entre la forêt et cet endroit dénonciateur, un peu en arrière de la terre écorchée, on retrouvait la même trace de petits pas que dans le jardin; la voiture s'était arrêtée. 

En cet endroit, Milady était sortie du bois et était montée dans la voiture. 

Satisfait de cette découverte qui confirmait tous ses soupçons, Athos revint à l'hôtel et trouva Planchet qui l'attendait avec impatience. 

Tout était comme l'avait prévu Athos. 

Planchet avait suivi la route, avait comme Athos remarqué les taches de sang, comme Athos il avait reconnu l'endroit où les chevaux s'étaient arrêtés; mais il avait poussé plus loin qu'Athos, de sorte qu'au village de Festubert, en buvant dans une auberge, il avait, sans avoir eu besoin de questionner, appris que la veille, à huit heures et demie du soir, un homme blessé, qui accompagnait une dame qui voyageait dans une chaise de poste, avait été obligé de s'arrêter, ne pouvant aller plus loin. L'accident avait été mis sur le compte de voleurs qui auraient arrêté la chaise dans le bois. L'homme était resté dans le village, la femme avait relayé et continué son chemin. 

Planchet se mit en quête du postillon qui avait conduit la chaise, et le retrouva. Il avait conduit la dame jusqu'à Fromelles, et de Fromelles elle était partie pour Armentières. Planchet prit la traverse, et à sept heures du matin il était à Armentières. 

Il n'y avait qu'un seul hôtel, celui de la Poste. Planchet alla s'y présenter comme un laquais sans place qui cherchait une condition. Il n'avait pas causé dix minutes avec les gens de l'auberge, qu'il savait qu'une femme seule était arrivée à onze heures du soir, avait pris une chambre, avait fait venir le maître d'hôtel et lui avait dit qu'elle désirerait demeurer quelque temps dans les environs. 

Planchet n'avait pas besoin d'en savoir davantage. Il courut au rendez-vous, trouva les trois laquais exacts à leur poste, les plaça en sentinelles à toutes les issues de l'hôtel, et vint trouver Athos, qui achevait de recevoir les renseignements de Planchet, lorsque ses amis rentrèrent. 

Tous les visages étaient sombres et crispés, même le doux visage d'Aramis. 

«Que faut-il faire? demanda d'Artagnan. 

\speak  Attendre», répondit Athos. 

Chacun se retira chez soi. 

À huit heures du soir, Athos donna l'ordre de seller les chevaux, et fit prévenir Lord de Winter et ses amis qu'ils eussent à se préparer pour l'expédition. 

En un instant tous cinq furent prêts. Chacun visita ses armes et les mit en état. Athos descendit le premier et trouva d'Artagnan déjà à cheval et s'impatientant. 

«Patience, dit Athos, il nous manque encore quelqu'un.» 

Les quatre cavaliers regardèrent autour d'eux avec étonnement, car ils cherchaient inutilement dans leur esprit quel était ce quelqu'un qui pouvait leur manquer. 

En ce moment Planchet amena le cheval d'Athos, le mousquetaire sauta légèrement en selle. 

«Attendez-moi, dit-il, je reviens.» 

Et il partit au galop. 

Un quart d'heure après, il revint effectivement accompagné d'un homme masqué et enveloppé d'un grand manteau rouge. 

Lord de Winter et les trois mousquetaires s'interrogèrent du regard. Nul d'entre eux ne put renseigner les autres, car tous ignoraient ce qu'était cet homme. Cependant ils pensèrent que cela devait être ainsi, puisque la chose se faisait par l'ordre d'Athos. 

À neuf heures, guidée par Planchet, la petite cavalcade se mit en route, prenant le chemin qu'avait suivi la voiture. 

C'était un triste aspect que celui de ces six hommes courant en silence, plongés chacun dans sa pensée, mornes comme le désespoir, sombres comme le châtiment.
%!TeX root=../musketeersfr.tex 

\chapter{Le Jugement} 
	
\lettrine{C}{'était} une nuit orageuse et sombre, de gros nuages couraient au ciel, voilant la clarté des étoiles; la lune ne devait se lever qu'à minuit. 

Parfois, à la lueur d'un éclair qui brillait à l'horizon, on apercevait la route qui se déroulait blanche et solitaire; puis, l'éclair éteint, tout rentrait dans l'obscurité. 

À chaque instant, Athos invitait d'Artagnan, toujours à la tête de la petite troupe, à reprendre son rang qu'au bout d'un instant il abandonnait de nouveau; il n'avait qu'une pensée, c'était d'aller en avant, et il allait. 

On traversa en silence le village de Festubert, où était resté le domestique blessé, puis on longea le bois de Richebourg; arrivés à Herlies, Planchet, qui dirigeait toujours la colonne, prit à gauche. 

Plusieurs fois, Lord de Winter, soit Porthos, soit Aramis, avaient essayé d'adresser la parole à l'homme au manteau rouge; mais à chaque interrogation qui lui avait été faite, il s'était incliné sans répondre. Les voyageurs avaient alors compris qu'il y avait quelque raison pour que l'inconnu gardât le silence, et ils avaient cessé de lui adresser la parole. 

D'ailleurs, l'orage grossissait, les éclairs se succédaient rapidement, le tonnerre commençait à gronder, et le vent, précurseur de l'ouragan, sifflait dans la plaine, agitant les plumes des cavaliers. 

La cavalcade prit le grand trot. 

Un peu au-delà de Fromelles, l'orage éclata; on déploya les manteaux; il restait encore trois lieues à faire: on les fit sous des torrents de pluie. 

D'Artagnan avait ôté son feutre et n'avait pas mis son manteau; il trouvait plaisir à laisser ruisseler l'eau sur son front brûlant et sur son corps agité de frissons fiévreux. 

Au moment où la petite troupe avait dépassé Goskal et allait arriver à la poste, un homme, abrité sous un arbre, se détacha du tronc avec lequel il était resté confondu dans l'obscurité, et s'avança jusqu'au milieu de la route, mettant son doigt sur ses lèvres. 

Athos reconnut Grimaud. 

«Qu'y a-t-il donc? s'écria d'Artagnan, aurait-elle quitté Armentières?» 

Grimaud fit de sa tête un signe affirmatif. D'Artagnan grinça des dents. 

«Silence, d'Artagnan! dit Athos, c'est moi qui me suis chargé de tout, c'est donc à moi d'interroger Grimaud. 

\speak  Où est-elle?» demanda Athos. 

Grimaud étendit la main dans la direction de la Lys. 

«Loin d'ici?» demanda Athos. 

Grimaud présenta à son maître son index plié. 

«Seule?» demanda Athos. 

Grimaud fit signe que oui. 

«Messieurs, dit Athos, elle est seule à une demi-lieue d'ici, dans la direction de la rivière. 

\speak  C'est bien, dit d'Artagnan, conduis-nous, Grimaud.» 

Grimaud prit à travers champs, et servit de guide à la cavalcade. 

Au bout de cinq cents pas à peu près, on trouva un ruisseau, que l'on traversa à gué. 

À la lueur d'un éclair, on aperçut le village d'Erquinghem. 

«Est-ce là?» demanda d'Artagnan. 

Grimaud secoua la tête en signe de négation. 

«Silence donc!» dit Athos. 

Et la troupe continua son chemin. 

Un autre éclair brilla; Grimaud étendit le bras, et à la lueur bleuâtre du serpent de feu on distingua une petite maison isolée, au bord de la rivière, à cent pas d'un bac. Une fenêtre était éclairée. 

«Nous y sommes», dit Athos. 

En ce moment, un homme couché dans le fossé se leva, c'était Mousqueton; il montra du doigt la fenêtre éclairée. 

«Elle est là, dit-il. 

\speak  Et Bazin? demanda Athos. 

\speak  Tandis que je gardais la fenêtre, il gardait la porte. 

\speak  Bien, dit Athos, vous êtes tous de fidèles serviteurs.» Athos sauta à bas de son cheval, dont il remit la bride aux mains de Grimaud, et s'avança vers la fenêtre après avoir fait signe au reste de la troupe de tourner du côté de la porte. 

La petite maison était entourée d'une haie vive, de deux ou trois pieds de haut. Athos franchit la haie, parvint jusqu'à la fenêtre privée de contrevents, mais dont les demi-rideaux étaient exactement tirés. 

Il monta sur le rebord de pierre, afin que son œil pût dépasser la hauteur des rideaux. 

À la lueur d'une lampe, il vit une femme enveloppée d'une mante de couleur sombre, assise sur un escabeau, près d'un feu mourant: ses coudes étaient posés sur une mauvaise table, et elle appuyait sa tête dans ses deux mains blanches comme l'ivoire. 

On ne pouvait distinguer son visage, mais un sourire sinistre passa sur les lèvres d'Athos, il n'y avait pas à s'y tromper, c'était bien celle qu'il cherchait. 

En ce moment un cheval hennit: Milady releva la tête, vit, collé à la vitre, le visage pâle d'Athos, et poussa un cri. 

Athos comprit qu'il était reconnu, poussa la fenêtre du genou et de la main, la fenêtre céda, les carreaux se rompirent. 

Et Athos, pareil au spectre de la vengeance, sauta dans la chambre. 

Milady courut à la porte et l'ouvrit; plus pâle et plus menaçant encore qu'Athos, d'Artagnan était sur le seuil. 

Milady recula en poussant un cri. D'Artagnan, croyant qu'elle avait quelque moyen de fuir et craignant qu'elle ne leur échappât, tira un pistolet de sa ceinture; mais Athos leva la main. 

«Remets cette arme à sa place, d'Artagnan, dit-il, il importe que cette femme soit jugée et non assassinée. Attends encore un instant, d'Artagnan, et tu seras satisfait. Entrez, messieurs.» 

D'Artagnan obéit, car Athos avait la voix solennelle et le geste puissant d'un juge envoyé par le Seigneur lui-même. Aussi, derrière d'Artagnan, entrèrent Porthos, Aramis, Lord de Winter et l'homme au manteau rouge. 

Les quatre valets gardaient la porte et la fenêtre. 

Milady était tombée sur sa chaise les mains étendues, comme pour conjurer cette terrible apparition; en apercevant son beau-frère, elle jeta un cri terrible. 

«Que demandez-vous? s'écria Milady. 

\speak  Nous demandons, dit Athos, Charlotte Backson, qui s'est appelée d'abord la comtesse de La Fère, puis Lady de Winter, baronne de Sheffield. 

\speak  C'est moi, c'est moi! murmura-t-elle au comble de la terreur, que me voulez-vous? 

\speak  Nous voulons vous juger selon vos crimes, dit Athos: vous serez libre de vous défendre, justifiez-vous si vous pouvez. Monsieur d'Artagnan, à vous d'accuser le premier.» 

D'Artagnan s'avança. 

«Devant Dieu et devant les hommes, dit-il, j'accuse cette femme d'avoir empoisonné Constance Bonacieux, morte hier soir.» 

Il se retourna vers Porthos et vers Aramis. 

«Nous attestons», dirent d'un seul mouvement les deux mousquetaires. 

D'Artagnan continua. 

«Devant Dieu et devant les hommes, j'accuse cette femme d'avoir voulu m'empoisonner moi-même, dans du vin qu'elle m'avait envoyé de Villeroi, avec une fausse lettre, comme si le vin venait de mes amis; Dieu m'a sauvé; mais un homme est mort à ma place, qui s'appelait Brisemont. 

\speak  Nous attestons, dirent de la même voix Porthos et Aramis. 

\speak  Devant Dieu et devant les hommes, j'accuse cette femme de m'avoir poussé au meurtre du baron de Wardes; et, comme personne n'est là pour attester la vérité de cette accusation, je l'atteste, moi. 

«J'ai dit.» 

Et d'Artagnan passa de l'autre côté de la chambre avec Porthos et Aramis. 

«À vous, Milord!» dit Athos. 

Le baron s'approcha à son tour. 

«Devant Dieu et devant les hommes, dit-il, j'accuse cette femme d'avoir fait assassiner le duc de Buckingham. 

\speak  Le duc de Buckingham assassiné? s'écrièrent d'un seul cri tous les assistants. 

\speak  Oui, dit le baron, assassiné! Sur la lettre d'avis que vous m'aviez écrite, j'avais fait arrêter cette femme, et je l'avais donnée en garde à un loyal serviteur; elle a corrompu cet homme, elle lui a mis le poignard dans la main, elle lui a fait tuer le duc, et dans ce moment peut-être Felton paie de sa tête le crime de cette furie.» 

Un frémissement courut parmi les juges à la révélation de ces crimes encore inconnus. 

«Ce n'est pas tout, reprit Lord de Winter, mon frère, qui vous avait faite son héritière, est mort en trois heures d'une étrange maladie qui laisse des taches livides sur tout le corps. Ma sœur, comment votre mari est-il mort? 

\speak  Horreur! s'écrièrent Porthos et Aramis. 

\speak  Assassin de Buckingham, assassin de Felton, assassin de mon frère, je demande justice contre vous, et je déclare que si on ne me la fait pas, je me la ferai.» 

Et Lord de Winter alla se ranger près de d'Artagnan, laissant la place libre à un autre accusateur. 

Milady laissa tomber son front dans ses deux mains et essaya de rappeler ses idées confondues par un vertige mortel. 

«À mon tour, dit Athos, tremblant lui-même comme le lion tremble à l'aspect du serpent, à mon tour. J'épousai cette femme quand elle était jeune fille, je l'épousai malgré toute ma famille; je lui donnai mon bien, je lui donnai mon nom; et un jour je m'aperçus que cette femme était flétrie: cette femme était marquée d'une fleur de lis sur l'épaule gauche. 

\speak  Oh! dit Milady en se levant, je défie de retrouver le tribunal qui a prononcé sur moi cette sentence infâme. Je défie de retrouver celui qui l'a exécutée. 

\speak  Silence, dit une voix. À ceci, c'est à moi de répondre!» 

Et l'homme au manteau rouge s'approcha à son tour. 

«Quel est cet homme, quel est cet homme?» s'écria Milady suffoquée par la terreur et dont les cheveux se dénouèrent et se dressèrent sur sa tête livide comme s'ils eussent été vivants. 

Tous les yeux se tournèrent sur cet homme, car à tous, excepté à Athos, il était inconnu. 

Encore Athos le regardait-il avec autant de stupéfaction que les autres, car il ignorait comment il pouvait se trouver mêlé en quelque chose à l'horrible drame qui se dénouait en ce moment. 

Après s'être approché de Milady, d'un pas lent et solennel, de manière que la table seule le séparât d'elle, l'inconnu ôta son masque. 

Milady regarda quelque temps avec une terreur croissante ce visage pâle encadré de cheveux et de favoris noirs, dont la seule expression était une impassibilité glacée, puis tout à coup: 

«Oh! non, non, dit-elle en se levant et en reculant jusqu'au mur; non, non, c'est une apparition infernale! ce n'est pas lui! à moi! à moi!» s'écria-t-elle d'une voix rauque en se retournant vers la muraille, comme si elle eût pu s'y ouvrir un passage avec ses mains. 

«Mais qui êtes-vous donc? s'écrièrent tous les témoins de cette scène. 

\speak  Demandez-le à cette femme, dit l'homme au manteau rouge, car vous voyez bien qu'elle m'a reconnu, elle. 

\speak  Le bourreau de Lille, le bourreau de Lille!» s'écria Milady en proie à une terreur insensée et se cramponnant des mains à la muraille pour ne pas tomber. 

Tout le monde s'écarta, et l'homme au manteau rouge resta seul debout au milieu de la salle. 

«Oh! grâce! grâce! pardon!» s'écria la misérable en tombant à genoux. 

L'inconnu laissa le silence se rétablir. 

«Je vous le disais bien qu'elle m'avait reconnu! reprit-il. Oui, je suis le bourreau de la ville de Lille, et voici mon histoire.» 

Tous les yeux étaient fixés sur cet homme dont on attendait les paroles avec une avide anxiété. 

«Cette jeune femme était autrefois une jeune fille aussi belle qu'elle est belle aujourd'hui. Elle était religieuse au couvent des bénédictines de Templemar. Un jeune prêtre au cœur simple et croyant desservait l'église de ce couvent; elle entreprit de le séduire et y réussit, elle eût séduit un saint. 

«Leurs vœux à tous deux étaient sacrés, irrévocables; leur liaison ne pouvait durer longtemps sans les perdre tous deux. Elle obtint de lui qu'ils quitteraient le pays; mais pour quitter le pays, pour fuir ensemble, pour gagner une autre partie de la France, où ils pussent vivre tranquilles parce qu'ils seraient inconnus, il fallait de l'argent; ni l'un ni l'autre n'en avait. Le prêtre vola les vases sacrés, les vendit; mais comme ils s'apprêtaient à partir ensemble, ils furent arrêtés tous deux. 

«Huit jours après, elle avait séduit le fils du geôlier et s'était sauvée. Le jeune prêtre fut condamné à dix ans de fers et à la flétrissure. J'étais le bourreau de la ville de Lille, comme dit cette femme. Je fus obligé de marquer le coupable, et le coupable, messieurs, c'était mon frère! 

«Je jurai alors que cette femme qui l'avait perdu, qui était plus que sa complice, puisqu'elle l'avait poussé au crime, partagerait au moins le châtiment. Je me doutai du lieu où elle était cachée, je la poursuivis, je l'atteignis, je la garrottai et lui imprimai la même flétrissure que j'avais imprimée à mon frère. 

«Le lendemain de mon retour à Lille, mon frère parvint à s'échapper à son tour, on m'accusa de complicité, et l'on me condamna à rester en prison à sa place tant qu'il ne se serait pas constitué prisonnier. Mon pauvre frère ignorait ce jugement; il avait rejoint cette femme, ils avaient fui ensemble dans le Berry; et là, il avait obtenu une petite cure. Cette femme passait pour sa sœur. 

«Le seigneur de la terre sur laquelle était située l'église du curé vit cette prétendue sœur et en devint amoureux, amoureux au point qu'il lui proposa de l'épouser. Alors elle quitta celui qu'elle avait perdu pour celui qu'elle devait perdre, et devint la comtesse de La Fère\dots» 

Tous les yeux se tournèrent vers Athos, dont c'était le véritable nom, et qui fit signe de la tête que tout ce qu'avait dit le bourreau était vrai. 

«Alors, reprit celui-ci, fou, désespéré, décidé à se débarrasser d'une existence à laquelle elle avait tout enlevé, honneur et bonheur, mon pauvre frère revint à Lille, et apprenant l'arrêt qui m'avait condamné à sa place, se constitua prisonnier et se pendit le même soir au soupirail de son cachot. 

«Au reste, c'est une justice à leur rendre, ceux qui m'avaient condamné me tinrent parole. À peine l'identité du cadavre fut-elle constatée qu'on me rendit ma liberté. 

«Voilà le crime dont je l'accuse, voilà la cause pour laquelle je l'ai marquée. 

\speak  Monsieur d'Artagnan, dit Athos, quelle est la peine que vous réclamez contre cette femme? 

\speak  La peine de mort, répondit d'Artagnan. 

\speak  Milord de Winter, continua Athos, quelle est la peine que vous réclamez contre cette femme? 

\speak  La peine de mort, reprit Lord de Winter. 

\speak  Messieurs Porthos et Aramis, reprit Athos, vous qui êtes ses juges, quelle est la peine que vous portez contre cette femme? 

\speak  La peine de mort», répondirent d'une voix sourde les deux mousquetaires. 

Milady poussa un hurlement affreux, et fit quelques pas vers ses juges en se traînant sur ses genoux. 

Athos étendit la main vers elle. 

«Anne de Breuil, comtesse de La Fère, Milady de Winter, dit-il, vos crimes ont lassé les hommes sur la terre et Dieu dans le ciel. Si vous savez quelque prière, dites-la, car vous êtes condamnée et vous allez mourir.» 

À ces paroles, qui ne lui laissaient aucun espoir, Milady se releva de toute sa hauteur et voulut parler, mais les forces lui manquèrent; elle sentit qu'une main puissante et implacable la saisissait par les cheveux et l'entraînait aussi irrévocablement que la fatalité entraîne l'homme: elle ne tenta donc pas même de faire résistance et sortit de la chaumière. 

Lord de Winter, d'Artagnan, Athos, Porthos et Aramis sortirent derrière elle. Les valets suivirent leurs maîtres et la chambre resta solitaire avec sa fenêtre brisée, sa porte ouverte et sa lampe fumeuse qui brûlait tristement sur la table. 

%!TeX root=../musketeersfr.tex 

\chapter{L'Exécution}

\lettrine{I}{l} était minuit à peu près; la lune, échancrée par sa décroissance et ensanglantée par les dernières traces de l'orage, se levait derrière la petite ville d'Armentières, qui détachait sur sa lueur blafarde la silhouette sombre de ses maisons et le squelette de son haut clocher découpé à jour. En face, la Lys roulait ses eaux pareilles à une rivière d'étain fondu; tandis que sur l'autre rive on voyait la masse noire des arbres se profiler sur un ciel orageux envahi par de gros nuages cuivrés qui faisaient une espèce de crépuscule au milieu de la nuit. À gauche s'élevait un vieux moulin abandonné, aux ailes immobiles, dans les ruines duquel une chouette faisait entendre son cri aigu, périodique et monotone. Çà et là dans la plaine, à droite et à gauche du chemin que suivait le lugubre cortège, apparaissaient quelques arbres bas et trapus, qui semblaient des nains difformes accroupis pour guetter les hommes à cette heure sinistre. 

De temps en temps un large éclair ouvrait l'horizon dans toute sa largeur, serpentait au-dessus de la masse noire des arbres et venait comme un effrayant cimeterre couper le ciel et l'eau en deux parties. Pas un souffle de vent ne passait dans l'atmosphère alourdie. Un silence de mort écrasait toute la nature; le sol était humide et glissant de la pluie qui venait de tomber, et les herbes ranimées jetaient leur parfum avec plus d'énergie. 

Deux valets traînaient Milady, qu'ils tenaient chacun par un bras; le bourreau marchait derrière, et Lord de Winter, d'Artagnan, Athos, Porthos et Aramis marchaient derrière le bourreau. 

Planchet et Bazin venaient les derniers. 

Les deux valets conduisaient Milady du côté de la rivière. Sa bouche était muette; mais ses yeux parlaient avec leur inexprimable éloquence, suppliant tour à tour chacun de ceux qu'elle regardait. 

Comme elle se trouvait de quelques pas en avant, elle dit aux valets: 

«Mille pistoles à chacun de vous si vous protégez ma fuite; mais si vous me livrez à vos maîtres, j'ai ici près des vengeurs qui vous feront payer cher ma mort.» 

Grimaud hésitait. Mousqueton tremblait de tous ses membres. 

Athos, qui avait entendu la voix de Milady, s'approcha vivement, Lord de Winter en fit autant. 

«Renvoyez ces valets, dit-il, elle leur a parlé, ils ne sont plus sûrs.» 

On appela Planchet et Bazin, qui prirent la place de Grimaud et de Mousqueton. 

Arrivés au bord de l'eau, le bourreau s'approcha de Milady et lui lia les pieds et les mains. 

Alors elle rompit le silence pour s'écrier: 

«Vous êtes des lâches, vous êtes des misérables assassins, vous vous mettez à dix pour égorger une femme; prenez garde, si je ne suis point secourue, je serai vengée. 

\speak  Vous n'êtes pas une femme, dit froidement Athos, vous n'appartenez pas à l'espèce humaine, vous êtes un démon échappé de l'enfer et que nous allons y faire rentrer. 

\speak  Ah! messieurs les hommes vertueux! dit Milady, faites attention que celui qui touchera un cheveu de ma tête est à son tour un assassin. 

\speak  Le bourreau peut tuer, sans être pour cela un assassin, madame, dit l'homme au manteau rouge en frappant sur sa large épée; c'est le dernier juge, voilà tout: \textit{Nachrichter}, comme disent nos voisins les Allemands.» 

Et, comme il la liait en disant ces paroles, Milady poussa deux ou trois cris sauvages, qui firent un effet sombre et étrange en s'envolant dans la nuit et en se perdant dans les profondeurs du bois. 

«Mais si je suis coupable, si j'ai commis les crimes dont vous m'accusez, hurlait Milady, conduisez-moi devant un tribunal, vous n'êtes pas des juges, vous, pour me condamner. 

\speak  Je vous avais proposé Tyburn, dit Lord de Winter, pourquoi n'avez-vous pas voulu? 

\speak  Parce que je ne veux pas mourir! s'écria Milady en se débattant, parce que je suis trop jeune pour mourir! 

\speak  La femme que vous avez empoisonnée à Béthune était plus jeune encore que vous, madame, et cependant elle est morte, dit d'Artagnan. 

\speak  J'entrerai dans un cloître, je me ferai religieuse, dit Milady. 

\speak  Vous étiez dans un cloître, dit le bourreau, et vous en êtes sortie pour perdre mon frère.» 

Milady poussa un cri d'effroi, et tomba sur ses genoux. 

Le bourreau la souleva sous les bras, et voulut l'emporter vers le bateau. 

«Oh! mon Dieu! s'écria-t-elle, mon Dieu! allez-vous donc me noyer!» 

Ces cris avaient quelque chose de si déchirant, que d'Artagnan, qui d'abord était le plus acharné à la poursuite de Milady, se laissa aller sur une souche, et pencha la tête, se bouchant les oreilles avec les paumes de ses mains; et cependant, malgré cela, il l'entendait encore menacer et crier. 

D'Artagnan était le plus jeune de tous ces hommes, le cœur lui manqua. 

«Oh! je ne puis voir cet affreux spectacle! je ne puis consentir à ce que cette femme meure ainsi!» 

Milady avait entendu ces quelques mots, et elle s'était reprise à une lueur d'espérance. 

«D'Artagnan! d'Artagnan! cria-t-elle, souviens-toi que je t'ai aimé!» 

Le jeune homme se leva et fit un pas vers elle. 

Mais Athos, brusquement, tira son épée, se mit sur son chemin. 

«Si vous faites un pas de plus, d'Artagnan, dit-il, nous croiserons le fer ensemble. 

D'Artagnan tomba à genoux et pria. 

«Allons, continua Athos, bourreau, fais ton devoir. 

\speak  Volontiers, Monseigneur, dit le bourreau, car aussi vrai que je suis bon catholique, je crois fermement être juste en accomplissant ma fonction sur cette femme. 

\speak  C'est bien.» 

Athos fit un pas vers Milady. 

«Je vous pardonne, dit-il, le mal que vous m'avez fait; je vous pardonne mon avenir brisé, mon honneur perdu, mon amour souillé et mon salut à jamais compromis par le désespoir où vous m'avez jeté. Mourez en paix.» 

Lord de Winter s'avança à son tour. 

«Je vous pardonne, dit-il, l'empoisonnement de mon frère, l'assassinat de Sa Grâce Lord Buckingham; je vous pardonne la mort du pauvre Felton, je vous pardonne vos tentatives sur ma personne. Mourez en paix. 

\speak  Et moi, dit d'Artagnan, pardonnez-moi, madame, d'avoir, par une fourberie indigne d'un gentilhomme, provoqué votre colère; et, en échange, je vous pardonne le meurtre de ma pauvre amie et vos vengeances cruelles pour moi, je vous pardonne et je pleure sur vous. Mourez en paix! 

\speak  \textit{I am lost!} murmura en anglais Milady. \textit{I must die.}» 

Alors elle se releva d'elle-même, jeta tout autour d'elle un de ces regards clairs qui semblaient jaillir d'un œil de flamme. 

Elle ne vit rien. 

Elle écouta et n'entendit rien. 

Elle n'avait autour d'elle que des ennemis. 

«Où vais-je mourir? dit-elle. 

\speak  Sur l'autre rive», répondit le bourreau. 

Alors il la fit entrer dans la barque, et, comme il allait y mettre le pied pour la suivre, Athos lui remit une somme d'argent. 

«Tenez, dit-il, voici le prix de l'exécution; que l'on voie bien que nous agissons en juges. 

\speak  C'est bien, dit le bourreau; et que maintenant, à son tour, cette femme sache que je n'accomplis pas mon métier, mais mon devoir.» 

Et il jeta l'argent dans la rivière. 

Le bateau s'éloigna vers la rive gauche de la Lys, emportant la coupable et l'exécuteur; tous les autres demeurèrent sur la rive droite, où ils étaient tombés à genoux. 

Le bateau glissait lentement le long de la corde du bac, sous le reflet d'un nuage pâle qui surplombait l'eau en ce moment. 

On le vit aborder sur l'autre rive; les personnages se dessinaient en noir sur l'horizon rougeâtre. 

Milady, pendant le trajet, était parvenue à détacher la corde qui liait ses pieds: en arrivant sur le rivage, elle sauta légèrement à terre et prit la fuite. 

Mais le sol était humide; en arrivant au haut du talus, elle glissa et tomba sur ses genoux. 

Une idée superstitieuse la frappa sans doute; elle comprit que le Ciel lui refusait son secours et resta dans l'attitude où elle se trouvait, la tête inclinée et les mains jointes. 

Alors on vit, de l'autre rive, le bourreau lever lentement ses deux bras, un rayon de lune se refléta sur la lame de sa large épée, les deux bras retombèrent; on entendit le sifflement du cimeterre et le cri de la victime, puis une masse tronquée s'affaissa sous le coup. 

Alors le bourreau détacha son manteau rouge, l'étendit à terre, y coucha le corps, y jeta la tête, le noua par les quatre coins, le chargea sur son épaule et remonta dans le bateau. 

Arrivé au milieu de la Lys, il arrêta la barque, et suspendant son fardeau au-dessus de la rivière: 

«Laissez passer la justice de Dieu!» cria-t-il à haute voix. 

Et il laissa tomber le cadavre au plus profond de l'eau, qui se referma sur lui. 

Trois jours après, les quatre mousquetaires rentraient à Paris; ils étaient restés dans les limites de leur congé, et le même soir ils allèrent faire leur visite accoutumée à M. de Tréville. 

«Eh bien, messieurs, leur demanda le brave capitaine, vous êtes-vous bien amusés dans votre excursion? 

\speak  Prodigieusement», répondit Athos, les dents serrées. 
%!TeX root=../musketeersfr.tex 

\chapter{Conclusion}

\lettrine{L}{e} 6 du mois suivant, le roi, tenant la promesse qu'il avait faite au cardinal de quitter Paris pour revenir à La Rochelle, sortit de sa capitale tout étourdi encore de la nouvelle qui venait de s'y répandre que Buckingham venait d'être assassiné. 

Quoique prévenue que l'homme qu'elle avait tant aimé courait un danger, la reine, lorsqu'on lui annonça cette mort, ne voulut pas la croire; il lui arriva même de s'écrier imprudemment: 

«C'est faux! il vient de m'écrire.» 

Mais le lendemain il lui fallut bien croire à cette fatale nouvelle; La Porte, retenu comme tout le monde en Angleterre par les ordres du roi Charles I\ier, arriva porteur du dernier et funèbre présent que Buckingham envoyait à la reine. 

La joie du roi avait été très vive; il ne se donna pas la peine de la dissimuler et la fit même éclater avec affectation devant la reine. Louis XIII, comme tous les cœurs faibles, manquait de générosité. 

Mais bientôt le roi redevint sombre et mal portant: son front n'était pas de ceux qui s'éclaircissent pour longtemps; il sentait qu'en retournant au camp il allait reprendre son esclavage, et cependant il y retournait. 

Le cardinal était pour lui le serpent fascinateur et il était, lui, l'oiseau qui voltige de branche en branche sans pouvoir lui échapper. 

Aussi le retour vers La Rochelle était-il profondément triste. Nos quatre amis surtout faisaient l'étonnement de leurs camarades; ils voyageaient ensemble, côte à côte, l'œil sombre et la tête baissée. Athos relevait seul de temps en temps son large front; un éclair brillait dans ses yeux, un sourire amer passait sur ses lèvres, puis, pareil à ses camarades, il se laissait de nouveau aller à ses rêveries. 

Aussitôt l'arrivée de l'escorte dans une ville, dès qu'ils avaient conduit le roi à son logis, les quatre amis se retiraient ou chez eux ou dans quelque cabaret écarté, où ils ne jouaient ni ne buvaient; seulement ils parlaient à voix basse en regardant avec attention si nul ne les écoutait. 

Un jour que le roi avait fait halte sur la route pour voler la pie, et que les quatre amis, selon leur habitude, au lieu de suivre la chasse, s'étaient arrêtés dans un cabaret sur la grande route, un homme, qui venait de La Rochelle à franc étrier, s'arrêta à la porte pour boire un verre de vin, et plongea son regard dans l'intérieur de la chambre où étaient attablés les quatre mousquetaires. 

«Holà! monsieur d'Artagnan! dit-il, n'est-ce point vous que je vois là-bas?» 

D'Artagnan leva la tête et poussa un cri de joie. Cet homme qu'il appelait son fantôme, c'était son inconnu de Meung, de la rue des Fossoyeurs et d'Arras. 

D'Artagnan tira son épée et s'élança vers la porte. 

Mais cette fois, au lieu de fuir, l'inconnu s'élança à bas de son cheval, et s'avança à la rencontre de d'Artagnan. 

«Ah! monsieur, dit le jeune homme, je vous rejoins donc enfin; cette fois vous ne m'échapperez pas. 

\speak  Ce n'est pas mon intention non plus, monsieur, car cette fois je vous cherchais; au nom du roi, je vous arrête et dis que vous ayez à me rendre votre épée, monsieur, et cela sans résistance; il y va de la tête, je vous en avertis. 

\speak  Qui êtes-vous donc? demanda d'Artagnan en baissant son épée, mais sans la rendre encore. 

\speak  Je suis le chevalier de Rochefort, répondit l'inconnu, l'écuyer de M. le cardinal de Richelieu, et j'ai ordre de vous ramener à Son Éminence. 

\speak  Nous retournons auprès de Son Éminence, monsieur le chevalier, dit Athos en s'avançant, et vous accepterez bien la parole de M. d'Artagnan, qu'il va se rendre en droite ligne à La Rochelle. 

\speak  Je dois le remettre entre les mains des gardes qui le ramèneront au camp. 

\speak  Nous lui en servirons, monsieur, sur notre parole de gentilshommes; mais sur notre parole de gentilshommes aussi, ajouta Athos en fronçant le sourcil, M. d'Artagnan ne nous quittera pas.» 

Le chevalier de Rochefort jeta un coup d'œil en arrière et vit que Porthos et Aramis s'étaient placés entre lui et la porte; il comprit qu'il était complètement à la merci de ces quatre hommes. 

«Messieurs, dit-il, si M. d'Artagnan veut me rendre son épée, et joindre sa parole à la vôtre, je me contenterai de votre promesse de conduire M. d'Artagnan au quartier de Mgr le cardinal. 

\speak  Vous avez ma parole, monsieur, dit d'Artagnan, et voici mon épée. 

\speak  Cela me va d'autant mieux, ajouta Rochefort, qu'il faut que je continue mon voyage. 

\speak  Si c'est pour rejoindre Milady, dit froidement Athos, c'est inutile, vous ne la retrouverez pas. 

\speak  Qu'est-elle donc devenue? demanda vivement Rochefort. 

\speak  Revenez au camp et vous le saurez.» 

Rochefort demeura un instant pensif, puis, comme on n'était plus qu'à une journée de Surgères, jusqu'où le cardinal devait venir au-devant du roi, il résolut de suivre le conseil d'Athos et de revenir avec eux. 

D'ailleurs ce retour lui offrait un avantage, c'était de surveiller lui-même son prisonnier. 

On se remit en route. 

Le lendemain, à trois heures de l'après-midi, on arriva à Surgères. Le cardinal y attendait Louis XIII. Le ministre et le roi y échangèrent force caresses, se félicitèrent de l'heureux hasard qui débarrassait la France de l'ennemi acharné qui ameutait l'Europe contre elle. Après quoi, le cardinal, qui avait été prévenu par Rochefort que d'Artagnan était arrêté, et qui avait hâte de le voir, prit congé du roi en l'invitant à venir voir le lendemain les travaux de la digue qui étaient achevés. 

En revenant le soir à son quartier du pont de La Pierre, le cardinal trouva debout, devant la porte de la maison qu'il habitait, d'Artagnan sans épée et les trois mousquetaires armés. 

Cette fois, comme il était en force, il les regarda sévèrement, et fit signe de l'œil et de la main à d'Artagnan de le suivre. 

D'Artagnan obéit. 

«Nous t'attendrons, d'Artagnan», dit Athos assez haut pour que le cardinal l'entendit. 

Son Éminence fronça le sourcil, s'arrêta un instant, puis continua son chemin sans prononcer une seule parole. 

D'Artagnan entra derrière le cardinal, et Rochefort derrière d'Artagnan; la porte fut gardée. 

Son Éminence se rendit dans la chambre qui lui servait de cabinet, et fit signe à Rochefort d'introduire le jeune mousquetaire. 

Rochefort obéit et se retira. 

D'Artagnan resta seul en face du cardinal; c'était sa seconde entrevue avec Richelieu, et il avoua depuis qu'il avait été bien convaincu que ce serait la dernière. 

Richelieu resta debout, appuyé contre la cheminée, une table était dressée entre lui et d'Artagnan. 

«Monsieur, dit le cardinal, vous avez été arrêté par mes ordres. 

\speak  On me l'a dit, Monseigneur. 

\speak  Savez-vous pourquoi? 

\speak  Non, Monseigneur; car la seule chose pour laquelle je pourrais être arrêté est encore inconnue de Son Éminence.» 

Richelieu regarda fixement le jeune homme. 

«Oh! Oh! dit-il, que veut dire cela? 

\speak  Si Monseigneur veut m'apprendre d'abord les crimes qu'on m'impute, je lui dirai ensuite les faits que j'ai accomplis. 

\speak  On vous impute des crimes qui ont fait choir des têtes plus hautes que la vôtre, monsieur! dit le cardinal. 

\speak  Lesquels, Monseigneur? demanda d'Artagnan avec un calme qui étonna le cardinal lui-même. 

\speak  On vous impute d'avoir correspondu avec les ennemis du royaume, on vous impute d'avoir surpris les secrets de l'État, on vous impute d'avoir essayé de faire avorter les plans de votre général. 

\speak  Et qui m'impute cela, Monseigneur? dit d'Artagnan, qui se doutait que l'accusation venait de Milady: une femme flétrie par la justice du pays, une femme qui a épousé un homme en France et un autre en Angleterre, une femme qui a empoisonné son second mari et qui a tenté de m'empoisonner moi-même! 

\speak  Que dites-vous donc là? Monsieur, s'écria le cardinal étonné, et de quelle femme parlez-vous ainsi? 

\speak  De Milady de Winter, répondit d'Artagnan; oui, de Milady de Winter, dont, sans doute, Votre Éminence ignorait tous les crimes lorsqu'elle l'a honorée de sa confiance. 

\speak  Monsieur, dit le cardinal, si Milady de Winter a commis les crimes que vous dites, elle sera punie. 

\speak  Elle l'est, Monseigneur. 

\speak  Et qui l'a punie? 

\speak  Nous. 

\speak  Elle est en prison? 

\speak  Elle est morte. 

\speak  Morte! répéta le cardinal, qui ne pouvait croire à ce qu'il entendait: morte! n'avez-vous pas dit qu'elle était morte? 

\speak  Trois fois elle avait essayé de me tuer, et je lui avais pardonné, mais elle a tué la femme que j'aimais. Alors, mes amis et moi, nous l'avons prise, jugée et condamnée.» 

D'Artagnan alors raconta l'empoisonnement de Mme Bonacieux dans le couvent des Carmélites de Béthune, le jugement de la maison isolée, l'exécution sur les bords de la Lys. 

Un frisson courut par tout le corps du cardinal, qui cependant ne frissonnait pas facilement. 

Mais tout à coup, comme subissant l'influence d'une pensée muette, la physionomie du cardinal, sombre jusqu'alors, s'éclaircit peu à peu et arriva à la plus parfaite sérénité. 

«Ainsi, dit-il avec une voix dont la douceur contrastait avec la sévérité de ses paroles, vous vous êtes constitués juges, sans penser que ceux qui n'ont pas mission de punir et qui punissent sont des assassins! 

\speak  Monseigneur, je vous jure que je n'ai pas eu un instant l'intention de défendre ma tête contre vous. Je subirai le châtiment que Votre Éminence voudra bien m'infliger. Je ne tiens pas assez à la vie pour craindre la mort. 

\speak  Oui, je le sais, vous êtes un homme de cœur, monsieur, dit le cardinal avec une voix presque affectueuse; je puis donc vous dire d'avance que vous serez jugé, condamné même. 

\speak  Un autre pourrait répondre à Votre Éminence qu'il a sa grâce dans sa poche; moi je me contenterai de vous dire: «Ordonnez, Monseigneur, je suis prêt.» 

\speak  Votre grâce? dit Richelieu surpris. 

\speak  Oui, Monseigneur, dit d'Artagnan. 

\speak  Et signée de qui? du roi?» 

Et le cardinal prononça ces mots avec une singulière expression de mépris. 

«Non, de Votre Éminence. 

\speak  De moi? vous êtes fou, monsieur? 

\speak  Monseigneur reconnaîtra sans doute son écriture.» 

Et d'Artagnan présenta au cardinal le précieux papier qu'Athos avait arraché à Milady, et qu'il avait donné à d'Artagnan pour lui servir de sauvegarde. 

Son Éminence prit le papier et lut d'une voix lente et en appuyant sur chaque syllabe: 

\begin{mail}{Au camp de la Rochelle, ce 5 août 1628.}
	
C'est par mon ordre et pour le bien de l'État que le porteur du présent a fait ce qu'il a fait.
\closeletter{Richelieu}
\end{mail}

Le cardinal, après avoir lu ces deux lignes, tomba dans une rêverie profonde, mais il ne rendit pas le papier à d'Artagnan. 

«Il médite de quel genre de supplice il me fera mourir, se dit tout bas d'Artagnan; eh bien, ma foi! il verra comment meurt un gentilhomme.» 

Le jeune mousquetaire était en excellente disposition pour trépasser héroïquement. 

Richelieu pensait toujours, roulait et déroulait le papier dans ses mains. Enfin il leva la tête, fixa son regard d'aigle sur cette physionomie loyale, ouverte, intelligente, lut sur ce visage sillonné de larmes toutes les souffrances qu'il avait endurées depuis un mois, et songea pour la troisième ou quatrième fois combien cet enfant de vingt et un ans avait d'avenir, et quelles ressources son activité, son courage et son esprit pouvaient offrir à un bon maître. 

D'un autre côté, les crimes, la puissance, le génie infernal de Milady l'avaient plus d'une fois épouvanté. Il sentait comme une joie secrète d'être à jamais débarrassé de ce complice dangereux. 

Il déchira lentement le papier que d'Artagnan lui avait si généreusement remis. 

«Je suis perdu», dit en lui-même d'Artagnan. 

Et il s'inclina profondément devant le cardinal en homme qui dit: «Seigneur, que votre volonté soit faite!» 

Le cardinal s'approcha de la table, et, sans s'asseoir, écrivit quelques lignes sur un parchemin dont les deux tiers étaient déjà remplis et y apposa son sceau. 

«Ceci est ma condamnation, dit d'Artagnan; il m'épargne l'ennui de la Bastille et les lenteurs d'un jugement. C'est encore fort aimable à lui.» 

«Tenez, monsieur, dit le cardinal au jeune homme, je vous ai pris un blanc-seing et je vous en rends un autre. Le nom manque sur ce brevet: vous l'écrirez vous-même.» 

D'Artagnan prit le papier en hésitant et jeta les yeux dessus. 

C'était une lieutenance dans les mousquetaires. 

D'Artagnan tomba aux pieds du cardinal. 

«Monseigneur, dit-il, ma vie est à vous; disposez-en désormais; mais cette faveur que vous m'accordez, je ne la mérite pas: j'ai trois amis qui sont plus méritants et plus dignes\dots 

\speak  Vous êtes un brave garçon, d'Artagnan, interrompit le cardinal en lui frappant familièrement sur l'épaule, charmé qu'il était d'avoir vaincu cette nature rebelle. Faites de ce brevet ce qu'il vous plaira. Seulement rappelez-vous que, quoique le nom soit en blanc, c'est à vous que je le donne. 

\speak  Je ne l'oublierai jamais, répondit d'Artagnan. Votre Éminence peut en être certaine.» 

Le cardinal se retourna et dit à haute voix: 

«Rochefort!» 

Le chevalier, qui sans doute était derrière la porte entra aussitôt. 

«Rochefort, dit le cardinal, vous voyez M. d'Artagnan; je le reçois au nombre de mes amis; ainsi donc que l'on s'embrasse et que l'on soit sage si l'on tient à conserver sa tête. 

Rochefort et d'Artagnan s'embrassèrent du bout des lèvres; mais le cardinal était là, qui les observait de son œil vigilant. 

Ils sortirent de la chambre en même temps. 

«Nous nous retrouverons, n'est-ce pas, monsieur? 

\speak  Quand il vous plaira, fit d'Artagnan. 

\speak  L'occasion viendra, répondit Rochefort. 

\speak  Hein?» fit Richelieu en ouvrant la porte. 

Les deux hommes se sourirent, se serrèrent la main et saluèrent Son Éminence. 

«Nous commencions à nous impatienter, dit Athos. 

\speak  Me voilà, mes amis! répondit d'Artagnan, non seulement libre, mais en faveur. 

\speak  Vous nous conterez cela? 

\speak  Dès ce soir.» 

En effet, dès le soir même d'Artagnan se rendit au logis d'Athos, qu'il trouva en train de vider sa bouteille de vin d'Espagne, occupation qu'il accomplissait religieusement tous les soirs. 

Il lui raconta ce qui s'était passé entre le cardinal et lui, et tirant le brevet de sa poche: 

«Tenez, mon cher Athos, voilà, dit-il, qui vous revient tout naturellement.» 

Athos sourit de son doux et charmant sourire. 

«Amis, dit-il, pour Athos c'est trop; pour le comte de La Fère, c'est trop peu. Gardez ce brevet, il est à vous; hélas, mon Dieu! vous l'avez acheté assez cher.» 

D'Artagnan sortit de la chambre d'Athos, et entra dans celle de Porthos. 

Il le trouva vêtu d'un magnifique habit, couvert de broderies splendides, et se mirant dans une glace. 

«Ah! ah! dit Porthos, c'est vous, cher ami! comment trouvez-vous que ce vêtement me va? 

\speak  À merveille, dit d'Artagnan, mais je viens vous proposer un habit qui vous ira mieux encore. 

\speak  Lequel? demanda Porthos. 

\speak  Celui de lieutenant aux mousquetaires. 

D'Artagnan raconta à Porthos son entrevue avec le cardinal, et tirant le brevet de sa poche: 

«Tenez, mon cher, dit-il, écrivez votre nom là-dessus, et soyez bon chef pour moi. 

Porthos jeta les yeux sur le brevet, et le rendit à d'Artagnan, au grand étonnement du jeune homme. 

«Oui, dit-il, cela me flatterait beaucoup, mais je n'aurais pas assez longtemps à jouir de cette faveur. Pendant notre expédition de Béthune, le mari de ma duchesse est mort; de sorte que, mon cher, le coffre du défunt me tendant les bras, j'épouse la veuve. Tenez, j'essayais mon habit de noce; gardez la lieutenance, mon cher, gardez.» 

Et il rendit le brevet à d'Artagnan. 

Le jeune homme entra chez Aramis. 

Il le trouva agenouillé devant un prie-Dieu, le front appuyé contre son livre d'heures ouvert. 

Il lui raconta son entrevue avec le cardinal, et tirant pour la troisième fois son brevet de sa poche: 

«Vous, notre ami, notre lumière, notre protecteur invisible, dit-il, acceptez ce brevet; vous l'avez mérité plus que personne, par votre sagesse et vos conseils toujours suivis de si heureux résultats. 

\speak  Hélas, cher ami! dit Aramis, nos dernières aventures m'ont dégoûté tout à fait de la vie d'homme d'épée. Cette fois, mon parti est pris irrévocablement, après le siège j'entre chez les lazaristes. Gardez ce brevet, d'Artagnan, le métier des armes vous convient, vous serez un brave et aventureux capitaine.» 

D'Artagnan, l'œil humide de reconnaissance et brillant de joie, revint à Athos, qu'il trouva toujours attablé et mirant son dernier verre de malaga à la lueur de la lampe. 

«Eh bien, dit-il, eux aussi m'ont refusé. 

\speak  C'est que personne, cher ami, n'en était plus digne que vous.» 

Il prit une plume, écrivit sur le brevet le nom de d'Artagnan, et le lui remit. 

«Je n'aurai donc plus d'amis, dit le jeune homme, hélas! plus rien, que d'amers souvenirs\dots» 

Et il laissa tomber sa tête entre ses deux mains, tandis que deux larmes roulaient le long de ses joues. 

«Vous êtes jeune, vous, répondit Athos, et vos souvenirs amers ont le temps de se changer en doux souvenirs!»

%!TeX root=../musketeersfr.tex 

\chapter{Épilogue}

\lettrine{L}{a} Rochelle, privée du secours de la flotte anglaise et de la division promise par Buckingham, se rendit après un siège d'un an. Le 28 octobre 1628, on signa la capitulation. 

\zz
Le roi fit son entrée à Paris le 23 décembre de la même année. On lui fit un triomphe comme s'il revenait de vaincre l'ennemi et non des Français. Il entra par le faubourg Saint-Jacques sous des arcs de verdure. 

D'Artagnan prit possession de son grade. Porthos quitta le service et épousa, dans le courant de l'année suivante, Mme Coquenard, le coffre tant convoité contenait huit cent mille livres. 

Mousqueton eut une livrée magnifique, et de plus la satisfaction, qu'il avait ambitionnée toute sa vie, de monter derrière un carrosse doré. 

Aramis, après un voyage en Lorraine, disparut tout à coup et cessa d'écrire à ses amis. On apprit plus tard, par Mme de Chevreuse, qui le dit à deux ou trois de ses amants, qu'il avait pris l'habit dans un couvent de Nancy. 

Bazin devint frère lai. 

Athos resta mousquetaire sous les ordres de d'Artagnan jusqu'en 1633, époque à laquelle, à la suite d'un voyage qu'il fit en Touraine, il quitta aussi le service sous prétexte qu'il venait de recueillir un petit héritage en Roussillon. 

Grimaud suivit Athos. 

D'Artagnan se battit trois fois avec Rochefort et le blessa trois fois. 

«Je vous tuerai probablement à la quatrième, lui dit-il en lui tendant la main pour le relever. 

\speak  Il vaut donc mieux, pour vous et pour moi, que nous en restions là, répondit le blessé. Corbleu! je suis plus votre ami que vous ne pensez, car dès la première rencontre j'aurais pu, en disant un mot au cardinal, vous faire couper le cou.» 

Ils s'embrassèrent cette fois, mais de bon cœur et sans arrière-pensée. 

Planchet obtint de Rochefort le grade de sergent dans les gardes. 

M. Bonacieux vivait fort tranquille, ignorant parfaitement ce qu'était devenue sa femme et ne s'en inquiétant guère. Un jour, il eut l'imprudence de se rappeler au souvenir du cardinal; le cardinal lui fit répondre qu'il allait pourvoir à ce qu'il ne manquât jamais de rien désormais. 

En effet, le lendemain, M. Bonacieux, étant sorti à sept heures du soir de chez lui pour se rendre au Louvre, ne reparut plus rue des Fossoyeurs; l'avis de ceux qui parurent les mieux informés fut qu'il était nourri et logé dans quelque château royal aux frais de sa généreuse Éminence.

\KOMAoptions{headings=openright}


\KOMAoptions{headings=openleft}
\chapter*{Colophon}

\centering

\vfill
\begin{minipage}{\textwidth}
\textit{The Three Musketeers} was serialised in the French newspaper \textit{Le Siècle} between March and July of 1844. Alexandre Dumas \textit{père} (1802--1870) would later call it his favourite work. As wtih most of Dumas' works, it was a collaboration with his writing partner Auguste Maquet (1813--1888).

Title page decoration is an engraving by Jules Huyot (1841--1921) based on a drawing by Maurice Leloir (1853--1940), and was originally published in 1894 by Calmann-Lévy in Paris (France).

\end{minipage}
\vfill
gutenberg.org/ebooks/1257
\vfill
\divider
\vfill
\begin{minipage}{\textwidth}
Text is set in <EB Garamond,> Georg Mayr-Duffner's free and open source implementation of Claude Garamond’s famous humanist typefaces from the mid-sixteenth century. This digital version reproduces the original design by Claude Garamont closely: the source for the letterforms is a scan of a specimen known as the <Berner specimen,> which was composed in 1592 by Conrad Berner, the son-in-law of Christian Egenolff and his successor at the Egenolff print office.

Title page is set in <Lombardic>, by Manfred Klein.
\end{minipage}
\vfill
github.com/georgd/EB-Garamond
\vfill
\divider
\vfill
\begin{minipage}{\textwidth}
This typeset is dedicated to the public domain under a Creative Commons CC0 1.0 Universal deed: creativecommons.org/publicdomain/zero/1.0/\\
\end{minipage}
\vfill
\divider
\vfill
Typeset in \LaTeX{}. Last revised \today.
\thispagestyle{empty}
\end{document}