\chapter{The Morrel Family} 

 \lettrine{I}{n} a very few minutes the count reached № 7 in the Rue Meslay. The house was of white stone, and in a small court before it were two small beds full of beautiful flowers. In the concierge that opened the gate the count recognized Cocles; but as he had but one eye, and that eye had become somewhat dim in the course of nine years, Cocles did not recognize the count. 

 The carriages that drove up to the door were compelled to turn, to avoid a fountain that played in a basin of rockwork,—an ornament that had excited the jealousy of the whole quarter, and had gained for the place the appellation of \textit{The Little Versailles}. It is needless to add that there were gold and silver fish in the basin. The house, with kitchens and cellars below, had above the ground floor, two stories and attics. The whole of the property, consisting of an immense workshop, two pavilions at the bottom of the garden, and the garden itself, had been purchased by Emmanuel, who had seen at a glance that he could make of it a profitable speculation. He had reserved the house and half the garden, and building a wall between the garden and the workshops, had let them upon lease with the pavilions at the bottom of the garden. So that for a trifling sum he was as well lodged, and as perfectly shut out from observation, as the inhabitants of the finest mansion in the Faubourg St. Germain. 

 The breakfast-room was finished in oak; the salon in mahogany, and the furnishings were of blue velvet; the bedroom was in citronwood and green damask. There was a study for Emmanuel, who never studied, and a music-room for Julie, who never played. The whole of the second story was set apart for Maximilian; it was precisely similar to his sister's apartments, except that for the breakfast-parlour he had a billiard-room, where he received his friends. He was superintending the grooming of his horse, and smoking his cigar at the entrance of the garden, when the count's carriage stopped at the gate. 

 Cocles opened the gate, and Baptistin, springing from the box, inquired whether Monsieur and Madame Herbault and Monsieur Maximilian Morrel would see his excellency the Count of Monte Cristo. 

 <The Count of Monte Cristo?> cried Morrel, throwing away his cigar and hastening to the carriage; <I should think we would see him. Ah, a thousand thanks, count, for not having forgotten your promise.> 

 And the young officer shook the count's hand so warmly, that Monte Cristo could not be mistaken as to the sincerity of his joy, and he saw that he had been expected with impatience, and was received with pleasure. 

 <Come, come,> said Maximilian, <I will serve as your guide; such a man as you are ought not to be introduced by a servant. My sister is in the garden plucking the dead roses; my brother is reading his two papers, \textit{la Presse} and \textit{les Débats}, within six steps of her; for wherever you see Madame Herbault, you have only to look within a circle of four yards and you will find M. Emmanuel, and <reciprocally,> as they say at the Polytechnic School.> 

 At the sound of their steps a young woman of twenty to five-and-twenty, dressed in a silk morning gown, and busily engaged in plucking the dead leaves off a noisette rose-tree, raised her head. This was Julie, who had become, as the clerk of the house of Thomson \& French had predicted, Madame Emmanuel Herbault. She uttered a cry of surprise at the sight of a stranger, and Maximilian began to laugh. 

 <Don't disturb yourself, Julie,> said he. <The count has only been two or three days in Paris, but he already knows what a fashionable woman of the Marais is, and if he does not, you will show him.> 

 <Ah, monsieur,> returned Julie, <it is treason in my brother to bring you thus, but he never has any regard for his poor sister. Penelon, Penelon!> 

 An old man, who was digging busily at one of the beds, stuck his spade in the earth, and approached, cap in hand, striving to conceal a quid of tobacco he had just thrust into his cheek. A few locks of gray mingled with his hair, which was still thick and matted, while his bronzed features and determined glance well suited an old sailor who had braved the heat of the equator and the storms of the tropics. 

 <I think you hailed me, Mademoiselle Julie?> said he. 

 Penelon had still preserved the habit of calling his master's daughter <Mademoiselle Julie,> and had never been able to change the name to Madame Herbault. 

 <Penelon,> replied Julie, <go and inform M. Emmanuel of this gentleman's visit, and Maximilian will conduct him to the salon.> 

 Then, turning to Monte Cristo,—<I hope you will permit me to leave you for a few minutes,> continued she; and without awaiting any reply, disappeared behind a clump of trees, and escaped to the house by a lateral alley.  <I am sorry to see,> observed Monte Cristo to Morrel, <that I cause no small disturbance in your house.> 

 <Look there,> said Maximilian, laughing; <there is her husband changing his jacket for a coat. I assure you, you are well known in the Rue Meslay.> 

 <Your family appears to be a very happy one,> said the count, as if speaking to himself. 

 <Oh, yes, I assure you, count, they want nothing that can render them happy; they are young and cheerful, they are tenderly attached to each other, and with twenty-five thousand francs a year they fancy themselves as rich as Rothschild.> 

 <Five-and-twenty thousand francs is not a large sum, however,> replied Monte Cristo, with a tone so sweet and gentle, that it went to Maximilian's heart like the voice of a father; <but they will not be content with that. Your brother-in-law is a barrister? a doctor?> 

\enquote{He was a merchant, monsieur, and had succeeded to the business of my poor father. M. Morrel, at his death, left 500,000 francs, which were divided between my sister and myself, for we were his only children. Her husband, who, when he married her, had no other patrimony than his noble probity, his first-rate ability, and his spotless reputation, wished to possess as much as his wife. He laboured and toiled until he had amassed 250,000 francs; six years sufficed to achieve this object. Oh, I assure you, sir, it was a touching spectacle to see these young creatures, destined by their talents for higher stations, toiling together, and through their unwillingness to change any of the customs of their paternal house, taking six years to accomplish what less scrupulous people would have effected in two or three. Marseilles resounded with their well-earned praises. At last, one day, Emmanuel came to his wife, who had just finished making up the accounts. 

 <Julie,> said he to her, <Cocles has just given me the last rouleau of a hundred francs; that completes the 250,000 francs we had fixed as the limits of our gains. Can you content yourself with the small fortune which we shall possess for the future? Listen to me. Our house transacts business to the amount of a million a year, from which we derive an income of 40,000 francs. We can dispose of the business, if we please, in an hour, for I have received a letter from M. Delaunay, in which he offers to purchase the good-will of the house, to unite with his own, for 300,000 francs. Advise me what I had better do.> 

 <Emmanuel,> returned my sister, <the house of Morrel can only be carried on by a Morrel. Is it not worth 300,000 francs to save our father's name from the chances of evil fortune and failure?> 

 <I thought so,> replied Emmanuel; <but I wished to have your advice.> 

 <This is my counsel:—Our accounts are made up and our bills paid; all we have to do is to stop the issue of any more, and close our office.> 

 This was done instantly. It was three o'clock; at a quarter past, a merchant presented himself to insure two ships; it was a clear profit of 15,000 francs. 

 <Monsieur,> said Emmanuel, <have the goodness to address yourself to M. Delaunay. We have quitted business.> 

 <How long?> inquired the astonished merchant.  
 
 <A quarter of an hour,> was the reply. 

 And this is the reason, monsieur,} continued Maximilian, <of my sister and brother-in-law having only 25,000 francs a year.> 

 Maximilian had scarcely finished his story, during which the count's heart had swelled within him, when Emmanuel entered wearing a hat and coat. He saluted the count with the air of a man who is aware of the rank of his guest; then, after having led Monte Cristo around the little garden, he returned to the house. 

 A large vase of Japan porcelain, filled with flowers that loaded the air with their perfume, stood in the salon. Julie, suitably dressed, and her hair arranged (she had accomplished this feat in less than ten minutes), received the count on his entrance. The songs of the birds were heard in an aviary hard by, and the branches of laburnums and rose acacias formed an exquisite framework to the blue velvet curtains. Everything in this charming retreat, from the warble of the birds to the smile of the mistress, breathed tranquillity and repose. 

 The count had felt the influence of this happiness from the moment he entered the house, and he remained silent and pensive, forgetting that he was expected to renew the conversation, which had ceased after the first salutations had been exchanged. The silence became almost painful when, by a violent effort, tearing himself from his pleasing reverie: 

 <Madame,> said he at length, <I pray you to excuse my emotion, which must astonish you who are only accustomed to the happiness I meet here; but contentment is so new a sight to me, that I could never be weary of looking at yourself and your husband.> 

 <We are very happy, monsieur,> replied Julie; <but we have also known unhappiness, and few have ever undergone more bitter sufferings than ourselves.> 

 The count's features displayed an expression of the most intense curiosity. 

 <Oh, all this is a family history, as Château-Renaud told you the other day,> observed Maximilian. <This humble picture would have but little interest for you, accustomed as you are to behold the pleasures and the misfortunes of the wealthy and industrious; but such as we are, we have experienced bitter sorrows.> 

 <And God has poured balm into your wounds, as he does into those of all who are in affliction?> said Monte Cristo inquiringly. 

 <Yes, count,> returned Julie, <we may indeed say he has, for he has done for us what he grants only to his chosen; he sent us one of his angels.> 

 The count's cheeks became scarlet, and he coughed, in order to have an excuse for putting his handkerchief to his mouth. 

 <Those born to wealth, and who have the means of gratifying every wish,> said Emmanuel, <know not what is the real happiness of life, just as those who have been tossed on the stormy waters of the ocean on a few frail planks can alone realize the blessings of fair weather.> 

 Monte Cristo rose, and without making any answer (for the tremulousness of his voice would have betrayed his emotion) walked up and down the apartment with a slow step. 

 <Our magnificence makes you smile, count,> said Maximilian, who had followed him with his eyes. 

 <No, no,> returned Monte Cristo, pale as death, pressing one hand on his heart to still its throbbings, while with the other he pointed to a crystal cover, beneath which a silken purse lay on a black velvet cushion. <I was wondering what could be the significance of this purse, with the paper at one end and the large diamond at the other.> 

 <Count,> replied Maximilian, with an air of gravity, <those are our most precious family treasures.> 

 <The stone seems very brilliant,> answered the count. 

 <Oh, my brother does not allude to its value, although it has been estimated at 100,000 francs; he means, that the articles contained in this purse are the relics of the angel I spoke of just now.> 

 <This I do not comprehend; and yet I may not ask for an explanation, madame,> replied Monte Cristo bowing. <Pardon me, I had no intention of committing an indiscretion.> 

 <Indiscretion,—oh, you make us happy by giving us an excuse for expatiating on this subject. If we wanted to conceal the noble action this purse commemorates, we should not expose it thus to view. Oh, would we could relate it everywhere, and to everyone, so that the emotion of our unknown benefactor might reveal his presence.> 

 <Ah, really,> said Monte Cristo in a half-stifled voice. 

 <Monsieur,> returned Maximilian, raising the glass cover, and respectfully kissing the silken purse, <this has touched the hand of a man who saved my father from suicide, us from ruin, and our name from shame and disgrace,—a man by whose matchless benevolence we poor children, doomed to want and wretchedness, can at present hear everyone envying our happy lot. This letter> (as he spoke, Maximilian drew a letter from the purse and gave it to the count)—<this letter was written by him the day that my father had taken a desperate resolution, and this diamond was given by the generous unknown to my sister as her dowry.> 

 Monte Cristo opened the letter, and read it with an indescribable feeling of delight. It was the letter written (as our readers know) to Julie, and signed <Sinbad the Sailor.> 

 <Unknown you say, is the man who rendered you this service—unknown to you?> 

 <Yes; we have never had the happiness of pressing his hand,> continued Maximilian. <We have supplicated Heaven in vain to grant us this favour, but the whole affair has had a mysterious meaning that we cannot comprehend—we have been guided by an invisible hand,—a hand as powerful as that of an enchanter.> 

 <Oh,> cried Julie, <I have not lost all hope of some day kissing that hand, as I now kiss the purse which he has touched. Four years ago, Penelon was at Trieste—Penelon, count, is the old sailor you saw in the garden, and who, from quartermaster, has become gardener—Penelon, when he was at Trieste, saw on the quay an Englishman, who was on the point of embarking on board a yacht, and he recognized him as the person who called on my father the fifth of June, 1829, and who wrote me this letter on the fifth of September. He felt convinced of his identity, but he did not venture to address him.> 

 <An Englishman,> said Monte Cristo, who grew uneasy at the attention with which Julie looked at him. <An Englishman you say?> 

 <Yes,> replied Maximilian, <an Englishman, who represented himself as the confidential clerk of the house of Thomson \& French, at Rome. It was this that made me start when you said the other day, at M. de Morcerf's, that Messrs. Thomson \& French were your bankers. That happened, as I told you, in 1829. For God's sake, tell me, did you know this Englishman?> 

 <But you tell me, also, that the house of Thomson \& French have constantly denied having rendered you this service?> 

 <Yes.> 

 <Then is it not probable that this Englishman may be someone who, grateful for a kindness your father had shown him, and which he himself had forgotten, has taken this method of requiting the obligation?> 

 <Everything is possible in this affair, even a miracle.> 

 <What was his name?> asked Monte Cristo. 

 <He gave no other name,> answered Julie, looking earnestly at the count, <than that at the end of his letter—<Sinbad the Sailor.>> 

 <Which is evidently not his real name, but a fictitious one.> 

 Then, noticing that Julie was struck with the sound of his voice: 

 <Tell me,> continued he, <was he not about my height, perhaps a little taller, with his chin imprisoned, as it were, in a high cravat; his coat closely buttoned up, and constantly taking out his pencil?> 

 <Oh, do you then know him?> cried Julie, whose eyes sparkled with joy. 

 <No,> returned Monte Cristo <I only guessed. I knew a Lord Wilmore, who was constantly doing actions of this kind.> 

 <Without revealing himself?> 

 <He was an eccentric being, and did not believe in the existence of gratitude.> 

 <Oh, Heaven,> exclaimed Julie, clasping her hands, <in what did he believe, then?>  <He did not credit it at the period which I knew him,> said Monte Cristo, touched to the heart by the accents of Julie's voice; <but, perhaps, since then he has had proofs that gratitude does exist.> 

 <And do you know this gentleman, monsieur?> inquired Emmanuel. 

 <Oh, if you do know him,> cried Julie, <can you tell us where he is—where we can find him? Maximilian—Emmanuel—if we do but discover him, he must believe in the gratitude of the heart!> 

 Monte Cristo felt tears start into his eyes, and he again walked hastily up and down the room. 

 <In the name of Heaven,> said Maximilian, <if you know anything of him, tell us what it is.> 

 <Alas,> cried Monte Cristo, striving to repress his emotion, <if Lord Wilmore was your unknown benefactor, I fear you will never see him again. I parted from him two years ago at Palermo, and he was then on the point of setting out for the most remote regions; so that I fear he will never return.> 

 <Oh, monsieur, this is cruel of you,> said Julie, much affected; and the young lady's eyes swam with tears. 

 <Madame,> replied Monte Cristo gravely, and gazing earnestly on the two liquid pearls that trickled down Julie's cheeks, <had Lord Wilmore seen what I now see, he would become attached to life, for the tears you shed would reconcile him to mankind;> and he held out his hand to Julie, who gave him hers, carried away by the look and accent of the count. 

 <But,> continued she, <Lord Wilmore had a family or friends, he must have known someone, can we not\longdash> 

 <Oh, it is useless to inquire,> returned the count; <perhaps, after all, he was not the man you seek for. He was my friend: he had no secrets from me, and if this had been so he would have confided in me.> 

 <And he told you nothing?> 

 <Not a word.> 

 <Nothing that would lead you to suppose?> 

 <Nothing.> 

 <And yet you spoke of him at once.> 

 <Ah, in such a case one supposes\longdash> 

 <Sister, sister,> said Maximilian, coming to the count's aid, <monsieur is quite right. Recollect what our excellent father so often told us, <It was no Englishman that thus saved us.>> 

 Monte Cristo started. <What did your father tell you, M. Morrel?> said he eagerly. 

 <My father thought that this action had been miraculously performed—he believed that a benefactor had arisen from the grave to save us. Oh, it was a touching superstition, monsieur, and although I did not myself believe it, I would not for the world have destroyed my father's faith. How often did he muse over it and pronounce the name of a dear friend—a friend lost to him forever; and on his death-bed, when the near approach of eternity seemed to have illumined his mind with supernatural light, this thought, which had until then been but a doubt, became a conviction, and his last words were, <Maximilian, it was Edmond Dantès!>> 

 At these words the count's paleness, which had for some time been increasing, became alarming; he could not speak; he looked at his watch like a man who has forgotten the hour, said a few hurried words to Madame Herbault, and pressing the hands of Emmanuel and Maximilian,—<Madame,> said he, <I trust you will allow me to visit you occasionally; I value your friendship, and feel grateful to you for your welcome, for this is the first time for many years that I have thus yielded to my feelings;> and he hastily quitted the apartment. 

 <This Count of Monte Cristo is a strange man,> said Emmanuel. 

 <Yes,> answered Maximilian, <but I feel sure he has an excellent heart, and that he likes us.> 

 <His voice went to my heart,> observed Julie; <and two or three times I fancied that I had heard it before.> 