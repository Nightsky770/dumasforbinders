\chapter{Dividing the Proceeds} 

 \lettrine{T}{he} apartment on the first floor of the house in the Rue Saint-Germain-des-Prés, where Albert de Morcerf had selected a home for his mother, was let to a very mysterious person. This was a man whose face the concierge himself had never seen, for in the winter his chin was buried in one of the large red handkerchiefs worn by gentlemen's coachmen on a cold night, and in the summer he made a point of always blowing his nose just as he approached the door. Contrary to custom, this gentleman had not been watched, for as the report ran that he was a person of high rank, and one who would allow no impertinent interference, his \textit{incognito} was strictly respected. 

 His visits were tolerably regular, though occasionally he appeared a little before or after his time, but generally, both in summer and winter, he took possession of his apartment about four o'clock, though he never spent the night there. At half-past three in the winter the fire was lighted by the discreet servant, who had the superintendence of the little apartment, and in the summer ices were placed on the table at the same hour. At four o'clock, as we have already stated, the mysterious personage arrived. 

 Twenty minutes afterwards a carriage stopped at the house, a lady alighted in a black or dark blue dress, and always thickly veiled; she passed like a shadow through the lodge, and ran upstairs without a sound escaping under the touch of her light foot. No one ever asked her where she was going. Her face, therefore, like that of the gentleman, was perfectly unknown to the two concierges, who were perhaps unequalled throughout the capital for discretion. We need not say she stopped at the first floor. Then she tapped in a peculiar manner at a door, which after being opened to admit her was again fastened, and curiosity penetrated no farther. They used the same precautions in leaving as in entering the house. The lady always left first, and as soon as she had stepped into her carriage, it drove away, sometimes towards the right hand, sometimes to the left; then about twenty minutes afterwards the gentleman would also leave, buried in his cravat or concealed by his handkerchief. 

 The day after Monte Cristo had called upon Danglars, the mysterious lodger entered at ten o'clock in the morning instead of four in the afternoon. Almost directly afterwards, without the usual interval of time, a cab arrived, and the veiled lady ran hastily upstairs. The door opened, but before it could be closed, the lady exclaimed: 

 <Oh, Lucien—oh, my friend!> 

 The concierge therefore heard for the first time that the lodger's name was Lucien; still, as he was the very perfection of a door-keeper, he made up his mind not to tell his wife. 

 <Well, what is the matter, my dear?> asked the gentleman whose name the lady's agitation revealed; <tell me what is the matter.> 

 <Oh, Lucien, can I confide in you?> 

 <Of course, you know you can do so. But what can be the matter? Your note of this morning has completely bewildered me. This precipitation—this unusual appointment. Come, ease me of my anxiety, or else frighten me at once.> 

 <Lucien, a great event has happened!> said the lady, glancing inquiringly at Lucien,—<M. Danglars left last night!> 

 <Left?—M. Danglars left? Where has he gone?> 

 <I do not know.> 

 <What do you mean? Has he gone intending not to return?> 

 <Undoubtedly;—at ten o'clock at night his horses took him to the barrier of Charenton; there a post-chaise was waiting for him—he entered it with his valet de chambre, saying that he was going to Fontainebleau.> 

 <Then what did you mean\longdash> 

 <Stay—he left a letter for me.> 

 <A letter?> 

 <Yes; read it.> 

 And the baroness took from her pocket a letter which she gave to Debray. Debray paused a moment before reading, as if trying to guess its contents, or perhaps while making up his mind how to act, whatever it might contain. No doubt his ideas were arranged in a few minutes, for he began reading the letter which caused so much uneasiness in the heart of the baroness, and which ran as follows: 

 <<Madame and most faithful wife.>> 

 Debray mechanically stopped and looked at the baroness, whose face became covered with blushes. 

 <Read,> she said. 

 Debray continued: 

 “<When you receive this, you will no longer have a husband. Oh, you need not be alarmed, you will only have lost him as you have lost your daughter; I mean that I shall be travelling on one of the thirty or forty roads leading out of France. I owe you some explanations for my conduct, and as you are a woman that can perfectly understand me, I will give them. Listen, then. I received this morning five millions which I paid away; almost directly afterwards another demand for the same sum was presented to me; I put this creditor off till tomorrow and I intend leaving today, to escape that tomorrow, which would be rather too unpleasant for me to endure. You understand this, do you not, my most precious wife? I say you understand this, because you are as conversant with my affairs as I am; indeed, I think you understand them better, since I am ignorant of what has become of a considerable portion of my fortune, once very tolerable, while I am sure, madame, that you know perfectly well. For women have infallible instincts; they can even explain the marvellous by an algebraic calculation they have invented; but I, who only understand my own figures, know nothing more than that one day these figures deceived me. Have you admired the rapidity of my fall? Have you been slightly dazzled at the sudden fusion of my ingots? I confess I have seen nothing but the fire; let us hope you have found some gold among the ashes. With this consoling idea, I leave you, madame, and most prudent wife, without any conscientious reproach for abandoning you; you have friends left, and the ashes I have already mentioned, and above all the liberty I hasten to restore to you. And here, madame, I must add another word of explanation. So long as I hoped you were working for the good of our house and for the fortune of our daughter, I philosophically closed my eyes; but as you have transformed that house into a vast ruin I will not be the foundation of another man>s fortune. You were rich when I married you, but little respected. Excuse me for speaking so very candidly, but as this is intended only for ourselves, I do not see why I should weigh my words. I have augmented our fortune, and it has continued to increase during the last fifteen years, till extraordinary and unexpected catastrophes have suddenly overturned it,—without any fault of mine, I can honestly declare. You, madame, have only sought to increase your own, and I am convinced that you have succeeded. I leave you, therefore, as I took you,—rich, but little respected. Adieu! I also intend from this time to work on my own account. Accept my acknowledgments for the example you have set me, and which I intend following. 

 “‘Your very devoted husband, 

 <<Baron Danglars.>> 

 The baroness had watched Debray while he read this long and painful letter, and saw him, notwithstanding his self-control, change colour once or twice. When he had ended the perusal, he folded the letter and resumed his pensive attitude. 

 <Well?> asked Madame Danglars, with an anxiety easy to be understood. 

 <Well, madame?> unhesitatingly repeated Debray. 

 <With what ideas does that letter inspire you?> 

 <Oh, it is simple enough, madame; it inspires me with the idea that M. Danglars has left suspiciously.> 

 <Certainly; but is this all you have to say to me?> 

 <I do not understand you,> said Debray with freezing coldness. 

 <He is gone! Gone, never to return!> 

 <Oh, madame, do not think that!> 

 <I tell you he will never return. I know his character; he is inflexible in any resolutions formed for his own interests. If he could have made any use of me, he would have taken me with him; he leaves me in Paris, as our separation will conduce to his benefit;—therefore he has gone, and I am free forever,> added Madame Danglars, in the same supplicating tone. 

 Debray, instead of answering, allowed her to remain in an attitude of nervous inquiry. 

 <Well?> she said at length, <do you not answer me?> 

 <I have but one question to ask you,—what do you intend to do?> 

 <I was going to ask you,> replied the baroness with a beating heart. 

 <Ah, then, you wish to ask advice of me?> 

 <Yes; I do wish to ask your advice,> said Madame Danglars with anxious expectation. 

 <Then if you wish to take my advice,> said the young man coldly, <I would recommend you to travel.> 

 <To travel!> she murmured. 

 <Certainly; as M. Danglars says, you are rich, and perfectly free. In my opinion, a withdrawal from Paris is absolutely necessary after the double catastrophe of Mademoiselle Danglars' broken contract and M. Danglars' disappearance. The world will think you abandoned and poor, for the wife of a bankrupt would never be forgiven, were she to keep up an appearance of opulence. You have only to remain in Paris for about a fortnight, telling the world you are abandoned, and relating the details of this desertion to your best friends, who will soon spread the report. Then you can quit your house, leaving your jewels and giving up your jointure, and everyone's mouth will be filled with praises of your disinterestedness. They will know you are deserted, and think you also poor, for I alone know your real financial position, and am quite ready to give up my accounts as an honest partner.> 

 The dread with which the pale and motionless baroness listened to this, was equalled by the calm indifference with which Debray had spoken. 

 <Deserted?> she repeated; <ah, yes, I am, indeed, deserted! You are right, sir, and no one can doubt my position.> 

 These were the only words that this proud and violently enamoured woman could utter in response to Debray.  <But then you are rich,—very rich, indeed,> continued Debray, taking out some papers from his pocket-book, which he spread upon the table. Madame Danglars did not see them; she was engaged in stilling the beatings of her heart, and restraining the tears which were ready to gush forth. At length a sense of dignity prevailed, and if she did not entirely master her agitation, she at least succeeded in preventing the fall of a single tear. 

 <Madame,> said Debray, <it is nearly six months since we have been associated. You furnished a principal of 100,000 francs. Our partnership began in the month of April. In May we commenced operations, and in the course of the month gained 450,000 francs. In June the profit amounted to 900,000. In July we added 1,700,000 francs,—it was, you know, the month of the Spanish bonds. In August we lost 300,000 francs at the beginning of the month, but on the 13th we made up for it, and we now find that our accounts, reckoning from the first day of partnership up to yesterday, when I closed them, showed a capital of 2,400,000 francs, that is, 1,200,000 for each of us. Now, madame,> said Debray, delivering up his accounts in the methodical manner of a stockbroker, <there are still 80,000 francs, the interest of this money, in my hands.> 

 <But,> said the baroness, <I thought you never put the money out to interest.> 

 <Excuse me, madame,> said Debray coldly, “I had your permission to do so, and I have made use of it. There are, then, 40,000 francs for your share, besides the 100,000 you furnished me to begin with, making in all 1,340,000 francs for your portion. Now, madame, I took the precaution of drawing out your money the day before yesterday; it is not long ago, you see, and I was in continual expectation of being called on to deliver up my accounts. There is your money,—half in bank-notes, the other half in checks payable to bearer. I say \textit{there}, for as I did not consider my house safe enough, or lawyers sufficiently discreet, and as landed property carries evidence with it, and moreover since you have no right to possess anything independent of your husband, I have kept this sum, now your whole fortune, in a chest concealed under that closet, and for greater security I myself concealed it there. 

 <Now, madame,> continued Debray, first opening the closet, then the chest;—<now, madame, here are 800 notes of 1,000 francs each, resembling, as you see, a large book bound in iron; to this I add a certificate in the funds of 25,000 francs; then, for the odd cash, making I think about 110,000 francs, here is a check upon my banker, who, not being M. Danglars, will pay you the amount, you may rest assured.> 

 Madame Danglars mechanically took the check, the bond, and the heap of bank-notes. This enormous fortune made no great appearance on the table. Madame Danglars, with tearless eyes, but with her breast heaving with concealed emotion, placed the bank-notes in her bag, put the certificate and check into her pocket-book, and then, standing pale and mute, awaited one kind word of consolation. 

 But she waited in vain. 

 <Now, madame,> said Debray, <you have a splendid fortune, an income of about 60,000 livres a year, which is enormous for a woman who cannot keep an establishment here for a year, at least. You will be able to indulge all your fancies; besides, should you find your income insufficient, you can, for the sake of the past, madame, make use of mine; and I am ready to offer you all I possess, on loan.> 

 <Thank you, sir—thank you,> replied the baroness; <you forget that what you have just paid me is much more than a poor woman requires, who intends for some time, at least, to retire from the world.> 

 Debray was, for a moment, surprised, but immediately recovering himself, he bowed with an air which seemed to say, <As you please, madame.> 

 Madame Danglars had until then, perhaps, hoped for something; but when she saw the careless bow of Debray, and the glance by which it was accompanied, together with his significant silence, she raised her head, and without passion or violence or even hesitation, ran downstairs, disdaining to address a last farewell to one who could thus part from her. 

 <Bah,> said Debray, when she had left, <these are fine projects! She will remain at home, read novels, and speculate at cards, since she can no longer do so on the Bourse.> 

 Then taking up his account book, he cancelled with the greatest care all the entries of the amounts he had just paid away. 

 <I have 1,060,000 francs remaining,> he said. <What a pity Mademoiselle de Villefort is dead! She suited me in every respect, and I would have married her.> 

 And he calmly waited until the twenty minutes had elapsed after Madame Danglars' departure before he left the house. During this time he occupied himself in making figures, with his watch by his side. 

 Asmodeus—that diabolical personage, who would have been created by every fertile imagination if Le Sage had not acquired the priority in his great masterpiece—would have enjoyed a singular spectacle, if he had lifted up the roof of the little house in the Rue Saint-Germain-des-Prés, while Debray was casting up his figures. 

 Above the room in which Debray had been dividing two millions and a half with Madame Danglars was another, inhabited by persons who have played too prominent a part in the incidents we have related for their appearance not to create some interest. 

 Mercédès and Albert were in that room. 

 Mercédès was much changed within the last few days; not that even in her days of fortune she had ever dressed with the magnificent display which makes us no longer able to recognize a woman when she appears in a plain and simple attire; nor indeed, had she fallen into that state of depression where it is impossible to conceal the garb of misery; no, the change in Mercédès was that her eye no longer sparkled, her lips no longer smiled, and there was now a hesitation in uttering the words which formerly sprang so fluently from her ready wit. 

 It was not poverty which had broken her spirit; it was not a want of courage which rendered her poverty burdensome. Mercédès, although deposed from the exalted position she had occupied, lost in the sphere she had now chosen, like a person passing from a room splendidly lighted into utter darkness, appeared like a queen, fallen from her palace to a hovel, and who, reduced to strict necessity, could neither become reconciled to the earthen vessels she was herself forced to place upon the table, nor to the humble pallet which had become her bed. 

 The beautiful Catalane and noble countess had lost both her proud glance and charming smile, because she saw nothing but misery around her; the walls were hung with one of the gray papers which economical landlords choose as not likely to show the dirt; the floor was uncarpeted; the furniture attracted the attention to the poor attempt at luxury; indeed, everything offended eyes accustomed to refinement and elegance. 

 Madame de Morcerf had lived there since leaving her house; the continual silence of the spot oppressed her; still, seeing that Albert continually watched her countenance to judge the state of her feelings, she constrained herself to assume a monotonous smile of the lips alone, which, contrasted with the sweet and beaming expression that usually shone from her eyes, seemed like <moonlight on a statue,>—yielding light without warmth. 

 Albert, too, was ill at ease; the remains of luxury prevented him from sinking into his actual position. If he wished to go out without gloves, his hands appeared too white; if he wished to walk through the town, his boots seemed too highly polished. Yet these two noble and intelligent creatures, united by the indissoluble ties of maternal and filial love, had succeeded in tacitly understanding one another, and economizing their stores, and Albert had been able to tell his mother without extorting a change of countenance: 

 <Mother, we have no more money.>  Mercédès had never known misery; she had often, in her youth, spoken of poverty, but between want and necessity, those synonymous words, there is a wide difference. 

 Amongst the Catalans, Mercédès wished for a thousand things, but still she never really wanted any. So long as the nets were good, they caught fish; and so long as they sold their fish, they were able to buy twine for new nets. And then, shut out from friendship, having but one affection, which could not be mixed up with her ordinary pursuits, she thought of herself—of no one but herself. Upon the little she earned she lived as well as she could; now there were two to be supported, and nothing to live upon. 

 Winter approached. Mercédès had no fire in that cold and naked room—she, who was accustomed to stoves which heated the house from the hall to the boudoir; she had not even one little flower—she whose apartment had been a conservatory of costly exotics. But she had her son. Hitherto the excitement of fulfilling a duty had sustained them. Excitement, like enthusiasm, sometimes renders us unconscious to the things of earth. But the excitement had calmed down, and they felt themselves obliged to descend from dreams to reality; after having exhausted the ideal, they found they must talk of the actual. 

 <Mother,> exclaimed Albert, just as Madame Danglars was descending the stairs, <let us reckon our riches, if you please; I want capital to build my plans upon.> 

 <Capital—nothing!> replied Mercédès with a mournful smile. 

 <No, mother,—capital 3,000 francs. And I have an idea of our leading a delightful life upon this 3,000 francs.> 

 <Child!> sighed Mercédès. 

 <Alas, dear mother,> said the young man, <I have unhappily spent too much of your money not to know the value of it. These 3,000 francs are enormous, and I intend building upon this foundation a miraculous certainty for the future.> 

 <You say this, my dear boy; but do you think we ought to accept these 3,000 francs?> said Mercédès, coloring. 

 <I think so,> answered Albert in a firm tone. <We will accept them the more readily, since we have them not here; you know they are buried in the garden of the little house in the Allées de Meilhan, at Marseilles. With 200 francs we can reach Marseilles.> 

 <With 200 francs?—are you sure, Albert?> 

 <Oh, as for that, I have made inquiries respecting the diligences and steamboats, and my calculations are made. You will take your place in the \textit{coupé} to Châlons. You see, mother, I treat you handsomely for thirty-five francs.> 

 Albert then took a pen, and wrote: 
 
 \begin{tabular} {l@{\dotfill}r} 
 
\multicolumn{1}{l}{~} & \multicolumn{1}{r}{Frs.} \\
\textit{Coupé}, thirty-five francs&35.\\
From Châlons to Lyons you will go on by the steamboat&6.\\
From Lyons to Avignon (still by steamboat)&16.\\
From Avignon to Marseilles, seven francs&7.\\
Expenses on the road, about fifty francs&50.\\
Total&114 frs.\\
\end{tabular}
\bigskip

<Let us put down 120,> added Albert, smiling. <You see I am generous, am I not, mother?> 

 <But you, my poor child?> 

 <I? do you not see that I reserve eighty francs for myself? A young man does not require luxuries; besides, I know what travelling is.> 

 <With a post-chaise and valet de chambre?> 

 <Any way, mother.> 

 <Well, be it so. But these 200 francs?> 

 <Here they are, and 200 more besides. See, I have sold my watch for 100 francs, and the guard and seals for 300. How fortunate that the ornaments were worth more than the watch. Still the same story of superfluities! Now I think we are rich, since instead of the 114 francs we require for the journey we find ourselves in possession of 250.> 

 <But we owe something in this house?> 

 <Thirty francs; but I pay that out of my 150 francs,—that is understood,—and as I require only eighty francs for my journey, you see I am overwhelmed with luxury. But that is not all. What do you say to this, mother?> 

 And Albert took out of a little pocket-book with golden clasps, a remnant of his old fancies, or perhaps a tender souvenir from one of the mysterious and veiled ladies who used to knock at his little door,—Albert took out of this pocket-book a note of 1,000 francs. 

 <What is this?> asked Mercédès. 

 <A thousand francs.> 

 <But whence have you obtained them?> 

 <Listen to me, mother, and do not yield too much to agitation.> And Albert, rising, kissed his mother on both cheeks, then stood looking at her. <You cannot imagine, mother, how beautiful I think you!> said the young man, impressed with a profound feeling of filial love. <You are, indeed, the most beautiful and most noble woman I ever saw!> 

 <Dear child!> said Mercédès, endeavouring in vain to restrain a tear which glistened in the corner of her eye. <Indeed, you only wanted misfortune to change my love for you to admiration. I am not unhappy while I possess my son!> 

 <Ah, just so,> said Albert; <here begins the trial. Do you know the decision we have come to, mother?> 

 <Have we come to any?> 

 <Yes; it is decided that you are to live at Marseilles, and that I am to leave for Africa, where I will earn for myself the right to use the name I now bear, instead of the one I have thrown aside.> Mercédès sighed. <Well, mother, I yesterday engaged myself as substitute in the Spahis,>\footnote{The Spahis are French cavalry reserved for service in Africa. } added the young man, lowering his eyes with a certain feeling of shame, for even he was unconscious of the sublimity of his self-abasement. <I thought my body was my own, and that I might sell it. I yesterday took the place of another. I sold myself for more than I thought I was worth,> he added, attempting to smile; <I fetched 2,000 francs.> 

 <Then these 1,000 francs\longdash> said Mercédès, shuddering. 

 <Are the half of the sum, mother; the other will be paid in a year.> 

 Mercédès raised her eyes to heaven with an expression it would be impossible to describe, and tears, which had hitherto been restrained, now yielded to her emotion, and ran down her cheeks. 

 <The price of his blood!> she murmured. 

 <Yes, if I am killed,> said Albert, laughing. <But I assure you, mother, I have a strong intention of defending my person, and I never felt half so strong an inclination to live as I do now.> 

 <Merciful Heavens!> 

 <Besides, mother, why should you make up your mind that I am to be killed? Has Lamoricière, that Ney of the South, been killed? Has Changarnier been killed? Has Bedeau been killed? Has Morrel, whom we know, been killed? Think of your joy, mother, when you see me return with an embroidered uniform! I declare, I expect to look magnificent in it, and chose that regiment only from vanity.> 

 Mercédès sighed while endeavouring to smile; the devoted mother felt that she ought not to allow the whole weight of the sacrifice to fall upon her son. 

 <Well, now you understand, mother!> continued Albert; <here are more than 4,000 francs settled on you; upon these you can live at least two years.> 

 <Do you think so?> said Mercédès. 

 These words were uttered in so mournful a tone that their real meaning did not escape Albert; he felt his heart beat, and taking his mother's hand within his own he said, tenderly: 

 <Yes, you will live!> 

 <I shall live!—then you will not leave me, Albert?> 

 <Mother, I must go,> said Albert in a firm, calm voice; <you love me too well to wish me to remain useless and idle with you; besides, I have signed.>  <You will obey your own wish and the will of Heaven!> 

 <Not my own wish, mother, but reason—necessity. Are we not two despairing creatures? What is life to you?—Nothing. What is life to me?—Very little without you, mother; for believe me, but for you I should have ceased to live on the day I doubted my father and renounced his name. Well, I will live, if you promise me still to hope; and if you grant me the care of your future prospects, you will redouble my strength. Then I will go to the governor of Algeria; he has a royal heart, and is essentially a soldier; I will tell him my gloomy story. I will beg him to turn his eyes now and then towards me, and if he keep his word and interest himself for me, in six months I shall be an officer, or dead. If I am an officer, your fortune is certain, for I shall have money enough for both, and, moreover, a name we shall both be proud of, since it will be our own. If I am killed—well then mother, you can also die, and there will be an end of our misfortunes.> 

 <It is well,> replied Mercédès, with her eloquent glance; <you are right, my love; let us prove to those who are watching our actions that we are worthy of compassion.> 

 <But let us not yield to gloomy apprehensions,> said the young man; <I assure you we are, or rather we shall be, very happy. You are a woman at once full of spirit and resignation; I have become simple in my tastes, and am without passion, I hope. Once in service, I shall be rich—once in M. Dantès' house, you will be at rest. Let us strive, I beseech you,—let us strive to be cheerful.> 

 <Yes, let us strive, for you ought to live, and to be happy, Albert.> 

 <And so our division is made, mother,> said the young man, affecting ease of mind. <We can now part; come, I shall engage your passage.> 

 <And you, my dear boy?> 

 <I shall stay here for a few days longer; we must accustom ourselves to parting. I want recommendations and some information relative to Africa. I will join you again at Marseilles.> 

 <Well, be it so—let us part,> said Mercédès, folding around her shoulders the only shawl she had taken away, and which accidentally happened to be a valuable black cashmere. Albert gathered up his papers hastily, rang the bell to pay the thirty francs he owed to the landlord, and offering his arm to his mother, they descended the stairs. 

 Someone was walking down before them, and this person, hearing the rustling of a silk dress, turned around. <Debray!> muttered Albert. 

 <You, Morcerf?> replied the secretary, resting on the stairs. Curiosity had vanquished the desire of preserving his \textit{incognito}, and he was recognized. It was, indeed, strange in this unknown spot to find the young man whose misfortunes had made so much noise in Paris. 

 <Morcerf!> repeated Debray. Then noticing in the dim light the still youthful and veiled figure of Madame de Morcerf: 

 <Pardon me,> he added with a smile, <I leave you, Albert.> Albert understood his thoughts. 

 <Mother,> he said, turning towards Mercédès, <this is M. Debray, secretary of the Minister for the Interior, once a friend of mine.> 

 <How once?> stammered Debray; <what do you mean?> 

 <I say so, M. Debray, because I have no friends now, and I ought not to have any. I thank you for having recognized me, sir.> Debray stepped forward, and cordially pressed the hand of his interlocutor. 

 <Believe me, dear Albert,> he said, with all the emotion he was capable of feeling,—<believe me, I feel deeply for your misfortunes, and if in any way I can serve you, I am yours.> 

 <Thank you, sir,> said Albert, smiling. <In the midst of our misfortunes, we are still rich enough not to require assistance from anyone. We are leaving Paris, and when our journey is paid, we shall have 5,000 francs left.> 

 The blood mounted to the temples of Debray, who held a million in his pocket-book, and unimaginative as he was he could not help reflecting that the same house had contained two women, one of whom, justly dishonored, had left it poor with 1,500,000 francs under her cloak, while the other, unjustly stricken, but sublime in her misfortune, was yet rich with a few deniers. This parallel disturbed his usual politeness, the philosophy he witnessed appalled him, he muttered a few words of general civility and ran downstairs. 

 That day the minister's clerks and the subordinates had a great deal to put up with from his ill-humor. But that same night, he found himself the possessor of a fine house, situated on the Boulevard de la Madeleine, and an income of 50,000 livres. 

 The next day, just as Debray was signing the deed, that is about five o'clock in the afternoon, Madame de Morcerf, after having affectionately embraced her son, entered the \textit{coupé} of the diligence, which closed upon her. 

 A man was hidden in Lafitte's banking-house, behind one of the little arched windows which are placed above each desk; he saw Mercédès enter the diligence, and he also saw Albert withdraw. Then he passed his hand across his forehead, which was clouded with doubt. 

 <Alas,> he exclaimed, <how can I restore the happiness I have taken away from these poor innocent creatures? God help me!> 