\chapter{The Villefort Family Vault} 

 \lettrine{T}{wo} days after, a considerable crowd was assembled, towards ten o'clock in the morning, around the door of M. de Villefort's house, and a long file of mourning-coaches and private carriages extended along the Faubourg Saint-Honoré and the Rue de la Pépinière. Among them was one of a very singular form, which appeared to have come from a distance. It was a kind of covered wagon, painted black, and was one of the first to arrive. Inquiry was made, and it was ascertained that, by a strange coincidence, this carriage contained the corpse of the Marquis de Saint-Méran, and that those who had come thinking to attend one funeral would follow two. Their number was great. The Marquis de Saint-Méran, one of the most zealous and faithful dignitaries of Louis XVIII. and King Charles X., had preserved a great number of friends, and these, added to the personages whom the usages of society gave Villefort a claim on, formed a considerable body. 

 Due information was given to the authorities, and permission obtained that the two funerals should take place at the same time. A second hearse, decked with the same funereal pomp, was brought to M. de Villefort's door, and the coffin removed into it from the post-wagon. The two bodies were to be interred in the cemetery of Père-Lachaise, where M. de Villefort had long since had a tomb prepared for the reception of his family. The remains of poor Renée were already deposited there, and now, after ten years of separation, her father and mother were to be reunited with her. 

 The Parisians, always curious, always affected by funereal display, looked on with religious silence while the splendid procession accompanied to their last abode two of the number of the old aristocracy—the greatest protectors of commerce and sincere devotees to their principles. 

 In one of the mourning-coaches Beauchamp, Debray, and Château-Renaud were talking of the very sudden death of the marchioness. 

 <I saw Madame de Saint-Méran only last year at Marseilles, when I was coming back from Algiers,> said Château-Renaud; <she looked like a woman destined to live to be a hundred years old, from her apparent sound health and great activity of mind and body. How old was she?> 

 <Franz assured me,> replied Albert, <that she was sixty-six years old. But she has not died of old age, but of grief; it appears that since the death of the marquis, which affected her very deeply, she has not completely recovered her reason.> 

 <But of what disease, then, did she die?> asked Debray. 

 <It is said to have been a congestion of the brain, or apoplexy, which is the same thing, is it not?> 

 <Nearly.> 

 <It is difficult to believe that it was apoplexy,> said Beauchamp. <Madame de Saint-Méran, whom I once saw, was short, of slender form, and of a much more nervous than sanguine temperament; grief could hardly produce apoplexy in such a constitution as that of Madame de Saint-Méran.> 

 <At any rate,> said Albert, <whatever disease or doctor may have killed her, M. de Villefort, or rather, Mademoiselle Valentine,—or, still rather, our friend Franz, inherits a magnificent fortune, amounting, I believe, to 80,000 livres per annum.> 

 <And this fortune will be doubled at the death of the old Jacobin, Noirtier.> 

 <That is a tenacious old grandfather,> said Beauchamp. <\textit{Tenacem propositi virum}. I think he must have made an agreement with death to outlive all his heirs, and he appears likely to succeed. He resembles the old Conventionalist of '93, who said to Napoleon, in 1814, <You bend because your empire is a young stem, weakened by rapid growth. Take the Republic for a tutor; let us return with renewed strength to the battle-field, and I promise you 500,000 soldiers, another Marengo, and a second Austerlitz. Ideas do not become extinct, sire; they slumber sometimes, but only revive the stronger before they sleep entirely.>> 

 <Ideas and men appeared the same to him,> said Albert. <One thing only puzzles me, namely, how Franz d'Épinay will like a grandfather who cannot be separated from his wife. But where is Franz?> 

 <In the first carriage, with M. de Villefort, who considers him already as one of the family.>  Such was the conversation in almost all the carriages; these two sudden deaths, so quickly following each other, astonished everyone, but no one suspected the terrible secret which M. d'Avrigny had communicated, in his nocturnal walk to M. de Villefort. They arrived in about an hour at the cemetery; the weather was mild, but dull, and in harmony with the funeral ceremony. Among the groups which flocked towards the family vault, Château-Renaud recognized Morrel, who had come alone in a cabriolet, and walked silently along the path bordered with yew-trees. 

 <You here?> said Château-Renaud, passing his arms through the young captain's; <are you a friend of Villefort's? How is it that I have never met you at his house?> 

 <I am no acquaintance of M. de Villefort's,> answered Morrel, <but I was of Madame de Saint-Méran.> Albert came up to them at this moment with Franz. 

 <The time and place are but ill-suited for an introduction.> said Albert; <but we are not superstitious. M. Morrel, allow me to present to you M. Franz d'Épinay, a delightful travelling companion, with whom I made the tour of Italy. My dear Franz, M. Maximilian Morrel, an excellent friend I have acquired in your absence, and whose name you will hear me mention every time I make any allusion to affection, wit, or amiability.> 

 Morrel hesitated for a moment; he feared it would be hypocritical to accost in a friendly manner the man whom he was tacitly opposing, but his oath and the gravity of the circumstances recurred to his memory; he struggled to conceal his emotion and bowed to Franz. 

 <Mademoiselle de Villefort is in deep sorrow, is she not?> said Debray to Franz. 

 <Extremely,> replied he; <she looked so pale this morning, I scarcely knew her.> 

 These apparently simple words pierced Morrel to the heart. This man had seen Valentine, and spoken to her! The young and high-spirited officer required all his strength of mind to resist breaking his oath. He took the arm of Château-Renaud, and turned towards the vault, where the attendants had already placed the two coffins. 

 <This is a magnificent habitation,> said Beauchamp, looking towards the mausoleum; <a summer and winter palace. You will, in turn, enter it, my dear d'Épinay, for you will soon be numbered as one of the family. I, as a philosopher, should like a little country-house, a cottage down there under the trees, without so many free-stones over my poor body. In dying, I will say to those around me what Voltaire wrote to Piron: <\textit{Eo rus}, and all will be over.> But come, Franz, take courage, your wife is an heiress.> 

 <Indeed, Beauchamp, you are unbearable. Politics has made you laugh at everything, and political men have made you disbelieve everything. But when you have the honour of associating with ordinary men, and the pleasure of leaving politics for a moment, try to find your affectionate heart, which you leave with your stick when you go to the Chamber.> 

 <But tell me,> said Beauchamp, <what is life? Is it not a halt in Death's anteroom?>  <I am prejudiced against Beauchamp,> said Albert, drawing Franz away, and leaving the former to finish his philosophical dissertation with Debray. 

 The Villefort vault formed a square of white stones, about twenty feet high; an interior partition separated the two families, and each apartment had its entrance door. Here were not, as in other tombs, ignoble drawers, one above another, where thrift bestows its dead and labels them like specimens in a museum; all that was visible within the bronze gates was a gloomy-looking room, separated by a wall from the vault itself. The two doors before mentioned were in the middle of this wall, and enclosed the Villefort and Saint-Méran coffins. There grief might freely expend itself without being disturbed by the trifling loungers who came from a picnic party to visit Père-Lachaise, or by lovers who make it their rendezvous. 

 The two coffins were placed on trestles previously prepared for their reception in the right-hand crypt belonging to the Saint-Méran family. Villefort, Franz, and a few near relatives alone entered the sanctuary. 

 As the religious ceremonies had all been performed at the door, and there was no address given, the party all separated; Château-Renaud, Albert, and Morrel, went one way, and Debray and Beauchamp the other. Franz remained with M. de Villefort; at the gate of the cemetery Morrel made an excuse to wait; he saw Franz and M. de Villefort get into the same mourning-coach, and thought this meeting forboded evil. He then returned to Paris, and although in the same carriage with Château-Renaud and Albert, he did not hear one word of their conversation. 

 As Franz was about to take leave of M. de Villefort, <When shall I see you again?> said the latter. 

 <At what time you please, sir,> replied Franz. 

 <As soon as possible.> 

 <I am at your command, sir; shall we return together?> 

 <If not unpleasant to you.> 

 <On the contrary, I shall feel much pleasure.> 

 Thus, the future father and son-in-law stepped into the same carriage, and Morrel, seeing them pass, became uneasy. Villefort and Franz returned to the Faubourg Saint-Honoré. The procureur, without going to see either his wife or his daughter, went at once to his study, and, offering the young man a chair: 

 <M. d'Épinay,> said he, <allow me to remind you at this moment,—which is perhaps not so ill-chosen as at first sight may appear, for obedience to the wishes of the departed is the first offering which should be made at their tomb,—allow me then to remind you of the wish expressed by Madame de Saint-Méran on her death-bed, that Valentine's wedding might not be deferred. You know the affairs of the deceased are in perfect order, and her will bequeaths to Valentine the entire property of the Saint-Méran family; the notary showed me the documents yesterday, which will enable us to draw up the contract immediately. You may call on the notary, M. Deschamps, Place Beauveau, Faubourg Saint-Honoré, and you have my authority to inspect those deeds.> 

 <Sir,> replied M. d'Épinay, <it is not, perhaps, the moment for Mademoiselle Valentine, who is in deep distress, to think of a husband; indeed, I fear\longdash>  <Valentine will have no greater pleasure than that of fulfilling her grandmother's last injunctions; there will be no obstacle from that quarter, I assure you.> 

 <In that case,> replied Franz, <as I shall raise none, you may make arrangements when you please; I have pledged my word, and shall feel pleasure and happiness in adhering to it.> 

 <Then,> said Villefort, <nothing further is required. The contract was to have been signed three days since; we shall find it all ready, and can sign it today.> 

 <But the mourning?> said Franz, hesitating. 

 <Don't be uneasy on that score,> replied Villefort; <no ceremony will be neglected in my house. Mademoiselle de Villefort may retire during the prescribed three months to her estate of Saint-Méran; I say hers, for she inherits it today. There, after a few days, if you like, the civil marriage shall be celebrated without pomp or ceremony. Madame de Saint-Méran wished her daughter should be married there. When that is over, you, sir, can return to Paris, while your wife passes the time of her mourning with her mother-in-law.> 

 <As you please, sir,> said Franz. 

 <Then,> replied M. de Villefort, <have the kindness to wait half an hour; Valentine shall come down into the drawing-room. I will send for M. Deschamps; we will read and sign the contract before we separate, and this evening Madame de Villefort shall accompany Valentine to her estate, where we will rejoin them in a week.> 

 <Sir,> said Franz, <I have one request to make.> 

 <What is it?> 

 <I wish Albert de Morcerf and Raoul de Château-Renaud to be present at this signature; you know they are my witnesses.> 

 <Half an hour will suffice to apprise them; will you go for them yourself, or shall you send?> 

 <I prefer going, sir.> 

 <I shall expect you, then, in half an hour, baron, and Valentine will be ready.> 

 Franz bowed and left the room. Scarcely had the door closed, when M. de Villefort sent to tell Valentine to be ready in the drawing-room in half an hour, as he expected the notary and M. d'Épinay and his witnesses. The news caused a great sensation throughout the house; Madame de Villefort would not believe it, and Valentine was thunderstruck. She looked around for help, and would have gone down to her grandfather's room, but on the stairs she met M. de Villefort, who took her arm and led her into the drawing-room. In the anteroom, Valentine met Barrois, and looked despairingly at the old servant. A moment later, Madame de Villefort entered the drawing-room with her little Edward. It was evident that she had shared the grief of the family, for she was pale and looked fatigued. She sat down, took Edward on her knees, and from time to time pressed this child, on whom her affections appeared centred, almost convulsively to her bosom. 

 Two carriages were soon heard to enter the courtyard. One was the notary's; the other, that of Franz and his friends. In a moment the whole party was assembled. Valentine was so pale one might trace the blue veins from her temples, round her eyes and down her cheeks. Franz was deeply affected. Château-Renaud and Albert looked at each other with amazement; the ceremony which was just concluded had not appeared more sorrowful than did that which was about to begin. Madame de Villefort had placed herself in the shadow behind a velvet curtain, and as she constantly bent over her child, it was difficult to read the expression of her face. M. de Villefort was, as usual, unmoved. 

 The notary, after having, according to the customary method, arranged the papers on the table, taken his place in an armchair, and raised his spectacles, turned towards Franz: 

 <Are you M. Franz de Quesnel, baron d'Épinay?> asked he, although he knew it perfectly. 

 <Yes, sir,> replied Franz. The notary bowed. 

 <I have, then, to inform you, sir, at the request of M. de Villefort, that your projected marriage with Mademoiselle de Villefort has changed the feeling of M. Noirtier towards his grandchild, and that he disinherits her entirely of the fortune he would have left her. Let me hasten to add,> continued he, <that the testator, having only the right to alienate a part of his fortune, and having alienated it all, the will will not bear scrutiny, and is declared null and void.> 

 <Yes.> said Villefort; <but I warn M. d'Épinay, that during my life-time my father's will shall never be questioned, my position forbidding any doubt to be entertained.>  <Sir,> said Franz, <I regret much that such a question has been raised in the presence of Mademoiselle Valentine; I have never inquired the amount of her fortune, which, however limited it may be, exceeds mine. My family has sought consideration in this alliance with M. de Villefort; all I seek is happiness.> 

 Valentine imperceptibly thanked him, while two silent tears rolled down her cheeks. 

 <Besides, sir,> said Villefort, addressing himself to his future son-in-law, <excepting the loss of a portion of your hopes, this unexpected will need not personally wound you; M. Noirtier's weakness of mind sufficiently explains it. It is not because Mademoiselle Valentine is going to marry you that he is angry, but because she will marry, a union with any other would have caused him the same sorrow. Old age is selfish, sir, and Mademoiselle de Villefort has been a faithful companion to M. Noirtier, which she cannot be when she becomes the Baroness d'Épinay. My father's melancholy state prevents our speaking to him on any subjects, which the weakness of his mind would incapacitate him from understanding, and I am perfectly convinced that at the present time, although, he knows that his granddaughter is going to be married, M. Noirtier has even forgotten the name of his intended grandson.> M. de Villefort had scarcely said this, when the door opened, and Barrois appeared. 

 <Gentlemen,> said he, in a tone strangely firm for a servant speaking to his masters under such solemn circumstances,—<gentlemen, M. Noirtier de Villefort wishes to speak immediately to M. Franz de Quesnel, baron d'Épinay.> He, as well as the notary, that there might be no mistake in the person, gave all his titles to the bridegroom elect. 

 Villefort started, Madame de Villefort let her son slip from her knees, Valentine rose, pale and dumb as a statue. Albert and Château-Renaud exchanged a second look, more full of amazement than the first. The notary looked at Villefort. 

 <It is impossible,> said the procureur. <M. d'Épinay cannot leave the drawing-room at present.> 

 <It is at this moment,> replied Barrois with the same firmness, <that M. Noirtier, my master, wishes to speak on important subjects to M. Franz d'Épinay.> 

 <Grandpapa Noirtier can speak now, then,> said Edward, with his habitual quickness. However, his remark did not make Madame de Villefort even smile, so much was every mind engaged, and so solemn was the situation. 

 <Tell M. Nortier,> resumed Villefort, <that what he demands is impossible.> 

 <Then, M. Nortier gives notice to these gentlemen,> replied Barrois, <that he will give orders to be carried to the drawing-room.> 

 Astonishment was at its height. Something like a smile was perceptible on Madame de Villefort's countenance. Valentine instinctively raised her eyes, as if to thank heaven. 

 <Pray go, Valentine,> said; M. de Villefort, <and see what this new fancy of your grandfather's is.> Valentine rose quickly, and was hastening joyfully towards the door, when M. de Villefort altered his intention. 

 <Stop,> said he; <I will go with you.> 

 <Excuse me, sir,> said Franz, <since M. Noirtier sent for me, I am ready to attend to his wish; besides, I shall be happy to pay my respects to him, not having yet had the honour of doing so.> 

 <Pray, sir,> said Villefort with marked uneasiness, <do not disturb yourself.>  <Forgive me, sir,> said Franz in a resolute tone. <I would not lose this opportunity of proving to M. Noirtier how wrong it would be of him to encourage feelings of dislike to me, which I am determined to conquer, whatever they may be, by my devotion.> 

 And without listening to Villefort he arose, and followed Valentine, who was running downstairs with the joy of a shipwrecked mariner who finds a rock to cling to. M. de Villefort followed them. Château-Renaud and Morcerf exchanged a third look of still increasing wonder. 