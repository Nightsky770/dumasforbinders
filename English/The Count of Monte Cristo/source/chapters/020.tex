\chapter{The Cemetery of the Château d'If}

 \lettrine{O}{n} the bed, at full length, and faintly illuminated by the pale light that came from the window, lay a sack of canvas, and under its rude folds was stretched a long and stiffened form; it was Faria's last winding-sheet,—a winding-sheet which, as the turnkey said, cost so little. Everything was in readiness. A barrier had been placed between Dantès and his old friend. No longer could Edmond look into those wide-open eyes which had seemed to be penetrating the mysteries of death; no longer could he clasp the hand which had done so much to make his existence blessed. Faria, the beneficent and cheerful companion, with whom he was accustomed to live so intimately, no longer breathed. He seated himself on the edge of that terrible bed, and fell into melancholy and gloomy reverie. 

 Alone! he was alone again! again condemned to silence—again face to face with nothingness! Alone!—never again to see the face, never again to hear the voice of the only human being who united him to earth! Was not Faria's fate the better, after all—to solve the problem of life at its source, even at the risk of horrible suffering? 

 The idea of suicide, which his friend had driven away and kept away by his cheerful presence, now hovered like a phantom over the abbé's dead body. 

 <If I could die,> he said, <I should go where he goes, and should assuredly find him again. But how to die? It is very easy,> he went on with a smile; <I will remain here, rush on the first person that opens the door, strangle him, and then they will guillotine me.> 

 But excessive grief is like a storm at sea, where the frail bark is tossed from the depths to the top of the wave. Dantès recoiled from the idea of so infamous a death, and passed suddenly from despair to an ardent desire for life and liberty. 

 <Die? oh, no,> he exclaimed—<not die now, after having lived and suffered so long and so much! Die? yes, had I died years ago; but now to die would be, indeed, to give way to the sarcasm of destiny. No, I want to live; I shall struggle to the very last; I will yet win back the happiness of which I have been deprived. Before I die I must not forget that I have my executioners to punish, and perhaps, too, who knows, some friends to reward. Yet they will forget me here, and I shall die in my dungeon like Faria.> 

 As he said this, he became silent and gazed straight before him like one overwhelmed with a strange and amazing thought. Suddenly he arose, lifted his hand to his brow as if his brain were giddy, paced twice or thrice round the dungeon, and then paused abruptly by the bed. 

 <Just God!> he muttered, <whence comes this thought? Is it from thee? Since none but the dead pass freely from this dungeon, let me take the place of the dead!> 

 Without giving himself time to reconsider his decision, and, indeed, that he might not allow his thoughts to be distracted from his desperate resolution, he bent over the appalling shroud, opened it with the knife which Faria had made, drew the corpse from the sack, and bore it along the tunnel to his own chamber, laid it on his couch, tied around its head the rag he wore at night around his own, covered it with his counterpane, once again kissed the ice-cold brow, and tried vainly to close the resisting eyes, which glared horribly, turned the head towards the wall, so that the jailer might, when he brought the evening meal, believe that he was asleep, as was his frequent custom; entered the tunnel again, drew the bed against the wall, returned to the other cell, took from the hiding-place the needle and thread, flung off his rags, that they might feel only naked flesh beneath the coarse canvas, and getting inside the sack, placed himself in the posture in which the dead body had been laid, and sewed up the mouth of the sack from the inside. 

 He would have been discovered by the beating of his heart, if by any mischance the jailers had entered at that moment. Dantès might have waited until the evening visit was over, but he was afraid that the governor would change his mind, and order the dead body to be removed earlier. In that case his last hope would have been destroyed. 

 Now his plans were fully made, and this is what he intended to do. If while he was being carried out the grave-diggers should discover that they were bearing a live instead of a dead body, Dantès did not intend to give them time to recognize him, but with a sudden cut of the knife, he meant to open the sack from top to bottom, and, profiting by their alarm, escape; if they tried to catch him, he would use his knife to better purpose.  If they took him to the cemetery and laid him in a grave, he would allow himself to be covered with earth, and then, as it was night, the grave-diggers could scarcely have turned their backs before he would have worked his way through the yielding soil and escaped. He hoped that the weight of earth would not be so great that he could not overcome it. If he was detected in this and the earth proved too heavy, he would be stifled, and then—so much the better, all would be over. 

 Dantès had not eaten since the preceding evening, but he had not thought of hunger, nor did he think of it now. His situation was too precarious to allow him even time to reflect on any thought but one. 

 The first risk that Dantès ran was, that the jailer, when he brought him his supper at seven o'clock, might perceive the change that had been made; fortunately, twenty times at least, from misanthropy or fatigue, Dantès had received his jailer in bed, and then the man placed his bread and soup on the table, and went away without saying a word. This time the jailer might not be as silent as usual, but speak to Dantès, and seeing that he received no reply, go to the bed, and thus discover all. 

 When seven o'clock came, Dantès' agony really began. His hand placed upon his heart was unable to redress its throbbings, while, with the other he wiped the perspiration from his temples. From time to time chills ran through his whole body, and clutched his heart in a grasp of ice. Then he thought he was going to die. Yet the hours passed on without any unusual disturbance, and Dantès knew that he had escaped the first peril. It was a good augury. 

 At length, about the hour the governor had appointed, footsteps were heard on the stairs. Edmond felt that the moment had arrived, summoned up all his courage, held his breath, and would have been happy if at the same time he could have repressed the throbbing of his veins. The footsteps—they were double—paused at the door—and Dantès guessed that the two grave-diggers had come to seek him—this idea was soon converted into certainty, when he heard the noise they made in putting down the hand-bier. 

 The door opened, and a dim light reached Dantès' eyes through the coarse sack that covered him; he saw two shadows approach his bed, a third remaining at the door with a torch in its hand. The two men, approaching the ends of the bed, took the sack by its extremities. 

 <He's heavy, though, for an old and thin man,> said one, as he raised the head. 

 <They say every year adds half a pound to the weight of the bones,> said another, lifting the feet. 

 <Have you tied the knot?> inquired the first speaker. 

 <What would be the use of carrying so much more weight?> was the reply, <I can do that when we get there.> 

 <Yes, you're right,> replied the companion. 

 <What's the knot for?> thought Dantès. 

 They deposited the supposed corpse on the bier. Edmond stiffened himself in order to play the part of a dead man, and then the party, lighted by the man with the torch, who went first, ascended the stairs. Suddenly he felt the fresh and sharp night air, and Dantès knew that the mistral was blowing. It was a sensation in which pleasure and pain were strangely mingled. 

 The bearers went on for twenty paces, then stopped, putting the bier down on the ground. One of them went away, and Dantès heard his shoes striking on the pavement. 

 <Where am I?> he asked himself. 

 <Really, he is by no means a light load!> said the other bearer, sitting on the edge of the hand-barrow. 

 Dantès' first impulse was to escape, but fortunately he did not attempt it. 

 <Give us a light,> said the other bearer, <or I shall never find what I am looking for.> 

 The man with the torch complied, although not asked in the most polite terms. 

 <What can he be looking for?> thought Edmond. <The spade, perhaps.> 

 An exclamation of satisfaction indicated that the grave-digger had found the object of his search. <Here it is at last,> he said, <not without some trouble, though.> 

 <Yes,> was the answer, <but it has lost nothing by waiting.> 

 As he said this, the man came towards Edmond, who heard a heavy metallic substance laid down beside him, and at the same moment a cord was fastened round his feet with sudden and painful violence. 

 <Well, have you tied the knot?> inquired the grave-digger, who was looking on. 

 <Yes, and pretty tight too, I can tell you,> was the answer. 

 <Move on, then.> And the bier was lifted once more, and they proceeded. 

 They advanced fifty paces farther, and then stopped to open a door, then went forward again. The noise of the waves dashing against the rocks on which the château is built, reached Dantès' ear distinctly as they went forward. 

 <Bad weather!> observed one of the bearers; <not a pleasant night for a dip in the sea.> 

 <Why, yes, the abbé runs a chance of being wet,> said the other; and then there was a burst of brutal laughter. 

 Dantès did not comprehend the jest, but his hair stood erect on his head. 

 <Well, here we are at last,> said one of them. 

 <A little farther—a little farther,> said the other. <You know very well that the last was stopped on his way, dashed on the rocks, and the governor told us next day that we were careless fellows.> 

 They ascended five or six more steps, and then Dantès felt that they took him, one by the head and the other by the heels, and swung him to and fro. 

 <One!> said the grave-diggers, <two! three!> 

 And at the same instant Dantès felt himself flung into the air like a wounded bird, falling, falling, with a rapidity that made his blood curdle. Although drawn downwards by the heavy weight which hastened his rapid descent, it seemed to him as if the fall lasted for a century. At last, with a horrible splash, he darted like an arrow into the ice-cold water, and as he did so he uttered a shrill cry, stifled in a moment by his immersion beneath the waves. 

 Dantès had been flung into the sea, and was dragged into its depths by a thirty-six-pound shot tied to his feet. 

 The sea is the cemetery of the Château d'If.