\chapter{The Promise} 

 \lettrine{I}{t} was indeed Maximilian Morrel, who had passed a wretched existence since the previous day. With the instinct peculiar to lovers he had anticipated after the return of Madame de Saint-Méran and the death of the marquis, that something would occur at M. de Villefort's in connection with his attachment for Valentine. His presentiments were realized, as we shall see, and his uneasy forebodings had goaded him pale and trembling to the gate under the chestnut-trees. 

 Valentine was ignorant of the cause of this sorrow and anxiety, and as it was not his accustomed hour for visiting her, she had gone to the spot simply by accident or perhaps through sympathy. Morrel called her, and she ran to the gate. 

 <You here at this hour?> said she. 

 <Yes, my poor girl,> replied Morrel; <I come to bring and to hear bad tidings.> 

 <This is, indeed, a house of mourning,> said Valentine; <speak, Maximilian, although the cup of sorrow seems already full.> 

 <Dear Valentine,> said Morrel, endeavouring to conceal his own emotion, <listen, I entreat you; what I am about to say is very serious. When are you to be married?> 

 <I will tell you all,> said Valentine; <from you I have nothing to conceal. This morning the subject was introduced, and my dear grandmother, on whom I depended as my only support, not only declared herself favourable to it, but is so anxious for it, that they only await the arrival of M. d'Épinay, and the following day the contract will be signed.> 

 A deep sigh escaped the young man, who gazed long and mournfully at her he loved. 

 <Alas,> replied he, <it is dreadful thus to hear my condemnation from your own lips. The sentence is passed, and, in a few hours, will be executed; it must be so, and I will not endeavour to prevent it. But, since you say nothing remains but for M. d'Épinay to arrive that the contract may be signed, and the following day you will be his, tomorrow you will be engaged to M. d'Épinay, for he came this morning to Paris.> Valentine uttered a cry. 

 <I was at the house of Monte Cristo an hour since,> said Morrel; <we were speaking, he of the sorrow your family had experienced, and I of your grief, when a carriage rolled into the courtyard. Never, till then, had I placed any confidence in presentiments, but now I cannot help believing them, Valentine. At the sound of that carriage I shuddered; soon I heard steps on the staircase, which terrified me as much as the footsteps of the commander did Don Juan. The door at last opened; Albert de Morcerf entered first, and I began to hope my fears were vain, when, after him, another young man advanced, and the count exclaimed: <Ah, here is the Baron Franz d'Épinay!> I summoned all my strength and courage to my support. Perhaps I turned pale and trembled, but certainly I smiled; and five minutes after I left, without having heard one word that had passed.> 

 <Poor Maximilian!> murmured Valentine. 

 <Valentine, the time has arrived when you must answer me. And remember my life depends on your answer. What do you intend doing?> Valentine held down her head; she was overwhelmed. 

 <Listen,> said Morrel; <it is not the first time you have contemplated our present position, which is a serious and urgent one; I do not think it is a moment to give way to useless sorrow; leave that for those who like to suffer at their leisure and indulge their grief in secret. There are such in the world, and God will doubtless reward them in heaven for their resignation on earth, but those who mean to contend must not lose one precious moment, but must return immediately the blow which fortune strikes. Do you intend to struggle against our ill-fortune? Tell me, Valentine for it is that I came to know.> 

 Valentine trembled, and looked at him with amazement. The idea of resisting her father, her grandmother, and all the family, had never occurred to her. 

 <What do you say, Maximilian?> asked Valentine. <What do you mean by a struggle? Oh, it would be a sacrilege. What? I resist my father's order, and my dying grandmother's wish? Impossible!> 

 Morrel started. 

 <You are too noble not to understand me, and you understand me so well that you already yield, dear Maximilian. No, no; I shall need all my strength to struggle with myself and support my grief in secret, as you say. But to grieve my father—to disturb my grandmother's last moments—never!> 

 <You are right,> said Morrel, calmly. 

 <In what a tone you speak!> cried Valentine. 

 <I speak as one who admires you, mademoiselle.> 

 <Mademoiselle,> cried Valentine; <mademoiselle! Oh, selfish man! he sees me in despair, and pretends he cannot understand me!> 

 <You mistake—I understand you perfectly. You will not oppose M. Villefort, you will not displease the marchioness, and tomorrow you will sign the contract which will bind you to your husband.> 

 <But, \textit{mon Dieu!} tell me, how can I do otherwise?> 

 <Do not appeal to me, mademoiselle; I shall be a bad judge in such a case; my selfishness will blind me,> replied Morrel, whose low voice and clenched hands announced his growing desperation. 

 <What would you have proposed, Maximilian, had you found me willing to accede?> 

 <It is not for me to say.> 

 <You are wrong; you must advise me what to do.> 

 <Do you seriously ask my advice, Valentine?> 

 <Certainly, dear Maximilian, for if it is good, I will follow it; you know my devotion to you.> 

 <Valentine,> said Morrel pushing aside a loose plank, <give me your hand in token of forgiveness of my anger; my senses are confused, and during the last hour the most extravagant thoughts have passed through my brain. Oh, if you refuse my advice\longdash> 

 <What do you advise?> said Valentine, raising her eyes to heaven and sighing. 

 <I am free,> replied Maximilian, <and rich enough to support you. I swear to make you my lawful wife before my lips even shall have approached your forehead.> 

 <You make me tremble!> said the young girl. 

 <Follow me,> said Morrel; <I will take you to my sister, who is worthy also to be yours. We will embark for Algiers, for England, for America, or, if you prefer it, retire to the country and only return to Paris when our friends have reconciled your family.> 

 Valentine shook her head. 

 <I feared it, Maximilian,> said she; <it is the counsel of a madman, and I should be more mad than you, did I not stop you at once with the word <Impossible, Morrel, impossible!>> 

 <You will then submit to what fate decrees for you without even attempting to contend with it?> said Morrel sorrowfully. 

 <Yes,—if I die!> 

 <Well, Valentine,> resumed Maximilian, <I can only say again that you are right. Truly, it is I who am mad, and you prove to me that passion blinds the most well-meaning. I appreciate your calm reasoning. It is then understood that tomorrow you will be irrevocably promised to M. Franz d'Épinay, not only by that theatrical formality invented to heighten the effect of a comedy called the signature of the contract, but your own will?> 

 <Again you drive me to despair, Maximilian,> said Valentine, <again you plunge the dagger into the wound! What would you do, tell me, if your sister listened to such a proposition?> 

 <Mademoiselle,> replied Morrel with a bitter smile, <I am selfish—you have already said so—and as a selfish man I think not of what others would do in my situation, but of what I intend doing myself. I think only that I have known you not a whole year. From the day I first saw you, all my hopes of happiness have been in securing your affection. One day you acknowledged that you loved me, and since that day my hope of future happiness has rested on obtaining you, for to gain you would be life to me. Now, I think no more; I say only that fortune has turned against me—I had thought to gain heaven, and now I have lost it. It is an every-day occurrence for a gambler to lose not only what he possesses but also what he has not.> 

 Morrel pronounced these words with perfect calmness; Valentine looked at him a moment with her large, scrutinizing eyes, endeavouring not to let Morrel discover the grief which struggled in her heart. 

 <But, in a word, what are you going to do?> asked she. 

 <I am going to have the honour of taking my leave of you, mademoiselle, solemnly assuring you that I wish your life may be so calm, so happy, and so fully occupied, that there may be no place for me even in your memory.> 

 <Oh!> murmured Valentine. 

 <Adieu, Valentine, adieu!> said Morrel, bowing. 

 <Where are you going?> cried the young girl, extending her hand through the opening, and seizing Maximilian by his coat, for she understood from her own agitated feelings that her lover's calmness could not be real; <where are you going?> 

 <I am going, that I may not bring fresh trouble into your family: and to set an example which every honest and devoted man, situated as I am, may follow.> 

 <Before you leave me, tell me what you are going to do, Maximilian.> The young man smiled sorrowfully. 

 <Speak, speak!> said Valentine; <I entreat you.> 

 <Has your resolution changed, Valentine?> 

 <It cannot change, unhappy man; you know it must not!> cried the young girl. 

 <Then adieu, Valentine!> 

 Valentine shook the gate with a strength of which she could not have been supposed to be possessed, as Morrel was going away, and passing both her hands through the opening, she clasped and wrung them. <I must know what you mean to do!> said she. <Where are you going?> 

 <Oh, fear not,> said Maximilian, stopping at a short distance, <I do not intend to render another man responsible for the rigorous fate reserved for me. Another might threaten to seek M. Franz, to provoke him, and to fight with him; all that would be folly. What has M. Franz to do with it? He saw me this morning for the first time, and has already forgotten he has seen me. He did not even know I existed when it was arranged by your two families that you should be united. I have no enmity against M. Franz, and promise you the punishment shall not fall on him.> 

 <On whom, then!—on me?> 

 <On you? Valentine! Oh, Heaven forbid! Woman is sacred; the woman one loves is holy.> 

 <On yourself, then, unhappy man; on yourself?> 

 <I am the only guilty person, am I not?> said Maximilian. 

 <Maximilian!> said Valentine, <Maximilian, come back, I entreat you!> 

 He drew near with his sweet smile, and but for his paleness one might have thought him in his usual happy mood. 

 <Listen, my dear, my adored Valentine,> said he in his melodious and grave tone; <those who, like us, have never had a thought for which we need blush before the world, such may read each other's hearts. I never was romantic, and am no melancholy hero. I imitate neither Manfred nor Anthony; but without words, protestations, or vows, my life has entwined itself with yours; you leave me, and you are right in doing so,—I repeat it, you are right; but in losing you, I lose my life. The moment you leave me, Valentine, I am alone in the world. My sister is happily married; her husband is only my brother-in-law, that is, a man whom the ties of social life alone attach to me; no one then longer needs my useless life. This is what I shall do; I will wait until the very moment you are married, for I will not lose the shadow of one of those unexpected chances which are sometimes reserved for us, since M. Franz may, after all, die before that time, a thunderbolt may fall even on the altar as you approach it,—nothing appears impossible to one condemned to die, and miracles appear quite reasonable when his escape from death is concerned. I will, then, wait until the last moment, and when my misery is certain, irremediable, hopeless, I will write a confidential letter to my brother-in-law, another to the prefect of police, to acquaint them with my intention, and at the corner of some wood, on the brink of some abyss, on the bank of some river, I will put an end to my existence, as certainly as I am the son of the most honest man who ever lived in France.>  Valentine trembled convulsively; she loosened her hold of the gate, her arms fell by her side, and two large tears rolled down her cheeks. The young man stood before her, sorrowful and resolute. 

 <Oh, for pity's sake,> said she, <you will live, will you not?> 

 <No, on my honour,> said Maximilian; <but that will not affect you. You have done your duty, and your conscience will be at rest.> 

 Valentine fell on her knees, and pressed her almost bursting heart. <Maximilian,> said she, <Maximilian, my friend, my brother on earth, my true husband in heaven, I entreat you, do as I do, live in suffering; perhaps we may one day be united.> 

 <Adieu, Valentine,> repeated Morrel. 

 <My God,> said Valentine, raising both her hands to heaven with a sublime expression, <I have done my utmost to remain a submissive daughter; I have begged, entreated, implored; he has regarded neither my prayers, my entreaties, nor my tears. It is done,> cried she, wiping away her tears, and resuming her firmness, <I am resolved not to die of remorse, but rather of shame. Live, Maximilian, and I will be yours. Say when shall it be? Speak, command, I will obey.> 

 Morrel, who had already gone some few steps away, again returned, and pale with joy extended both hands towards Valentine through the opening. 

 <Valentine,> said he, <dear Valentine, you must not speak thus—rather let me die. Why should I obtain you by violence, if our love is mutual? Is it from mere humanity you bid me live? I would then rather die.> 

 <Truly,> murmured Valentine, <who on this earth cares for me, if he does not? Who has consoled me in my sorrow but he? On whom do my hopes rest? On whom does my bleeding heart repose? On him, on him, always on him! Yes, you are right, Maximilian, I will follow you. I will leave the paternal home, I will give up all. Oh, ungrateful girl that I am,> cried Valentine, sobbing, <I will give up all, even my dear old grandfather, whom I had nearly forgotten.> 

 <No,> said Maximilian, <you shall not leave him. M. Noirtier has evinced, you say, a kind feeling towards me. Well, before you leave, tell him all; his consent would be your justification in God's sight. As soon as we are married, he shall come and live with us, instead of one child, he shall have two. You have told me how you talk to him and how he answers you; I shall very soon learn that language by signs, Valentine, and I promise you solemnly, that instead of despair, it is happiness that awaits us.> 

 <Oh, see, Maximilian, see the power you have over me, you almost make me believe you; and yet, what you tell me is madness, for my father will curse me—he is inflexible—he will never pardon me. Now listen to me, Maximilian; if by artifice, by entreaty, by accident—in short, if by any means I can delay this marriage, will you wait?> 

 <Yes, I promise you, as faithfully as you have promised me that this horrible marriage shall not take place, and that if you are dragged before a magistrate or a priest, you will refuse.> 

 <I promise you by all that is most sacred to me in the world, namely, by my mother.> 

 <We will wait, then,> said Morrel. 

 <Yes, we will wait,> replied Valentine, who revived at these words; <there are so many things which may save unhappy beings such as we are.> 

 <I rely on you, Valentine,> said Morrel; <all you do will be well done; only if they disregard your prayers, if your father and Madame de Saint-Méran insist that M. d'Épinay should be called tomorrow to sign the contract\longdash> 

 <Then you have my promise, Maximilian.> 

 <Instead of signing\longdash> 

 <I will go to you, and we will fly; but from this moment until then, let us not tempt Providence, let us not see each other. It is a miracle, it is a providence that we have not been discovered. If we were surprised, if it were known that we met thus, we should have no further resource.> 

 <You are right, Valentine; but how shall I ascertain?> 

 <From the notary, M. Deschamps.> 

 <I know him.> 

 <And for myself—I will write to you, depend on me. I dread this marriage, Maximilian, as much as you.> 

 <Thank you, my adored Valentine, thank you; that is enough. When once I know the hour, I will hasten to this spot, you can easily get over this fence with my assistance, a carriage will await us at the gate, in which you will accompany me to my sister's; there living, retired or mingling in society, as you wish, we shall be enabled to use our power to resist oppression, and not suffer ourselves to be put to death like sheep, which only defend themselves by sighs.> 

 <Yes,> said Valentine, <I will now acknowledge you are right, Maximilian; and now are you satisfied with your betrothal?> said the young girl sorrowfully. 

 <My adored Valentine, words cannot express one half of my satisfaction.> 

 Valentine had approached, or rather, had placed her lips so near the fence, that they nearly touched those of Morrel, which were pressed against the other side of the cold and inexorable barrier. 

 <Adieu, then, till we meet again,> said Valentine, tearing herself away. <I shall hear from you?> 

 <Yes.> 

 <Thanks, thanks, dear love, adieu!> 

 The sound of a kiss was heard, and Valentine fled through the avenue. Morrel listened to catch the last sound of her dress brushing the branches, and of her footstep on the gravel, then raised his eyes with an ineffable smile of thankfulness to heaven for being permitted to be thus loved, and then also disappeared. 

 The young man returned home and waited all the evening and all the next day without getting any message. It was only on the following day, at about ten o'clock in the morning, as he was starting to call on M. Deschamps, the notary, that he received from the postman a small billet, which he knew to be from Valentine, although he had not before seen her writing. It was to this effect: 

 “Tears, entreaties, prayers, have availed me nothing. Yesterday, for two hours, I was at the church of Saint-Philippe-du-Roule, and for two hours I prayed most fervently. Heaven is as inflexible as man, and the signature of the contract is fixed for this evening at nine o'clock. I have but one promise and but one heart to give; that promise is pledged to you, that heart is also yours. This evening, then, at a quarter to nine at the gate. 

 “Your betrothed, 

 <Valentine de Villefort.> 

 <P.S.—My poor grandmother gets worse and worse; yesterday her fever amounted to delirium; today her delirium is almost madness. You will be very kind to me, will you not, Morrel, to make me forget my sorrow in leaving her thus? I think it is kept a secret from grandpapa Noirtier, that the contract is to be signed this evening.> 

 Morrel went also to the notary, who confirmed the news that the contract was to be signed that evening. Then he went to call on Monte Cristo and heard still more. Franz had been to announce the ceremony, and Madame de Villefort had also written to beg the count to excuse her not inviting him; the death of M. de Saint-Méran and the dangerous illness of his widow would cast a gloom over the meeting which she would regret should be shared by the count whom she wished every happiness. 

 The day before Franz had been presented to Madame de Saint-Méran, who had left her bed to receive him, but had been obliged to return to it immediately after. 

 It is easy to suppose that Morrel's agitation would not escape the count's penetrating eye. Monte Cristo was more affectionate than ever,—indeed, his manner was so kind that several times Morrel was on the point of telling him all. But he recalled the promise he had made to Valentine, and kept his secret. 

 The young man read Valentine's letter twenty times in the course of the day. It was her first, and on what an occasion! Each time he read it he renewed his vow to make her happy. How great is the power of a woman who has made so courageous a resolution! What devotion does she deserve from him for whom she has sacrificed everything! How ought she really to be supremely loved! She becomes at once a queen and a wife, and it is impossible to thank and love her sufficiently. 

 Morrel longed intensely for the moment when he should hear Valentine say, <Here I am, Maximilian; come and help me.> He had arranged everything for her escape; two ladders were hidden in the clover-field; a cabriolet was ordered for Maximilian alone, without a servant, without lights; at the turning of the first street they would light the lamps, as it would be foolish to attract the notice of the police by too many precautions. Occasionally he shuddered; he thought of the moment when, from the top of that wall, he should protect the descent of his dear Valentine, pressing in his arms for the first time her of whom he had yet only kissed the delicate hand. 

 When the afternoon arrived and he felt that the hour was drawing near, he wished for solitude, his agitation was extreme; a simple question from a friend would have irritated him. He shut himself in his room, and tried to read, but his eye glanced over the page without understanding a word, and he threw away the book, and for the second time sat down to sketch his plan, the ladders and the fence. 

 At length the hour drew near. Never did a man deeply in love allow the clocks to go on peacefully. Morrel tormented his so effectually that they struck eight at half-past six. He then said, <It is time to start; the signature was indeed fixed to take place at nine o'clock, but perhaps Valentine will not wait for that.> Consequently, Morrel, having left the Rue Meslay at half-past eight by his timepiece, entered the clover-field while the clock of Saint-Philippe-du-Roule was striking eight. The horse and cabriolet were concealed behind a small ruin, where Morrel had often waited. 

 The night gradually drew on, and the foliage in the garden assumed a deeper hue. Then Morrel came out from his hiding-place with a beating heart, and looked through the small opening in the gate; there was yet no one to be seen. 

 The clock struck half-past eight, and still another half-hour was passed in waiting, while Morrel walked to and fro, and gazed more and more frequently through the opening. The garden became darker still, but in the darkness he looked in vain for the white dress, and in the silence he vainly listened for the sound of footsteps. The house, which was discernible through the trees, remained in darkness, and gave no indication that so important an event as the signature of a marriage-contract was going on. Morrel looked at his watch, which wanted a quarter to ten; but soon the same clock he had already heard strike two or three times rectified the error by striking half-past nine. 

 This was already half an hour past the time Valentine had fixed. It was a terrible moment for the young man. The slightest rustling of the foliage, the least whistling of the wind, attracted his attention, and drew the perspiration to his brow; then he tremblingly fixed his ladder, and, not to lose a moment, placed his foot on the first step. Amidst all these alternations of hope and fear, the clock struck ten. <It is impossible,> said Maximilian, <that the signing of a contract should occupy so long a time without unexpected interruptions. I have weighed all the chances, calculated the time required for all the forms; something must have happened.> 

 And then he walked rapidly to and fro, and pressed his burning forehead against the fence. Had Valentine fainted? or had she been discovered and stopped in her flight? These were the only obstacles which appeared possible to the young man. 

 The idea that her strength had failed her in attempting to escape, and that she had fainted in one of the paths, was the one that most impressed itself upon his mind. <In that case,> said he, <I should lose her, and by my own fault.> He dwelt on this idea for a moment, then it appeared reality. He even thought he could perceive something on the ground at a distance; he ventured to call, and it seemed to him that the wind wafted back an almost inarticulate sigh. 

 At last the half-hour struck. It was impossible to wait longer, his temples throbbed violently, his eyes were growing dim; he passed one leg over the wall, and in a moment leaped down on the other side. He was on Villefort's premises—had arrived there by scaling the wall. What might be the consequences? However, he had not ventured thus far to draw back. He followed a short distance close under the wall, then crossed a path, hid entered a clump of trees. In a moment he had passed through them, and could see the house distinctly.  Then Morrel saw that he had been right in believing that the house was not illuminated. Instead of lights at every window, as is customary on days of ceremony, he saw only a gray mass, which was veiled also by a cloud, which at that moment obscured the moon's feeble light. A light moved rapidly from time to time past three windows of the second floor. These three windows were in Madame de Saint-Méran's room. Another remained motionless behind some red curtains which were in Madame de Villefort's bedroom. Morrel guessed all this. So many times, in order to follow Valentine in thought at every hour in the day, had he made her describe the whole house, that without having seen it he knew it all. 

 This darkness and silence alarmed Morrel still more than Valentine's absence had done. Almost mad with grief, and determined to venture everything in order to see Valentine once more, and be certain of the misfortune he feared, Morrel gained the edge of the clump of trees, and was going to pass as quickly as possible through the flower-garden, when the sound of a voice, still at some distance, but which was borne upon the wind, reached him. At this sound, as he was already partially exposed to view, he stepped back and concealed himself completely, remaining perfectly motionless. 

 He had formed his resolution. If it was Valentine alone, he would speak as she passed; if she was accompanied, and he could not speak, still he should see her, and know that she was safe; if they were strangers, he would listen to their conversation, and might understand something of this hitherto incomprehensible mystery. 

 The moon had just then escaped from behind the cloud which had concealed it, and Morrel saw Villefort come out upon the steps, followed by a gentleman in black. They descended, and advanced towards the clump of trees, and Morrel soon recognized the other gentleman as Doctor d'Avrigny.  The young man, seeing them approach, drew back mechanically, until he found himself stopped by a sycamore-tree in the centre of the clump; there he was compelled to remain. Soon the two gentlemen stopped also. 

 <Ah, my dear doctor,> said the procureur, <Heaven declares itself against my house! What a dreadful death—what a blow! Seek not to console me; alas, nothing can alleviate so great a sorrow—the wound is too deep and too fresh! Dead, dead!> 

 The cold sweat sprang to the young man's brow, and his teeth chattered. Who could be dead in that house, which Villefort himself had called accursed? 

 <My dear M. de Villefort,> replied the doctor, with a tone which redoubled the terror of the young man, <I have not led you here to console you; on the contrary\longdash> 

 <What can you mean?> asked the procureur, alarmed. 

 <I mean that behind the misfortune which has just happened to you, there is another, perhaps, still greater.> 

 <Can it be possible?> murmured Villefort, clasping his hands. <What are you going to tell me?> 

 <Are we quite alone, my friend?> 

 <Yes, quite; but why all these precautions?> 

 <Because I have a terrible secret to communicate to you,> said the doctor. <Let us sit down.> 

 Villefort fell, rather than seated himself. The doctor stood before him, with one hand placed on his shoulder. Morrel, horrified, supported his head with one hand, and with the other pressed his heart, lest its beatings should be heard. <Dead, dead!> repeated he within himself; and he felt as if he were also dying. 

 <Speak, doctor—I am listening,> said Villefort; <strike—I am prepared for everything!> 

 <Madame de Saint-Méran was, doubtless, advancing in years, but she enjoyed excellent health.> Morrel began again to breathe freely, which he had not done during the last ten minutes. 

 <Grief has consumed her,> said Villefort—<yes, grief, doctor! After living forty years with the marquis\longdash> 

 <It is not grief, my dear Villefort,> said the doctor; <grief may kill, although it rarely does, and never in a day, never in an hour, never in ten minutes.> Villefort answered nothing, he simply raised his head, which had been cast down before, and looked at the doctor with amazement. 

 <Were you present during the last struggle?> asked M. d'Avrigny. 

 <I was,> replied the procureur; <you begged me not to leave.> 

 <Did you notice the symptoms of the disease to which Madame de Saint-Méran has fallen a victim?> 

 <I did. Madame de Saint-Méran had three successive attacks, at intervals of some minutes, each one more serious than the former. When you arrived, Madame de Saint-Méran had already been panting for breath some minutes; she then had a fit, which I took to be simply a nervous attack, and it was only when I saw her raise herself in the bed, and her limbs and neck appear stiffened, that I became really alarmed. Then I understood from your countenance there was more to fear than I had thought. This crisis past, I endeavoured to catch your eye, but could not. You held her hand—you were feeling her pulse—and the second fit came on before you had turned towards me. This was more terrible than the first; the same nervous movements were repeated, and the mouth contracted and turned purple.> 

 <And at the third she expired.> 

 <At the end of the first attack I discovered symptoms of tetanus; you confirmed my opinion.> 

 <Yes, before others,> replied the doctor; <but now we are alone\longdash> 

 <What are you going to say? Oh, spare me!> 

 <That the symptoms of tetanus and poisoning by vegetable substances are the same.> 

 M. de Villefort started from his seat, then in a moment fell down again, silent and motionless. Morrel knew not if he were dreaming or awake. 

 <Listen,> said the doctor; <I know the full importance of the statement I have just made, and the disposition of the man to whom I have made it.> 

 <Do you speak to me as a magistrate or as a friend?> asked Villefort. 

 <As a friend, and only as a friend, at this moment. The similarity in the symptoms of tetanus and poisoning by vegetable substances is so great, that were I obliged to affirm by oath what I have now stated, I should hesitate; I therefore repeat to you, I speak not to a magistrate, but to a friend. And to that friend I say, <During the three-quarters of an hour that the struggle continued, I watched the convulsions and the death of Madame de Saint-Méran, and am thoroughly convinced that not only did her death proceed from poison, but I could also specify the poison.>> 

 <Can it be possible?> 

 <The symptoms are marked, do you see?—sleep broken by nervous spasms, excitation of the brain, torpor of the nerve centres. Madame de Saint-Méran succumbed to a powerful dose of brucine or of strychnine, which by some mistake, perhaps, has been given to her.> 

 Villefort seized the doctor's hand. 

 <Oh, it is impossible,> said he, <I must be dreaming! It is frightful to hear such things from such a man as you! Tell me, I entreat you, my dear doctor, that you may be deceived.> 

 <Doubtless I may, but\longdash> 

 <But?>  <But I do not think so.> 

 <Have pity on me doctor! So many dreadful things have happened to me lately that I am on the verge of madness.> 

 <Has anyone besides me seen Madame de Saint-Méran?> 

 <No.> 

 <Has anything been sent for from a chemist's that I have not examined?> 

 <Nothing.> 

 <Had Madame de Saint-Méran any enemies?> 

 <Not to my knowledge.> 

 <Would her death affect anyone's interest?> 

 <It could not indeed, my daughter is her only heiress—Valentine alone. Oh, if such a thought could present itself, I would stab myself to punish my heart for having for one instant harbored it.> 

 <Indeed, my dear friend,> said M. d'Avrigny, <I would not accuse anyone; I speak only of an accident, you understand,—of a mistake,—but whether accident or mistake, the fact is there; it is on my conscience and compels me to speak aloud to you. Make inquiry.> 

 <Of whom?—how?—of what?> 

 <May not Barrois, the old servant, have made a mistake, and have given Madame de Saint-Méran a dose prepared for his master?> 

 <For my father?> 

 <Yes.> 

 <But how could a dose prepared for M. Noirtier poison Madame de Saint-Méran?> 

 <Nothing is more simple. You know poisons become remedies in certain diseases, of which paralysis is one. For instance, having tried every other remedy to restore movement and speech to M. Noirtier, I resolved to try one last means, and for three months I have been giving him brucine; so that in the last dose I ordered for him there were six grains. This quantity, which is perfectly safe to administer to the paralysed frame of M. Noirtier, which has become gradually accustomed to it, would be sufficient to kill another person.> 

 <My dear doctor, there is no communication between M. Noirtier's apartment and that of Madame de Saint-Méran, and Barrois never entered my mother-in-law's room. In short, doctor although I know you to be the most conscientious man in the world, and although I place the utmost reliance in you, I want, notwithstanding my conviction, to believe this axiom, \textit{errare humanum est}.> 

 <Is there one of my brethren in whom you have equal confidence with myself?> 

 <Why do you ask me that?—what do you wish?> 

 <Send for him; I will tell him what I have seen, and we will consult together, and examine the body.> 

 <And you will find traces of poison?> 

 <No, I did not say of poison, but we can prove what was the state of the body; we shall discover the cause of her sudden death, and we shall say, <Dear Villefort, if this thing has been caused by negligence, watch over your servants; if from hatred, watch your enemies.>> 

 <What do you propose to me, d'Avrigny?> said Villefort in despair; <so soon as another is admitted into our secret, an inquest will become necessary; and an inquest in my house—impossible! Still,> continued the procureur, looking at the doctor with uneasiness, <if you wish it—if you demand it, why then it shall be done. But, doctor, you see me already so grieved—how can I introduce into my house so much scandal, after so much sorrow? My wife and my daughter would die of it! And I, doctor—you know a man does not arrive at the post I occupy—one has not been king's attorney twenty-five years without having amassed a tolerable number of enemies; mine are numerous. Let this affair be talked of, it will be a triumph for them, which will make them rejoice, and cover me with shame. Pardon me, doctor, these worldly ideas; were you a priest I should not dare tell you that, but you are a man, and you know mankind. Doctor, pray recall your words; you have said nothing, have you?> 

 <My dear M. de Villefort,> replied the doctor, <my first duty is to humanity. I would have saved Madame de Saint-Méran, if science could have done it; but she is dead and my duty regards the living. Let us bury this terrible secret in the deepest recesses of our hearts; I am willing, if anyone should suspect this, that my silence on the subject should be imputed to my ignorance. Meanwhile, sir, watch always—watch carefully, for perhaps the evil may not stop here. And when you have found the culprit, if you find him, I will say to you, <You are a magistrate, do as you will!>> 

 <I thank you, doctor,> said Villefort with indescribable joy; <I never had a better friend than you.> And, as if he feared Doctor d'Avrigny would recall his promise, he hurried him towards the house. 

 When they were gone, Morrel ventured out from under the trees, and the moon shone upon his face, which was so pale it might have been taken for that of a ghost. 

 <I am manifestly protected in a most wonderful, but most terrible manner,> said he; <but Valentine, poor girl, how will she bear so much sorrow?> 

 As he thought thus, he looked alternately at the window with red curtains and the three windows with white curtains. The light had almost disappeared from the former; doubtless Madame de Villefort had just put out her lamp, and the nightlamp alone reflected its dull light on the window. At the extremity of the building, on the contrary, he saw one of the three windows open. A wax-light placed on the mantle-piece threw some of its pale rays without, and a shadow was seen for one moment on the balcony. Morrel shuddered; he thought he heard a sob. 

 It cannot be wondered at that his mind, generally so courageous, but now disturbed by the two strongest human passions, love and fear, was weakened even to the indulgence of superstitious thoughts. Although it was impossible that Valentine should see him, hidden as he was, he thought he heard the shadow at the window call him; his disturbed mind told him so. This double error became an irresistible reality, and by one of the incomprehensible transports of youth, he bounded from his hiding-place, and with two strides, at the risk of being seen, at the risk of alarming Valentine, at the risk of being discovered by some exclamation which might escape the young girl, he crossed the flower-garden, which by the light of the moon resembled a large white lake, and having passed the rows of orange-trees which extended in front of the house, he reached the step, ran quickly up and pushed the door, which opened without offering any resistance. 

 Valentine had not seen him. Her eyes, raised towards heaven, were watching a silvery cloud gliding over the azure, its form that of a shadow mounting towards heaven. Her poetic and excited mind pictured it as the soul of her grandmother. 

 Meanwhile, Morrel had traversed the anteroom and found the staircase, which, being carpeted, prevented his approach being heard, and he had regained that degree of confidence that the presence of M. de Villefort even would not have alarmed him. He was quite prepared for any such encounter. He would at once approach Valentine's father and acknowledge all, begging Villefort to pardon and sanction the love which united two fond and loving hearts. Morrel was mad. 

 Happily he did not meet anyone. Now, especially, did he find the description Valentine had given of the interior of the house useful to him; he arrived safely at the top of the staircase, and while he was feeling his way, a sob indicated the direction he was to take. He turned back, a door partly open enabled him to see his road, and to hear the voice of one in sorrow. He pushed the door open and entered. At the other end of the room, under a white sheet which covered it, lay the corpse, still more alarming to Morrel since the account he had so unexpectedly overheard. By its side, on her knees, and with her head buried in the cushion of an easy-chair, was Valentine, trembling and sobbing, her hands extended above her head, clasped and stiff. She had turned from the window, which remained open, and was praying in accents that would have affected the most unfeeling; her words were rapid, incoherent, unintelligible, for the burning weight of grief almost stopped her utterance. 

 The moon shining through the open blinds made the lamp appear to burn paler, and cast a sepulchral hue over the whole scene. Morrel could not resist this; he was not exemplary for piety, he was not easily impressed, but Valentine suffering, weeping, wringing her hands before him, was more than he could bear in silence. He sighed, and whispered a name, and the head bathed in tears and pressed on the velvet cushion of the chair—a head like that of a Magdalen by Correggio—was raised and turned towards him. Valentine perceived him without betraying the least surprise. A heart overwhelmed with one great grief is insensible to minor emotions. Morrel held out his hand to her. Valentine, as her only apology for not having met him, pointed to the corpse under the sheet, and began to sob again. 

 Neither dared for some time to speak in that room. They hesitated to break the silence which death seemed to impose; at length Valentine ventured. 

 <My friend,> said she, <how came you here? Alas, I would say you are welcome, had not death opened the way for you into this house.> 

 <Valentine,> said Morrel with a trembling voice, <I had waited since half-past eight, and did not see you come; I became uneasy, leaped the wall, found my way through the garden, when voices conversing about the fatal event\longdash> 

 <What voices?> asked Valentine. Morrel shuddered as he thought of the conversation of the doctor and M. de Villefort, and he thought he could see through the sheet the extended hands, the stiff neck, and the purple lips. 

 <Your servants,> said he, <who were repeating the whole of the sorrowful story; from them I learned it all.> 

 <But it was risking the failure of our plan to come up here, love.> 

 <Forgive me,> replied Morrel; <I will go away.> 

 <No,> said Valentine, <you might meet someone; stay.> 

 <But if anyone should come here\longdash> 

 The young girl shook her head. <No one will come,> said she; <do not fear, there is our safeguard,> pointing to the bed. 

 <But what has become of M. d'Épinay?> replied Morrel.  <M. Franz arrived to sign the contract just as my dear grandmother was dying.> 

 <Alas,> said Morrel with a feeling of selfish joy; for he thought this death would cause the wedding to be postponed indefinitely. 

 <But what redoubles my sorrow,> continued the young girl, as if this feeling was to receive its immediate punishment, <is that the poor old lady, on her death-bed, requested that the marriage might take place as soon as possible; she also, thinking to protect me, was acting against me.> 

 <Hark!> said Morrel. They both listened; steps were distinctly heard in the corridor and on the stairs. 

 <It is my father, who has just left his study.> 

 <To accompany the doctor to the door,> added Morrel. 

 <How do you know it is the doctor?> asked Valentine, astonished. 

 <I imagined it must be,> said Morrel. 

 Valentine looked at the young man; they heard the street door close, then M. de Villefort locked the garden door, and returned upstairs. He stopped a moment in the anteroom, as if hesitating whether to turn to his own apartment or into Madame de Saint-Méran's; Morrel concealed himself behind a door; Valentine remained motionless, grief seeming to deprive her of all fear. M. de Villefort passed on to his own room. 

 <Now,> said Valentine, <you can neither go out by the front door nor by the garden.> 

 Morrel looked at her with astonishment. 

 <There is but one way left you that is safe,> said she; <it is through my grandfather's room.> She rose. <Come,> she added. 

 <Where?> asked Maximilian. 

 <To my grandfather's room.> 

 <I in M. Noirtier's apartment?> 

 <Yes.> 

 <Can you mean it, Valentine?> 

 <I have long wished it; he is my only remaining friend and we both need his help,—come.> 

 <Be careful, Valentine,> said Morrel, hesitating to comply with the young girl's wishes; <I now see my error—I acted like a madman in coming in here. Are you sure you are more reasonable?> 

 <Yes,> said Valentine; <and I have but one scruple,—that of leaving my dear grandmother's remains, which I had undertaken to watch.> 

 <Valentine,> said Morrel, <death is in itself sacred.> 

 <Yes,> said Valentine; <besides, it will not be for long.> 

 She then crossed the corridor, and led the way down a narrow staircase to M. Noirtier's room; Morrel followed her on tiptoe; at the door they found the old servant. 

 <Barrois,> said Valentine, <shut the door, and let no one come in.> 

 She passed first. 

 Noirtier, seated in his chair, and listening to every sound, was watching the door; he saw Valentine, and his eye brightened. There was something grave and solemn in the approach of the young girl which struck the old man, and immediately his bright eye began to interrogate.  
 
 
 <Dear grandfather.> said she hurriedly, <you know poor grandmamma died an hour since, and now I have no friend in the world but you.> 

 His expressive eyes evinced the greatest tenderness. 

 <To you alone, then, may I confide my sorrows and my hopes?> 

 The paralytic motioned <Yes.> 

 Valentine took Maximilian's hand. 

 <Look attentively, then, at this gentleman.> 

 The old man fixed his scrutinizing gaze with slight astonishment on Morrel. 

 <It is M. Maximilian Morrel,> said she; <the son of that good merchant of Marseilles, whom you doubtless recollect.> 

 <Yes,> said the old man. 

 <He brings an irreproachable name, which Maximilian is likely to render glorious, since at thirty years of age he is a captain, an officer of the Legion of honour.> 

 The old man signified that he recollected him. 

 <Well, grandpapa,> said Valentine, kneeling before him, and pointing to Maximilian, <I love him, and will be only his; were I compelled to marry another, I would destroy myself.> 

 The eyes of the paralytic expressed a multitude of tumultuous thoughts. 

 <You like M. Maximilian Morrel, do you not, grandpapa?> asked Valentine. 

 <Yes.> 

 <And you will protect us, who are your children, against the will of my father?> 

 Noirtier cast an intelligent glance at Morrel, as if to say, <perhaps I may.> 

 Maximilian understood him. 

 <Mademoiselle,> said he, <you have a sacred duty to fulfil in your deceased grandmother's room, will you allow me the honour of a few minutes' conversation with M. Noirtier?> 

 <That is it,> said the old man's eye. Then he looked anxiously at Valentine. 

 <Do you fear he will not understand?> 

 <Yes.> 

 <Oh, we have so often spoken of you, that he knows exactly how I talk to you.> Then turning to Maximilian, with an adorable smile; although shaded by sorrow,—<He knows everything I know,> said she. 

 Valentine arose, placed a chair for Morrel, requested Barrois not to admit anyone, and having tenderly embraced her grandfather, and sorrowfully taken leave of Morrel, she went away. To prove to Noirtier that he was in Valentine's confidence and knew all their secrets, Morrel took the dictionary, a pen, and some paper, and placed them all on a table where there was a light. 

 <But first,> said Morrel, <allow me, sir, to tell you who I am, how much I love Mademoiselle Valentine, and what are my designs respecting her.> 

 Noirtier made a sign that he would listen. 

 It was an imposing sight to witness this old man, apparently a mere useless burden, becoming the sole protector, support, and adviser of the lovers who were both young, beautiful, and strong. His remarkably noble and austere expression struck Morrel, who began his story with trembling. He related the manner in which he had become acquainted with Valentine, and how he had loved her, and that Valentine, in her solitude and her misfortune, had accepted the offer of his devotion. He told him his birth, his position, his fortune, and more than once, when he consulted the look of the paralytic, that look answered, <That is good, proceed.> 

 <And now,> said Morrel, when he had finished the first part of his recital, <now I have told you of my love and my hopes, may I inform you of my intentions?> 

 <Yes,> signified the old man. 

 <This was our resolution; a cabriolet was in waiting at the gate, in which I intended to carry off Valentine to my sister's house, to marry her, and to wait respectfully M. de Villefort's pardon.> 

 <No,> said Noirtier. 

 <We must not do so?> 

 <No.> 

 <You do not sanction our project?> 

 <No.> 

 <There is another way,> said Morrel. The old man's interrogative eye said, <Which?> 

 <I will go,> continued Maximilian, <I will seek M. Franz d'Épinay—I am happy to be able to mention this in Mademoiselle de Villefort's absence—and will conduct myself toward him so as to compel him to challenge me.> Noirtier's look continued to interrogate. 

 <You wish to know what I will do?> 

 <Yes.> 

 <I will find him, as I told you. I will tell him the ties which bind me to Mademoiselle Valentine; if he be a sensible man, he will prove it by renouncing of his own accord the hand of his betrothed, and will secure my friendship, and love until death; if he refuse, either through interest or ridiculous pride, after I have proved to him that he would be forcing my wife from me, that Valentine loves me, and will have no other, I will fight with him, give him every advantage, and I shall kill him, or he will kill me; if I am victorious, he will not marry Valentine, and if I die, I am very sure Valentine will not marry him.> 

 Noirtier watched, with indescribable pleasure, this noble and sincere countenance, on which every sentiment his tongue uttered was depicted, adding by the expression of his fine features all that coloring adds to a sound and faithful drawing. 

 Still, when Morrel had finished, he shut his eyes several times, which was his manner of saying <No.> 

 <No?> said Morrel; <you disapprove of this second project, as you did of the first?> 

 <I do,> signified the old man. 

 <But what then must be done?> asked Morrel. <Madame de Saint-Méran's last request was, that the marriage might not be delayed; must I let things take their course?> Noirtier did not move. <I understand,> said Morrel; <I am to wait.> 

 <Yes.> 

 <But delay may ruin our plan, sir,> replied the young man. <Alone, Valentine has no power; she will be compelled to submit. I am here almost miraculously, and can scarcely hope for so good an opportunity to occur again. Believe me, there are only the two plans I have proposed to you; forgive my vanity, and tell me which you prefer. Do you authorize Mademoiselle Valentine to intrust herself to my honour?> 

 <No.> 

 <Do you prefer I should seek M. d'Épinay?> 

 <No.> 

 <Whence then will come the help we need—from chance?> resumed Morrel. 

 <No.> 

 <From you?> 

 <Yes.> 

 <You thoroughly understand me, sir? Pardon my eagerness, for my life depends on your answer. Will our help come from you?> 

 <Yes.> 

 <You are sure of it?> 

 <Yes.> There was so much firmness in the look which gave this answer, no one could, at any rate, doubt his will, if they did his power. 

 <Oh, thank you a thousand times! But how, unless a miracle should restore your speech, your gesture, your movement, how can you, chained to that armchair, dumb and motionless, oppose this marriage?> A smile lit up the old man's face, a strange smile of the eyes in a paralysed face. 

 <Then I must wait?> asked the young man. 

 <Yes.> 

 <But the contract?> The same smile returned. <Will you assure me it shall not be signed?> 

 <Yes,> said Noirtier. 

 <The contract shall not be signed!> cried Morrel. <Oh, pardon me, sir; I can scarcely realize so great a happiness. Will they not sign it?> 

 <No,> said the paralytic. Notwithstanding that assurance, Morrel still hesitated. This promise of an impotent old man was so strange that, instead of being the result of the power of his will, it might emanate from enfeebled organs. Is it not natural that the madman, ignorant of his folly, should attempt things beyond his power? The weak man talks of burdens he can raise, the timid of giants he can confront, the poor of treasures he spends, the most humble peasant, in the height of his pride, calls himself Jupiter. Whether Noirtier understood the young man's indecision, or whether he had not full confidence in his docility, he looked uneasily at him. 

 <What do you wish, sir?> asked Morrel; <that I should renew my promise of remaining tranquil?> Noirtier's eye remained fixed and firm, as if to imply that a promise did not suffice; then it passed from his face to his hands. 

 <Shall I swear to you, sir?> asked Maximilian. 

 <Yes,> said the paralytic with the same solemnity. Morrel understood that the old man attached great importance to an oath. He extended his hand. 

 <I swear to you, on my honour,> said he, <to await your decision respecting the course I am to pursue with M. d'Épinay.> 

 <That is right,> said the old man. 

 <Now,> said Morrel, <do you wish me to retire?> 

 <Yes.> 

 <Without seeing Mademoiselle Valentine?> 

 <Yes.> 

 Morrel made a sign that he was ready to obey. <But,> said he, <first allow me to embrace you as your daughter did just now.> Noirtier's expression could not be understood. The young man pressed his lips on the same spot, on the old man's forehead, where Valentine's had been. Then he bowed a second time and retired. 

 He found outside the door the old servant, to whom Valentine had given directions. Morrel was conducted along a dark passage, which led to a little door opening on the garden, soon found the spot where he had entered, with the assistance of the shrubs gained the top of the wall, and by his ladder was in an instant in the clover-field where his cabriolet was still waiting for him. He got in it, and thoroughly wearied by so many emotions, arrived about midnight in the Rue Meslay, threw himself on his bed and slept soundly. 