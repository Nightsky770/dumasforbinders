\chapter{Father and Daughter} 

 \lettrine{W}{e} saw in a preceding chapter how Madame Danglars went formally to announce to Madame de Villefort the approaching marriage of Eugénie Danglars and M. Andrea Cavalcanti. This formal announcement, which implied or appeared to imply, the approval of all the persons concerned in this momentous affair, had been preceded by a scene to which our readers must be admitted. We beg them to take one step backward, and to transport themselves, the morning of that day of great catastrophes, into the showy, gilded salon we have before shown them, and which was the pride of its owner, Baron Danglars. 

 In this room, at about ten o'clock in the morning, the banker himself had been walking to and fro for some minutes thoughtfully and in evident uneasiness, watching both doors, and listening to every sound. When his patience was exhausted, he called his valet. 

 <Étienne,> said he, <see why Mademoiselle Eugénie has asked me to meet her in the drawing-room, and why she makes me wait so long.> 

 Having given this vent to his ill-humor, the baron became more calm; Mademoiselle Danglars had that morning requested an interview with her father, and had fixed on the gilded drawing-room as the spot. The singularity of this step, and above all its formality, had not a little surprised the banker, who had immediately obeyed his daughter by repairing first to the drawing-room. Étienne soon returned from his errand. 

 <Mademoiselle's lady's maid says, sir, that mademoiselle is finishing her toilette, and will be here shortly.> 

 Danglars nodded, to signify that he was satisfied. To the world and to his servants Danglars assumed the character of the good-natured man and the indulgent father. This was one of his parts in the popular comedy he was performing,—a make-up he had adopted and which suited him about as well as the masks worn on the classic stage by paternal actors, who seen from one side, were the image of geniality, and from the other showed lips drawn down in chronic ill-temper. Let us hasten to say that in private the genial side descended to the level of the other, so that generally the indulgent man disappeared to give place to the brutal husband and domineering father. 

 <Why the devil does that foolish girl, who pretends to wish to speak to me, not come into my study? and why on earth does she want to speak to me at all?> 

 He was turning this thought over in his brain for the twentieth time, when the door opened and Eugénie appeared, attired in a figured black satin dress, her hair dressed and gloves on, as if she were going to the Italian Opera. 

 <Well, Eugénie, what is it you want with me? and why in this solemn drawing-room when the study is so comfortable?> 

 <I quite understand why you ask, sir,> said Eugénie, making a sign that her father might be seated, <and in fact your two questions suggest fully the theme of our conversation. I will answer them both, and contrary to the usual method, the last first, because it is the least difficult. I have chosen the drawing-room, sir, as our place of meeting, in order to avoid the disagreeable impressions and influences of a banker's study. Those gilded cashbooks, drawers locked like gates of fortresses, heaps of bank-bills, come from I know not where, and the quantities of letters from England, Holland, Spain, India, China, and Peru, have generally a strange influence on a father's mind, and make him forget that there is in the world an interest greater and more sacred than the good opinion of his correspondents. I have, therefore, chosen this drawing-room, where you see, smiling and happy in their magnificent frames, your portrait, mine, my mother's, and all sorts of rural landscapes and touching pastorals. I rely much on external impressions; perhaps, with regard to you, they are immaterial, but I should be no artist if I had not some fancies.> 

 <Very well,> replied M. Danglars, who had listened to all this preamble with imperturbable coolness, but without understanding a word, since like every man burdened with thoughts of the past, he was occupied with seeking the thread of his own ideas in those of the speaker. 

 <There is, then, the second point cleared up, or nearly so,> said Eugénie, without the least confusion, and with that masculine pointedness which distinguished her gesture and her language; <and you appear satisfied with the explanation. Now, let us return to the first. You ask me why I have requested this interview; I will tell you in two words, sir; I will not marry count Andrea Cavalcanti.> 

 Danglars leaped from his chair and raised his eyes and arms towards heaven.  <Yes, indeed, sir,> continued Eugénie, still quite calm; <you are astonished, I see; for since this little affair began, I have not manifested the slightest opposition, and yet I am always sure, when the opportunity arrives, to oppose a determined and absolute will to people who have not consulted me, and things which displease me. However, this time, my tranquillity, or passiveness as philosophers say, proceeded from another source; it proceeded from a wish, like a submissive and devoted daughter> (a slight smile was observable on the purple lips of the young girl), <to practice obedience.> 

 <Well?> asked Danglars. 

 <Well, sir,> replied Eugénie, <I have tried to the very last and now that the moment has come, I feel in spite of all my efforts that it is impossible.> 

 <But,> said Danglars, whose weak mind was at first quite overwhelmed with the weight of this pitiless logic, marking evident premeditation and force of will, <what is your reason for this refusal, Eugénie? what reason do you assign?> 

 <My reason?> replied the young girl. <Well, it is not that the man is more ugly, more foolish, or more disagreeable than any other; no, M. Andrea Cavalcanti may appear to those who look at men's faces and figures as a very good specimen of his kind. It is not, either, that my heart is less touched by him than any other; that would be a schoolgirl's reason, which I consider quite beneath me. I actually love no one, sir; you know it, do you not? I do not then see why, without real necessity, I should encumber my life with a perpetual companion. Has not some sage said, <Nothing too much>? and another, <I carry all my effects with me>? I have been taught these two aphorisms in Latin and in Greek; one is, I believe, from Phædrus, and the other from Bias. Well, my dear father, in the shipwreck of life—for life is an eternal shipwreck of our hopes—I cast into the sea my useless encumbrance, that is all, and I remain with my own will, disposed to live perfectly alone, and consequently perfectly free.> 

 <Unhappy girl, unhappy girl!> murmured Danglars, turning pale, for he knew from long experience the solidity of the obstacle he had so suddenly encountered. 

 <Unhappy girl,> replied Eugénie, <unhappy girl, do you say, sir? No, indeed; the exclamation appears quite theatrical and affected. Happy, on the contrary, for what am I in want of? The world calls me beautiful. It is something to be well received. I like a favourable reception; it expands the countenance, and those around me do not then appear so ugly. I possess a share of wit, and a certain relative sensibility, which enables me to draw from life in general, for the support of mine, all I meet with that is good, like the monkey who cracks the nut to get at its contents. I am rich, for you have one of the first fortunes in France. I am your only daughter, and you are not so exacting as the fathers of the Porte Saint-Martin and Gaîté, who disinherit their daughters for not giving them grandchildren. Besides, the provident law has deprived you of the power to disinherit me, at least entirely, as it has also of the power to compel me to marry Monsieur This or Monsieur That. And so—being, beautiful, witty, somewhat talented, as the comic operas say, and rich—and that is happiness, sir—why do you call me unhappy?> 

 Danglars, seeing his daughter smiling, and proud even to insolence, could not entirely repress his brutal feelings, but they betrayed themselves only by an exclamation. Under the fixed and inquiring gaze levelled at him from under those beautiful black eyebrows, he prudently turned away, and calmed himself immediately, daunted by the power of a resolute mind. 

 <Truly, my daughter,> replied he with a smile, <you are all you boast of being, excepting one thing; I will not too hastily tell you which, but would rather leave you to guess it.> 

 Eugénie looked at Danglars, much surprised that one flower of her crown of pride, with which she had so superbly decked herself, should be disputed. 

 <My daughter,> continued the banker, <you have perfectly explained to me the sentiments which influence a girl like you, who is determined she will not marry; now it remains for me to tell you the motives of a father like me, who has decided that his daughter shall marry.> 

 Eugénie bowed, not as a submissive daughter, but as an adversary prepared for a discussion. 

 <My daughter,> continued Danglars, <when a father asks his daughter to choose a husband, he has always some reason for wishing her to marry. Some are affected with the mania of which you spoke just now, that of living again in their grandchildren. This is not my weakness, I tell you at once; family joys have no charm for me. I may acknowledge this to a daughter whom I know to be philosophical enough to understand my indifference, and not to impute it to me as a crime.> 

 <This is not to the purpose,> said Eugénie; <let us speak candidly, sir; I admire candour.> 

 <Oh,> said Danglars, <I can, when circumstances render it desirable, adopt your system, although it may not be my general practice. I will therefore proceed. I have proposed to you to marry, not for your sake, for indeed I did not think of you in the least at the moment (you admire candour, and will now be satisfied, I hope); but because it suited me to marry you as soon as possible, on account of certain commercial speculations I am desirous of entering into.> Eugénie became uneasy.  <It is just as I tell you, I assure you, and you must not be angry with me, for you have sought this disclosure. I do not willingly enter into arithmetical explanations with an artist like you, who fears to enter my study lest she should imbibe disagreeable or anti-poetic impressions and sensations. But in that same banker's study, where you very willingly presented yourself yesterday to ask for the thousand francs I give you monthly for pocket-money, you must know, my dear young lady, that many things may be learned, useful even to a girl who will not marry. There one may learn, for instance, what, out of regard to your nervous susceptibility, I will inform you of in the drawing-room, namely, that the credit of a banker is his physical and moral life; that credit sustains him as breath animates the body; and M. de Monte Cristo once gave me a lecture on that subject, which I have never forgotten. There we may learn that as credit sinks, the body becomes a corpse, and this is what must happen very soon to the banker who is proud to own so good a logician as you for his daughter.> 

 But Eugénie, instead of stooping, drew herself up under the blow. <Ruined?> said she. 

 <Exactly, my daughter; that is precisely what I mean,> said Danglars, almost digging his nails into his breast, while he preserved on his harsh features the smile of the heartless though clever man; <ruined—yes, that is it.> 

 <Ah!> said Eugénie. 

 <Yes, ruined! Now it is revealed, this secret so full of horror, as the tragic poet says. Now, my daughter, learn from my lips how you may alleviate this misfortune, so far as it will affect you.> 

 <Oh,> cried Eugénie, <you are a bad physiognomist, if you imagine I deplore on my own account the catastrophe of which you warn me. I ruined? and what will that signify to me? Have I not my talent left? Can I not, like Pasta, Malibran, Grisi, acquire for myself what you would never have given me, whatever might have been your fortune, a hundred or a hundred and fifty thousand livres per annum, for which I shall be indebted to no one but myself; and which, instead of being given as you gave me those poor twelve thousand francs, with sour looks and reproaches for my prodigality, will be accompanied with acclamations, with bravos, and with flowers? And if I do not possess that talent, which your smiles prove to me you doubt, should I not still have that ardent love of independence, which will be a substitute for wealth, and which in my mind supersedes even the instinct of self-preservation? No, I grieve not on my own account, I shall always find a resource; my books, my pencils, my piano, all the things which cost but little, and which I shall be able to procure, will remain my own. 

 Do you think that I sorrow for Madame Danglars? Undeceive yourself again; either I am greatly mistaken, or she has provided against the catastrophe which threatens you, and, which will pass over without affecting her. She has taken care for herself,—at least I hope so,—for her attention has not been diverted from her projects by watching over me. She has fostered my independence by professedly indulging my love for liberty. Oh, no, sir; from my childhood I have seen too much, and understood too much, of what has passed around me, for misfortune to have an undue power over me. From my earliest recollections, I have been beloved by no one—so much the worse; that has naturally led me to love no one—so much the better—now you have my profession of faith.> 

 <Then,> said Danglars, pale with anger, which was not at all due to offended paternal love,—<then, mademoiselle, you persist in your determination to accelerate my ruin?> 

 <Your ruin? I accelerate your ruin? What do you mean? I do not understand you.> 

 <So much the better, I have a ray of hope left; listen.> 

 <I am all attention,> said Eugénie, looking so earnestly at her father that it was an effort for the latter to endure her unrelenting gaze. 

 <M. Cavalcanti,> continued Danglars, <is about to marry you, and will place in my hands his fortune, amounting to three million livres.> 

 <That is admirable!> said Eugénie with sovereign contempt, smoothing her gloves out one upon the other. 

 <You think I shall deprive you of those three millions,> said Danglars; <but do not fear it. They are destined to produce at least ten. I and a brother banker have obtained a grant of a railway, the only industrial enterprise which in these days promises to make good the fabulous prospects that Law once held out to the eternally deluded Parisians, in the fantastic Mississippi scheme. As I look at it, a millionth part of a railway is worth fully as much as an acre of waste land on the banks of the Ohio. We make in our case a deposit, on a mortgage, which is an advance, as you see, since we gain at least ten, fifteen, twenty, or a hundred livres' worth of iron in exchange for our money. Well, within a week I am to deposit four millions for my share; the four millions, I promise you, will produce ten or twelve.> 

 <But during my visit to you the day before yesterday, sir, which you appear to recollect so well,> replied Eugénie, <I saw you arranging a deposit—is not that the term?—of five millions and a half; you even pointed it out to me in two drafts on the treasury, and you were astonished that so valuable a paper did not dazzle my eyes like lightning.> 

 <Yes, but those five millions and a half are not mine, and are only a proof of the great confidence placed in me; my title of popular banker has gained me the confidence of charitable institutions, and the five millions and a half belong to them; at any other time I should not have hesitated to make use of them, but the great losses I have recently sustained are well known, and, as I told you, my credit is rather shaken. That deposit may be at any moment withdrawn, and if I had employed it for another purpose, I should bring on me a disgraceful bankruptcy. I do not despise bankruptcies, believe me, but they must be those which enrich, not those which ruin. Now, if you marry M. Cavalcanti, and I get the three millions, or even if it is thought I am going to get them, my credit will be restored, and my fortune, which for the last month or two has been swallowed up in gulfs which have been opened in my path by an inconceivable fatality, will revive. Do you understand me?> 

 <Perfectly; you pledge me for three millions, do you not?> 

 <The greater the amount, the more flattering it is to you; it gives you an idea of your value.> 

 <Thank you. One word more, sir; do you promise me to make what use you can of the report of the fortune M. Cavalcanti will bring without touching the money? This is no act of selfishness, but of delicacy. I am willing to help rebuild your fortune, but I will not be an accomplice in the ruin of others.> 

 <But since I tell you,> cried Danglars, <that with these three million\longdash> 

 <Do you expect to recover your position, sir, without touching those three million?> 

 <I hope so, if the marriage should take place and confirm my credit.> 

 <Shall you be able to pay M. Cavalcanti the five hundred thousand francs you promise for my dowry?> 

 <He shall receive them on returning from the mayor's\footnote{The performance of the civil marriage. }.> 

 <Very well!> 

 <What next? what more do you want?> 

 <I wish to know if, in demanding my signature, you leave me entirely free in my person?> 

 <Absolutely.> 

 <Then, as I said before, sir,—very well; I am ready to marry M. Cavalcanti.> 

 <But what are you up to?> 

 <Ah, that is my affair. What advantage should I have over you, if knowing your secret I were to tell you mine?> 

 Danglars bit his lips. <Then,> said he, <you are ready to pay the official visits, which are absolutely indispensable?> 

 <Yes,> replied Eugénie. 

 <And to sign the contract in three days?> 

 <Yes.> 

 <Then, in my turn, I also say, very well!> 

 Danglars pressed his daughter's hand in his. But, extraordinary to relate, the father did not say, <Thank you, my child,> nor did the daughter smile at her father. 

 <Is the conference ended?> asked Eugénie, rising. 

 Danglars motioned that he had nothing more to say. Five minutes afterwards the piano resounded to the touch of Mademoiselle d'Armilly's fingers, and Mademoiselle Danglars was singing Brabantio's malediction on Desdemona. At the end of the piece Étienne entered, and announced to Eugénie that the horses were to the carriage, and that the baroness was waiting for her to pay her visits. We have seen them at Villefort's; they proceeded then on their course. 