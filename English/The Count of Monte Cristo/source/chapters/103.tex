\chapter{Maximilian} 

 \lettrine{V}{illefort} rose, half-ashamed of being surprised in such a paroxysm of grief. The terrible office he had held for twenty-five years had succeeded in making him more or less than man. His glance, at first wandering, fixed itself upon Morrel. <Who are you, sir,> he asked, <that forget that this is not the manner to enter a house stricken with death? Go, sir, go!> 

 But Morrel remained motionless; he could not detach his eyes from that disordered bed, and the pale corpse of the young girl who was lying on it. 

 <Go!—do you hear?> said Villefort, while d'Avrigny advanced to lead Morrel out. Maximilian stared for a moment at the corpse, gazed all around the room, then upon the two men; he opened his mouth to speak, but finding it impossible to give utterance to the innumerable ideas that occupied his brain, he went out, thrusting his hands through his hair in such a manner that Villefort and d'Avrigny, for a moment diverted from the engrossing topic, exchanged glances, which seemed to say,—<He is mad!> 

 But in less than five minutes the staircase groaned beneath an extraordinary weight. Morrel was seen carrying, with superhuman strength, the armchair containing Noirtier upstairs. When he reached the landing he placed the armchair on the floor and rapidly rolled it into Valentine's room. This could only have been accomplished by means of unnatural strength supplied by powerful excitement. But the most fearful spectacle was Noirtier being pushed towards the bed, his face expressing all his meaning, and his eyes supplying the want of every other faculty. That pale face and flaming glance appeared to Villefort like a frightful apparition. Each time he had been brought into contact with his father, something terrible had happened. 

 <See what they have done!> cried Morrel, with one hand leaning on the back of the chair, and the other extended towards Valentine. <See, my father, see!> 

 Villefort drew back and looked with astonishment on the young man, who, almost a stranger to him, called Noirtier his father. At this moment the whole soul of the old man seemed centred in his eyes which became bloodshot; the veins of the throat swelled; his cheeks and temples became purple, as though he was struck with epilepsy; nothing was wanting to complete this but the utterance of a cry. And the cry issued from his pores, if we may thus speak—a cry frightful in its silence. D'Avrigny rushed towards the old man and made him inhale a powerful restorative. 

 <Sir,> cried Morrel, seizing the moist hand of the paralytic, <they ask me who I am, and what right I have to be here. Oh, you know it, tell them, tell them!> And the young man's voice was choked by sobs. 

 As for the old man, his chest heaved with his panting respiration. One could have thought that he was undergoing the agonies preceding death. At length, happier than the young man, who sobbed without weeping, tears glistened in the eyes of Noirtier. 

 <Tell them,> said Morrel in a hoarse voice, <tell them that I am her betrothed. Tell them she was my beloved, my noble girl, my only blessing in the world. Tell them—oh, tell them, that corpse belongs to me!> 

 The young man overwhelmed by the weight of his anguish, fell heavily on his knees before the bed, which his fingers grasped with convulsive energy. D'Avrigny, unable to bear the sight of this touching emotion, turned away; and Villefort, without seeking any further explanation, and attracted towards him by the irresistible magnetism which draws us towards those who have loved the people for whom we mourn, extended his hand towards the young man. 

 But Morrel saw nothing; he had grasped the hand of Valentine, and unable to weep vented his agony in groans as he bit the sheets. For some time nothing was heard in that chamber but sobs, exclamations, and prayers. At length Villefort, the most composed of all, spoke: 

 <Sir,> said he to Maximilian, <you say you loved Valentine, that you were betrothed to her. I knew nothing of this engagement, of this love, yet I, her father, forgive you, for I see that your grief is real and deep; and besides my own sorrow is too great for anger to find a place in my heart. But you see that the angel whom you hoped for has left this earth—she has nothing more to do with the adoration of men. Take a last farewell, sir, of her sad remains; take the hand you expected to possess once more within your own, and then separate yourself from her forever. Valentine now requires only the ministrations of the priest.>  <You are mistaken, sir,> exclaimed Morrel, raising himself on one knee, his heart pierced by a more acute pang than any he had yet felt—<you are mistaken; Valentine, dying as she has, not only requires a priest, but an avenger. \textit{You}, M. de Villefort, send for the priest; \textit{I}will be the avenger.>

 <What do you mean, sir?> asked Villefort, trembling at the new idea inspired by the delirium of Morrel. 

 <I tell you, sir, that two persons exist in you; the father has mourned sufficiently, now let the procureur fulfil his office.> 

 The eyes of Noirtier glistened, and d'Avrigny approached. 

 <Gentlemen,> said Morrel, reading all that passed through the minds of the witnesses to the scene, <I know what I am saying, and you know as well as I do what I am about to say—Valentine has been assassinated!> 

 Villefort hung his head, d'Avrigny approached nearer, and Noirtier said <Yes> with his eyes. 

 <Now, sir,> continued Morrel, <in these days no one can disappear by violent means without some inquiries being made as to the cause of her disappearance, even were she not a young, beautiful, and adorable creature like Valentine. Now, M. le Procureur du Roi,> said Morrel with increasing vehemence, <no mercy is allowed; I denounce the crime; it is your place to seek the assassin.> 

 The young man's implacable eyes interrogated Villefort, who, on his side, glanced from Noirtier to d'Avrigny. But instead of finding sympathy in the eyes of the doctor and his father, he only saw an expression as inflexible as that of Maximilian. 

 <Yes,> indicated the old man. 

 <Assuredly,> said d'Avrigny. 

 <Sir,> said Villefort, striving to struggle against this triple force and his own emotion,—<sir, you are deceived; no one commits crimes here. I am stricken by fate. It is horrible, indeed, but no one assassinates.> 

 The eyes of Noirtier lighted up with rage, and d'Avrigny prepared to speak. Morrel, however, extended his arm, and commanded silence. 

 <And I say that murders \textit{are} committed here,> said Morrel, whose voice, though lower in tone, lost none of its terrible distinctness: <I tell you that this is the fourth victim within the last four months. I tell you, Valentine's life was attempted by poison four days ago, though she escaped, owing to the precautions of M. Noirtier. I tell you that the dose has been double, the poison changed, and that this time it has succeeded. I tell you that you know these things as well as I do, since this gentleman has forewarned you, both as a doctor and as a friend.> 

 <Oh, you rave, sir,> exclaimed Villefort, in vain endeavouring to escape the net in which he was taken.  <I rave?> said Morrel; <well, then, I appeal to M. d'Avrigny himself. Ask him, sir, if he recollects the words he uttered in the garden of this house on the night of Madame de Saint-Méran's death. You thought yourselves alone, and talked about that tragical death, and the fatality you mentioned then is the same which has caused the murder of Valentine.> Villefort and d'Avrigny exchanged looks. 

 <Yes, yes,> continued Morrel; <recall the scene, for the words you thought were only given to silence and solitude fell into my ears. Certainly, after witnessing the culpable indolence manifested by M. de Villefort towards his own relations, I ought to have denounced him to the authorities; then I should not have been an accomplice to thy death, as I now am, sweet, beloved Valentine; but the accomplice shall become the avenger. This fourth murder is apparent to all, and if thy father abandon thee, Valentine, it is I, and I swear it, that shall pursue the assassin.> 

 And this time, as though nature had at least taken compassion on the vigourous frame, nearly bursting with its own strength, the words of Morrel were stifled in his throat; his breast heaved; the tears, so long rebellious, gushed from his eyes; and he threw himself weeping on his knees by the side of the bed. 

 Then d'Avrigny spoke. <And I, too,> he exclaimed in a low voice, <I unite with M. Morrel in demanding justice for crime; my blood boils at the idea of having encouraged a murderer by my cowardly concession.> 

 <Oh, merciful Heavens!> murmured Villefort. Morrel raised his head, and reading the eyes of the old man, which gleamed with unnatural lustre,— 

 <Stay,> he said, <M. Noirtier wishes to speak.> 

 <Yes,> indicated Noirtier, with an expression the more terrible, from all his faculties being centred in his glance. 

 <Do you know the assassin?> asked Morrel. 

 <Yes,> replied Noirtier. 

 <And will you direct us?> exclaimed the young man. <Listen, M. d'Avrigny, listen!> 

 Noirtier looked upon Morrel with one of those melancholy smiles which had so often made Valentine happy, and thus fixed his attention. Then, having riveted the eyes of his interlocutor on his own, he glanced towards the door. 

 <Do you wish me to leave?> said Morrel, sadly. 

 <Yes,> replied Noirtier. 

 <Alas, alas, sir, have pity on me!> 

 The old man's eyes remained fixed on the door. 

 <May I, at least, return?> asked Morrel. 

 <Yes.> 

 <Must I leave alone?> 

 <No.> 

 <Whom am I to take with me? The procureur?> 

 <No.> 

 <The doctor?> 

 <Yes.> 

 <You wish to remain alone with M. de Villefort?> 

 <Yes.> 

 <But can he understand you?> 

 <Yes.> 

 <Oh,> said Villefort, inexpressibly delighted to think that the inquiries were to be made by him alone,—<oh, be satisfied, I can understand my father.> While uttering these words with this expression of joy, his teeth clashed together violently. 

 D'Avrigny took the young man's arm, and led him out of the room. A more than deathlike silence then reigned in the house. At the end of a quarter of an hour a faltering footstep was heard, and Villefort appeared at the door of the apartment where d'Avrigny and Morrel had been staying, one absorbed in meditation, the other in grief. 

 <You can come,> he said, and led them back to Noirtier. 

 Morrel looked attentively on Villefort. His face was livid, large drops rolled down his face, and in his fingers he held the fragments of a quill pen which he had torn to atoms. 

 <Gentlemen,> he said in a hoarse voice, <give me your word of honour that this horrible secret shall forever remain buried amongst ourselves!> The two men drew back. 

 <I entreat you\longdash> continued Villefort. 

 <But,> said Morrel, <the culprit—the murderer—the assassin.> 

 <Do not alarm yourself, sir; justice will be done,> said Villefort. <My father has revealed the culprit's name; my father thirsts for revenge as much as you do, yet even he conjures you as I do to keep this secret. Do you not, father?> 

 <Yes,> resolutely replied Noirtier. Morrel suffered an exclamation of horror and surprise to escape him. 

 <Oh, sir,> said Villefort, arresting Maximilian by the arm, <if my father, the inflexible man, makes this request, it is because he knows, be assured, that Valentine will be terribly revenged. Is it not so, father?> 

 The old man made a sign in the affirmative. Villefort continued: 

 <He knows me, and I have pledged my word to him. Rest assured, gentlemen, that within three days, in a less time than justice would demand, the revenge I shall have taken for the murder of my child will be such as to make the boldest heart tremble;> and as he spoke these words he ground his teeth, and grasped the old man's senseless hand. 

 <Will this promise be fulfilled, M. Noirtier?> asked Morrel, while d'Avrigny looked inquiringly. 

 <Yes,> replied Noirtier with an expression of sinister joy. 

 <Swear, then,> said Villefort, joining the hands of Morrel and d'Avrigny, <swear that you will spare the honour of my house, and leave me to avenge my child.> 

 D'Avrigny turned round and uttered a very feeble <Yes,> but Morrel, disengaging his hand, rushed to the bed, and after having pressed the cold lips of Valentine with his own, hurriedly left, uttering a long, deep groan of despair and anguish. 

 We have before stated that all the servants had fled. M. de Villefort was therefore obliged to request M. d'Avrigny to superintend all the arrangements consequent upon a death in a large city, more especially a death under such suspicious circumstances. 

 It was something terrible to witness the silent agony, the mute despair of Noirtier, whose tears silently rolled down his cheeks. Villefort retired to his study, and d'Avrigny left to summon the doctor of the mayoralty, whose office it is to examine bodies after decease, and who is expressly named <the doctor of the dead.> M. Noirtier could not be persuaded to quit his grandchild. At the end of a quarter of an hour M. d'Avrigny returned with his associate; they found the outer gate closed, and not a servant remaining in the house; Villefort himself was obliged to open to them. But he stopped on the landing; he had not the courage to again visit the death chamber. The two doctors, therefore, entered the room alone. Noirtier was near the bed, pale, motionless, and silent as the corpse. The district doctor approached with the indifference of a man accustomed to spend half his time amongst the dead; he then lifted the sheet which was placed over the face, and just unclosed the lips. 

 <Alas,> said d'Avrigny, <she is indeed dead, poor child!>  <Yes,> answered the doctor laconically, dropping the sheet he had raised. Noirtier uttered a kind of hoarse, rattling sound; the old man's eyes sparkled, and the good doctor understood that he wished to behold his child. He therefore approached the bed, and while his companion was dipping the fingers with which he had touched the lips of the corpse in chloride of lime, he uncovered the calm and pale face, which looked like that of a sleeping angel. 

 A tear, which appeared in the old man's eye, expressed his thanks to the doctor. The doctor of the dead then laid his permit on the corner of the table, and having fulfilled his duty, was conducted out by d'Avrigny. Villefort met them at the door of his study; having in a few words thanked the district doctor, he turned to d'Avrigny, and said: 

 <And now the priest.> 

 <Is there any particular priest you wish to pray with Valentine?> asked d'Avrigny. 

 <No.> said Villefort; <fetch the nearest.> 

 <The nearest,> said the district doctor, <is a good Italian abbé, who lives next door to you. Shall I call on him as I pass?> 

 <D'Avrigny,> said Villefort, <be so kind, I beseech you, as to accompany this gentleman. Here is the key of the door, so that you can go in and out as you please; you will bring the priest with you, and will oblige me by introducing him into my child's room.>  <Do you wish to see him?> 

 <I only wish to be alone. You will excuse me, will you not? A priest can understand a father's grief.> 

 And M. de Villefort, giving the key to d'Avrigny, again bade farewell to the strange doctor, and retired to his study, where he began to work. For some temperaments work is a remedy for all afflictions. 

 As the doctors entered the street, they saw a man in a cassock standing on the threshold of the next door. 

 <This is the abbé of whom I spoke,> said the doctor to d'Avrigny. D'Avrigny accosted the priest. 

 <Sir,> he said, <are you disposed to confer a great obligation on an unhappy father who has just lost his daughter? I mean M. de Villefort, the king's attorney.> 

 <Ah,> said the priest, in a marked Italian accent; <yes, I have heard that death is in that house.> 

 <Then I need not tell you what kind of service he requires of you.> 

 <I was about to offer myself, sir,> said the priest; <it is our mission to forestall our duties.> 

 <It is a young girl.> 

 <I know it, sir; the servants who fled from the house informed me. I also know that her name is Valentine, and I have already prayed for her.> 

 <Thank you, sir,> said d'Avrigny; <since you have commenced your sacred office, deign to continue it. Come and watch by the dead, and all the wretched family will be grateful to you.> 

 <I am going, sir; and I do not hesitate to say that no prayers will be more fervent than mine.> 

 D'Avrigny took the priest's hand, and without meeting Villefort, who was engaged in his study, they reached Valentine's room, which on the following night was to be occupied by the undertakers. On entering the room, Noirtier's eyes met those of the abbé, and no doubt he read some particular expression in them, for he remained in the room. D'Avrigny recommended the attention of the priest to the living as well as to the dead, and the abbé promised to devote his prayers to Valentine and his attentions to Noirtier. 

 In order, doubtless, that he might not be disturbed while fulfilling his sacred mission, the priest rose as soon as d'Avrigny departed, and not only bolted the door through which the doctor had just left, but also that leading to Madame de Villefort's room. 