\chapter{Toxicology} 

 \lettrine{I}{t} was really the Count of Monte Cristo who had just arrived at Madame de Villefort's for the purpose of returning the procureur's visit, and at his name, as may be easily imagined, the whole house was in confusion. 

 Madame de Villefort, who was alone in her drawing-room when the count was announced, desired that her son might be brought thither instantly to renew his thanks to the count; and Edward, who heard this great personage talked of for two whole days, made all possible haste to come to him, not from obedience to his mother, or out of any feeling of gratitude to the count, but from sheer curiosity, and that some chance remark might give him the opportunity for making one of the impertinent speeches which made his mother say: 

 <Oh, that naughty child! But I can't be severe with him, he is really \textit{so} bright.> 

 After the usual civilities, the count inquired after M. de Villefort. 

 <My husband dines with the chancellor,> replied the young lady; <he has just gone, and I am sure he'll be exceedingly sorry not to have had the pleasure of seeing you before he went.> 

 Two visitors who were there when the count arrived, having gazed at him with all their eyes, retired after that reasonable delay which politeness admits and curiosity requires. 

 <What is your sister Valentine doing?> inquired Madame de Villefort of Edward; <tell someone to bid her come here, that I may have the honour of introducing her to the count.> 

 <You have a daughter, then, madame?> inquired the count; <very young, I presume?> 

 <The daughter of M. de Villefort by his first marriage,> replied the young wife, <a fine well-grown girl.> 

 <But melancholy,> interrupted Master Edward, snatching the feathers out of the tail of a splendid paroquet that was screaming on its gilded perch, in order to make a plume for his hat. 

 Madame de Villefort merely cried, <Be still, Edward!> She then added, <This young madcap is, however, very nearly right, and merely re-echoes what he has heard me say with pain a hundred times; for Mademoiselle de Villefort is, in spite of all we can do to rouse her, of a melancholy disposition and taciturn habit, which frequently injure the effect of her beauty. But what detains her? Go, Edward, and see.> 

 <Because they are looking for her where she is not to be found.> 

 <And where are they looking for her?> 

 <With grandpapa Noirtier.> 

 <And do you think she is not there?> 

 <No, no, no, no, no, she is not there,> replied Edward, singing his words. 

 <And where is she, then? If you know, why don't you tell?> 

 <She is under the big chestnut-tree,> replied the spoiled brat, as he gave, in spite of his mother's commands, live flies to the parrot, which seemed keenly to relish such fare. 

 Madame de Villefort stretched out her hand to ring, intending to direct her waiting-maid to the spot where she would find Valentine, when the young lady herself entered the apartment. She appeared much dejected; and any person who considered her attentively might have observed the traces of recent tears in her eyes. 

 Valentine, whom we have in the rapid march of our narrative presented to our readers without formally introducing her, was a tall and graceful girl of nineteen, with bright chestnut hair, deep blue eyes, and that reposeful air of quiet distinction which characterized her mother. Her white and slender fingers, her pearly neck, her cheeks tinted with varying hues reminded one of the lovely Englishwomen who have been so poetically compared in their manner to the gracefulness of a swan. 

 She entered the apartment, and seeing near her stepmother the stranger of whom she had already heard so much, saluted him without any girlish awkwardness, or even lowering her eyes, and with an elegance that redoubled the count's attention. 

 He rose to return the salutation. 

 <Mademoiselle de Villefort, my step-daughter,> said Madame de Villefort to Monte Cristo, leaning back on her sofa and motioning towards Valentine with her hand. 

 <And M. de Monte Cristo, King of China, Emperor of Cochin-China,> said the young imp, looking slyly towards his sister. 

 Madame de Villefort at this really did turn pale, and was very nearly angry with this household plague, who answered to the name of Edward; but the count, on the contrary, smiled, and appeared to look at the boy complacently, which caused the maternal heart to bound again with joy and enthusiasm. 

 <But, madame,> replied the count, continuing the conversation, and looking by turns at Madame de Villefort and Valentine, <have I not already had the honour of meeting yourself and mademoiselle before? I could not help thinking so just now; the idea came over my mind, and as mademoiselle entered the sight of her was an additional ray of light thrown on a confused remembrance; excuse the remark.> 

 <I do not think it likely, sir; Mademoiselle de Villefort is not very fond of society, and we very seldom go out,> said the young lady. 

 <Then it was not in society that I met with mademoiselle or yourself, madame, or this charming little merry boy. Besides, the Parisian world is entirely unknown to me, for, as I believe I told you, I have been in Paris but very few days. No,—but, perhaps, you will permit me to call to mind—stay!> 

 The Count placed his hand on his brow as if to collect his thoughts. 

 <No—it was somewhere—away from here—it was—I do not know—but it appears that this recollection is connected with a lovely sky and some religious \textit{fête}; mademoiselle was holding flowers in her hand, the interesting boy was chasing a beautiful peacock in a garden, and you, madame, were under the trellis of some arbour. Pray come to my aid, madame; do not these circumstances appeal to your memory?> 

 <No, indeed,> replied Madame de Villefort; <and yet it appears to me, sir, that if I had met you anywhere, the recollection of you must have been imprinted on my memory.> 

 <Perhaps the count saw us in Italy,> said Valentine timidly. 

 <Yes, in Italy; it was in Italy most probably,> replied Monte Cristo; <you have travelled then in Italy, mademoiselle?> 

 <Yes; madame and I were there two years ago. The doctors, anxious for my lungs, had prescribed the air of Naples. We went by Bologna, Perugia, and Rome.> 

 <Ah, yes—true, mademoiselle,> exclaimed Monte Cristo as if this simple explanation was sufficient to revive the recollection he sought. <It was at Perugia on Corpus Christi Day, in the garden of the Hôtel des Postes, when chance brought us together; you, Madame de Villefort, and her son; I now remember having had the honour of meeting you.> 

 <I perfectly well remember Perugia, sir, and the Hôtel des Postes, and the festival of which you speak,> said Madame de Villefort, <but in vain do I tax my memory, of whose treachery I am ashamed, for I really do not recall to mind that I ever had the pleasure of seeing you before.> 

 <It is strange, but neither do I recollect meeting with you,> observed Valentine, raising her beautiful eyes to the count.  <But I remember it perfectly,> interposed the darling Edward. 

 <I will assist your memory, madame,> continued the count; <the day had been burning hot; you were waiting for horses, which were delayed in consequence of the festival. Mademoiselle was walking in the shade of the garden, and your son disappeared in pursuit of the peacock.> 

 <And I caught it, mamma, don't you remember?> interposed Edward, <and I pulled three such beautiful feathers out of his tail.> 

 <You, madame, remained under the arbour; do you not remember, that while you were seated on a stone bench, and while, as I told you, Mademoiselle de Villefort and your young son were absent, you conversed for a considerable time with somebody?> 

 <Yes, in truth, yes,> answered the young lady, turning very red, <I do remember conversing with a person wrapped in a long woollen mantle; he was a medical man, I think.> 

 <Precisely so, madame; this man was myself; for a fortnight I had been at that hotel, during which period I had cured my valet de chambre of a fever, and my landlord of the jaundice, so that I really acquired a reputation as a skilful physician. We discoursed a long time, madame, on different subjects; of Perugino, of Raphael, of manners, customs, of the famous \textit{aqua Tofana}, of which they had told you, I think you said, that certain individuals in Perugia had preserved the secret.> 

 <Yes, true,> replied Madame de Villefort, somewhat uneasily, <I remember now.> 

 <I do not recollect now all the various subjects of which we discoursed, madame,> continued the count with perfect calmness; <but I perfectly remember that, falling into the error which others had entertained respecting me, you consulted me as to the health of Mademoiselle de Villefort.> 

 <Yes, really, sir, you were in fact a medical man,> said Madame de Villefort, <since you had cured the sick.> 

 <Molière or Beaumarchais would reply to you, madame, that it was precisely because I was not, that I had cured my patients; for myself, I am content to say to you that I have studied chemistry and the natural sciences somewhat deeply, but still only as an amateur, you understand.> 

 At this moment the clock struck six. 

 <It is six o'clock,> said Madame de Villefort, evidently agitated. <Valentine, will you not go and see if your grandpapa will have his dinner?> 

 Valentine rose, and saluting the count, left the apartment without speaking. 

 <Oh, madame,> said the count, when Valentine had left the room, <was it on my account that you sent Mademoiselle de Villefort away?> 

 <By no means,> replied the young lady quickly; <but this is the hour when we usually give M. Noirtier the unwelcome meal that sustains his pitiful existence. You are aware, sir, of the deplorable condition of my husband's father?> 

 <Yes, madame, M. de Villefort spoke of it to me—a paralysis, I think.> 

 <Alas, yes; the poor old gentleman is entirely helpless; the mind alone is still active in this human machine, and that is faint and flickering, like the light of a lamp about to expire. But excuse me, sir, for talking of our domestic misfortunes; I interrupted you at the moment when you were telling me that you were a skilful chemist.> 

 <No, madame, I did not say as much as that,> replied the count with a smile; <quite the contrary. I have studied chemistry because, having determined to live in eastern climates I have been desirous of following the example of King Mithridates.> 

 <\textit{Mithridates, rex Ponticus},> said the young scamp, as he tore some beautiful portraits out of a splendid album, <the individual who took cream in his cup of poison every morning at breakfast.> 

 <Edward, you naughty boy,> exclaimed Madame de Villefort, snatching the mutilated book from the urchin's grasp, <you are positively past bearing; you really disturb the conversation; go, leave us, and join your sister Valentine in dear grandpapa Noirtier's room.> 

 <The album,> said Edward sulkily. 

 <What do you mean?—the album!> 

 <I want the album.> 

 <How dare you tear out the drawings?> 

 <Oh, it amuses me.> 

 <Go—go at once.> 

 <I won't go unless you give me the album,> said the boy, seating himself doggedly in an armchair, according to his habit of never giving way. 

 <Take it, then, and pray disturb us no longer,> said Madame de Villefort, giving the album to Edward, who then went towards the door, led by his mother. The count followed her with his eyes. 

 <Let us see if she shuts the door after him,> he muttered. 

 Madame de Villefort closed the door carefully after the child, the count appearing not to notice her; then casting a scrutinizing glance around the chamber, the young wife returned to her chair, in which she seated herself. 

 <Allow me to observe, madame,> said the count, with that kind tone he could assume so well, <you are really very severe with that dear clever child.> 

 <Oh, sometimes severity is quite necessary,> replied Madame de Villefort, with all a mother's real firmness. 

 <It was his Cornelius Nepos that Master Edward was repeating when he referred to King Mithridates,> continued the count, <and you interrupted him in a quotation which proves that his tutor has by no means neglected him, for your son is really advanced for his years.> 

 <The fact is, count,> answered the mother, agreeably flattered, <he has great aptitude, and learns all that is set before him. He has but one fault, he is somewhat wilful; but really, on referring for the moment to what he said, do you truly believe that Mithridates used these precautions, and that these precautions were efficacious?> 

 <I think so, madame, because I myself have made use of them, that I might not be poisoned at Naples, at Palermo, and at Smyrna—that is to say, on three several occasions when, but for these precautions, I must have lost my life.> 

 <And your precautions were successful?> 

 <Completely so.> 

 <Yes, I remember now your mentioning to me at Perugia something of this sort.> 

 <Indeed?> said the count with an air of surprise, remarkably well counterfeited; <I really did not remember.> 

 <I inquired of you if poisons acted equally, and with the same effect, on men of the North as on men of the South; and you answered me that the cold and sluggish habits of the North did not present the same aptitude as the rich and energetic temperaments of the natives of the South.> 

 <And that is the case,> observed Monte Cristo. <I have seen Russians devour, without being visibly inconvenienced, vegetable substances which would infallibly have killed a Neapolitan or an Arab.> 

 <And you really believe the result would be still more sure with us than in the East, and in the midst of our fogs and rains a man would habituate himself more easily than in a warm latitude to this progressive absorption of poison?> 

 <Certainly; it being at the same time perfectly understood that he should have been duly fortified against the poison to which he had not been accustomed.> 

 <Yes, I understand that; and how would you habituate yourself, for instance, or rather, how did you habituate yourself to it?> 

 <Oh, very easily. Suppose you knew beforehand the poison that would be made use of against you; suppose the poison was, for instance, brucine\longdash> 

 <Brucine is extracted from the false angostura\footnote{\itshape Brucea ferruginea.} is it not?> inquired Madame de Villefort. 

 <Precisely, madame,> replied Monte Cristo; <but I perceive I have not much to teach you. Allow me to compliment you on your knowledge; such learning is very rare among ladies.> 

 <Oh, I am aware of that,> said Madame de Villefort; <but I have a passion for the occult sciences, which speak to the imagination like poetry, and are reducible to figures, like an algebraic equation; but go on, I beg of you; what you say interests me to the greatest degree.>  
 
 <Well,> replied Monte Cristo <suppose, then, that this poison was brucine, and you were to take a milligramme the first day, two milligrammes the second day, and so on. Well, at the end of ten days you would have taken a centigramme, at the end of twenty days, increasing another milligramme, you would have taken three hundred centigrammes; that is to say, a dose which you would support without inconvenience, and which would be very dangerous for any other person who had not taken the same precautions as yourself. Well, then, at the end of a month, when drinking water from the same carafe, you would kill the person who drank with you, without your perceiving, otherwise than from slight inconvenience, that there was any poisonous substance mingled with this water.> 

 <Do you know any other counter-poisons?> 

 <I do not.> 

 <I have often read, and read again, the history of Mithridates,> said Madame de Villefort in a tone of reflection, <and had always considered it a fable.> 

 <No, madame, contrary to most history, it is true; but what you tell me, madame, what you inquire of me, is not the result of a chance query, for two years ago you asked me the same questions, and said then, that for a very long time this history of Mithridates had occupied your mind.> 

 <True, sir. The two favourite studies of my youth were botany and mineralogy, and subsequently, when I learned that the use of simples frequently explained the whole history of a people, and the entire life of individuals in the East, as flowers betoken and symbolize a love affair, I have regretted that I was not a man, that I might have been a Flamel, a Fontana, or a Cabanis.> 

 <And the more, madame,> said Monte Cristo, <as the Orientals do not confine themselves, as did Mithridates, to make a cuirass of his poisons, but they also made them a dagger. Science becomes, in their hands, not only a defensive weapon, but still more frequently an offensive one; the one serves against all their physical sufferings, the other against all their enemies. With opium, belladonna, brucea, snake-wood, and the cherry-laurel, they put to sleep all who stand in their way. There is not one of those women, Egyptian, Turkish, or Greek, whom here you call <good women,> who do not know how, by means of chemistry, to stupefy a doctor, and in psychology to amaze a confessor.> 

 <Really,> said Madame de Villefort, whose eyes sparkled with strange fire at this conversation. 

 <Oh, yes, indeed, madame,> continued Monte Cristo, <the secret dramas of the East begin with a love philtre and end with a death potion—begin with paradise and end with—hell. There are as many elixirs of every kind as there are caprices and peculiarities in the physical and moral nature of humanity; and I will say further—the art of these chemists is capable with the utmost precision to accommodate and proportion the remedy and the bane to yearnings for love or desires for vengeance.> 

 <But, sir,> remarked the young woman, <these Eastern societies, in the midst of which you have passed a portion of your existence, are as fantastic as the tales that come from their strange land. A man can easily be put out of the way there, then; it is, indeed, the Bagdad and Bassora of the \textit{Thousand and One Nights}. The sultans and viziers who rule over society there, and who constitute what in France we call the government, are really Haroun-al-Raschids and Giaffars, who not only pardon a poisoner, but even make him a prime minister, if his crime has been an ingenious one, and who, under such circumstances, have the whole story written in letters of gold, to divert their hours of idleness and \textit{ennui}.> 

 <By no means, madame; the fanciful exists no longer in the East. There, disguised under other names, and concealed under other costumes, are police agents, magistrates, attorneys-general, and bailiffs. They hang, behead, and impale their criminals in the most agreeable possible manner; but some of these, like clever rogues, have contrived to escape human justice, and succeed in their fraudulent enterprises by cunning stratagems. Amongst us a simpleton, possessed by the demon of hate or cupidity, who has an enemy to destroy, or some near relation to dispose of, goes straight to the grocer's or druggist's, gives a false name, which leads more easily to his detection than his real one, and under the pretext that the rats prevent him from sleeping, purchases five or six grammes of arsenic—if he is really a cunning fellow, he goes to five or six different druggists or grocers, and thereby becomes only five or six times more easily traced;—then, when he has acquired his specific, he administers duly to his enemy, or near kinsman, a dose of arsenic which would make a mammoth or mastodon burst, and which, without rhyme or reason, makes his victim utter groans which alarm the entire neighbourhood. Then arrive a crowd of policemen and constables. They fetch a doctor, who opens the dead body, and collects from the entrails and stomach a quantity of arsenic in a spoon. Next day a hundred newspapers relate the fact, with the names of the victim and the murderer. The same evening the grocer or grocers, druggist or druggists, come and say, <It was I who sold the arsenic to the gentleman;> and rather than not recognize the guilty purchaser, they will recognize twenty. Then the foolish criminal is taken, imprisoned, interrogated, confronted, confounded, condemned, and cut off by hemp or steel; or if she be a woman of any consideration, they lock her up for life. This is the way in which you Northerns understand chemistry, madame. Desrues was, however, I must confess, more skilful.> 

 <What would you have, sir?> said the lady, laughing; <we do what we can. All the world has not the secret of the Medicis or the Borgias.> 

 <Now,> replied the count, shrugging his shoulders, <shall I tell you the cause of all these stupidities? It is because, at your theatres, by what at least I could judge by reading the pieces they play, they see persons swallow the contents of a phial, or suck the button of a ring, and fall dead instantly. Five minutes afterwards the curtain falls, and the spectators depart. They are ignorant of the consequences of the murder; they see neither the police commissary with his badge of office, nor the corporal with his four men; and so the poor fools believe that the whole thing is as easy as lying. But go a little way from France—go either to Aleppo or Cairo, or only to Naples or Rome, and you will see people passing by you in the streets—people erect, smiling, and fresh-coloured, of whom Asmodeus, if you were holding on by the skirt of his mantle, would say, <That man was poisoned three weeks ago; he will be a dead man in a month.>> 

 <Then,> remarked Madame de Villefort, <they have again discovered the secret of the famous \textit{aqua Tofana} that they said was lost at Perugia.> 

 <Ah, but madame, does mankind ever lose anything? The arts change about and make a tour of the world; things take a different name, and the vulgar do not follow them—that is all; but there is always the same result. Poisons act particularly on some organ or another—one on the stomach, another on the brain, another on the intestines. Well, the poison brings on a cough, the cough an inflammation of the lungs, or some other complaint catalogued in the book of science, which, however, by no means precludes it from being decidedly mortal; and if it were not, would be sure to become so, thanks to the remedies applied by foolish doctors, who are generally bad chemists, and which will act in favour of or against the malady, as you please; and then there is a human being killed according to all the rules of art and skill, and of whom justice learns nothing, as was said by a terrible chemist of my acquaintance, the worthy Abbé Adelmonte of Taormina, in Sicily, who has studied these national phenomena very profoundly.> 

 <It is quite frightful, but deeply interesting,> said the young lady, motionless with attention. <I thought, I must confess, that these tales, were inventions of the Middle Ages.> 

 <Yes, no doubt, but improved upon by ours. What is the use of time, rewards of merit, medals, crosses, Monthyon prizes, if they do not lead society towards more complete perfection? Yet man will never be perfect until he learns to create and destroy; he does know how to destroy, and that is half the battle.> 

 <So,> added Madame de Villefort, constantly returning to her object, <the poisons of the Borgias, the Medicis, the Renées, the Ruggieris, and later, probably, that of Baron de Trenck, whose story has been so misused by modern drama and romance\longdash> 

 <Were objects of art, madame, and nothing more,> replied the count. <Do you suppose that the real \textit{savant} addresses himself stupidly to the mere individual? By no means. Science loves eccentricities, leaps and bounds, trials of strength, fancies, if I may be allowed so to term them. Thus, for instance, the excellent Abbé Adelmonte, of whom I spoke just now, made in this way some marvellous experiments.> 

 <Really?> 

 <Yes; I will mention one to you. He had a remarkably fine garden, full of vegetables, flowers, and fruit. From amongst these vegetables he selected the most simple—a cabbage, for instance. For three days he watered this cabbage with a distillation of arsenic; on the third, the cabbage began to droop and turn yellow. At that moment he cut it. In the eyes of everybody it seemed fit for table, and preserved its wholesome appearance. It was only poisoned to the Abbé Adelmonte. He then took the cabbage to the room where he had rabbits—for the Abbé Adelmonte had a collection of rabbits, cats, and guinea-pigs, fully as fine as his collection of vegetables, flowers, and fruit. Well, the Abbé Adelmonte took a rabbit, and made it eat a leaf of the cabbage. The rabbit died. What magistrate would find, or even venture to insinuate, anything against this? What procureur has ever ventured to draw up an accusation against M. Magendie or M. Flourens, in consequence of the rabbits, cats, and guinea-pigs they have killed?—not one. So, then, the rabbit dies, and justice takes no notice. This rabbit dead, the Abbé Adelmonte has its entrails taken out by his cook and thrown on the dunghill; on this dunghill is a hen, who, pecking these intestines, is in her turn taken ill, and dies next day. At the moment when she is struggling in the convulsions of death, a vulture is flying by (there are a good many vultures in Adelmonte's country); this bird darts on the dead fowl, and carries it away to a rock, where it dines off its prey. Three days afterwards, this poor vulture, which has been very much indisposed since that dinner, suddenly feels very giddy while flying aloft in the clouds, and falls heavily into a fish-pond. The pike, eels, and carp eat greedily always, as everybody knows—well, they feast on the vulture. Now suppose that next day, one of these eels, or pike, or carp, poisoned at the fourth remove, is served up at your table. Well, then, your guest will be poisoned at the fifth remove, and die, at the end of eight or ten days, of pains in the intestines, sickness, or abscess of the pylorus. The doctors open the body and say with an air of profound learning, <The subject has died of a tumour on the liver, or of typhoid fever!>> 

 <But,> remarked Madame de Villefort, <all these circumstances which you link thus to one another may be broken by the least accident; the vulture may not see the fowl, or may fall a hundred yards from the fish-pond.> 

 <Ah, that is where the art comes in. To be a great chemist in the East, one must direct chance; and this is to be achieved.> 

 Madame de Villefort was in deep thought, yet listened attentively. 

 <But,> she exclaimed, suddenly, <arsenic is indelible, indestructible; in whatsoever way it is absorbed, it will be found again in the body of the victim from the moment when it has been taken in sufficient quantity to cause death.> 

 <Precisely so,> cried Monte Cristo—<precisely so; and this is what I said to my worthy Adelmonte. He reflected, smiled, and replied to me by a Sicilian proverb, which I believe is also a French proverb, <My son, the world was not made in a day—but in seven. Return on Sunday.> On the Sunday following I did return to him. Instead of having watered his cabbage with arsenic, he had watered it this time with a solution of salts, having their basis in strychnine, \textit{strychnos colubrina}, as the learned term it. Now, the cabbage had not the slightest appearance of disease in the world, and the rabbit had not the smallest distrust; yet, five minutes afterwards, the rabbit was dead. The fowl pecked at the rabbit, and the next day was a dead hen. This time we were the vultures; so we opened the bird, and this time all special symptoms had disappeared, there were only general symptoms. There was no peculiar indication in any organ—an excitement of the nervous system—that was it; a case of cerebral congestion—nothing more. The fowl had not been poisoned—she had died of apoplexy. Apoplexy is a rare disease among fowls, I believe, but very common among men.> 

 Madame de Villefort appeared more and more thoughtful. 

 <It is very fortunate,> she observed, <that such substances could only be prepared by chemists; otherwise, all the world would be poisoning each other.> 

 <By chemists and persons who have a taste for chemistry,> said Monte Cristo carelessly. 

 <And then,> said Madame de Villefort, endeavouring by a struggle, and with effort, to get away from her thoughts, <however skilfully it is prepared, crime is always crime, and if it avoid human scrutiny, it does not escape the eye of God. The Orientals are stronger than we are in cases of conscience, and, very prudently, have no hell—that is the point.>  
 
 <Really, madame, this is a scruple which naturally must occur to a pure mind like yours, but which would easily yield before sound reasoning. The bad side of human thought will always be defined by the paradox of Jean Jacques Rousseau,—you remember,—the mandarin who is killed five hundred leagues off by raising the tip of the finger. Man's whole life passes in doing these things, and his intellect is exhausted by reflecting on them. You will find very few persons who will go and brutally thrust a knife in the heart of a fellow-creature, or will administer to him, in order to remove him from the surface of the globe on which we move with life and animation, that quantity of arsenic of which we just now talked. Such a thing is really out of rule—eccentric or stupid. To attain such a point, the blood must be heated to thirty-six degrees, the pulse be, at least, at ninety, and the feelings excited beyond the ordinary limit. But suppose one pass, as is permissible in philology, from the word itself to its softened synonym, then, instead of committing an ignoble assassination you make an <elimination;> you merely and simply remove from your path the individual who is in your way, and that without shock or violence, without the display of the sufferings which, in the case of becoming a punishment, make a martyr of the victim, and a butcher, in every sense of the word, of him who inflicts them. Then there will be no blood, no groans, no convulsions, and above all, no consciousness of that horrid and compromising moment of accomplishing the act,—then one escapes the clutch of the human law, which says, <Do not disturb society!> This is the mode in which they manage these things, and succeed in Eastern climes, where there are grave and phlegmatic persons who care very little for the questions of time in conjunctures of importance.> 

 <Yet conscience remains,> remarked Madame de Villefort in an agitated voice, and with a stifled sigh. 

 <Yes,> answered Monte Cristo <happily, yes, conscience does remain; and if it did not, how wretched we should be! After every action requiring exertion, it is conscience that saves us, for it supplies us with a thousand good excuses, of which we alone are judges; and these reasons, howsoever excellent in producing sleep, would avail us but very little before a tribunal, when we were tried for our lives. Thus Richard \textsc{iii.}, for instance, was marvellously served by his conscience after the putting away of the two children of Edward \textsc{iv.}; in fact, he could say, <These two children of a cruel and persecuting king, who have inherited the vices of their father, which I alone could perceive in their juvenile propensities—these two children are impediments in my way of promoting the happiness of the English people, whose unhappiness they (the children) would infallibly have caused.' Thus was Lady Macbeth served by her conscience, when she sought to give her son, and not her husband (whatever Shakespeare may say), a throne. Ah, maternal love is a great virtue, a powerful motive—so powerful that it excuses a multitude of things, even if, after Duncan>s death, Lady Macbeth had been at all pricked by her conscience.> 

 Madame de Villefort listened with avidity to these appalling maxims and horrible paradoxes, delivered by the count with that ironical simplicity which was peculiar to him. 

 After a moment's silence, the lady inquired: 

 <Do you know, my dear count,> she said, <that you are a very terrible reasoner, and that you look at the world through a somewhat distempered medium? Have you really measured the world by scrutinies, or through alembics and crucibles? For you must indeed be a great chemist, and the elixir you administered to my son, which recalled him to life almost instantaneously\longdash>  
 
 <Oh, do not place any reliance on that, madame; \textit{one} drop of that elixir sufficed to recall life to a dying child, but three drops would have impelled the blood into his lungs in such a way as to have produced most violent palpitations; six would have suspended his respiration, and caused syncope more serious than that in which he was; ten would have destroyed him. You know, madame, how suddenly I snatched him from those phials which he so imprudently touched?> 

 <Is it then so terrible a poison?> 

 <Oh, no! In the first place, let us agree that the word poison does not exist, because in medicine use is made of the most violent poisons, which become, according as they are employed, most salutary remedies.> 

 <What, then, is it?> 

 <A skilful preparation of my friend's the worthy Abbé Adelmonte, who taught me the use of it.> 

 <Oh,> observed Madame de Villefort, <it must be an admirable anti-spasmodic.> 

 <Perfect, madame, as you have seen,> replied the count; <and I frequently make use of it—with all possible prudence though, be it observed,> he added with a smile of intelligence. 

 <Most assuredly,> responded Madame de Villefort in the same tone. <As for me, so nervous, and so subject to fainting fits, I should require a Doctor Adelmonte to invent for me some means of breathing freely and tranquillizing my mind, in the fear I have of dying some fine day of suffocation. In the meanwhile, as the thing is difficult to find in France, and your abbé is not probably disposed to make a journey to Paris on my account, I must continue to use Monsieur Planche's anti-spasmodics; and mint and Hoffman's drops are among my favourite remedies. Here are some lozenges which I have made up on purpose; they are compounded doubly strong.> 

 Monte Cristo opened the tortoise-shell box, which the lady presented to him, and inhaled the odor of the lozenges with the air of an amateur who thoroughly appreciated their composition. 

 <They are indeed exquisite,> he said; <but as they are necessarily submitted to the process of deglutition—a function which it is frequently impossible for a fainting person to accomplish—I prefer my own specific.> 

 <Undoubtedly, and so should I prefer it, after the effects I have seen produced; but of course it is a secret, and I am not so indiscreet as to ask it of you.> 

 <But I,> said Monte Cristo, rising as he spoke—<I am gallant enough to offer it you.>

<How kind you are.> 

 <Only remember one thing—a small dose is a remedy, a large one is poison. One drop will restore life, as you have seen; five or six will inevitably kill, and in a way the more terrible inasmuch as, poured into a glass of wine, it would not in the slightest degree affect its flavour. But I say no more, madame; it is really as if I were prescribing for you.> 

 The clock struck half-past six, and a lady was announced, a friend of Madame de Villefort, who came to dine with her. 

 <If I had had the honour of seeing you for the third or fourth time, count, instead of only for the second,> said Madame de Villefort; <if I had had the honour of being your friend, instead of only having the happiness of being under an obligation to you, I should insist on detaining you to dinner, and not allow myself to be daunted by a first refusal.> 

 <A thousand thanks, madame,> replied Monte Cristo <but I have an engagement which I cannot break. I have promised to escort to the Académie a Greek princess of my acquaintance who has never seen your grand opera, and who relies on me to conduct her thither.> 

 <Adieu, then, sir, and do not forget the prescription.> 

 <Ah, in truth, madame, to do that I must forget the hour's conversation I have had with you, which is indeed impossible.> 

 Monte Cristo bowed, and left the house. Madame de Villefort remained immersed in thought. 

 <He is a very strange man,> she said, <and in my opinion is himself the Adelmonte he talks about.> 

 As to Monte Cristo the result had surpassed his utmost expectations. 

 <Good,> said he, as he went away; <this is a fruitful soil, and I feel certain that the seed sown will not be cast on barren ground.> 

 Next morning, faithful to his promise, he sent the prescription requested. 