\chapter{A Learned Italian} 

 \lettrine{S}{eizing} in his arms the friend so long and ardently desired, Dantès almost carried him towards the window, in order to obtain a better view of his features by the aid of the imperfect light that struggled through the grating. 

 He was a man of small stature, with hair blanched rather by suffering and sorrow than by age. He had a deep-set, penetrating eye, almost buried beneath the thick gray eyebrow, and a long (and still black) beard reaching down to his breast. His thin face, deeply furrowed by care, and the bold outline of his strongly marked features, betokened a man more accustomed to exercise his mental faculties than his physical strength. Large drops of perspiration were now standing on his brow, while the garments that hung about him were so ragged that one could only guess at the pattern upon which they had originally been fashioned. 

 The stranger might have numbered sixty or sixty-five years; but a certain briskness and appearance of vigour in his movements made it probable that he was aged more from captivity than the course of time. He received the enthusiastic greeting of his young acquaintance with evident pleasure, as though his chilled affections were rekindled and invigourated by his contact with one so warm and ardent. He thanked him with grateful cordiality for his kindly welcome, although he must at that moment have been suffering bitterly to find another dungeon where he had fondly reckoned on discovering a means of regaining his liberty. 

 <Let us first see,> said he, <whether it is possible to remove the traces of my entrance here—our future tranquillity depends upon our jailers being entirely ignorant of it.> 

 Advancing to the opening, he stooped and raised the stone easily in spite of its weight; then, fitting it into its place, he said: 

 <You removed this stone very carelessly; but I suppose you had no tools to aid you.> 

 <Why,> exclaimed Dantès, with astonishment, <do you possess any?> 

 <I made myself some; and with the exception of a file, I have all that are necessary,—a chisel, pincers, and lever.>  <Oh, how I should like to see these products of your industry and patience.> 

 <Well, in the first place, here is my chisel.> 

 So saying, he displayed a sharp strong blade, with a handle made of beechwood. 

 <And with what did you contrive to make that?> inquired Dantès. 

 <With one of the clamps of my bedstead; and this very tool has sufficed me to hollow out the road by which I came hither, a distance of about fifty feet.> 

 <Fifty feet!> responded Dantès, almost terrified. 

 <Do not speak so loud, young man—don't speak so loud. It frequently occurs in a state prison like this, that persons are stationed outside the doors of the cells purposely to overhear the conversation of the prisoners.> 

 <But they believe I am shut up alone here.> 

 <That makes no difference.> 

 <And you say that you dug your way a distance of fifty feet to get here?> 

 <I do; that is about the distance that separates your chamber from mine; only, unfortunately, I did not curve aright; for want of the necessary geometrical instruments to calculate my scale of proportion, instead of taking an ellipsis of forty feet, I made it fifty. I expected, as I told you, to reach the outer wall, pierce through it, and throw myself into the sea; I have, however, kept along the corridor on which your chamber opens, instead of going beneath it. My labour is all in vain, for I find that the corridor looks into a courtyard filled with soldiers.> 

 <That's true,> said Dantès; <but the corridor you speak of only bounds \textit{one} side of my cell; there are three others—do you know anything of their situation?> 

 <This one is built against the solid rock, and it would take ten experienced miners, duly furnished with the requisite tools, as many years to perforate it. This adjoins the lower part of the governor's apartments, and were we to work our way through, we should only get into some lock-up cellars, where we must necessarily be recaptured. The fourth and last side of your cell faces on—faces on—stop a minute, now where does it face?> 

 The wall of which he spoke was the one in which was fixed the loophole by which light was admitted to the chamber. This loophole, which gradually diminished in size as it approached the outside, to an opening through which a child could not have passed, was, for better security, furnished with three iron bars, so as to quiet all apprehensions even in the mind of the most suspicious jailer as to the possibility of a prisoner's escape. As the stranger asked the question, he dragged the table beneath the window. 

 <Climb up,> said he to Dantès. 

 The young man obeyed, mounted on the table, and, divining the wishes of his companion, placed his back securely against the wall and held out both hands. The stranger, whom as yet Dantès knew only by the number of his cell, sprang up with an agility by no means to be expected in a person of his years, and, light and steady on his feet as a cat or a lizard, climbed from the table to the outstretched hands of Dantès, and from them to his shoulders; then, bending double, for the ceiling of the dungeon prevented him from holding himself erect, he managed to slip his head between the upper bars of the window, so as to be able to command a perfect view from top to bottom. 

 An instant afterwards he hastily drew back his head, saying, <I thought so!> and sliding from the shoulders of Dantès as dextrously as he had ascended, he nimbly leaped from the table to the ground. 

 <What was it that you thought?> asked the young man anxiously, in his turn descending from the table. 

 The elder prisoner pondered the matter. <Yes,> said he at length, <it is so. This side of your chamber looks out upon a kind of open gallery, where patrols are continually passing, and sentries keep watch day and night.> 

 <Are you quite sure of that?> 

 <Certain. I saw the soldier's shape and the top of his musket; that made me draw in my head so quickly, for I was fearful he might also see me.> 

 <Well?> inquired Dantès. 

 <You perceive then the utter impossibility of escaping through your dungeon?> 

 <Then\longdash> pursued the young man eagerly. 

 <Then,> answered the elder prisoner, <the will of God be done!> And as the old man slowly pronounced those words, an air of profound resignation spread itself over his careworn countenance. Dantès gazed on the man who could thus philosophically resign hopes so long and ardently nourished with an astonishment mingled with admiration. 

 <Tell me, I entreat of you, who and what you are?> said he at length. <Never have I met with so remarkable a person as yourself.> 

 <Willingly,> answered the stranger; <if, indeed, you feel any curiosity respecting one, now, alas, powerless to aid you in any way.> 

 <Say not so; you can console and support me by the strength of your own powerful mind. Pray let me know who you really are?> 

 The stranger smiled a melancholy smile. <Then listen,> said he. <I am the Abbé Faria, and have been imprisoned as you know in this Château d'If since the year 1811; previously to which I had been confined for three years in the fortress of Fenestrelle. In the year 1811 I was transferred to Piedmont in France. It was at this period I learned that the destiny which seemed subservient to every wish formed by Napoleon, had bestowed on him a son, named king of Rome even in his cradle. I was very far then from expecting the change you have just informed me of; namely, that four years afterwards, this colossus of power would be overthrown. Then who reigns in France at this moment—Napoleon II.?> 

 <No, Louis XVIII.> 

 <The brother of Louis XVI.! How inscrutable are the ways of Providence—for what great and mysterious purpose has it pleased Heaven to abase the man once so elevated, and raise up him who was so abased?> 

 Dantès' whole attention was riveted on a man who could thus forget his own misfortunes while occupying himself with the destinies of others. 

 <Yes, yes,> continued he, <'Twill be the same as it was in England. After Charles I., Cromwell; after Cromwell, Charles II., and then James II., and then some son-in-law or relation, some Prince of Orange, a stadtholder who becomes a king. Then new concessions to the people, then a constitution, then liberty. Ah, my friend!> said the abbé, turning towards Dantès, and surveying him with the kindling gaze of a prophet, <you are young, you will see all this come to pass.> 

 <Probably, if ever I get out of prison!> 

 <True,> replied Faria, <we are prisoners; but I forget this sometimes, and there are even moments when my mental vision transports me beyond these walls, and I fancy myself at liberty.> 

 <But wherefore are you here?> 

 <Because in 1807 I dreamed of the very plan Napoleon tried to realize in 1811; because, like Machiavelli, I desired to alter the political face of Italy, and instead of allowing it to be split up into a quantity of petty principalities, each held by some weak or tyrannical ruler, I sought to form one large, compact, and powerful empire; and, lastly, because I fancied I had found my Cæsar Borgia in a crowned simpleton, who feigned to enter into my views only to betray me. It was the plan of Alexander VI. and Clement VII., but it will never succeed now, for they attempted it fruitlessly, and Napoleon was unable to complete his work. Italy seems fated to misfortune.> And the old man bowed his head. 

 Dantès could not understand a man risking his life for such matters. Napoleon certainly he knew something of, inasmuch as he had seen and spoken with him; but of Clement VII. and Alexander VI. he knew nothing. 

 <Are you not,> he asked, <the priest who here in the Château d'If is generally thought to be—ill?> 

 <Mad, you mean, don't you?> 

 <I did not like to say so,> answered Dantès, smiling. 

 <Well, then,> resumed Faria with a bitter smile, <let me answer your question in full, by acknowledging that I am the poor mad prisoner of the Château d'If, for many years permitted to amuse the different visitors with what is said to be my insanity; and, in all probability, I should be promoted to the honour of making sport for the children, if such innocent beings could be found in an abode devoted like this to suffering and despair.> 

 Dantès remained for a short time mute and motionless; at length he said: 

 <Then you abandon all hope of escape?> 

 <I perceive its utter impossibility; and I consider it impious to attempt that which the Almighty evidently does not approve.> 

 <Nay, be not discouraged. Would it not be expecting too much to hope to succeed at your first attempt? Why not try to find an opening in another direction from that which has so unfortunately failed?> 

 <Alas, it shows how little notion you can have of all it has cost me to effect a purpose so unexpectedly frustrated, that you talk of beginning over again. In the first place, I was four years making the tools I possess, and have been two years scraping and digging out earth, hard as granite itself; then what toil and fatigue has it not been to remove huge stones I should once have deemed impossible to loosen. Whole days have I passed in these Titanic efforts, considering my labour well repaid if, by night-time I had contrived to carry away a square inch of this hard-bound cement, changed by ages into a substance unyielding as the stones themselves; then to conceal the mass of earth and rubbish I dug up, I was compelled to break through a staircase, and throw the fruits of my labour into the hollow part of it; but the well is now so completely choked up, that I scarcely think it would be possible to add another handful of dust without leading to discovery. Consider also that I fully believed I had accomplished the end and aim of my undertaking, for which I had so exactly husbanded my strength as to make it just hold out to the termination of my enterprise; and now, at the moment when I reckoned upon success, my hopes are forever dashed from me. No, I repeat again, that nothing shall induce me to renew attempts evidently at variance with the Almighty's pleasure.> 

 Dantès held down his head, that the other might not see how joy at the thought of having a companion outweighed the sympathy he felt for the failure of the abbé's plans. 

 The abbé sank upon Edmond's bed, while Edmond himself remained standing. Escape had never once occurred to him. There are, indeed, some things which appear so impossible that the mind does not dwell on them for an instant. To undermine the ground for fifty feet—to devote three years to a labour which, if successful, would conduct you to a precipice overhanging the sea—to plunge into the waves from the height of fifty, sixty, perhaps a hundred feet, at the risk of being dashed to pieces against the rocks, should you have been fortunate enough to have escaped the fire of the sentinels; and even, supposing all these perils past, then to have to swim for your life a distance of at least three miles ere you could reach the shore—were difficulties so startling and formidable that Dantès had never even dreamed of such a scheme, resigning himself rather to death. 

 But the sight of an old man clinging to life with so desperate a courage, gave a fresh turn to his ideas, and inspired him with new courage. Another, older and less strong than he, had attempted what he had not had sufficient resolution to undertake, and had failed only because of an error in calculation. This same person, with almost incredible patience and perseverance, had contrived to provide himself with tools requisite for so unparalleled an attempt. Another had done all this; why, then, was it impossible to Dantès? Faria had dug his way through fifty feet, Dantès would dig a hundred; Faria, at the age of fifty, had devoted three years to the task; he, who was but half as old, would sacrifice six; Faria, a priest and savant, had not shrunk from the idea of risking his life by trying to swim a distance of three miles to one of the islands—Daume, Rattonneau, or Lemaire; should a hardy sailor, an experienced diver, like himself, shrink from a similar task; should he, who had so often for mere amusement's sake plunged to the bottom of the sea to fetch up the bright coral branch, hesitate to entertain the same project? He could do it in an hour, and how many times had he, for pure pastime, continued in the water for more than twice as long! At once Dantès resolved to follow the brave example of his energetic companion, and to remember that what has once been done may be done again. 

 After continuing some time in profound meditation, the young man suddenly exclaimed, <I have found what you were in search of!> 

 Faria started: <Have you, indeed?> cried he, raising his head with quick anxiety; <pray, let me know what it is you have discovered?> 

 <The corridor through which you have bored your way from the cell you occupy here, extends in the same direction as the outer gallery, does it not?>  <It does.> 

 <And is not above fifteen feet from it?> 

 <About that.> 

 <Well, then, I will tell you what we must do. We must pierce through the corridor by forming a side opening about the middle, as it were the top part of a cross. This time you will lay your plans more accurately; we shall get out into the gallery you have described; kill the sentinel who guards it, and make our escape. All we require to insure success is courage, and that you possess, and strength, which I am not deficient in; as for patience, you have abundantly proved yours—you shall now see me prove mine.> 

 <One instant, my dear friend,> replied the abbé; <it is clear you do not understand the nature of the courage with which I am endowed, and what use I intend making of my strength. As for patience, I consider that I have abundantly exercised that in beginning every morning the task of the night before, and every night renewing the task of the day. But then, young man (and I pray of you to give me your full attention), then I thought I could not be doing anything displeasing to the Almighty in trying to set an innocent being at liberty—one who had committed no offence, and merited not condemnation.> 

 <And have your notions changed?> asked Dantès with much surprise; <do you think yourself more guilty in making the attempt since you have encountered me?> 

 <No; neither do I wish to incur guilt. Hitherto I have fancied myself merely waging war against circumstances, not men. I have thought it no sin to bore through a wall, or destroy a staircase; but I cannot so easily persuade myself to pierce a heart or take away a life.> 

 A slight movement of surprise escaped Dantès. 

 <Is it possible,> said he, <that where your liberty is at stake you can allow any such scruple to deter you from obtaining it?> 

 <Tell me,> replied Faria, <what has hindered you from knocking down your jailer with a piece of wood torn from your bedstead, dressing yourself in his clothes, and endeavouring to escape?> 

 <Simply the fact that the idea never occurred to me,> answered Dantès. 

 <Because,> said the old man, <the natural repugnance to the commission of such a crime prevented you from thinking of it; and so it ever is because in simple and allowable things our natural instincts keep us from deviating from the strict line of duty. The tiger, whose nature teaches him to delight in shedding blood, needs but the sense of smell to show him when his prey is within his reach, and by following this instinct he is enabled to measure the leap necessary to permit him to spring on his victim; but man, on the contrary, loathes the idea of blood—it is not alone that the laws of social life inspire him with a shrinking dread of taking life; his natural construction and physiological formation\longdash> 

 Dantès was confused and silent at this explanation of the thoughts which had unconsciously been working in his mind, or rather soul; for there are two distinct sorts of ideas, those that proceed from the head and those that emanate from the heart.  <Since my imprisonment,> said Faria, <I have thought over all the most celebrated cases of escape on record. They have rarely been successful. Those that have been crowned with full success have been long meditated upon, and carefully arranged; such, for instance, as the escape of the Duc de Beaufort from the Château de Vincennes, that of the Abbé Dubuquoi from For l'Evêque; of Latude from the Bastille. Then there are those for which chance sometimes affords opportunity, and those are the best of all. Let us, therefore, wait patiently for some favourable moment, and when it presents itself, profit by it.> 

 <Ah,> said Dantès, <you might well endure the tedious delay; you were constantly employed in the task you set yourself, and when weary with toil, you had your hopes to refresh and encourage you.> 

 <I assure you,> replied the old man, <I did not turn to that source for recreation or support.> 

 <What did you do then?> 

 <I wrote or studied.> 

 <Were you then permitted the use of pens, ink, and paper?> 

 <Oh, no,> answered the abbé; <I had none but what I made for myself.> 

 <You made paper, pens and ink?> 

 <Yes.> 

 Dantès gazed with admiration, but he had some difficulty in believing. Faria saw this. 

 <When you pay me a visit in my cell, my young friend,> said he, <I will show you an entire work, the fruits of the thoughts and reflections of my whole life; many of them meditated over in the shades of the Colosseum at Rome, at the foot of St. Mark's column at Venice, and on the borders of the Arno at Florence, little imagining at the time that they would be arranged in order within the walls of the Château d'If. The work I speak of is called \textit{A Treatise on the Possibility of a General Monarchy in Italy}, and will make one large quarto volume.> 

 <And on what have you written all this?> 

 <On two of my shirts. I invented a preparation that makes linen as smooth and as easy to write on as parchment.> 

 <You are, then, a chemist?> 

 <Somewhat; I know Lavoisier, and was the intimate friend of Cabanis.> 

 <But for such a work you must have needed books—had you any?> 

 <I had nearly five thousand volumes in my library at Rome; but after reading them over many times, I found out that with one hundred and fifty well-chosen books a man possesses, if not a complete summary of all human knowledge, at least all that a man need really know. I devoted three years of my life to reading and studying these one hundred and fifty volumes, till I knew them nearly by heart; so that since I have been in prison, a very slight effort of memory has enabled me to recall their contents as readily as though the pages were open before me. I could recite you the whole of Thucydides, Xenophon, Plutarch, Titus Livius, Tacitus, Strada, Jornandes, Dante, Montaigne, Shakespeare, Spinoza, Machiavelli, and Bossuet. I name only the most important.> 

 <You are, doubtless, acquainted with a variety of languages, so as to have been able to read all these?> 

 <Yes, I speak five of the modern tongues—that is to say, German, French, Italian, English, and Spanish; by the aid of ancient Greek I learned modern Greek—I don't speak it so well as I could wish, but I am still trying to improve myself.> 

 <Improve yourself!> repeated Dantès; <why, how can you manage to do so?> 

 <Why, I made a vocabulary of the words I knew; turned, returned, and arranged them, so as to enable me to express my thoughts through their medium. I know nearly one thousand words, which is all that is absolutely necessary, although I believe there are nearly one hundred thousand in the dictionaries. I cannot hope to be very fluent, but I certainly should have no difficulty in explaining my wants and wishes; and that would be quite as much as I should ever require.> 

 Stronger grew the wonder of Dantès, who almost fancied he had to do with one gifted with supernatural powers; still hoping to find some imperfection which might bring him down to a level with human beings, he added, <Then if you were not furnished with pens, how did you manage to write the work you speak of?> 

 <I made myself some excellent ones, which would be universally preferred to all others if once known. You are aware what huge whitings are served to us on \textit{maigre} days. Well, I selected the cartilages of the heads of these fishes, and you can scarcely imagine the delight with which I welcomed the arrival of each Wednesday, Friday, and Saturday, as affording me the means of increasing my stock of pens; for I will freely confess that my historical labors have been my greatest solace and relief. While retracing the past, I forget the present; and traversing at will the path of history I cease to remember that I am myself a prisoner.> 

 <But the ink,> said Dantès; <of what did you make your ink?> 

 <There was formerly a fireplace in my dungeon,> replied Faria, <but it was closed up long ere I became an occupant of this prison. Still, it must have been many years in use, for it was thickly covered with a coating of soot; this soot I dissolved in a portion of the wine brought to me every Sunday, and I assure you a better ink cannot be desired. For very important notes, for which closer attention is required, I pricked one of my fingers, and wrote with my own blood.> 

 <And when,> asked Dantès, <may I see all this?> 

 <Whenever you please,> replied the abbé. 

 <Oh, then let it be directly!> exclaimed the young man. 

 <Follow me, then,> said the abbé, as he re-entered the subterranean passage, in which he soon disappeared, followed by Dantès. 