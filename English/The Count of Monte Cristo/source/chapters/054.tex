\chapter{A Flurry in Stocks} 

 \lettrine{S}{ome} days after this meeting, Albert de Morcerf visited the Count of Monte Cristo at his house in the Champs-Élysées, which had already assumed that palace-like appearance which the count's princely fortune enabled him to give even to his most temporary residences. He came to renew the thanks of Madame Danglars which had been already conveyed to the count through the medium of a letter, signed <Baronne Danglars, \textit{née} Hermine de Servieux.> 

 Albert was accompanied by Lucien Debray, who, joining in his friend's conversation, added some passing compliments, the source of which the count's talent for finesse easily enabled him to guess. He was convinced that Lucien's visit was due to a double feeling of curiosity, the larger half of which sentiment emanated from the Rue de la Chaussée d'Antin. In short, Madame Danglars, not being able personally to examine in detail the domestic economy and household arrangements of a man who gave away horses worth 30,000 francs and who went to the opera with a Greek slave wearing diamonds to the amount of a million of money, had deputed those eyes, by which she was accustomed to see, to give her a faithful account of the mode of life of this incomprehensible person. But the count did not appear to suspect that there could be the slightest connection between Lucien's visit and the curiosity of the baroness. 

 <You are in constant communication with the Baron Danglars?> the count inquired of Albert de Morcerf. 

 <Yes, count, you know what I told you?> 

 <All remains the same, then, in that quarter?> 

 <It is more than ever a settled thing,> said Lucien,—and, considering that this remark was all that he was at that time called upon to make, he adjusted the glass to his eye, and biting the top of his gold headed cane, began to make the tour of the apartment, examining the arms and the pictures. 

 <Ah,> said Monte Cristo <I did not expect that the affair would be so promptly concluded.> 

 <Oh, things take their course without our assistance. While we are forgetting them, they are falling into their appointed order; and when, again, our attention is directed to them, we are surprised at the progress they have made towards the proposed end. My father and M. Danglars served together in Spain, my father in the army and M. Danglars in the commissariat department. It was there that my father, ruined by the revolution, and M. Danglars, who never had possessed any patrimony, both laid the foundations of their different fortunes.> 

 <Yes,> said Monte Cristo <I think M. Danglars mentioned that in a visit which I paid him; and,> continued he, casting a side-glance at Lucien, who was turning over the leaves of an album, <Mademoiselle Eugénie is pretty—I think I remember that to be her name.> 

 <Very pretty, or rather, very beautiful,> replied Albert, <but of that style of beauty which I do not appreciate; I am an ungrateful fellow.> 

 <You speak as if you were already her husband.> 

 <Ah,> returned Albert, in his turn looking around to see what Lucien was doing. 

 <Really,> said Monte Cristo, lowering his voice, <you do not appear to me to be very enthusiastic on the subject of this marriage.> 

 <Mademoiselle Danglars is too rich for me,> replied Morcerf, <and that frightens me.> 

 <Bah,> exclaimed Monte Cristo, <that's a fine reason to give. Are you not rich yourself?> 

 <My father's income is about 50,000 francs per annum; and he will give me, perhaps, ten or twelve thousand when I marry.> 

 <That, perhaps, might not be considered a large sum, in Paris especially,> said the count; <but everything does not depend on wealth, and it is a fine thing to have a good name, and to occupy a high station in society. Your name is celebrated, your position magnificent; and then the Comte de Morcerf is a soldier, and it is pleasing to see the integrity of a Bayard united to the poverty of a Duguesclin; disinterestedness is the brightest ray in which a noble sword can shine. As for me, I consider the union with Mademoiselle Danglars a most suitable one; she will enrich you, and you will ennoble her.> 

 Albert shook his head, and looked thoughtful. 

 <There is still something else,> said he. 

 <I confess,> observed Monte Cristo, <that I have some difficulty in comprehending your objection to a young lady who is both rich and beautiful.> 

 <Oh,> said Morcerf, <this repugnance, if repugnance it may be called, is not all on my side.> 

 <Whence can it arise, then? for you told me your father desired the marriage.> 

 <It is my mother who dissents; she has a clear and penetrating judgment, and does not smile on the proposed union. I cannot account for it, but she seems to entertain some prejudice against the Danglars.>  
 
 <Ah,> said the count, in a somewhat forced tone, <that may be easily explained; the Comtesse de Morcerf, who is aristocracy and refinement itself, does not relish the idea of being allied by your marriage with one of ignoble birth; that is natural enough.> 

 <I do not know if that is her reason,> said Albert, <but one thing I do know, that if this marriage be consummated, it will render her quite miserable. There was to have been a meeting six weeks ago in order to talk over and settle the affair; but I had such a sudden attack of indisposition\longdash> 

 <Real?> interrupted the count, smiling. 

 <Oh, real enough, from anxiety doubtless,—at any rate they postponed the matter for two months. There is no hurry, you know. I am not yet twenty-one, and Eugénie is only seventeen; but the two months expire next week. It must be done. My dear count, you cannot imagine how my mind is harassed. How happy you are in being exempt from all this!> 

 <Well, and why should not you be free, too? What prevents you from being so?> 

 <Oh, it will be too great a disappointment to my father if I do not marry Mademoiselle Danglars.> 

 <Marry her then,> said the count, with a significant shrug of the shoulders. 

 <Yes,> replied Morcerf, <but that will plunge my mother into positive grief.> 

 <Then do not marry her,> said the count. 

 <Well, I shall see. I will try and think over what is the best thing to be done; you will give me your advice, will you not, and if possible extricate me from my unpleasant position? I think, rather than give pain to my dear mother, I would run the risk of offending the count.> 

 Monte Cristo turned away; he seemed moved by this last remark. 

 <Ah,> said he to Debray, who had thrown himself into an easy-chair at the farthest extremity of the salon, and who held a pencil in his right hand and an account book in his left, <what are you doing there? Are you making a sketch after Poussin?> 

 <Oh, no,> was the tranquil response; <I am too fond of art to attempt anything of that sort. I am doing a little sum in arithmetic.> 

 <In arithmetic?> 

 <Yes; I am calculating—by the way, Morcerf, that indirectly concerns you—I am calculating what the house of Danglars must have gained by the last rise in Haiti bonds; from 206 they have risen to 409 in three days, and the prudent banker had purchased at 206; therefore he must have made 300,000 livres.> 

 <That is not his biggest scoop,> said Morcerf; <did he not make a million in Spaniards this last year?> 

 <My dear fellow,> said Lucien, 'here is the Count of Monte Cristo, who will say to you, as the Italians do,—  
 
 \begin{verse}
Denaro e santità,\\
Metà della metà.\footnote{<Money and sanctity, Each in a moiety.>}'
\end{verse}
%manual quotes due to straddling an environment boundary

 <When they tell me such things, I only shrug my shoulders and say nothing.> 

 <But you were speaking of Haitians?> said Monte Cristo. 

 <Ah, Haitians,—that is quite another thing! Haitians are the \textit{écarté} of French stock-jobbing. We may like bouillotte, delight in whist, be enraptured with boston, and yet grow tired of them all; but we always come back to \textit{écarté}—it is not only a game, it is a \textit{hors-d'œuvre}! M. Danglars sold yesterday at 405, and pockets 300,000 francs. Had he but waited till today, the price would have fallen to 205, and instead of gaining 300,000 francs, he would have lost 20 or 25,000.> 

 <And what has caused the sudden fall from 409 to 206?> asked Monte Cristo. <I am profoundly ignorant of all these stock-jobbing intrigues.> 

 <Because,> said Albert, laughing, <one piece of news follows another, and there is often great dissimilarity between them.> 

 <Ah,> said the count, <I see that M. Danglars is accustomed to play at gaining or losing 300,000 francs in a day; he must be enormously rich.> 

 <It is not he who plays!> exclaimed Lucien; <it is Madame Danglars; she is indeed daring.> 

 <But you who are a reasonable being, Lucien, and who knows how little dependence is to be placed on the news, since you are at the fountain-head, surely you ought to prevent it,> said Morcerf, with a smile. 

 <How can I, if her husband fails in controlling her?> asked Lucien; <you know the character of the baroness—no one has any influence with her, and she does precisely what she pleases.> 

 <Ah, if I were in your place\longdash> said Albert. 

 <Well?> 

 <I would reform her; it would be rendering a service to her future son-in-law.> 

 <How would you set about it?> 

 <Ah, that would be easy enough—I would give her a lesson.> 

 <A lesson?> 

 <Yes. Your position as secretary to the minister renders your authority great on the subject of political news; you never open your mouth but the stockbrokers immediately stenograph your words. Cause her to lose a hundred thousand francs, and that would teach her prudence.> 

 <I do not understand,> stammered Lucien. 

 <It is very clear, notwithstanding,> replied the young man, with an artlessness wholly free from affectation; <tell her some fine morning an unheard-of piece of intelligence—some telegraphic despatch, of which you alone are in possession; for instance, that Henri \textsc{iv.} was seen yesterday at Gabrielle's. That would boom the market; she will buy heavily, and she will certainly lose when Beauchamp announces the following day, in his gazette, <The report circulated by some usually well-informed persons that the king was seen yesterday at Gabrielle's house, is totally without foundation. We can positively assert that his majesty did not quit the Pont-Neuf.>> 

 Lucien half smiled. Monte Cristo, although apparently indifferent, had not lost one word of this conversation, and his penetrating eye had even read a hidden secret in the embarrassed manner of the secretary. This embarrassment had completely escaped Albert, but it caused Lucien to shorten his visit; he was evidently ill at ease. The count, in taking leave of him, said something in a low voice, to which he answered, <Willingly, count; I accept.> The count returned to young Morcerf. 

 <Do you not think, on reflection,> said he to him, <that you have done wrong in thus speaking of your mother-in-law in the presence of M. Debray?> 

 <My dear count,> said Morcerf, <I beg of you not to apply that title so prematurely.> 

 <Now, speaking without any exaggeration, is your mother really so very much averse to this marriage?> 

 <So much so that the baroness very rarely comes to the house, and my mother, has not, I think, visited Madame Danglars twice in her whole life.> 

 <Then,> said the count, <I am emboldened to speak openly to you. M. Danglars is my banker; M. de Villefort has overwhelmed me with politeness in return for a service which a casual piece of good fortune enabled me to render him. I predict from all this an avalanche of dinners and routs. Now, in order not to presume on this, and also to be beforehand with them, I have, if agreeable to you, thought of inviting M. and Madame Danglars, and M. and Madame de Villefort, to my country-house at Auteuil. If I were to invite you and the Count and Countess of Morcerf to this dinner, I should give it the appearance of being a matrimonial meeting, or at least Madame de Morcerf would look upon the affair in that light, especially if Baron Danglars did me the honour to bring his daughter. In that case your mother would hold me in aversion, and I do not at all wish that; on the contrary, I desire to stand high in her esteem.> 

 <Indeed, count,> said Morcerf, <I thank you sincerely for having used so much candour towards me, and I gratefully accept the exclusion which you propose. You say you desire my mother's good opinion; I assure you it is already yours to a very unusual extent.>

<Do you think so?> said Monte Cristo, with interest. 

 <Oh, I am sure of it; we talked of you an hour after you left us the other day. But to return to what we were saying. If my mother could know of this attention on your part—and I will venture to tell her—I am sure that she will be most grateful to you; it is true that my father will be equally angry.> The count laughed. 

 <Well,> said he to Morcerf, <but I think your father will not be the only angry one; M. and Madame Danglars will think me a very ill-mannered person. They know that I am intimate with you—that you are, in fact; one of the oldest of my Parisian acquaintances—and they will not find you at my house; they will certainly ask me why I did not invite you. Be sure to provide yourself with some previous engagement which shall have a semblance of probability, and communicate the fact to me by a line in writing. You know that with bankers nothing but a written document will be valid.> 

 <I will do better than that,> said Albert; <my mother is wishing to go to the sea-side—what day is fixed for your dinner?> 

 <Saturday.> 

 <This is Tuesday—well, tomorrow evening we leave, and the day after we shall be at Tréport. Really, count, you have a delightful way of setting people at their ease.> 

 <Indeed, you give me more credit than I deserve; I only wish to do what will be agreeable to you, that is all.> 

 <When shall you send your invitations?> 

 <This very day.> 

 <Well, I will immediately call on M. Danglars, and tell him that my mother and myself must leave Paris tomorrow. I have not seen you, consequently I know nothing of your dinner.> 

 <How foolish you are! Have you forgotten that M. Debray has just seen you at my house?> 

 <Ah, true.> 

 <Fix it this way. I have seen you, and invited you without any ceremony, when you instantly answered that it would be impossible for you to accept, as you were going to Tréport.> 

 <Well, then, that is settled; but you will come and call on my mother before tomorrow?> 

 <Before tomorrow?—that will be a difficult matter to arrange, besides, I shall just be in the way of all the preparations for departure.> 

 <Well, you can do better. You were only a charming man before, but, if you accede to my proposal, you will be adorable.> 

 <What must I do to attain such sublimity?> 

 <You are today free as air—come and dine with me; we shall be a small party—only yourself, my mother, and \textsc{i.} You have scarcely seen my mother; you shall have an opportunity of observing her more closely. She is a remarkable woman, and I only regret that there does not exist another like her, about twenty years younger; in that case, I assure you, there would very soon be a Countess and Viscountess of Morcerf. As to my father, you will not see him; he is officially engaged, and dines with the chief referendary. We will talk over our travels; and you, who have seen the whole world, will relate your adventures—you shall tell us the history of the beautiful Greek who was with you the other night at the Opera, and whom you call your slave, and yet treat like a princess. We will talk Italian and Spanish. Come, accept my invitation, and my mother will thank you.> 

 <A thousand thanks,> said the count, <your invitation is most gracious, and I regret exceedingly that it is not in my power to accept it. I am not so much at liberty as you suppose; on the contrary, I have a most important engagement.> 

 <Ah, take care, you were teaching me just now how, in case of an invitation to dinner, one might creditably make an excuse. I require the proof of a pre-engagement. I am not a banker, like M. Danglars, but I am quite as incredulous as he is.> 

 <I am going to give you a proof,> replied the count, and he rang the bell. 

 <Humph,> said Morcerf, <this is the second time you have refused to dine with my mother; it is evident that you wish to avoid her.> 

 Monte Cristo started. <Oh, you do not mean that,> said he; <besides, here comes the confirmation of my assertion.> 

 Baptistin entered, and remained standing at the door. 

 <I had no previous knowledge of your visit, had I?> 

 <Indeed, you are such an extraordinary person, that I would not answer for it.> 

 <At all events, I could not guess that you would invite me to dinner.> 

 <Probably not.> 

 <Well, listen, Baptistin, what did I tell you this morning when I called you into my laboratory?> 

 <To close the door against visitors as soon as the clock struck five,> replied the valet. 

 <What then?> 

 <Ah, my dear count,> said Albert. 

 <No, no, I wish to do away with that mysterious reputation that you have given me, my dear viscount; it is tiresome to be always acting Manfred. I wish my life to be free and open. Go on, Baptistin.> 

 <Then to admit no one except Major Bartolomeo Cavalcanti and his son.> 

 <You hear—Major Bartolomeo Cavalcanti—a man who ranks amongst the most ancient nobility of Italy, whose name Dante has celebrated in the tenth canto of \textit{The Inferno}, you remember it, do you not? Then there is his son, Andrea, a charming young man, about your own age, viscount, bearing the same title as yourself, and who is making his entry into the Parisian world, aided by his father's millions. The major will bring his son with him this evening, the \textit{contino}, as we say in Italy; he confides him to my care. If he proves himself worthy of it, I will do what I can to advance his interests. You will assist me in the work, will you not?> 

 <Most undoubtedly. This Major Cavalcanti is an old friend of yours, then?> 

 <By no means. He is a perfect nobleman, very polite, modest, and agreeable, such as may be found constantly in Italy, descendants of very ancient families. I have met him several times at Florence, Bologna and Lucca, and he has now communicated to me the fact of his arrival in Paris. The acquaintances one makes in travelling have a sort of claim on one; they everywhere expect to receive the same attention which you once paid them by chance, as though the civilities of a passing hour were likely to awaken any lasting interest in favour of the man in whose society you may happen to be thrown in the course of your journey. This good Major Cavalcanti is come to take a second view of Paris, which he only saw in passing through in the time of the Empire, when he was on his way to Moscow. I shall give him a good dinner, he will confide his son to my care, I will promise to watch over him, I shall let him follow in whatever path his folly may lead him, and then I shall have done my part.> 

 <Certainly; I see you are a model Mentor,> said Albert <Good-bye, we shall return on Sunday. By the way, I have received news of Franz.> 

 <Have you? Is he still amusing himself in Italy?> 

 <I believe so; however, he regrets your absence extremely. He says you were the sun of Rome, and that without you all appears dark and cloudy; I do not know if he does not even go so far as to say that it rains.> 

 <His opinion of me is altered for the better, then?> 

 <No, he still persists in looking upon you as the most incomprehensible and mysterious of beings.> 

 <He is a charming young man,> said Monte Cristo <and I felt a lively interest in him the very first evening of my introduction, when I met him in search of a supper, and prevailed upon him to accept a portion of mine. He is, I think, the son of General d'Épinay?> 

 <He is.> 

 <The same who was so shamefully assassinated in 1815?> 

 <By the Bonapartists.> 

 <Yes. Really I like him extremely; is there not also a matrimonial engagement contemplated for him?> 

 <Yes, he is to marry Mademoiselle de Villefort.> 

 <Indeed?>

<And you know I am to marry Mademoiselle Danglars,> said Albert, laughing. 

 <You smile.> 

 <Yes.> 

 <Why do you do so?> 

 <I smile because there appears to me to be about as much inclination for the consummation of the engagement in question as there is for my own. But really, my dear count, we are talking as much of women as they do of us; it is unpardonable.> 

 Albert rose. 

 <Are you going?> 

 <Really, that is a good idea!—two hours have I been boring you to death with my company, and then you, with the greatest politeness, ask me if I am going. Indeed, count, you are the most polished man in the world. And your servants, too, how very well behaved they are; there is quite a style about them. Monsieur Baptistin especially; I could never get such a man as that. My servants seem to imitate those you sometimes see in a play, who, because they have only a word or two to say, acquit themselves in the most awkward manner possible. Therefore, if you part with M. Baptistin, give me the refusal of him.> 

 <By all means.> 

 <That is not all; give my compliments to your illustrious Luccanese, Cavalcante of the Cavalcanti; and if by any chance he should be wishing to establish his son, find him a wife very rich, very noble on her mother's side at least, and a baroness in right of her father, I will help you in the search.> 

 <Ah, ha; you will do as much as that, will you?> 

 <Yes.> 

 <Well, really, nothing is certain in this world.> 

 <Oh, count, what a service you might render me! I should like you a hundred times better if, by your intervention, I could manage to remain a bachelor, even were it only for ten years.> 

 <Nothing is impossible,> gravely replied Monte Cristo; and taking leave of Albert, he returned into the house, and struck the gong three times. Bertuccio appeared. 

 <Monsieur Bertuccio, you understand that I intend entertaining company on Saturday at Auteuil.> Bertuccio slightly started. <I shall require your services to see that all be properly arranged. It is a beautiful house, or at all events may be made so.> 

 <There must be a good deal done before it can deserve that title, your excellency, for the tapestried hangings are very old.> 

 <Let them all be taken away and changed, then, with the exception of the sleeping-chamber which is hung with red damask; you will leave that exactly as it is.> Bertuccio bowed. <You will not touch the garden either; as to the yard, you may do what you please with it; I should prefer that being altered beyond all recognition.> 

 <I will do everything in my power to carry out your wishes, your excellency. I should be glad, however, to receive your excellency's commands concerning the dinner.> 

 <Really, my dear M. Bertuccio,> said the count, <since you have been in Paris, you have become quite nervous, and apparently out of your element; you no longer seem to understand me.> 

 <But surely your excellency will be so good as to inform me whom you are expecting to receive?> 

 <I do not yet know myself, neither is it necessary that you should do so. <Lucullus dines with Lucullus,> that is quite sufficient.> 

 Bertuccio bowed, and left the room. 