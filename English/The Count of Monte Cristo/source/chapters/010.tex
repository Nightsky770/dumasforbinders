\chapter{The King's Closet at the Tuileries} 

 \lettrine{W}{e} will leave Villefort on the road to Paris, travelling—thanks to trebled fees—with all speed, and passing through two or three apartments, enter at the Tuileries the little room with the arched window, so well known as having been the favourite closet of Napoleon and Louis XVIII., and now of Louis Philippe. 

 There, seated before a walnut table he had brought with him from Hartwell, and to which, from one of those fancies not uncommon to great people, he was particularly attached, the king, Louis XVIII., was carelessly listening to a man of fifty or fifty-two years of age, with gray hair, aristocratic bearing, and exceedingly gentlemanly attire, and meanwhile making a marginal note in a volume of Gryphius's rather inaccurate, but much sought-after, edition of Horace—a work which was much indebted to the sagacious observations of the philosophical monarch. 

 <You say, sir\longdash> said the king. 

 <That I am exceedingly disquieted, sire.> 

 <Really, have you had a vision of the seven fat kine and the seven lean kine?> 

 <No, sire, for that would only betoken for us seven years of plenty and seven years of scarcity; and with a king as full of foresight as your majesty, scarcity is not a thing to be feared.> 

 <Then of what other scourge are you afraid, my dear Blacas?> 

 <Sire, I have every reason to believe that a storm is brewing in the south.> 

 <Well, my dear duke,> replied Louis XVIII., <I think you are wrongly informed, and know positively that, on the contrary, it is very fine weather in that direction.> Man of ability as he was, Louis XVIII. liked a pleasant jest. 

 <Sire,> continued M. de Blacas, <if it only be to reassure a faithful servant, will your majesty send into Languedoc, Provence, and Dauphiné, trusty men, who will bring you back a faithful report as to the feeling in these three provinces?> 

 <\textit{Canimus surdis},> replied the king, continuing the annotations in his Horace. 

 <Sire,> replied the courtier, laughing, in order that he might seem to comprehend the quotation, <your majesty may be perfectly right in relying on the good feeling of France, but I fear I am not altogether wrong in dreading some desperate attempt.> 

 <By whom?> 

 <By Bonaparte, or, at least, by his adherents.> 

 <My dear Blacas,> said the king, <you with your alarms prevent me from working.> 

 <And you, sire, prevent me from sleeping with your security.> 

 <Wait, my dear sir, wait a moment; for I have such a delightful note on the \textit{Pastor quum traheret}—wait, and I will listen to you afterwards.> 

 There was a brief pause, during which Louis XVIII. wrote, in a hand as small as possible, another note on the margin of his Horace, and then looking at the duke with the air of a man who thinks he has an idea of his own, while he is only commenting upon the idea of another, said: 

 <Go on, my dear duke, go on—I listen.> 

 <Sire,> said Blacas, who had for a moment the hope of sacrificing Villefort to his own profit, <I am compelled to tell you that these are not mere rumours destitute of foundation which thus disquiet me; but a serious-minded man, deserving all my confidence, and charged by me to watch over the south> (the duke hesitated as he pronounced these words), <has arrived by post to tell me that a great peril threatens the king, and so I hastened to you, sire.> 

 <\textit{Mala ducis avi domum},> continued Louis XVIII., still annotating. 

 <Does your majesty wish me to drop the subject?> 

 <By no means, my dear duke; but just stretch out your hand.> 

 <Which?> 

 <Whichever you please—there to the left.> 

 <Here, sire?> 

 <I tell you to the left, and you are looking to the right; I mean on my left—yes, there. You will find yesterday's report of the minister of police. But here is M. Dandré himself;> and M. Dandré, announced by the chamberlain-in-waiting, entered. 

 <Come in,> said Louis XVIII., with repressed smile, <come in, Baron, and tell the duke all you know—the latest news of M. de Bonaparte; do not conceal anything, however serious,—let us see, the Island of Elba is a volcano, and we may expect to have issuing thence flaming and bristling war—\textit{bella, horrida bella}.> 

 M. Dandré leaned very respectfully on the back of a chair with his two hands, and said: 

 <Has your majesty perused yesterday's report?> 

 <Yes, yes; but tell the duke himself, who cannot find anything, what the report contains—give him the particulars of what the usurper is doing in his islet.> 

 <Monsieur,> said the baron to the duke, <all the servants of his majesty must approve of the latest intelligence which we have from the Island of Elba. Bonaparte\longdash> 

 M. Dandré looked at Louis XVIII., who, employed in writing a note, did not even raise his head. <Bonaparte,> continued the baron, <is mortally wearied, and passes whole days in watching his miners at work at Porto-Longone.> 

 <And scratches himself for amusement,> added the king. 

 <Scratches himself?> inquired the duke, <what does your majesty mean?> 

 <Yes, indeed, my dear duke. Did you forget that this great man, this hero, this demigod, is attacked with a malady of the skin which worries him to death, \textit{prurigo}?> 

 <And, moreover, my dear duke,> continued the minister of police, <we are almost assured that, in a very short time, the usurper will be insane.> 

 <Insane?> 

 <Raving mad; his head becomes weaker. Sometimes he weeps bitterly, sometimes laughs boisterously, at other time he passes hours on the seashore, flinging stones in the water and when the flint makes <duck-and-drake> five or six times, he appears as delighted as if he had gained another Marengo or Austerlitz. Now, you must agree that these are indubitable symptoms of insanity.> 

 <Or of wisdom, my dear baron—or of wisdom,> said Louis XVIII., laughing; <the greatest captains of antiquity amused themselves by casting pebbles into the ocean—see Plutarch's life of Scipio Africanus.> 

 M. de Blacas pondered deeply between the confident monarch and the truthful minister. Villefort, who did not choose to reveal the whole secret, lest another should reap all the benefit of the disclosure, had yet communicated enough to cause him the greatest uneasiness. 

 <Well, well, Dandré,> said Louis XVIII., <Blacas is not yet convinced; let us proceed, therefore, to the usurper's conversion.> The minister of police bowed. 

 <The usurper's conversion!> murmured the duke, looking at the king and Dandré, who spoke alternately, like Virgil's shepherds. <The usurper converted!> 

 <Decidedly, my dear duke.> 

 <In what way converted?> 

 <To good principles. Tell him all about it, baron.> 

 <Why, this is the way of it,> said the minister, with the gravest air in the world: <Napoleon lately had a review, and as two or three of his old veterans expressed a desire to return to France, he gave them their dismissal, and exhorted them to <serve the good king.> These were his own words, of that I am certain.> 

 <Well, Blacas, what think you of this?> inquired the king triumphantly, and pausing for a moment from the voluminous scholiast before him. 

 <I say, sire, that the minister of police is greatly deceived or I am; and as it is impossible it can be the minister of police as he has the guardianship of the safety and honour of your majesty, it is probable that I am in error. However, sire, if I might advise, your majesty will interrogate the person of whom I spoke to you, and I will urge your majesty to do him this honour.> 

 <Most willingly, duke; under your auspices I will receive any person you please, but you must not expect me to be too confiding. Baron, have you any report more recent than this, dated the 20th February, and this is the 3rd of March?> 

 <No, sire, but I am hourly expecting one; it may have arrived since I left my office.> 

 <Go thither, and if there be none—well, well,> continued Louis XVIII., <make one; that is the usual way, is it not?> and the king laughed facetiously. 

 <Oh, sire,> replied the minister, <we have no occasion to invent any; every day our desks are loaded with most circumstantial denunciations, coming from hosts of people who hope for some return for services which they seek to render, but cannot; they trust to fortune, and rely upon some unexpected event in some way to justify their predictions.> 

 <Well, sir, go,> said Louis XVIII., <and remember that I am waiting for you.> 

 <I will but go and return, sire; I shall be back in ten minutes.> 

 <And I, sire,> said M. de Blacas, <will go and find my messenger.> 

 <Wait, sir, wait,> said Louis XVIII. <Really, M. de Blacas, I must change your armorial bearings; I will give you an eagle with outstretched wings, holding in its claws a prey which tries in vain to escape, and bearing this device—\textit{Tenax}.>  
 
 <Sire, I listen,> said De Blacas, biting his nails with impatience. 

 <I wish to consult you on this passage, <\textit{Molli fugiens anhelitu},> you know it refers to a stag flying from a wolf. Are you not a sportsman and a great wolf-hunter? Well, then, what do you think of the \textit{molli anhelitu}?> 

 <Admirable, sire; but my messenger is like the stag you refer to, for he has posted two hundred and twenty leagues in scarcely three days.> 

 <Which is undergoing great fatigue and anxiety, my dear duke, when we have a telegraph which transmits messages in three or four hours, and that without getting in the least out of breath.> 

 <Ah, sire, you recompense but badly this poor young man, who has come so far, and with so much ardour, to give your majesty useful information. If only for the sake of M. de Salvieux, who recommends him to me, I entreat your majesty to receive him graciously.> 

 <M. de Salvieux, my brother's chamberlain?> 

 <Yes, sire.> 

 <He is at Marseilles.> 

 <And writes me thence.> 

 <Does he speak to you of this conspiracy?> 

 <No; but strongly recommends M. de Villefort, and begs me to present him to your majesty.> 

 <M. de Villefort!> cried the king, <is the messenger's name M. de Villefort?> 

 <Yes, sire.> 

 <And he comes from Marseilles?> 

 <In person.> 

 <Why did you not mention his name at once?> replied the king, betraying some uneasiness. 

 <Sire, I thought his name was unknown to your majesty.> 

 <No, no, Blacas; he is a man of strong and elevated understanding, ambitious, too, and, \textit{pardieu!} you know his father's name!> 

 <His father?> 

 <Yes, Noirtier.> 

 <Noirtier the Girondin?—Noirtier the senator?> 

 <He himself.> 

 <And your majesty has employed the son of such a man?> 

 <Blacas, my friend, you have but limited comprehension. I told you Villefort was ambitious, and to attain this ambition Villefort would sacrifice everything, even his father.> 

 <Then, sire, may I present him?> 

 <This instant, duke! Where is he?> 

 <Waiting below, in my carriage.> 

 <Seek him at once.> 

 <I hasten to do so.> 

 The duke left the royal presence with the speed of a young man; his really sincere royalism made him youthful again. Louis XVIII. remained alone, and turning his eyes on his half-opened Horace, muttered: 

 <\textit{Justum et tenacem propositi virum}.> 

 M. de Blacas returned as speedily as he had departed, but in the antechamber he was forced to appeal to the king's authority. Villefort's dusty garb, his costume, which was not of courtly cut, excited the susceptibility of M. de Brezé, who was all astonishment at finding that this young man had the audacity to enter before the king in such attire. The duke, however, overcame all difficulties with a word—his majesty's order; and, in spite of the protestations which the master of ceremonies made for the honour of his office and principles, Villefort was introduced. 

 The king was seated in the same place where the duke had left him. On opening the door, Villefort found himself facing him, and the young magistrate's first impulse was to pause. 

 <Come in, M. de Villefort,> said the king, <come in.> 

 Villefort bowed, and advancing a few steps, waited until the king should interrogate him. 

 <M. de Villefort,> said Louis XVIII., <the Duc de Blacas assures me you have some interesting information to communicate.> 

 <Sire, the duke is right, and I believe your majesty will think it equally important.>  <In the first place, and before everything else, sir, is the news as bad in your opinion as I am asked to believe?> 

 <Sire, I believe it to be most urgent, but I hope, by the speed I have used, that it is not irreparable.> 

 <Speak as fully as you please, sir,> said the king, who began to give way to the emotion which had showed itself in Blacas's face and affected Villefort's voice. <Speak, sir, and pray begin at the beginning; I like order in everything.> 

 <Sire,> said Villefort, <I will render a faithful report to your majesty, but I must entreat your forgiveness if my anxiety leads to some obscurity in my language.> A glance at the king after this discreet and subtle exordium, assured Villefort of the benignity of his august auditor, and he went on: 

 <Sire, I have come as rapidly to Paris as possible, to inform your majesty that I have discovered, in the exercise of my duties, not a commonplace and insignificant plot, such as is every day got up in the lower ranks of the people and in the army, but an actual conspiracy—a storm which menaces no less than your majesty's throne. Sire, the usurper is arming three ships, he meditates some project, which, however mad, is yet, perhaps, terrible. At this moment he will have left Elba, to go whither I know not, but assuredly to attempt a landing either at Naples, or on the coast of Tuscany, or perhaps on the shores of France. Your majesty is well aware that the sovereign of the Island of Elba has maintained his relations with Italy and France?> 

 <I am, sir,> said the king, much agitated; <and recently we have had information that the Bonapartist clubs have had meetings in the Rue Saint-Jacques. But proceed, I beg of you. How did you obtain these details?> 

 <Sire, they are the results of an examination which I have made of a man of Marseilles, whom I have watched for some time, and arrested on the day of my departure. This person, a sailor, of turbulent character, and whom I suspected of Bonapartism, has been secretly to the Island of Elba. There he saw the grand-marshal, who charged him with an oral message to a Bonapartist in Paris, whose name I could not extract from him; but this mission was to prepare men's minds for a return (it is the man who says this, sire)—a return which will soon occur.> 

 <And where is this man?> 

 <In prison, sire.> 

 <And the matter seems serious to you?> 

 <So serious, sire, that when the circumstance surprised me in the midst of a family festival, on the very day of my betrothal, I left my bride and friends, postponing everything, that I might hasten to lay at your majesty's feet the fears which impressed me, and the assurance of my devotion.> 

 <True,> said Louis XVIII., <was there not a marriage engagement between you and Mademoiselle de Saint-Méran?> 

 <Daughter of one of your majesty's most faithful servants.> 

 <Yes, yes; but let us talk of this plot, M. de Villefort.> 

 <Sire, I fear it is more than a plot; I fear it is a conspiracy.> 

 <A conspiracy in these times,> said Louis XVIII., smiling, <is a thing very easy to meditate, but more difficult to conduct to an end, inasmuch as, re-established so recently on the throne of our ancestors, we have our eyes open at once upon the past, the present, and the future. For the last ten months my ministers have redoubled their vigilance, in order to watch the shore of the Mediterranean. If Bonaparte landed at Naples, the whole coalition would be on foot before he could even reach Piombino; if he land in Tuscany, he will be in an unfriendly territory; if he land in France, it must be with a handful of men, and the result of that is easily foretold, execrated as he is by the population. Take courage, sir; but at the same time rely on our royal gratitude.> 

 <Ah, here is M. Dandré!> cried de Blacas. At this instant the minister of police appeared at the door, pale, trembling, and as if ready to faint. Villefort was about to retire, but M. de Blacas, taking his hand, restrained him. 