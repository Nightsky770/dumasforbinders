\chapter{A Signed Statement} 

 \lettrine{N}{oirtier} was prepared to receive them, dressed in black, and installed in his armchair. When the three persons he expected had entered, he looked at the door, which his valet immediately closed. 

\zz
 <Listen,> whispered Villefort to Valentine, who could not conceal her joy; <if M. Noirtier wishes to communicate anything which would delay your marriage, I forbid you to understand him.> 

 Valentine blushed, but did not answer. Villefort, approached Noirtier. 

 <Here is M. Franz d'Épinay,> said he; <you requested to see him. We have all wished for this interview, and I trust it will convince you how ill-formed are your objections to Valentine's marriage.> 

 Noirtier answered only by a look which made Villefort's blood run cold. He motioned to Valentine to approach. In a moment, thanks to her habit of conversing with her grandfather, she understood that he asked for a key. Then his eye was fixed on the drawer of a small chest between the windows. She opened the drawer, and found a key; and, understanding that was what he wanted, again watched his eyes, which turned toward an old secretaire which had been neglected for many years and was supposed to contain nothing but useless documents. 

 <Shall I open the secretaire?> asked Valentine. 

 <Yes,> said the old man. 

 <And the drawers?> 

 <Yes.> 

 <Those at the side?> 

 <No.> 

 <The middle one?> 

 <Yes.> 

 Valentine opened it and drew out a bundle of papers. <Is that what you wish for?> asked she. 

 <No.> 

 She took successively all the other papers out till the drawer was empty. <But there are no more,> said she. Noirtier's eye was fixed on the dictionary. 

 <Yes, I understand, grandfather,> said the young girl.  She pointed to each letter of the alphabet. At the letter S the old man stopped her. She opened, and found the word <secret.> 

 <Ah! is there a secret spring?> said Valentine. 

 <Yes,> said Noirtier. 

 <And who knows it?> Noirtier looked at the door where the servant had gone out. 

 <Barrois?> said she. 

 <Yes.> 

 <Shall I call him?> 

 <Yes.> 

 Valentine went to the door, and called Barrois. Villefort's impatience during this scene made the perspiration roll from his forehead, and Franz was stupefied. The old servant came. 

 <Barrois,> said Valentine, <my grandfather has told me to open that drawer in the secretaire, but there is a secret spring in it, which you know—will you open it?> 

 Barrois looked at the old man. <Obey,> said Noirtier's intelligent eye. Barrois touched a spring, the false bottom came out, and they saw a bundle of papers tied with a black string. 

 <Is that what you wish for?> said Barrois. 

 <Yes.> 

 <Shall I give these papers to M. de Villefort?> 

 <No.> 

 <To Mademoiselle Valentine?> 

 <No.> 

 <To M. Franz d'Épinay?> 

 <Yes.> 

 Franz, astonished, advanced a step. <To me, sir?> said he. 

 <Yes.> 

 Franz took them from Barrois and casting a glance at the cover, read: 

\begin{mail}{}{To be given, after my death, to General Durand, who shall bequeath the packet to his son, with an injunction to preserve it as containing an important document.}
	
\end{mail} 

<Well, sir,> asked Franz, <what do you wish me to do with this paper?> 

 <To preserve it, sealed up as it is, doubtless,> said the procureur. 

 <No,> replied Noirtier eagerly. 

 <Do you wish him to read it?> said Valentine. 

 <Yes,> replied the old man. 

 <You understand, baron, my grandfather wishes you to read this paper,> said Valentine. 

 <Then let us sit down,> said Villefort impatiently, <for it will take some time.> 

 <Sit down,> said the old man. Villefort took a chair, but Valentine remained standing by her father's side, and Franz before him, holding the mysterious paper in his hand. <Read,> said the old man. Franz untied it, and in the midst of the most profound silence read: 
 
\begin{center} 
 \textit{Extract of the report of a meeting of the Bonapartist Club in the Rue Saint-Jacques, held February 5th, 1815}. 
\end{center}
 Franz stopped. <February 5th, 1815!> said he; <it is the day my father was murdered.> Valentine and Villefort were dumb; the eye of the old man alone seemed to say clearly, <Go on.> 

 <But it was on leaving this club,> said he, <my father disappeared.> 

 Noirtier's eye continued to say, <Read.> He resumed:— 

\begin{quotation}
The undersigned Louis-Jacques Beaurepaire, lieutenant-colonel of artillery, Étienne Duchampy, general of brigade, and Claude Lecharpal, keeper of woods and forests, declare, that on the 4th of February, a letter arrived from the Island of Elba, recommending to the kindness and the confidence of the Bonapartist Club, General Flavien de Quesnel, who having served the emperor from 1804 to 1814 was supposed to be devoted to the interests of the Napoleon dynasty, notwithstanding the title of baron which Louis \textsc{xviii.} had just granted to him with his estate of Épinay.  

A note was in consequence addressed to General de Quesnel, begging him to be present at the meeting next day, the 5th. The note indicated neither the street nor the number of the house where the meeting was to be held; it bore no signature, but it announced to the general that someone would call for him if he would be ready at nine o'clock. The meetings were always held from that time till midnight. At nine o'clock the president of the club presented himself; the general was ready, the president informed him that one of the conditions of his introduction was that he should be eternally ignorant of the place of meeting, and that he would allow his eyes to be bandaged, swearing that he would not endeavour to take off the bandage. General de Quesnel accepted the condition, and promised on his honour not to seek to discover the road they took. The general's carriage was ready, but the president told him it was impossible for him to use it, since it was useless to blindfold the master if the coachman knew through what streets he went. 

<What must be done then?> asked the general.

<I have my carriage here,> said the president. 

<Have you, then, so much confidence in your servant that you can intrust him with a secret you will not allow me to know?> 

<Our coachman is a member of the club,> said the president; <we shall be driven by a State-Councillor.> 

<Then we run another risk,> said the general, laughing, <that of being upset.>

We insert this joke to prove that the general was not in the least compelled to attend the meeting, but that he came willingly. When they were seated in the carriage the president reminded the general of his promise to allow his eyes to be bandaged, to which he made no opposition. On the road the president thought he saw the general make an attempt to remove the handkerchief, and reminded him of his oath. <Sure enough,> said the general. The carriage stopped at an alley leading out of the Rue Saint-Jacques. The general alighted, leaning on the arm of the president, of whose dignity he was not aware, considering him simply as a member of the club; they went through the alley, mounted a flight of stairs, and entered the assembly-room. 

The deliberations had already begun. The members, apprised of the sort of presentation which was to be made that evening, were all in attendance. When in the middle of the room the general was invited to remove his bandage, he did so immediately, and was surprised to see so many well-known faces in a society of whose existence he had till then been ignorant. They questioned him as to his sentiments, but he contented himself with answering, that the letters from the Island of Elba ought to have informed them\longdash
\end{quotation}
 Franz interrupted himself by saying, <My father was a royalist; they need not have asked his sentiments, which were well known.> 

 <And hence,> said Villefort, <arose my affection for your father, my dear M. Franz. Opinions held in common are a ready bond of union.> 

 <Read again,> said the old man. 

 Franz continued: 
\begin{quotation}
The president then sought to make him speak more explicitly, but M. de Quesnel replied that he wished first to know what they wanted with him. He was then informed of the contents of the letter from the Island of Elba, in which he was recommended to the club as a man who would be likely to advance the interests of their party. One paragraph spoke of the return of Bonaparte and promised another letter and further details, on the arrival of the \textit{Pharaon} belonging to the shipbuilder Morrel, of Marseilles, whose captain was entirely devoted to the emperor. During all this time, the general, on whom they thought to have relied as on a brother, manifested evidently signs of discontent and repugnance. When the reading was finished, he remained silent, with knitted brows. 

<Well,> asked the president, <what do you say to this letter, general?> 

<I say that it is too soon after declaring myself for Louis \textsc{xviii.} to break my vow in behalf of the ex-emperor.> This answer was too clear to permit of any mistake as to his sentiments. <General,> said the president, <we acknowledge no King Louis \textsc{xviii.}, or an ex-emperor, but his majesty the emperor and king, driven from France, which is his kingdom, by violence and treason.> 

<Excuse me, gentlemen,> said the general; <you may not acknowledge Louis \textsc{xviii.}, but I do, as he has made me a baron and a field-marshal, and I shall never forget that for these two titles I am indebted to his happy return to France.> 

<Sir,> said the president, rising with gravity, <be careful what you say; your words clearly show us that they are deceived concerning you in the Island of Elba, and have deceived us! The communication has been made to you in consequence of the confidence placed in you, and which does you honour. Now we discover our error; a title and promotion attach you to the government we wish to overturn. We will not constrain you to help us; we enroll no one against his conscience, but we will compel you to act generously, even if you are not disposed to do so.> 

<You would call acting generously, knowing your conspiracy and not informing against you, that is what I should call becoming your accomplice. You see I am more candid than you.>
\end{quotation}

 <Ah, my father!> said Franz, interrupting himself. <I understand now why they murdered him.> Valentine could not help casting one glance towards the young man, whose filial enthusiasm it was delightful to behold. Villefort walked to and fro behind them. Noirtier watched the expression of each one, and preserved his dignified and commanding attitude. Franz returned to the manuscript, and continued: 


\begin{quotation}
<Sir,> said the president, <you have been invited to join this assembly—you were not forced here; it was proposed to you to come blindfolded—you accepted. When you complied with this twofold request you well knew we did not wish to secure the throne of Louis \textsc{xviii.}, or we should not take so much care to avoid the vigilance of the police. It would be conceding too much to allow you to put on a mask to aid you in the discovery of our secret, and then to remove it that you may ruin those who have confided in you. No, no, you must first say if you declare yourself for the king of a day who now reigns, or for his majesty the emperor.> 

<I am a royalist,> replied the general; <I have taken the oath of allegiance to Louis \textsc{xviii.}, and I will adhere to it.> These words were followed by a general murmur, and it was evident that several of the members were discussing the propriety of making the general repent of his rashness. 

The president again arose, and having imposed silence, said, <Sir, you are too serious and too sensible a man not to understand the consequences of our present situation, and your candour has already dictated to us the conditions which remain for us to offer you.> The general, putting his hand on his sword, exclaimed, <If you talk of honour, do not begin by disavowing its laws, and impose nothing by violence.> 

<And you, sir,> continued the president, with a calmness still more terrible than the general's anger, <I advise you not to touch your sword.> The general looked around him with slight uneasiness; however he did not yield, but calling up all his fortitude, said, <I will not swear.> 

<Then you must die,> replied the president calmly. M. d'Épinay became very pale; he looked round him a second time, several members of the club were whispering, and getting their arms from under their cloaks. <General,> said the president, <do not alarm yourself; you are among men of honour who will use every means to convince you before resorting to the last extremity, but as you have said, you are among conspirators, you are in possession of our secret, and you must restore it to us.> A significant silence followed these words, and as the general did not reply, <Close the doors,> said the president to the door-keeper.  

The same deadly silence succeeded these words. Then the general advanced, and making a violent effort to control his feelings, <I have a son,> said he, <and I ought to think of him, finding myself among assassins.> 

<General,> said the chief of the assembly, <one man may insult fifty—it is the privilege of weakness. But he does wrong to use his privilege. Follow my advice, swear, and do not insult.> The general, again daunted by the superiority of the chief, hesitated a moment; then advancing to the president's desk, <What is the form,> said he. 

<It is this:—<I swear by my honour not to reveal to anyone what I have seen and heard on the 5th of February, 1815, between nine and ten o'clock in the evening; and I plead guilty of death should I ever violate this oath.>>

The general appeared to be affected by a nervous tremor, which prevented his answering for some moments; then, overcoming his manifest repugnance, he pronounced the required oath, but in so low a tone as to be scarcely audible to the majority of the members, who insisted on his repeating it clearly and distinctly, which he did. 

<Now am I at liberty to retire?> said the general. The president rose, appointed three members to accompany him, and got into the carriage with the general after bandaging his eyes. One of those three members was the coachman who had driven them there. The other members silently dispersed. 

<Where do you wish to be taken?> asked the president. 

<Anywhere out of your presence,> replied M. d'Épinay. 

<Beware, sir,> replied the president, <you are no longer in the assembly, and have only to do with individuals; do not insult them unless you wish to be held responsible.> 

But instead of listening, M. d'Épinay went on, <You are still as brave in your carriage as in your assembly because you are still four against one.> The president stopped the coach. They were at that part of the Quai des Ormes where the steps lead down to the river. 

<Why do you stop here?> asked d'Épinay. 

<Because, sir,> said the president, <you have insulted a man, and that man will not go one step farther without demanding honourable reparation.> 

<Another method of assassination?> said the general, shrugging his shoulders. 

<Make no noise, sir, unless you wish me to consider you as one of the men of whom you spoke just now as cowards, who take their weakness for a shield. You are alone, one alone shall answer you; you have a sword by your side, I have one in my cane; you have no witness, one of these gentlemen will serve you. Now, if you please, remove your bandage.> The general tore the handkerchief from his eyes. <At last,> said he, <I shall know with whom I have to do.> They opened the door and the four men alighted.
\end{quotation}


 Franz again interrupted himself, and wiped the cold drops from his brow; there was something awful in hearing the son read aloud in trembling pallor these details of his father's death, which had hitherto been a mystery. Valentine clasped her hands as if in prayer. Noirtier looked at Villefort with an almost sublime expression of contempt and pride. 

 Franz continued: 

\begin{quotation}
It was, as we said, the fifth of February. For three days the mercury had been five or six degrees below freezing and the steps were covered with ice. The general was stout and tall, the president offered him the side of the railing to assist him in getting down. The two witnesses followed. 

It was a dark night. The ground from the steps to the river was covered with snow and hoarfrost, the water of the river looked black and deep. One of the seconds went for a lantern in a coal-barge near, and by its light they examined the weapons. 

The president's sword, which was simply, as he had said, one he carried in his cane, was five inches shorter than the general's, and had no guard. 

The general proposed to cast lots for the swords, but the president said it was he who had given the provocation, and when he had given it he had supposed each would use his own arms. The witnesses endeavoured to insist, but the president bade them be silent. The lantern was placed on the ground, the two adversaries took their stations, and the duel began. The light made the two swords appear like flashes of lightning; as for the men, they were scarcely perceptible, the darkness was so great.  

General d'Épinay passed for one of the best swordsmen in the army, but he was pressed so closely in the onset that he missed his aim and fell. The witnesses thought he was dead, but his adversary, who knew he had not struck him, offered him the assistance of his hand to rise. The circumstance irritated instead of calming the general, and he rushed on his adversary. 

But his opponent did not allow his guard to be broken. He received him on his sword and three times the general drew back on finding himself too closely engaged, and then returned to the charge. At the third he fell again. 

They thought he slipped, as at first, and the witnesses, seeing he did not move, approached and endeavoured to raise him, but the one who passed his arm around the body found it was moistened with blood. The general, who had almost fainted, revived. <Ah,> said he, <they have sent some fencing-master to fight with me.> The president, without answering, approached the witness who held the lantern, and raising his sleeve, showed him two wounds he had received in his arm; then opening his coat, and unbuttoning his waistcoat, displayed his side, pierced with a third wound. Still he had not even uttered a sigh. General d'Épinay died five minutes after.
\end{quotation}

 Franz read these last words in a voice so choked that they were hardly audible, and then stopped, passing his hand over his eyes as if to dispel a cloud; but after a moment's silence, he continued: 

\begin{mail}{}{}
The president went up the steps, after pushing his sword into his cane; a track of blood on the snow marked his course. He had scarcely arrived at the top when he heard a heavy splash in the water—it was the general's body, which the witnesses had just thrown into the river after ascertaining that he was dead. The general fell, then, in a loyal duel, and not in ambush as it might have been reported. In proof of this we have signed this paper to establish the truth of the facts, lest the moment should arrive when either of the actors in this terrible scene should be accused of premeditated murder or of infringement of the laws of honour. 

\closeletter[Signed,]{Beaurepaire, Duchampy, and Lecharpal.}
\end{mail}

 When Franz had finished reading this account, so dreadful for a son; when Valentine, pale with emotion, had wiped away a tear; when Villefort, trembling, and crouched in a corner, had endeavoured to lessen the storm by supplicating glances at the implacable old man,— 

 <Sir,> said d'Épinay to Noirtier, <since you are well acquainted with all these details, which are attested by honourable signatures,—since you appear to take some interest in me, although you have only manifested it hitherto by causing me sorrow, refuse me not one final satisfaction—tell me the name of the president of the club, that I may at least know who killed my father.> 

 Villefort mechanically felt for the handle of the door; Valentine, who understood sooner than anyone her grandfather's answer, and who had often seen two scars upon his right arm, drew back a few steps. 

 <Mademoiselle,> said Franz, turning towards Valentine, <unite your efforts with mine to find out the name of the man who made me an orphan at two years of age.> Valentine remained dumb and motionless. 

 <Hold, sir,> said Villefort, <do not prolong this dreadful scene. The names have been purposely concealed; my father himself does not know who this president was, and if he knows, he cannot tell you; proper names are not in the dictionary.> 

 <Oh, misery,> cried Franz: <the only hope which sustained me and enabled me to read to the end was that of knowing, at least, the name of him who killed my father! Sir, sir,> cried he, turning to Noirtier, <do what you can—make me understand in some way!> 

 <Yes,> replied Noirtier. 

 <Oh, mademoiselle, mademoiselle!> cried Franz, <your grandfather says he can indicate the person. Help me,—lend me your assistance!> 

 Noirtier looked at the dictionary. Franz took it with a nervous trembling, and repeated the letters of the alphabet successively, until he came to M. At that letter the old man signified <Yes.> 

 <M,> repeated Franz. The young man's finger, glided over the words, but at each one Noirtier answered by a negative sign. Valentine hid her head between her hands. At length, Franz arrived at the word \textsc{myself}.  <Yes!> 

 <You!> cried Franz, whose hair stood on end; <you, M. Noirtier—you killed my father?> 

 <Yes!> replied Noirtier, fixing a majestic look on the young man. Franz fell powerless on a chair; Villefort opened the door and escaped, for the idea had entered his mind to stifle the little remaining life in the heart of this terrible old man.