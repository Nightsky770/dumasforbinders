\chapter{The Guests} 

 \lettrine{I}{n} the house in the Rue du Helder, where Albert had invited the Count of Monte Cristo, everything was being prepared on the morning of the 21st of May to do honour to the occasion. Albert de Morcerf inhabited a pavilion situated at the corner of a large court, and directly opposite another building, in which were the servants' apartments. Two windows only of the pavilion faced the street; three other windows looked into the court, and two at the back into the garden. 

 Between the court and the garden, built in the heavy style of the imperial architecture, was the large and fashionable dwelling of the Count and Countess of Morcerf. 

 A high wall surrounded the whole of the property, surmounted at intervals by vases filled with flowers, and broken in the centre by a large gate of gilded iron, which served as the carriage entrance. A small door, close to the lodge of the \textit{concierge}, gave ingress and egress to the servants and masters when they were on foot. 

 It was easy to discover that the delicate care of a mother, unwilling to part from her son, and yet aware that a young man of the viscount's age required the full exercise of his liberty, had chosen this habitation for Albert. There were not lacking, however, evidences of what we may call the intelligent egoism of a youth who is charmed with the indolent, careless life of an only son, and who lives as it were in a gilded cage. By means of the two windows looking into the street, Albert could see all that passed; the sight of what is going on is necessary to young men, who always want to see the world traverse their horizon, even if that horizon is only a public thoroughfare. Then, should anything appear to merit a more minute examination, Albert de Morcerf could follow up his researches by means of a small gate, similar to that close to the \textit{concierge's} door, and which merits a particular description. 

 It was a little entrance that seemed never to have been opened since the house was built, so entirely was it covered with dust and dirt; but the well-oiled hinges and locks told quite another story. This door was a mockery to the \textit{concierge}, from whose vigilance and jurisdiction it was free, and, like that famous portal in the \textit{Arabian Nights}, opening at the <\textit{Sesame}> of Ali Baba, it was wont to swing backward at a cabalistic word or a concerted tap from without from the sweetest voices or whitest fingers in the world. 

 At the end of a long corridor, with which the door communicated, and which formed the antechamber, was, on the right, Albert's breakfast-room, looking into the court, and on the left the salon, looking into the garden. Shrubs and creeping plants covered the windows, and hid from the garden and court these two apartments, the only rooms into which, as they were on the ground floor, the prying eyes of the curious could penetrate. 

 On the floor above were similar rooms, with the addition of a third, formed out of the antechamber; these three rooms were a salon, a boudoir, and a bedroom. The salon downstairs was only an Algerian divan, for the use of smokers. The boudoir upstairs communicated with the bedchamber by an invisible door on the staircase; it was evident that every precaution had been taken. Above this floor was a large \textit{atelier}, which had been increased in size by pulling down the partitions—a pandemonium, in which the artist and the dandy strove for pre-eminence. 

 There were collected and piled up all Albert's successive caprices, hunting-horns, bass-viols, flutes—a whole orchestra, for Albert had had not a taste but a fancy for music; easels, palettes, brushes, pencils—for music had been succeeded by painting; foils, boxing-gloves, broadswords, and single-sticks—for, following the example of the fashionable young men of the time, Albert de Morcerf cultivated, with far more perseverance than music and drawing, the three arts that complete a dandy's education, i.e., fencing, boxing, and single-stick; and it was here that he received Grisier, Cooks, and Charles Leboucher. 

 The rest of the furniture of this privileged apartment consisted of old cabinets, filled with Chinese porcelain and Japanese vases, Lucca della Robbia \textit{faïences}, and Palissy platters; of old armchairs, in which perhaps had sat Henry \textsc{iv.} or Sully, Louis \textsc{xiii.} or Richelieu—for two of these armchairs, adorned with a carved shield, on which were engraved the fleur-de-lis of France on an azure field, evidently came from the Louvre, or, at least, some royal residence. 

 Over these dark and sombre chairs were thrown splendid stuffs, dyed beneath Persia's sun, or woven by the fingers of the women of Calcutta or of Chandernagor. What these stuffs did there, it was impossible to say; they awaited, while gratifying the eyes, a destination unknown to their owner himself; in the meantime they filled the place with their golden and silky reflections. 

 In the centre of the room was a Roller and Blanchet <baby grand> piano in rosewood, but holding the potentialities of an orchestra in its narrow and sonorous cavity, and groaning beneath the weight of the \textit{chefs-d'œuvre} of Beethoven, Weber, Mozart, Haydn, Grétry, and Porpora. 

 On the walls, over the doors, on the ceiling, were swords, daggers, Malay creeses, maces, battle-axes; gilded, damasked, and inlaid suits of armor; dried plants, minerals, and stuffed birds, their flame-coloured wings outspread in motionless flight, and their beaks forever open. This was Albert's favourite lounging place. 

 However, the morning of the appointment, the young man had established himself in the small salon downstairs. There, on a table, surrounded at some distance by a large and luxurious divan, every species of tobacco known,—from the yellow tobacco of Petersburg to the black of Sinai, and so on along the scale from Maryland and Porto Rico, to Latakia,—was exposed in pots of crackled earthenware of which the Dutch are so fond; beside them, in boxes of fragrant wood, were ranged, according to their size and quality, puros, regalias, havanas, and manillas; and, in an open cabinet, a collection of German pipes, of chibouques, with their amber mouth-pieces ornamented with coral, and of narghiles, with their long tubes of morocco, awaiting the caprice or the sympathy of the smokers. 

 Albert had himself presided at the arrangement, or, rather, the symmetrical derangement, which, after coffee, the guests at a breakfast of modern days love to contemplate through the vapor that escapes from their mouths, and ascends in long and fanciful wreaths to the ceiling. 

 At a quarter to ten, a valet entered; he composed, with a little groom named John, and who only spoke English, all Albert's establishment, although the cook of the hotel was always at his service, and on great occasions the count's \textit{chasseur} also. This valet, whose name was Germain, and who enjoyed the entire confidence of his young master, held in one hand a number of papers, and in the other a packet of letters, which he gave to Albert. Albert glanced carelessly at the different missives, selected two written in a small and delicate hand, and enclosed in scented envelopes, opened them and perused their contents with some attention. 

 <How did these letters come?> said he. 

 <One by the post, Madame Danglars' footman left the other.> 

 <Let Madame Danglars know that I accept the place she offers me in her box. Wait; then, during the day, tell Rosa that when I leave the Opera I will sup with her as she wishes. Take her six bottles of different wine—Cyprus, sherry, and Malaga, and a barrel of Ostend oysters; get them at Borel's, and be sure you say they are for me.> 

 <At what o'clock, sir, do you breakfast?>

<What time is it now?> 

 <A quarter to ten.> 

 <Very well, at half past ten. Debray will, perhaps, be obliged to go to the minister—and besides> (Albert looked at his tablets), <it is the hour I told the count, 21st May, at half past ten; and though I do not much rely upon his promise, I wish to be punctual. Is the countess up yet?> 

 <If you wish, I will inquire.> 

 <Yes, ask her for one of her \textit{liqueur} cellarets, mine is incomplete; and tell her I shall have the honour of seeing her about three o'clock, and that I request permission to introduce someone to her.> 

 The valet left the room. Albert threw himself on the divan, tore off the cover of two or three of the papers, looked at the theatre announcements, made a face seeing they gave an opera, and not a ballet; hunted vainly amongst the advertisements for a new tooth-powder of which he had heard, and threw down, one after the other, the three leading papers of Paris, muttering, 

 <These papers become more and more stupid every day.> 

 A moment after, a carriage stopped before the door, and the servant announced M. Lucien Debray. A tall young man, with light hair, clear gray eyes, and thin and compressed lips, dressed in a blue coat with beautifully carved gold buttons, a white neckcloth, and a tortoiseshell eye-glass suspended by a silken thread, and which, by an effort of the superciliary and zygomatic muscles, he fixed in his eye, entered, with a half-official air, without smiling or speaking. 

 <Good-morning, Lucien, good-morning,> said Albert; <your punctuality really alarms me. What do I say? punctuality! You, whom I expected last, you arrive at five minutes to ten, when the time fixed was half-past! Has the ministry resigned?> 

 <No, my dear fellow,> returned the young man, seating himself on the divan; <reassure yourself; we are tottering always, but we never fall, and I begin to believe that we shall pass into a state of immobility, and then the affairs of the Peninsula will completely consolidate us.> 

 <Ah, true; you drive Don Carlos out of Spain.> 

 <No, no, my dear fellow, do not confound our plans. We take him to the other side of the French frontier, and offer him hospitality at Bourges.> 

 <At Bourges?> 

 <Yes, he has not much to complain of; Bourges is the capital of Charles \textsc{vii.} Do you not know that all Paris knew it yesterday, and the day before it had already transpired on the Bourse, and M. Danglars (I do not know by what means that man contrives to obtain intelligence as soon as we do) made a million!> 

 <And you another order, for I see you have a blue ribbon at your button-hole.> 

 <Yes; they sent me the order of Charles \textsc{iii.},> returned Debray carelessly. 

 <Come, do not affect indifference, but confess you were pleased to have it.> 

 <Oh, it is very well as a finish to the toilet. It looks very neat on a black coat buttoned up.> 

 <And makes you resemble the Prince of Wales or the Duke of Reichstadt.> 

 <It is for that reason you see me so early.> 

 <Because you have the order of Charles \textsc{iii.}, and you wish to announce the good news to me?> 

 <No, because I passed the night writing letters,—five-and-twenty despatches. I returned home at daybreak, and strove to sleep; but my head ached and I got up to have a ride for an hour. At the Bois de Boulogne, \textit{ennui} and hunger attacked me at once,—two enemies who rarely accompany each other, and who are yet leagued against me, a sort of Carlo-republican alliance. I then recollected you gave a breakfast this morning, and here I am. I am hungry, feed me; I am bored, amuse me.> 

 <It is my duty as your host,> returned Albert, ringing the bell, while Lucien turned over, with his gold-mounted cane, the papers that lay on the table. <Germain, a glass of sherry and a biscuit. In the meantime, my dear Lucien, here are cigars—contraband, of course—try them, and persuade the minister to sell us such instead of poisoning us with cabbage leaves.> 

 <\textit{Peste!} I will do nothing of the kind; the moment they come from government you would find them execrable. Besides, that does not concern the home but the financial department. Address yourself to M. Humann, section of the indirect contributions, corridor A., № 26.> 

 <On my word,> said Albert, <you astonish me by the extent of your knowledge. Take a cigar.> 

 <Really, my dear Albert,> replied Lucien, lighting a manilla at a rose-coloured taper that burnt in a beautifully enamelled stand—<how happy you are to have nothing to do. You do not know your own good fortune!> 

 <And what would you do, my dear diplomatist,> replied Morcerf, with a slight degree of irony in his voice, <if you did nothing? What? private secretary to a minister, plunged at once into European cabals and Parisian intrigues; having kings, and, better still, queens, to protect, parties to unite, elections to direct; making more use of your cabinet with your pen and your telegraph than Napoleon did of his battle-fields with his sword and his victories; possessing five-and-twenty thousand francs a year, besides your place; a horse, for which Château-Renaud offered you four hundred louis, and which you would not part with; a tailor who never disappoints you; with the opera, the jockey-club, and other diversions, can you not amuse yourself? Well, I will amuse you.> 

 <How?> 

 <By introducing to you a new acquaintance.> 

 <A man or a woman?> 

 <A man.> 

 <I know so many men already.> 

 <But you do not know this man.> 

 <Where does he come from—the end of the world?> 

 <Farther still, perhaps.> 

 <The deuce! I hope he does not bring our breakfast with him.> 

 <Oh, no; our breakfast comes from my father's kitchen. Are you hungry?> 

 <Humiliating as such a confession is, I am. But I dined at M. de Villefort's, and lawyers always give you very bad dinners. You would think they felt some remorse; did you ever remark that?> 

 <Ah, depreciate other persons' dinners; you ministers give such splendid ones.> 

 <Yes; but we do not invite people of fashion. If we were not forced to entertain a parcel of country boobies because they think and vote with us, we should never dream of dining at home, I assure you.> 

 <Well, take another glass of sherry and another biscuit.> 

 <Willingly. Your Spanish wine is excellent. You see we were quite right to pacify that country.> 

 <Yes; but Don Carlos?> 

 <Well, Don Carlos will drink Bordeaux, and in ten years we will marry his son to the little queen.> 

 <You will then obtain the Golden Fleece, if you are still in the ministry.> 

 <I think, Albert, you have adopted the system of feeding me on smoke this morning.> 

 <Well, you must allow it is the best thing for the stomach; but I hear Beauchamp in the next room; you can dispute together, and that will pass away the time.> 

 <About what?> 

 <About the papers.> 

 <My dear friend,> said Lucien with an air of sovereign contempt, <do I ever read the papers?> 

 <Then you will dispute the more.> 

 <M. Beauchamp,> announced the servant. <Come in, come in,> said Albert, rising and advancing to meet the young man. <Here is Debray, who detests you without reading you, so he says.> 

 <He is quite right,> returned Beauchamp; <for I criticise him without knowing what he does. Good-day, commander!> 

 <Ah, you know that already,> said the private secretary, smiling and shaking hands with him. 

 <\textit{Pardieu!}> 

 <And what do they say of it in the world?> 

 <In which world? we have so many worlds in the year of grace 1838.> 

 <In the entire political world, of which you are one of the leaders.> 

 <They say that it is quite fair, and that sowing so much red, you ought to reap a little blue.> 

 <Come, come, that is not bad!> said Lucien. <Why do you not join our party, my dear Beauchamp? With your talents you would make your fortune in three or four years.> 

 <I only await one thing before following your advice; that is, a minister who will hold office for six months. My dear Albert, one word, for I must give poor Lucien a respite. Do we breakfast or dine? I must go to the Chamber, for our life is not an idle one.> 

 <You only breakfast; I await two persons, and the instant they arrive we shall sit down to table.> 