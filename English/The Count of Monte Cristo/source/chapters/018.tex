\chapter{The Treasure} 

 \lettrine{W}{hen} Dantès returned next morning to the chamber of his companion in captivity, he found Faria seated and looking composed. In the ray of light which entered by the narrow window of his cell, he held open in his left hand, of which alone, it will be recollected, he retained the use, a sheet of paper, which, from being constantly rolled into a small compass, had the form of a cylinder, and was not easily kept open. He did not speak, but showed the paper to Dantès. 

 <What is that?> he inquired. 

 <Look at it,> said the abbé with a smile. 

 <I have looked at it with all possible attention,> said Dantès, <and I only see a half-burnt paper, on which are traces of Gothic characters inscribed with a peculiar kind of ink.> 

 <This paper, my friend,> said Faria, <I may now avow to you, since I have the proof of your fidelity—this paper is my treasure, of which, from this day forth, one-half belongs to you.> 

 The sweat started forth on Dantès' brow. Until this day and for how long a time!—he had refrained from talking of the treasure, which had brought upon the abbé the accusation of madness. With his instinctive delicacy Edmond had preferred avoiding any touch on this painful chord, and Faria had been equally silent. He had taken the silence of the old man for a return to reason; and now these few words uttered by Faria, after so painful a crisis, seemed to indicate a serious relapse into mental alienation. 

 <Your treasure?> stammered Dantès. Faria smiled. 

 <Yes,> said he. <You have, indeed, a noble nature, Edmond, and I see by your paleness and agitation what is passing in your heart at this moment. No, be assured, I am not mad. This treasure exists, Dantès, and if I have not been allowed to possess it, you will. Yes—you. No one would listen or believe me, because everyone thought me mad; but you, who must know that I am not, listen to me, and believe me so afterwards if you will.> 

 <Alas,> murmured Edmond to himself, <this is a terrible relapse! There was only this blow wanting.> Then he said aloud, <My dear friend, your attack has, perhaps, fatigued you; had you not better repose awhile? Tomorrow, if you will, I will hear your narrative; but today I wish to nurse you carefully. Besides,> he said, <a treasure is not a thing we need hurry about.> 

 <On the contrary, it is a matter of the utmost importance, Edmond!> replied the old man. <Who knows if tomorrow, or the next day after, the third attack may not come on? and then must not all be over? Yes, indeed, I have often thought with a bitter joy that these riches, which would make the wealth of a dozen families, will be forever lost to those men who persecute me. This idea was one of vengeance to me, and I tasted it slowly in the night of my dungeon and the despair of my captivity. But now I have forgiven the world for the love of you; now that I see you, young and with a promising future,—now that I think of all that may result to you in the good fortune of such a disclosure, I shudder at any delay, and tremble lest I should not assure to one as worthy as yourself the possession of so vast an amount of hidden wealth.> 

 Edmond turned away his head with a sigh. 

 <You persist in your incredulity, Edmond,> continued Faria. <My words have not convinced you. I see you require proofs. Well, then, read this paper, which I have never shown to anyone.> 

 <Tomorrow, my dear friend,> said Edmond, desirous of not yielding to the old man's madness. <I thought it was understood that we should not talk of that until tomorrow.> 

 <Then we will not talk of it until tomorrow; but read this paper today.> 

 <I will not irritate him,> thought Edmond, and taking the paper, of which half was wanting,—having been burnt, no doubt, by some accident,—he read:  
 
 \begin{blockquote}\oldfont
 this treasure, which may amount to two\dots \\ of Roman crowns in the most distant a\dots \\ of the second opening wh\dots \\ declare to belong to him alo\dots \\ heir.\\                     <25th April, 149'> 
 \end{blockquote}

 <Well!> said Faria, when the young man had finished reading it. 

 <Why,> replied Dantès, <I see nothing but broken lines and unconnected words, which are rendered illegible by fire.> 

 <Yes, to you, my friend, who read them for the first time; but not for me, who have grown pale over them by many nights' study, and have reconstructed every phrase, completed every thought.> 

 <And do you believe you have discovered the hidden meaning?> 

 <I am sure I have, and you shall judge for yourself; but first listen to the history of this paper.> 

 <Silence!> exclaimed Dantès. <Steps approach—I go—adieu!> 

 And Dantès, happy to escape the history and explanation which would be sure to confirm his belief in his friend's mental instability, glided like a snake along the narrow passage; while Faria, restored by his alarm to a certain amount of activity, pushed the stone into place with his foot, and covered it with a mat in order the more effectually to avoid discovery. 

 It was the governor, who, hearing of Faria's illness from the jailer, had come in person to see him. 

 Faria sat up to receive him, avoiding all gestures in order that he might conceal from the governor the paralysis that had already half stricken him with death. His fear was lest the governor, touched with pity, might order him to be removed to better quarters, and thus separate him from his young companion. But fortunately this was not the case, and the governor left him, convinced that the poor madman, for whom in his heart he felt a kind of affection, was only troubled with a slight indisposition. 

 During this time, Edmond, seated on his bed with his head in his hands, tried to collect his scattered thoughts. Faria, since their first acquaintance, had been on all points so rational and logical, so wonderfully sagacious, in fact, that he could not understand how so much wisdom on all points could be allied with madness. Was Faria deceived as to his treasure, or was all the world deceived as to Faria? 

 Dantès remained in his cell all day, not daring to return to his friend, thinking thus to defer the moment when he should be convinced, once for all, that the abbé was mad—such a conviction would be so terrible! 

 But, towards the evening after the hour for the customary visit had gone by, Faria, not seeing the young man appear, tried to move and get over the distance which separated them. Edmond shuddered when he heard the painful efforts which the old man made to drag himself along; his leg was inert, and he could no longer make use of one arm. Edmond was obliged to assist him, for otherwise he would not have been able to enter by the small aperture which led to Dantès' chamber. 

 <Here I am, pursuing you remorselessly,> he said with a benignant smile. <You thought to escape my munificence, but it is in vain. Listen to me.> 

 Edmond saw there was no escape, and placing the old man on his bed, he seated himself on the stool beside him. 

 <You know,> said the abbé, \enquote{that I was the secretary and intimate friend of Cardinal Spada, the last of the princes of that name. I owe to this worthy lord all the happiness I ever knew. He was not rich, although the wealth of his family had passed into a proverb, and I heard the phrase very often, <As rich as a Spada.> But he, like public rumour, lived on this reputation for wealth; his palace was my paradise. I was tutor to his nephews, who are dead; and when he was alone in the world, I tried by absolute devotion to his will, to make up to him all he had done for me during ten years of unremitting kindness. The cardinal's house had no secrets for me. I had often seen my noble patron annotating ancient volumes, and eagerly searching amongst dusty family manuscripts. One day when I was reproaching him for his unavailing searches, and deploring the prostration of mind that followed them, he looked at me, and, smiling bitterly, opened a volume relating to the History of the City of Rome. There, in the twentieth chapter of the Life of Pope Alexander VI., were the following lines, which I can never forget:— 

<The great wars of Romagna had ended; Cæsar Borgia, who had completed his conquest, had need of money to purchase all Italy. The pope had also need of money to bring matters to an end with Louis XII. King of France, who was formidable still in spite of his recent reverses; and it was necessary, therefore, to have recourse to some profitable scheme, which was a matter of great difficulty in the impoverished condition of exhausted Italy. His holiness had an idea. He determined to make two cardinals.> 

By choosing two of the greatest personages of Rome, especially rich men—\textit{this} was the return the Holy Father looked for. In the first place, he could sell the great appointments and splendid offices which the cardinals already held; and then he had the two hats to sell besides. There was a third point in view, which will appear hereafter. 

The pope and Cæsar Borgia first found the two future cardinals; they were Giovanni Rospigliosi, who held four of the highest dignities of the Holy See, and Cæsar Spada, one of the noblest and richest of the Roman nobility; both felt the high honour of such a favour from the pope. They were ambitious, and Cæsar Borgia soon found purchasers for their appointments. The result was, that Rospigliosi and Spada paid for being cardinals, and eight other persons paid for the offices the cardinals held before their elevation, and thus eight hundred thousand crowns entered into the coffers of the speculators.  

It is time now to proceed to the last part of the speculation. The pope heaped attentions upon Rospigliosi and Spada, conferred upon them the insignia of the cardinalate, and induced them to arrange their affairs and take up their residence at Rome. Then the pope and Cæsar Borgia invited the two cardinals to dinner. This was a matter of dispute between the Holy Father and his son. Cæsar thought they could make use of one of the means which he always had ready for his friends, that is to say, in the first place, the famous key which was given to certain persons with the request that they go and open a designated cupboard. This key was furnished with a small iron point,—a negligence on the part of the locksmith. When this was pressed to effect the opening of the cupboard, of which the lock was difficult, the person was pricked by this small point, and died next day. Then there was the ring with the lion's head, which Cæsar wore when he wanted to greet his friends with a clasp of the hand. The lion bit the hand thus favoured, and at the end of twenty-four hours, the bite was mortal. 

Cæsar proposed to his father, that they should either ask the cardinals to open the cupboard, or shake hands with them; but Alexander VI. replied: <Now as to the worthy cardinals, Spada and Rospigliosi, let us ask both of them to dinner, something tells me that we shall get that money back. Besides, you forget, Cæsar, an indigestion declares itself immediately, while a prick or a bite occasions a delay of a day or two.> Cæsar gave way before such cogent reasoning, and the cardinals were consequently invited to dinner. 

The table was laid in a vineyard belonging to the pope, near San Pierdarena, a charming retreat which the cardinals knew very well by report. Rospigliosi, quite set up with his new dignities, went with a good appetite and his most ingratiating manner. Spada, a prudent man, and greatly attached to his only nephew, a young captain of the highest promise, took paper and pen, and made his will. He then sent word to his nephew to wait for him near the vineyard; but it appeared the servant did not find him. 

Spada knew what these invitations meant; since Christianity, so eminently civilizing, had made progress in Rome, it was no longer a centurion who came from the tyrant with a message, <Cæsar wills that you die.> but it was a legate \textit{à latere}, who came with a smile on his lips to say from the pope, <His holiness requests you to dine with him.> 

Spada set out about two o'clock to San Pierdarena. The pope awaited him. The first sight that attracted the eyes of Spada was that of his nephew, in full costume, and Cæsar Borgia paying him most marked attentions. Spada turned pale, as Cæsar looked at him with an ironical air, which proved that he had anticipated all, and that the snare was well spread. 

They began dinner and Spada was only able to inquire of his nephew if he had received his message. The nephew replied no; perfectly comprehending the meaning of the question. It was too late, for he had already drunk a glass of excellent wine, placed for him expressly by the pope's butler. Spada at the same moment saw another bottle approach him, which he was pressed to taste. An hour afterwards a physician declared they were both poisoned through eating mushrooms. Spada died on the threshold of the vineyard; the nephew expired at his own door, making signs which his wife could not comprehend. 

Then Cæsar and the pope hastened to lay hands on the heritage, under pretense of seeking for the papers of the dead man. But the inheritance consisted in this only, a scrap of paper on which Spada had written:—<I bequeath to my beloved nephew my coffers, my books, and, amongst others, my breviary with the gold corners, which I beg he will preserve in remembrance of his affectionate uncle.> 

The heirs sought everywhere, admired the breviary, laid hands on the furniture, and were greatly astonished that Spada, the rich man, was really the most miserable of uncles—no treasures—unless they were those of science, contained in the library and laboratories. That was all. Cæsar and his father searched, examined, scrutinized, but found nothing, or at least very little; not exceeding a few thousand crowns in plate, and about the same in ready money; but the nephew had time to say to his wife before he expired: <Look well among my uncle's papers; there is a will.> 

They sought even more thoroughly than the august heirs had done, but it was fruitless. There were two palaces and a vineyard behind the Palatine Hill; but in these days landed property had not much value, and the two palaces and the vineyard remained to the family since they were beneath the rapacity of the pope and his son. Months and years rolled on. Alexander VI. died, poisoned,—you know by what mistake. Cæsar, poisoned at the same time, escaped by shedding his skin like a snake; but the new skin was spotted by the poison till it looked like a tiger's. Then, compelled to quit Rome, he went and got himself obscurely killed in a night skirmish, scarcely noticed in history. 

After the pope's death and his son's exile, it was supposed that the Spada family would resume the splendid position they had held before the cardinal's time; but this was not the case. The Spadas remained in doubtful ease, a mystery hung over this dark affair, and the public rumour was, that Cæsar, a better politician than his father, had carried off from the pope the fortune of the two cardinals. I say the two, because Cardinal Rospigliosi, who had not taken any precaution, was completely despoiled.}

 <Up to this point,> said Faria, interrupting the thread of his narrative, <this seems to you very meaningless, no doubt, eh?> 

 <Oh, my friend,> cried Dantès, <on the contrary, it seems as if I were reading a most interesting narrative; go on, I beg of you.> 

\enquote{I will. The family began to get accustomed to their obscurity. Years rolled on, and amongst the descendants some were soldiers, others diplomatists; some churchmen, some bankers; some grew rich, and some were ruined. I come now to the last of the family, whose secretary I was—the Count of Spada. I had often heard him complain of the disproportion of his rank with his fortune; and I advised him to invest all he had in an annuity. He did so, and thus doubled his income. The celebrated breviary remained in the family, and was in the count's possession. It had been handed down from father to son; for the singular clause of the only will that had been found, had caused it to be regarded as a genuine relic, preserved in the family with superstitious veneration. It was an illuminated book, with beautiful Gothic characters, and so weighty with gold, that a servant always carried it before the cardinal on days of great solemnity. 

At the sight of papers of all sorts,—titles, contracts, parchments, which were kept in the archives of the family, all descending from the poisoned cardinal, I in my turn examined the immense bundles of documents, like twenty servitors, stewards, secretaries before me; but in spite of the most exhaustive researches, I found—nothing. Yet I had read, I had even written a precise history of the Borgia family, for the sole purpose of assuring myself whether any increase of fortune had occurred to them on the death of the Cardinal Cæsar Spada; but could only trace the acquisition of the property of the Cardinal Rospigliosi, his companion in misfortune. 

I was then almost assured that the inheritance had neither profited the Borgias nor the family, but had remained unpossessed like the treasures of the Arabian Nights, which slept in the bosom of the earth under the eyes of the genie. I searched, ransacked, counted, calculated a thousand and a thousand times the income and expenditure of the family for three hundred years. It was useless. I remained in my ignorance, and the Count of Spada in his poverty. 

My patron died. He had reserved from his annuity his family papers, his library, composed of five thousand volumes, and his famous breviary. All these he bequeathed to me, with a thousand Roman crowns, which he had in ready money, on condition that I would have anniversary masses said for the repose of his soul, and that I would draw up a genealogical tree and history of his house. All this I did scrupulously. Be easy, my dear Edmond, we are near the conclusion. 

In 1807, a month before I was arrested, and a fortnight after the death of the Count of Spada, on the 25th of December (you will see presently how the date became fixed in my memory), I was reading, for the thousandth time, the papers I was arranging, for the palace was sold to a stranger, and I was going to leave Rome and settle at Florence, intending to take with me twelve thousand francs I possessed, my library, and the famous breviary, when, tired with my constant labour at the same thing, and overcome by a heavy dinner I had eaten, my head dropped on my hands, and I fell asleep about three o'clock in the afternoon.  

I awoke as the clock was striking six. I raised my head; I was in utter darkness. I rang for a light, but, as no one came, I determined to find one for myself. It was indeed but anticipating the simple manners which I should soon be under the necessity of adopting. I took a wax-candle in one hand, and with the other groped about for a piece of paper (my match-box being empty), with which I proposed to get a light from the small flame still playing on the embers. Fearing, however, to make use of any valuable piece of paper, I hesitated for a moment, then recollected that I had seen in the famous breviary, which was on the table beside me, an old paper quite yellow with age, and which had served as a marker for centuries, kept there by the request of the heirs. I felt for it, found it, twisted it up together, and putting it into the expiring flame, set light to it. 

But beneath my fingers, as if by magic, in proportion as the fire ascended, I saw yellowish characters appear on the paper. I grasped it in my hand, put out the flame as quickly as I could, lighted my taper in the fire itself, and opened the crumpled paper with inexpressible emotion, recognizing, when I had done so, that these characters had been traced in mysterious and sympathetic ink, only appearing when exposed to the fire; nearly one-third of the paper had been consumed by the flame. It was that paper you read this morning; read it again, Dantès, and then I will complete for you the incomplete words and unconnected sense.} 

 Faria, with an air of triumph, offered the paper to Dantès, who this time read the following words, traced with an ink of a reddish colour resembling rust:  
\begin{letter}
\begin{quotation}\raggedright\oldfont

	This 25th day of April, 1498, be\dots \\      
	Alexander VI., and fearing that not\dots \\      
	he may desire to become my heir, and re\dots \\      
	and Bentivoglio, who were poisoned,\dots \\      
	my sole heir, that I have bu\dots \\      
	and has visited with me, that is, in\dots \\      
	Island of Monte Cristo, all I poss\dots \\      
	jewels, diamonds, gems; that I alone\dots \\      
	may amount to nearly two mil\dots \\      
	will find on raising the twentieth ro\dots \\      
	creek to the east in a right line. Two open\dots \\      
	in these caves; the treasure is in the furthest a\dots \\      
	which treasure I bequeath and leave en\dots \\      
	as my sole heir.\\      
	25th April, 1498.\hspace{2em} Cæs\dots  
 \end{quotation}

\end{letter}

\begin{a4}
	\centerline{
\begin{minipage}{0.75\textwidth}\raggedright\oldfont

	This 25th day of April, 1498, be\dots \\      
	Alexander VI., and fearing that not\dots \\      
	he may desire to become my heir, and re\dots \\      
	and Bentivoglio, who were poisoned,\dots \\      
	my sole heir, that I have bu\dots \\      
	and has visited with me, that is, in\dots \\      
	Island of Monte Cristo, all I poss\dots \\      
	jewels, diamonds, gems; that I alone\dots \\      
	may amount to nearly two mil\dots \\      
	will find on raising the twentieth ro\dots \\      
	creek to the east in a right line. Two open\dots \\      
	in these caves; the treasure is in the furthest a\dots \\      
	which treasure I bequeath and leave en\dots \\      
	as my sole heir.\\      
	25th April, 1498.\\      
	Cæs\dots  
 \end{minipage}
}
\end{a4}

 <And now,> said the abbé, <read this other paper;> and he presented to Dantès a second leaf with fragments of lines written on it, which Edmond read as follows:  
 
\begin{quotation}\raggedleft\oldfont
 \dots ing invited to dine by his Holiness\\         
 \dots con\-tent with making me pay for my hat,\\    
 \dots serves for me the fate of Cardinals Cap\-ra\-ra\\            
 \dots I declare to my nephew, Guido Spada\\                       
 \dots ried in a place he knows\\                         
 \dots the caves of the small\\                  
 \dots essed of ingots, gold, money,\\  
 \dots know of the existence of this treasure, which\\            
 \dots lions of Roman crowns, and which he\\                              
 \dots ck from the small\\                            
 \dots ings have been made\\                            
 \dots ngle in the second;\\                                    
 \dots tire to him\\                                     
 \dots ar  Spada.\\
 \end{quotation}

 Faria followed him with an excited look. 

 <And now,> he said, when he saw that Dantès had read the last line, <put the two fragments together, and judge for yourself.> Dantès obeyed, and the conjointed pieces gave the following:  
 
% \vfill

\begin{quotation}\oldfont
 This 25th day of April, 1498, being invited to dine by his Holiness Alexander VI., and fearing that not content with making me pay for my hat, he may desire to become my heir, and reserves for me the fate of Cardinals Caprara and Bentivoglio, who were poisoned, I declare to my nephew, Guido Spada, my sole heir, that I have buried in a place he knows and has visited with me, that is, in the caves of the small Island of Monte Cristo, all I possessed of ingots, gold, money, jewels, diamonds, gems; that I alone know of the existence of this treasure, which may amount to nearly two millions of Roman crowns, and which he will find on raising the twentieth rock from the small creek to the east in a right line. Two openings have been made in these caves; the treasure is in the furthest angle in the second; which treasure I bequeath and leave entire to him as my sole heir. \\
25th April, 1498.\hfill Cæsar  Spada.
 \end{quotation}


 <Well, do you comprehend now?> inquired Faria. 

 <It is the declaration of Cardinal Spada, and the will so long sought for,> replied Edmond, still incredulous. 

 <Yes; a thousand times, yes!> 

 <And who completed it as it now is?> 

 <I did. Aided by the remaining fragment, I guessed the rest; measuring the length of the lines by those of the paper, and divining the hidden meaning by means of what was in part revealed, as we are guided in a cavern by the small ray of light above us.> 

 <And what did you do when you arrived at this conclusion?> 

 <I resolved to set out, and did set out at that very instant, carrying with me the beginning of my great work, the unity of the Italian kingdom; but for some time the imperial police (who at this period, quite contrary to what Napoleon desired so soon as he had a son born to him, wished for a partition of provinces) had their eyes on me; and my hasty departure, the cause of which they were unable to guess, having aroused their suspicions, I was arrested at the very moment I was leaving Piombino. >

 <Now,> continued Faria, addressing Dantès with an almost paternal expression, <now, my dear fellow, you know as much as I do myself. If we ever escape together, half this treasure is yours; if I die here, and you escape alone, the whole belongs to you.> 

 <But,> inquired Dantès hesitating, <has this treasure no more legitimate possessor in the world than ourselves?> 

 <No, no, be easy on that score; the family is extinct. The last Count of Spada, moreover, made me his heir, bequeathing to me this symbolic breviary, he bequeathed to me all it contained; no, no, make your mind satisfied on that point. If we lay hands on this fortune, we may enjoy it without remorse.> 

 <And you say this treasure amounts to\longdash> 

 <Two millions of Roman crowns; nearly thirteen millions of our money.>\footnote{Approximately \$429 million/\texteuro 376 million/\textsterling 317 million, based on the price of gold in 1815 and 2025.} 

 <Impossible!> said Dantès, staggered at the enormous amount. 

 <Impossible? and why?> asked the old man. <The Spada family was one of the oldest and most powerful families of the fifteenth century; and in those times, when other opportunities for investment were wanting, such accumulations of gold and jewels were by no means rare; there are at this day Roman families perishing of hunger, though possessed of nearly a million in diamonds and jewels, handed down by entail, and which they cannot touch.> 

 Edmond thought he was in a dream—he wavered between incredulity and joy. 

 <I have only kept this secret so long from you,> continued Faria, <that I might test your character, and then surprise you. Had we escaped before my attack of catalepsy, I should have conducted you to Monte Cristo; now,> he added, with a sigh, <it is you who will conduct me thither. Well, Dantès, you do not thank me?> 

 <This treasure belongs to you, my dear friend,> replied Dantès, <and to you only. I have no right to it. I am no relation of yours.> 

 <You are my son, Dantès,> exclaimed the old man. <You are the child of my captivity. My profession condemns me to celibacy. God has sent you to me to console, at one and the same time, the man who could not be a father, and the prisoner who could not get free.> 

 And Faria extended the arm of which alone the use remained to him to the young man, who threw himself upon his neck and wept. 