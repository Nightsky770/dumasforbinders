\chapter{Danglars' Signature}

 \lettrine{T}{he} next morning dawned dull and cloudy. During the night the undertakers had executed their melancholy office, and wrapped the corpse in the winding-sheet, which, whatever may be said about the equality of death, is at least a last proof of the luxury so pleasing in life. This winding-sheet was nothing more than a beautiful piece of cambric, which the young girl had bought a fortnight before. 

 During the evening two men, engaged for the purpose, had carried Noirtier from Valentine's room into his own, and contrary to all expectation there was no difficulty in withdrawing him from his child. The Abbé Busoni had watched till daylight, and then left without calling anyone. D'Avrigny returned about eight o'clock in the morning; he met Villefort on his way to Noirtier's room, and accompanied him to see how the old man had slept. They found him in the large armchair, which served him for a bed, enjoying a calm, nay, almost a smiling sleep. They both stood in amazement at the door. 

 <See,> said d'Avrigny to Villefort, <nature knows how to alleviate the deepest sorrow. No one can say that M. Noirtier did not love his child, and yet he sleeps.> 

 <Yes, you are right,> replied Villefort, surprised; <he sleeps, indeed! And this is the more strange, since the least contradiction keeps him awake all night.> 

 <Grief has stunned him,> replied d'Avrigny; and they both returned thoughtfully to the procureur's study. 

 <See, I have not slept,> said Villefort, showing his undisturbed bed; <grief does not stun me. I have not been in bed for two nights; but then look at my desk; see what I have written during these two days and nights. I have filled those papers, and have made out the accusation against the assassin Benedetto. Oh, work, work,—my passion, my joy, my delight,—it is for thee to alleviate my sorrows!> and he convulsively grasped the hand of d'Avrigny. 

 <Do you require my services now?> asked d'Avrigny. 

 <No,> said Villefort; <only return again at eleven o'clock; at twelve the—the—oh, Heavens, my poor, poor child!> and the procureur again becoming a man, lifted up his eyes and groaned. 

 <Shall you be present in the reception-room?> 

 <No; I have a cousin who has undertaken this sad office. I shall work, doctor—when I work I forget everything.> 

 And, indeed, no sooner had the doctor left the room, than he was again absorbed in work. On the doorsteps d'Avrigny met the cousin whom Villefort had mentioned, a personage as insignificant in our story as in the world he occupied—one of those beings designed from their birth to make themselves useful to others. He was punctual, dressed in black, with crape around his hat, and presented himself at his cousin's with a face made up for the occasion, and which he could alter as might be required. 

 At eleven o'clock the mourning-coaches rolled into the paved court, and the Rue du Faubourg Saint-Honoré was filled with a crowd of idlers, equally pleased to witness the festivities or the mourning of the rich, and who rush with the same avidity to a funeral procession as to the marriage of a duchess. 

 Gradually the reception-room filled, and some of our old friends made their appearance—we mean Debray, Château-Renaud, and Beauchamp, accompanied by all the leading men of the day at the bar, in literature, or the army, for M. de Villefort moved in the first Parisian circles, less owing to his social position than to his personal merit. 

 The cousin standing at the door ushered in the guests, and it was rather a relief to the indifferent to see a person as unmoved as themselves, and who did not exact a mournful face or force tears, as would have been the case with a father, a brother, or a lover. Those who were acquainted soon formed into little groups. One of them was made of Debray, Château-Renaud, and Beauchamp. 

 <Poor girl,> said Debray, like the rest, paying an involuntary tribute to the sad event,—<poor girl, so young, so rich, so beautiful! Could you have imagined this scene, Château-Renaud, when we saw her, at the most three weeks ago, about to sign that contract?> 

 <Indeed, no,> said Château-Renaud.” 

 <Did you know her?> 

 <I spoke to her once or twice at Madame de Morcerf's, among the rest; she appeared to me charming, though rather melancholy. Where is her stepmother? Do you know?> 

 <She is spending the day with the wife of the worthy gentleman who is receiving us.>

<Who is he?> 

 <Whom do you mean?> 

 <The gentleman who receives us? Is he a deputy?> 

 <Oh, no. I am condemned to witness those gentlemen every day,> said Beauchamp; <but he is perfectly unknown to me.> 

 <Have you mentioned this death in your paper?> 

 <It has been mentioned, but the article is not mine; indeed, I doubt if it will please M. Villefort, for it says that if four successive deaths had happened anywhere else than in the house of the king's attorney, he would have interested himself somewhat more about it.> 

 <Still,> said Château-Renaud, <Dr. d'Avrigny, who attends my mother, declares he is in despair about it. But whom are you seeking, Debray?> 

 <I am seeking the Count of Monte Cristo> said the young man. 

 <I met him on the boulevard, on my way here,> said Beauchamp. <I think he is about to leave Paris; he was going to his banker.> 

 <His banker? Danglars is his banker, is he not?> asked Château-Renaud of Debray. 

 <I believe so,> replied the secretary with slight uneasiness. <But Monte Cristo is not the only one I miss here; I do not see Morrel.> 

 <Morrel? Do they know him?> asked Château-Renaud. <I think he has only been introduced to Madame de Villefort.> 

 <Still, he ought to have been here,> said Debray; <I wonder what will be talked about tonight; this funeral is the news of the day. But hush, here comes our minister of justice; he will feel obliged to make some little speech to the cousin,> and the three young men drew near to listen. 

 Beauchamp told the truth when he said that on his way to the funeral he had met Monte Cristo, who was directing his steps towards the Rue de la Chaussée d'Antin, to M. Danglars'. The banker saw the carriage of the count enter the courtyard, and advanced to meet him with a sad, though affable smile. 

 <Well,> said he, extending his hand to Monte Cristo, <I suppose you have come to sympathize with me, for indeed misfortune has taken possession of my house. When I perceived you, I was just asking myself whether I had not wished harm towards those poor Morcerfs, which would have justified the proverb of <He who wishes misfortunes to happen to others experiences them himself.> Well, on my word of honour, I answered, <No!> I wished no ill to Morcerf; he was a little proud, perhaps, for a man who like myself has risen from nothing; but we all have our faults. Do you know, count, that persons of our time of life—not that you belong to the class, you are still a young man,—but as I was saying, persons of our time of life have been very unfortunate this year. For example, look at the puritanical procureur, who has just lost his daughter, and in fact nearly all his family, in so singular a manner; Morcerf dishonored and dead; and then myself covered with ridicule through the villany of Benedetto; besides\longdash> 

 <Besides what?> asked the Count. 

 <Alas, do you not know?> 

 <What new calamity?> 

 <My daughter\longdash> 

 <Mademoiselle Danglars?> 

 <Eugénie has left us!> 

 <Good heavens, what are you telling me?> 

 <The truth, my dear count. Oh, how happy you must be in not having either wife or children!> 

 <Do you think so?> 

 <Indeed I do.> 

 <And so Mademoiselle Danglars\longdash> 

 <She could not endure the insult offered to us by that wretch, so she asked permission to travel.> 

 <And is she gone?> 

 <The other night she left.> 

 <With Madame Danglars?> 

 <No, with a relation. But still, we have quite lost our dear Eugénie; for I doubt whether her pride will ever allow her to return to France.> 

 <Still, baron,> said Monte Cristo, <family griefs, or indeed any other affliction which would crush a man whose child was his only treasure, are endurable to a millionaire. Philosophers may well say, and practical men will always support the opinion, that money mitigates many trials; and if you admit the efficacy of this sovereign balm, you ought to be very easily consoled—you, the king of finance, the focus of immeasurable power.> 

 Danglars looked at him askance, as though to ascertain whether he spoke seriously. 

 <Yes,> he answered, <if a fortune brings consolation, I ought to be consoled; I am rich.> 

 <So rich, dear sir, that your fortune resembles the pyramids; if you wished to demolish them you could not, and if it were possible, you would not dare!> 

 Danglars smiled at the good-natured pleasantry of the count. <That reminds me,> he said, <that when you entered I was on the point of signing five little bonds; I have already signed two: will you allow me to do the same to the others?> 

 <Pray do so.> 

 There was a moment's silence, during which the noise of the banker's pen was alone heard, while Monte Cristo examined the gilt mouldings on the ceiling. 

 <Are they Spanish, Haitian, or Neapolitan bonds?> said Monte Cristo. 

 <No,> said Danglars, smiling, <they are bonds on the bank of France, payable to bearer. Stay, count,> he added, <you, who may be called the emperor, if I claim the title of king of finance, have you many pieces of paper of this size, each worth a million?> 

 The count took into his hands the papers, which Danglars had so proudly presented to him, and read:— 

\begin{mail}{}{To the Governor of the Bank}
	
Please pay to my order, from the fund deposited by me, the sum of a million, and charge the same to my account. 

\closeletter{Baron Danglars.}
\end{mail}

 <One, two, three, four, five,> said Monte Cristo; <five millions—why what a Crœsus you are!> 

 <This is how I transact business,> said Danglars. 

 <It is really wonderful,> said the count; <above all, if, as I suppose, it is payable at sight.> 

 <It is, indeed,> said Danglars. 

 <It is a fine thing to have such credit; really, it is only in France these things are done. Five millions on five little scraps of paper!—it must be seen to be believed.> 

 <You do not doubt it?> 

 <No!> 

 <You say so with an accent—stay, you shall be convinced; take my clerk to the bank, and you will see him leave it with an order on the Treasury for the same sum.> 

 <No,> said Monte Cristo folding the five notes, <most decidedly not; the thing is so curious, I will make the experiment myself. I am credited on you for six millions. I have drawn nine hundred thousand francs, you therefore still owe me five millions and a hundred thousand francs. I will take the five scraps of paper that I now hold as bonds, with your signature alone, and here is a receipt in full for the six millions between us. I had prepared it beforehand, for I am much in want of money today.> 

 And Monte Cristo placed the bonds in his pocket with one hand, while with the other he held out the receipt to Danglars. If a thunderbolt had fallen at the banker's feet, he could not have experienced greater terror. 

 <What,> he stammered, <do you mean to keep that money? Excuse me, excuse me, but I owe this money to the charity fund,—a deposit which I promised to pay this morning.> 

 <Oh, well, then,> said Monte Cristo, <I am not particular about these five notes, pay me in a different form; I wished, from curiosity, to take these, that I might be able to say that without any advice or preparation the house of Danglars had paid me five millions without a minute's delay; it would have been remarkable. But here are your bonds; pay me differently;> and he held the bonds towards Danglars, who seized them like a vulture extending its claws to withhold the food that is being wrested from its grasp. 

 Suddenly he rallied, made a violent effort to restrain himself, and then a smile gradually widened the features of his disturbed countenance.  <Certainly,> he said, <your receipt is money.> 

 <Oh dear, yes; and if you were at Rome, the house of Thomson \& French would make no more difficulty about paying the money on my receipt than you have just done.> 

 <Pardon me, count, pardon me.> 

 <Then I may keep this money?> 

 <Yes,> said Danglars, while the perspiration started from the roots of his hair. <Yes, keep it—keep it.> 

 Monte Cristo replaced the notes in his pocket with that indescribable expression which seemed to say, <Come, reflect; if you repent there is still time.> 

 <No,> said Danglars, <no, decidedly no; keep my signatures. But you know none are so formal as bankers in transacting business; I intended this money for the charity fund, and I seemed to be robbing them if I did not pay them with these precise bonds. How absurd—as if one crown were not as good as another. Excuse me;> and he began to laugh loudly, but nervously. 

 <Certainly, I excuse you,> said Monte Cristo graciously, <and pocket them.> And he placed the bonds in his pocket-book. 

 <But,> said Danglars, <there is still a sum of one hundred thousand francs?> 

 <Oh, a mere nothing,> said Monte Cristo. <The balance would come to about that sum; but keep it, and we shall be quits.> 

 <Count,> said Danglars, <are you speaking seriously?> 

 <I never joke with bankers,> said Monte Cristo in a freezing manner, which repelled impertinence; and he turned to the door, just as the valet de chambre announced: 

 <M. de Boville, Receiver-General of the charities.> 

 <\textit{Ma foi},> said Monte Cristo; <I think I arrived just in time to obtain your signatures, or they would have been disputed with me.> 

 Danglars again became pale, and hastened to conduct the count out. Monte Cristo exchanged a ceremonious bow with M. de Boville, who was standing in the waiting-room, and who was introduced into Danglars' room as soon as the count had left. 

 The count's serious face was illumined by a faint smile, as he noticed the portfolio which the receiver-general held in his hand. At the door he found his carriage, and was immediately driven to the bank. Meanwhile Danglars, repressing all emotion, advanced to meet the receiver-general. We need not say that a smile of condescension was stamped upon his lips. 

 <Good-morning, creditor,> said he; <for I wager anything it is the creditor who visits me.> 

 <You are right, baron,> answered M. de Boville; <the charities present themselves to you through me; the widows and orphans depute me to receive alms to the amount of five millions from you.> 

 <And yet they say orphans are to be pitied,> said Danglars, wishing to prolong the jest. <Poor things!> 

 <Here I am in their name,> said M. de Boville; <but did you receive my letter yesterday?> 

 <Yes.> 

 <I have brought my receipt.>

<My dear M. de Boville, your widows and orphans must oblige me by waiting twenty-four hours, since M. de Monte Cristo whom you just saw leaving here—you did see him, I think?> 

 <Yes; well?> 

 <Well, M. de Monte Cristo has just carried off their five millions.> 

 <How so?> 

 <The count has an unlimited credit upon me; a credit opened by Thomson \& French, of Rome; he came to demand five millions at once, which I paid him with checks on the bank. My funds are deposited there, and you can understand that if I draw out ten millions on the same day it will appear rather strange to the governor. Two days will be a different thing,> said Danglars, smiling. 

 <Come,> said Boville, with a tone of entire incredulity, <five millions to that gentleman who just left, and who bowed to me as though he knew me?> 

 <Perhaps he knows you, though you do not know him; M. de Monte Cristo knows everybody.> 

 <Five millions!> 

 <Here is his receipt. Believe your own eyes.> M. de Boville took the paper Danglars presented him, and read: 

 <Received of Baron Danglars the sum of five million one hundred thousand francs, to be repaid on demand by the house of Thomson \& French of Rome.> 

 <It is really true,> said M. de Boville. 

 <Do you know the house of Thomson \& French?> 

 <Yes, I once had business to transact with it to the amount of 200,000 francs; but since then I have not heard it mentioned.> 

 <It is one of the best houses in Europe,> said Danglars, carelessly throwing down the receipt on his desk. 

 <And he had five millions in your hands alone! Why, this Count of Monte Cristo must be a nabob?> 

 <Indeed I do not know what he is; he has three unlimited credits—one on me, one on Rothschild, one on Lafitte; and, you see,> he added carelessly, <he has given me the preference, by leaving a balance of 100,000 francs.> 

 M. de Boville manifested signs of extraordinary admiration. 

 <I must visit him,> he said, <and obtain some pious grant from him.> 

 <Oh, you may make sure of him; his charities alone amount to 20,000 francs a month.> 

 <It is magnificent! I will set before him the example of Madame de Morcerf and her son.> 

 <What example?> 

 <They gave all their fortune to the hospitals.> 

 <What fortune?> 

 <Their own—M. de Morcerf's, who is deceased.> 

 <For what reason?> 

 <Because they would not spend money so guiltily acquired.> 

 <And what are they to live upon?> 

 <The mother retires into the country, and the son enters the army.>

<Well, I must confess, these are scruples.> 

 <I registered their deed of gift yesterday.> 

 <And how much did they possess?> 

 <Oh, not much—from twelve to thirteen hundred thousand francs. But to return to our millions.> 

 <Certainly,> said Danglars, in the most natural tone in the world. <Are you then pressed for this money?> 

 <Yes; for the examination of our cash takes place tomorrow.> 

 <Tomorrow? Why did you not tell me so before? Why, it is as good as a century! At what hour does the examination take place?> 

 <At two o'clock.> 

 <Send at twelve,> said Danglars, smiling. 

 M. de Boville said nothing, but nodded his head, and took up the portfolio. 

 <Now I think of it, you can do better,> said Danglars. 

 <How do you mean?> 

 <The receipt of M. de Monte Cristo is as good as money; take it to Rothschild's or Lafitte's, and they will take it off your hands at once.> 

 <What, though payable at Rome?> 

 <Certainly; it will only cost you a discount of 5,000 or 6,000 francs.> 

 The receiver started back. 

 <\textit{Ma foi!}> he said, <I prefer waiting till tomorrow. What a proposition!> 

 <I thought, perhaps,> said Danglars with supreme impertinence, <that you had a deficiency to make up?> 

 <Indeed,> said the receiver. 

 <And if that were the case it would be worth while to make some sacrifice.> 

 <Thank you, no, sir.> 

 <Then it will be tomorrow.> 

 <Yes; but without fail.> 

 <Ah, you are laughing at me; send tomorrow at twelve, and the bank shall be notified.> 

 <I will come myself.> 

 <Better still, since it will afford me the pleasure of seeing you.> They shook hands. 

 <By the way,> said M. de Boville, <are you not going to the funeral of poor Mademoiselle de Villefort, which I met on my road here?> 

 <No,> said the banker; <I have appeared rather ridiculous since that affair of Benedetto, so I remain in the background.> 

 <Bah, you are wrong. How were you to blame in that affair?> 

 <Listen—when one bears an irreproachable name, as I do, one is rather sensitive.> 

 <Everybody pities you, sir; and, above all, Mademoiselle Danglars!> 

 <Poor Eugénie!> said Danglars; <do you know she is going to embrace a religious life?> 

 <No.> 

 <Alas, it is unhappily but too true. The day after the event, she decided on leaving Paris with a nun of her acquaintance; they are gone to seek a very strict convent in Italy or Spain.> 

 <Oh, it is terrible!> and M. de Boville retired with this exclamation, after expressing acute sympathy with the father. But he had scarcely left before Danglars, with an energy of action those can alone understand who have seen Robert Macaire represented by Frédérick,\footnote{Frédérick Lemaître—French actor (1800-1876). Robert Macaire is the hero of two favourite melodramas—<Chien de Montargis> and <Chien d'Aubry>—and the name is applied to bold criminals as a term of derision.} exclaimed: 

 <Fool!> 

 Then enclosing Monte Cristo's receipt in a little pocket-book, he added:—<Yes, come at twelve o'clock; I shall then be far away.> 

 Then he double-locked his door, emptied all his drawers, collected about fifty thousand francs in bank-notes, burned several papers, left others exposed to view, and then commenced writing a letter which he addressed: 

 <To Madame la Baronne Danglars.> 

 <I will place it on her table myself tonight,> he murmured. Then taking a passport from his drawer he said,—<Good, it is available for two months longer.> 