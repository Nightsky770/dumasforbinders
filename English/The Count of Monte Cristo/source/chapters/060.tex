\chapter{The Telegraph} 
	
\lettrine{M}{.} and Madame de Villefort found on their return that the Count of Monte Cristo, who had come to visit them in their absence, had been ushered into the drawing-room, and was still awaiting them there. Madame de Villefort, who had not yet sufficiently recovered from her late emotion to allow of her entertaining visitors so immediately, retired to her bedroom, while the procureur, who could better depend upon himself, proceeded at once to the salon. 

 Although M. de Villefort flattered himself that, to all outward view, he had completely masked the feelings which were passing in his mind, he did not know that the cloud was still lowering on his brow, so much so that the count, whose smile was radiant, immediately noticed his sombre and thoughtful air. 

 <\textit{Ma foi!}> said Monte Cristo, after the first compliments were over, <what is the matter with you, M. de Villefort? Have I arrived at the moment when you were drawing up an indictment for a capital crime?> 

 Villefort tried to smile. 

 <No, count,> he replied, <I am the only victim in this case. It is I who lose my cause, and it is ill-luck, obstinacy, and folly which have caused it to be decided against me.> 

 <To what do you refer?> said Monte Cristo with well-feigned interest. <Have you really met with some great misfortune?> 

 <Oh, no, monsieur,> said Villefort with a bitter smile; <it is only a loss of money which I have sustained—nothing worth mentioning, I assure you.> 

 <True,> said Monte Cristo, <the loss of a sum of money becomes almost immaterial with a fortune such as you possess, and to one of your philosophic spirit.> 

 <It is not so much the loss of the money that vexes me,> said Villefort, <though, after all, 900,000 francs are worth regretting; but I am the more annoyed with this fate, chance, or whatever you please to call the power which has destroyed my hopes and my fortune, and may blast the prospects of my child also, as it is all occasioned by an old man relapsed into second childhood.>  <What do you say?> said the count; <900,000 francs? It is indeed a sum which might be regretted even by a philosopher. And who is the cause of all this annoyance?> 

 <My father, as I told you.> 

 <M. Noirtier? But I thought you told me he had become entirely paralysed, and that all his faculties were completely destroyed?> 

 <Yes, his bodily faculties, for he can neither move nor speak, nevertheless he thinks, acts, and wills in the manner I have described. I left him about five minutes ago, and he is now occupied in dictating his will to two notaries.> 

 <But to do this he must have spoken?> 

 <He has done better than that—he has made himself understood.> 

 <How was such a thing possible?> 

 <By the help of his eyes, which are still full of life, and, as you perceive, possess the power of inflicting mortal injury.> 

 <My dear,> said Madame de Villefort, who had just entered the room, <perhaps you exaggerate the evil.> 

 <Good-morning, madame,> said the count, bowing. 

 

 Madame de Villefort acknowledged the salutation with one of her most gracious smiles. 

 <What is this that M. de Villefort has been telling me?> demanded Monte Cristo <and what incomprehensible misfortune\longdash> 

 <Incomprehensible is the word!> interrupted the procureur, shrugging his shoulders. <It is an old man's caprice!> 

 <And is there no means of making him revoke his decision?> 

 <Yes,> said Madame de Villefort; <and it is still entirely in the power of my husband to cause the will, which is now in prejudice of Valentine, to be altered in her favour.> 

 The count, who perceived that M. and Madame de Villefort were beginning to speak in parables, appeared to pay no attention to the conversation, and feigned to be busily engaged in watching Edward, who was mischievously pouring some ink into the bird's water-glass. 

 <My dear,> said Villefort, in answer to his wife, <you know I have never been accustomed to play the patriarch in my family, nor have I ever considered that the fate of a universe was to be decided by my nod. Nevertheless, it is necessary that my will should be respected in my family, and that the folly of an old man and the caprice of a child should not be allowed to overturn a project which I have entertained for so many years. The Baron d'Épinay was my friend, as you know, and an alliance with his son is the most suitable thing that could possibly be arranged.> 

 <Do you think,> said Madame de Villefort, <that Valentine is in league with him? She has always been opposed to this marriage, and I should not be at all surprised if what we have just seen and heard is nothing but the execution of a plan concerted between them.> 

 <Madame,> said Villefort, <believe me, a fortune of 900,000 francs is not so easily renounced.> 

 <She could, nevertheless, make up her mind to renounce the world, sir, since it is only about a year ago that she herself proposed entering a convent.> 

 <Never mind,> replied Villefort; <I say that this marriage \textit{shall} be consummated.> 

 <Notwithstanding your father's wishes to the contrary?> said Madame de Villefort, selecting a new point of attack. <That is a serious thing.> 

 Monte Cristo, who pretended not to be listening, heard however, every word that was said.  <Madame,> replied Villefort <I can truly say that I have always entertained a high respect for my father, because, to the natural feeling of relationship was added the consciousness of his moral superiority. The name of father is sacred in two senses; he should be reverenced as the author of our being and as a master whom we ought to obey. But, under the present circumstances, I am justified in doubting the wisdom of an old man who, because he hated the father, vents his anger on the son. It would be ridiculous in me to regulate my conduct by such caprices. I shall still continue to preserve the same respect toward M. Noirtier; I will suffer, without complaint, the pecuniary deprivation to which he has subjected me; but I shall remain firm in my determination, and the world shall see which party has reason on his side. Consequently I shall marry my daughter to the Baron Franz d'Épinay, because I consider it would be a proper and eligible match for her to make, and, in short, because I choose to bestow my daughter's hand on whomever I please.> 

 <What?> said the count, the approbation of whose eye Villefort had frequently solicited during this speech. <What? Do you say that M. Noirtier disinherits Mademoiselle de Villefort because she is going to marry M. le Baron Franz d'Épinay?> 

 <Yes, sir, that is the reason,> said Villefort, shrugging his shoulders. 

 <The apparent reason, at least,> said Madame de Villefort. 

 <The \textit{real} reason, madame, I can assure you; I know my father.> 

 <But I want to know in what way M. d'Épinay can have displeased your father more than any other person?> 

 <I believe I know M. Franz d'Épinay,> said the count; <is he not the son of General de Quesnel, who was created Baron d'Épinay by Charles X.?> 

 <The same,> said Villefort. 

 <Well, but he is a charming young man, according to my ideas.> 

 <He is, which makes me believe that it is only an excuse of M. Noirtier to prevent his granddaughter marrying; old men are always so selfish in their affection,> said Madame de Villefort. 

 <But,> said Monte Cristo <do you not know any cause for this hatred?> 

 <Ah, \textit{ma foi!} who is to know?> 

 <Perhaps it is some political difference?> 

 <My father and the Baron d'Épinay lived in the stormy times of which I only saw the ending,> said Villefort. 

 <Was not your father a Bonapartist?> asked Monte Cristo; <I think I remember that you told me something of that kind.> 

 <My father has been a Jacobin more than anything else,> said Villefort, carried by his emotion beyond the bounds of prudence; <and the senator's robe, which Napoleon cast on his shoulders, only served to disguise the old man without in any degree changing him. When my father conspired, it was not for the emperor, it was against the Bourbons; for M. Noirtier possessed this peculiarity, he never projected any Utopian schemes which could never be realized, but strove for possibilities, and he applied to the realization of these possibilities the terrible theories of The Mountain,—theories that never shrank from any means that were deemed necessary to bring about the desired result.> 

 <Well,> said Monte Cristo, <it is just as I thought; it was politics which brought Noirtier and M. d'Épinay into personal contact. Although General d'Épinay served under Napoleon, did he not still retain royalist sentiments? And was he not the person who was assassinated one evening on leaving a Bonapartist meeting to which he had been invited on the supposition that he favoured the cause of the emperor?> 

 Villefort looked at the count almost with terror. 

 <Am I mistaken, then?> said Monte Cristo. 

 <No, sir, the facts were precisely what you have stated,> said Madame de Villefort; <and it was to prevent the renewal of old feuds that M. de Villefort formed the idea of uniting in the bonds of affection the two children of these inveterate enemies.> 

 <It was a sublime and charitable thought,> said Monte Cristo, <and the whole world should applaud it. It would be noble to see Mademoiselle Noirtier de Villefort assuming the title of Madame Franz d'Épinay.> 

 Villefort shuddered and looked at Monte Cristo as if he wished to read in his countenance the real feelings which had dictated the words he had just uttered. But the count completely baffled the procureur, and prevented him from discovering anything beneath the never-varying smile he was so constantly in the habit of assuming. 

 <Although,> said Villefort, <it will be a serious thing for Valentine to lose her grandfather's fortune, I do not think that M. d'Épinay will be frightened at this pecuniary loss. He will, perhaps, hold me in greater esteem than the money itself, seeing that I sacrifice everything in order to keep my word with him. Besides, he knows that Valentine is rich in right of her mother, and that she will, in all probability, inherit the fortune of M. and Madame de Saint-Méran, her mother's parents, who both love her tenderly.> 

 <And who are fully as well worth loving and tending as M. Noirtier,> said Madame de Villefort; <besides, they are to come to Paris in about a month, and Valentine, after the affront she has received, need not consider it necessary to continue to bury herself alive by being shut up with M. Noirtier.> 

 The count listened with satisfaction to this tale of wounded self-love and defeated ambition. 

 <But it seems to me,> said Monte Cristo, <and I must begin by asking your pardon for what I am about to say, that if M. Noirtier disinherits Mademoiselle de Villefort because she is going to marry a man whose father he detested, he cannot have the same cause of complaint against this dear Edward.> 

 <True,> said Madame de Villefort, with an intonation of voice which it is impossible to describe; <is it not unjust—shamefully unjust? Poor Edward is as much M. Noirtier's grandchild as Valentine, and yet, if she had not been going to marry M. Franz, M. Noirtier would have left her all his money; and supposing Valentine to be disinherited by her grandfather, she will still be three times richer than he.> 

 The count listened and said no more. 

 <Count,> said Villefort, <we will not entertain you any longer with our family misfortunes. It is true that my patrimony will go to endow charitable institutions, and my father will have deprived me of my lawful inheritance without any reason for doing so, but I shall have the satisfaction of knowing that I have acted like a man of sense and feeling. M. d'Épinay, to whom I had promised the interest of this sum, shall receive it, even if I endure the most cruel privations.> 

 <However,> said Madame de Villefort, returning to the one idea which incessantly occupied her mind, <perhaps it would be better to explain this unlucky affair to M. d'Épinay, in order to give him the opportunity of himself renouncing his claim to the hand of Mademoiselle de Villefort.> 

 <Ah, that would be a great pity,> said Villefort. 

 <A great pity,> said Monte Cristo. 

 <Undoubtedly,> said Villefort, moderating the tones of his voice, <a marriage once concerted and then broken off, throws a sort of discredit on a young lady; then again, the old reports, which I was so anxious to put an end to, will instantly gain ground. No, it will all go well; M. d'Épinay, if he is an honourable man, will consider himself more than ever pledged to Mademoiselle de Villefort, unless he were actuated by a decided feeling of avarice, but that is impossible.> 

 <I agree with M. de Villefort,> said Monte Cristo, fixing his eyes on Madame de Villefort; <and if I were sufficiently intimate with him to allow of giving my advice, I would persuade him, since I have been told M. d'Épinay is coming back, to settle this affair at once beyond all possibility of revocation. I will answer for the success of a project which will reflect so much honour on M. de Villefort.> 

 The procureur arose, delighted with the proposition, but his wife slightly changed colour. 

 <Well, that is all that I wanted, and I will be guided by a counsellor such as you are,> said he, extending his hand to Monte Cristo. <Therefore let everyone here look upon what has passed today as if it had not happened, and as though we had never thought of such a thing as a change in our original plans.> 

 <Sir,> said the count, <the world, unjust as it is, will be pleased with your resolution; your friends will be proud of you, and M. d'Épinay, even if he took Mademoiselle de Villefort without any dowry, which he will not do, would be delighted with the idea of entering a family which could make such sacrifices in order to keep a promise and fulfil a duty.> 

 At the conclusion of these words, the count rose to depart. 

 <Are you going to leave us, count?> said Madame de Villefort. 

 <I am sorry to say I must do so, madame, I only came to remind you of your promise for Saturday.> 

 <Did you fear that we should forget it?> 

 <You are very good, madame, but M. de Villefort has so many important and urgent occupations.> 

 <My husband has given me his word, sir,> said Madame de Villefort; <you have just seen him resolve to keep it when he has everything to lose, and surely there is more reason for his doing so where he has everything to gain.> 

 <And,> said Villefort, <is it at your house in the Champs-Élysées that you receive your visitors?> 

 <No,> said Monte Cristo, <which is precisely the reason which renders your kindness more meritorious,—it is in the country.> 

 <In the country?> 

 <Yes.> 

 <Where is it, then? Near Paris, is it not?> 

 <Very near, only half a league from the Barriers,—it is at Auteuil.> 

 <At Auteuil?> said Villefort; <true, Madame de Villefort told me you lived at Auteuil, since it was to your house that she was taken. And in what part of Auteuil do you reside?> 

 <Rue de la Fontaine.> 

 <Rue de la Fontaine!> exclaimed Villefort in an agitated tone; <at what number?> 

 <№ 28.> 

 <Then,> cried Villefort, <was it you who bought M. de Saint-Méran's house!> 

 <Did it belong to M. de Saint-Méran?> demanded Monte Cristo. 

 <Yes,> replied Madame de Villefort; <and, would you believe it, count\longdash> 

 <Believe what?> 

 <You think this house pretty, do you not?> 

 <I think it charming.> 

 <Well, my husband would never live in it.> 

 <Indeed?> returned Monte Cristo, <that is a prejudice on your part, M. de Villefort, for which I am quite at a loss to account.> 

 <I do not like Auteuil, sir,> said the procureur, making an evident effort to appear calm. 

 <But I hope you will not carry your antipathy so far as to deprive me of the pleasure of your company, sir,> said Monte Cristo. 

 <No, count,—I hope—I assure you I shall do my best,> stammered Villefort. 

 <Oh,> said Monte Cristo, <I allow of no excuse. On Saturday, at six o'clock. I shall be expecting you, and if you fail to come, I shall think—for how do I know to the contrary?—that this house, which has remained uninhabited for twenty years, must have some gloomy tradition or dreadful legend connected with it.> 

 <I will come, count,—I will be sure to come,> said Villefort eagerly. 

 <Thank you,> said Monte Cristo; <now you must permit me to take my leave of you.> 

 <You said before that you were obliged to leave us, monsieur,> said Madame de Villefort, <and you were about to tell us why when your attention was called to some other subject.> 

 <Indeed madame,> said Monte Cristo: <I scarcely know if I dare tell you where I am going.> 

 <Nonsense; say on.> 

 <Well, then, it is to see a thing on which I have sometimes mused for hours together.> 

 <What is it?> 

 <A telegraph. So now I have told my secret.> 

 <A telegraph?> repeated Madame de Villefort. 

 <Yes, a telegraph. I had often seen one placed at the end of a road on a hillock, and in the light of the sun its black arms, bending in every direction, always reminded me of the claws of an immense beetle, and I assure you it was never without emotion that I gazed on it, for I could not help thinking how wonderful it was that these various signs should be made to cleave the air with such precision as to convey to the distance of three hundred leagues the ideas and wishes of a man sitting at a table at one end of the line to another man similarly placed at the opposite extremity, and all this effected by a simple act of volition on the part of the sender of the message. I began to think of genii, sylphs, gnomes, in short, of all the ministers of the occult sciences, until I laughed aloud at the freaks of my own imagination. Now, it never occurred to me to wish for a nearer inspection of these large insects, with their long black claws, for I always feared to find under their stone wings some little human genius fagged to death with cabals, factions, and government intrigues. But one fine day I learned that the mover of this telegraph was only a poor wretch, hired for twelve hundred francs a year, and employed all day, not in studying the heavens like an astronomer, or in gazing on the water like an angler, or even in enjoying the privilege of observing the country around him, but all his monotonous life was passed in watching his white-bellied, black-clawed fellow insect, four or five leagues distant from him. At length I felt a desire to study this living chrysalis more closely, and to endeavour to understand the secret part played by these insect-actors when they occupy themselves simply with pulling different pieces of string.>  <And are you going there?> 

 <I am.> 

 <What telegraph do you intend visiting? that of the home department, or of the observatory?> 

 <Oh, no; I should find there people who would force me to understand things of which I would prefer to remain ignorant, and who would try to explain to me, in spite of myself, a mystery which even they do not understand. \textit{Ma foi!} I should wish to keep my illusions concerning insects unimpaired; it is quite enough to have those dissipated which I had formed of my fellow-creatures. I shall, therefore, not visit either of these telegraphs, but one in the open country where I shall find a good-natured simpleton, who knows no more than the machine he is employed to work.> 

 <You are a singular man,> said Villefort. 

 <What line would you advise me to study?> 

 <The one that is most in use just at this time.> 

 <The Spanish one, you mean, I suppose?> 

 <Yes; should you like a letter to the minister that they might explain to you\longdash> 

 <No,> said Monte Cristo; <since, as I told you before, I do not wish to comprehend it. The moment I understand it there will no longer exist a telegraph for me; it will be nothing more than a sign from M. Duchâtel, or from M. Montalivet, transmitted to the prefect of Bayonne, mystified by two Greek words, \textit{têle}, \textit{graphein}. It is the insect with black claws, and the awful word which I wish to retain in my imagination in all its purity and all its importance.> 

 <Go then; for in the course of two hours it will be dark, and you will not be able to see anything.> 

 <\textit{Ma foi!} you frighten me. Which is the nearest way? Bayonne?> 

 <Yes; the road to Bayonne.> 

 <And afterwards the road to Châtillon?> 

 <Yes.> 

 <By the tower of Montlhéry, you mean?> 

 <Yes.> 

 <Thank you. Good-bye. On Saturday I will tell you my impressions concerning the telegraph.> 

 At the door the count was met by the two notaries, who had just completed the act which was to disinherit Valentine, and who were leaving under the conviction of having done a thing which could not fail of redounding considerably to their credit. 