\chapter{The Pardon} 

 \lettrine{T}{he} next day Danglars was again hungry; certainly the air of that dungeon was very provocative of appetite. The prisoner expected that he would be at no expense that day, for like an economical man he had concealed half of his fowl and a piece of the bread in the corner of his cell. But he had no sooner eaten than he felt thirsty; he had forgotten that. He struggled against his thirst till his tongue clave to the roof of his mouth; then, no longer able to resist, he called out. The sentinel opened the door; it was a new face. He thought it would be better to transact business with his old acquaintance, so he sent for Peppino. 

 <Here I am, your excellency,> said Peppino, with an eagerness which Danglars thought favourable to him. <What do you want?> 

 <Something to drink.> 

 <Your excellency knows that wine is beyond all price near Rome.> 

 <Then give me water,> cried Danglars, endeavouring to parry the blow. 

 <Oh, water is even more scarce than wine, your excellency,—there has been such a drought.> 

 <Come,> thought Danglars, <it is the same old story.> And while he smiled as he attempted to regard the affair as a joke, he felt his temples get moist with perspiration. 

 <Come, my friend,> said Danglars, seeing that he made no impression on Peppino, <you will not refuse me a glass of wine?> 

 <I have already told you that we do not sell at retail.> 

 <Well, then, let me have a bottle of the least expensive.> 

 <They are all the same price.> 

 <And what is that?> 

 <Twenty-five thousand francs a bottle.> 

 <Tell me,> cried Danglars, in a tone whose bitterness Harpagon\footnote{The miser in Molière's comedy of \textit{L'Avare}.—Ed. } alone has been capable of revealing—<tell me that you wish to despoil me of all; it will be sooner over than devouring me piecemeal.> 

 <It is possible such may be the master's intention.> 

 <The master?—who is he?> 

 <The person to whom you were conducted yesterday.> 

 <Where is he?> 

 <Here.> 

 <Let me see him.> 

 <Certainly.> 

 And the next moment Luigi Vampa appeared before Danglars. 

 <You sent for me?> he said to the prisoner. 

 <Are you, sir, the chief of the people who brought me here?> 

 <Yes, your excellency. What then?> 

 <How much do you require for my ransom?> 

 <Merely the 5,000,000 you have about you.> Danglars felt a dreadful spasm dart through his heart. 

 <But this is all I have left in the world,> he said, <out of an immense fortune. If you deprive me of that, take away my life also.> 

 <We are forbidden to shed your blood.> 

 <And by whom are you forbidden?> 

 <By him we obey.> 

 <You do, then, obey someone?> 

 <Yes, a chief.> 

 <I thought you said you were the chief?> 

 <So I am of these men; but there is another over me.> 

 <And did your superior order you to treat me in this way?> 

 <Yes.> 

 <But my purse will be exhausted.> 

 <Probably.> 

 <Come,> said Danglars, <will you take a million?> 

 <No.> 

 <Two millions?—three?—four? Come, four? I will give them to you on condition that you let me go.> 

 <Why do you offer me 4,000,000 for what is worth 5,000,000? This is a kind of usury, banker, that I do not understand.> 

 <Take all, then—take all, I tell you, and kill me!> 

 <Come, come, calm yourself. You will excite your blood, and that would produce an appetite it would require a million a day to satisfy. Be more economical.> 

 <But when I have no more money left to pay you?> asked the infuriated Danglars. 

 <Then you must suffer hunger.> 

 <Suffer hunger?> said Danglars, becoming pale. 

 <Most likely,> replied Vampa coolly. 

 <But you say you do not wish to kill me?> 

 <No.> 

 <And yet you will let me perish with hunger?> 

 <Ah, that is a different thing.> 

 <Well, then, wretches,> cried Danglars, <I will defy your infamous calculations—I would rather die at once! You may torture, torment, kill me, but you shall not have my signature again!> 

 <As your excellency pleases,> said Vampa, as he left the cell. 

 Danglars, raving, threw himself on the goat-skin. Who could these men be? Who was the invisible chief? What could be his intentions towards him? And why, when everyone else was allowed to be ransomed, might he not also be? Oh, yes; certainly a speedy, violent death would be a fine means of deceiving these remorseless enemies, who appeared to pursue him with such incomprehensible vengeance. But to die? For the first time in his life, Danglars contemplated death with a mixture of dread and desire; the time had come when the implacable spectre, which exists in the mind of every human creature, arrested his attention and called out with every pulsation of his heart, <Thou shalt die!> 

 Danglars resembled a timid animal excited in the chase; first it flies, then despairs, and at last, by the very force of desperation, sometimes succeeds in eluding its pursuers. Danglars meditated an escape; but the walls were solid rock, a man was sitting reading at the only outlet to the cell, and behind that man shapes armed with guns continually passed. His resolution not to sign lasted two days, after which he offered a million for some food. They sent him a magnificent supper, and took his million. 

 From this time the prisoner resolved to suffer no longer, but to have everything he wanted. At the end of twelve days, after having made a splendid dinner, he reckoned his accounts, and found that he had only 50,000 francs left. Then a strange reaction took place; he who had just abandoned 5,000,000 endeavoured to save the 50,000 francs he had left, and sooner than give them up he resolved to enter again upon a life of privation—he was deluded by the hopefulness that is a premonition of madness. 

 He, who for so long a time had forgotten God, began to think that miracles were possible—that the accursed cavern might be discovered by the officers of the Papal States, who would release him; that then he would have 50,000 remaining, which would be sufficient to save him from starvation; and finally he prayed that this sum might be preserved to him, and as he prayed he wept. Three days passed thus, during which his prayers were frequent, if not heartfelt. Sometimes he was delirious, and fancied he saw an old man stretched on a pallet; he, also, was dying of hunger. 

 On the fourth, he was no longer a man, but a living corpse. He had picked up every crumb that had been left from his former meals, and was beginning to eat the matting which covered the floor of his cell. Then he entreated Peppino, as he would a guardian angel, to give him food; he offered him 1,000 francs for a mouthful of bread. But Peppino did not answer. On the fifth day he dragged himself to the door of the cell. 

 <Are you not a Christian?> he said, falling on his knees. <Do you wish to assassinate a man who, in the eyes of Heaven, is a brother? Oh, my former friends, my former friends!> he murmured, and fell with his face to the ground. Then rising in despair, he exclaimed, <The chief, the chief!> 

 <Here I am,> said Vampa, instantly appearing; <what do you want?> 

 <Take my last gold,> muttered Danglars, holding out his pocket-book, <and let me live here; I ask no more for liberty—I only ask to live!> 

 <Then you suffer a great deal?> 

 <Oh, yes, yes, cruelly!> 

 <Still, there have been men who suffered more than you.> 

 <I do not think so.> 

 <Yes; those who have died of hunger.> 

 Danglars thought of the old man whom, in his hours of delirium, he had seen groaning on his bed. He struck his forehead on the ground and groaned. <Yes,> he said, <there have been some who have suffered more than I have, but then they must have been martyrs at least.> 

 <Do you repent?> asked a deep, solemn voice, which caused Danglars' hair to stand on end. His feeble eyes endeavoured to distinguish objects, and behind the bandit he saw a man enveloped in a cloak, half lost in the shadow of a stone column. 

 <Of what must I repent?> stammered Danglars. 

 <Of the evil you have done,> said the voice. 

 <Oh, yes; oh, yes, I do indeed repent.> And he struck his breast with his emaciated fist. 

 <Then I forgive you,> said the man, dropping his cloak, and advancing to the light. 

 <The Count of Monte Cristo!> said Danglars, more pale from terror than he had been just before from hunger and misery. 

 <You are mistaken—I am not the Count of Monte Cristo.> 

 <Then who are you?>

<I am he whom you sold and dishonored—I am he whose betrothed you prostituted—I am he upon whom you trampled that you might raise yourself to fortune—I am he whose father you condemned to die of hunger—I am he whom you also condemned to starvation, and who yet forgives you, because he hopes to be forgiven—I am Edmond Dantès!> 

 Danglars uttered a cry, and fell prostrate. 

 <Rise,> said the count, <your life is safe; the same good fortune has not happened to your accomplices—one is mad, the other dead. Keep the 50,000 francs you have left—I give them to you. The 5,000,000 you stole from the hospitals has been restored to them by an unknown hand. And now eat and drink; I will entertain you tonight. Vampa, when this man is satisfied, let him be free.> 

 Danglars remained prostrate while the count withdrew; when he raised his head he saw disappearing down the passage nothing but a shadow, before which the bandits bowed. 

 According to the count's directions, Danglars was waited on by Vampa, who brought him the best wine and fruits of Italy; then, having conducted him to the road, and pointed to the post-chaise, left him leaning against a tree. He remained there all night, not knowing where he was. When daylight dawned he saw that he was near a stream; he was thirsty, and dragged himself towards it. As he stooped down to drink, he saw that his hair had become entirely white. 