\chapter{Italy: Sinbad the Sailor} 

 \lettrine{T}{owards} the beginning of the year 1838, two young men belonging to the first society of Paris, the Viscount Albert de Morcerf and the Baron Franz d'Épinay, were at Florence. They had agreed to see the Carnival at Rome that year, and that Franz, who for the last three or four years had inhabited Italy, should act as \textit{cicerone} to Albert. 

 As it is no inconsiderable affair to spend the Carnival at Rome, especially when you have no great desire to sleep on the Piazza del Popolo, or the Campo Vaccino, they wrote to Signor Pastrini, the proprietor of the Hôtel de Londres, Piazza di Spagna, to reserve comfortable apartments for them. Signor Pastrini replied that he had only two rooms and a parlour on the third floor, which he offered at the low charge of a louis per diem. They accepted his offer; but wishing to make the best use of the time that was left, Albert started for Naples. As for Franz, he remained at Florence, and after having passed a few days in exploring the paradise of the Cascine, and spending two or three evenings at the houses of the Florentine nobility, he took a fancy into his head (having already visited Corsica, the cradle of Bonaparte) to visit Elba, the waiting-place of Napoleon. 

 One evening he cast off the painter of a sailboat from the iron ring that secured it to the dock at Leghorn, wrapped himself in his coat and lay down, and said to the crew,—<To the Island of Elba!> 

 The boat shot out of the harbour like a bird and the next morning Franz disembarked at Porto-Ferrajo. He traversed the island, after having followed the traces which the footsteps of the giant have left, and re-embarked for Marciana. 

 Two hours after he again landed at Pianosa, where he was assured that red partridges abounded. The sport was bad; Franz only succeeded in killing a few partridges, and, like every unsuccessful sportsman, he returned to the boat very much out of temper. 

 <Ah, if your excellency chose,> said the captain, <you might have capital sport.> 

 <Where?> 

 <Do you see that island?> continued the captain, pointing to a conical pile rising from the indigo sea. 

 <Well, what is this island?> 

 <The Island of Monte Cristo.> 

 <But I have no permission to shoot over this island.> 

 <Your excellency does not require a permit, for the island is uninhabited.> 

 <Ah, indeed!> said the young man. <A desert island in the midst of the Mediterranean must be a curiosity.> 

 <It is very natural; this island is a mass of rocks, and does not contain an acre of land capable of cultivation.> 

 <To whom does this island belong?> 

 <To Tuscany.> 

 <What game shall I find there!> 

 <Thousands of wild goats.> 

 <Who live upon the stones, I suppose,> said Franz with an incredulous smile. 

 <No, but by browsing the shrubs and trees that grow out of the crevices of the rocks.> 

 <Where can I sleep?> 

 <On shore in the grottos, or on board in your cloak; besides, if your excellency pleases, we can leave as soon as you like—we can sail as well by night as by day, and if the wind drops we can use our oars.> 

 As Franz had sufficient time, and his apartments at Rome were not yet available, he accepted the proposition. Upon his answer in the affirmative, the sailors exchanged a few words together in a low tone. <Well,> asked he, <what now? Is there any difficulty in the way?> 

 <No.> replied the captain, <but we must warn your excellency that the island is an infected port.> 

 <What do you mean?> 

 <Monte Cristo although uninhabited, yet serves occasionally as a refuge for the smugglers and pirates who come from Corsica, Sardinia, and Africa, and if it becomes known that we have been there, we shall have to perform quarantine for six days on our return to Leghorn.> 

 <The deuce! That puts a different face on the matter. Six days! Why, that's as long as the Almighty took to make the world! Too long a wait—too long.> 

 <But who will say your excellency has been to Monte Cristo?> 

 <Oh, I shall not,> cried Franz. 

 <Nor I, nor I,> chorused the sailors. 

 <Then steer for Monte Cristo.> 

 The captain gave his orders, the helm was put up, and the boat was soon sailing in the direction of the island. Franz waited until all was in order, and when the sail was filled, and the four sailors had taken their places—three forward, and one at the helm—he resumed the conversation. <Gaetano,> said he to the captain, <you tell me Monte Cristo serves as a refuge for pirates, who are, it seems to me, a very different kind of game from the goats.> 

 <Yes, your excellency, and it is true.> 

 <I knew there were smugglers, but I thought that since the capture of Algiers, and the destruction of the regency, pirates existed only in the romances of Cooper and Captain Marryat.> 

 <Your excellency is mistaken; there are pirates, like the bandits who were believed to have been exterminated by Pope Leo XII., and who yet, every day, rob travellers at the gates of Rome. Has not your excellency heard that the French \textit{chargé d'affaires} was robbed six months ago within five hundred paces of Velletri?> 

 <Oh, yes, I heard that.> 

 <Well, then, if, like us, your excellency lived at Leghorn, you would hear, from time to time, that a little merchant vessel, or an English yacht that was expected at Bastia, at Porto-Ferrajo, or at Civita Vecchia, has not arrived; no one knows what has become of it, but, doubtless, it has struck on a rock and foundered. Now this rock it has met has been a long and narrow boat, manned by six or eight men, who have surprised and plundered it, some dark and stormy night, near some desert and gloomy island, as bandits plunder a carriage in the recesses of a forest.> 

 <But,> asked Franz, who lay wrapped in his cloak at the bottom of the boat, <why do not those who have been plundered complain to the French, Sardinian, or Tuscan governments?> 

 <Why?> said Gaetano with a smile. 

 <Yes, why?> 

 <Because, in the first place, they transfer from the vessel to their own boat whatever they think worth taking, then they bind the crew hand and foot, they attach to everyone's neck a four-and-twenty-pound ball, a large hole is chopped in the vessel's bottom, and then they leave her. At the end of ten minutes the vessel begins to roll heavily and settle down. First one gun'l goes under, then the other. Then they lift and sink again, and both go under at once. All at once there's a noise like a cannon—that's the air blowing up the deck. Soon the water rushes out of the scupper-holes like a whale spouting, the vessel gives a last groan, spins round and round, and disappears, forming a vast whirlpool in the ocean, and then all is over, so that in five minutes nothing but the eye of God can see the vessel where she lies at the bottom of the sea. Do you understand now,> said the captain, <why no complaints are made to the government, and why the vessel never reaches port?> 

 It is probable that if Gaetano had related this previous to proposing the expedition, Franz would have hesitated, but now that they had started, he thought it would be cowardly to draw back. He was one of those men who do not rashly court danger, but if danger presents itself, combat it with the most unalterable coolness. Calm and resolute, he treated any peril as he would an adversary in a duel,—calculated its probable method of approach; retreated, if at all, as a point of strategy and not from cowardice; was quick to see an opening for attack, and won victory at a single thrust. 

 <Bah!> said he, <I have travelled through Sicily and Calabria—I have sailed two months in the Archipelago, and yet I never saw even the shadow of a bandit or a pirate.> 

 <I did not tell your excellency this to deter you from your project,> replied Gaetano, <but you questioned me, and I have answered; that's all.> 

 <Yes, and your conversation is most interesting; and as I wish to enjoy it as long as possible, steer for Monte Cristo.> 

 The wind blew strongly, the boat made six or seven knots an hour, and they were rapidly reaching the end of their voyage. As they drew near the island seemed to lift from the sea, and the air was so clear that they could already distinguish the rocks heaped on one another, like cannon balls in an arsenal, with green bushes and trees growing in the crevices. As for the sailors, although they appeared perfectly tranquil yet it was evident that they were on the alert, and that they carefully watched the glassy surface over which they were sailing, and on which a few fishing-boats, with their white sails, were alone visible. 

 They were within fifteen miles of Monte Cristo when the sun began to set behind Corsica, whose mountains appeared against the sky, showing their rugged peaks in bold relief; this mass of rock, like the giant Adamastor, rose dead ahead, a formidable barrier, and intercepting the light that gilded its massive peaks so that the voyagers were in shadow. Little by little the shadow rose higher and seemed to drive before it the last rays of the expiring day; at last the reflection rested on the summit of the mountain, where it paused an instant, like the fiery crest of a volcano, then gloom gradually covered the summit as it had covered the base, and the island now only appeared to be a gray mountain that grew continually darker; half an hour after, the night was quite dark. 

 Fortunately, the mariners were used to these latitudes, and knew every rock in the Tuscan Archipelago; for in the midst of this obscurity Franz was not without uneasiness—Corsica had long since disappeared, and Monte Cristo itself was invisible; but the sailors seemed, like the lynx, to see in the dark, and the pilot who steered did not evince the slightest hesitation. 

 An hour had passed since the sun had set, when Franz fancied he saw, at a quarter of a mile to the left, a dark mass, but he could not precisely make out what it was, and fearing to excite the mirth of the sailors by mistaking a floating cloud for land, he remained silent; suddenly a great light appeared on the strand; land might resemble a cloud, but the fire was not a meteor. 

 <What is this light?> asked he. 

 <Hush!> said the captain; <it is a fire.> 

 <But you told me the island was uninhabited?> 

 <I said there were no fixed habitations on it, but I said also that it served sometimes as a harbour for smugglers.> 

 <And for pirates?> 

 <And for pirates,> returned Gaetano, repeating Franz's words. <It is for that reason I have given orders to pass the island, for, as you see, the fire is behind us.> 

 <But this fire?> continued Franz. <It seems to me rather reassuring than otherwise; men who did not wish to be seen would not light a fire.> 

 <Oh, that goes for nothing,> said Gaetano. <If you can guess the position of the island in the darkness, you will see that the fire cannot be seen from the side or from Pianosa, but only from the sea.> 

 <You think, then, this fire indicates the presence of unpleasant neighbours?> 

 <That is what we must find out,> returned Gaetano, fixing his eyes on this terrestrial star. 

 <How can you find out?> 

 <You shall see.> 

 Gaetano consulted with his companions, and after five minutes' discussion a manœuvre was executed which caused the vessel to tack about, they returned the way they had come, and in a few minutes the fire disappeared, hidden by an elevation of the land. The pilot again changed the course of the boat, which rapidly approached the island, and was soon within fifty paces of it. Gaetano lowered the sail, and the boat came to rest. All this was done in silence, and from the moment that their course was changed not a word was spoken. 

 Gaetano, who had proposed the expedition, had taken all the responsibility on himself; the four sailors fixed their eyes on him, while they got out their oars and held themselves in readiness to row away, which, thanks to the darkness, would not be difficult. As for Franz, he examined his arms with the utmost coolness; he had two double-barrelled guns and a rifle; he loaded them, looked at the priming, and waited quietly. 

 During this time the captain had thrown off his vest and shirt, and secured his trousers round his waist; his feet were naked, so he had no shoes and stockings to take off; after these preparations he placed his finger on his lips, and lowering himself noiselessly into the sea, swam towards the shore with such precaution that it was impossible to hear the slightest sound; he could only be traced by the phosphorescent line in his wake. This track soon disappeared; it was evident that he had touched the shore. 

 Everyone on board remained motionless for half an hour, when the same luminous track was again observed, and the swimmer was soon on board. 

 <Well?> exclaimed Franz and the sailors in unison. 

 <They are Spanish smugglers,> said he; <they have with them two Corsican bandits.> 

 <And what are these Corsican bandits doing here with Spanish smugglers?> 

 <Alas,> returned the captain with an accent of the most profound pity, <we ought always to help one another. Very often the bandits are hard pressed by gendarmes or carbineers; well, they see a vessel, and good fellows like us on board, they come and demand hospitality of us; you can't refuse help to a poor hunted devil; we receive them, and for greater security we stand out to sea. This costs us nothing, and saves the life, or at least the liberty, of a fellow-creature, who on the first occasion returns the service by pointing out some safe spot where we can land our goods without interruption.> 

 <Ah!> said Franz, <then you are a smuggler occasionally, Gaetano?> 

 <Your excellency, we must live somehow,> returned the other, smiling impenetrably. 

 <Then you know the men who are now on Monte Cristo?> 

 <Oh, yes, we sailors are like freemasons, and recognize each other by signs.> 

 <And do you think we have nothing to fear if we land?> 

 <Nothing at all; smugglers are not thieves.> 

 <But these two Corsican bandits?> said Franz, calculating the chances of peril. 

 <It is not their fault that they are bandits, but that of the authorities.> 

 <How so?> 

 <Because they are pursued for having made a stiff, as if it was not in a Corsican's nature to revenge himself.> 

 <What do you mean by having made a stiff?—having assassinated a man?> said Franz, continuing his investigation. 

 <I mean that they have killed an enemy, which is a very different thing,> returned the captain. 

 <Well,> said the young man, <let us demand hospitality of these smugglers and bandits. Do you think they will grant it?> 

 <Without doubt.> 

 <How many are they?> 

 <Four, and the two bandits make six.> 

 <Just our number, so that if they prove troublesome, we shall be able to hold them in check; so, for the last time, steer to Monte Cristo.> 

 <Yes, but your excellency will permit us to take all due precautions.> 

 <By all means, be as wise as Nestor and as prudent as Ulysses; I do more than permit, I exhort you.> 

 <Silence, then!> said Gaetano. 

 Everyone obeyed. For a man who, like Franz, viewed his position in its true light, it was a grave one. He was alone in the darkness with sailors whom he did not know, and who had no reason to be devoted to him; who knew that he had several thousand francs in his belt, and who had often examined his weapons,—which were very beautiful,—if not with envy, at least with curiosity. On the other hand, he was about to land, without any other escort than these men, on an island which had, indeed, a very religious name, but which did not seem to Franz likely to afford him much hospitality, thanks to the smugglers and bandits. The history of the scuttled vessels, which had appeared improbable during the day, seemed very probable at night; placed as he was between two possible sources of danger, he kept his eye on the crew, and his gun in his hand. 

 The sailors had again hoisted sail, and the vessel was once more cleaving the waves. Through the darkness Franz, whose eyes were now more accustomed to it, could see the looming shore along which the boat was sailing, and then, as they rounded a rocky point, he saw the fire more brilliant than ever, and about it five or six persons seated. The blaze illumined the sea for a hundred paces around. Gaetano skirted the light, carefully keeping the boat in the shadow; then, when they were opposite the fire, he steered to the centre of the circle, singing a fishing song, of which his companions sung the chorus. 

 At the first words of the song the men seated round the fire arose and approached the landing-place, their eyes fixed on the boat, evidently seeking to know who the new-comers were and what were their intentions. They soon appeared satisfied and returned (with the exception of one, who remained at the shore) to their fire, at which the carcass of a goat was roasting. When the boat was within twenty paces of the shore, the man on the beach, who carried a carbine, presented arms after the manner of a sentinel, and cried, <Who comes there?> in Sardinian. 

 Franz coolly cocked both barrels. Gaetano then exchanged a few words with this man which the traveller did not understand, but which evidently concerned him. 

 <Will your excellency give your name, or remain \textit{incognito}?> asked the captain. 

 <My name must rest unknown,> replied Franz; <merely say I am a Frenchman travelling for pleasure.> 

 As soon as Gaetano had transmitted this answer, the sentinel gave an order to one of the men seated round the fire, who rose and disappeared among the rocks. Not a word was spoken, everyone seemed occupied, Franz with his disembarkment, the sailors with their sails, the smugglers with their goat; but in the midst of all this carelessness it was evident that they mutually observed each other. 

 The man who had disappeared returned suddenly on the opposite side to that by which he had left; he made a sign with his head to the sentinel, who, turning to the boat, said, <\textit{S'accommodi}.> The Italian \textit{s'accommodi} is untranslatable; it means at once, <Come, enter, you are welcome; make yourself at home; you are the master.> It is like that Turkish phrase of Molière's that so astonished the bourgeois gentleman by the number of things implied in its utterance. 

 The sailors did not wait for a second invitation; four strokes of the oar brought them to land; Gaetano sprang to shore, exchanged a few words with the sentinel, then his comrades disembarked, and lastly came Franz. One of his guns was swung over his shoulder, Gaetano had the other, and a sailor held his rifle; his dress, half artist, half dandy, did not excite any suspicion, and, consequently, no disquietude. The boat was moored to the shore, and they advanced a few paces to find a comfortable bivouac; but, doubtless, the spot they chose did not suit the smuggler who filled the post of sentinel, for he cried out: 

 <Not that way, if you please.> 

 Gaetano faltered an excuse, and advanced to the opposite side, while two sailors kindled torches at the fire to light them on their way. 

 They advanced about thirty paces, and then stopped at a small esplanade surrounded with rocks, in which seats had been cut, not unlike sentry-boxes. Around in the crevices of the rocks grew a few dwarf oaks and thick bushes of myrtles. Franz lowered a torch, and saw by the mass of cinders that had accumulated that he was not the first to discover this retreat, which was, doubtless, one of the halting-places of the wandering visitors of Monte Cristo. 

 As for his suspicions, once on \textit{terra firma}, once that he had seen the indifferent, if not friendly, appearance of his hosts, his anxiety had quite disappeared, or rather, at sight of the goat, had turned to appetite. He mentioned this to Gaetano, who replied that nothing could be more easy than to prepare a supper when they had in their boat, bread, wine, half a dozen partridges, and a good fire to roast them by. 

 <Besides,> added he, <if the smell of their roast meat tempts you, I will go and offer them two of our birds for a slice.> 

 <You are a born diplomat,> returned Franz; <go and try.> 

 Meanwhile the sailors had collected dried sticks and branches with which they made a fire. Franz waited impatiently, inhaling the aroma of the roasted meat, when the captain returned with a mysterious air. 

 <Well,> said Franz, <anything new?—do they refuse?> 

 <On the contrary,> returned Gaetano, <the chief, who was told you were a young Frenchman, invites you to sup with him.> 

 <Well,> observed Franz, <this chief is very polite, and I see no objection—the more so as I bring my share of the supper.> 

 <Oh, it is not that; he has plenty, and to spare, for supper; but he makes one condition, and rather a peculiar one, before he will receive you at his house.> 

 <His house? Has he built one here, then?> 

 <No, but he has a very comfortable one all the same, so they say.> 

 <You know this chief, then?> 

 <I have heard talk of him.> 

 <favourably or otherwise?> 

 <Both.> 

 <The deuce!—and what is this condition?> 

 <That you are blindfolded, and do not take off the bandage until he himself bids you.> 

 Franz looked at Gaetano, to see, if possible, what he thought of this proposal. <Ah,> replied he, guessing Franz's thought, <I know this is a serious matter.> 

 <What should you do in my place?> 

 <I, who have nothing to lose,—I should go.>  <You would accept?> 

 <Yes, were it only out of curiosity.> 

 <There is something very peculiar about this chief, then?> 

 <Listen,> said Gaetano, lowering his voice, <I do not know if what they say is true>—he stopped to see if anyone was near. 

 <What do they say?> 

 <That this chief inhabits a cavern to which the Pitti Palace is nothing.> 

 <What nonsense!> said Franz, reseating himself. 

 <It is no nonsense; it is quite true. Cama, the pilot of the \textit{Saint Ferdinand}, went in once, and he came back amazed, vowing that such treasures were only to be heard of in fairy tales.> 

 <Do you know,> observed Franz, <that with such stories you make me think of Ali Baba's enchanted cavern?> 

 <I tell you what I have been told.> 

 <Then you advise me to accept?> 

 <Oh, I don't say that; your excellency will do as you please; I should be sorry to advise you in the matter.> 

 Franz pondered the matter for a few moments, concluded that a man so rich could not have any intention of plundering him of what little he had, and seeing only the prospect of a good supper, accepted. Gaetano departed with the reply. Franz was prudent, and wished to learn all he possibly could concerning his host. He turned towards the sailor, who, during this dialogue, had sat gravely plucking the partridges with the air of a man proud of his office, and asked him how these men had landed, as no vessel of any kind was visible. 

 <Never mind that,> returned the sailor, <I know their vessel.> 

 <Is it a very beautiful vessel?> 

 <I would not wish for a better to sail round the world.> 

 <Of what burden is she?> 

 <About a hundred tons; but she is built to stand any weather. She is what the English call a yacht.> 

 <Where was she built?> 

 <I know not; but my own opinion is she is a Genoese.> 

 <And how did a leader of smugglers,> continued Franz, <venture to build a vessel designed for such a purpose at Genoa?> 

 <I did not say that the owner was a smuggler,> replied the sailor. 

 <No; but Gaetano did, I thought.> 

 <Gaetano had only seen the vessel from a distance, he had not then spoken to anyone.> 

 <And if this person be not a smuggler, who is he?> 

 <A wealthy signor, who travels for his pleasure.> 

 <Come,> thought Franz, <he is still more mysterious, since the two accounts do not agree.> 

 <What is his name?> 

 <If you ask him, he says Sinbad the Sailor; but I doubt if it be his real name.> 

 <Sinbad the Sailor?> 

 <Yes.> 

 <And where does he reside?> 

 <On the sea.> 

 <What country does he come from?> 

 <I do not know.> 

 <Have you ever seen him?> 

 <Sometimes.> 

 <What sort of a man is he?> 

 <Your excellency will judge for yourself.> 

 <Where will he receive me?> 

 <No doubt in the subterranean palace Gaetano told you of.> 

 <Have you never had the curiosity, when you have landed and found this island deserted, to seek for this enchanted palace?> 

 <Oh, yes, more than once, but always in vain; we examined the grotto all over, but we never could find the slightest trace of any opening; they say that the door is not opened by a key, but a magic word.> 

 <Decidedly,> muttered Franz, <this is an Arabian Nights' adventure.> 

 <His excellency waits for you,> said a voice, which he recognized as that of the sentinel. He was accompanied by two of the yacht's crew. 

 Franz drew his handkerchief from his pocket, and presented it to the man who had spoken to him. Without uttering a word, they bandaged his eyes with a care that showed their apprehensions of his committing some indiscretion. Afterwards he was made to promise that he would not make the least attempt to raise the bandage. He promised. 

 Then his two guides took his arms, and he went on, guided by them, and preceded by the sentinel. After going about thirty paces, he smelt the appetizing odor of the kid that was roasting, and knew thus that he was passing the bivouac; they then led him on about fifty paces farther, evidently advancing towards that part of the shore where they would not allow Gaetano to go—a refusal he could now comprehend. 

 Presently, by a change in the atmosphere, he knew that they were entering a cave; after going on for a few seconds more he heard a crackling, and it seemed to him as though the atmosphere again changed, and became balmy and perfumed. At length his feet touched on a thick and soft carpet, and his guides let go their hold of him. There was a moment's silence, and then a voice, in excellent French, although, with a foreign accent, said: 

 <Welcome, sir. I beg you will remove your bandage.> 

 It may be supposed, then, Franz did not wait for a repetition of this permission, but took off the handkerchief, and found himself in the presence of a man from thirty-eight to forty years of age, dressed in a Tunisian costume, that is to say, a red cap with a long blue silk tassel, a vest of black cloth embroidered with gold, pantaloons of deep red, large and full gaiters of the same colour, embroidered with gold like the vest, and yellow slippers; he had a splendid cashmere round his waist, and a small sharp and crooked cangiar was passed through his girdle. 

 Although of a paleness that was almost livid, this man had a remarkably handsome face; his eyes were penetrating and sparkling; his nose, quite straight, and projecting direct from the brow, was of the pure Greek type, while his teeth, as white as pearls, were set off to admiration by the black moustache that encircled them. 

 His pallor was so peculiar, that it seemed to pertain to one who had been long entombed, and who was incapable of resuming the healthy glow and hue of life. He was not particularly tall, but extremely well made, and, like the men of the South, had small hands and feet. But what astonished Franz, who had treated Gaetano's description as a fable, was the splendour of the apartment in which he found himself. 

 The entire chamber was lined with crimson brocade, worked with flowers of gold. In a recess was a kind of divan, surmounted with a stand of Arabian swords in silver scabbards, and the handles resplendent with gems; from the ceiling hung a lamp of Venetian glass, of beautiful shape and colour, while the feet rested on a Turkey carpet, in which they sunk to the instep; tapestry hung before the door by which Franz had entered, and also in front of another door, leading into a second apartment which seemed to be brilliantly illuminated. 

 The host gave Franz time to recover from his surprise, and, moreover, returned look for look, not even taking his eyes off him. 

 <Sir,> he said, after a pause, <a thousand excuses for the precaution taken in your introduction hither; but as, during the greater portion of the year, this island is deserted, if the secret of this abode were discovered, I should doubtless, find on my return my temporary retirement in a state of great disorder, which would be exceedingly annoying, not for the loss it occasioned me, but because I should not have the certainty I now possess of separating myself from all the rest of mankind at pleasure. Let me now endeavour to make you forget this temporary unpleasantness, and offer you what no doubt you did not expect to find here—that is to say, a tolerable supper and pretty comfortable beds.> 

 <\textit{Ma foi}, my dear sir,> replied Franz, <make no apologies. I have always observed that they bandage people's eyes who penetrate enchanted palaces, for instance, those of Raoul in the \textit{Huguenots}, and really I have nothing to complain of, for what I see makes me think of the wonders of the \textit{Arabian Nights}.>  <Alas! I may say with Lucullus, if I could have anticipated the honour of your visit, I would have prepared for it. But such as is my hermitage, it is at your disposal; such as is my supper, it is yours to share, if you will. Ali, is the supper ready?> 

 At this moment the tapestry moved aside, and a Nubian, black as ebony, and dressed in a plain white tunic, made a sign to his master that all was prepared in the dining-room. 

 <Now,> said the unknown to Franz, <I do not know if you are of my opinion, but I think nothing is more annoying than to remain two or three hours together without knowing by name or appellation how to address one another. Pray observe, that I too much respect the laws of hospitality to ask your name or title. I only request you to give me one by which I may have the pleasure of addressing you. As for myself, that I may put you at your ease, I tell you that I am generally called <Sinbad the Sailor.>> 

 <And I,> replied Franz, <will tell you, as I only require his wonderful lamp to make me precisely like Aladdin, that I see no reason why at this moment I should not be called Aladdin. That will keep us from going away from the East whither I am tempted to think I have been conveyed by some good genius.> 

 <Well, then, Signor Aladdin,> replied the singular Amphitryon, <you heard our repast announced, will you now take the trouble to enter the dining-room, your humble servant going first to show the way?> 

 At these words, moving aside the tapestry, Sinbad preceded his guest. Franz now looked upon another scene of enchantment; the table was splendidly covered, and once convinced of this important point he cast his eyes around him. The dining-room was scarcely less striking than the room he had just left; it was entirely of marble, with antique bas-reliefs of priceless value; and at the four corners of this apartment, which was oblong, were four magnificent statues, having baskets in their hands. These baskets contained four pyramids of most splendid fruit; there were Sicily pine-apples, pomegranates from Malaga, oranges from the Balearic Isles, peaches from France, and dates from Tunis. 

 The supper consisted of a roast pheasant garnished with Corsican blackbirds; a boar's ham with jelly, a quarter of a kid with tartar sauce, a glorious turbot, and a gigantic lobster. Between these large dishes were smaller ones containing various dainties. The dishes were of silver, and the plates of Japanese china. 

 Franz rubbed his eyes in order to assure himself that this was not a dream. Ali alone was present to wait at table, and acquitted himself so admirably, that the guest complimented his host thereupon. 

 <Yes,> replied he, while he did the honours of the supper with much ease and grace—<yes, he is a poor devil who is much devoted to me, and does all he can to prove it. He remembers that I saved his life, and as he has a regard for his head, he feels some gratitude towards me for having kept it on his shoulders.> 

 Ali approached his master, took his hand, and kissed it. 

 <Would it be impertinent, Signor Sinbad,> said Franz, <to ask you the particulars of this kindness?>  <Oh, they are simple enough,> replied the host. <It seems the fellow had been caught wandering nearer to the harem of the Bey of Tunis than etiquette permits to one of his colour, and he was condemned by the Bey to have his tongue cut out, and his hand and head cut off; the tongue the first day, the hand the second, and the head the third. I always had a desire to have a mute in my service, so learning the day his tongue was cut out, I went to the Bey, and proposed to give him for Ali a splendid double-barreled gun, which I knew he was very desirous of having. He hesitated a moment, he was so very desirous to complete the poor devil's punishment. But when I added to the gun an English cutlass with which I had shivered his highness's yataghan to pieces, the Bey yielded, and agreed to forgive the hand and head, but on condition that the poor fellow never again set foot in Tunis. This was a useless clause in the bargain, for whenever the coward sees the first glimpse of the shores of Africa, he runs down below, and can only be induced to appear again when we are out of sight of that quarter of the globe.> 

 Franz remained a moment silent and pensive, hardly knowing what to think of the half-kindness, half-cruelty, with which his host related the brief narrative. 

 <And like the celebrated sailor whose name you have assumed,> he said, by way of changing the conversation, <you pass your life in travelling?> 

 <Yes. I made a vow at a time when I little thought I should ever be able to accomplish it,> said the unknown with a singular smile; <and I made some others also which I hope I may fulfil in due season.> 

 Although Sinbad pronounced these words with much calmness, his eyes gave forth gleams of extraordinary ferocity. 

 <You have suffered a great deal, sir?> said Franz inquiringly. 

 Sinbad started and looked fixedly at him, as he replied, <What makes you suppose so?> 

 <Everything,> answered Franz,—<your voice, your look, your pallid complexion, and even the life you lead.> 

 <I?—I live the happiest life possible, the real life of a pasha. I am king of all creation. I am pleased with one place, and stay there; I get tired of it, and leave it; I am free as a bird and have wings like one; my attendants obey my slightest wish. Sometimes I amuse myself by delivering some bandit or criminal from the bonds of the law. Then I have my mode of dispensing justice, silent and sure, without respite or appeal, which condemns or pardons, and which no one sees. Ah, if you had tasted my life, you would not desire any other, and would never return to the world unless you had some great project to accomplish there.> 

 <Revenge, for instance!> observed Franz. 

 The unknown fixed on the young man one of those looks which penetrate into the depth of the heart and thoughts. <And why revenge?> he asked. 

 <Because,> replied Franz, <you seem to me like a man who, persecuted by society, has a fearful account to settle with it.> 

 <Ah!> responded Sinbad, laughing with his singular laugh, which displayed his white and sharp teeth. <You have not guessed rightly. Such as you see me I am, a sort of philosopher, and one day perhaps I shall go to Paris to rival Monsieur Appert, and the man in the little blue cloak.> 

 <And will that be the first time you ever took that journey?> 

 <Yes; it will. I must seem to you by no means curious, but I assure you that it is not my fault I have delayed it so long—it will happen one day or the other.>  <And do you propose to make this journey very shortly?> 

 <I do not know; it depends on circumstances which depend on certain arrangements.> 

 <I should like to be there at the time you come, and I will endeavour to repay you, as far as lies in my power, for your liberal hospitality displayed to me at Monte Cristo.> 

 <I should avail myself of your offer with pleasure,> replied the host, <but, unfortunately, if I go there, it will be, in all probability, \textit{incognito}.> 

 The supper appeared to have been supplied solely for Franz, for the unknown scarcely touched one or two dishes of the splendid banquet to which his guest did ample justice. Then Ali brought on the dessert, or rather took the baskets from the hands of the statues and placed them on the table. Between the two baskets he placed a small silver cup with a silver cover. The care with which Ali placed this cup on the table roused Franz's curiosity. He raised the cover and saw a kind of greenish paste, something like preserved angelica, but which was perfectly unknown to him. He replaced the lid, as ignorant of what the cup contained as he was before he had looked at it, and then casting his eyes towards his host he saw him smile at his disappointment. 

 <You cannot guess,> said he, <what there is in that small vase, can you?> 

 <No, I really cannot.> 

 <Well, then, that green preserve is nothing less than the ambrosia which Hebe served at the table of Jupiter.> 

 <But,> replied Franz, <this ambrosia, no doubt, in passing through mortal hands has lost its heavenly appellation and assumed a human name; in vulgar phrase, what may you term this composition, for which, to tell the truth, I do not feel any particular desire?> 

 <Ah, thus it is that our material origin is revealed,> cried Sinbad; <we frequently pass so near to happiness without seeing, without regarding it, or if we do see and regard it, yet without recognizing it. Are you a man for the substantials, and is gold your god? taste this, and the mines of Peru, Guzerat, and Golconda are opened to you. Are you a man of imagination—a poet? taste this, and the boundaries of possibility disappear; the fields of infinite space open to you, you advance free in heart, free in mind, into the boundless realms of unfettered reverie. Are you ambitious, and do you seek after the greatnesses of the earth? taste this, and in an hour you will be a king, not a king of a petty kingdom hidden in some corner of Europe like France, Spain, or England, but king of the world, king of the universe, king of creation; without bowing at the feet of Satan, you will be king and master of all the kingdoms of the earth. Is it not tempting what I offer you, and is it not an easy thing, since it is only to do thus? look!> 

 At these words he uncovered the small cup which contained the substance so lauded, took a teaspoonful of the magic sweetmeat, raised it to his lips, and swallowed it slowly with his eyes half shut and his head bent backwards. Franz did not disturb him whilst he absorbed his favourite sweetmeat, but when he had finished, he inquired: 

 <What, then, is this precious stuff?> 

 <Did you ever hear,> he replied, <of the Old Man of the Mountain, who attempted to assassinate Philippe Auguste?> 

 <Of course I have.> 

 <Well, you know he reigned over a rich valley which was overhung by the mountain whence he derived his picturesque name. In this valley were magnificent gardens planted by Hassen-ben-Sabah, and in these gardens isolated pavilions. Into these pavilions he admitted the elect, and there, says Marco Polo, gave them to eat a certain herb, which transported them to Paradise, in the midst of ever-blooming shrubs, ever-ripe fruit, and ever-lovely virgins. What these happy persons took for reality was but a dream; but it was a dream so soft, so voluptuous, so enthralling, that they sold themselves body and soul to him who gave it to them, and obedient to his orders as to those of a deity, struck down the designated victim, died in torture without a murmur, believing that the death they underwent was but a quick transition to that life of delights of which the holy herb, now before you, had given them a slight foretaste.> 

 <Then,> cried Franz, <it is hashish! I know that—by name at least.> 

 <That is it precisely, Signor Aladdin; it is hashish—the purest and most unadulterated hashish of Alexandria,—the hashish of Abou-Gor, the celebrated maker, the only man, the man to whom there should be built a palace, inscribed with these words, \textit{A grateful world to the dealer in happiness}.> 

 <Do you know,> said Franz, <I have a very great inclination to judge for myself of the truth or exaggeration of your eulogies.> 

 <Judge for yourself, Signor Aladdin—judge, but do not confine yourself to one trial. Like everything else, we must habituate the senses to a fresh impression, gentle or violent, sad or joyous. There is a struggle in nature against this divine substance,—in nature which is not made for joy and clings to pain. Nature subdued must yield in the combat, the dream must succeed to reality, and then the dream reigns supreme, then the dream becomes life, and life becomes the dream. But what changes occur! It is only by comparing the pains of actual being with the joys of the assumed existence, that you would desire to live no longer, but to dream thus forever. When you return to this mundane sphere from your visionary world, you would seem to leave a Neapolitan spring for a Lapland winter—to quit paradise for earth—heaven for hell! Taste the hashish, guest of mine—taste the hashish.> 

 Franz's only reply was to take a teaspoonful of the marvellous preparation, about as much in quantity as his host had eaten, and lift it to his mouth. 

 <\textit{Diable!}> he said, after having swallowed the divine preserve. <I do not know if the result will be as agreeable as you describe, but the thing does not appear to me as palatable as you say.> 

 <Because your palate his not yet been attuned to the sublimity of the substances it flavors. Tell me, the first time you tasted oysters, tea, porter, truffles, and sundry other dainties which you now adore, did you like them? Could you comprehend how the Romans stuffed their pheasants with assafœtida, and the Chinese eat swallows' nests? Eh? no! Well, it is the same with hashish; only eat for a week, and nothing in the world will seem to you to equal the delicacy of its flavour, which now appears to you flat and distasteful. Let us now go into the adjoining chamber, which is your apartment, and Ali will bring us coffee and pipes.> 

 They both arose, and while he who called himself Sinbad—and whom we have occasionally named so, that we might, like his guest, have some title by which to distinguish him—gave some orders to the servant, Franz entered still another apartment. 

 It was simply yet richly furnished. It was round, and a large divan completely encircled it. Divan, walls, ceiling, floor, were all covered with magnificent skins as soft and downy as the richest carpets; there were heavy-maned lion-skins from Atlas, striped tiger-skins from Bengal; panther-skins from the Cape, spotted beautifully, like those that appeared to Dante; bear-skins from Siberia, fox-skins from Norway, and so on; and all these skins were strewn in profusion one on the other, so that it seemed like walking over the most mossy turf, or reclining on the most luxurious bed. 

 Both laid themselves down on the divan; chibouques with jasmine tubes and amber mouthpieces were within reach, and all prepared so that there was no need to smoke the same pipe twice. Each of them took one, which Ali lighted and then retired to prepare the coffee. 

 There was a moment's silence, during which Sinbad gave himself up to thoughts that seemed to occupy him incessantly, even in the midst of his conversation; and Franz abandoned himself to that mute reverie, into which we always sink when smoking excellent tobacco, which seems to remove with its fume all the troubles of the mind, and to give the smoker in exchange all the visions of the soul. Ali brought in the coffee. 

 <How do you take it?> inquired the unknown; <in the French or Turkish style, strong or weak, sugar or none, cool or boiling? As you please; it is ready in all ways.> 

 <I will take it in the Turkish style,> replied Franz. 

 <And you are right,> said his host; <it shows you have a tendency for an Oriental life. Ah, those Orientals; they are the only men who know how to live. As for me,> he added, with one of those singular smiles which did not escape the young man, <when I have completed my affairs in Paris, I shall go and die in the East; and should you wish to see me again, you must seek me at Cairo, Bagdad, or Ispahan.>  <\textit{Ma foi},> said Franz, <it would be the easiest thing in the world; for I feel eagle's wings springing out at my shoulders, and with those wings I could make a tour of the world in four-and-twenty hours.> 

 <Ah, yes, the hashish is beginning its work. Well, unfurl your wings, and fly into superhuman regions; fear nothing, there is a watch over you; and if your wings, like those of Icarus, melt before the sun, we are here to ease your fall.> 

 He then said something in Arabic to Ali, who made a sign of obedience and withdrew, but not to any distance. 

 As to Franz a strange transformation had taken place in him. All the bodily fatigue of the day, all the preoccupation of mind which the events of the evening had brought on, disappeared as they do at the first approach of sleep, when we are still sufficiently conscious to be aware of the coming of slumber. His body seemed to acquire an airy lightness, his perception brightened in a remarkable manner, his senses seemed to redouble their power, the horizon continued to expand; but it was not the gloomy horizon of vague alarms, and which he had seen before he slept, but a blue, transparent, unbounded horizon, with all the blue of the ocean, all the spangles of the sun, all the perfumes of the summer breeze; then, in the midst of the songs of his sailors,—songs so clear and sonorous, that they would have made a divine harmony had their notes been taken down,—he saw the Island of Monte Cristo, no longer as a threatening rock in the midst of the waves, but as an oasis in the desert; then, as his boat drew nearer, the songs became louder, for an enchanting and mysterious harmony rose to heaven, as if some Loreley had decreed to attract a soul thither, or Amphion, the enchanter, intended there to build a city. 

 At length the boat touched the shore, but without effort, without shock, as lips touch lips; and he entered the grotto amidst continued strains of most delicious melody. He descended, or rather seemed to descend, several steps, inhaling the fresh and balmy air, like that which may be supposed to reign around the grotto of Circe, formed from such perfumes as set the mind a-dreaming, and such fires as burn the very senses; and he saw again all he had seen before his sleep, from Sinbad, his singular host, to Ali, the mute attendant; then all seemed to fade away and become confused before his eyes, like the last shadows of the magic lantern before it is extinguished, and he was again in the chamber of statues, lighted only by one of those pale and antique lamps which watch in the dead of the night over the sleep of pleasure. 

 They were the same statues, rich in form, in attraction, and poesy, with eyes of fascination, smiles of love, and bright and flowing hair. They were Phryne, Cleopatra, Messalina, those three celebrated courtesans. Then among them glided like a pure ray, like a Christian angel in the midst of Olympus, one of those chaste figures, those calm shadows, those soft visions, which seemed to veil its virgin brow before these marble wantons. 

 Then the three statues advanced towards him with looks of love, and approached the couch on which he was reposing, their feet hidden in their long white tunics, their throats bare, hair flowing like waves, and assuming attitudes which the gods could not resist, but which saints withstood, and looks inflexible and ardent like those with which the serpent charms the bird; and then he gave way before looks that held him in a torturing grasp and delighted his senses as with a voluptuous kiss. 

 It seemed to Franz that he closed his eyes, and in a last look about him saw the vision of modesty completely veiled; and then followed a dream of passion like that promised by the Prophet to the elect. Lips of stone turned to flame, breasts of ice became like heated lava, so that to Franz, yielding for the first time to the sway of the drug, love was a sorrow and voluptuousness a torture, as burning mouths were pressed to his thirsty lips, and he was held in cool serpent-like embraces. The more he strove against this unhallowed passion the more his senses yielded to its thrall, and at length, weary of a struggle that taxed his very soul, he gave way and sank back breathless and exhausted beneath the kisses of these marble goddesses, and the enchantment of his marvellous dream. 