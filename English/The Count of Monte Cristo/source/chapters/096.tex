\chapter{The Contract} 

 \lettrine{T}{hree} days after the scene we have just described, namely towards five o'clock in the afternoon of the day fixed for the signature of the contract between Mademoiselle Eugénie Danglars and Andrea Cavalcanti, whom the banker persisted in calling prince, a fresh breeze was stirring the leaves in the little garden in front of the Count of Monte Cristo's house, and the count was preparing to go out. While his horses were impatiently pawing the ground, held in by the coachman, who had been seated a quarter of an hour on his box, the elegant phaeton with which we are familiar rapidly turned the angle of the entrance-gate, and cast out on the doorsteps M. Andrea Cavalcanti, as decked up and gay as if he were going to marry a princess. 

 He inquired after the count with his usual familiarity, and ascending lightly to the first story met him at the top of the stairs. 

 The count stopped on seeing the young man. As for Andrea, he was launched, and when he was once launched nothing stopped him. 

 <Ah, good morning, my dear count,> said he. 

 <Ah, M. Andrea,> said the latter, with his half-jesting tone; <how do you do?> 

 <Charmingly, as you see. I am come to talk to you about a thousand things; but, first tell me, were you going out or just returned?> 

 <I was going out, sir.> 

 <Then, in order not to hinder you, I will get up with you if you please in your carriage, and Tom shall follow with my phaeton in tow.> 

 <No,> said the count, with an imperceptible smile of contempt, for he had no wish to be seen in the young man's society,—<no; I prefer listening to you here, my dear M. Andrea; we can chat better in-doors, and there is no coachman to overhear our conversation.> 

 The count returned to a small drawing-room on the first floor, sat down, and crossing his legs motioned to the young man to take a seat also. Andrea assumed his gayest manner. 

 <You know, my dear count,> said he, <the ceremony is to take place this evening. At nine o'clock the contract is to be signed at my father-in-law's.> 

 <Ah, indeed?> said Monte Cristo. 

 <What; is it news to you? Has not M. Danglars informed you of the ceremony?> 

 <Oh, yes,> said the count; <I received a letter from him yesterday, but I do not think the hour was mentioned.> 

 <Possibly my father-in-law trusted to its general notoriety.> 

 <Well,> said Monte Cristo, <you are fortunate, M. Cavalcanti; it is a most suitable alliance you are contracting, and Mademoiselle Danglars is a handsome girl.> 

 <Yes, indeed she is,> replied Cavalcanti, in a very modest tone. 

 <Above all, she is very rich,—at least, I believe so,> said Monte Cristo. 

 <Very rich, do you think?> replied the young man. 

 <Doubtless; it is said M. Danglars conceals at least half of his fortune.> 

 <And he acknowledges fifteen or twenty millions,> said Andrea with a look sparkling with joy. 

 <Without reckoning,> added Monte Cristo, <that he is on the eve of entering into a sort of speculation already in vogue in the United States and in England, but quite novel in France.> 

 <Yes, yes, I know what you mean,—the railway, of which he has obtained the grant, is it not?> 

 <Precisely; it is generally believed he will gain ten millions by that affair.> 

 <Ten millions! Do you think so? It is magnificent!> said Cavalcanti, who was quite confounded at the metallic sound of these golden words. 

 <Without reckoning,> replied Monte Cristo, <that all his fortune will come to you, and justly too, since Mademoiselle Danglars is an only daughter. Besides, your own fortune, as your father assured me, is almost equal to that of your betrothed. But enough of money matters. Do you know, M. Andrea, I think you have managed this affair rather skilfully?> 

 <Not badly, by any means,> said the young man; <I was born for a diplomatist.> 

 <Well, you must become a diplomatist; diplomacy, you know, is something that is not to be acquired; it is instinctive. Have you lost your heart?> 

 <Indeed, I fear it,> replied Andrea, in the tone in which he had heard Dorante or Valère reply to Alceste\footnote{In Molière's comedy, \textit{Le Misanthrope}.} at the Théâtre Français. 

 <Is your love returned?> 

 <I suppose so,> said Andrea with a triumphant smile, <since I am accepted. But I must not forget one grand point.> 

 <Which?> 

 <That I have been singularly assisted.> 

 <Nonsense.> 

 <I have, indeed.> 

 <By circumstances?> 

 <No; by you.> 

 <By me? Not at all, prince,> said Monte Cristo laying a marked stress on the title, <what have I done for you? Are not your name, your social position, and your merit sufficient?> 

 <No,> said Andrea,—<no; it is useless for you to say so, count. I maintain that the position of a man like you has done more than my name, my social position, and my merit.> 

 <You are completely mistaken, sir,> said Monte Cristo coldly, who felt the perfidious manœuvre of the young man, and understood the bearing of his words; <you only acquired my protection after the influence and fortune of your father had been ascertained; for, after all, who procured for me, who had never seen either you or your illustrious father, the pleasure of your acquaintance?—two of my good friends, Lord Wilmore and the Abbé Busoni. What encouraged me not to become your surety, but to patronize you?—your father's name, so well known in Italy and so highly honoured. Personally, I do not know you.> 

 This calm tone and perfect ease made Andrea feel that he was, for the moment, restrained by a more muscular hand than his own, and that the restraint could not be easily broken through. 

 <Oh, then my father has really a very large fortune, count?> 

 <It appears so, sir,> replied Monte Cristo. 

 <Do you know if the marriage settlement he promised me has come?> 

 <I have been advised of it.> 

 <But the three millions?> 

 <The three millions are probably on the road.> 

 <Then I shall really have them?> 

 <Oh, well,> said the count, <I do not think you have yet known the want of money.> 

 Andrea was so surprised that he pondered the matter for a moment. Then, arousing from his reverie: 

 <Now, sir, I have one request to make to you, which you will understand, even if it should be disagreeable to you.> 

 <Proceed,> said Monte Cristo. 

 <I have formed an acquaintance, thanks to my good fortune, with many noted persons, and have, at least for the moment, a crowd of friends. But marrying, as I am about to do, before all Paris, I ought to be supported by an illustrious name, and in the absence of the paternal hand some powerful one ought to lead me to the altar; now, my father is not coming to Paris, is he?> 

 <He is old, covered with wounds, and suffers dreadfully, he says, in travelling.> 

 <I understand; well, I am come to ask a favour of you.> 

 <Of me?> 

 <Yes, of you.> 

 <And pray what may it be?> 

 <Well, to take his part.> 

 <Ah, my dear sir! What?—after the varied relations I have had the happiness to sustain towards you, can it be that you know me so little as to ask such a thing? Ask me to lend you half a million and, although such a loan is somewhat rare, on my honour, you would annoy me less! Know, then, what I thought I had already told you, that in participation in this world's affairs, more especially in their moral aspects, the Count of Monte Cristo has never ceased to entertain the scruples and even the superstitions of the East. I, who have a seraglio at Cairo, one at Smyrna, and one at Constantinople, preside at a wedding?—never!> 

 <Then you refuse me?> 

 <Decidedly; and were you my son or my brother I would refuse you in the same way.> 

 <But what must be done?> said Andrea, disappointed. 

 <You said just now that you had a hundred friends.> 

 <Very true, but you introduced me at M. Danglars'.> 

 <Not at all! Let us recall the exact facts. You met him at a dinner party at my house, and you introduced yourself at his house; that is a totally different affair.> 

 <Yes, but, by my marriage, you have forwarded that.> 

 <I?—not in the least, I beg you to believe. Recollect what I told you when you asked me to propose you. <Oh, I never make matches, my dear prince, it is my settled principle.>> Andrea bit his lips.  <But, at least, you will be there?> 

 <Will all Paris be there?> 

 <Oh, certainly.> 

 <Well, like all Paris, I shall be there too,> said the count. 

 <And will you sign the contract?> 

 <I see no objection to that; my scruples do not go thus far.> 

 <Well, since you will grant me no more, I must be content with what you give me. But one word more, count.> 

 <What is it?> 

 <Advice.> 

 <Be careful; advice is worse than a service.> 

 <Oh, you can give me this without compromising yourself.> 

 <Tell me what it is.> 

 <Is my wife's fortune five hundred thousand livres?> 

 <That is the sum M. Danglars himself announced.> 

 <Must I receive it, or leave it in the hands of the notary?> 

 <This is the way such affairs are generally arranged when it is wished to do them stylishly: Your two solicitors appoint a meeting, when the contract is signed, for the next or the following day; then they exchange the two portions, for which they each give a receipt; then, when the marriage is celebrated, they place the amount at your disposal as the chief member of the alliance.> 

 <Because,> said Andrea, with a certain ill-concealed uneasiness, <I thought I heard my father-in-law say that he intended embarking our property in that famous railway affair of which you spoke just now.> 

 <Well,> replied Monte Cristo, <it will be the way, everybody says, of trebling your fortune in twelve months. Baron Danglars is a good father, and knows how to calculate.> 

 <In that case,> said Andrea, <everything is all right, excepting your refusal, which quite grieves me.> 

 <You must attribute it only to natural scruples under similar circumstances.> 

 <Well,> said Andrea, <let it be as you wish. This evening, then, at nine o'clock.> 

 <Adieu till then.> 

 Notwithstanding a slight resistance on the part of Monte Cristo, whose lips turned pale, but who preserved his ceremonious smile, Andrea seized the count's hand, pressed it, jumped into his phaeton, and disappeared. 

 The four or five remaining hours before nine o'clock arrived, Andrea employed in riding, paying visits,—designed to induce those of whom he had spoken to appear at the banker's in their gayest equipages,—dazzling them by promises of shares in schemes which have since turned every brain, and in which Danglars was just taking the initiative. 

 In fact, at half-past eight in the evening the grand salon, the gallery adjoining, and the three other drawing-rooms on the same floor, were filled with a perfumed crowd, who sympathized but little in the event, but who all participated in that love of being present wherever there is anything fresh to be seen. An Academician would say that the entertainments of the fashionable world are collections of flowers which attract inconstant butterflies, famished bees, and buzzing drones.  No one could deny that the rooms were splendidly illuminated; the light streamed forth on the gilt mouldings and the silk hangings; and all the bad taste of decorations, which had only their richness to boast of, shone in its splendour. Mademoiselle Eugénie was dressed with elegant simplicity in a figured white silk dress, and a white rose half concealed in her jet black hair was her only ornament, unaccompanied by a single jewel. Her eyes, however, betrayed that perfect confidence which contradicted the girlish simplicity of this modest attire. 

 Madame Danglars was chatting at a short distance with Debray, Beauchamp, and Château-Renaud. Debray was admitted to the house for this grand ceremony, but on the same plane with everyone else, and without any particular privilege. M. Danglars, surrounded by deputies and men connected with the revenue, was explaining a new theory of taxation which he intended to adopt when the course of events had compelled the government to call him into the ministry. Andrea, on whose arm hung one of the most consummate dandies of the Opera, was explaining to him rather cleverly, since he was obliged to be bold to appear at ease, his future projects, and the new luxuries he meant to introduce to Parisian fashions with his hundred and seventy-five thousand livres per annum. 

 The crowd moved to and fro in the rooms like an ebb and flow of turquoises, rubies, emeralds, opals, and diamonds. As usual, the oldest women were the most decorated, and the ugliest the most conspicuous. If there was a beautiful lily, or a sweet rose, you had to search for it, concealed in some corner behind a mother with a turban, or an aunt with a bird-of-paradise. 

 At each moment, in the midst of the crowd, the buzzing, and the laughter, the door-keeper's voice was heard announcing some name well known in the financial department, respected in the army, or illustrious in the literary world, and which was acknowledged by a slight movement in the different groups. But for one whose privilege it was to agitate that ocean of human waves, how many were received with a look of indifference or a sneer of disdain! 

 At the moment when the hand of the massive time-piece, representing Endymion asleep, pointed to nine on its golden face, and the hammer, the faithful type of mechanical thought, struck nine times, the name of the Count of Monte Cristo resounded in its turn, and as if by an electric shock all the assembly turned towards the door. The count was dressed in black and with his habitual simplicity; his white waistcoat displayed his expansive noble chest and his black stock was singularly noticeable because of its contrast with the deadly paleness of his face. His only jewellery was a chain, so fine that the slender gold thread was scarcely perceptible on his white waistcoat. 

 A circle was immediately formed around the door. The count perceived at one glance Madame Danglars at one end of the drawing-room, M. Danglars at the other, and Eugénie in front of him. He first advanced towards the baroness, who was chatting with Madame de Villefort, who had come alone, Valentine being still an invalid; and without turning aside, so clear was the road left for him, he passed from the baroness to Eugénie, whom he complimented in such rapid and measured terms, that the proud artist was quite struck. Near her was Mademoiselle Louise d'Armilly, who thanked the count for the letters of introduction he had so kindly given her for Italy, which she intended immediately to make use of. On leaving these ladies he found himself with Danglars, who had advanced to meet him. 

 Having accomplished these three social duties, Monte Cristo stopped, looking around him with that expression peculiar to a certain class, which seems to say, <I have done my duty, now let others do theirs.> 

 Andrea, who was in an adjoining room, had shared in the sensation caused by the arrival of Monte Cristo, and now came forward to pay his respects to the count. He found him completely surrounded; all were eager to speak to him, as is always the case with those whose words are few and weighty. The solicitors arrived at this moment and arranged their scrawled papers on the velvet cloth embroidered with gold which covered the table prepared for the signature; it was a gilt table supported on lions' claws. One of the notaries sat down, the other remained standing. They were about to proceed to the reading of the contract, which half Paris assembled was to sign. All took their places, or rather the ladies formed a circle, while the gentlemen (more indifferent to the restraints of what Boileau calls the \textit{style énergique}) commented on the feverish agitation of Andrea, on M. Danglars' riveted attention, Eugénie's composure, and the light and sprightly manner in which the baroness treated this important affair. 

 The contract was read during a profound silence. But as soon as it was finished, the buzz was redoubled through all the drawing-rooms; the brilliant sums, the rolling millions which were to be at the command of the two young people, and which crowned the display of the wedding presents and the young lady's diamonds, which had been made in a room entirely appropriated for that purpose, had exercised to the full their delusions over the envious assembly. 

 Mademoiselle Danglars' charms were heightened in the opinion of the young men, and for the moment seemed to outvie the sun in splendour. As for the ladies, it is needless to say that while they coveted the millions, they thought they did not need them for themselves, as they were beautiful enough without them. Andrea, surrounded by his friends, complimented, flattered, beginning to believe in the reality of his dream, was almost bewildered. The notary solemnly took the pen, flourished it above his head, and said: 

 <Gentlemen, we are about to sign the contract.> 

 The baron was to sign first, then the representative of M. Cavalcanti, senior, then the baroness, afterwards the <future couple,> as they are styled in the abominable phraseology of legal documents. 

 The baron took the pen and signed, then the representative. The baroness approached, leaning on Madame de Villefort's arm. 

 <My dear,> said she, as she took the pen, <is it not vexatious? An unexpected incident, in the affair of murder and theft at the Count of Monte Cristo's, in which he nearly fell a victim, deprives us of the pleasure of seeing M. de Villefort.> 

 <Indeed?> said M. Danglars, in the same tone in which he would have said, <Oh, well, what do I care?> 

 <As a matter of fact,> said Monte Cristo, approaching, <I am much afraid that I am the involuntary cause of his absence.> 

 <What, you, count?> said Madame Danglars, signing; <if you are, take care, for I shall never forgive you.> 

 Andrea pricked up his ears. 

 <But it is not my fault, as I shall endeavour to prove.> 

 Everyone listened eagerly; Monte Cristo who so rarely opened his lips, was about to speak. 

 <You remember,> said the count, during the most profound silence, <that the unhappy wretch who came to rob me died at my house; the supposition is that he was stabbed by his accomplice, on attempting to leave it.> 

 <Yes,> said Danglars. 

 <In order that his wounds might be examined he was undressed, and his clothes were thrown into a corner, where the police picked them up, with the exception of the waistcoat, which they overlooked.> 

 Andrea turned pale, and drew towards the door; he saw a cloud rising in the horizon, which appeared to forebode a coming storm. 

 <Well, this waistcoat was discovered today, covered with blood, and with a hole over the heart.> The ladies screamed, and two or three prepared to faint. <It was brought to me. No one could guess what the dirty rag could be; I alone suspected that it was the waistcoat of the murdered man. My valet, in examining this mournful relic, felt a paper in the pocket and drew it out; it was a letter addressed to you, baron.> 

 <To me?> cried Danglars. 

 <Yes, indeed, to you; I succeeded in deciphering your name under the blood with which the letter was stained,> replied Monte Cristo, amid the general outburst of amazement. 

 <But,> asked Madame Danglars, looking at her husband with uneasiness, <how could that prevent M. de Villefort\longdash> 

 <In this simple way, madame,> replied Monte Cristo; <the waistcoat and the letter were both what is termed circumstantial evidence; I therefore sent them to the king's attorney. You understand, my dear baron, that legal methods are the safest in criminal cases; it was, perhaps, some plot against you.> Andrea looked steadily at Monte Cristo and disappeared in the second drawing-room. 

 <Possibly,> said Danglars; <was not this murdered man an old galley-slave?>  <Yes,> replied the count; <a felon named Caderousse.> Danglars turned slightly pale; Andrea reached the anteroom beyond the little drawing-room. 

 <But go on signing,> said Monte Cristo; <I perceive that my story has caused a general emotion, and I beg to apologize to you, baroness, and to Mademoiselle Danglars.> 

 The baroness, who had signed, returned the pen to the notary. 

 <Prince Cavalcanti,> said the latter; <Prince Cavalcanti, where are you?> 

 <Andrea, Andrea,> repeated several young people, who were already on sufficiently intimate terms with him to call him by his Christian name. 

 <Call the prince; inform him that it is his turn to sign,> cried Danglars to one of the floorkeepers. 

 But at the same instant the crowd of guests rushed in alarm into the principal salon as if some frightful monster had entered the apartments, \textit{quærens quem devoret}. There was, indeed, reason to retreat, to be alarmed, and to scream. An officer was placing two soldiers at the door of each drawing-room, and was advancing towards Danglars, preceded by a commissary of police, girded with his scarf. Madame Danglars uttered a scream and fainted. Danglars, who thought himself threatened (certain consciences are never calm),—Danglars even before his guests showed a countenance of abject terror. 

 <What is the matter, sir?> asked Monte Cristo, advancing to meet the commissioner. 

 <Which of you gentlemen,> asked the magistrate, without replying to the count, <answers to the name of Andrea Cavalcanti?> 

 A cry of astonishment was heard from all parts of the room. They searched; they questioned. 

 <But who then is Andrea Cavalcanti?> asked Danglars in amazement. 

 <A galley-slave, escaped from confinement at Toulon.> 

 <And what crime has he committed?> 

 <He is accused,> said the commissary with his inflexible voice, <of having assassinated the man named Caderousse, his former companion in prison, at the moment he was making his escape from the house of the Count of Monte Cristo.> 

 Monte Cristo cast a rapid glance around him. Andrea was gone. 