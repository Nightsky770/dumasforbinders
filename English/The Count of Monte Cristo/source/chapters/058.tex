\chapter{M. Noirtier de Villefort}

 \lettrine{W}{e} will now relate what was passing in the house of the king's attorney after the departure of Madame Danglars and her daughter, and during the time of the conversation between Maximilian and Valentine, which we have just detailed. 

 M. de Villefort entered his father's room, followed by Madame de Villefort. Both of the visitors, after saluting the old man and speaking to Barrois, a faithful servant, who had been twenty-five years in his service, took their places on either side of the paralytic. 

 M. Noirtier was sitting in an armchair, which moved upon casters, in which he was wheeled into the room in the morning, and in the same way drawn out again at night. He was placed before a large glass, which reflected the whole apartment, and so, without any attempt to move, which would have been impossible, he could see all who entered the room and everything which was going on around him. M. Noirtier, although almost as immovable as a corpse, looked at the new-comers with a quick and intelligent expression, perceiving at once, by their ceremonious courtesy, that they were come on business of an unexpected and official character. 

 Sight and hearing were the only senses remaining, and they, like two solitary sparks, remained to animate the miserable body which seemed fit for nothing but the grave; it was only, however, by means of one of these senses that he could reveal the thoughts and feelings that still occupied his mind, and the look by which he gave expression to his inner life was like the distant gleam of a candle which a traveller sees by night across some desert place, and knows that a living being dwells beyond the silence and obscurity. 

 Noirtier's hair was long and white, and flowed over his shoulders; while in his eyes, shaded by thick black lashes, was concentrated, as it often happens with an organ which is used to the exclusion of the others, all the activity, address, force, and intelligence which were formerly diffused over his whole body; and so although the movement of the arm, the sound of the voice, and the agility of the body, were wanting, the speaking eye sufficed for all. He commanded with it; it was the medium through which his thanks were conveyed. In short, his whole appearance produced on the mind the impression of a corpse with living eyes, and nothing could be more startling than to observe the expression of anger or joy suddenly lighting up these organs, while the rest of the rigid and marble-like features were utterly deprived of the power of participation. Three persons only could understand this language of the poor paralytic; these were Villefort, Valentine, and the old servant of whom we have already spoken. But as Villefort saw his father but seldom, and then only when absolutely obliged, and as he never took any pains to please or gratify him when he was there, all the old man's happiness was centred in his granddaughter. Valentine, by means of her love, her patience, and her devotion, had learned to read in Noirtier's look all the varied feelings which were passing in his mind. To this dumb language, which was so unintelligible to others, she answered by throwing her whole soul into the expression of her countenance, and in this manner were the conversations sustained between the blooming girl and the helpless invalid, whose body could scarcely be called a living one, but who, nevertheless, possessed a fund of knowledge and penetration, united with a will as powerful as ever although clogged by a body rendered utterly incapable of obeying its impulses. 

 Valentine had solved the problem, and was able easily to understand his thoughts, and to convey her own in return, and, through her untiring and devoted assiduity, it was seldom that, in the ordinary transactions of every-day life, she failed to anticipate the wishes of the living, thinking mind, or the wants of the almost inanimate body. 

 As to the servant, he had, as we have said, been with his master for five-and-twenty years, therefore he knew all his habits, and it was seldom that Noirtier found it necessary to ask for anything, so prompt was he in administering to all the necessities of the invalid. 

 Villefort did not need the help of either Valentine or the domestic in order to carry on with his father the strange conversation which he was about to begin. As we have said, he perfectly understood the old man's vocabulary, and if he did not use it more often, it was only indifference and \textit{ennui} which prevented him from so doing. He therefore allowed Valentine to go into the garden, sent away Barrois, and after having seated himself at his father's right hand, while Madame de Villefort placed herself on the left, he addressed him thus: 

 <I trust you will not be displeased, sir, that Valentine has not come with us, or that I dismissed Barrois, for our conference will be one which could not with propriety be carried on in the presence of either. Madame de Villefort and I have a communication to make to you.> 

 Noirtier's face remained perfectly passive during this long preamble, while, on the contrary, Villefort's eye was endeavouring to penetrate into the inmost recesses of the old man's heart. 

 <This communication,> continued the procureur, in that cold and decisive tone which seemed at once to preclude all discussion, <will, we are sure, meet with your approbation.> 

 The eye of the invalid still retained that vacancy of expression which prevented his son from obtaining any knowledge of the feelings which were passing in his mind; he listened, nothing more. 

 <Sir,> resumed Villefort, <we are thinking of marrying Valentine.> Had the old man's face been moulded in wax it could not have shown less emotion at this news than was now to be traced there. <The marriage will take place in less than three months,> said Villefort. 

 Noirtier's eye still retained its inanimate expression. 

 Madame de Villefort now took her part in the conversation and added: 

 <We thought this news would possess an interest for you, sir, who have always entertained a great affection for Valentine; it therefore only now remains for us to tell you the name of the young man for whom she is destined. It is one of the most desirable connections which could possibly be formed; he possesses fortune, a high rank in society, and every personal qualification likely to render Valentine supremely happy,—his name, moreover, cannot be wholly unknown to you. It is M. Franz de Quesnel, Baron d'Épinay.> 

 While his wife was speaking, Villefort had narrowly watched the old man's countenance. When Madame de Villefort pronounced the name of Franz, the pupil of M. Noirtier's eye began to dilate, and his eyelids trembled with the same movement that may be perceived on the lips of an individual about to speak, and he darted a lightning glance at Madame de Villefort and his son. The procureur, who knew the political hatred which had formerly existed between M. Noirtier and the elder d'Épinay, well understood the agitation and anger which the announcement had produced; but, feigning not to perceive either, he immediately resumed the narrative begun by his wife. 

 <Sir,> said he, <you are aware that Valentine is about to enter her nineteenth year, which renders it important that she should lose no time in forming a suitable alliance. Nevertheless, you have not been forgotten in our plans, and we have fully ascertained beforehand that Valentine's future husband will consent, not to live in this house, for that might not be pleasant for the young people, but that you should live with them; so that you and Valentine, who are so attached to each other, would not be separated, and you would be able to pursue exactly the same course of life which you have hitherto done, and thus, instead of losing, you will be a gainer by the change, as it will secure to you two children instead of one, to watch over and comfort you.>  Noirtier's look was furious; it was very evident that something desperate was passing in the old man's mind, for a cry of anger and grief rose in his throat, and not being able to find vent in utterance, appeared almost to choke him, for his face and lips turned quite purple with the struggle. Villefort quietly opened a window, saying, <It is very warm, and the heat affects M. Noirtier.> He then returned to his place, but did not sit down. 

 <This marriage,> added Madame de Villefort, <is quite agreeable to the wishes of M. d'Épinay and his family; besides, he had no relations nearer than an uncle and aunt, his mother having died at his birth, and his father having been assassinated in 1815, that is to say, when he was but two years old; it naturally followed that the child was permitted to choose his own pursuits, and he has, therefore, seldom acknowledged any other authority but that of his own will.> 

 <That assassination was a mysterious affair,> said Villefort, <and the perpetrators have hitherto escaped detection, although suspicion has fallen on the head of more than one person.> 

 Noirtier made such an effort that his lips expanded into a smile. 

 <Now,> continued Villefort, <those to whom the guilt really belongs, by whom the crime was committed, on whose heads the justice of man may probably descend here, and the certain judgment of God hereafter, would rejoice in the opportunity thus afforded of bestowing such a peace-offering as Valentine on the son of him whose life they so ruthlessly destroyed.> Noirtier had succeeded in mastering his emotion more than could have been deemed possible with such an enfeebled and shattered frame. 

 <Yes, I understand,> was the reply contained in his look; and this look expressed a feeling of strong indignation, mixed with profound contempt. Villefort fully understood his father's meaning, and answered by a slight shrug of his shoulders. He then motioned to his wife to take leave. 

 <Now sir,> said Madame de Villefort, <I must bid you farewell. Would you like me to send Edward to you for a short time?> 

 It had been agreed that the old man should express his approbation by closing his eyes, his refusal by winking them several times, and if he had some desire or feeling to express, he raised them to heaven. If he wanted Valentine, he closed his right eye only, and if Barrois, the left. At Madame de Villefort's proposition he instantly winked his eyes. 

 Provoked by a complete refusal, she bit her lip and said, <Then shall I send Valentine to you?> The old man closed his eyes eagerly, thereby intimating that such was his wish. 

 M. and Madame de Villefort bowed and left the room, giving orders that Valentine should be summoned to her grandfather's presence, and feeling sure that she would have much to do to restore calmness to the perturbed spirit of the invalid. Valentine, with a colour still heightened by emotion, entered the room just after her parents had quitted it. One look was sufficient to tell her that her grandfather was suffering, and that there was much on his mind which he was wishing to communicate to her. 

 <Dear grandpapa,> cried she, <what has happened? They have vexed you, and you are angry?> 

 The paralytic closed his eyes in token of assent. 

 <Who has displeased you? Is it my father?> 

 <No.> 

 <Madame de Villefort?> 

 <No.> 

 <Me?> The former sign was repeated. 

 <Are you displeased with me?> cried Valentine in astonishment. M. Noirtier again closed his eyes. 

 <And what have I done, dear grandpapa, that you should be angry with me?> cried Valentine. 

 There was no answer, and she continued: 

 <I have not seen you all day. Has anyone been speaking to you against me?> 

 <Yes,> said the old man's look, with eagerness. 

 <Let me think a moment. I do assure you, grandpapa—Ah—M. and Madame de Villefort have just left this room, have they not?> 

 <Yes.> 

 <And it was they who told you something which made you angry? What was it then? May I go and ask them, that I may have the opportunity of making my peace with you?> 

 <No, no,> said Noirtier's look. 

 <Ah, you frighten me. What can they have said?> and she again tried to think what it could be. 

 <Ah, I know,> said she, lowering her voice and going close to the old man. <They have been speaking of my marriage,—have they not?> 

 <Yes,> replied the angry look. 

 <I understand; you are displeased at the silence I have preserved on the subject. The reason of it was, that they had insisted on my keeping the matter a secret, and begged me not to tell you anything of it. They did not even acquaint me with their intentions, and I only discovered them by chance, that is why I have been so reserved with you, dear grandpapa. Pray forgive me.> 

 But there was no look calculated to reassure her; all it seemed to say was, <It is not only your reserve which afflicts me.> 

 <What is it, then?> asked the young girl. <Perhaps you think I shall abandon you, dear grandpapa, and that I shall forget you when I am married?> 

 <No.> 

 <They told you, then, that M. d'Épinay consented to our all living together?> 

 <Yes.> 

 <Then why are you still vexed and grieved?> The old man's eyes beamed with an expression of gentle affection. 

 <Yes, I understand,> said Valentine; <it is because you love me.> The old man assented. 

 <And you are afraid I shall be unhappy?> 

 <Yes.> 

 <You do not like M. Franz?> The eyes repeated several times, <No, no, no.> 

 <Then you are vexed with the engagement?> 

 <Yes.> 

 <Well, listen,> said Valentine, throwing herself on her knees, and putting her arm round her grandfather's neck, <I am vexed, too, for I do not love M. Franz d'Épinay.> 

 An expression of intense joy illumined the old man's eyes. 

 <When I wished to retire into a convent, you remember how angry you were with me?> A tear trembled in the eye of the invalid. <Well,> continued Valentine, <the reason of my proposing it was that I might escape this hateful marriage, which drives me to despair.> Noirtier's breathing came thick and short. 

 <Then the idea of this marriage really grieves you too? Ah, if you could but help me—if we could both together defeat their plan! But you are unable to oppose them,—you, whose mind is so quick, and whose will is so firm are nevertheless, as weak and unequal to the contest as I am myself. Alas, you, who would have been such a powerful protector to me in the days of your health and strength, can now only sympathize in my joys and sorrows, without being able to take any active part in them. However, this is much, and calls for gratitude and Heaven has not taken away all my blessings when it leaves me your sympathy and kindness.> 

 At these words there appeared in Noirtier's eye an expression of such deep meaning that the young girl thought she could read these words there: <You are mistaken; I can still do much for you.> 

 <Do you think you can help me, dear grandpapa?> said Valentine. 

 <Yes.> Noirtier raised his eyes, it was the sign agreed on between him and Valentine when he wanted anything. 

 <What is it you want, dear grandpapa?> said Valentine, and she endeavoured to recall to mind all the things which he would be likely to need; and as the ideas presented themselves to her mind, she repeated them aloud, then,—finding that all her efforts elicited nothing but a constant \textit{<No,>}—she said, <Come, since this plan does not answer, I will have recourse to another.> 

 She then recited all the letters of the alphabet from A down to N. When she arrived at that letter the paralytic made her understand that she had spoken the initial letter of the thing he wanted. 

 <Ah,> said Valentine, <the thing you desire begins with the letter N; it is with N that we have to do, then. Well, let me see, what can you want that begins with N? Na—Ne—Ni—No\longdash> 

 <Yes, yes, yes,> said the old man's eye. 

 <Ah, it is No, then?> 

 <Yes.> 

 Valentine fetched a dictionary, which she placed on a desk before Noirtier; she opened it, and, seeing that the old man's eye was thoroughly fixed on its pages, she ran her finger quickly up and down the columns. During the six years which had passed since Noirtier first fell into this sad state, Valentine's powers of invention had been too often put to the test not to render her expert in devising expedients for gaining a knowledge of his wishes, and the constant practice had so perfected her in the art that she guessed the old man's meaning as quickly as if he himself had been able to seek for what he wanted. At the word \textit{Notary}, Noirtier made a sign to her to stop.  <Notary,> said she, <do you want a notary, dear grandpapa?> The old man again signified that it was a notary he desired. 

 <You would wish a notary to be sent for then?> said Valentine. 

 <Yes.> 

 <Shall my father be informed of your wish?> 

 <Yes.> 

 <Do you wish the notary to be sent for immediately?> 

 <Yes.> 

 <Then they shall go for him directly, dear grandpapa. Is that all you want?> 

 <Yes.> Valentine rang the bell, and ordered the servant to tell Monsieur or Madame de Villefort that they were requested to come to M. Noirtier's room. 

 <Are you satisfied now?> inquired Valentine. 

 <Yes.> 

 <I am sure you are; it is not very difficult to discover that.> And the young girl smiled on her grandfather, as if he had been a child. M. de Villefort entered, followed by Barrois. 

 <What do you want me for, sir?> demanded he of the paralytic. 

 <Sir,> said Valentine, <my grandfather wishes for a notary.> At this strange and unexpected demand M. de Villefort and his father exchanged looks. 

 <Yes,> motioned the latter, with a firmness which seemed to declare that with the help of Valentine and his old servant, who both knew what his wishes were, he was quite prepared to maintain the contest. 

 <Do you wish for a notary?> asked Villefort. 

 <Yes.> 

 <What to do?> 

 Noirtier made no answer. 

 <What do you want with a notary?> again repeated Villefort. The invalid's eye remained fixed, by which expression he intended to intimate that his resolution was unalterable. 

 <Is it to do us some ill turn? Do you think it is worth while?> said Villefort. 

 <Still,> said Barrois, with the freedom and fidelity of an old servant, <if M. Noirtier asks for a notary, I suppose he really wishes for a notary; therefore I shall go at once and fetch one.> Barrois acknowledged no master but Noirtier, and never allowed his desires in any way to be contradicted.  <Yes, I do want a notary,> motioned the old man, shutting his eyes with a look of defiance, which seemed to say, <and I should like to see the person who dares to refuse my request.> 

 <You shall have a notary, as you absolutely wish for one, sir,> said Villefort; <but I shall explain to him your state of health, and make excuses for you, for the scene cannot fail of being a most ridiculous one.> 

 <Never mind that,> said Barrois; <I shall go and fetch a notary, nevertheless.> And the old servant departed triumphantly on his mission. 