\chapter{The Secret Cave} 

 \lettrine{T}{he} sun had nearly reached the meridian, and his scorching rays fell full on the rocks, which seemed themselves sensible of the heat. Thousands of grasshoppers, hidden in the bushes, chirped with a monotonous and dull note; the leaves of the myrtle and olive trees waved and rustled in the wind. At every step that Edmond took he disturbed the lizards glittering with the hues of the emerald; afar off he saw the wild goats bounding from crag to crag. In a word, the island was inhabited, yet Edmond felt himself alone, guided by the hand of God. 

 He felt an indescribable sensation somewhat akin to dread—that dread of the daylight which even in the desert makes us fear we are watched and observed. This feeling was so strong that at the moment when Edmond was about to begin his labour, he stopped, laid down his pickaxe, seized his gun, mounted to the summit of the highest rock, and from thence gazed round in every direction. 

 But it was not upon Corsica, the very houses of which he could distinguish; or on Sardinia; or on the Island of Elba, with its historical associations; or upon the almost imperceptible line that to the experienced eye of a sailor alone revealed the coast of Genoa the proud, and Leghorn the commercial, that he gazed. It was at the brigantine that had left in the morning, and the tartan that had just set sail, that Edmond fixed his eyes. 

 The first was just disappearing in the straits of Bonifacio; the other, following an opposite direction, was about to round the Island of Corsica. 

 This sight reassured him. He then looked at the objects near him. He saw that he was on the highest point of the island,—a statue on this vast pedestal of granite, nothing human appearing in sight, while the blue ocean beat against the base of the island, and covered it with a fringe of foam. Then he descended with cautious and slow step, for he dreaded lest an accident similar to that he had so adroitly feigned should happen in reality. 

 Dantès, as we have said, had traced the marks along the rocks, and he had noticed that they led to a small creek, which was hidden like the bath of some ancient nymph. This creek was sufficiently wide at its mouth, and deep in the centre, to admit of the entrance of a small vessel of the lugger class, which would be perfectly concealed from observation. 

 Then following the clew that, in the hands of the Abbé Faria, had been so skilfully used to guide him through the Dædalian labyrinth of probabilities, he thought that the Cardinal Spada, anxious not to be watched, had entered the creek, concealed his little barque, followed the line marked by the notches in the rock, and at the end of it had buried his treasure. It was this idea that had brought Dantès back to the circular rock. One thing only perplexed Edmond, and destroyed his theory. How could this rock, which weighed several tons, have been lifted to this spot, without the aid of many men? 

 Suddenly an idea flashed across his mind. Instead of raising it, thought he, they have lowered it. And he sprang from the rock in order to inspect the base on which it had formerly stood. 

 He soon perceived that a slope had been formed, and the rock had slid along this until it stopped at the spot it now occupied. A large stone had served as a wedge; flints and pebbles had been inserted around it, so as to conceal the orifice; this species of masonry had been covered with earth, and grass and weeds had grown there, moss had clung to the stones, myrtle-bushes had taken root, and the old rock seemed fixed to the earth.  Dantès dug away the earth carefully, and detected, or fancied he detected, the ingenious artifice. He attacked this wall, cemented by the hand of time, with his pickaxe. After ten minutes' labour the wall gave way, and a hole large enough to insert the arm was opened. 

 Dantès went and cut the strongest olive-tree he could find, stripped off its branches, inserted it in the hole, and used it as a lever. But the rock was too heavy, and too firmly wedged, to be moved by anyone man, were he Hercules himself. Dantès saw that he must attack the wedge. But how? 

 He cast his eyes around, and saw the horn full of powder which his friend Jacopo had left him. He smiled; the infernal invention would serve him for this purpose. 

 With the aid of his pickaxe, Dantès, after the manner of a labour-saving pioneer, dug a mine between the upper rock and the one that supported it, filled it with powder, then made a match by rolling his handkerchief in saltpetre. He lighted it and retired. 

 The explosion soon followed; the upper rock was lifted from its base by the terrific force of the powder; the lower one flew into pieces; thousands of insects escaped from the aperture Dantès had previously formed, and a huge snake, like the guardian demon of the treasure, rolled himself along in darkening coils, and disappeared. 

 Dantès approached the upper rock, which now, without any support, leaned towards the sea. The intrepid treasure-seeker walked round it, and, selecting the spot from whence it appeared most susceptible to attack, placed his lever in one of the crevices, and strained every nerve to move the mass. 

 The rock, already shaken by the explosion, tottered on its base. Dantès redoubled his efforts; he seemed like one of the ancient Titans, who uprooted the mountains to hurl against the father of the gods. The rock yielded, rolled over, bounded from point to point, and finally disappeared in the ocean. 

 On the spot it had occupied was a circular space, exposing an iron ring let into a square flag-stone. 

 Dantès uttered a cry of joy and surprise; never had a first attempt been crowned with more perfect success. He would fain have continued, but his knees trembled, and his heart beat so violently, and his sight became so dim, that he was forced to pause. 

 This feeling lasted but for a moment. Edmond inserted his lever in the ring and exerted all his strength; the flag-stone yielded, and disclosed steps that descended until they were lost in the obscurity of a subterraneous grotto. 

 Anyone else would have rushed on with a cry of joy. Dantès turned pale, hesitated, and reflected. 

 <Come,> said he to himself, <be a man. I am accustomed to adversity. I must not be cast down by the discovery that I have been deceived. What, then, would be the use of all I have suffered? The heart breaks when, after having been elated by flattering hopes, it sees all its illusions destroyed. Faria has dreamed this; the Cardinal Spada buried no treasure here; perhaps he never came here, or if he did, Cæsar Borgia, the intrepid adventurer, the stealthy and indefatigable plunderer, has followed him, discovered his traces, pursued them as I have done, raised the stone, and descending before me, has left me nothing.> 

 He remained motionless and pensive, his eyes fixed on the gloomy aperture that was open at his feet. 

 <Now that I expect nothing, now that I no longer entertain the slightest hopes, the end of this adventure becomes simply a matter of curiosity.> And he remained again motionless and thoughtful. 

 <Yes, yes; this is an adventure worthy a place in the varied career of that royal bandit. This fabulous event formed but a link in a long chain of marvels. Yes, Borgia has been here, a torch in one hand, a sword in the other, and within twenty paces, at the foot of this rock, perhaps two guards kept watch on land and sea, while their master descended, as I am about to descend, dispelling the darkness before his awe-inspiring progress.>

<But what was the fate of the guards who thus possessed his secret?> asked Dantès of himself. 

 <The fate,> replied he, smiling, <of those who buried Alaric, and were interred with the corpse.> 

 <Yet, had he come,> thought Dantès, <he would have found the treasure, and Borgia, he who compared Italy to an artichoke, which he could devour leaf by leaf, knew too well the value of time to waste it in replacing this rock. I will go down.> 

 Then he descended, a smile on his lips, and murmuring that last word of human philosophy, <Perhaps!> 

 But instead of the darkness, and the thick and mephitic atmosphere he had expected to find, Dantès saw a dim and bluish light, which, as well as the air, entered, not merely by the aperture he had just formed, but by the interstices and crevices of the rock which were visible from without, and through which he could distinguish the blue sky and the waving branches of the evergreen oaks, and the tendrils of the creepers that grew from the rocks. 

 After having stood a few minutes in the cavern, the atmosphere of which was rather warm than damp, Dantès' eye, habituated as it was to darkness, could pierce even to the remotest angles of the cavern, which was of granite that sparkled like diamonds. 

 <Alas,> said Edmond, smiling, <these are the treasures the cardinal has left; and the good abbé, seeing in a dream these glittering walls, has indulged in fallacious hopes.> 

 But he called to mind the words of the will, which he knew by heart. <In the farthest angle of the second opening,> said the cardinal's will. He had only found the first grotto; he had now to seek the second. Dantès continued his search. He reflected that this second grotto must penetrate deeper into the island; he examined the stones, and sounded one part of the wall where he fancied the opening existed, masked for precaution's sake. 

 The pickaxe struck for a moment with a dull sound that drew out of Dantès' forehead large drops of perspiration. At last it seemed to him that one part of the wall gave forth a more hollow and deeper echo; he eagerly advanced, and with the quickness of perception that no one but a prisoner possesses, saw that there, in all probability, the opening must be. 

 However, he, like Cæsar Borgia, knew the value of time; and, in order to avoid fruitless toil, he sounded all the other walls with his pickaxe, struck the earth with the butt of his gun, and finding nothing that appeared suspicious, returned to that part of the wall whence issued the consoling sound he had before heard. 

 He again struck it, and with greater force. Then a singular thing occurred. As he struck the wall, pieces of stucco similar to that used in the ground work of arabesques broke off, and fell to the ground in flakes, exposing a large white stone. The aperture of the rock had been closed with stones, then this stucco had been applied, and painted to imitate granite. Dantès struck with the sharp end of his pickaxe, which entered someway between the interstices. 

 It was there he must dig. 

 But by some strange play of emotion, in proportion as the proofs that Faria, had not been deceived became stronger, so did his heart give way, and a feeling of discouragement stole over him. This last proof, instead of giving him fresh strength, deprived him of it; the pickaxe descended, or rather fell; he placed it on the ground, passed his hand over his brow, and remounted the stairs, alleging to himself, as an excuse, a desire to be assured that no one was watching him, but in reality because he felt that he was about to faint. 

 The island was deserted, and the sun seemed to cover it with its fiery glance; afar off, a few small fishing boats studded the bosom of the blue ocean. 

 Dantès had tasted nothing, but he thought not of hunger at such a moment; he hastily swallowed a few drops of rum, and again entered the cavern. 

 The pickaxe that had seemed so heavy, was now like a feather in his grasp; he seized it, and attacked the wall. After several blows he perceived that the stones were not cemented, but had been merely placed one upon the other, and covered with stucco; he inserted the point of his pickaxe, and using the handle as a lever, with joy soon saw the stone turn as if on hinges, and fall at his feet. 

 He had nothing more to do now, but with the iron tooth of the pickaxe to draw the stones towards him one by one. The aperture was already sufficiently large for him to enter, but by waiting, he could still cling to hope, and retard the certainty of deception. At last, after renewed hesitation, Dantès entered the second grotto. 

 The second grotto was lower and more gloomy than the first; the air that could only enter by the newly formed opening had the mephitic smell Dantès was surprised not to find in the outer cavern. He waited in order to allow pure air to displace the foul atmosphere, and then went on. 

 At the left of the opening was a dark and deep angle. But to Dantès' eye there was no darkness. He glanced around this second grotto; it was, like the first, empty. 

 The treasure, if it existed, was buried in this corner. The time had at length arrived; two feet of earth removed, and Dantès' fate would be decided. 

 He advanced towards the angle, and summoning all his resolution, attacked the ground with the pickaxe. At the fifth or sixth blow the pickaxe struck against an iron substance. Never did funeral knell, never did alarm-bell, produce a greater effect on the hearer. Had Dantès found nothing he could not have become more ghastly pale. 

 He again struck his pickaxe into the earth, and encountered the same resistance, but not the same sound. 

 <It is a casket of wood bound with iron,> thought he. 

 At this moment a shadow passed rapidly before the opening; Dantès seized his gun, sprang through the opening, and mounted the stair. A wild goat had passed before the mouth of the cave, and was feeding at a little distance. This would have been a favourable occasion to secure his dinner; but Dantès feared lest the report of his gun should attract attention. 

 He thought a moment, cut a branch of a resinous tree, lighted it at the fire at which the smugglers had prepared their breakfast, and descended with this torch. 

 He wished to see everything. He approached the hole he had dug, and now, with the aid of the torch, saw that his pickaxe had in reality struck against iron and wood. He planted his torch in the ground and resumed his labour. 

 In an instant a space three feet long by two feet broad was cleared, and Dantès could see an oaken coffer, bound with cut steel; in the middle of the lid he saw engraved on a silver plate, which was still untarnished, the arms of the Spada family—viz., a sword, \textit{en pale}, on an oval shield, like all the Italian armorial bearings, and surmounted by a cardinal's hat. 

 Dantès easily recognized them, Faria had so often drawn them for him. There was no longer any doubt: the treasure was there—no one would have been at such pains to conceal an empty casket. In an instant he had cleared every obstacle away, and he saw successively the lock, placed between two padlocks, and the two handles at each end, all carved as things were carved at that epoch, when art rendered the commonest metals precious. 

 Dantès seized the handles, and strove to lift the coffer; it was impossible. He sought to open it; lock and padlock were fastened; these faithful guardians seemed unwilling to surrender their trust. Dantès inserted the sharp end of the pickaxe between the coffer and the lid, and pressing with all his force on the handle, burst open the fastenings. The hinges yielded in their turn and fell, still holding in their grasp fragments of the wood, and the chest was open.  Edmond was seized with vertigo; he cocked his gun and laid it beside him. He then closed his eyes as children do in order that they may see in the resplendent night of their own imagination more stars than are visible in the firmament; then he re-opened them, and stood motionless with amazement. 

 Three compartments divided the coffer. In the first, blazed piles of golden coin; in the second, were ranged bars of unpolished gold, which possessed nothing attractive save their value; in the third, Edmond grasped handfuls of diamonds, pearls, and rubies, which, as they fell on one another, sounded like hail against glass. 

 After having touched, felt, examined these treasures, Edmond rushed through the caverns like a man seized with frenzy; he leaped on a rock, from whence he could behold the sea. He was alone—alone with these countless, these unheard-of treasures! Was he awake, or was it but a dream? Was it a transient vision, or was he face to face with reality? 

 He would fain have gazed upon his gold, and yet he had not strength enough; for an instant he leaned his head in his hands as if to prevent his senses from leaving him, and then rushed madly about the rocks of Monte Cristo, terrifying the wild goats and scaring the sea-fowls with his wild cries and gestures; then he returned, and, still unable to believe the evidence of his senses, rushed into the grotto, and found himself before this mine of gold and jewels. 

 This time he fell on his knees, and, clasping his hands convulsively, uttered a prayer intelligible to God alone. He soon became calmer and more happy, for only now did he begin to realize his felicity. 

 He then set himself to work to count his fortune. There were a thousand ingots of gold, each weighing from two to three pounds; then he piled up twenty-five thousand crowns, each worth about eighty francs of our money, and bearing the effigies of Alexander \textsc{vi.} and his predecessors; and he saw that the complement was not half empty. And he measured ten double handfuls of pearls, diamonds, and other gems, many of which, mounted by the most famous workmen, were valuable beyond their intrinsic worth. 

 Dantès saw the light gradually disappear, and fearing to be surprised in the cavern, left it, his gun in his hand. A piece of biscuit and a small quantity of rum formed his supper, and he snatched a few hours' sleep, lying over the mouth of the cave. 

 It was a night of joy and terror, such as this man of stupendous emotions had already experienced twice or thrice in his lifetime. 