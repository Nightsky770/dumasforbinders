\chapter{The Waking} 

 \lettrine{W}{hen} Franz returned to himself, he seemed still to be in a dream. He thought himself in a sepulchre, into which a ray of sunlight in pity scarcely penetrated. He stretched forth his hand, and touched stone; he rose to his seat, and found himself lying on his bournous in a bed of dry heather, very soft and odoriferous. The vision had fled; and as if the statues had been but shadows from the tomb, they had vanished at his waking. 

 He advanced several paces towards the point whence the light came, and to all the excitement of his dream succeeded the calmness of reality. He found that he was in a grotto, went towards the opening, and through a kind of fanlight saw a blue sea and an azure sky. The air and water were shining in the beams of the morning sun; on the shore the sailors were sitting, chatting and laughing; and at ten yards from them the boat was at anchor, undulating gracefully on the water. 

 There for some time he enjoyed the fresh breeze which played on his brow, and listened to the dash of the waves on the beach, that left against the rocks a lace of foam as white as silver. He was for some time without reflection or thought for the divine charm which is in the things of nature, specially after a fantastic dream; then gradually this view of the outer world, so calm, so pure, so grand, reminded him of the illusiveness of his vision, and once more awakened memory. He recalled his arrival on the island, his presentation to a smuggler chief, a subterranean palace full of splendour, an excellent supper, and a spoonful of hashish. 

 It seemed, however, even in the very face of open day, that at least a year had elapsed since all these things had passed, so deep was the impression made in his mind by the dream, and so strong a hold had it taken of his imagination. Thus every now and then he saw in fancy amid the sailors, seated on a rock, or undulating in the vessel, one of the shadows which had shared his dream with looks and kisses. Otherwise, his head was perfectly clear, and his body refreshed; he was free from the slightest headache; on the contrary, he felt a certain degree of lightness, a faculty for absorbing the pure air, and enjoying the bright sunshine more vividly than ever. 

 He went gayly up to the sailors, who rose as soon as they perceived him; and the patron, accosting him, said: 

 <The Signor Sinbad has left his compliments for your excellency, and desires us to express the regret he feels at not being able to take his leave in person; but he trusts you will excuse him, as very important business calls him to Malaga.> 

 <So, then, Gaetano,> said Franz, <this is, then, all reality; there exists a man who has received me in this island, entertained me right royally, and has departed while I was asleep?> 

 <He exists as certainly as that you may see his small yacht with all her sails spread; and if you will use your glass, you will, in all probability, recognize your host in the midst of his crew.> 

 So saying, Gaetano pointed in a direction in which a small vessel was making sail towards the southern point of Corsica. Franz adjusted his telescope, and directed it towards the yacht. Gaetano was not mistaken. At the stern the mysterious stranger was standing up looking towards the shore, and holding a spy-glass in his hand. He was attired as he had been on the previous evening, and waved his pocket-handkerchief to his guest in token of adieu. Franz returned the salute by shaking his handkerchief as an exchange of signals. After a second, a slight cloud of smoke was seen at the stern of the vessel, which rose gracefully as it expanded in the air, and then Franz heard a slight report. 

 <There, do you hear?> observed Gaetano; <he is bidding you adieu.> 

 The young man took his carbine and fired it in the air, but without any idea that the noise could be heard at the distance which separated the yacht from the shore. 

 <What are your excellency's orders?> inquired Gaetano. 

 <In the first place, light me a torch.> 

 <Ah, yes, I understand,> replied the patron, <to find the entrance to the enchanted apartment. With much pleasure, your excellency, if it would amuse you; and I will get you the torch you ask for. But I too have had the idea you have, and two or three times the same fancy has come over me; but I have always given it up. Giovanni, light a torch,> he added, <and give it to his excellency.> 

 Giovanni obeyed. Franz took the lamp, and entered the subterranean grotto, followed by Gaetano. He recognized the place where he had awaked by the bed of heather that was there; but it was in vain that he carried his torch all round the exterior surface of the grotto. He saw nothing, unless that, by traces of smoke, others had before him attempted the same thing, and, like him, in vain. Yet he did not leave a foot of this granite wall, as impenetrable as futurity, without strict scrutiny; he did not see a fissure without introducing the blade of his hunting sword into it, or a projecting point on which he did not lean and press in the hopes it would give way. All was vain; and he lost two hours in his attempts, which were at last utterly useless. At the end of this time he gave up his search, and Gaetano smiled. 

 When Franz appeared again on the shore, the yacht only seemed like a small white speck on the horizon. He looked again through his glass, but even then he could not distinguish anything. 

 Gaetano reminded him that he had come for the purpose of shooting goats, which he had utterly forgotten. He took his fowling-piece, and began to hunt over the island with the air of a man who is fulfilling a duty, rather than enjoying a pleasure; and at the end of a quarter of an hour he had killed a goat and two kids. These animals, though wild and agile as chamois, were too much like domestic goats, and Franz could not consider them as game. Moreover, other ideas, much more enthralling, occupied his mind. Since, the evening before, he had really been the hero of one of the tales of the \textit{Thousand and One Nights}, and he was irresistibly attracted towards the grotto. 

 Then, in spite of the failure of his first search, he began a second, after having told Gaetano to roast one of the two kids. The second visit was a long one, and when he returned the kid was roasted and the repast ready. Franz was sitting on the spot where he was on the previous evening when his mysterious host had invited him to supper; and he saw the little yacht, now like a sea-gull on the wave, continuing her flight towards Corsica. 

 <Why,> he remarked to Gaetano, <you told me that Signor Sinbad was going to Malaga, while it seems he is in the direction of Porto-Vecchio.> 

 <Don't you remember,> said the patron, <I told you that among the crew there were two Corsican brigands?> 

 <True; and he is going to land them,> added Franz. 

 <Precisely so,> replied Gaetano. <Ah, he is one who fears neither God nor Satan, they say, and would at any time run fifty leagues out of his course to do a poor devil a service.>  <But such services as these might involve him with the authorities of the country in which he practices this kind of philanthropy,> said Franz. 

 <And what cares he for that,> replied Gaetano with a laugh, <or any authorities? He smiles at them. Let them try to pursue him! Why, in the first place, his yacht is not a ship, but a bird, and he would beat any frigate three knots in every nine; and if he were to throw himself on the coast, why, is he not certain of finding friends everywhere?> 

 It was perfectly clear that the Signor Sinbad, Franz's host, had the honour of being on excellent terms with the smugglers and bandits along the whole coast of the Mediterranean, and so enjoyed exceptional privileges. As to Franz, he had no longer any inducement to remain at Monte Cristo. He had lost all hope of detecting the secret of the grotto; he consequently despatched his breakfast, and, his boat being ready, he hastened on board, and they were soon under way. At the moment the boat began her course they lost sight of the yacht, as it disappeared in the gulf of Porto-Vecchio. With it was effaced the last trace of the preceding night; and then supper, Sinbad, hashish, statues,—all became a dream for Franz. 

 The boat sailed on all day and all night, and next morning, when the sun rose, they had lost sight of Monte Cristo. 

 When Franz had once again set foot on shore, he forgot, for the moment at least, the events which had just passed, while he finished his affairs of pleasure at Florence, and then thought of nothing but how he should rejoin his companion, who was awaiting him at Rome. 

 He set out, and on the Saturday evening reached the Place de la Douane by the mail-coach. An apartment, as we have said, had been retained beforehand, and thus he had but to go to Signor Pastrini's hotel. But this was not so easy a matter, for the streets were thronged with people, and Rome was already a prey to that low and feverish murmur which precedes all great events; and at Rome there are four great events in every year,—the Carnival, Holy Week, Corpus Christi, and the Feast of St. Peter. 

 All the rest of the year the city is in that state of dull apathy, between life and death, which renders it similar to a kind of station between this world and the next—a sublime spot, a resting-place full of poetry and character, and at which Franz had already halted five or six times, and at each time found it more marvellous and striking. 

 At last he made his way through the mob, which was continually increasing and getting more and more turbulent, and reached the hotel. On his first inquiry he was told, with the impertinence peculiar to hired hackney-coachmen and innkeepers with their houses full, that there was no room for him at the Hôtel de Londres. Then he sent his card to Signor Pastrini, and asked for Albert de Morcerf. This plan succeeded; and Signor Pastrini himself ran to him, excusing himself for having made his excellency wait, scolding the waiters, taking the candlestick from the porter, who was ready to pounce on the traveller and was about to lead him to Albert, when Morcerf himself appeared. 

 The apartment consisted of two small rooms and a parlour. The two rooms looked on to the street—a fact which Signor Pastrini commented upon as an inappreciable advantage. The rest of the floor was hired by a very rich gentleman who was supposed to be a Sicilian or Maltese; but the host was unable to decide to which of the two nations the traveller belonged. 

 <Very good, signor Pastrini,> said Franz; <but we must have some supper instantly, and a carriage for tomorrow and the following days.> 

 <As to supper,> replied the landlord, <you shall be served immediately; but as for the carriage\longdash> 

 <What as to the carriage?> exclaimed Albert. <Come, come, Signor Pastrini, no joking; we must have a carriage.> 

 <Sir,> replied the host, <we will do all in our power to procure you one—this is all I can say.> 

 <And when shall we know?> inquired Franz. 

 <Tomorrow morning,> answered the innkeeper. 

 <Oh, the deuce! then we shall pay the more, that's all, I see plainly enough. At Drake's or Aaron's one pays twenty-five lire for common days, and thirty or thirty-five lire a day more for Sundays and feast days; add five lire a day more for extras, that will make forty, and there's an end of it.> 

 <I am afraid if we offer them double that we shall not procure a carriage.> 

 <Then they must put horses to mine. It is a little worse for the journey, but that's no matter.> 

 <There are no horses.> 

 Albert looked at Franz like a man who hears a reply he does not understand. 

 <Do you understand that, my dear Franz—no horses?> he said, <but can't we have post-horses?> 

 <They have been all hired this fortnight, and there are none left but those absolutely requisite for posting.> 

 <What are we to say to this?> asked Franz. 

 <I say, that when a thing completely surpasses my comprehension, I am accustomed not to dwell on that thing, but to pass to another. Is supper ready, Signor Pastrini?> 

 <Yes, your excellency.> 

 <Well, then, let us sup.> 

 <But the carriage and horses?> said Franz. 

 <Be easy, my dear boy; they will come in due season; it is only a question of how much shall be charged for them.> Morcerf then, with that delighted philosophy which believes that nothing is impossible to a full purse or well-lined pocketbook, supped, went to bed, slept soundly, and dreamed he was racing all over Rome at Carnival time in a coach with six horses. 