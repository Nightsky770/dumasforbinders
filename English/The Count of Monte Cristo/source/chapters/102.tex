\chapter{Valentine} 

 \lettrine{T}{he} night-light continued to burn on the chimney-piece, exhausting the last drops of oil which floated on the surface of the water. The globe of the lamp appeared of a reddish hue, and the flame, brightening before it expired, threw out the last flickerings which in an inanimate object have been so often compared with the convulsions of a human creature in its final agonies. A dull and dismal light was shed over the bedclothes and curtains surrounding the young girl. All noise in the streets had ceased, and the silence was frightful. 

 It was then that the door of Edward's room opened, and a head we have before noticed appeared in the glass opposite; it was Madame de Villefort, who came to witness the effects of the drink she had prepared. She stopped in the doorway, listened for a moment to the flickering of the lamp, the only sound in that deserted room, and then advanced to the table to see if Valentine's glass were empty. It was still about a quarter full, as we before stated. Madame de Villefort emptied the contents into the ashes, which she disturbed that they might the more readily absorb the liquid; then she carefully rinsed the glass, and wiping it with her handkerchief replaced it on the table. 

 If anyone could have looked into the room just then he would have noticed the hesitation with which Madame de Villefort approached the bed and looked fixedly on Valentine. The dim light, the profound silence, and the gloomy thoughts inspired by the hour, and still more by her own conscience, all combined to produce a sensation of fear; the poisoner was terrified at the contemplation of her own work. 

 At length she rallied, drew aside the curtain, and leaning over the pillow gazed intently on Valentine. The young girl no longer breathed, no breath issued through the half-closed teeth; the white lips no longer quivered—the eyes were suffused with a bluish vapor, and the long black lashes rested on a cheek white as wax. Madame de Villefort gazed upon the face so expressive even in its stillness; then she ventured to raise the coverlet and press her hand upon the young girl's heart. It was cold and motionless. She only felt the pulsation in her own fingers, and withdrew her hand with a shudder. One arm was hanging out of the bed; from shoulder to elbow it was moulded after the arms of Germain Pillon's <Graces,>\footnote{Germain Pillon was a famous French sculptor (1535-1598). His best known work is <The Three Graces,> now in the Louvre.} but the fore-arm seemed to be slightly distorted by convulsion, and the hand, so delicately formed, was resting with stiff outstretched fingers on the framework of the bed. The nails, too, were turning blue. 

 Madame de Villefort had no longer any doubt; all was over—she had consummated the last terrible work she had to accomplish. There was no more to do in the room, so the poisoner retired stealthily, as though fearing to hear the sound of her own footsteps; but as she withdrew she still held aside the curtain, absorbed in the irresistible attraction always exerted by the picture of death, so long as it is merely mysterious and does not excite disgust. 

 The minutes passed; Madame de Villefort could not drop the curtain which she held like a funeral pall over the head of Valentine. She was lost in reverie, and the reverie of crime is remorse. 

 Just then the lamp again flickered; the noise startled Madame de Villefort, who shuddered and dropped the curtain. Immediately afterwards the light expired, and the room was plunged in frightful obscurity, while the clock at that minute struck half-past four. 

 Overpowered with agitation, the poisoner succeeded in groping her way to the door, and reached her room in an agony of fear. The darkness lasted two hours longer; then by degrees a cold light crept through the Venetian blinds, until at length it revealed the objects in the room. 

 About this time the nurse's cough was heard on the stairs and the woman entered the room with a cup in her hand. To the tender eye of a father or a lover, the first glance would have sufficed to reveal Valentine's condition; but to this hireling, Valentine only appeared to sleep. 

 <Good,> she exclaimed, approaching the table, <she has taken part of her draught; the glass is three-quarters empty.> 

 Then she went to the fireplace and lit the fire, and although she had just left her bed, she could not resist the temptation offered by Valentine's sleep, so she threw herself into an armchair to snatch a little more rest. The clock striking eight awoke her. Astonished at the prolonged slumber of the patient, and frightened to see that the arm was still hanging out of the bed, she advanced towards Valentine, and for the first time noticed the white lips. She tried to replace the arm, but it moved with a frightful rigidity which could not deceive a sick-nurse. She screamed aloud; then running to the door exclaimed: 

 <Help, help!>  <What is the matter?> asked M. d'Avrigny, at the foot of the stairs, it being the hour he usually visited her. 

 <What is it?> asked Villefort, rushing from his room. <Doctor, do you hear them call for help?> 

 <Yes, yes; let us hasten up; it was in Valentine's room.> 

 But before the doctor and the father could reach the room, the servants who were on the same floor had entered, and seeing Valentine pale and motionless on her bed, they lifted up their hands towards heaven and stood transfixed, as though struck by lightening. 

 <Call Madame de Villefort!—Wake Madame de Villefort!> cried the procureur from the door of his chamber, which apparently he scarcely dared to leave. But instead of obeying him, the servants stood watching M. d'Avrigny, who ran to Valentine, and raised her in his arms. 

 <What?—this one, too?> he exclaimed. <Oh, where will be the end?> 

 Villefort rushed into the room. 

 <What are you saying, doctor?> he exclaimed, raising his hands to heaven. 

 <I say that Valentine is dead!> replied d'Avrigny, in a voice terrible in its solemn calmness.  M. de Villefort staggered and buried his head in the bed. On the exclamation of the doctor and the cry of the father, the servants all fled with muttered imprecations; they were heard running down the stairs and through the long passages, then there was a rush in the court, afterwards all was still; they had, one and all, deserted the accursed house. 

 Just then, Madame de Villefort, in the act of slipping on her dressing-gown, threw aside the drapery and for a moment stood motionless, as though interrogating the occupants of the room, while she endeavoured to call up some rebellious tears. On a sudden she stepped, or rather bounded, with outstretched arms, towards the table. She saw d'Avrigny curiously examining the glass, which she felt certain of having emptied during the night. It was now a third full, just as it was when she threw the contents into the ashes. The spectre of Valentine rising before the poisoner would have alarmed her less. It was, indeed, the same colour as the draught she had poured into the glass, and which Valentine had drunk; it was indeed the poison, which could not deceive M. d'Avrigny, which he now examined so closely; it was doubtless a miracle from heaven, that, notwithstanding her precautions, there should be some trace, some proof remaining to reveal the crime. 

 While Madame de Villefort remained rooted to the spot like a statue of terror, and Villefort, with his head hidden in the bedclothes, saw nothing around him, d'Avrigny approached the window, that he might the better examine the contents of the glass, and dipping the tip of his finger in, tasted it. 

 <Ah,> he exclaimed, <it is no longer brucine that is used; let me see what it is!> 

 Then he ran to one of the cupboards in Valentine's room, which had been transformed into a medicine closet, and taking from its silver case a small bottle of nitric acid, dropped a little of it into the liquor, which immediately changed to a blood-red colour. 

 <Ah,> exclaimed d'Avrigny, in a voice in which the horror of a judge unveiling the truth was mingled with the delight of a student making a discovery. 

 Madame de Villefort was overpowered; her eyes first flashed and then swam, she staggered towards the door and disappeared. Directly afterwards the distant sound of a heavy weight falling on the ground was heard, but no one paid any attention to it; the nurse was engaged in watching the chemical analysis, and Villefort was still absorbed in grief. M. d'Avrigny alone had followed Madame de Villefort with his eyes, and watched her hurried retreat. He lifted up the drapery over the entrance to Edward's room, and his eye reaching as far as Madame de Villefort's apartment, he beheld her extended lifeless on the floor. 

 <Go to the assistance of Madame de Villefort,> he said to the nurse. <Madame de Villefort is ill.>  <But Mademoiselle de Villefort\longdash> stammered the nurse. 

 <Mademoiselle de Villefort no longer requires help,> said d'Avrigny, <since she is dead.> 

 <Dead,—dead!> groaned forth Villefort, in a paroxysm of grief, which was the more terrible from the novelty of the sensation in the iron heart of that man. 

 <Dead!> repeated a third voice. <Who said Valentine was dead?> 

 The two men turned round, and saw Morrel standing at the door, pale and terror-stricken. This is what had happened. At the usual time, Morrel had presented himself at the little door leading to Noirtier's room. Contrary to custom, the door was open, and having no occasion to ring he entered. He waited for a moment in the hall and called for a servant to conduct him to M. Noirtier; but no one answered, the servants having, as we know, deserted the house. Morrel had no particular reason for uneasiness; Monte Cristo had promised him that Valentine should live, and so far he had always fulfilled his word. Every night the count had given him news, which was the next morning confirmed by Noirtier. Still this extraordinary silence appeared strange to him, and he called a second and third time; still no answer. Then he determined to go up. Noirtier's room was opened, like all the rest. The first thing he saw was the old man sitting in his armchair in his usual place, but his eyes expressed alarm, which was confirmed by the pallor which overspread his features. 

 <How are you, sir?> asked Morrel, with a sickness of heart. 

 <Well,> answered the old man, by closing his eyes; but his appearance manifested increasing uneasiness. 

 <You are thoughtful, sir,> continued Morrel; <you want something; shall I call one of the servants?> 

 <Yes,> replied Noirtier. 

 Morrel pulled the bell, but though he nearly broke the cord no one answered. He turned towards Noirtier; the pallor and anguish expressed on his countenance momentarily increased. 

 <Oh,> exclaimed Morrel, <why do they not come? Is anyone ill in the house?> The eyes of Noirtier seemed as though they would start from their sockets. <What is the matter? You alarm me. Valentine? Valentine?> 

 <Yes, yes,> signed Noirtier. 

 Maximilian tried to speak, but he could articulate nothing; he staggered, and supported himself against the wainscot. Then he pointed to the door. 

 <Yes, yes, yes!> continued the old man.  Maximilian rushed up the little staircase, while Noirtier's eyes seemed to say,—<Quicker, quicker!> 

 In a minute the young man darted through several rooms, till at length he reached Valentine's. 

 There was no occasion to push the door, it was wide open. A sob was the only sound he heard. He saw as though in a mist, a black figure kneeling and buried in a confused mass of white drapery. A terrible fear transfixed him. It was then he heard a voice exclaim <Valentine is dead!> and another voice which, like an echo repeated: 

 <Dead,—dead!>  