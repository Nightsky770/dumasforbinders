\chapter{The Hundred Days} 
	
	\lettrine{M}{.} Noirtier was a true prophet, and things progressed rapidly, as he had predicted. Everyone knows the history of the famous return from Elba, a return which was unprecedented in the past, and will probably remain without a counterpart in the future. 

 Louis \textsc{xviii.} made but a faint attempt to parry this unexpected blow; the monarchy he had scarcely reconstructed tottered on its precarious foundation, and at a sign from the emperor the incongruous structure of ancient prejudices and new ideas fell to the ground. Villefort, therefore, gained nothing save the king's gratitude (which was rather likely to injure him at the present time) and the cross of the Legion of honour, which he had the prudence not to wear, although M. de Blacas had duly forwarded the brevet. 

 Napoleon would, doubtless, have deprived Villefort of his office had it not been for Noirtier, who was all powerful at court, and thus the Girondin of '93 and the Senator of 1806 protected him who so lately had been his protector. All Villefort's influence barely enabled him to stifle the secret Dantès had so nearly divulged. The king's procureur alone was deprived of his office, being suspected of royalism. 

 However, scarcely was the imperial power established—that is, scarcely had the emperor re-entered the Tuileries and begun to issue orders from the closet into which we have introduced our readers,—he found on the table there Louis \textsc{xviii.}'s half-filled snuff-box,—scarcely had this occurred when Marseilles began, in spite of the authorities, to rekindle the flames of civil war, always smouldering in the south, and it required but little to excite the populace to acts of far greater violence than the shouts and insults with which they assailed the royalists whenever they ventured abroad.  Owing to this change, the worthy shipowner became at that moment—we will not say all powerful, because Morrel was a prudent and rather a timid man, so much so, that many of the most zealous partisans of Bonaparte accused him of <moderation>—but sufficiently influential to make a demand in favour of Dantès. 

 Villefort retained his place, but his marriage was put off until a more favourable opportunity. If the emperor remained on the throne, Gérard required a different alliance to aid his career; if Louis \textsc{xviii.} returned, the influence of M. de Saint-Méran, like his own, could be vastly increased, and the marriage be still more suitable. The deputy procureur was, therefore, the first magistrate of Marseilles, when one morning his door opened, and M. Morrel was announced. 

 Anyone else would have hastened to receive him; but Villefort was a man of ability, and he knew this would be a sign of weakness. He made Morrel wait in the antechamber, although he had no one with him, for the simple reason that the king's procureur always makes everyone wait, and after passing a quarter of an hour in reading the papers, he ordered M. Morrel to be admitted. 

 Morrel expected Villefort would be dejected; he found him as he had found him six weeks before, calm, firm, and full of that glacial politeness, that most insurmountable barrier which separates the well-bred from the vulgar man. 

 He had entered Villefort's office expecting that the magistrate would tremble at the sight of him; on the contrary, he felt a cold shudder all over him when he saw Villefort sitting there with his elbow on his desk, and his head leaning on his hand. He stopped at the door; Villefort gazed at him as if he had some difficulty in recognizing him; then, after a brief interval, during which the honest shipowner turned his hat in his hands, 

 <M. Morrel, I believe?> said Villefort. 

 <Yes, sir.> 

 <Come nearer,> said the magistrate, with a patronizing wave of the hand, <and tell me to what circumstance I owe the honour of this visit.> 

 <Do you not guess, monsieur?> asked Morrel. 

 <Not in the least; but if I can serve you in any way I shall be delighted.> 

 <Everything depends on you.> 

 <Explain yourself, pray.> 

 <Monsieur,> said Morrel, recovering his assurance as he proceeded, <do you recollect that a few days before the landing of his majesty the emperor, I came to intercede for a young man, the mate of my ship, who was accused of being concerned in correspondence with the Island of Elba? What was the other day a crime is today a title to favour. You then served Louis \textsc{xviii.}, and you did not show any favour—it was your duty; today you serve Napoleon, and you ought to protect him—it is equally your duty; I come, therefore, to ask what has become of him?>  Villefort by a strong effort sought to control himself. <What is his name?> said he. <Tell me his name.> 

 <Edmond Dantès.> 

 Villefort would probably have rather stood opposite the muzzle of a pistol at five-and-twenty paces than have heard this name spoken; but he did not blanch. 

 <Dantès,> repeated he, <Edmond Dantès.> 

 <Yes, monsieur.> Villefort opened a large register, then went to a table, from the table turned to his registers, and then, turning to Morrel, 

 <Are you quite sure you are not mistaken, monsieur?> said he, in the most natural tone in the world. 

 Had Morrel been a more quick-sighted man, or better versed in these matters, he would have been surprised at the king's procureur answering him on such a subject, instead of referring him to the governors of the prison or the prefect of the department. But Morrel, disappointed in his expectations of exciting fear, was conscious only of the other's condescension. Villefort had calculated rightly. 

 <No,> said Morrel; <I am not mistaken. I have known him for ten years, the last four of which he was in my service. Do not you recollect, I came about six weeks ago to plead for clemency, as I come today to plead for justice. You received me very coldly. Oh, the royalists were very severe with the Bonapartists in those days.> 

 <Monsieur,> returned Villefort, <I was then a royalist, because I believed the Bourbons not only the heirs to the throne, but the chosen of the nation. The miraculous return of Napoleon has conquered me, the legitimate monarch is he who is loved by his people.> 

 <That's right!> cried Morrel. <I like to hear you speak thus, and I augur well for Edmond from it.> 

 <Wait a moment,> said Villefort, turning over the leaves of a register; <I have it—a sailor, who was about to marry a young Catalan girl. I recollect now; it was a very serious charge.> 

 <How so?> 

 <You know that when he left here he was taken to the Palais de Justice.> 

 <Well?> 

 <I made my report to the authorities at Paris, and a week after he was carried off.> 

 <Carried off!> said Morrel. <What can they have done with him?> 

 <Oh, he has been taken to Fenestrelles, to Pignerol, or to the Sainte-Marguérite islands. Some fine morning he will return to take command of your vessel.> 

 <Come when he will, it shall be kept for him. But how is it he is not already returned? It seems to me the first care of government should be to set at liberty those who have suffered for their adherence to it.> 

 <Do not be too hasty, M. Morrel,> replied Villefort. <The order of imprisonment came from high authority, and the order for his liberation must proceed from the same source; and, as Napoleon has scarcely been reinstated a fortnight, the letters have not yet been forwarded.> 

 <But,> said Morrel, <is there no way of expediting all these formalities—of releasing him from arrest?> 

 <There has been no arrest.> 

 <How?> 

 <It is sometimes essential to government to cause a man's disappearance without leaving any traces, so that no written forms or documents may defeat their wishes.> 

 <It might be so under the Bourbons, but at present\longdash> 

 <It has always been so, my dear Morrel, since the reign of Louis \textsc{xiv.} The emperor is more strict in prison discipline than even Louis himself, and the number of prisoners whose names are not on the register is incalculable.> Had Morrel even any suspicions, so much kindness would have dispelled them. 

 <Well, M. de Villefort, how would you advise me to act?> asked he. 

 <Petition the minister.> 

 <Oh, I know what that is; the minister receives two hundred petitions every day, and does not read three.> 

 <That is true; but he will read a petition countersigned and presented by me.> 

 <And will you undertake to deliver it?> 

 <With the greatest pleasure. Dantès was then guilty, and now he is innocent, and it is as much my duty to free him as it was to condemn him.> Villefort thus forestalled any danger of an inquiry, which, however improbable it might be, if it did take place would leave him defenceless. 

 <But how shall I address the minister?> 

 <Sit down there,> said Villefort, giving up his place to Morrel, <and write what I dictate.> 

 <Will you be so good?> 

 <Certainly. But lose no time; we have lost too much already.> 

 <That is true. Only think what the poor fellow may even now be suffering.> 

 Villefort shuddered at the suggestion; but he had gone too far to draw back. Dantès must be crushed to gratify Villefort's ambition. 

 Villefort dictated a petition, in which, from an excellent intention, no doubt, Dantès' patriotic services were exaggerated, and he was made out one of the most active agents of Napoleon's return. It was evident that at the sight of this document the minister would instantly release him. The petition finished, Villefort read it aloud. 

 <That will do,> said he; <leave the rest to me.> 

 <Will the petition go soon?> 

 <Today.> 

 <Countersigned by you?> 

 <The best thing I can do will be to certify the truth of the contents of your petition.> And, sitting down, Villefort wrote the certificate at the bottom. 

 <What more is to be done?> 

 <I will do whatever is necessary.> This assurance delighted Morrel, who took leave of Villefort, and hastened to announce to old Dantès that he would soon see his son. 

 As for Villefort, instead of sending to Paris, he carefully preserved the petition that so fearfully compromised Dantès, in the hopes of an event that seemed not unlikely,—that is, a second restoration. Dantès remained a prisoner, and heard not the noise of the fall of Louis \textsc{xviii.}'s throne, or the still more tragic destruction of the empire. 

 Twice during the Hundred Days had Morrel renewed his demand, and twice had Villefort soothed him with promises. At last there was Waterloo, and Morrel came no more; he had done all that was in his power, and any fresh attempt would only compromise himself uselessly. 

 Louis \textsc{xviii.} remounted the throne; Villefort, to whom Marseilles had become filled with remorseful memories, sought and obtained the situation of king's procureur at Toulouse, and a fortnight afterwards he married Mademoiselle de Saint-Méran, whose father now stood higher at court than ever. 

 And so Dantès, after the Hundred Days and after Waterloo, remained in his dungeon, forgotten of earth and heaven. 

 Danglars comprehended the full extent of the wretched fate that overwhelmed Dantès; and, when Napoleon returned to France, he, after the manner of mediocre minds, termed the coincidence, \textit{a decree of Providence}. But when Napoleon returned to Paris, Danglars' heart failed him, and he lived in constant fear of Dantès' return on a mission of vengeance. He therefore informed M. Morrel of his wish to quit the sea, and obtained a recommendation from him to a Spanish merchant, into whose service he entered at the end of March, that is, ten or twelve days after Napoleon's return. He then left for Madrid, and was no more heard of. 

 Fernand understood nothing except that Dantès was absent. What had become of him he cared not to inquire. Only, during the respite the absence of his rival afforded him, he reflected, partly on the means of deceiving Mercédès as to the cause of his absence, partly on plans of emigration and abduction, as from time to time he sat sad and motionless on the summit of Cape Pharo, at the spot from whence Marseilles and the Catalans are visible, watching for the apparition of a young and handsome man, who was for him also the messenger of vengeance. Fernand's mind was made up; he would shoot Dantès, and then kill himself. But Fernand was mistaken; a man of his disposition never kills himself, for he constantly hopes. 

 During this time the empire made its last conscription, and every man in France capable of bearing arms rushed to obey the summons of the emperor. Fernand departed with the rest, bearing with him the terrible thought that while he was away, his rival would perhaps return and marry Mercédès. Had Fernand really meant to kill himself, he would have done so when he parted from Mercédès. His devotion, and the compassion he showed for her misfortunes, produced the effect they always produce on noble minds—Mercédès had always had a sincere regard for Fernand, and this was now strengthened by gratitude. 

 <My brother,> said she, as she placed his knapsack on his shoulders, <be careful of yourself, for if you are killed, I shall be alone in the world.> These words carried a ray of hope into Fernand's heart. Should Dantès not return, Mercédès might one day be his.  Mercédès was left alone face to face with the vast plain that had never seemed so barren, and the sea that had never seemed so vast. Bathed in tears she wandered about the Catalan village. Sometimes she stood mute and motionless as a statue, looking towards Marseilles, at other times gazing on the sea, and debating as to whether it were not better to cast herself into the abyss of the ocean, and thus end her woes. It was not want of courage that prevented her putting this resolution into execution; but her religious feelings came to her aid and saved her. 

 Caderousse was, like Fernand, enrolled in the army, but, being married and eight years older, he was merely sent to the frontier. Old Dantès, who was only sustained by hope, lost all hope at Napoleon's downfall. Five months after he had been separated from his son, and almost at the hour of his arrest, he breathed his last in Mercédès' arms. M. Morrel paid the expenses of his funeral, and a few small debts the poor old man had contracted. 

 There was more than benevolence in this action; there was courage; the south was aflame, and to assist, even on his death-bed, the father of so dangerous a Bonapartist as Dantès, was stigmatized as a crime. 