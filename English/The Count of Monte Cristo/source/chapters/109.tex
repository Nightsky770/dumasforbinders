\chapter{The Assizes} 

 \lettrine{T}{he} Benedetto affair, as it was called at the Palais, and by people in general, had produced a tremendous sensation. Frequenting the Café de Paris, the Boulevard de Gand, and the Bois de Boulogne, during his brief career of splendour, the false Cavalcanti had formed a host of acquaintances. The papers had related his various adventures, both as the man of fashion and the galley-slave; and as everyone who had been personally acquainted with Prince Andrea Cavalcanti experienced a lively curiosity in his fate, they all determined to spare no trouble in endeavouring to witness the trial of M. Benedetto for the murder of his comrade in chains. 

 In the eyes of many, Benedetto appeared, if not a victim to, at least an instance of, the fallibility of the law. M. Cavalcanti, his father, had been seen in Paris, and it was expected that he would re-appear to claim the illustrious outcast. Many, also, who were not aware of the circumstances attending his withdrawal from Paris, were struck with the worthy appearance, the gentlemanly bearing, and the knowledge of the world displayed by the old patrician, who certainly played the nobleman very well, so long as he said nothing, and made no arithmetical calculations. 

 As for the accused himself, many remembered him as being so amiable, so handsome, and so liberal, that they chose to think him the victim of some conspiracy, since in this world large fortunes frequently excite the malevolence and jealousy of some unknown enemy. 

 Everyone, therefore, ran to the court; some to witness the sight, others to comment upon it. From seven o'clock in the morning a crowd was stationed at the iron gates, and an hour before the trial commenced the hall was full of the privileged. Before the entrance of the magistrates, and indeed frequently afterwards, a court of justice, on days when some especial trial is to take place, resembles a drawing-room where many persons recognize each other and converse if they can do so without losing their seats; or, if they are separated by too great a number of lawyers, communicate by signs. 

 It was one of the magnificent autumn days which make amends for a short summer; the clouds which M. de Villefort had perceived at sunrise had all disappeared as if by magic, and one of the softest and most brilliant days of September shone forth in all its splendour. 

 Beauchamp, one of the kings of the press, and therefore claiming the right of a throne everywhere, was eying everybody through his monocle. He perceived Château-Renaud and Debray, who had just gained the good graces of a sergeant-at-arms, and who had persuaded the latter to let them stand before, instead of behind him, as they ought to have done. The worthy sergeant had recognized the minister's secretary and the millionnaire, and, by way of paying extra attention to his noble neighbours, promised to keep their places while they paid a visit to Beauchamp. 

 <Well,> said Beauchamp, <we shall see our friend!> 

 <Yes, indeed!> replied Debray. <That worthy prince. Deuce take those Italian princes!> 

 <A man, too, who could boast of Dante for a genealogist, and could reckon back to the \textit{Divina Comedia}.> 

 <A nobility of the rope!> said Château-Renaud phlegmatically. 

 <He will be condemned, will he not?> asked Debray of Beauchamp. 

 <My dear fellow, I think we should ask you that question; you know such news much better than we do. Did you see the president at the minister's last night?> 

 <Yes.> 

 <What did he say?> 

 <Something which will surprise you.> 

 <Oh, make haste and tell me, then; it is a long time since that has happened.> 

 <Well, he told me that Benedetto, who is considered a serpent of subtlety and a giant of cunning, is really but a very commonplace, silly rascal, and altogether unworthy of the experiments that will be made on his phrenological organs after his death.> 

 <Bah,> said Beauchamp, <he played the prince very well.> 

 <Yes, for you who detest those unhappy princes, Beauchamp, and are always delighted to find fault with them; but not for me, who discover a gentleman by instinct, and who scent out an aristocratic family like a very bloodhound of heraldry.> 

 <Then you never believed in the principality?> 

 <Yes.—in the principality, but not in the prince.> 

 <Not so bad,> said Beauchamp; <still, I assure you, he passed very well with many people; I saw him at the ministers' houses.> 

 <Ah, yes,> said Château-Renaud. <The idea of thinking ministers understand anything about princes!> 

 <There is something in what you have just said,> said Beauchamp, laughing. 

 <But,> said Debray to Beauchamp, <if I spoke to the president, \textit{you} must have been with the procureur.> 

 <It was an impossibility; for the last week M. de Villefort has secluded himself. It is natural enough; this strange chain of domestic afflictions, followed by the no less strange death of his daughter\longdash> 

 <Strange? What do you mean, Beauchamp?> 

 <Oh, yes; do you pretend that all this has been unobserved at the minister's?> said Beauchamp, placing his eye-glass in his eye, where he tried to make it remain. 

 <My dear sir,> said Château-Renaud, <allow me to tell you that you do not understand that manœuvre with the eye-glass half so well as Debray. Give him a lesson, Debray.> 

 <Stay,> said Beauchamp, <surely I am not deceived.> 

 <What is it?> 

 <It is she!> 

 <Whom do you mean?> 

 <They said she had left.> 

 <Mademoiselle Eugénie?> said Château-Renaud; <has she returned?> 

 <No, but her mother.> 

 <Madame Danglars? Nonsense! Impossible!> said Château-Renaud; <only ten days after the flight of her daughter, and three days from the bankruptcy of her husband?> 

 Debray coloured slightly, and followed with his eyes the direction of Beauchamp's glance. 

 <Come,> he said, <it is only a veiled lady, some foreign princess, perhaps the mother of Cavalcanti. But you were just speaking on a very interesting topic, Beauchamp.> 

 <I?> 

 <Yes; you were telling us about the extraordinary death of Valentine.> 

 <Ah, yes, so I was. But how is it that Madame de Villefort is not here?> 

 <Poor, dear woman,> said Debray, <she is no doubt occupied in distilling balm for the hospitals, or in making cosmetics for herself or friends. Do you know she spends two or three thousand crowns a year in this amusement? But I wonder she is not here. I should have been pleased to see her, for I like her very much.> 

 <And I hate her,> said Château-Renaud. 

 <Why?> 

 <I do not know. Why do we love? Why do we hate? I detest her, from antipathy.> 

 <Or, rather, by instinct.> 

 <Perhaps so. But to return to what you were saying, Beauchamp.> 

 <Well, do you know why they die so multitudinously at M. de Villefort's?> 

 <<Multitudinously> is good,> said Château-Renaud. 

 <My good fellow, you'll find the word in Saint-Simon.> 

 <But the thing itself is at M. de Villefort's; but let's get back to the subject.> 

 <Talking of that,> said Debray, <Madame was making inquiries about that house, which for the last three months has been hung with black.> 

 <Who is Madame?> asked Château-Renaud. 

 <The minister's wife, \textit{pardieu!}> 

 <Oh, your pardon! I never visit ministers; I leave that to the princes.> 

 <Really, you were only before sparkling, but now you are brilliant; take compassion on us, or, like Jupiter, you will wither us up.> 

 <I will not speak again,> said Château-Renaud; <pray have compassion upon me, and do not take up every word I say.> 

 <Come, let us endeavour to get to the end of our story, Beauchamp; I told you that yesterday Madame made inquiries of me upon the subject; enlighten me, and I will then communicate my information to her.> 

 <Well, gentlemen, the reason people die so multitudinously (I like the word) at M. de Villefort's is that there is an assassin in the house!> 

 The two young men shuddered, for the same idea had more than once occurred to them. 

 <And who is the assassin;> they asked together. 

 <Young Edward!> A burst of laughter from the auditors did not in the least disconcert the speaker, who continued,—<Yes, gentlemen; Edward, the infant phenomenon, who is quite an adept in the art of killing.> 

 <You are jesting.> 

 <Not at all. I yesterday engaged a servant, who had just left M. de Villefort—I intend sending him away tomorrow, for he eats so enormously, to make up for the fast imposed upon him by his terror in that house. Well, now listen.> 

 <We are listening.> 

 <It appears the dear child has obtained possession of a bottle containing some drug, which he every now and then uses against those who have displeased him. First, M. and Madame de Saint-Méran incurred his displeasure, so he poured out three drops of his elixir—three drops were sufficient; then followed Barrois, the old servant of M. Noirtier, who sometimes rebuffed this little wretch—he therefore received the same quantity of the elixir; the same happened to Valentine, of whom he was jealous; he gave her the same dose as the others, and all was over for her as well as the rest.> 

 <Why, what nonsense are you telling us?> said Château-Renaud. 

 <Yes, it is an extraordinary story,> said Beauchamp; <is it not?> 

 <It is absurd,> said Debray. 

 <Ah,> said Beauchamp, <you doubt me? Well, you can ask my servant, or rather him who will no longer be my servant tomorrow, it was the talk of the house.> 

 <And this elixir, where is it? what is it?> 

 <The child conceals it.> 

 <But where did he find it?> 

 <In his mother's laboratory.> 

 <Does his mother then, keep poisons in her laboratory?> 

 <How can I tell? You are questioning me like a king's attorney. I only repeat what I have been told, and like my informant I can do no more. The poor devil would eat nothing, from fear.> 

 <It is incredible!> 

 <No, my dear fellow, it is not at all incredible. You saw the child pass through the Rue Richelieu last year, who amused himself with killing his brothers and sisters by sticking pins in their ears while they slept. The generation who follow us are very precocious.> 

 <Come, Beauchamp,> said Château-Renaud, <I will bet anything you do not believe a word of all you have been telling us. But I do not see the Count of Monte Cristo here.> 

 <He is worn out,> said Debray; <besides, he could not well appear in public, since he has been the dupe of the Cavalcanti, who, it appears, presented themselves to him with false letters of credit, and cheated him out of 100,000 francs upon the hypothesis of this principality.> 

 <By the way, M. de Château-Renaud,> asked Beauchamp, <how is Morrel?> 

 <\textit{Ma foi}, I have called three times without once seeing him. Still, his sister did not seem uneasy, and told me that though she had not seen him for two or three days, she was sure he was well.> 

 <Ah, now I think of it, the Count of Monte Cristo cannot appear in the hall,> said Beauchamp. 

 <Why not?> 

 <Because he is an actor in the drama.> 

 <Has he assassinated anyone, then?> 

 <No, on the contrary, they wished to assassinate him. You know that it was in leaving his house that M. de Caderousse was murdered by his friend Benedetto. You know that the famous waistcoat was found in his house, containing the letter which stopped the signature of the marriage-contract. Do you see the waistcoat? There it is, all blood-stained, on the desk, as a testimony of the crime.> 

 <Ah, very good.> 

 <Hush, gentlemen, here is the court; let us go back to our places.> 

 A noise was heard in the hall; the sergeant called his two patrons with an energetic <hem!> and the door-keeper appearing, called out with that shrill voice peculiar to his order, ever since the days of Beaumarchais: 

 <The court, gentlemen!> 