\chapter{The Indictment} 

 \lettrine{T}{he} judges took their places in the midst of the most profound silence; the jury took their seats; M. de Villefort, the object of unusual attention, and we had almost said of general admiration, sat in the armchair and cast a tranquil glance around him. Everyone looked with astonishment on that grave and severe face, whose calm expression personal griefs had been unable to disturb, and the aspect of a man who was a stranger to all human emotions excited something very like terror. 

 <Gendarmes,> said the president, <lead in the accused.> 

 At these words the public attention became more intense, and all eyes were turned towards the door through which Benedetto was to enter. The door soon opened and the accused appeared. 

 The same impression was experienced by all present, and no one was deceived by the expression of his countenance. His features bore no sign of that deep emotion which stops the beating of the heart and blanches the cheek. His hands, gracefully placed, one upon his hat, the other in the opening of his white waistcoat, were not at all tremulous; his eye was calm and even brilliant. Scarcely had he entered the hall when he glanced at the whole body of magistrates and assistants; his eye rested longer on the president, and still more so on the king's attorney. 

 By the side of Andrea was stationed the lawyer who was to conduct his defence, and who had been appointed by the court, for Andrea disdained to pay any attention to those details, to which he appeared to attach no importance. The lawyer was a young man with light hair whose face expressed a hundred times more emotion than that which characterized the prisoner.  The president called for the indictment, revised as we know, by the clever and implacable pen of Villefort. During the reading of this, which was long, the public attention was continually drawn towards Andrea, who bore the inspection with Spartan unconcern. Villefort had never been so concise and eloquent. The crime was depicted in the most vivid colours; the former life of the prisoner, his transformation, a review of his life from the earliest period, were set forth with all the talent that a knowledge of human life could furnish to a mind like that of the procureur. Benedetto was thus forever condemned in public opinion before the sentence of the law could be pronounced. 

 Andrea paid no attention to the successive charges which were brought against him. M. de Villefort, who examined him attentively, and who no doubt practised upon him all the psychological studies he was accustomed to use, in vain endeavoured to make him lower his eyes, notwithstanding the depth and profundity of his gaze. At length the reading of the indictment was ended. 

 <Accused,> said the president, <your name and surname?> 

 Andrea arose. 

 <Excuse me, Mr. President,> he said, in a clear voice, <but I see you are going to adopt a course of questions through which I cannot follow you. I have an idea, which I will explain by and by, of making an exception to the usual form of accusation. Allow me, then, if you please, to answer in different order, or I will not do so at all.> 

 The astonished president looked at the jury, who in turn looked at Villefort. The whole assembly manifested great surprise, but Andrea appeared quite unmoved. 

 <Your age?> said the president; <will you answer that question?> 

 <I will answer that question, as well as the rest, Mr. President, but in its turn.> 

 <Your age?> repeated the president. 

 <I am twenty-one years old, or rather I shall be in a few days, as I was born the night of the 27th of September, 1817.> 

 M. de Villefort, who was busy taking down some notes, raised his head at the mention of this date. 

 <Where were you born?> continued the president. 

 <At Auteuil, near Paris.> 

 M. de Villefort a second time raised his head, looked at Benedetto as if he had been gazing at the head of Medusa, and became livid. As for Benedetto, he gracefully wiped his lips with a fine cambric pocket-handkerchief. 

 <Your profession?> 

 <First I was a forger,> answered Andrea, as calmly as possible; <then I became a thief, and lately have become an assassin.> 

 A murmur, or rather storm, of indignation burst from all parts of the assembly. The judges themselves appeared to be stupefied, and the jury manifested tokens of disgust for a cynicism so unexpected in a man of fashion. M. de Villefort pressed his hand upon his brow, which, at first pale, had become red and burning; then he suddenly arose and looked around as though he had lost his senses—he wanted air.  <Are you looking for anything, Mr. Procureur?> asked Benedetto, with his most ingratiating smile. 

 M. de Villefort answered nothing, but sat, or rather threw himself down again upon his chair. 

 <And now, prisoner, will you consent to tell your name?> said the president. <The brutal affectation with which you have enumerated and classified your crimes calls for a severe reprimand on the part of the court, both in the name of morality, and for the respect due to humanity. You appear to consider this a point of honour, and it may be for this reason, that you have delayed acknowledging your name. You wished it to be preceded by all these titles.> 

 <It is quite wonderful, Mr. President, how entirely you have read my thoughts,> said Benedetto, in his softest voice and most polite manner. <This is, indeed, the reason why I begged you to alter the order of the questions.> 

 The public astonishment had reached its height. There was no longer any deceit or bravado in the manner of the accused. The audience felt that a startling revelation was to follow this ominous prelude. 

 <Well,> said the president; <your name?> 

 <I cannot tell you my name, since I do not know it; but I know my father's, and can tell it to you.> 

 A painful giddiness overwhelmed Villefort; great drops of acrid sweat fell from his face upon the papers which he held in his convulsed hand. 

 <Repeat your father's name,> said the president. 

 Not a whisper, not a breath, was heard in that vast assembly; everyone waited anxiously. 

 <My father is king's attorney,> replied Andrea calmly.  <King's attorney?> said the president, stupefied, and without noticing the agitation which spread over the face of M. de Villefort; <king's attorney?> 

 <Yes; and if you wish to know his name, I will tell it,—he is named Villefort.> 

 The explosion, which had been so long restrained from a feeling of respect to the court of justice, now burst forth like thunder from the breasts of all present; the court itself did not seek to restrain the feelings of the audience. The exclamations, the insults addressed to Benedetto, who remained perfectly unconcerned, the energetic gestures, the movement of the gendarmes, the sneers of the scum of the crowd always sure to rise to the surface in case of any disturbance—all this lasted five minutes, before the door-keepers and magistrates were able to restore silence. In the midst of this tumult the voice of the president was heard to exclaim: 

 <Are you playing with justice, accused, and do you dare set your fellow-citizens an example of disorder which even in these times has never been equalled?> 

 Several persons hurried up to M. de Villefort, who sat half bowed over in his chair, offering him consolation, encouragement, and protestations of zeal and sympathy. Order was re-established in the hall, except that a few people still moved about and whispered to one another. A lady, it was said, had just fainted; they had supplied her with a smelling-bottle, and she had recovered. During the scene of tumult, Andrea had turned his smiling face towards the assembly; then, leaning with one hand on the oaken rail of the dock, in the most graceful attitude possible, he said: 

 <Gentlemen, I assure you I had no idea of insulting the court, or of making a useless disturbance in the presence of this honourable assembly. They ask my age; I tell it. They ask where I was born; I answer. They ask my name, I cannot give it, since my parents abandoned me. But though I cannot give my own name, not possessing one, I can tell them my father's. Now I repeat, my father is named M. de Villefort, and I am ready to prove it.> 

 There was an energy, a conviction, and a sincerity in the manner of the young man, which silenced the tumult. All eyes were turned for a moment towards the procureur, who sat as motionless as though a thunderbolt had changed him into a corpse. 

 <Gentlemen,> said Andrea, commanding silence by his voice and manner; <I owe you the proofs and explanations of what I have said.> 

 <But,> said the irritated president, <you called yourself Benedetto, declared yourself an orphan, and claimed Corsica as your country.> 

 <I said anything I pleased, in order that the solemn declaration I have just made should not be withheld, which otherwise would certainly have been the case. I now repeat that I was born at Auteuil on the night of the 27th of September, 1817, and that I am the son of the procureur, M. de Villefort. Do you wish for any further details? I will give them. I was born in № 28, Rue de la Fontaine, in a room hung with red damask; my father took me in his arms, telling my mother I was dead, wrapped me in a napkin marked with an H and an N, and carried me into a garden, where he buried me alive.> 

 A shudder ran through the assembly when they saw that the confidence of the prisoner increased in proportion to the terror of M. de Villefort. 

 <But how have you become acquainted with all these details?> asked the president. 

 <I will tell you, Mr. President. A man who had sworn vengeance against my father, and had long watched his opportunity to kill him, had introduced himself that night into the garden in which my father buried me. He was concealed in a thicket; he saw my father bury something in the ground, and stabbed him; then thinking the deposit might contain some treasure he turned up the ground, and found me still living. The man carried me to the foundling asylum, where I was registered under the number 37. Three months afterwards, a woman travelled from Rogliano to Paris to fetch me, and having claimed me as her son, carried me away. Thus, you see, though born in Paris, I was brought up in Corsica.> 

 There was a moment's silence, during which one could have fancied the hall empty, so profound was the stillness. 

 <Proceed,> said the president. 

 <Certainly, I might have lived happily amongst those good people, who adored me, but my perverse disposition prevailed over the virtues which my adopted mother endeavoured to instil into my heart. I increased in wickedness till I committed crime. One day when I cursed Providence for making me so wicked, and ordaining me to such a fate, my adopted father said to me, <Do not blaspheme, unhappy child, the crime is that of your father, not yours,—of your father, who consigned you to hell if you died, and to misery if a miracle preserved you alive.> After that I ceased to blaspheme, but I cursed my father. That is why I have uttered the words for which you blame me; that is why I have filled this whole assembly with horror. If I have committed an additional crime, punish me, but if you will allow that ever since the day of my birth my fate has been sad, bitter, and lamentable, then pity me.> 

 <But your mother?> asked the president. 

 <My mother thought me dead; she is not guilty. I did not even wish to know her name, nor do I know it.> 

 Just then a piercing cry, ending in a sob, burst from the centre of the crowd, who encircled the lady who had before fainted, and who now fell into a violent fit of hysterics. She was carried out of the hall, the thick veil which concealed her face dropped off, and Madame Danglars was recognized. Notwithstanding his shattered nerves, the ringing sensation in his ears, and the madness which turned his brain, Villefort rose as he perceived her. 

 <The proofs, the proofs!> said the president; <remember this tissue of horrors must be supported by the clearest proofs.> 

 <The proofs?> said Benedetto, laughing; <do you want proofs?> 

 <Yes.> 

 <Well, then, look at M. de Villefort, and then ask me for proofs.> 

 Everyone turned towards the procureur, who, unable to bear the universal gaze now riveted on him alone, advanced staggering into the midst of the tribunal, with his hair dishevelled and his face indented with the mark of his nails. The whole assembly uttered a long murmur of astonishment. 

 <Father,> said Benedetto, <I am asked for proofs, do you wish me to give them?> 

 <No, no, it is useless,> stammered M. de Villefort in a hoarse voice; <no, it is useless!> 

 <How useless?> cried the president, <what do you mean?> 

 <I mean that I feel it impossible to struggle against this deadly weight which crushes me. Gentlemen, I know I am in the hands of an avenging God! We need no proofs; everything relating to this young man is true.> 

 A dull, gloomy silence, like that which precedes some awful phenomenon of nature, pervaded the assembly, who shuddered in dismay. 

 <What, M. de Villefort,> cried the president, <do you yield to an hallucination? What, are you no longer in possession of your senses? This strange, unexpected, terrible accusation has disordered your reason. Come, recover.> 

 The procureur dropped his head; his teeth chattered like those of a man under a violent attack of fever, and yet he was deadly pale. 

 <I am in possession of all my senses, sir,> he said; <my body alone suffers, as you may suppose. I acknowledge myself guilty of all the young man has brought against me, and from this hour hold myself under the authority of the procureur who will succeed me.> 

 And as he spoke these words with a hoarse, choking voice, he staggered towards the door, which was mechanically opened by a door-keeper. The whole assembly were dumb with astonishment at the revelation and confession which had produced a catastrophe so different from that which had been expected during the last fortnight by the Parisian world. 

 <Well,> said Beauchamp, <let them now say that drama is unnatural!> 

 <\textit{Ma foi!}> said Château-Renaud, <I would rather end my career like M. de Morcerf; a pistol-shot seems quite delightful compared with this catastrophe.> 

 <And moreover, it kills,> said Beauchamp. 

 <And to think that I had an idea of marrying his daughter,> said Debray. <She did well to die, poor girl!> 

 <The sitting is adjourned, gentlemen,> said the president; <fresh inquiries will be made, and the case will be tried next session by another magistrate.> 

 As for Andrea, who was calm and more interesting than ever, he left the hall, escorted by gendarmes, who involuntarily paid him some attention. 

 <Well, what do you think of this, my fine fellow?> asked Debray of the sergeant-at-arms, slipping a louis into his hand. 

 <There will be extenuating circumstances,> he replied. 