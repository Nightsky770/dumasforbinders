\chapter{The Cemetery of Père-Lachaise} 
	
\lettrine{M}{.} de Boville had indeed met the funeral procession which was taking Valentine to her last home on earth. The weather was dull and stormy, a cold wind shook the few remaining yellow leaves from the boughs of the trees, and scattered them among the crowd which filled the boulevards. M. de Villefort, a true Parisian, considered the cemetery of Père-Lachaise alone worthy of receiving the mortal remains of a Parisian family; there alone the corpses belonging to him would be surrounded by worthy associates. He had therefore purchased a vault, which was quickly occupied by members of his family. On the front of the monument was inscribed: <The families of Saint-Méran and Villefort,> for such had been the last wish expressed by poor Renée, Valentine's mother. The pompous procession therefore wended its way towards Père-Lachaise from the Faubourg Saint-Honoré. Having crossed Paris, it passed through the Faubourg du Temple, then leaving the exterior boulevards, it reached the cemetery. More than fifty private carriages followed the twenty mourning-coaches, and behind them more than five hundred persons joined in the procession on foot.  These last consisted of all the young people whom Valentine's death had struck like a thunderbolt, and who, notwithstanding the raw chilliness of the season, could not refrain from paying a last tribute to the memory of the beautiful, chaste, and adorable girl, thus cut off in the flower of her youth. 

 As they left Paris, an equipage with four horses, at full speed, was seen to draw up suddenly; it contained Monte Cristo. The count left the carriage and mingled in the crowd who followed on foot. Château-Renaud perceived him and immediately alighting from his \textit{coupé}, joined him; Beauchamp did the same. 

 The count looked attentively through every opening in the crowd; he was evidently watching for someone, but his search ended in disappointment.  <Where is Morrel?> he asked; <do either of these gentlemen know where he is?> 

 <We have already asked that question,> said Château-Renaud, <for none of us has seen him.> 

 The count was silent, but continued to gaze around him. At length they arrived at the cemetery. The piercing eye of Monte Cristo glanced through clusters of bushes and trees, and was soon relieved from all anxiety, for seeing a shadow glide between the yew-trees, Monte Cristo recognized him whom he sought. 

 One funeral is generally very much like another in this magnificent metropolis. Black figures are seen scattered over the long white avenues; the silence of earth and heaven is alone broken by the noise made by the crackling branches of hedges planted around the monuments; then follows the melancholy chant of the priests, mingled now and then with a sob of anguish, escaping from some woman concealed behind a mass of flowers. 

 The shadow Monte Cristo had noticed passed rapidly behind the tomb of Abélard and Héloïse, placed itself close to the heads of the horses belonging to the hearse, and following the undertaker's men, arrived with them at the spot appointed for the burial. Each person's attention was occupied. Monte Cristo saw nothing but the shadow, which no one else observed. Twice the count left the ranks to see whether the object of his interest had any concealed weapon beneath his clothes. When the procession stopped, this shadow was recognized as Morrel, who, with his coat buttoned up to his throat, his face livid, and convulsively crushing his hat between his fingers, leaned against a tree, situated on an elevation commanding the mausoleum, so that none of the funeral details could escape his observation. 

 Everything was conducted in the usual manner. A few men, the least impressed of all by the scene, pronounced a discourse, some deploring this premature death, others expatiating on the grief of the father, and one very ingenious person quoting the fact that Valentine had solicited pardon of her father for criminals on whom the arm of justice was ready to fall—until at length they exhausted their stores of metaphor and mournful speeches, elaborate variations on the stanzas of Malherbe to Du Périer. 

 Monte Cristo heard and saw nothing, or rather he only saw Morrel, whose calmness had a frightful effect on those who knew what was passing in his heart. 

 <See,> said Beauchamp, pointing out Morrel to Debray. <What is he doing up there?> And they called Château-Renaud's attention to him. 

 <How pale he is!> said Château-Renaud, shuddering. 

 <He is cold,> said Debray. 

 <Not at all,> said Château-Renaud, slowly; <I think he is violently agitated. He is very susceptible.> 

 <Bah,> said Debray; <he scarcely knew Mademoiselle de Villefort; you said so yourself.> 

 <True. Still I remember he danced three times with her at Madame de Morcerf's. Do you recollect that ball, count, where you produced such an effect?>

<No, I do not,> replied Monte Cristo, without even knowing of what or to whom he was speaking, so much was he occupied in watching Morrel, who was holding his breath with emotion. 

 <The discourse is over; farewell, gentlemen,> said the count, unceremoniously. 

 And he disappeared without anyone seeing whither he went. 

 The funeral being over, the guests returned to Paris. Château-Renaud looked for a moment for Morrel; but while they were watching the departure of the count, Morrel had quitted his post, and Château-Renaud, failing in his search, joined Debray and Beauchamp. 

 Monte Cristo concealed himself behind a large tomb and awaited the arrival of Morrel, who by degrees approached the tomb now abandoned by spectators and workmen. Morrel threw a glance around, but before it reached the spot occupied by Monte Cristo the latter had advanced yet nearer, still unperceived. The young man knelt down. The count, with outstretched neck and glaring eyes, stood in an attitude ready to pounce upon Morrel upon the first occasion. Morrel bent his head till it touched the stone, then clutching the grating with both hands, he murmured: 

 <Oh, Valentine!> 

 The count's heart was pierced by the utterance of these two words; he stepped forward, and touching the young man's shoulder, said: 

 <I was looking for you, my friend.> Monte Cristo expected a burst of passion, but he was deceived, for Morrel turning round, said calmly,— 

 <You see I was praying.> The scrutinizing glance of the count searched the young man from head to foot. He then seemed more easy. 

 <Shall I drive you back to Paris?> he asked. 

 <No, thank you.> 

 <Do you wish anything?> 

 <Leave me to pray.> 

 The count withdrew without opposition, but it was only to place himself in a situation where he could watch every movement of Morrel, who at length arose, brushed the dust from his knees, and turned towards Paris, without once looking back. He walked slowly down the Rue de la Roquette. The count, dismissing his carriage, followed him about a hundred paces behind. Maximilian crossed the canal and entered the Rue Meslay by the boulevards. 

 Five minutes after the door had been closed on Morrel's entrance, it was again opened for the count. Julie was at the entrance of the garden, where she was attentively watching Penelon, who, entering with zeal into his profession of gardener, was very busy grafting some Bengal roses. <Ah, count,> she exclaimed, with the delight manifested by every member of the family whenever he visited the Rue Meslay. 

 <Maximilian has just returned, has he not, madame?> asked the count.  <Yes, I think I saw him pass; but pray, call Emmanuel.> 

 <Excuse me, madame, but I must go up to Maximilian's room this instant,> replied Monte Cristo, <I have something of the greatest importance to tell him.> 

 <Go, then,> she said with a charming smile, which accompanied him until he had disappeared. 

 Monte Cristo soon ran up the staircase conducting from the ground floor to Maximilian's room; when he reached the landing he listened attentively, but all was still. Like many old houses occupied by a single family, the room door was panelled with glass; but it was locked, Maximilian was shut in, and it was impossible to see what was passing in the room, because a red curtain was drawn before the glass. The count's anxiety was manifested by a bright colour which seldom appeared on the face of that imperturbable man. 

 <What shall I do!> he uttered, and reflected for a moment; <shall I ring? No, the sound of a bell, announcing a visitor, will but accelerate the resolution of one in Maximilian's situation, and then the bell would be followed by a louder noise.> 

 Monte Cristo trembled from head to foot and as if his determination had been taken with the rapidity of lightning, he struck one of the panes of glass with his elbow; the glass was shivered to atoms, then withdrawing the curtain he saw Morrel, who had been writing at his desk, bound from his seat at the noise of the broken window. 

 <I beg a thousand pardons,> said the count, <there is nothing the matter, but I slipped down and broke one of your panes of glass with my elbow. Since it is opened, I will take advantage of it to enter your room; do not disturb yourself—do not disturb yourself!> 

 And passing his hand through the broken glass, the count opened the door. Morrel, evidently discomposed, came to meet Monte Cristo less with the intention of receiving him than to exclude his entry. 

 <\textit{Ma foi},> said Monte Cristo, rubbing his elbow, <it's all your servant's fault; your stairs are so polished, it is like walking on glass.> 

 <Are you hurt, sir?> coldly asked Morrel. 

 <I believe not. But what are you about there? You were writing.> 

 <I?> 

 <Your fingers are stained with ink.> 

 <Ah, true, I was writing. I do sometimes, soldier though I am.> 

 Monte Cristo advanced into the room; Maximilian was obliged to let him pass, but he followed him. 

 <You were writing?> said Monte Cristo with a searching look. 

 <I have already had the honour of telling you I was,> said Morrel. 

 The count looked around him. 

 <Your pistols are beside your desk,> said Monte Cristo, pointing with his finger to the pistols on the table. 

 <I am on the point of starting on a journey,> replied Morrel disdainfully. 

 <My friend,> exclaimed Monte Cristo in a tone of exquisite sweetness. 

 <Sir?> 

 <My friend, my dear Maximilian, do not make a hasty resolution, I entreat you.> 

 <I make a hasty resolution?> said Morrel, shrugging his shoulders; <is there anything extraordinary in a journey?>

<Maximilian,> said the count, <let us both lay aside the mask we have assumed. You no more deceive me with that false calmness than I impose upon you with my frivolous solicitude. You can understand, can you not, that to have acted as I have done, to have broken that glass, to have intruded on the solitude of a friend—you can understand that, to have done all this, I must have been actuated by real uneasiness, or rather by a terrible conviction. Morrel, you are going to destroy yourself!> 

 <Indeed, count,> said Morrel, shuddering; <what has put this into your head?> 

 <I tell you that you are about to destroy yourself,> continued the count, <and here is proof of what I say;> and, approaching the desk, he removed the sheet of paper which Morrel had placed over the letter he had begun, and took the latter in his hands. 

 Morrel rushed forward to tear it from him, but Monte Cristo perceiving his intention, seized his wrist with his iron grasp. 

 <You wish to destroy yourself,> said the count; <you have written it.> 

 <Well,> said Morrel, changing his expression of calmness for one of violence—<well, and if I do intend to turn this pistol against myself, who shall prevent me—who will dare prevent me? All my hopes are blighted, my heart is broken, my life a burden, everything around me is sad and mournful; earth has become distasteful to me, and human voices distract me. It is a mercy to let me die, for if I live I shall lose my reason and become mad. When, sir, I tell you all this with tears of heartfelt anguish, can you reply that I am wrong, can you prevent my putting an end to my miserable existence? Tell me, sir, could you have the courage to do so?> 

 <Yes, Morrel,> said Monte Cristo, with a calmness which contrasted strangely with the young man's excitement; <yes, I would do so.> 

 <You?> exclaimed Morrel, with increasing anger and reproach—<you, who have deceived me with false hopes, who have cheered and soothed me with vain promises, when I might, if not have saved her, at least have seen her die in my arms! You, who pretend to understand everything, even the hidden sources of knowledge,—and who enact the part of a guardian angel upon earth, and could not even find an antidote to a poison administered to a young girl! Ah, sir, indeed you would inspire me with pity, were you not hateful in my eyes.> 

 <Morrel\longdash> 

 <Yes; you tell me to lay aside the mask, and I will do so, be satisfied! When you spoke to me at the cemetery, I answered you—my heart was softened; when you arrived here, I allowed you to enter. But since you abuse my confidence, since you have devised a new torture after I thought I had exhausted them all, then, Count of Monte Cristo my pretended benefactor—then, Count of Monte Cristo, the universal guardian, be satisfied, you shall witness the death of your friend;> and Morrel, with a maniacal laugh, again rushed towards the pistols. 

 <And I again repeat, you shall not commit suicide.> 

 <Prevent me, then!> replied Morrel, with another struggle, which, like the first, failed in releasing him from the count's iron grasp. 

 <I will prevent you.>

<And who are you, then, that arrogate to yourself this tyrannical right over free and rational beings?> 

 <Who am I?> repeated Monte Cristo. <Listen; I am the only man in the world having the right to say to you, <Morrel, your father's son shall not die today;>> and Monte Cristo, with an expression of majesty and sublimity, advanced with arms folded toward the young man, who, involuntarily overcome by the commanding manner of this man, recoiled a step. 

 <Why do you mention my father?> stammered he; <why do you mingle a recollection of him with the affairs of today?> 

 <Because I am he who saved your father's life when he wished to destroy himself, as you do today—because I am the man who sent the purse to your young sister, and the \textit{Pharaon} to old Morrel—because I am the Edmond Dantès who nursed you, a child, on my knees.> 

 Morrel made another step back, staggering, breathless, crushed; then all his strength give way, and he fell prostrate at the feet of Monte Cristo. Then his admirable nature underwent a complete and sudden revulsion; he arose, rushed out of the room and to the stairs, exclaiming energetically, <Julie, Julie—Emmanuel, Emmanuel!> 

 Monte Cristo endeavoured also to leave, but Maximilian would have died rather than relax his hold of the handle of the door, which he closed upon the count. Julie, Emmanuel, and some of the servants, ran up in alarm on hearing the cries of Maximilian. Morrel seized their hands, and opening the door exclaimed in a voice choked with sobs: 

 <On your knees—on your knees—he is our benefactor—the saviour of our father! He is\longdash> 

 He would have added <Edmond Dantès,> but the count seized his arm and prevented him. 

 Julie threw herself into the arms of the count; Emmanuel embraced him as a guardian angel; Morrel again fell on his knees, and struck the ground with his forehead. Then the iron-hearted man felt his heart swell in his breast; a flame seemed to rush from his throat to his eyes, he bent his head and wept. For a while nothing was heard in the room but a succession of sobs, while the incense from their grateful hearts mounted to heaven. Julie had scarcely recovered from her deep emotion when she rushed out of the room, descended to the next floor, ran into the drawing-room with childlike joy and raised the crystal globe which covered the purse given by the unknown of the Allées de Meilhan. Meanwhile, Emmanuel in a broken voice said to the count: 

 <Oh, count, how could you, hearing us so often speak of our unknown benefactor, seeing us pay such homage of gratitude and adoration to his memory,—how could you continue so long without discovering yourself to us? Oh, it was cruel to us, and—dare I say it?—to you also.> 

 <Listen, my friends,> said the count—<I may call you so since we have really been friends for the last eleven years—the discovery of this secret has been occasioned by a great event which you must never know. I wished to bury it during my whole life in my own bosom, but your brother Maximilian wrested it from me by a violence he repents of now, I am sure.> 

 Then turning around, and seeing that Morrel, still on his knees, had thrown himself into an armchair, he added in a low voice, pressing Emmanuel's hand significantly, <Watch over him.> 

 <Why so?> asked the young man, surprised. 

 <I cannot explain myself; but watch over him.> Emmanuel looked around the room and caught sight of the pistols; his eyes rested on the weapons, and he pointed to them. Monte Cristo bent his head. Emmanuel went towards the pistols. 

 <Leave them,> said Monte Cristo. Then walking towards Morrel, he took his hand; the tumultuous agitation of the young man was succeeded by a profound stupor. Julie returned, holding the silken purse in her hands, while tears of joy rolled down her cheeks, like dewdrops on the rose. 

 <Here is the relic,> she said; <do not think it will be less dear to us now we are acquainted with our benefactor!> 

 <My child,> said Monte Cristo, coloring, <allow me to take back that purse? Since you now know my face, I wish to be remembered alone through the affection I hope you will grant me.>

 <Oh,> said Julie, pressing the purse to her heart, <no, no, I beseech you do not take it, for some unhappy day you will leave us, will you not?> 

 <You have guessed rightly, madame,> replied Monte Cristo, smiling; <in a week I shall have left this country, where so many persons who merit the vengeance of Heaven lived happily, while my father perished of hunger and grief.> 

 While announcing his departure, the count fixed his eyes on Morrel, and remarked that the words, <I shall have left this country,> had failed to rouse him from his lethargy. He then saw that he must make another struggle against the grief of his friend, and taking the hands of Emmanuel and Julie, which he pressed within his own, he said with the mild authority of a father: 

 <My kind friends, leave me alone with Maximilian.> 

 Julie saw the means offered of carrying off her precious relic, which Monte Cristo had forgotten. She drew her husband to the door. <Let us leave them,> she said. 

 The count was alone with Morrel, who remained motionless as a statue. 

 <Come,> said Monte-Cristo, touching his shoulder with his finger, <are you a man again, Maximilian?> 

 <Yes; for I begin to suffer again.> 

 The count frowned, apparently in gloomy hesitation. 

 <Maximilian, Maximilian,> he said, <the ideas you yield to are unworthy of a Christian.> 

 <Oh, do not fear, my friend,> said Morrel, raising his head, and smiling with a sweet expression on the count; <I shall no longer attempt my life.> 

 <Then we are to have no more pistols—no more despair?> 

 <No; I have found a better remedy for my grief than either a bullet or a knife.> 

 <Poor fellow, what is it?> 

 <My grief will kill me of itself.> 

 <My friend,> said Monte Cristo, with an expression of melancholy equal to his own, <listen to me. One day, in a moment of despair like yours, since it led to a similar resolution, I also wished to kill myself; one day your father, equally desperate, wished to kill himself too. If anyone had said to your father, at the moment he raised the pistol to his head—if anyone had told me, when in my prison I pushed back the food I had not tasted for three days—if anyone had said to either of us then, <Live—the day will come when you will be happy, and will bless life!>—no matter whose voice had spoken, we should have heard him with the smile of doubt, or the anguish of incredulity,—and yet how many times has your father blessed life while embracing you—how often have I myself\longdash> 

 <Ah,> exclaimed Morrel, interrupting the count, <you had only lost your liberty, my father had only lost his fortune, but I have lost Valentine.> 

 <Look at me,> said Monte Cristo, with that expression which sometimes made him so eloquent and persuasive—<look at me. There are no tears in my eyes, nor is there fever in my veins, yet I see you suffer—you, Maximilian, whom I love as my own son. Well, does not this tell you that in grief, as in life, there is always something to look forward to beyond? Now, if I entreat, if I order you to live, Morrel, it is in the conviction that one day you will thank me for having preserved your life.> 

 <Oh, heavens,> said the young man, <oh, heavens—what are you saying, count? Take care. But perhaps you have never loved!> 

 <Child!> replied the count. 

 <I mean, as I love. You see, I have been a soldier ever since I attained manhood. I reached the age of twenty-nine without loving, for none of the feelings I before then experienced merit the appellation of love. Well, at twenty-nine I saw Valentine; for two years I have loved her, for two years I have seen written in her heart, as in a book, all the virtues of a daughter and wife. Count, to possess Valentine would have been a happiness too infinite, too ecstatic, too complete, too divine for this world, since it has been denied me; but without Valentine the earth is desolate.> 

 <I have told you to hope,> said the count. 

 <Then have a care, I repeat, for you seek to persuade me, and if you succeed I should lose my reason, for I should hope that I could again behold Valentine.> 

 The count smiled. 

 <My friend, my father,> said Morrel with excitement, <have a care, I again repeat, for the power you wield over me alarms me. Weigh your words before you speak, for my eyes have already become brighter, and my heart beats strongly; be cautious, or you will make me believe in supernatural agencies. I must obey you, though you bade me call forth the dead or walk upon the water.> 

 <Hope, my friend,> repeated the count. 

 <Ah,> said Morrel, falling from the height of excitement to the abyss of despair—<ah, you are playing with me, like those good, or rather selfish mothers who soothe their children with honeyed words, because their screams annoy them. No, my friend, I was wrong to caution you; do not fear, I will bury my grief so deep in my heart, I will disguise it so, that you shall not even care to sympathize with me. Adieu, my friend, adieu!> 

 <On the contrary,> said the count, <after this time you must live with me—you must not leave me, and in a week we shall have left France behind us.> 

 <And you still bid me hope?> 

 <I tell you to hope, because I have a method of curing you.> 

 <Count, you render me sadder than before, if it be possible. You think the result of this blow has been to produce an ordinary grief, and you would cure it by an ordinary remedy—change of scene.> And Morrel dropped his head with disdainful incredulity. 

 <What can I say more?> asked Monte Cristo. <I have confidence in the remedy I propose, and only ask you to permit me to assure you of its efficacy.> 

 <Count, you prolong my agony.> 

 <Then,> said the count, <your feeble spirit will not even grant me the trial I request? Come—do you know of what the Count of Monte Cristo is capable? do you know that he holds terrestrial beings under his control? nay, that he can almost work a miracle? Well, wait for the miracle I hope to accomplish, or\longdash> 

 <Or?> repeated Morrel. 

 <Or, take care, Morrel, lest I call you ungrateful.> 

 <Have pity on me, count!> 

 <I feel so much pity towards you, Maximilian, that—listen to me attentively—if I do not cure you in a month, to the day, to the very hour, mark my words, Morrel, I will place loaded pistols before you, and a cup of the deadliest Italian poison—a poison more sure and prompt than that which has killed Valentine.> 

 <Will you promise me?> 

 <Yes; for I am a man, and have suffered like yourself, and also contemplated suicide; indeed, often since misfortune has left me I have longed for the delights of an eternal sleep.> 

 <But you are sure you will promise me this?> said Morrel, intoxicated. 

 <I not only promise, but swear it!> said Monte Cristo extending his hand. 

 <In a month, then, on your honour, if I am not consoled, you will let me take my life into my own hands, and whatever may happen you will not call me ungrateful?> 

 <In a month, to the day, the very hour and the date is a sacred one, Maximilian. I do not know whether you remember that this is the 5th of September; it is ten years today since I saved your father's life, who wished to die.> 

 Morrel seized the count's hand and kissed it; the count allowed him to pay the homage he felt due to him. 

 <In a month you will find on the table, at which we shall be then sitting, good pistols and a delicious draught; but, on the other hand, you must promise me not to attempt your life before that time.> 

 <Oh, I also swear it!> 

 Monte Cristo drew the young man towards him, and pressed him for some time to his heart. <And now,> he said, <after today, you will come and live with me; you can occupy Haydée's apartment, and my daughter will at least be replaced by my son.> 

 <Haydée?> said Morrel, <what has become of her?> 

 <She departed last night.> 

 <To leave you?> 

 <To wait for me. Hold yourself ready then to join me at the Champs-Élysées, and lead me out of this house without anyone seeing my departure.> 

 Maximilian hung his head, and obeyed with childlike reverence. 