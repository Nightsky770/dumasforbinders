\chapter{The Insult} 

 \lettrine{A}{t} the banker's door Beauchamp stopped Morcerf. 

\zz
 <Listen,> said he; <just now I told you it was of M. de Monte Cristo you must demand an explanation.> 

\zz
 <Yes; and we are going to his house.> 

 <Reflect, Morcerf, one moment before you go.> 

 <On what shall I reflect?> 

 <On the importance of the step you are taking.> 

 <Is it more serious than going to M. Danglars?> 

 <Yes; M. Danglars is a money-lover, and those who love money, you know, think too much of what they risk to be easily induced to fight a duel. The other is, on the contrary, to all appearance a true nobleman; but do you not fear to find him a bully?> 

 <I only fear one thing; namely, to find a man who will not fight.> 

 <Do not be alarmed,> said Beauchamp; <he will meet you. My only fear is that he will be too strong for you.> 

 <My friend,> said Morcerf, with a sweet smile, <that is what I wish. The happiest thing that could occur to me, would be to die in my father's stead; that would save us all.> 

 <Your mother would die of grief.> 

 <My poor mother!> said Albert, passing his hand across his eyes, <I know she would; but better so than die of shame.> 

 <Are you quite decided, Albert?> 

 <Yes; let us go.> 

 <But do you think we shall find the count at home?> 

 <He intended returning some hours after me, and doubtless he is now at home.> 

 They ordered the driver to take them to № 30 Champs-Élysées. Beauchamp wished to go in alone, but Albert observed that as this was an unusual circumstance he might be allowed to deviate from the usual etiquette of duels. The cause which the young man espoused was one so sacred that Beauchamp had only to comply with all his wishes; he yielded and contented himself with following Morcerf. Albert sprang from the porter's lodge to the steps. He was received by Baptistin. The count had, indeed, just arrived, but he was in his bath, and had forbidden that anyone should be admitted. 

 <But after his bath?> asked Morcerf. 

 <My master will go to dinner.> 

 <And after dinner?> 

 <He will sleep an hour.> 

 <Then?> 

 <He is going to the Opera.> 

 <Are you sure of it?> asked Albert. 

 <Quite, sir; my master has ordered his horses at eight o'clock precisely.> 

 <Very good,> replied Albert; <that is all I wished to know.> 

 Then, turning towards Beauchamp, <If you have anything to attend to, Beauchamp, do it directly; if you have any appointment for this evening, defer it till tomorrow. I depend on you to accompany me to the Opera; and if you can, bring Château-Renaud with you.> 

 Beauchamp availed himself of Albert's permission, and left him, promising to call for him at a quarter before eight. On his return home, Albert expressed his wish to Franz Debray, and Morrel, to see them at the Opera that evening. Then he went to see his mother, who since the events of the day before had refused to see anyone, and had kept her room. He found her in bed, overwhelmed with grief at this public humiliation. 

 The sight of Albert produced the effect which might naturally be expected on Mercédès; she pressed her son's hand and sobbed aloud, but her tears relieved her. Albert stood one moment speechless by the side of his mother's bed. It was evident from his pale face and knit brows that his resolution to revenge himself was growing weaker. 

 <My dear mother,> said he, <do you know if M. de Morcerf has any enemy?> 

 Mercédès started; she noticed that the young man did not say <my father.> 

 <My son,> she said, <persons in the count's situation have many secret enemies. Those who are known are not the most dangerous.> 

 <I know it, and appeal to your penetration. You are of so superior a mind, nothing escapes you.> 

 <Why do you say so?> 

 <Because, for instance, you noticed on the evening of the ball we gave, that M. de Monte Cristo would eat nothing in our house.> 

 Mercédès raised herself on her feverish arm. 

 <M. de Monte Cristo!> she exclaimed; <and how is he connected with the question you asked me?>

<You know, mother, M. de Monte Cristo is almost an Oriental, and it is customary with the Orientals to secure full liberty for revenge by not eating or drinking in the houses of their enemies.> 

 <Do you say M. de Monte Cristo is our enemy?> replied Mercédès, becoming paler than the sheet which covered her. <Who told you so? Why, you are mad, Albert! M. de Monte Cristo has only shown us kindness. M. de Monte Cristo saved your life; you yourself presented him to us. Oh, I entreat you, my son, if you had entertained such an idea, dispel it; and my counsel to you—nay, my prayer—is to retain his friendship.> 

 <Mother,> replied the young man, <you have special reasons for telling me to conciliate that man.> 

 <I?> said Mercédès, blushing as rapidly as she had turned pale, and again becoming paler than ever. 

 <Yes, doubtless; and is it not that he may never do us any harm?> 

 Mercédès shuddered, and, fixing on her son a scrutinizing gaze, <You speak strangely,> said she to Albert, <and you appear to have some singular prejudices. What has the count done? Three days since you were with him in Normandy; only three days since we looked on him as our best friend.> 

 An ironical smile passed over Albert's lips. Mercédès saw it and with the double instinct of woman and mother guessed all; but as she was prudent and strong-minded she concealed both her sorrows and her fears. Albert was silent; an instant after, the countess resumed: 

 <You came to inquire after my health; I will candidly acknowledge that I am not well. You should install yourself here, and cheer my solitude. I do not wish to be left alone.> 

 <Mother,> said the young man, <you know how gladly I would obey your wish, but an urgent and important affair obliges me to leave you for the whole evening.> 

 <Well,> replied Mercédès, sighing, <go, Albert; I will not make you a slave to your filial piety.> 

 Albert pretended he did not hear, bowed to his mother, and quitted her. Scarcely had he shut her door, when Mercédès called a confidential servant, and ordered him to follow Albert wherever he should go that evening, and to come and tell her immediately what he observed. Then she rang for her lady's maid, and, weak as she was, she dressed, in order to be ready for whatever might happen. The footman's mission was an easy one. Albert went to his room, and dressed with unusual care. At ten minutes to eight Beauchamp arrived; he had seen Château-Renaud, who had promised to be in the orchestra before the curtain was raised. Both got into Albert's \textit{coupé}; and, as the young man had no reason to conceal where he was going, he called aloud, <To the Opera.> In his impatience he arrived before the beginning of the performance.  Château-Renaud was at his post; apprised by Beauchamp of the circumstances, he required no explanation from Albert. The conduct of the son in seeking to avenge his father was so natural that Château-Renaud did not seek to dissuade him, and was content with renewing his assurances of devotion. Debray was not yet come, but Albert knew that he seldom lost a scene at the Opera. 

 Albert wandered about the theatre until the curtain was drawn up. He hoped to meet with M. de Monte Cristo either in the lobby or on the stairs. The bell summoned him to his seat, and he entered the orchestra with Château-Renaud and Beauchamp. But his eyes scarcely quitted the box between the columns, which remained obstinately closed during the whole of the first act. At last, as Albert was looking at his watch for about the hundredth time, at the beginning of the second act the door opened, and Monte Cristo entered, dressed in black, and, leaning over the front of the box, looked around the pit. Morrel followed him, and looked also for his sister and brother in-law; he soon discovered them in another box, and kissed his hand to them. 

 The count, in his survey of the pit, encountered a pale face and threatening eyes, which evidently sought to gain his attention. He recognized Albert, but thought it better not to notice him, as he looked so angry and discomposed. Without communicating his thoughts to his companion, he sat down, drew out his opera-glass, and looked another way. Although apparently not noticing Albert, he did not, however, lose sight of him, and when the curtain fell at the end of the second act, he saw him leave the orchestra with his two friends. Then his head was seen passing at the back of the boxes, and the count knew that the approaching storm was intended to fall on him. He was at the moment conversing cheerfully with Morrel, but he was well prepared for what might happen. 

 The door opened, and Monte Cristo, turning round, saw Albert, pale and trembling, followed by Beauchamp and Château-Renaud. 

 <Well,> cried he, with that benevolent politeness which distinguished his salutation from the common civilities of the world, <my cavalier has attained his object. Good-evening, M. de Morcerf.> 

 The countenance of this man, who possessed such extraordinary control over his feelings, expressed the most perfect cordiality. Morrel only then recollected the letter he had received from the viscount, in which, without assigning any reason, he begged him to go to the Opera, but he understood that something terrible was brooding. 

 <We are not come here, sir, to exchange hypocritical expressions of politeness, or false professions of friendship,> said Albert, <but to demand an explanation.> 

 The young man's trembling voice was scarcely audible. 

 <An explanation at the Opera?> said the count, with that calm tone and penetrating eye which characterize the man who knows his cause is good. <Little acquainted as I am with the habits of Parisians, I should not have thought this the place for such a demand.> 

 <Still, if people will shut themselves up,> said Albert, <and cannot be seen because they are bathing, dining, or asleep, we must avail ourselves of the opportunity whenever they are to be seen.> 

 <I am not difficult of access, sir; for yesterday, if my memory does not deceive me, you were at my house.> 

 <Yesterday I was at your house, sir,> said the young man; <because then I knew not who you were.> 

 In pronouncing these words Albert had raised his voice so as to be heard by those in the adjoining boxes and in the lobby. Thus the attention of many was attracted by this altercation. 

<Where are you come from, sir?> said Monte Cristo <You do not appear to be in the possession of your senses.> 

 <Provided I understand your perfidy, sir, and succeed in making you understand that I will be revenged, I shall be reasonable enough,> said Albert furiously. 

 <I do not understand you, sir,> replied Monte Cristo; <and if I did, your tone is too high. I am at home here, and I alone have a right to raise my voice above another's. Leave the box, sir!> 

 Monte Cristo pointed towards the door with the most commanding dignity. 

 <Ah, I shall know how to make you leave your home!> replied Albert, clasping in his convulsed grasp the glove, which Monte Cristo did not lose sight of. 

 <Well, well,> said Monte Cristo quietly, <I see you wish to quarrel with me; but I would give you one piece of advice, which you will do well to keep in mind. It is in poor taste to make a display of a challenge. Display is not becoming to everyone, M. de Morcerf.> 

 At this name a murmur of astonishment passed around the group of spectators of this scene. They had talked of no one but Morcerf the whole day. Albert understood the allusion in a moment, and was about to throw his glove at the count, when Morrel seized his hand, while Beauchamp and Château-Renaud, fearing the scene would surpass the limits of a challenge, held him back. But Monte Cristo, without rising, and leaning forward in his chair, merely stretched out his arm and, taking the damp, crushed glove from the clenched hand of the young man: 

 <Sir,> said he in a solemn tone, <I consider your glove thrown, and will return it to you wrapped around a bullet. Now leave me or I will summon my servants to throw you out at the door.> 

 Wild, almost unconscious, and with eyes inflamed, Albert stepped back, and Morrel closed the door. Monte Cristo took up his glass again as if nothing had happened; his face was like marble, and his heart was like bronze. Morrel whispered, <What have you done to him?> 

 <I? Nothing—at least personally,> said Monte Cristo. 

 <But there must be some cause for this strange scene.> 

 <The Count of Morcerf's adventure exasperates the young man.> 

 <Have you anything to do with it?> 

 <It was through Haydée that the Chamber was informed of his father's treason.> 

 <Indeed?> said Morrel. <I had been told, but would not credit it, that the Grecian slave I have seen with you here in this very box was the daughter of Ali Pasha.> 

 <It is true, nevertheless.> 

 <Then,> said Morrel, <I understand it all, and this scene was premeditated.> 

 <How so?> 

 <Yes. Albert wrote to request me to come to the Opera, doubtless that I might be a witness to the insult he meant to offer you.> 

 <Probably,> said Monte Cristo with his imperturbable tranquillity. 

 <But what shall you do with him?> 

 <With whom?> 

 <With Albert.> 

 <What shall I do with Albert? As certainly, Maximilian, as I now press your hand, I shall kill him before ten o'clock tomorrow morning.> Morrel, in his turn, took Monte Cristo's hand in both of his, and he shuddered to feel how cold and steady it was. 

 <Ah, count,> said he, <his father loves him so much!> 

 <Do not speak to me of that,> said Monte Cristo, with the first movement of anger he had betrayed; <I will make him suffer.> 

 Morrel, amazed, let fall Monte Cristo's hand. <Count, count!> said he. 

 <Dear Maximilian,> interrupted the count, `listen how adorably Duprez is singing that line,—  
 \begin{verse}
 O Mathilde! idole de mon âme!'
 \end{verse}

 <I was the first to discover Duprez at Naples, and the first to applaud him. Bravo, bravo!> 

 Morrel saw it was useless to say more, and refrained. The curtain, which had risen at the close of the scene with Albert, again fell, and a rap was heard at the door. 

 <Come in,> said Monte Cristo with a voice that betrayed not the least emotion; and immediately Beauchamp appeared. <Good-evening, M. Beauchamp,> said Monte Cristo, as if this was the first time he had seen the journalist that evening; <be seated.> 

 Beauchamp bowed, and, sitting down, <Sir,> said he, <I just now accompanied M. de Morcerf, as you saw.> 

 <And that means,> replied Monte Cristo, laughing, <that you had, probably, just dined together. I am happy to see, M. Beauchamp, that you are more sober than he was.> 

 <Sir,> said M. Beauchamp, <Albert was wrong, I acknowledge, to betray so much anger, and I come, on my own account, to apologize for him. And having done so, entirely on my own account, be it understood, I would add that I believe you too gentlemanly to refuse giving him some explanation concerning your connection with Yanina. Then I will add two words about the young Greek girl.> 

 Monte Cristo motioned him to be silent. <Come,> said he, laughing, <there are all my hopes about to be destroyed.> 

 <How so?> asked Beauchamp. 

 <Doubtless you wish to make me appear a very eccentric character. I am, in your opinion, a Lara, a Manfred, a Lord Ruthven; then, just as I am arriving at the climax, you defeat your own end, and seek to make an ordinary man of me. You bring me down to your own level, and demand explanations! Indeed, M. Beauchamp, it is quite laughable.> 

 <Yet,> replied Beauchamp haughtily, <there are occasions when probity commands\longdash> 

 <M. Beauchamp,> interposed this strange man, <the Count of Monte Cristo bows to none but the Count of Monte Cristo himself. Say no more, I entreat you. I do what I please, M. Beauchamp, and it is always well done.> 

 <Sir,> replied the young man, <honest men are not to be paid with such coin. I require honourable guaranties.> 

 <I am, sir, a living guaranty,> replied Monte Cristo, motionless, but with a threatening look; <we have both blood in our veins which we wish to shed—that is our mutual guaranty. Tell the viscount so, and that tomorrow, before ten o'clock, I shall see what colour his is.> 

 <Then I have only to make arrangements for the duel,> said Beauchamp. 

 <It is quite immaterial to me,> said Monte Cristo, <and it was very unnecessary to disturb me at the Opera for such a trifle. In France people fight with the sword or pistol, in the colonies with the carbine, in Arabia with the dagger. Tell your client that, although I am the insulted party, in order to carry out my eccentricity, I leave him the choice of arms, and will accept without discussion, without dispute, anything, even combat by drawing lots, which is always stupid, but with me different from other people, as I am sure to gain.> 

 <Sure to gain!> repeated Beauchamp, looking with amazement at the count. 

 <Certainly,> said Monte Cristo, slightly shrugging his shoulders; <otherwise I would not fight with M. de Morcerf. I shall kill him—I cannot help it. Only by a single line this evening at my house let me know the arms and the hour; I do not like to be kept waiting.> 

 <Pistols, then, at eight o'clock, in the Bois de Vincennes,> said Beauchamp, quite disconcerted, not knowing if he was dealing with an arrogant braggadocio or a supernatural being. 

 <Very well, sir,> said Monte Cristo. <Now all that is settled, do let me see the performance, and tell your friend Albert not to come any more this evening; he will hurt himself with all his ill-chosen barbarisms: let him go home and go to sleep.> 

 Beauchamp left the box, perfectly amazed. 

 <Now,> said Monte Cristo, turning towards Morrel, <I may depend upon you, may I not?> 

 <Certainly,> said Morrel, <I am at your service, count; still\longdash> 

 <What?> 

 <It is desirable I should know the real cause.> 

 <That is to say, you would rather not?> 

 <No.> 

 <The young man himself is acting blindfolded, and knows not the true cause, which is known only to God and to me; but I give you my word, Morrel, that God, who does know it, will be on our side.> 

 <Enough,> said Morrel; <who is your second witness?> 

 <I know no one in Paris, Morrel, on whom I could confer that honour besides you and your brother Emmanuel. Do you think Emmanuel would oblige me?> 

 <I will answer for him, count.> 

 <Well? that is all I require. Tomorrow morning, at seven o'clock, you will be with me, will you not?> 

 <We will.> 

 <Hush, the curtain is rising. Listen! I never lose a note of this opera if I can avoid it; the music of \textit{William Tell} is so sweet.> 