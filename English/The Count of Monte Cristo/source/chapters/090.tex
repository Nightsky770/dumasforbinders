\chapter{The Meeting} 

 \lettrine{A}{fter} Mercédès had left Monte Cristo, he fell into profound gloom. Around him and within him the flight of thought seemed to have stopped; his energetic mind slumbered, as the body does after extreme fatigue. 

 <What?> said he to himself, while the lamp and the wax lights were nearly burnt out, and the servants were waiting impatiently in the anteroom; “what? this edifice which I have been so long preparing, which I have reared with so much care and toil, is to be crushed by a single touch, a word, a breath! Yes, this self, of whom I thought so much, of whom I was so proud, who had appeared so worthless in the dungeons of the Château d'If, and whom I had succeeded in making so great, will be but a lump of clay tomorrow. Alas, it is not the death of the body I regret; for is not the destruction of the vital principle, the repose to which everything is tending, to which every unhappy being aspires,—is not this the repose of matter after which I so long sighed, and which I was seeking to attain by the painful process of starvation when Faria appeared in my dungeon? What is death for me? One step farther into rest,—two, perhaps, into silence. No, it is not existence, then, that I regret, but the ruin of projects so slowly carried out, so laboriously framed. Providence is now opposed to them, when I most thought it would be propitious. It is not God's will that they should be accomplished. This burden, almost as heavy as a world, which I had raised, and I had thought to bear to the end, was too great for my strength, and I was compelled to lay it down in the middle of my career. Oh, shall I then, again become a fatalist, whom fourteen years of despair and ten of hope had rendered a believer in Providence? 

 “And all this—all this, because my heart, which I thought dead, was only sleeping; because it has awakened and has begun to beat again, because I have yielded to the pain of the emotion excited in my breast by a woman's voice. 

 <Yet,> continued the count, becoming each moment more absorbed in the anticipation of the dreadful sacrifice for the morrow, which Mercédès had accepted, <yet, it is impossible that so noble-minded a woman should thus through selfishness consent to my death when I am in the prime of life and strength; it is impossible that she can carry to such a point maternal love, or rather delirium. There are virtues which become crimes by exaggeration. No, she must have conceived some pathetic scene; she will come and throw herself between us; and what would be sublime here will there appear ridiculous.> 

 The blush of pride mounted to the count's forehead as this thought passed through his mind. 

 <Ridiculous?> repeated he; <and the ridicule will fall on me. I ridiculous? No, I would rather die.> 

 By thus exaggerating to his own mind the anticipated ill-fortune of the next day, to which he had condemned himself by promising Mercédès to spare her son, the count at last exclaimed: 

 <Folly, folly, folly!—to carry generosity so far as to put myself up as a mark for that young man to aim at. He will never believe that my death was suicide; and yet it is important for the honour of my memory,—and this surely is not vanity, but a justifiable pride,—it is important the world should know that I have consented, by my free will, to stop my arm, already raised to strike, and that with the arm which has been so powerful against others I have struck myself. It must be; it shall be.> 

 Seizing a pen, he drew a paper from a secret drawer in his desk, and wrote at the bottom of the document (which was no other than his will, made since his arrival in Paris) a sort of codicil, clearly explaining the nature of his death. 

 <I do this, Oh, my God,> said he, with his eyes raised to heaven, <as much for thy honour as for mine. I have during ten years considered myself the agent of thy vengeance, and other wretches, like Morcerf, Danglars, Villefort, even Morcerf himself, must not imagine that chance has freed them from their enemy. Let them know, on the contrary, that their punishment, which had been decreed by Providence, is only delayed by my present determination, and although they escape it in this world, it awaits them in another, and that they are only exchanging time for eternity.> 

 While he was thus agitated by gloomy uncertainties,—wretched waking dreams of grief,—the first rays of morning pierced his windows, and shone upon the pale blue paper on which he had just inscribed his justification of Providence. 

 It was just five o'clock in the morning when a slight noise like a stifled sigh reached his ear. He turned his head, looked around him, and saw no one; but the sound was repeated distinctly enough to convince him of its reality. 

 He arose, and quietly opening the door of the drawing-room, saw Haydée, who had fallen on a chair, with her arms hanging down and her beautiful head thrown back. She had been standing at the door, to prevent his going out without seeing her, until sleep, which the young cannot resist, had overpowered her frame, wearied as she was with watching. The noise of the door did not awaken her, and Monte Cristo gazed at her with affectionate regret. 

 <She remembered that she had a son,> said he; <and I forgot I had a daughter.> Then, shaking his head sorrowfully, <Poor Haydée,> said he; <she wished to see me, to speak to me; she has feared or guessed something. Oh, I cannot go without taking leave of her; I cannot die without confiding her to someone.> 

 He quietly regained his seat, and wrote under the other lines:  <p class="letter"> <I bequeath to Maximilian Morrel, captain of Spahis,—and son of my former patron, Pierre Morrel, shipowner at Marseilles,—the sum of twenty millions, a part of which may be offered to his sister Julie and brother-in-law Emmanuel, if he does not fear this increase of fortune may mar their happiness. These twenty millions are concealed in my grotto at Monte Cristo, of which Bertuccio knows the secret. If his heart is free, and he will marry Haydée, the daughter of Ali Pasha of Yanina, whom I have brought up with the love of a father, and who has shown the love and tenderness of a daughter for me, he will thus accomplish my last wish. This will has already constituted Haydée heiress of the rest of my fortune, consisting of lands, funds in England, Austria, and Holland, furniture in my different palaces and houses, and which without the twenty millions and the legacies to my servants, may still amount to sixty millions.>  He was finishing the last line when a cry behind him made him start, and the pen fell from his hand. 

 <Haydée,> said he, <did you read it?> 

 <Oh, my lord,> said she, <why are you writing thus at such an hour? Why are you bequeathing all your fortune to me? Are you going to leave me?> 

 <I am going on a journey, dear child,> said Monte Cristo, with an expression of infinite tenderness and melancholy; <and if any misfortune should happen to me\longdash> 

 The count stopped. 

 <Well?> asked the young girl, with an authoritative tone the count had never observed before, and which startled him. 

 <Well, if any misfortune happen to me,> replied Monte Cristo, <I wish my daughter to be happy.> Haydée smiled sorrowfully, and shook her head. 

 <Do you think of dying, my lord?> said she. 

 <The wise man, my child, has said, <It is good to think of death.>> 

 <Well, if you die,> said she, <bequeath your fortune to others, for if you die I shall require nothing;> and, taking the paper, she tore it in four pieces, and threw it into the middle of the room. Then, the effort having exhausted her strength, she fell, not asleep this time, but fainting on the floor. 

 The count leaned over her and raised her in his arms; and seeing that sweet pale face, those lovely eyes closed, that beautiful form motionless and to all appearance lifeless, the idea occurred to him for the first time, that perhaps she loved him otherwise than as a daughter loves a father. 

 <Alas,> murmured he, with intense suffering, <I might, then, have been happy yet.> 

 Then he carried Haydée to her room, resigned her to the care of her attendants, and returning to his study, which he shut quickly this time, he again copied the destroyed will. As he was finishing, the sound of a cabriolet entering the yard was heard. Monte Cristo approached the window, and saw Maximilian and Emmanuel alight. <Good,> said he; <it was time,>—and he sealed his will with three seals. 

 A moment afterwards he heard a noise in the drawing-room, and went to open the door himself. Morrel was there; he had come twenty minutes before the time appointed. 

 <I am perhaps come too soon, count,> said he, <but I frankly acknowledge that I have not closed my eyes all night, nor has anyone in my house. I need to see you strong in your courageous assurance, to recover myself.> 

 Monte Cristo could not resist this proof of affection; he not only extended his hand to the young man, but flew to him with open arms. 

 <Morrel,> said he, <it is a happy day for me, to feel that I am beloved by such a man as you. Good-morning, Emmanuel; you will come with me then, Maximilian?> 

 <Did you doubt it?> said the young captain. 

 <But if I were wrong\longdash> 

 <I watched you during the whole scene of that challenge yesterday; I have been thinking of your firmness all night, and I said to myself that justice must be on your side, or man's countenance is no longer to be relied on.> 

 <But, Morrel, Albert is your friend?> 

 <Simply an acquaintance, sir.> 

 <You met on the same day you first saw me?> 

 <Yes, that is true; but I should not have recollected it if you had not reminded me.> 

 <Thank you, Morrel.> Then ringing the bell once, <Look.> said he to Ali, who came immediately, <take that to my solicitor. It is my will, Morrel. When I am dead, you will go and examine it.> 

 <What?> said Morrel, <you dead?> 

 <Yes; must I not be prepared for everything, dear friend? But what did you do yesterday after you left me?> 

 <I went to Tortoni's, where, as I expected, I found Beauchamp and Château-Renaud. I own I was seeking them.> 

 <Why, when all was arranged?> 

 <Listen, count; the affair is serious and unavoidable.> 

 <Did you doubt it!> 

 <No; the offence was public, and everyone is already talking of it.> 

 <Well?> 

 <Well, I hoped to get an exchange of arms,—to substitute the sword for the pistol; the pistol is blind.> 

 <Have you succeeded?> asked Monte Cristo quickly, with an imperceptible gleam of hope. 

 <No; for your skill with the sword is so well known.> 

 <Ah?—who has betrayed me?> 

 <The skilful swordsman whom you have conquered.> 

 <And you failed?> 

 <They positively refused.> 

 <Morrel,> said the count, <have you ever seen me fire a pistol?> 

 <Never.> 

 <Well, we have time; look.> Monte Cristo took the pistols he held in his hand when Mercédès entered, and fixing an ace of clubs against the iron plate, with four shots he successively shot off the four sides of the club. At each shot Morrel turned pale. He examined the bullets with which Monte Cristo performed this dexterous feat, and saw that they were no larger than buckshot. 

 <It is astonishing,> said he. <Look, Emmanuel.> Then turning towards Monte Cristo, <Count,> said he, <in the name of all that is dear to you, I entreat you not to kill Albert!—the unhappy youth has a mother.> 

 <You are right,> said Monte Cristo; <and I have none.> These words were uttered in a tone which made Morrel shudder. 

 <You are the offended party, count.> 

 <Doubtless; what does that imply?> 

 <That you will fire first.> 

 <I fire first?> 

 <Oh, I obtained, or rather claimed that; we had conceded enough for them to yield us that.> 

 <And at what distance?> 

 <Twenty paces.> A smile of terrible import passed over the count's lips. 

 <Morrel,> said he, <do not forget what you have just seen.> 

 <The only chance for Albert's safety, then, will arise from your emotion.> 

 <I suffer from emotion?> said Monte Cristo. 

 <Or from your generosity, my friend; to so good a marksman as you are, I may say what would appear absurd to another.> 

 <What is that?> 

 <Break his arm—wound him—but do not kill him.> 

 <I will tell you, Morrel,> said the count, <that I do not need entreating to spare the life of M. de Morcerf; he shall be so well spared, that he will return quietly with his two friends, while I\longdash> 

 <And you?> 

 <That will be another thing; I shall be brought home.> 

 <No, no,> cried Maximilian, quite unable to restrain his feelings. 

 <As I told you, my dear Morrel, M. de Morcerf will kill me.> 

 Morrel looked at him in utter amazement. <But what has happened, then, since last evening, count?> 

 <The same thing that happened to Brutus the night before the battle of Philippi; I have seen a ghost.> 

 <And that ghost\longdash> 

 <Told me, Morrel, that I had lived long enough.> 

 Maximilian and Emmanuel looked at each other. Monte Cristo drew out his watch. <Let us go,> said he; <it is five minutes past seven, and the appointment was for eight o'clock.> 

 A carriage was in readiness at the door. Monte Cristo stepped into it with his two friends. He had stopped a moment in the passage to listen at a door, and Maximilian and Emmanuel, who had considerately passed forward a few steps, thought they heard him answer by a sigh to a sob from within. As the clock struck eight they drove up to the place of meeting. 

 <We are first,> said Morrel, looking out of the window. 

 <Excuse me, sir,> said Baptistin, who had followed his master with indescribable terror, <but I think I see a carriage down there under the trees.> 

 Monte Cristo sprang lightly from the carriage, and offered his hand to assist Emmanuel and Maximilian. The latter retained the count's hand between his. 

 <I like,> said he, <to feel a hand like this, when its owner relies on the goodness of his cause.> 

 <It seems to me,> said Emmanuel, <that I see two young men down there, who are evidently, waiting.> 

 Monte Cristo drew Morrel a step or two behind his brother-in-law. 

 <Maximilian,> said he, <are your affections disengaged?> Morrel looked at Monte Cristo with astonishment. <I do not seek your confidence, my dear friend. I only ask you a simple question; answer it;—that is all I require.> 

 <I love a young girl, count.> 

 <Do you love her much?> 

 <More than my life.> 

 <Another hope defeated!> said the count. Then, with a sigh, <Poor Haydée!> murmured he. 

 <To tell the truth, count, if I knew less of you, I should think that you were less brave than you are.> 

 <Because I sigh when thinking of someone I am leaving? Come, Morrel, it is not like a soldier to be so bad a judge of courage. Do I regret life? What is it to me, who have passed twenty years between life and death? Moreover, do not alarm yourself, Morrel; this weakness, if it is such, is betrayed to you alone. I know the world is a drawing-room, from which we must retire politely and honestly; that is, with a bow, and our debts of honour paid.> 

 <That is to the purpose. Have you brought your arms?> 

 <I?—what for? I hope these gentlemen have theirs.> 

 <I will inquire,> said Morrel. 

 <Do; but make no treaty—you understand me?> 

 <You need not fear.> Morrel advanced towards Beauchamp and Château-Renaud, who, seeing his intention, came to meet him. The three young men bowed to each other courteously, if not affably. 

 <Excuse me, gentlemen,> said Morrel, <but I do not see M. de Morcerf.> 

 <He sent us word this morning,> replied Château-Renaud, <that he would meet us on the ground.> 

 <Ah,> said Morrel. Beauchamp pulled out his watch. 

 <It is only five minutes past eight,> said he to Morrel; <there is not much time lost yet.> 

 <Oh, I made no allusion of that kind,> replied Morrel. 

 <There is a carriage coming,> said Château-Renaud. It advanced rapidly along one of the avenues leading towards the open space where they were assembled. 

 <You are doubtless provided with pistols, gentlemen? M. de Monte Cristo yields his right of using his.> 

 <We had anticipated this kindness on the part of the count,> said Beauchamp, <and I have brought some weapons which I bought eight or ten days since, thinking to want them on a similar occasion. They are quite new, and have not yet been used. Will you examine them.> 

 <Oh, M. Beauchamp, if you assure me that M. de Morcerf does not know these pistols, you may readily believe that your word will be quite sufficient.> 

 <Gentlemen,> said Château-Renaud, <it is not Morcerf coming in that carriage;—faith, it is Franz and Debray!> 

 The two young men he announced were indeed approaching. <What chance brings you here, gentlemen?> said Château-Renaud, shaking hands with each of them. 

 <Because,> said Debray, <Albert sent this morning to request us to come.> Beauchamp and Château-Renaud exchanged looks of astonishment. <I think I understand his reason,> said Morrel. 

 <What is it?> 

 <Yesterday afternoon I received a letter from M. de Morcerf, begging me to attend the Opera.> 

 <And I,> said Debray. 

 <And I also,> said Franz. 

 <And we, too,> added Beauchamp and Château-Renaud. 

 <Having wished you all to witness the challenge, he now wishes you to be present at the combat.> 

 <Exactly so,> said the young men; <you have probably guessed right.> 

 <But, after all these arrangements, he does not come himself,> said Château-Renaud. <Albert is ten minutes after time.> 

 <There he comes,> said Beauchamp, <on horseback, at full gallop, followed by a servant.> 

 <How imprudent,> said Château-Renaud, <to come on horseback to fight a duel with pistols, after all the instructions I had given him.> 

 <And besides,> said Beauchamp, <with a collar above his cravat, an open coat and white waistcoat! Why has he not painted a spot upon his heart?—it would have been more simple.> 

 Meanwhile Albert had arrived within ten paces of the group formed by the five young men. He jumped from his horse, threw the bridle on his servant's arms, and joined them. He was pale, and his eyes were red and swollen; it was evident that he had not slept. A shade of melancholy gravity overspread his countenance, which was not natural to him. 

 <I thank you, gentlemen,> said he, <for having complied with my request; I feel extremely grateful for this mark of friendship.> Morrel had stepped back as Morcerf approached, and remained at a short distance. <And to you also, M. Morrel, my thanks are due. Come, there cannot be too many.>  <Sir,> said Maximilian, <you are not perhaps aware that I am M. de Monte Cristo's friend?> 

 <I was not sure, but I thought it might be so. So much the better; the more honourable men there are here the better I shall be satisfied.> 

 <M. Morrel,> said Château-Renaud, <will you apprise the Count of Monte Cristo that M. de Morcerf is arrived, and we are at his disposal?> 

 Morrel was preparing to fulfil his commission. Beauchamp had meanwhile drawn the box of pistols from the carriage. 

 <Stop, gentlemen,> said Albert; <I have two words to say to the Count of Monte Cristo.> 

 <In private?> asked Morrel. 

 <No, sir; before all who are here.> 

 Albert's witnesses looked at each other. Franz and Debray exchanged some words in a whisper, and Morrel, rejoiced at this unexpected incident, went to fetch the count, who was walking in a retired path with Emmanuel. 

 <What does he want with me?> said Monte Cristo. 

 <I do not know, but he wishes to speak to you.> 

 <Ah?> said Monte Cristo, <I trust he is not going to tempt me by some fresh insult!> 

 <I do not think that such is his intention,> said Morrel. 

 The count advanced, accompanied by Maximilian and Emmanuel. His calm and serene look formed a singular contrast to Albert's grief-stricken face, who approached also, followed by the other four young men. 

 When at three paces distant from each other, Albert and the count stopped. 

 <Approach, gentlemen,> said Albert; <I wish you not to lose one word of what I am about to have the honour of saying to the Count of Monte Cristo, for it must be repeated by you to all who will listen to it, strange as it may appear to you.> 

 <Proceed, sir,> said the count. 

 <Sir,> said Albert, at first with a tremulous voice, but which gradually became firmer, <I reproached you with exposing the conduct of M. de Morcerf in Epirus, for guilty as I knew he was, I thought you had no right to punish him; but I have since learned that you had that right. It is not Fernand Mondego's treachery towards Ali Pasha which induces me so readily to excuse you, but the treachery of the fisherman Fernand towards you, and the almost unheard-of miseries which were its consequences; and I say, and proclaim it publicly, that you were justified in revenging yourself on my father, and I, his son, thank you for not using greater severity.> 

 Had a thunderbolt fallen in the midst of the spectators of this unexpected scene, it would not have surprised them more than did Albert's declaration. As for Monte Cristo, his eyes slowly rose towards heaven with an expression of infinite gratitude. He could not understand how Albert's fiery nature, of which he had seen so much among the Roman bandits, had suddenly stooped to this humiliation. He recognized the influence of Mercédès, and saw why her noble heart had not opposed the sacrifice she knew beforehand would be useless. 

 <Now, sir,> said Albert, <if you think my apology sufficient, pray give me your hand. Next to the merit of infallibility which you appear to possess, I rank that of candidly acknowledging a fault. But this confession concerns me only. I acted well as a man, but you have acted better than man. An angel alone could have saved one of us from death—that angel came from heaven, if not to make us friends (which, alas, fatality renders impossible), at least to make us esteem each other.> 

 Monte Cristo, with moistened eye, heaving breast, and lips half open, extended to Albert a hand which the latter pressed with a sentiment resembling respectful fear. 

 <Gentlemen,> said he, <M. de Monte Cristo receives my apology. I had acted hastily towards him. Hasty actions are generally bad ones. Now my fault is repaired. I hope the world will not call me cowardly for acting as my conscience dictated. But if anyone should entertain a false opinion of me,> added he, drawing himself up as if he would challenge both friends and enemies, <I shall endeavour to correct his mistake.> 

 <What happened during the night?> asked Beauchamp of Château-Renaud; <we appear to make a very sorry figure here.> 

 <In truth, what Albert has just done is either very despicable or very noble,> replied the baron. 

 <What can it mean?> said Debray to Franz. 

 <The Count of Monte Cristo acts dishonorably to M. de Morcerf, and is justified by his son! Had I ten Yaninas in my family, I should only consider myself the more bound to fight ten times.> 

 As for Monte Cristo, his head was bent down, his arms were powerless. Bowing under the weight of twenty-four years' reminiscences, he thought not of Albert, of Beauchamp, of Château-Renaud, or of any of that group; but he thought of that courageous woman who had come to plead for her son's life, to whom he had offered his, and who had now saved it by the revelation of a dreadful family secret, capable of destroying forever in that young man's heart every feeling of filial piety. 

 <Providence still,> murmured he; <now only am I fully convinced of being the emissary of God!> 