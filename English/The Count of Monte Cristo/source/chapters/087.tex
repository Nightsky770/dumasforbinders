\chapter{The Challenge} 

 \lettrine[ante=`]{T}{hen},' continued Beauchamp, <I took advantage of the silence and the darkness to leave the house without being seen. The usher who had introduced me was waiting for me at the door, and he conducted me through the corridors to a private entrance opening into the Rue de Vaugirard. I left with mingled feelings of sorrow and delight. Excuse me, Albert,—sorrow on your account, and delight with that noble girl, thus pursuing paternal vengeance. Yes, Albert, from whatever source the blow may have proceeded—it may be from an enemy, but that enemy is only the agent of Providence.> 

 Albert held his head between his hands; he raised his face, red with shame and bathed in tears, and seizing Beauchamp's arm: 

 <My friend,> said he, <my life is ended. I cannot calmly say with you, <Providence has struck the blow;> but I must discover who pursues me with this hatred, and when I have found him I shall kill him, or he will kill me. I rely on your friendship to assist me, Beauchamp, if contempt has not banished it from your heart.> 

 <Contempt, my friend? How does this misfortune affect you? No, happily that unjust prejudice is forgotten which made the son responsible for the father's actions. Review your life, Albert; although it is only just beginning, did a lovely summer's day ever dawn with greater purity than has marked the commencement of your career? No, Albert, take my advice. You are young and rich—leave Paris—all is soon forgotten in this great Babylon of excitement and changing tastes. You will return after three or four years with a Russian princess for a bride, and no one will think more of what occurred yesterday than if it had happened sixteen years ago.> 

 <Thank you, my dear Beauchamp, thank you for the excellent feeling which prompts your advice; but it cannot be. I have told you my wish, or rather my determination. You understand that, interested as I am in this affair, I cannot see it in the same light as you do. What appears to you to emanate from a celestial source, seems to me to proceed from one far less pure. Providence appears to me to have no share in this affair; and happily so, for instead of the invisible, impalpable agent of celestial rewards and punishments, I shall find one both palpable and visible, on whom I shall revenge myself, I assure you, for all I have suffered during the last month. Now, I repeat, Beauchamp, I wish to return to human and material existence, and if you are still the friend you profess to be, help me to discover the hand that struck the blow.> 

 <Be it so,> said Beauchamp; <if you must have me descend to earth, I submit; and if you will seek your enemy, I will assist you, and I will engage to find him, my honour being almost as deeply interested as yours.> 

 <Well, then, you understand, Beauchamp, that we begin our search immediately. Each moment's delay is an eternity for me. The calumniator is not yet punished, and he may hope that he will not be; but, on my honour, if he thinks so, he deceives himself.> 

 <Well, listen, Morcerf.> 

 <Ah, Beauchamp, I see you know something already; you will restore me to life.> 

 <I do not say there is any truth in what I am going to tell you, but it is, at least, a ray of light in a dark night; by following it we may, perhaps, discover something more certain.> 

 <Tell me; satisfy my impatience.> 

 <Well, I will tell you what I did not like to mention on my return from Yanina.> 

 <Say on.> 

 <I went, of course, to the chief banker of the town to make inquiries. At the first word, before I had even mentioned your father's name>— 

 “<Ah,> said he. <I guess what brings you here.> 

 “<How, and why?> 

 “<Because a fortnight since I was questioned on the same subject.> 

 “<By whom?> 

 “<By a banker of Paris, my correspondent.> 

 “<Whose name is\longdash> 

 <<Danglars.>> 

 <He!> cried Albert; <yes, it is indeed he who has so long pursued my father with jealous hatred. He, the man who would be popular, cannot forgive the Count of Morcerf for being created a peer; and this marriage broken off without a reason being assigned—yes, it is all from the same cause.> 

 <Make inquiries, Albert, but do not be angry without reason; make inquiries, and if it be true\longdash> 

 <Oh, yes, if it be true,> cried the young man, <he shall pay me all I have suffered.> 

 <Beware, Morcerf, he is already an old man.> 

 <I will respect his age as he has respected the honour of my family; if my father had offended him, why did he not attack him personally? Oh, no, he was afraid to encounter him face to face.> 

 <I do not condemn you, Albert; I only restrain you. Act prudently.> 

 <Oh, do not fear; besides, you will accompany me. Beauchamp, solemn transactions should be sanctioned by a witness. Before this day closes, if M. Danglars is guilty, he shall cease to live, or I shall die. \textit{Pardieu}, Beauchamp, mine shall be a splendid funeral!> 

 <When such resolutions are made, Albert, they should be promptly executed. Do you wish to go to M. Danglars? Let us go immediately.> 

 They sent for a cabriolet. On entering the banker's mansion, they perceived the phaeton and servant of M. Andrea Cavalcanti. 

 <Ah! \textit{parbleu!} that's good,> said Albert, with a gloomy tone. <If M. Danglars will not fight with me, I will kill his son-in-law; Cavalcanti will certainly fight.> 

 The servant announced the young man; but the banker, recollecting what had transpired the day before, did not wish him admitted. It was, however, too late; Albert had followed the footman, and, hearing the order given, forced the door open, and followed by Beauchamp found himself in the banker's study. 

 <Sir,> cried the latter, <am I no longer at liberty to receive whom I choose in my house? You appear to forget yourself sadly.> 

 <No, sir,> said Albert, coldly; <there are circumstances in which one cannot, except through cowardice,—I offer you that refuge,—refuse to admit certain persons at least.> 

 <What is your errand, then, with me, sir?> 

 <I mean,> said Albert, drawing near, and without apparently noticing Cavalcanti, who stood with his back towards the fireplace—<I mean to propose a meeting in some retired corner where no one will interrupt us for ten minutes; that will be sufficient—where two men having met, one of them will remain on the ground.> 

 Danglars turned pale; Cavalcanti moved a step forward, and Albert turned towards him. 

 <And you, too,> said he, <come, if you like, monsieur; you have a claim, being almost one of the family, and I will give as many rendezvous of that kind as I can find persons willing to accept them.> 

 Cavalcanti looked at Danglars with a stupefied air, and the latter, making an effort, arose and stepped between the two young men. Albert's attack on Andrea had placed him on a different footing, and he hoped this visit had another cause than that he had at first supposed. 

 <Indeed, sir,> said he to Albert, <if you are come to quarrel with this gentleman because I have preferred him to you, I shall resign the case to the king's attorney.> 

 <You mistake, sir,> said Morcerf with a gloomy smile; <I am not referring in the least to matrimony, and I only addressed myself to M. Cavalcanti because he appeared disposed to interfere between us. In one respect you are right, for I am ready to quarrel with everyone today; but you have the first claim, M. Danglars.>  <Sir,> replied Danglars, pale with anger and fear, <I warn you, when I have the misfortune to meet with a mad dog, I kill it; and far from thinking myself guilty of a crime, I believe I do society a kindness. Now, if you are mad and try to bite me, I will kill you without pity. Is it my fault that your father has dishonored himself?> 

 <Yes, miserable wretch!> cried Morcerf, <it is your fault.> 

 Danglars retreated a few steps. <My fault?> said he; <you must be mad! What do I know of the Grecian affair? Have I travelled in that country? Did I advise your father to sell the castle of Yanina—to betray\longdash> 

 <Silence!> said Albert, with a thundering voice. <No; it is not you who have directly made this exposure and brought this sorrow on us, but you hypocritically provoked it.> 

 <I?> 

 <Yes; you! How came it known?> 

 <I suppose you read it in the paper in the account from Yanina?> 

 <Who wrote to Yanina?> 

 <To Yanina?> 

 <Yes. Who wrote for particulars concerning my father?> 

 <I imagine anyone may write to Yanina.> 

 <But one person only wrote!> 

 <One only?> 

 <Yes; and that was you!> 

 <I, doubtless, wrote. It appears to me that when about to marry your daughter to a young man, it is right to make some inquiries respecting his family; it is not only a right, but a duty.> 

 <You wrote, sir, knowing what answer you would receive.> 

 <I, indeed? I assure you,> cried Danglars, with a confidence and security proceeding less from fear than from the interest he really felt for the young man, <I solemnly declare to you, that I should never have thought of writing to Yanina, did I know anything of Ali Pasha's misfortunes.> 

 <Who, then, urged you to write? Tell me.> 

 <\textit{Pardieu!} it was the most simple thing in the world. I was speaking of your father's past history. I said the origin of his fortune remained obscure. The person to whom I addressed my scruples asked me where your father had acquired his property? I answered, <In Greece.>—<Then,> said he, <write to Yanina.>> 

 <And who thus advised you?> 

 <No other than your friend, Monte Cristo.> 

 <The Count of Monte Cristo told you to write to Yanina?> 

 <Yes; and I wrote, and will show you my correspondence, if you like.> 

 Albert and Beauchamp looked at each other. 

 <Sir,> said Beauchamp, who had not yet spoken, <you appear to accuse the count, who is absent from Paris at this moment, and cannot justify himself.> 

 <I accuse no one, sir,> said Danglars; <I relate, and I will repeat before the count what I have said to you.> 

 <Does the count know what answer you received?> 

 <Yes; I showed it to him.> 

 <Did he know my father's Christian name was Fernand, and his family name Mondego?> 

 <Yes, I had told him that long since, and I did only what any other would have done in my circumstances, and perhaps less. When, the day after the arrival of this answer, your father came by the advice of Monte Cristo to ask my daughter's hand for you, I decidedly refused him, but without any explanation or exposure. In short, why should I have any more to do with the affair? How did the honour or disgrace of M. de Morcerf affect me? It neither increased nor decreased my income.> 

 Albert felt the blood mounting to his brow; there was no doubt upon the subject. Danglars defended himself with the baseness, but at the same time with the assurance, of a man who speaks the truth, at least in part, if not wholly—not for conscience' sake, but through fear. Besides, what was Morcerf seeking? It was not whether Danglars or Monte Cristo was more or less guilty; it was a man who would answer for the offence, whether trifling or serious; it was a man who would fight, and it was evident Danglars would not fight. 

 In addition to this, everything forgotten or unperceived before presented itself now to his recollection. Monte Cristo knew everything, as he had bought the daughter of Ali Pasha; and, knowing everything, he had advised Danglars to write to Yanina. The answer known, he had yielded to Albert's wish to be introduced to Haydée, and allowed the conversation to turn on the death of Ali, and had not opposed Haydée's recital (but having, doubtless, warned the young girl, in the few Romaic words he spoke to her, not to implicate Morcerf's father). Besides, had he not begged of Morcerf not to mention his father's name before Haydée? Lastly, he had taken Albert to Normandy when he knew the final blow was near. There could be no doubt that all had been calculated and previously arranged; Monte Cristo then was in league with his father's enemies. Albert took Beauchamp aside, and communicated these ideas to him. 

 <You are right,> said the latter; <M. Danglars has only been a secondary agent in this sad affair, and it is of M. de Monte Cristo that you must demand an explanation.> 

 Albert turned. 

 <Sir,> said he to Danglars, <understand that I do not take a final leave of you; I must ascertain if your insinuations are just, and am going now to inquire of the Count of Monte Cristo.> 

 He bowed to the banker, and went out with Beauchamp, without appearing to notice Cavalcanti. Danglars accompanied him to the door, where he again assured Albert that no motive of personal hatred had influenced him against the Count of Morcerf. 