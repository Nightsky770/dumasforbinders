\chapter{The Lions' Den}

 \lettrine{O}{ne} division of La Force, in which the most dangerous and desperate prisoners are confined, is called the court of Saint-Bernard. The prisoners, in their expressive language, have named it the <Lions' Den,> probably because the captives possess teeth which frequently gnaw the bars, and sometimes the keepers also. It is a prison within a prison; the walls are double the thickness of the rest. The gratings are every day carefully examined by jailers, whose herculean proportions and cold pitiless expression prove them to have been chosen to reign over their subjects for their superior activity and intelligence. 

 The courtyard of this quarter is enclosed by enormous walls, over which the sun glances obliquely, when it deigns to penetrate into this gulf of moral and physical deformity. On this paved yard are to be seen,—pacing to and fro from morning till night, pale, careworn, and haggard, like so many shadows,—the men whom justice holds beneath the steel she is sharpening. There, crouched against the side of the wall which attracts and retains the most heat, they may be seen sometimes talking to one another, but more frequently alone, watching the door, which sometimes opens to call forth one from the gloomy assemblage, or to throw in another outcast from society. 

 The court of Saint-Bernard has its own particular apartment for the reception of guests; it is a long rectangle, divided by two upright gratings placed at a distance of three feet from one another to prevent a visitor from shaking hands with or passing anything to the prisoners. It is a wretched, damp, nay, even horrible spot, more especially when we consider the agonizing conferences which have taken place between those iron bars. And yet, frightful though this spot may be, it is looked upon as a kind of paradise by the men whose days are numbered; it is so rare for them to leave the Lions' Den for any other place than the barrier Saint-Jacques, the galleys! or solitary confinement. 

 In the court which we have attempted to describe, and from which a damp vapor was rising, a young man with his hands in his pockets, who had excited much curiosity among the inhabitants of the <Den,> might be seen walking. The cut of his clothes would have made him pass for an elegant man, if those clothes had not been torn to shreds; still they did not show signs of wear, and the fine cloth, beneath the careful hands of the prisoner, soon recovered its gloss in the parts which were still perfect, for the wearer tried his best to make it assume the appearance of a new coat. He bestowed the same attention upon the cambric front of a shirt, which had considerably changed in colour since his entrance into the prison, and he polished his varnished boots with the corner of a handkerchief embroidered with initials surmounted by a coronet. 

 Some of the inmates of the <Lions' Den> were watching the operations of the prisoner's toilet with considerable interest. 

 <See, the prince is pluming himself,> said one of the thieves. 

 <He's a fine looking fellow,> said another; <if he had only a comb and hair-grease, he'd take the shine off the gentlemen in white kids.> 

 <His coat looks nearly new, and his boots are brilliant. It is pleasant to have such well-dressed brethren; and those gendarmes behaved shamefully. What jealousy; to tear such clothes!> 

 <He looks like a big-bug,> said another; <dresses in fine style. And, then, to be here so young! Oh, what larks!> 

 Meanwhile the object of this hideous admiration approached the wicket, against which one of the keepers was leaning. 

 <Come, sir,> he said, <lend me twenty francs; you will soon be paid; you run no risks with me. Remember, I have relations who possess more millions than you have deniers. Come, I beseech you, lend me twenty francs, so that I may buy a dressing-gown; it is intolerable always to be in a coat and boots! And what a coat, sir, for a prince of the Cavalcanti!> 

 The keeper turned his back, and shrugged his shoulders; he did not even laugh at what would have caused anyone else to do so; he had heard so many utter the same things,—indeed, he heard nothing else. 

 <Come,> said Andrea, <you are a man void of compassion; I'll have you turned out.> 

 This made the keeper turn around, and he burst into a loud laugh. The prisoners then approached and formed a circle. 

 <I tell you that with that wretched sum,> continued Andrea, <I could obtain a coat, and a room in which to receive the illustrious visitor I am daily expecting.> 

 <Of course—of course,> said the prisoners;—<anyone can see he's a gentleman!> 

 <Well, then, lend him the twenty francs,> said the keeper, leaning on the other shoulder; <surely you will not refuse a comrade!>

<I am no comrade of these people,> said the young man, proudly, <you have no right to insult me thus.> 

 The thieves looked at one another with low murmurs, and a storm gathered over the head of the aristocratic prisoner, raised less by his own words than by the manner of the keeper. The latter, sure of quelling the tempest when the waves became too violent, allowed them to rise to a certain pitch that he might be revenged on the importunate Andrea, and besides it would afford him some recreation during the long day. 

 The thieves had already approached Andrea, some screaming, \textit{<La savate—La savate!>}\footnote{\textit{Savate}: an old shoe. } a cruel operation, which consists in cuffing a comrade who may have fallen into disgrace, not with an old shoe, but with an iron-heeled one. Others proposed the \textit{anguille}, another kind of recreation, in which a handkerchief is filled with sand, pebbles, and two-sous pieces, when they have them, which the wretches beat like a flail over the head and shoulders of the unhappy sufferer. 

 <Let us horsewhip the fine gentleman!> said others. 

 But Andrea, turning towards them, winked his eyes, rolled his tongue around his cheeks, and smacked his lips in a manner equivalent to a hundred words among the bandits when forced to be silent. It was a Masonic sign Caderousse had taught him. He was immediately recognized as one of them; the handkerchief was thrown down, and the iron-heeled shoe replaced on the foot of the wretch to whom it belonged. 

 Some voices were heard to say that the gentleman was right; that he intended to be civil, in his way, and that they would set the example of liberty of conscience,—and the mob retired. The keeper was so stupefied at this scene that he took Andrea by the hands and began examining his person, attributing the sudden submission of the inmates of the Lions' Den to something more substantial than mere fascination. 

 Andrea made no resistance, although he protested against it. Suddenly a voice was heard at the wicket. 

 <Benedetto!> exclaimed an inspector. The keeper relaxed his hold. 

 <I am called,> said Andrea. 

 <To the visitors' room!> said the same voice. 

 <You see someone pays me a visit. Ah, my dear sir, you will see whether a Cavalcanti is to be treated like a common person!> 

 And Andrea, gliding through the court like a black shadow, rushed out through the wicket, leaving his comrades, and even the keeper, lost in wonder. Certainly a call to the visitors' room had scarcely astonished Andrea less than themselves, for the wily youth, instead of making use of his privilege of waiting to be claimed on his entry into La Force, had maintained a rigid silence.  <Everything,> he said, <proves me to be under the protection of some powerful person,—this sudden fortune, the facility with which I have overcome all obstacles, an unexpected family and an illustrious name awarded to me, gold showered down upon me, and the most splendid alliances about to be entered into. An unhappy lapse of fortune and the absence of my protector have cast me down, certainly, but not forever. The hand which has retreated for a while will be again stretched forth to save me at the very moment when I shall think myself sinking into the abyss. Why should I risk an imprudent step? It might alienate my protector. He has two means of extricating me from this dilemma,—the one by a mysterious escape, managed through bribery; the other by buying off my judges with gold. I will say and do nothing until I am convinced that he has quite abandoned me, and then\longdash> 

 Andrea had formed a plan which was tolerably clever. The unfortunate youth was intrepid in the attack, and rude in the defence. He had borne with the public prison, and with privations of all sorts; still, by degrees nature, or rather custom, had prevailed, and he suffered from being naked, dirty, and hungry. It was at this moment of discomfort that the inspector's voice called him to the visiting-room. Andrea felt his heart leap with joy. It was too soon for a visit from the examining magistrate, and too late for one from the director of the prison, or the doctor; it must, then, be the visitor he hoped for. Behind the grating of the room into which Andrea had been led, he saw, while his eyes dilated with surprise, the dark and intelligent face of M. Bertuccio, who was also gazing with sad astonishment upon the iron bars, the bolted doors, and the shadow which moved behind the other grating. 

 <Ah,> said Andrea, deeply affected. 

 <Good morning, Benedetto,> said Bertuccio, with his deep, hollow voice. 

 <You—you?> said the young man, looking fearfully around him. 

 <Do you not recognize me, unhappy child?> 

 <Silence,—be silent!> said Andrea, who knew the delicate sense of hearing possessed by the walls; <for Heaven's sake, do not speak so loud!> 

 <You wish to speak with me alone, do you not?> said Bertuccio. 

 <Oh, yes.> 

 <That is well.> 

 And Bertuccio, feeling in his pocket, signed to a keeper whom he saw through the window of the wicket. 

 <Read?> he said. 

 <What is that?> asked Andrea. 

 <An order to conduct you to a room, and to leave you there to talk to me.> 

 <Oh,> cried Andrea, leaping with joy. Then he mentally added,—<Still my unknown protector! I am not forgotten. They wish for secrecy, since we are to converse in a private room. I understand, Bertuccio has been sent by my protector.> 

 The keeper spoke for a moment with an official, then opened the iron gates and conducted Andrea to a room on the first floor. The room was whitewashed, as is the custom in prisons, but it looked quite brilliant to a prisoner, though a stove, a bed, a chair, and a table formed the whole of its sumptuous furniture. Bertuccio sat down upon the chair, Andrea threw himself upon the bed; the keeper retired. 

 <Now,> said the steward, <what have you to tell me?> 

 <And you?> said Andrea. 

 <You speak first.> 

 <Oh, no. You must have much to tell me, since you have come to seek me.>  
 
 <Well, be it so. You have continued your course of villany; you have robbed—you have assassinated.> 

 <Well, I should say! If you had me taken to a private room only to tell me this, you might have saved yourself the trouble. I know all these things. But there are some with which, on the contrary, I am not acquainted. Let us talk of those, if you please. Who sent you?> 

 <Come, come, you are going on quickly, M. Benedetto!> 

 <Yes, and to the point. Let us dispense with useless words. Who sends you?> 

 <No one.> 

 <How did you know I was in prison?> 

 <I recognized you, some time since, as the insolent dandy who so gracefully mounted his horse in the Champs-Élysées.> 

 <Oh, the Champs-Élysées? Ah, yes; we burn, as they say at the game of pincette. The Champs-Élysées? Come, let us talk a little about my father.> 

 <Who, then, am I?> 

 <You, sir?—you are my adopted father. But it was not you, I presume, who placed at my disposal 100,000 francs, which I spent in four or five months; it was not you who manufactured an Italian gentleman for my father; it was not you who introduced me into the world, and had me invited to a certain dinner at Auteuil, which I fancy I am eating at this moment, in company with the most distinguished people in Paris—amongst the rest with a certain procureur, whose acquaintance I did very wrong not to cultivate, for he would have been very useful to me just now;—it was not you, in fact, who bailed me for one or two millions, when the fatal discovery of my little secret took place. Come, speak, my worthy Corsican, speak!> 

 <What do you wish me to say?> 

 <I will help you. You were speaking of the Champs-Élysées just now, worthy foster-father.> 

 <Well?> 

 <Well, in the Champs-Élysées there resides a very rich gentleman.> 

 <At whose house you robbed and murdered, did you not?> 

 <I believe I did.> 

 <The Count of Monte Cristo?> 

 <'Tis you who have named him, as M. Racine says. Well, am I to rush into his arms, and strain him to my heart, crying, <My father, my father!> like Monsieur Pixérécourt.>\footnote{Guilbert de Pixérécourt, French dramatist (1773-1844). } 

 <Do not let us jest,> gravely replied Bertuccio, <and dare not to utter that name again as you have pronounced it.> 

 <Bah,> said Andrea, a little overcome, by the solemnity of Bertuccio's manner, <why not?> 

 <Because the person who bears it is too highly favoured by Heaven to be the father of such a wretch as you.> 

 <Oh, these are fine words.> 

 <And there will be fine doings, if you do not take care.> 

 <Menaces—I do not fear them. I will say\longdash> 

 <Do you think you are engaged with a pygmy like yourself?> said Bertuccio, in so calm a tone, and with so steadfast a look, that Andrea was moved to the very soul. <Do you think you have to do with galley-slaves, or novices in the world? Benedetto, you are fallen into terrible hands; they are ready to open for you—make use of them. Do not play with the thunderbolt they have laid aside for a moment, but which they can take up again instantly, if you attempt to intercept their movements.>  
 
<My father—I will know who my father is,> said the obstinate youth; <I will perish if I must, but I \textit{will} know it. What does scandal signify to me? What possessions, what reputation, what <pull,> as Beauchamp says,—have I? You great people always lose something by scandal, notwithstanding your millions. Come, who is my father?> 

 <I came to tell you.> 

 <Ah,> cried Benedetto, his eyes sparkling with joy. Just then the door opened, and the jailer, addressing himself to Bertuccio, said: 

 <Excuse me, sir, but the examining magistrate is waiting for the prisoner.> 

 <And so closes our interview,> said Andrea to the worthy steward; <I wish the troublesome fellow were at the devil!> 

 <I will return tomorrow,> said Bertuccio. 

 <Good! Gendarmes, I am at your service. Ah, sir, do leave a few crowns for me at the gate that I may have some things I am in need of!> 

 <It shall be done,> replied Bertuccio. 

 Andrea extended his hand; Bertuccio kept his own in his pocket, and merely jingled a few pieces of money. 

 <That's what I mean,> said Andrea, endeavouring to smile, quite overcome by the strange tranquillity of Bertuccio. 

 <Can I be deceived?> he murmured, as he stepped into the oblong and grated vehicle which they call <the salad basket.> 

 <Never mind, we shall see! Tomorrow, then!> he added, turning towards Bertuccio. 

 <Tomorrow!> replied the steward. 