\chapter{The Departure} 

 \lettrine{T}{he} recent events formed the theme of conversation throughout all Paris. Emmanuel and his wife conversed with natural astonishment in their little apartment in the Rue Meslay upon the three successive, sudden, and most unexpected catastrophes of Morcerf, Danglars, and Villefort. Maximilian, who was paying them a visit, listened to their conversation, or rather was present at it, plunged in his accustomed state of apathy. 

 <Indeed,> said Julie, <might we not almost fancy, Emmanuel, that those people, so rich, so happy but yesterday, had forgotten in their prosperity that an evil genius—like the wicked fairies in Perrault's stories who present themselves unbidden at a wedding or baptism—hovered over them, and appeared all at once to revenge himself for their fatal neglect?> 

 <What a dire misfortune!> said Emmanuel, thinking of Morcerf and Danglars. 

 <What dreadful sufferings!> said Julie, remembering Valentine, but whom, with a delicacy natural to women, she did not name before her brother. 

 <If the Supreme Being has directed the fatal blow,> said Emmanuel, <it must be that he in his great goodness has perceived nothing in the past lives of these people to merit mitigation of their awful punishment.> 

 <Do you not form a very rash judgment, Emmanuel?> said Julie. <When my father, with a pistol in his hand, was once on the point of committing suicide, had anyone then said, <This man deserves his misery,> would not that person have been deceived?> 

 <Yes; but your father was not allowed to fall. A being was commissioned to arrest the fatal hand of death about to descend on him.> 

 Emmanuel had scarcely uttered these words when the sound of the bell was heard, the well-known signal given by the porter that a visitor had arrived. Nearly at the same instant the door was opened and the Count of Monte Cristo appeared on the threshold. The young people uttered a cry of joy, while Maximilian raised his head, but let it fall again immediately.  <Maximilian,> said the count, without appearing to notice the different impressions which his presence produced on the little circle, <I come to seek you.> 

 <To seek me?> repeated Morrel, as if awakening from a dream. 

 <Yes,> said Monte Cristo; <has it not been agreed that I should take you with me, and did I not tell you yesterday to prepare for departure?> 

 <I am ready,> said Maximilian; <I came expressly to wish them farewell.> 

 <Whither are you going, count?> asked Julie. 

 <In the first instance to Marseilles, madame.> 

 <To Marseilles!> exclaimed the young couple. 

 <Yes, and I take your brother with me.> 

 <Oh, count.> said Julie, <will you restore him to us cured of his melancholy?> Morrel turned away to conceal the confusion of his countenance. 

 <You perceive, then, that he is not happy?> said the count. 

 <Yes,> replied the young woman; <and fear much that he finds our home but a dull one.> 

 <I will undertake to divert him,> replied the count. 

 <I am ready to accompany you, sir,> said Maximilian. <Adieu, my kind friends! Emmanuel—Julie—farewell!> 

 <How farewell?> exclaimed Julie; <do you leave us thus, so suddenly, without any preparations for your journey, without even a passport?> 

 <Needless delays but increase the grief of parting,> said Monte Cristo, <and Maximilian has doubtless provided himself with everything requisite; at least, I advised him to do so.> 

 <I have a passport, and my clothes are ready packed,> said Morrel in his tranquil but mournful manner. 

 <Good,> said Monte Cristo, smiling; <in these prompt arrangements we recognize the order of a well-disciplined soldier.> 

 <And you leave us,> said Julie, <at a moment's warning? you do not give us a day—no, not even an hour before your departure?> 

 <My carriage is at the door, madame, and I must be in Rome in five days.> 

 <But does Maximilian go to Rome?> exclaimed Emmanuel. 

 <I am going wherever it may please the count to take me,> said Morrel, with a smile full of grief; <I am under his orders for the next month.> 

 <Oh, heavens, how strangely he expresses himself, count!> said Julie. 

 <Maximilian goes with \textit{me},> said the count, in his kindest and most persuasive manner; <therefore do not make yourself uneasy on your brother's account.> 

 <Once more farewell, my dear sister; Emmanuel, adieu!> Morrel repeated. 

 <His carelessness and indifference touch me to the heart,> said Julie. <Oh, Maximilian, Maximilian, you are certainly concealing something from us.> 

 <Pshaw!> said Monte Cristo, <you will see him return to you gay, smiling, and joyful.> 

 Maximilian cast a look of disdain, almost of anger, on the count. 

 <We must leave you,> said Monte Cristo.  <Before you quit us, count,> said Julie, <will you permit us to express to you all that the other day\longdash> 

 <Madame,> interrupted the count, taking her two hands in his, <all that you could say in words would never express what I read in your eyes; the thoughts of your heart are fully understood by mine. Like benefactors in romances, I should have left you without seeing you again, but that would have been a virtue beyond my strength, because I am a weak and vain man, fond of the tender, kind, and thankful glances of my fellow-creatures. On the eve of departure I carry my egotism so far as to say, <Do not forget me, my kind friends, for probably you will never see me again.>> 

 <Never see you again?> exclaimed Emmanuel, while two large tears rolled down Julie's cheeks, <never behold you again? It is not a man, then, but some angel that leaves us, and this angel is on the point of returning to heaven after having appeared on earth to do good.> 

 <Say not so,> quickly returned Monte Cristo—<say not so, my friends; angels never err, celestial beings remain where they wish to be. Fate is not more powerful than they; it is they who, on the contrary, overcome fate. No, Emmanuel, I am but a man, and your admiration is as unmerited as your words are sacrilegious.> 

 And pressing his lips on the hand of Julie, who rushed into his arms, he extended his other hand to Emmanuel; then tearing himself from this abode of peace and happiness, he made a sign to Maximilian, who followed him passively, with the indifference which had been perceptible in him ever since the death of Valentine had so stunned him. 

 <Restore my brother to peace and happiness,> whispered Julie to Monte Cristo. And the count pressed her hand in reply, as he had done eleven years before on the staircase leading to Morrel's study. 

 <You still confide, then, in Sinbad the Sailor?> asked he, smiling. 

 <Oh, yes,> was the ready answer. 

 <Well, then, sleep in peace, and put your trust in the Lord.> 

 As we have before said, the post-chaise was waiting; four powerful horses were already pawing the ground with impatience, while Ali, apparently just arrived from a long walk, was standing at the foot of the steps, his face bathed in perspiration. 

 <Well,> asked the count in Arabic, <have you been to see the old man?> Ali made a sign in the affirmative. 

 <And have you placed the letter before him, as I ordered you to do?> 

 The slave respectfully signalized that he had. 

 <And what did he say, or rather do?> Ali placed himself in the light, so that his master might see him distinctly, and then imitating in his intelligent manner the countenance of the old man, he closed his eyes, as Noirtier was in the custom of doing when saying <Yes.> 

 <Good; he accepts,> said Monte Cristo. <Now let us go.>  These words had scarcely escaped him, when the carriage was on its way, and the feet of the horses struck a shower of sparks from the pavement. Maximilian settled himself in his corner without uttering a word. Half an hour had passed when the carriage stopped suddenly; the count had just pulled the silken check-string, which was fastened to Ali's finger. The Nubian immediately descended and opened the carriage door. It was a lovely starlight night—they had just reached the top of the hill Villejuif, from whence Paris appears like a sombre sea tossing its millions of phosphoric waves into light—waves indeed more noisy, more passionate, more changeable, more furious, more greedy, than those of the tempestuous ocean,—waves which never rest as those of the sea sometimes do,—waves ever dashing, ever foaming, ever ingulfing what falls within their grasp. 

 The count stood alone, and at a sign from his hand, the carriage went on for a short distance. With folded arms, he gazed for some time upon the great city. When he had fixed his piercing look on this modern Babylon, which equally engages the contemplation of the religious enthusiast, the materialist, and the scoffer,— 

 <Great city,> murmured he, inclining his head, and joining his hands as if in prayer, <less than six months have elapsed since first I entered thy gates. I believe that the Spirit of God led my steps to thee and that he also enables me to quit thee in triumph; the secret cause of my presence within thy walls I have confided alone to him who only has had the power to read my heart. God only knows that I retire from thee without pride or hatred, but not without many regrets; he only knows that the power confided to me has never been made subservient to my personal good or to any useless cause. Oh, great city, it is in thy palpitating bosom that I have found that which I sought; like a patient miner, I have dug deep into thy very entrails to root out evil thence. Now my work is accomplished, my mission is terminated, now thou canst neither afford me pain nor pleasure. Adieu, Paris, adieu!> 

 His look wandered over the vast plain like that of some genius of the night; he passed his hand over his brow, got into the carriage, the door was closed on him, and the vehicle quickly disappeared down the other side of the hill in a whirlwind of dust and noise. 

 Ten leagues were passed and not a single word was uttered. Morrel was dreaming, and Monte Cristo was looking at the dreamer. 

 <Morrel,> said the count to him at length, <do you repent having followed me?> 

 <No, count; but to leave Paris\longdash> 

 <If I thought happiness might await you in Paris, Morrel, I would have left you there.> 

 <Valentine reposes within the walls of Paris, and to leave Paris is like losing her a second time.> 

 <Maximilian,> said the count, <the friends that we have lost do not repose in the bosom of the earth, but are buried deep in our hearts, and it has been thus ordained that we may always be accompanied by them. I have two friends, who in this way never depart from me; the one who gave me being, and the other who conferred knowledge and intelligence on me. Their spirits live in me. I consult them when doubtful, and if I ever do any good, it is due to their beneficent counsels. Listen to the voice of your heart, Morrel, and ask it whether you ought to preserve this melancholy exterior towards me.> 

 <My friend,> said Maximilian, <the voice of my heart is very sorrowful, and promises me nothing but misfortune.> 

 <It is the way of weakened minds to see everything through a black cloud. The soul forms its own horizons; your soul is darkened, and consequently the sky of the future appears stormy and unpromising.> 

 <That may possibly be true,> said Maximilian, and he again subsided into his thoughtful mood. 

 The journey was performed with that marvellous rapidity which the unlimited power of the count ever commanded. Towns fled from them like shadows on their path, and trees shaken by the first winds of autumn seemed like giants madly rushing on to meet them, and retreating as rapidly when once reached. The following morning they arrived at Châlons, where the count's steamboat waited for them. Without the loss of an instant, the carriage was placed on board and the two travellers embarked without delay. The boat was built for speed; her two paddle-wheels were like two wings with which she skimmed the water like a bird. 

 Morrel was not insensible to that sensation of delight which is generally experienced in passing rapidly through the air, and the wind which occasionally raised the hair from his forehead seemed on the point of dispelling momentarily the clouds collected there. 

 As the distance increased between the travellers and Paris, almost superhuman serenity appeared to surround the count; he might have been taken for an exile about to revisit his native land. 

 Ere long Marseilles presented herself to view,—Marseilles, white, fervid, full of life and energy,—Marseilles, the younger sister of Tyre and Carthage, the successor to them in the empire of the Mediterranean,—Marseilles, old, yet always young. Powerful memories were stirred within them by the sight of the round tower, Fort Saint-Nicolas, the City Hall designed by Puget,\footnote{Gaspard Puget, the sculptor-architect, was born at Marseilles in 1615. } the port with its brick quays, where they had both played in childhood, and it was with one accord that they stopped on the Canebière. 

 A vessel was setting sail for Algiers, on board of which the bustle usually attending departure prevailed. The passengers and their relations crowded on the deck, friends taking a tender but sorrowful leave of each other, some weeping, others noisy in their grief, the whole forming a spectacle that might be exciting even to those who witnessed similar sights daily, but which had no power to disturb the current of thought that had taken possession of the mind of Maximilian from the moment he had set foot on the broad pavement of the quay. 

 <Here,> said he, leaning heavily on the arm of Monte Cristo,—<here is the spot where my father stopped, when the \textit{Pharaon} entered the port; it was here that the good old man, whom you saved from death and dishonor, threw himself into my arms. I yet feel his warm tears on my face, and his were not the only tears shed, for many who witnessed our meeting wept also.> 

 Monte Cristo gently smiled and said,—<I was there;> at the same time pointing to the corner of a street. As he spoke, and in the very direction he indicated, a groan, expressive of bitter grief, was heard, and a woman was seen waving her hand to a passenger on board the vessel about to sail. Monte Cristo looked at her with an emotion that must have been remarked by Morrel had not his eyes been fixed on the vessel. 

 <Oh, heavens!> exclaimed Morrel, <I do not deceive myself—that young man who is waving his hat, that youth in the uniform of a lieutenant, is Albert de Morcerf!> 

 <Yes,> said Monte Cristo, <I recognized him.> 

 <How so?—you were looking the other way.>  The count smiled, as he was in the habit of doing when he did not want to make any reply, and he again turned towards the veiled woman, who soon disappeared at the corner of the street. Turning to his friend: 

 <Dear Maximilian,> said the count, <have you nothing to do in this land?> 

 <I have to weep over the grave of my father,> replied Morrel in a broken voice. 

 <Well, then, go,—wait for me there, and I will soon join you.> 

 <You leave me, then?> 

 <Yes; I also have a pious visit to pay.> 

 Morrel allowed his hand to fall into that which the count extended to him; then with an inexpressibly sorrowful inclination of the head he quitted the count and bent his steps to the east of the city. Monte Cristo remained on the same spot until Maximilian was out of sight; he then walked slowly towards the Allées de Meilhan to seek out a small house with which our readers were made familiar at the beginning of this story. 

 It yet stood, under the shade of the fine avenue of lime-trees, which forms one of the most frequent walks of the idlers of Marseilles, covered by an immense vine, which spreads its aged and blackened branches over the stone front, burnt yellow by the ardent sun of the south. Two stone steps worn away by the friction of many feet led to the door, which was made of three planks; the door had never been painted or varnished, so great cracks yawned in it during the dry season to close again when the rains came on. The house, with all its crumbling antiquity and apparent misery, was yet cheerful and picturesque, and was the same that old Dantès formerly inhabited—the only difference being that the old man occupied merely the garret, while the whole house was now placed at the command of Mercédès by the count. 

 The woman whom the count had seen leave the ship with so much regret entered this house; she had scarcely closed the door after her when Monte Cristo appeared at the corner of a street, so that he found and lost her again almost at the same instant. The worn out steps were old acquaintances of his; he knew better than anyone else how to open that weather-beaten door with the large headed nail which served to raise the latch within. He entered without knocking, or giving any other intimation of his presence, as if he had been a friend or the master of the place. At the end of a passage paved with bricks, was a little garden, bathed in sunshine, and rich in warmth and light. In this garden Mercédès had found, at the place indicated by the count, the sum of money which he, through a sense of delicacy, had described as having been placed there twenty-four years previously. The trees of the garden were easily seen from the steps of the street-door. 

 Monte Cristo, on stepping into the house, heard a sigh that was almost a deep sob; he looked in the direction whence it came, and there under an arbour of Virginia jessamine,\footnote{The Carolina—not Virginia—jessamine, \textit{gelsemium sempervirens}(properly speaking not a jessamine at all) has yellow blossoms. The reference is no doubt to the \textit{Wistaria frutescens}.—Ed.} with its thick foliage and beautiful long purple flowers, he saw Mercédès seated, with her head bowed, and weeping bitterly. She had raised her veil, and with her face hidden by her hands was giving free scope to the sighs and tears which had been so long restrained by the presence of her son. 

 Monte Cristo advanced a few steps, which were heard on the gravel. Mercédès raised her head, and uttered a cry of terror on beholding a man before her.  <Madame,> said the count, <it is no longer in my power to restore you to happiness, but I offer you consolation; will you deign to accept it as coming from a friend?> 

 <I am, indeed, most wretched,> replied Mercédès. <Alone in the world, I had but my son, and he has left me!> 

 <He possesses a noble heart, madame,> replied the count, <and he has acted rightly. He feels that every man owes a tribute to his country; some contribute their talents, others their industry; these devote their blood, those their nightly labors, to the same cause. Had he remained with you, his life must have become a hateful burden, nor would he have participated in your griefs. He will increase in strength and honour by struggling with adversity, which he will convert into prosperity. Leave him to build up the future for you, and I venture to say you will confide it to safe hands.> 

 <Oh,> replied the wretched woman, mournfully shaking her head, <the prosperity of which you speak, and which, from the bottom of my heart, I pray God in his mercy to grant him, I can never enjoy. The bitter cup of adversity has been drained by me to the very dregs, and I feel that the grave is not far distant. You have acted kindly, count, in bringing me back to the place where I have enjoyed so much bliss. I ought to meet death on the same spot where happiness was once all my own.> 

 <Alas,> said Monte Cristo, <your words sear and embitter my heart, the more so as you have every reason to hate me. I have been the cause of all your misfortunes; but why do you pity, instead of blaming me? You render me still more unhappy\longdash> 

 <Hate you, blame you—\textit{you}, Edmond! Hate, reproach, the man that has spared my son's life! For was it not your fatal and sanguinary intention to destroy that son of whom M. de Morcerf was so proud? Oh, look at me closely, and discover, if you can, even the semblance of a reproach in me.> 

 The count looked up and fixed his eyes on Mercédès, who arose partly from her seat and extended both her hands towards him. 

 <Oh, look at me,> continued she, with a feeling of profound melancholy, <my eyes no longer dazzle by their brilliancy, for the time has long fled since I used to smile on Edmond Dantès, who anxiously looked out for me from the window of yonder garret, then inhabited by his old father. Years of grief have created an abyss between those days and the present. I neither reproach you nor hate you, my friend. Oh, no, Edmond, it is myself that I blame, myself that I hate! Oh, miserable creature that I am!> cried she, clasping her hands, and raising her eyes to heaven. <I once possessed piety, innocence, and love, the three ingredients of the happiness of angels, and now what am I?> 

 Monte Cristo approached her, and silently took her hand. 

 <No,> said she, withdrawing it gently—<no, my friend, touch me not. You have spared me, yet of all those who have fallen under your vengeance I was the most guilty. They were influenced by hatred, by avarice, and by self-love; but I was base, and for want of courage acted against my judgment. Nay, do not press my hand, Edmond; you are thinking, I am sure, of some kind speech to console me, but do not utter it to me, reserve it for others more worthy of your kindness. See> (and she exposed her face completely to view)—<see, misfortune has silvered my hair, my eyes have shed so many tears that they are encircled by a rim of purple, and my brow is wrinkled. You, Edmond, on the contrary,—you are still young, handsome, dignified; it is because you have had faith; because you have had strength, because you have had trust in God, and God has sustained you. But as for me, I have been a coward; I have denied God and he has abandoned me.>  Mercédès burst into tears; her woman's heart was breaking under its load of memories. Monte Cristo took her hand and imprinted a kiss on it; but she herself felt that it was a kiss of no greater warmth than he would have bestowed on the hand of some marble statue of a saint. 

 <It often happens,> continued she, <that a first fault destroys the prospects of a whole life. I believed you dead; why did I survive you? What good has it done me to mourn for you eternally in the secret recesses of my heart?—only to make a woman of thirty-nine look like a woman of fifty. Why, having recognized you, and I the only one to do so—why was I able to save my son alone? Ought I not also to have rescued the man that I had accepted for a husband, guilty though he were? Yet I let him die! What do I say? Oh, merciful heavens, was I not accessory to his death by my supine insensibility, by my contempt for him, not remembering, or not willing to remember, that it was for my sake he had become a traitor and a perjurer? In what am I benefited by accompanying my son so far, since I now abandon him, and allow him to depart alone to the baneful climate of Africa? Oh, I have been base, cowardly, I tell you; I have abjured my affections, and like all renegades I am of evil omen to those who surround me!> 

 <No, Mercédès,> said Monte Cristo, <no; you judge yourself with too much severity. You are a noble-minded woman, and it was your grief that disarmed me. Still I was but an agent, led on by an invisible and offended Deity, who chose not to withhold the fatal blow that I was destined to hurl. I take that God to witness, at whose feet I have prostrated myself daily for the last ten years, that I would have sacrificed my life to you, and with my life the projects that were indissolubly linked with it. But—and I say it with some pride, Mercédès—God needed me, and I lived. Examine the past and the present, and endeavour to dive into futurity, and then say whether I am not a divine instrument. The most dreadful misfortunes, the most frightful sufferings, the abandonment of all those who loved me, the persecution of those who did not know me, formed the trials of my youth; when suddenly, from captivity, solitude, misery, I was restored to light and liberty, and became the possessor of a fortune so brilliant, so unbounded, so unheard-of, that I must have been blind not to be conscious that God had endowed me with it to work out his own great designs. From that time I looked upon this fortune as something confided to me for a particular purpose. Not a thought was given to a life which you once, Mercédès, had the power to render blissful; not one hour of peaceful calm was mine; but I felt myself driven on like an exterminating angel. Like adventurous captains about to embark on some enterprise full of danger, I laid in my provisions, I loaded my weapons, I collected every means of attack and defence; I inured my body to the most violent exercises, my soul to the bitterest trials; I taught my arm to slay, my eyes to behold excruciating sufferings, and my mouth to smile at the most horrid spectacles. Good-natured, confiding, and forgiving as I had been, I became revengeful, cunning, and wicked, or rather, immovable as fate. Then I launched out into the path that was opened to me. I overcame every obstacle, and reached the goal; but woe to those who stood in my pathway!>

<Enough,> said Mercédès; <enough, Edmond! Believe me, that she who alone recognized you has been the only one to comprehend you; and had she crossed your path, and you had crushed her like glass, still, Edmond, still she must have admired you! Like the gulf between me and the past, there is an abyss between you, Edmond, and the rest of mankind; and I tell you freely that the comparison I draw between you and other men will ever be one of my greatest tortures. No, there is nothing in the world to resemble you in worth and goodness! But we must say farewell, Edmond, and let us part.> 

 <Before I leave you, Mercédès, have you no request to make?> said the count. 

 <I desire but one thing in this world, Edmond,—the happiness of my son.> 

 <Pray to the Almighty to spare his life, and I will take upon myself to promote his happiness.> 

 <Thank you, Edmond.> 

 <But have you no request to make for yourself, Mercédès?> 

 <For myself I want nothing. I live, as it were, between two graves. One is that of Edmond Dantès, lost to me long, long since. He had my love! That word ill becomes my faded lip now, but it is a memory dear to my heart, and one that I would not lose for all that the world contains. The other grave is that of the man who met his death from the hand of Edmond Dantès. I approve of the deed, but I must pray for the dead.> 

 <Your son shall be happy, Mercédès,> repeated the count. 

 <Then I shall enjoy as much happiness as this world can possibly confer.> 

 <But what are your intentions?> 

 Mercédès smiled sadly. 

 <To say that I shall live here, like the Mercédès of other times, gaining my bread by labour, would not be true, nor would you believe me. I have no longer the strength to do anything but to spend my days in prayer. However, I shall have no occasion to work, for the little sum of money buried by you, and which I found in the place you mentioned, will be sufficient to maintain me. rumour will probably be busy respecting me, my occupations, my manner of living—that will signify but little, that concerns God, you, and myself.> 

 <Mercédès,> said the count, <I do not say it to blame you, but you made an unnecessary sacrifice in relinquishing the whole of the fortune amassed by M. de Morcerf; half of it at least by right belonged to you, in virtue of your vigilance and economy.> 

 <I perceive what you are intending to propose to me; but I cannot accept it, Edmond—my son would not permit it.> 

 <Nothing shall be done without the full approbation of Albert de Morcerf. I will make myself acquainted with his intentions and will submit to them. But if he be willing to accept my offers, will you oppose them?> 

 <You well know, Edmond, that I am no longer a reasoning creature; I have no will, unless it be the will never to decide. I have been so overwhelmed by the many storms that have broken over my head, that I am become passive in the hands of the Almighty, like a sparrow in the talons of an eagle. I live, because it is not ordained for me to die. If succor be sent to me, I will accept it.> 

 <Ah, madame,> said Monte Cristo, <you should not talk thus! It is not so we should evince our resignation to the will of heaven; on the contrary, we are all free agents.> 

 <Alas!> exclaimed Mercédès, <if it were so, if I possessed free-will, but without the power to render that will efficacious, it would drive me to despair.> 

 Monte Cristo dropped his head and shrank from the vehemence of her grief. 

 <Will you not even say you will see me again?> he asked. 

 <On the contrary, we shall meet again,> said Mercédès, pointing to heaven with solemnity. <I tell you so to prove to you that I still hope.> 

 And after pressing her own trembling hand upon that of the count, Mercédès rushed up the stairs and disappeared. Monte Cristo slowly left the house and turned towards the quay. But Mercédès did not witness his departure, although she was seated at the little window of the room which had been occupied by old Dantès. Her eyes were straining to see the ship which was carrying her son over the vast sea; but still her voice involuntarily murmured softly: 

 <Edmond, Edmond, Edmond!> 