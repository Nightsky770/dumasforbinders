\chapter{Expiation} 

 \lettrine{N}{otwithstanding} the density of the crowd, M. de Villefort saw it open before him. There is something so awe-inspiring in great afflictions that even in the worst times the first emotion of a crowd has generally been to sympathize with the sufferer in a great catastrophe. Many people have been assassinated in a tumult, but even criminals have rarely been insulted during trial. Thus Villefort passed through the mass of spectators and officers of the Palais, and withdrew. Though he had acknowledged his guilt, he was protected by his grief. There are some situations which men understand by instinct, but which reason is powerless to explain; in such cases the greatest poet is he who gives utterance to the most natural and vehement outburst of sorrow. Those who hear the bitter cry are as much impressed as if they listened to an entire poem, and when the sufferer is sincere they are right in regarding his outburst as sublime. 

 It would be difficult to describe the state of stupor in which Villefort left the Palais. Every pulse beat with feverish excitement, every nerve was strained, every vein swollen, and every part of his body seemed to suffer distinctly from the rest, thus multiplying his agony a thousand-fold. He made his way along the corridors through force of habit; he threw aside his magisterial robe, not out of deference to etiquette, but because it was an unbearable burden, a veritable garb of Nessus, insatiate in torture. Having staggered as far as the Rue Dauphine, he perceived his carriage, awoke his sleeping coachman by opening the door himself, threw himself on the cushions, and pointed towards the Faubourg Saint-Honoré; the carriage drove on. 

 All the weight of his fallen fortune seemed suddenly to crush him; he could not foresee the consequences; he could not contemplate the future with the indifference of the hardened criminal who merely faces a contingency already familiar. 

 

 God was still in his heart. <God,> he murmured, not knowing what he said,—<God—God!> Behind the event that had overwhelmed him he saw the hand of God. The carriage rolled rapidly onward. Villefort, while turning restlessly on the cushions, felt something press against him. He put out his hand to remove the object; it was a fan which Madame de Villefort had left in the carriage; this fan awakened a recollection which darted through his mind like lightning. He thought of his wife.  <Oh!> he exclaimed, as though a red-hot iron were piercing his heart. 

 During the last hour his own crime had alone been presented to his mind; now another object, not less terrible, suddenly presented itself. His wife! He had just acted the inexorable judge with her, he had condemned her to death, and she, crushed by remorse, struck with terror, covered with the shame inspired by the eloquence of \textit{his} irreproachable virtue,—she, a poor, weak woman, without help or the power of defending herself against his absolute and supreme will,—she might at that very moment, perhaps, be preparing to die! 

 An hour had elapsed since her condemnation; at that moment, doubtless, she was recalling all her crimes to her memory; she was asking pardon for her sins; perhaps she was even writing a letter imploring forgiveness from her virtuous husband—a forgiveness she was purchasing with her death! Villefort again groaned with anguish and despair. 

 <Ah,> he exclaimed, <that woman became criminal only from associating with me! I carried the infection of crime with me, and she has caught it as she would the typhus fever, the cholera, the plague! And yet I have punished her—I have dared to tell her—\textit{I} have—<Repent and die!> But no, she must not die; she shall live, and with me. We will flee from Paris and go as far as the earth reaches. I told her of the scaffold; oh, Heavens, I forgot that it awaits me also! How could I pronounce that word? Yes, we will fly; I will confess all to her,—I will tell her daily that I also have committed a crime!—Oh, what an alliance—the tiger and the serpent; worthy wife of such as I am! She \textit{must} live that my infamy may diminish hers.> 

 And Villefort dashed open the window in front of the carriage. 

 <Faster, faster!> he cried, in a tone which electrified the coachman. The horses, impelled by fear, flew towards the house. 

 <Yes, yes,> repeated Villefort, as he approached his home—<yes, that woman must live; she must repent, and educate my son, the sole survivor, with the exception of the indestructible old man, of the wreck of my house. She loves him; it was for his sake she has committed these crimes. We ought never to despair of softening the heart of a mother who loves her child. She will repent, and no one will know that she has been guilty. The events which have taken place in my house, though they now occupy the public mind, will be forgotten in time, or if, indeed, a few enemies should persist in remembering them, why then I will add them to my list of crimes. What will it signify if one, two, or three more are added? My wife and child shall escape from this gulf, carrying treasures with them; she will live and may yet be happy, since her child, in whom all her love is centred, will be with her. I shall have performed a good action, and my heart will be lighter.> 

 And the procureur breathed more freely than he had done for some time.  The carriage stopped at the door of the house. Villefort leaped out of the carriage, and saw that his servants were surprised at his early return; he could read no other expression on their features. Neither of them spoke to him; they merely stood aside to let him pass by, as usual, nothing more. As he passed by M. Noirtier's room, he perceived two figures through the half-open door; but he experienced no curiosity to know who was visiting his father; anxiety carried him on further. 

 <Come,> he said, as he ascended the stairs leading to his wife's room, <nothing is changed here.> 

 He then closed the door of the landing. 

 <No one must disturb us,> he said; <I must speak freely to her, accuse myself, and say>—he approached the door, touched the crystal handle, which yielded to his hand. <Not locked,> he cried; <that is well.> 

 And he entered the little room in which Edward slept; for though the child went to school during the day, his mother could not allow him to be separated from her at night. With a single glance Villefort's eye ran through the room. 

 <Not here,> he said; <doubtless she is in her bedroom.> He rushed towards the door, found it bolted, and stopped, shuddering. 

 <Héloïse!> he cried. He fancied he heard the sound of a piece of furniture being removed. 

 <Héloïse!> he repeated. 

 <Who is there?> answered the voice of her he sought. He thought that voice more feeble than usual. 

 <Open the door!> cried Villefort. <Open; it is \textsc{i.}> 

 But notwithstanding this request, notwithstanding the tone of anguish in which it was uttered, the door remained closed. Villefort burst it open with a violent blow. At the entrance of the room which led to her boudoir, Madame de Villefort was standing erect, pale, her features contracted, and her eyes glaring horribly. 

 <Héloïse, Héloïse!> he said, <what is the matter? Speak!> The young woman extended her stiff white hands towards him. 

 <It is done, monsieur,> she said with a rattling noise which seemed to tear her throat. <What more do you want?> and she fell full length on the floor. 

 Villefort ran to her and seized her hand, which convulsively clasped a crystal bottle with a golden stopper. Madame de Villefort was dead. Villefort, maddened with horror, stepped back to the threshhold of the door, fixing his eyes on the corpse. 

 <My son!> he exclaimed suddenly, <where is my son?—Edward, Edward!> and he rushed out of the room, still crying, <Edward, Edward!> The name was pronounced in such a tone of anguish that the servants ran up. 

 <Where is my son?> asked Villefort; <let him be removed from the house, that he may not see\longdash> 

 <Master Edward is not downstairs, sir,> replied the valet. 

 <Then he must be playing in the garden; go and see.>

<No, sir; Madame de Villefort sent for him half an hour ago; he went into her room, and has not been downstairs since.> 

 A cold perspiration burst out on Villefort's brow; his legs trembled, and his thoughts flew about madly in his brain like the wheels of a disordered watch. 

 <In Madame de Villefort's room?> he murmured and slowly returned, with one hand wiping his forehead, and with the other supporting himself against the wall. To enter the room he must again see the body of his unfortunate wife. To call Edward he must reawaken the echo of that room which now appeared like a sepulchre; to speak seemed like violating the silence of the tomb. His tongue was paralysed in his mouth. 

 <Edward!> he stammered—<Edward!> 

 The child did not answer. Where, then, could he be, if he had entered his mother's room and not since returned? He stepped forward. The corpse of Madame de Villefort was stretched across the doorway leading to the room in which Edward must be; those glaring eyes seemed to watch over the threshold, and the lips bore the stamp of a terrible and mysterious irony. Through the open door was visible a portion of the boudoir, containing an upright piano and a blue satin couch. Villefort stepped forward two or three paces, and beheld his child lying—no doubt asleep—on the sofa. The unhappy man uttered an exclamation of joy; a ray of light seemed to penetrate the abyss of despair and darkness. He had only to step over the corpse, enter the boudoir, take the child in his arms, and flee far, far away. 

 Villefort was no longer the civilized man; he was a tiger hurt unto death, gnashing his teeth in his wound. He no longer feared realities, but phantoms. He leaped over the corpse as if it had been a burning brazier. He took the child in his arms, embraced him, shook him, called him, but the child made no response. He pressed his burning lips to the cheeks, but they were icy cold and pale; he felt the stiffened limbs; he pressed his hand upon the heart, but it no longer beat,—the child was dead. 

 A folded paper fell from Edward's breast. Villefort, thunderstruck, fell upon his knees; the child dropped from his arms, and rolled on the floor by the side of its mother. He picked up the paper, and, recognizing his wife's writing, ran his eyes rapidly over its contents; it ran as follows: 

 <You know that I was a good mother, since it was for my son's sake I became criminal. A good mother cannot depart without her son.> 

 Villefort could not believe his eyes,—he could not believe his reason; he dragged himself towards the child's body, and examined it as a lioness contemplates its dead cub. Then a piercing cry escaped from his breast, and he cried, 

 <Still the hand of God.> 

 The presence of the two victims alarmed him; he could not bear solitude shared only by two corpses. Until then he had been sustained by rage, by his strength of mind, by despair, by the supreme agony which led the Titans to scale the heavens, and Ajax to defy the gods. He now arose, his head bowed beneath the weight of grief, and, shaking his damp, dishevelled hair, he who had never felt compassion for anyone determined to seek his father, that he might have someone to whom he could relate his misfortunes,—someone by whose side he might weep.  He descended the little staircase with which we are acquainted, and entered Noirtier's room. The old man appeared to be listening attentively and as affectionately as his infirmities would allow to the Abbé Busoni, who looked cold and calm, as usual. Villefort, perceiving the abbé, passed his hand across his brow. The past came to him like one of those waves whose wrath foams fiercer than the others. 

 He recollected the call he had made upon him after the dinner at Auteuil, and then the visit the abbé had himself paid to his house on the day of Valentine's death. 

 <You here, sir!> he exclaimed; <do you, then, never appear but to act as an escort to death?> 

 Busoni turned around, and, perceiving the excitement depicted on the magistrate's face, the savage lustre of his eyes, he understood that the revelation had been made at the assizes; but beyond this he was ignorant. 

 <I came to pray over the body of your daughter.> 

 <And now why are you here?> 

 <I come to tell you that you have sufficiently repaid your debt, and that from this moment I will pray to God to forgive you, as I do.> 

 <Good heavens!> exclaimed Villefort, stepping back fearfully, <surely that is not the voice of the Abbé Busoni!> 

 <No!> The abbé threw off his wig, shook his head, and his hair, no longer confined, fell in black masses around his manly face. 

 <It is the face of the Count of Monte Cristo!> exclaimed the procureur, with a haggard expression. 

 <You are not exactly right, M. Procureur; you must go farther back.> 

 <That voice, that voice!—where did I first hear it?> 

 <You heard it for the first time at Marseilles, twenty-three years ago, the day of your marriage with Mademoiselle de Saint-Méran. Refer to your papers.> 

 <You are not Busoni?—you are not Monte Cristo? Oh, heavens! you are, then, some secret, implacable, and mortal enemy! I must have wronged you in some way at Marseilles. Oh, woe to me!> 

 <Yes; you are now on the right path,> said the count, crossing his arms over his broad chest; <search—search!> 

 <But what have I done to you?> exclaimed Villefort, whose mind was balancing between reason and insanity, in that cloud which is neither a dream nor reality; <what have I done to you? Tell me, then! Speak!> 

 <You condemned me to a horrible, tedious death; you killed my father; you deprived me of liberty, of love, and happiness.> 

 <Who are you, then? Who are you?> 

 <I am the spectre of a wretch you buried in the dungeons of the Château d'If. God gave that spectre the form of the Count of Monte Cristo when he at length issued from his tomb, enriched him with gold and diamonds, and led him to you!> 

 <Ah, I recognize you—I recognize you!> exclaimed the king's attorney; <you are\longdash> 

 <I am Edmond Dantès!> 

 <You are Edmond Dantès,> cried Villefort, seizing the count by the wrist; <then come here!> 

 And up the stairs he dragged Monte Cristo; who, ignorant of what had happened, followed him in astonishment, foreseeing some new catastrophe. 

 <There, Edmond Dantès!> he said, pointing to the bodies of his wife and child, <see, are you well avenged?> 

 Monte Cristo became pale at this horrible sight; he felt that he had passed beyond the bounds of vengeance, and that he could no longer say, <God is for and with me.> With an expression of indescribable anguish he threw himself upon the body of the child, reopened its eyes, felt its pulse, and then rushed with him into Valentine's room, of which he double-locked the door. 

 <My child,> cried Villefort, <he carries away the body of my child! Oh, curses, woe, death to you!> 

 He tried to follow Monte Cristo; but as though in a dream he was transfixed to the spot,—his eyes glared as though they were starting through the sockets; he griped the flesh on his chest until his nails were stained with blood; the veins of his temples swelled and boiled as though they would burst their narrow boundary, and deluge his brain with living fire. This lasted several minutes, until the frightful overturn of reason was accomplished; then uttering a loud cry followed by a burst of laughter, he rushed down the stairs. 

 A quarter of an hour afterwards the door of Valentine's room opened, and Monte Cristo reappeared. Pale, with a dull eye and heavy heart, all the noble features of that face, usually so calm and serene, were overcast by grief. In his arms he held the child, whom no skill had been able to recall to life. Bending on one knee, he placed it reverently by the side of its mother, with its head upon her breast. Then, rising, he went out, and meeting a servant on the stairs, he asked: 

 <Where is M. de Villefort?> 

 The servant, instead of answering, pointed to the garden. Monte Cristo ran down the steps, and advancing towards the spot designated beheld Villefort, encircled by his servants, with a spade in his hand, and digging the earth with fury. 

 <It is not here!> he cried. <It is not here!>  And then he moved farther on, and began again to dig. 

 Monte Cristo approached him, and said in a low voice, with an expression almost humble: 

 <Sir, you have indeed lost a son; but\longdash> 

 Villefort interrupted him; he had neither listened nor heard. 

 <Oh, I \textit{will} find it,> he cried; <you may pretend he is not here, but I \textit{will} find him, though I dig forever!> 

 Monte Cristo drew back in horror. 

 <Oh,> he said, <he is mad!> And as though he feared that the walls of the accursed house would crumble around him, he rushed into the street, for the first time doubting whether he had the right to do as he had done. <Oh, enough of this,—enough of this,> he cried; <let me save the last.> On entering his house, he met Morrel, who wandered about like a ghost awaiting the heavenly mandate for return to the tomb. 

 <Prepare yourself, Maximilian,> he said with a smile; <we leave Paris tomorrow.> 

 <Have you nothing more to do there?> asked Morrel. 

 <No,> replied Monte Cristo; <God grant I may not have done too much already.> 

 The next day they indeed left, accompanied only by Baptistin. Haydée had taken away Ali, and Bertuccio remained with Noirtier. 