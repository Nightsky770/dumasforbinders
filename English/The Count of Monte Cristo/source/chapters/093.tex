\chapter{Valentine} 

 \lettrine{W}{e} may easily conceive where Morrel's appointment was. On leaving Monte Cristo he walked slowly towards Villefort's; we say slowly, for Morrel had more than half an hour to spare to go five hundred steps, but he had hastened to take leave of Monte Cristo because he wished to be alone with his thoughts. He knew his time well—the hour when Valentine was giving Noirtier his breakfast, and was sure not to be disturbed in the performance of this pious duty. Noirtier and Valentine had given him leave to go twice a week, and he was now availing himself of that permission. 

 He arrived; Valentine was expecting him. Uneasy and almost crazed, she seized his hand and led him to her grandfather. This uneasiness, amounting almost to frenzy, arose from the report Morcerf's adventure had made in the world, for the affair at the Opera was generally known. No one at Villefort's doubted that a duel would ensue from it. Valentine, with her woman's instinct, guessed that Morrel would be Monte Cristo's second, and from the young man's well-known courage and his great affection for the count, she feared that he would not content himself with the passive part assigned to him. We may easily understand how eagerly the particulars were asked for, given, and received; and Morrel could read an indescribable joy in the eyes of his beloved, when she knew that the termination of this affair was as happy as it was unexpected. 

 <Now,> said Valentine, motioning to Morrel to sit down near her grandfather, while she took her seat on his footstool,—<now let us talk about our own affairs. You know, Maximilian, grandpapa once thought of leaving this house, and taking an apartment away from M. de Villefort's.> 

 <Yes,> said Maximilian, <I recollect the project, of which I highly approved.> 

 <Well,> said Valentine, <you may approve again, for grandpapa is again thinking of it.> 

 <Bravo,> said Maximilian.  <And do you know,> said Valentine, <what reason grandpapa gives for leaving this house.> Noirtier looked at Valentine to impose silence, but she did not notice him; her looks, her eyes, her smile, were all for Morrel. 

 <Oh, whatever may be M. Noirtier's reason,> answered Morrel, <I can readily believe it to be a good one.> 

 <An excellent one,> said Valentine. <He pretends the air of the Faubourg Saint-Honoré is not good for me.> 

 <Indeed?> said Morrel; <in that M. Noirtier may be right; you have not seemed to be well for the last fortnight.> 

 <Not very,> said Valentine. <And grandpapa has become my physician, and I have the greatest confidence in him, because he knows everything.> 

 <Do you then really suffer?> asked Morrel quickly. 

 <Oh, it must not be called suffering; I feel a general uneasiness, that is all. I have lost my appetite, and my stomach feels as if it were struggling to get accustomed to something.> Noirtier did not lose a word of what Valentine said. 

 <And what treatment do you adopt for this singular complaint?> 

 <A very simple one,> said Valentine. <I swallow every morning a spoonful of the mixture prepared for my grandfather. When I say one spoonful, I began by one—now I take four. Grandpapa says it is a panacea.> Valentine smiled, but it was evident that she suffered. 

 Maximilian, in his devotedness, gazed silently at her. She was very beautiful, but her usual pallor had increased; her eyes were more brilliant than ever, and her hands, which were generally white like mother-of-pearl, now more resembled wax, to which time was adding a yellowish hue. 

 From Valentine the young man looked towards Noirtier. The latter watched with strange and deep interest the young girl, absorbed by her affection, and he also, like Morrel, followed those traces of inward suffering which was so little perceptible to a common observer that they escaped the notice of everyone but the grandfather and the lover. 

 <But,> said Morrel, <I thought this mixture, of which you now take four spoonfuls, was prepared for M. Noirtier?> 

 <I know it is very bitter,> said Valentine; <so bitter, that all I drink afterwards appears to have the same taste.> Noirtier looked inquiringly at his granddaughter. <Yes, grandpapa,> said Valentine; <it is so. Just now, before I came down to you, I drank a glass of sugared water; I left half, because it seemed so bitter.> Noirtier turned pale, and made a sign that he wished to speak. 

 Valentine rose to fetch the dictionary. Noirtier watched her with evident anguish. In fact, the blood was rushing to the young girl's head already, her cheeks were becoming red. 

 <Oh,> cried she, without losing any of her cheerfulness, <this is singular! I can't see! Did the sun shine in my eyes?> And she leaned against the window. 

 <The sun is not shining,> said Morrel, more alarmed by Noirtier's expression than by Valentine's indisposition. He ran towards her. The young girl smiled. 

 <Cheer up,> said she to Noirtier. <Do not be alarmed, Maximilian; it is nothing, and has already passed away. But listen! Do I not hear a carriage in the courtyard?> She opened Noirtier's door, ran to a window in the passage, and returned hastily. <Yes,> said she, <it is Madame Danglars and her daughter, who have come to call on us. Good-bye;—I must run away, for they would send here for me, or, rather, farewell till I see you again. Stay with grandpapa, Maximilian; I promise you not to persuade them to stay.>  Morrel watched her as she left the room; he heard her ascend the little staircase which led both to Madame de Villefort's apartments and to hers. As soon as she was gone, Noirtier made a sign to Morrel to take the dictionary. Morrel obeyed; guided by Valentine, he had learned how to understand the old man quickly. Accustomed, however, as he was to the work, he had to repeat most of the letters of the alphabet and to find every word in the dictionary, so that it was ten minutes before the thought of the old man was translated by these words, 

 <Fetch the glass of water and the decanter from Valentine's room.> 

 Morrel rang immediately for the servant who had taken Barrois's situation, and in Noirtier's name gave that order. The servant soon returned. The decanter and the glass were completely empty. Noirtier made a sign that he wished to speak. 

 <Why are the glass and decanter empty?> asked he; <Valentine said she only drank half the glassful.> 

 The translation of this new question occupied another five minutes. 

 <I do not know,> said the servant, <but the housemaid is in Mademoiselle Valentine's room: perhaps she has emptied them.> 

 <Ask her,> said Morrel, translating Noirtier's thought this time by his look. The servant went out, but returned almost immediately. <Mademoiselle Valentine passed through the room to go to Madame de Villefort's,> said he; <and in passing, as she was thirsty, she drank what remained in the glass; as for the decanter, Master Edward had emptied that to make a pond for his ducks.> 

 Noirtier raised his eyes to heaven, as a gambler does who stakes his all on one stroke. From that moment the old man's eyes were fixed on the door, and did not quit it. 

 It was indeed Madame Danglars and her daughter whom Valentine had seen; they had been ushered into Madame de Villefort's room, who had said she would receive them there. That is why Valentine passed through her room, which was on a level with Valentine's, and only separated from it by Edward's. The two ladies entered the drawing-room with that sort of official stiffness which preludes a formal communication. Among worldly people manner is contagious. Madame de Villefort received them with equal solemnity. Valentine entered at this moment, and the formalities were resumed. 

 <My dear friend,> said the baroness, while the two young people were shaking hands, <I and Eugénie are come to be the first to announce to you the approaching marriage of my daughter with Prince Cavalcanti.> Danglars kept up the title of prince. The popular banker found that it answered better than count. 

 <Allow me to present you my sincere congratulations,> replied Madame de Villefort. <Prince Cavalcanti appears to be a young man of rare qualities.>

<Listen,> said the baroness, smiling; <speaking to you as a friend I can say that the prince does not yet appear all he will be. He has about him a little of that foreign manner by which French persons recognize, at first sight, the Italian or German nobleman. Besides, he gives evidence of great kindness of disposition, much keenness of wit, and as to suitability, M. Danglars assures me that his fortune is majestic—that is his word.> 

 <And then,> said Eugénie, while turning over the leaves of Madame de Villefort's album, <add that you have taken a great fancy to the young man.> 

 <And,> said Madame de Villefort, <I need not ask you if you share that fancy.> 

 <I?> replied Eugénie with her usual candour. <Oh, not the least in the world, madame! My wish was not to confine myself to domestic cares, or the caprices of any man, but to be an artist, and consequently free in heart, in person, and in thought.> 

 Eugénie pronounced these words with so firm a tone that the colour mounted to Valentine's cheeks. The timid girl could not understand that vigourous nature which appeared to have none of the timidities of woman. 

 <At any rate,> said she, <since I am to be married whether I will or not, I ought to be thankful to Providence for having released me from my engagement with M. Albert de Morcerf, or I should this day have been the wife of a dishonored man.> 

 <It is true,> said the baroness, with that strange simplicity sometimes met with among fashionable ladies, and of which plebeian intercourse can never entirely deprive them,—<it is very true that had not the Morcerfs hesitated, my daughter would have married Monsieur Albert. The general depended much on it; he even came to force M. Danglars. We have had a narrow escape.> 

 <But,> said Valentine, timidly, <does all the father's shame revert upon the son? Monsieur Albert appears to me quite innocent of the treason charged against the general.> 

 <Excuse me,> said the implacable young girl, <Monsieur Albert claims and well deserves his share. It appears that after having challenged M. de Monte Cristo at the Opera yesterday, he apologized on the ground today.> 

 <Impossible,> said Madame de Villefort. 

 <Ah, my dear friend,> said Madame Danglars, with the same simplicity we before noticed, <it is a fact. I heard it from M. Debray, who was present at the explanation.> 

 Valentine also knew the truth, but she did not answer. A single word had reminded her that Morrel was expecting her in M. Noirtier's room. Deeply engaged with a sort of inward contemplation, Valentine had ceased for a moment to join in the conversation. She would, indeed, have found it impossible to repeat what had been said the last few minutes, when suddenly Madame Danglars' hand, pressed on her arm, aroused her from her lethargy. 

 <What is it?> said she, starting at Madame Danglars' touch as she would have done from an electric shock. 

 <It is, my dear Valentine,> said the baroness, <that you are, doubtless, suffering.>

<I?> said the young girl, passing her hand across her burning forehead. 

 <Yes, look at yourself in that glass; you have turned pale and then red successively, three or four times in one minute.> 

 <Indeed,> cried Eugénie, <you are very pale!> 

 <Oh, do not be alarmed; I have been so for many days.> Artless as she was, the young girl knew that this was an opportunity to leave, and besides, Madame de Villefort came to her assistance. 

 <Retire, Valentine,> said she; <you are really suffering, and these ladies will excuse you; drink a glass of pure water, it will restore you.> 

 Valentine kissed Eugénie, bowed to Madame Danglars, who had already risen to take her leave, and went out. 

 <That poor child,> said Madame de Villefort when Valentine was gone, <she makes me very uneasy, and I should not be astonished if she had some serious illness.> 

 Meanwhile, Valentine, in a sort of excitement which she could not quite understand, had crossed Edward's room without noticing some trick of the child, and through her own had reached the little staircase. 

 She was within three steps of the bottom; she already heard Morrel's voice, when suddenly a cloud passed over her eyes, her stiffened foot missed the step, her hands had no power to hold the baluster, and falling against the wall she lost her balance wholly and toppled to the floor. Morrel bounded to the door, opened it, and found Valentine stretched out at the bottom of the stairs. Quick as a flash, he raised her in his arms and placed her in a chair. Valentine opened her eyes. 

 <Oh, what a clumsy thing I am,> said she with feverish volubility; <I don't know my way. I forgot there were three more steps before the landing.> 

 <You have hurt yourself, perhaps,> said Morrel. <What can I do for you, Valentine?> 

 Valentine looked around her; she saw the deepest terror depicted in Noirtier's eyes. 

 <Don't worry, dear grandpapa,> said she, endeavouring to smile; <it is nothing—it is nothing; I was giddy, that is all.> 

 <Another attack of giddiness,> said Morrel, clasping his hands. <Oh, attend to it, Valentine, I entreat you.> 

 <But no,> said Valentine,—<no, I tell you it is all past, and it was nothing. Now, let me tell you some news; Eugénie is to be married in a week, and in three days there is to be a grand feast, a betrothal festival. We are all invited, my father, Madame de Villefort, and I—at least, I understood it so.> 

 <When will it be our turn to think of these things? Oh, Valentine, you who have so much influence over your grandpapa, try to make him answer—Soon.> 

 <And do you,> said Valentine, <depend on me to stimulate the tardiness and arouse the memory of grandpapa?> 

 <Yes,> cried Morrel, <make haste. So long as you are not mine, Valentine, I shall always think I may lose you.> 

 <Oh,> replied Valentine with a convulsive movement, <oh, indeed, Maximilian, you are too timid for an officer, for a soldier who, they say, never knows fear. Ha, ha, ha!> 

 She burst into a forced and melancholy laugh, her arms stiffened and twisted, her head fell back on her chair, and she remained motionless. The cry of terror which was stopped on Noirtier's lips, seemed to start from his eyes. Morrel understood it; he knew he must call assistance. The young man rang the bell violently; the housemaid who had been in Mademoiselle Valentine's room, and the servant who had replaced Barrois, ran in at the same moment. Valentine was so pale, so cold, so inanimate that without listening to what was said to them they were seized with the fear which pervaded that house, and they flew into the passage crying for help. Madame Danglars and Eugénie were going out at that moment; they heard the cause of the disturbance. 

 <I told you so!> exclaimed Madame de Villefort. <Poor child!>  