\chapter{The Fifth of September} 

 \lettrine{T}{he} extension provided for by the agent of Thomson \& French, at the moment when Morrel expected it least, was to the poor shipowner so decided a stroke of good fortune that he almost dared to believe that fate was at length grown weary of wasting her spite upon him. The same day he told his wife, Emmanuel, and his daughter all that had occurred; and a ray of hope, if not of tranquillity, returned to the family. Unfortunately, however, Morrel had not only engagements with the house of Thomson \& French, who had shown themselves so considerate towards him; and, as he had said, in business he had correspondents, and not friends. When he thought the matter over, he could by no means account for this generous conduct on the part of Thomson \& French towards him; and could only attribute it to some such selfish argument as this: <We had better help a man who owes us nearly 300,000 francs, and have those 300,000 francs at the end of three months than hasten his ruin, and get only six or eight per cent of our money back again.> 

 Unfortunately, whether through envy or stupidity, all Morrel's correspondents did not take this view; and some even came to a contrary decision. The bills signed by Morrel were presented at his office with scrupulous exactitude, and, thanks to the delay granted by the Englishman, were paid by Cocles with equal punctuality. Cocles thus remained in his accustomed tranquillity. It was Morrel alone who remembered with alarm, that if he had to repay on the 15th the 50,000 francs of M. de Boville, and on the 30th the 32,500 francs of bills, for which, as well as the debt due to the inspector of prisons, he had time granted, he must be a ruined man. 

 The opinion of all the commercial men was that, under the reverses which had successively weighed down Morrel, it was impossible for him to remain solvent. Great, therefore, was the astonishment when at the end of the month, he cancelled all his obligations with his usual punctuality. Still confidence was not restored to all minds, and the general opinion was that the complete ruin of the unfortunate shipowner had been postponed only until the end of the month. 

 The month passed, and Morrel made extraordinary efforts to get in all his resources. Formerly his paper, at any date, was taken with confidence, and was even in request. Morrel now tried to negotiate bills at ninety days only, and none of the banks would give him credit. Fortunately, Morrel had some funds coming in on which he could rely; and, as they reached him, he found himself in a condition to meet his engagements when the end of July came. 

 The agent of Thomson \& French had not been again seen at Marseilles; the day after, or two days after his visit to Morrel, he had disappeared; and as in that city he had had no intercourse but with the mayor, the inspector of prisons, and M. Morrel, his departure left no trace except in the memories of these three persons. As to the sailors of the \textit{Pharaon}, they must have found snug berths elsewhere, for they also had disappeared. 

 Captain Gaumard, recovered from his illness, had returned from Palma. He delayed presenting himself at Morrel's, but the owner, hearing of his arrival, went to see him. The worthy shipowner knew, from Penelon's recital, of the captain's brave conduct during the storm, and tried to console him. He brought him also the amount of his wages, which Captain Gaumard had not dared to apply for. 

 As he descended the staircase, Morrel met Penelon, who was going up. Penelon had, it would seem, made good use of his money, for he was newly clad. When he saw his employer, the worthy tar seemed much embarrassed, drew on one side into the corner of the landing-place, passed his quid from one cheek to the other, stared stupidly with his great eyes, and only acknowledged the squeeze of the hand which Morrel as usual gave him by a slight pressure in return. Morrel attributed Penelon's embarrassment to the elegance of his attire; it was evident the good fellow had not gone to such an expense on his own account; he was, no doubt, engaged on board some other vessel, and thus his bashfulness arose from the fact of his not having, if we may so express ourselves, worn mourning for the \textit{Pharaon} longer. Perhaps he had come to tell Captain Gaumard of his good luck, and to offer him employment from his new master. 

 <Worthy fellows!> said Morrel,  as he went away, <may your new master love you as I loved you, and be more fortunate than I have been!>  August rolled by in unceasing efforts on the part of Morrel to renew his credit or revive the old. On the 20th of August it was known at Marseilles that he had left town in the mailcoach, and then it was said that the bills would go to protest at the end of the month, and that Morrel had gone away and left his chief clerk Emmanuel, and his cashier Cocles, to meet the creditors. But, contrary to all expectation, when the 31st of August came, the house opened as usual, and Cocles appeared behind the grating of the counter, examined all bills presented with the usual scrutiny, and, from first to last, paid all with the usual precision. There came in, moreover, two drafts which M. Morrel had fully anticipated, and which Cocles paid as punctually as the bills which the shipowner had accepted. All this was incomprehensible, and then, with the tenacity peculiar to prophets of bad news, the failure was put off until the end of September. 

 On the 1st, Morrel returned; he was awaited by his family with extreme anxiety, for from this journey to Paris they hoped great things. Morrel had thought of Danglars, who was now immensely rich, and had lain under great obligations to Morrel in former days, since to him it was owing that Danglars entered the service of the Spanish banker, with whom he had laid the foundations of his vast wealth. It was said at this moment that Danglars was worth from six to eight millions of francs, and had unlimited credit. Danglars, then, without taking a crown from his pocket, could save Morrel; he had but to pass his word for a loan, and Morrel was saved. Morrel had long thought of Danglars, but had kept away from some instinctive motive, and had delayed as long as possible availing himself of this last resource. And Morrel was right, for he returned home crushed by the humiliation of a refusal. 

 Yet, on his arrival, Morrel did not utter a complaint, or say one harsh word. He embraced his weeping wife and daughter, pressed Emmanuel's hand with friendly warmth, and then going to his private room on the second floor had sent for Cocles. 

 <Then,> said the two women to Emmanuel, <we are indeed ruined.> 

 It was agreed in a brief council held among them, that Julie should write to her brother, who was in garrison at Nîmes, to come to them as speedily as possible. The poor women felt instinctively that they required all their strength to support the blow that impended. Besides, Maximilian Morrel, though hardly two-and-twenty, had great influence over his father. 

 He was a strong-minded, upright young man. At the time when he decided on his profession his father had no desire to choose for him, but had consulted young Maximilian's taste. He had at once declared for a military life, and had in consequence studied hard, passed brilliantly through the Polytechnic School, and left it as sub-lieutenant of the 53rd of the line. For a year he had held this rank, and expected promotion on the first vacancy. In his regiment Maximilian Morrel was noted for his rigid observance, not only of the obligations imposed on a soldier, but also of the duties of a man; and he thus gained the name of <the stoic.> We need hardly say that many of those who gave him this epithet repeated it because they had heard it, and did not even know what it meant. 

 This was the young man whom his mother and sister called to their aid to sustain them under the serious trial which they felt they would soon have to endure. They had not mistaken the gravity of this event, for the moment after Morrel had entered his private office with Cocles, Julie saw the latter leave it pale, trembling, and his features betraying the utmost consternation. She would have questioned him as he passed by her, but the worthy creature hastened down the staircase with unusual precipitation, and only raised his hands to heaven and exclaimed: 

 <Oh, mademoiselle, mademoiselle, what a dreadful misfortune! Who could ever have believed it!> 

 A moment afterwards Julie saw him go upstairs carrying two or three heavy ledgers, a portfolio, and a bag of money. 

 Morrel examined the ledgers, opened the portfolio, and counted the money. All his funds amounted to 6,000 or 8,000 francs, his bills receivable up to the 5th to 4,000 or 5,000, which, making the best of everything, gave him 14,000 francs to meet debts amounting to 287,500 francs. He had not even the means for making a possible settlement on account. 

 However, when Morrel went down to his dinner, he appeared very calm. This calmness was more alarming to the two women than the deepest dejection would have been. After dinner Morrel usually went out and used to take his coffee at the club of the Phocéens, and read the \textit{Semaphore}; this day he did not leave the house, but returned to his office. 

 As to Cocles, he seemed completely bewildered. For part of the day he went into the courtyard, seated himself on a stone with his head bare and exposed to the blazing sun. Emmanuel tried to comfort the women, but his eloquence faltered. The young man was too well acquainted with the business of the house, not to feel that a great catastrophe hung over the Morrel family. Night came, the two women had watched, hoping that when he left his room Morrel would come to them, but they heard him pass before their door, and trying to conceal the noise of his footsteps. They listened; he went into his sleeping-room, and fastened the door inside. Madame Morrel sent her daughter to bed, and half an hour after Julie had retired, she rose, took off her shoes, and went stealthily along the passage, to see through the keyhole what her husband was doing. 

 In the passage she saw a retreating shadow; it was Julie, who, uneasy herself, had anticipated her mother. The young lady went towards Madame Morrel. 

 <He is writing,> she said. 

 They had understood each other without speaking. Madame Morrel looked again through the keyhole, Morrel was writing; but Madame Morrel remarked, what her daughter had not observed, that her husband was writing on stamped paper. The terrible idea that he was writing his will flashed across her; she shuddered, and yet had not strength to utter a word. 

 Next day M. Morrel seemed as calm as ever, went into his office as usual, came to his breakfast punctually, and then, after dinner, he placed his daughter beside him, took her head in his arms, and held her for a long time against his bosom. In the evening, Julie told her mother, that although he was apparently so calm, she had noticed that her father's heart beat violently. 

 The next two days passed in much the same way. On the evening of the 4th of September, M. Morrel asked his daughter for the key of his study. Julie trembled at this request, which seemed to her of bad omen. Why did her father ask for this key which she always kept, and which was only taken from her in childhood as a punishment? The young girl looked at Morrel. 

 <What have I done wrong, father,> she said, <that you should take this key from me?> 

 <Nothing, my dear,> replied the unhappy man, the tears starting to his eyes at this simple question,—<nothing, only I want it.> 

 Julie made a pretence to feel for the key. <I must have left it in my room,> she said. 

 And she went out, but instead of going to her apartment she hastened to consult Emmanuel. 

 <Do not give this key to your father,> said he, <and tomorrow morning, if possible, do not quit him for a moment.> 

 She questioned Emmanuel, but he knew nothing, or would not say what he knew. 

 During the night, between the 4th and 5th of September, Madame Morrel remained listening for every sound, and, until three o'clock in the morning, she heard her husband pacing the room in great agitation. It was three o'clock when he threw himself on the bed. The mother and daughter passed the night together. They had expected Maximilian since the previous evening. At eight o'clock in the morning Morrel entered their chamber. He was calm; but the agitation of the night was legible in his pale and careworn visage. They did not dare to ask him how he had slept. Morrel was kinder to his wife, more affectionate to his daughter, than he had ever been. He could not cease gazing at and kissing the sweet girl. Julie, mindful of Emmanuel's request, was following her father when he quitted the room, but he said to her quickly: 

 <Remain with your mother, dearest.> Julie wished to accompany him. <I wish you to do so,> said he. 

 This was the first time Morrel had ever so spoken, but he said it in a tone of paternal kindness, and Julie did not dare to disobey. She remained at the same spot standing mute and motionless. An instant afterwards the door opened, she felt two arms encircle her, and a mouth pressed her forehead. She looked up and uttered an exclamation of joy.  <Maximilian, my dearest brother!> she cried. 

 At these words Madame Morrel rose, and threw herself into her son's arms. 

 <Mother,> said the young man, looking alternately at Madame Morrel and her daughter, <what has occurred—what has happened? Your letter has frightened me, and I have come hither with all speed.> 

 <Julie,> said Madame Morrel, making a sign to the young man, <go and tell your father that Maximilian has just arrived.> 

 The young lady rushed out of the apartment, but on the first step of the staircase she found a man holding a letter in his hand. 

 <Are you not Mademoiselle Julie Morrel?> inquired the man, with a strong Italian accent. 

 <Yes, sir,> replied Julie with hesitation; <what is your pleasure? I do not know you.> 

 <Read this letter,> he said, handing it to her. Julie hesitated. <It concerns the best interests of your father,> said the messenger. 

 The young girl hastily took the letter from him. She opened it quickly and read: 

\begin{mail}{}{}
	Go this moment to the Allées de Meilhan, enter the house № 15, ask the porter for the key of the room on the fifth floor, enter the apartment, take from the corner of the mantelpiece a purse netted in red silk, and give it to your father. It is important that he should receive it before eleven o'clock. You promised to obey me implicitly. Remember your oath. 
\closeletter{Sinbad the Sailor.}
\end{mail}

 The young girl uttered a joyful cry, raised her eyes, looked round to question the messenger, but he had disappeared. She cast her eyes again over the note to peruse it a second time, and saw there was a postscript. She read: 

 <It is important that you should fulfil this mission in person and alone. If you go accompanied by any other person, or should anyone else go in your place, the porter will reply that he does not know anything about it.> 

 This postscript decreased greatly the young girl's happiness. Was there nothing to fear? was there not some snare laid for her? Her innocence had kept her in ignorance of the dangers that might assail a young girl of her age. But there is no need to know danger in order to fear it; indeed, it may be observed, that it is usually unknown perils that inspire the greatest terror. 

 Julie hesitated, and resolved to take counsel. Yet, through a singular impulse, it was neither to her mother nor her brother that she applied, but to Emmanuel. She hastened down and told him what had occurred on the day when the agent of Thomson \& French had come to her father's, related the scene on the staircase, repeated the promise she had made, and showed him the letter. 

 <You must go, then, mademoiselle,> said Emmanuel. 

 <Go there?> murmured Julie. 

 <Yes; I will accompany you.> 

 <But did you not read that I must be alone?> said Julie. 

 <And you shall be alone,> replied the young man. <I will await you at the corner of the Rue du Musée, and if you are so long absent as to make me uneasy, I will hasten to rejoin you, and woe to him of whom you shall have cause to complain to me!> 

 <Then, Emmanuel?> said the young girl with hesitation, <it is your opinion that I should obey this invitation?> 

 <Yes. Did not the messenger say your father's safety depended upon it?> 

 <But what danger threatens him, then, Emmanuel?> she asked. 

 Emmanuel hesitated a moment, but his desire to make Julie decide immediately made him reply. 

 <Listen,> he said; <today is the 5th of September, is it not?> 

 <Yes.> 

 <Today, then, at eleven o'clock, your father has nearly three hundred thousand francs to pay?> 

 <Yes, we know that.> 

 <Well, then,> continued Emmanuel, <we have not fifteen thousand francs in the house.> 

 <What will happen then?> 

 <Why, if today before eleven o'clock your father has not found someone who will come to his aid, he will be compelled at twelve o'clock to declare himself a bankrupt.> 

 <Oh, come, then, come!> cried she, hastening away with the young man. 

 During this time, Madame Morrel had told her son everything. The young man knew quite well that, after the succession of misfortunes which had befallen his father, great changes had taken place in the style of living and housekeeping; but he did not know that matters had reached such a point. He was thunderstruck. Then, rushing hastily out of the apartment, he ran upstairs, expecting to find his father in his study, but he rapped there in vain. 

 While he was yet at the door of the study he heard the bedroom door open, turned, and saw his father. Instead of going direct to his study, M. Morrel had returned to his bedchamber, which he was only this moment quitting. Morrel uttered a cry of surprise at the sight of his son, of whose arrival he was ignorant. He remained motionless on the spot, pressing with his left hand something he had concealed under his coat. Maximilian sprang down the staircase, and threw his arms round his father's neck; but suddenly he recoiled, and placed his right hand on Morrel's breast. 

 <Father,> he exclaimed, turning pale as death, <what are you going to do with that brace of pistols under your coat?> 

 <Oh, this is what I feared!> said Morrel. 

 <Father, father, in Heaven's name,> exclaimed the young man, <what are these weapons for?> 

 <Maximilian,> replied Morrel, looking fixedly at his son, <you are a man, and a man of honour. Come, and I will explain to you.> 

 And with a firm step Morrel went up to his study, while Maximilian followed him, trembling as he went. Morrel opened the door, and closed it behind his son; then, crossing the anteroom, went to his desk on which he placed the pistols, and pointed with his finger to an open ledger. In this ledger was made out an exact balance-sheet of his affairs. Morrel had to pay, within half an hour, 287,500 francs. All he possessed was 15,257 francs. 

 <Read!> said Morrel. 

 The young man was overwhelmed as he read. Morrel said not a word. What could he say? What need he add to such a desperate proof in figures? 

 <And have you done all that is possible, father, to meet this disastrous result?> asked the young man, after a moment's pause. 

 <I have,> replied Morrel. 

 <You have no money coming in on which you can rely?> 

 <None.> 

 <You have exhausted every resource?> 

 <All.> 

 <And in half an hour,> said Maximilian in a gloomy voice, <our name is dishonored!> 

 <Blood washes out dishonor,> said Morrel. 

 <You are right, father; I understand you.> Then extending his hand towards one of the pistols, he said, <There is one for you and one for me—thanks!> 

 Morrel caught his hand. <Your mother—your sister! Who will support them?> 

 A shudder ran through the young man's frame. <Father,> he said, <do you reflect that you are bidding me to live?> 

 <Yes, I do so bid you,> answered Morrel, <it is your duty. You have a calm, strong mind, Maximilian. Maximilian, you are no ordinary man. I make no requests or commands; I only ask you to examine my position as if it were your own, and then judge for yourself.> 

 The young man reflected for a moment, then an expression of sublime resignation appeared in his eyes, and with a slow and sad gesture he took off his two epaulets, the insignia of his rank. 

 <Be it so, then, my father,> he said, extending his hand to Morrel, <die in peace, my father; I will live.> 

 Morrel was about to cast himself on his knees before his son, but Maximilian caught him in his arms, and those two noble hearts were pressed against each other for a moment. 

 <You know it is not my fault,> said Morrel.  Maximilian smiled. <I know, father, you are the most honourable man I have ever known.> 

 <Good, my son. And now there is no more to be said; go and rejoin your mother and sister.> 

 <My father,> said the young man, bending his knee, <bless me!> Morrel took the head of his son between his two hands, drew him forward, and kissing his forehead several times said: 

 <Oh, yes, yes, I bless you in my own name, and in the name of three generations of irreproachable men, who say through me, <The edifice which misfortune has destroyed, Providence may build up again.> On seeing me die such a death, the most inexorable will have pity on you. To you, perhaps, they will accord the time they have refused to me. Then do your best to keep our name free from dishonor. Go to work, labour, young man, struggle ardently and courageously; live, yourself, your mother and sister, with the most rigid economy, so that from day to day the property of those whom I leave in your hands may augment and fructify. Reflect how glorious a day it will be, how grand, how solemn, that day of complete restoration, on which you will say in this very office, <My father died because he could not do what I have this day done; but he died calmly and peaceably, because in dying he knew what I should do.>> 

 <My father, my father!> cried the young man, <why should you not live?> 

 <If I live, all would be changed; if I live, interest would be converted into doubt, pity into hostility; if I live I am only a man who has broken his word, failed in his engagements—in fact, only a bankrupt. If, on the contrary, I die, remember, Maximilian, my corpse is that of an honest but unfortunate man. Living, my best friends would avoid my house; dead, all Marseilles will follow me in tears to my last home. Living, you would feel shame at my name; dead, you may raise your head and say, <I am the son of him you killed, because, for the first time, he has been compelled to break his word.>> 

 The young man uttered a groan, but appeared resigned. 

 <And now,> said Morrel, <leave me alone, and endeavour to keep your mother and sister away.> 

 <Will you not see my sister once more?> asked Maximilian. A last but final hope was concealed by the young man in the effect of this interview, and therefore he had suggested it. Morrel shook his head. <I saw her this morning, and bade her adieu.> 

 <Have you no particular commands to leave with me, my father?> inquired Maximilian in a faltering voice. 

 <Yes; my son, and a sacred command.> 

 <Say it, my father.> 

 <The house of Thomson \& French is the only one who, from humanity, or, it may be, selfishness—it is not for me to read men's hearts—has had any pity for me. Its agent, who will in ten minutes present himself to receive the amount of a bill of 287,500 francs, I will not say granted, but offered me three months. Let this house be the first repaid, my son, and respect this man.> 

 <Father, I will,> said Maximilian. 

 <And now, once more, adieu,> said Morrel. <Go, leave me; I would be alone. You will find my will in the secretaire in my bedroom.> 

 The young man remained standing and motionless, having but the force of will and not the power of execution. 

 <Hear me, Maximilian,> said his father. <Suppose I were a soldier like you, and ordered to carry a certain redoubt, and you knew I must be killed in the assault, would you not say to me, as you said just now, <Go, father; for you are dishonored by delay, and death is preferable to shame!>> 

 <Yes, yes,> said the young man, <yes;> and once again embracing his father with convulsive pressure, he said, <Be it so, my father.> 

 And he rushed out of the study. When his son had left him, Morrel remained an instant standing with his eyes fixed on the door; then putting forth his arm, he pulled the bell. After a moment's interval, Cocles appeared. 

 It was no longer the same man—the fearful revelations of the three last days had crushed him. This thought—the house of Morrel is about to stop payment—bent him to the earth more than twenty years would otherwise have done. 

 <My worthy Cocles,> said Morrel in a tone impossible to describe, <do you remain in the antechamber. When the gentleman who came three months ago—the agent of Thomson \& French—arrives, announce his arrival to me.> 

 Cocles made no reply; he made a sign with his head, went into the anteroom, and seated himself. Morrel fell back in his chair, his eyes fixed on the clock; there were seven minutes left, that was all. The hand moved on with incredible rapidity, he seemed to see its motion. 

 What passed in the mind of this man at the supreme moment of his agony cannot be told in words. He was still comparatively young, he was surrounded by the loving care of a devoted family, but he had convinced himself by a course of reasoning, illogical perhaps, yet certainly plausible, that he must separate himself from all he held dear in the world, even life itself. To form the slightest idea of his feelings, one must have seen his face with its expression of enforced resignation and its tear-moistened eyes raised to heaven. The minute hand moved on. The pistols were loaded; he stretched forth his hand, took one up, and murmured his daughter's name. Then he laid it down, seized his pen, and wrote a few words. It seemed to him as if he had not taken a sufficient farewell of his beloved daughter. Then he turned again to the clock, counting time now not by minutes, but by seconds. 

 He took up the deadly weapon again, his lips parted and his eyes fixed on the clock, and then shuddered at the click of the trigger as he cocked the pistol. At this moment of mortal anguish the cold sweat came forth upon his brow, a pang stronger than death clutched at his heart-strings. He heard the door of the staircase creak on its hinges—the clock gave its warning to strike eleven—the door of his study opened. Morrel did not turn round—he expected these words of Cocles, <The agent of Thomson \& French.> 

 He placed the muzzle of the pistol between his teeth. Suddenly he heard a cry—it was his daughter's voice. He turned and saw Julie. The pistol fell from his hands. 

 <My father!> cried the young girl, out of breath, and half dead with joy—<saved, you are saved!> And she threw herself into his arms, holding in her extended hand a red, netted silk purse.  <Saved, my child!> said Morrel; <what do you mean?> 

 <Yes, saved—saved! See, see!> said the young girl. 

 Morrel took the purse, and started as he did so, for a vague remembrance reminded him that it once belonged to himself. At one end was the receipted bill for the 287,000 francs, and at the other was a diamond as large as a hazel-nut, with these words on a small slip of parchment: \textit{Julie's Dowry}. 

 Morrel passed his hand over his brow; it seemed to him a dream. At this moment the clock struck eleven. He felt as if each stroke of the hammer fell upon his heart. 

 <Explain, my child,> he said, <Explain, my child,> he said, <explain—where did you find this purse?> 

 <In a house in the Allées de Meilhan, № 15, on the corner of a mantelpiece in a small room on the fifth floor.> 

 <But,> cried Morrel, <this purse is not yours!> Julie handed to her father the letter she had received in the morning. 

 <And did you go alone?> asked Morrel, after he had read it. 

 <Emmanuel accompanied me, father. He was to have waited for me at the corner of the Rue du Musée, but, strange to say, he was not there when I returned.> 

 <Monsieur Morrel!> exclaimed a voice on the stairs; <Monsieur Morrel!> 

 <It is his voice!> said Julie. At this moment Emmanuel entered, his countenance full of animation and joy. 

 <The \textit{Pharaon}!> he cried; <the \textit{Pharaon}!> 

 <What!—what!—the \textit{Pharaon}! Are you mad, Emmanuel? You know the vessel is lost.> 

 <The \textit{Pharaon}, sir—they signal the \textit{Pharaon}! The \textit{Pharaon} is entering the harbour!> 

 Morrel fell back in his chair, his strength was failing him; his understanding weakened by such events, refused to comprehend such incredible, unheard-of, fabulous facts. But his son came in. 

 <Father,> cried Maximilian, <how could you say the \textit{Pharaon} was lost? The lookout has signalled her, and they say she is now coming into port.>  <My dear friends,> said Morrel, <if this be so, it must be a miracle of heaven! Impossible, impossible!> 

 But what was real and not less incredible was the purse he held in his hand, the acceptance receipted—the splendid diamond. 

 <Ah, sir,> exclaimed Cocles, <what can it mean?—the \textit{Pharaon}?> 

 <Come, dear ones,> said Morrel, rising from his seat, <let us go and see, and Heaven have pity upon us if it be false intelligence!> 

 They all went out, and on the stairs met Madame Morrel, who had been afraid to go up into the study. In a moment they were at the Canebière. There was a crowd on the pier. All the crowd gave way before Morrel. <The \textit{Pharaon}! the \textit{Pharaon}!> said every voice. 

 And, wonderful to see, in front of the tower of Saint-Jean, was a ship bearing on her stern these words, printed in white letters, <The \textit{Pharaon}, Morrel \& Son, of Marseilles.> She was the exact duplicate of the other \textit{Pharaon}, and loaded, as that had been, with cochineal and indigo. She cast anchor, clued up sails, and on the deck was Captain Gaumard giving orders, and good old Penelon making signals to M. Morrel. To doubt any longer was impossible; there was the evidence of the senses, and ten thousand persons who came to corroborate the testimony. 

 As Morrel and his son embraced on the pier-head, in the presence and amid the applause of the whole city witnessing this event, a man, with his face half-covered by a black beard, and who, concealed behind the sentry-box, watched the scene with delight, uttered these words in a low tone: 

 <Be happy, noble heart, be blessed for all the good thou hast done and wilt do hereafter, and let my gratitude remain in obscurity like your good deeds.>  And with a smile expressive of supreme content, he left his hiding-place, and without being observed, descended one of the flights of steps provided for debarkation, and hailing three times, shouted <Jacopo, Jacopo, Jacopo!> 

 Then a launch came to shore, took him on board, and conveyed him to a yacht splendidly fitted up, on whose deck he sprung with the activity of a sailor; thence he once again looked towards Morrel, who, weeping with joy, was shaking hands most cordially with all the crowd around him, and thanking with a look the unknown benefactor whom he seemed to be seeking in the skies. 

 <And now,> said the unknown, <farewell kindness, humanity, and gratitude! Farewell to all the feelings that expand the heart! I have been Heaven's substitute to recompense the good—now the god of vengeance yields to me his power to punish the wicked!> 

 At these words he gave a signal, and, as if only awaiting this signal, the yacht instantly put out to sea. 