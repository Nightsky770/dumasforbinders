\chapter{The Departure for Belgium}
	
	\lettrine{A}{} few minutes after the scene of confusion produced in the salons of M. Danglars by the unexpected appearance of the brigade of soldiers, and by the disclosure which had followed, the mansion was deserted with as much rapidity as if a case of plague or of cholera morbus had broken out among the guests. 

 In a few minutes, through all the doors, down all the staircases, by every exit, everyone hastened to retire, or rather to fly; for it was a situation where the ordinary condolences,—which even the best friends are so eager to offer in great catastrophes,—were seen to be utterly futile. There remained in the banker's house only Danglars, closeted in his study, and making his statement to the officer of gendarmes; Madame Danglars, terrified, in the boudoir with which we are acquainted; and Eugénie, who with haughty air and disdainful lip had retired to her room with her inseparable companion, Mademoiselle Louise d'Armilly. 

 As for the numerous servants (more numerous that evening than usual, for their number was augmented by cooks and butlers from the Café de Paris), venting on their employers their anger at what they termed the insult to which they had been subjected, they collected in groups in the hall, in the kitchens, or in their rooms, thinking very little of their duty, which was thus naturally interrupted. Of all this household, only two persons deserve our notice; these are Mademoiselle Eugénie Danglars and Mademoiselle Louise d'Armilly. 

 The betrothed had retired, as we said, with haughty air, disdainful lip, and the demeanour of an outraged queen, followed by her companion, who was paler and more disturbed than herself. On reaching her room Eugénie locked her door, while Louise fell on a chair. 

 <Ah, what a dreadful thing,> said the young musician; <who would have suspected it? M. Andrea Cavalcanti a murderer—a galley-slave escaped—a convict!> 

 An ironical smile curled the lip of Eugénie. <In truth, I was fated,> said she. <I escaped the Morcerf only to fall into the Cavalcanti.> 

 <Oh, do not confound the two, Eugénie.> 

 <Hold your tongue! The men are all infamous, and I am happy to be able now to do more than detest them—I despise them.> 

 <What shall we do?> asked Louise. 

 <What shall we do?> 

 <Yes.> 

 <Why, the same we had intended doing three days since—set off.> 

 <What?—although you are not now going to be married, you intend still\longdash> 

 <Listen, Louise. I hate this life of the fashionable world, always ordered, measured, ruled, like our music-paper. What I have always wished for, desired, and coveted, is the life of an artist, free and independent, relying only on my own resources, and accountable only to myself. Remain here? What for?—that they may try, a month hence, to marry me again; and to whom?—M. Debray, perhaps, as it was once proposed. No, Louise, no! This evening's adventure will serve for my excuse. I did not seek one, I did not ask for one. God sends me this, and I hail it joyfully!> 

 <How strong and courageous you are!> said the fair, frail girl to her brunette companion. 

 <Did you not yet know me? Come, Louise, let us talk of our affairs. The post-chaise\longdash> 

 <Was happily bought three days since.> 

 <Have you had it sent where we are to go for it?> 

 <Yes.> 

 <Our passport?> 

 <Here it is.> 

 And Eugénie, with her usual precision, opened a printed paper, and read: 

 <M. Léon d'Armilly, twenty years of age; profession, artist; hair black, eyes black; travelling with his sister.> 

 <Capital! How did you get this passport?> 

 <When I went to ask M. de Monte Cristo for letters to the directors of the theatres at Rome and Naples, I expressed my fears of travelling as a woman; he perfectly understood them, and undertook to procure for me a man's passport, and two days after I received this, to which I have added with my own hand, <travelling with his sister.>>  <Well,> said Eugénie cheerfully, <we have then only to pack up our trunks; we shall start the evening of the signing of the contract, instead of the evening of the wedding—that is all.> 

 <But consider the matter seriously, Eugénie!> 

 <Oh, I am done with considering! I am tired of hearing only of market reports, of the end of the month, of the rise and fall of Spanish funds, of Haitian bonds. Instead of that, Louise—do you understand?—air, liberty, melody of birds, plains of Lombardy, Venetian canals, Roman palaces, the Bay of Naples. How much have we, Louise?> 

 The young girl to whom this question was addressed drew from an inlaid secretaire a small portfolio with a lock, in which she counted twenty-three bank-notes. 

 <Twenty-three thousand francs,> said she. 

 <And as much, at least, in pearls, diamonds, and jewels,> said Eugénie. <We are rich. With forty-five thousand francs we can live like princesses for two years, and comfortably for four; but before six months—you with your music, and I with my voice—we shall double our capital. Come, you shall take charge of the money, I of the jewel-box; so that if one of us had the misfortune to lose her treasure, the other would still have hers left. Now, the portmanteau—let us make haste—the portmanteau!> 

 <Stop!> said Louise, going to listen at Madame Danglars' door. 

 <What do you fear?> 

 <That we may be discovered.> 

 <The door is locked.> 

 <They may tell us to open it.> 

 <They may if they like, but we will not.> 

 <You are a perfect Amazon, Eugénie!> And the two young girls began to heap into a trunk all the things they thought they should require. 

 <There now,> said Eugénie, <while I change my costume do you lock the portmanteau.> Louise pressed with all the strength of her little hands on the top of the portmanteau. 

 <But I cannot,> said she; <I am not strong enough; do you shut it.> 

 <Ah, you do well to ask,> said Eugénie, laughing; <I forgot that I was Hercules, and you only the pale Omphale!> 

 And the young girl, kneeling on the top, pressed the two parts of the portmanteau together, and Mademoiselle d'Armilly passed the bolt of the padlock through. When this was done, Eugénie opened a drawer, of which she kept the key, and took from it a wadded violet silk travelling cloak. 

 <Here,> said she, <you see I have thought of everything; with this cloak you will not be cold.> 

 <But you?> 

 <Oh, I am never cold, you know! Besides, with these men's clothes\longdash> 

 <Will you dress here?> 

 <Certainly.> 

 <Shall you have time?> 

 <Do not be uneasy, you little coward! All our servants are busy, discussing the grand affair. Besides, what is there astonishing, when you think of the grief I ought to be in, that I shut myself up?—tell me!> 

 <No, truly—you comfort me.> 

 <Come and help me.> 

 From the same drawer she took a man's complete costume, from the boots to the coat, and a provision of linen, where there was nothing superfluous, but every requisite. Then, with a promptitude which indicated that this was not the first time she had amused herself by adopting the garb of the opposite sex, Eugénie drew on the boots and pantaloons, tied her cravat, buttoned her waistcoat up to the throat, and put on a coat which admirably fitted her beautiful figure. 

 <Oh, that is very good—indeed, it is very good!> said Louise, looking at her with admiration; <but that beautiful black hair, those magnificent braids, which made all the ladies sigh with envy,—will they go under a man's hat like the one I see down there?> 

 <You shall see,> said Eugénie. And with her left hand seizing the thick mass, which her long fingers could scarcely grasp, she took in her right hand a pair of long scissors, and soon the steel met through the rich and splendid hair, which fell in a cluster at her feet as she leaned back to keep it from her coat. Then she grasped the front hair, which she also cut off, without expressing the least regret; on the contrary, her eyes sparkled with greater pleasure than usual under her ebony eyebrows.  <Oh, the magnificent hair!> said Louise, with regret. 

 <And am I not a hundred times better thus?> cried Eugénie, smoothing the scattered curls of her hair, which had now quite a masculine appearance; <and do you not think me handsomer so?> 

 <Oh, you are beautiful—always beautiful!> cried Louise. <Now, where are you going?> 

 <To Brussels, if you like; it is the nearest frontier. We can go to Brussels, Liège, Aix-la-Chapelle; then up the Rhine to Strasbourg. We will cross Switzerland, and go down into Italy by the Saint-Gothard. Will that do?> 

 <Yes.> 

 <What are you looking at?> 

 <I am looking at you; indeed you are adorable like that! One would say you were carrying me off.> 

 <And they would be right, \textit{pardieu!}> 

 <Oh, I think you swore, Eugénie.> 

 And the two young girls, whom everyone might have thought plunged in grief, the one on her own account, the other from interest in her friend, burst out laughing, as they cleared away every visible trace of the disorder which had naturally accompanied the preparations for their escape. Then, having blown out the lights, the two fugitives, looking and listening eagerly, with outstretched necks, opened the door of a dressing-room which led by a side staircase down to the yard,—Eugénie going first, and holding with one arm the portmanteau, which by the opposite handle Mademoiselle d'Armilly scarcely raised with both hands. The yard was empty; the clock was striking twelve. The porter was not yet gone to bed. Eugénie approached softly, and saw the old man sleeping soundly in an armchair in his lodge. She returned to Louise, took up the portmanteau, which she had placed for a moment on the ground, and they reached the archway under the shadow of the wall. 

 Eugénie concealed Louise in an angle of the gateway, so that if the porter chanced to awake he might see but one person. Then placing herself in the full light of the lamp which lit the yard: 

 <Gate!> cried she, with her finest contralto voice, and rapping at the window. 

 The porter got up as Eugénie expected, and even advanced some steps to recognize the person who was going out, but seeing a young man striking his boot impatiently with his riding-whip, he opened it immediately. Louise slid through the half-open gate like a snake, and bounded lightly forward. Eugénie, apparently calm, although in all probability her heart beat somewhat faster than usual, went out in her turn. 

 A porter was passing and they gave him the portmanteau; then the two young girls, having told him to take it to № 36, Rue de la Victoire, walked behind this man, whose presence comforted Louise. As for Eugénie, she was as strong as a Judith or a Delilah. They arrived at the appointed spot. Eugénie ordered the porter to put down the portmanteau, gave him some pieces of money, and having rapped at the shutter sent him away. The shutter where Eugénie had rapped was that of a little laundress, who had been previously warned, and was not yet gone to bed. She opened the door. 

 <Mademoiselle,> said Eugénie, <let the porter get the post-chaise from the coach-house, and fetch some post-horses from the hotel. Here are five francs for his trouble.> 

 <Indeed,> said Louise, <I admire you, and I could almost say respect you.> The laundress looked on in astonishment, but as she had been promised twenty louis, she made no remark. 

 In a quarter of an hour the porter returned with a post-boy and horses, which were harnessed, and put in the post-chaise in a minute, while the porter fastened the portmanteau on with the assistance of a cord and strap. 

 <Here is the passport,> said the postilion, <which way are we going, young gentleman?> 

 <To Fontainebleau,> replied Eugénie with an almost masculine voice. 

 <What do you say?> said Louise. 

 <I am giving them the slip,> said Eugénie; <this woman to whom we have given twenty louis may betray us for forty; we will soon alter our direction.> 

 And the young girl jumped into the britzka, which was admirably arranged for sleeping in, without scarcely touching the step. 

 <You are always right,> said the music teacher, seating herself by the side of her friend. 

 A quarter of an hour afterwards the postilion, having been put in the right road, passed with a crack of his whip through the gateway of the Barrière Saint-Martin. 

 <Ah,> said Louise, breathing freely, <here we are out of Paris.> 

 <Yes, my dear, the abduction is an accomplished fact,> replied Eugénie. 

 <Yes, and without violence,> said Louise. 

 <I shall bring that forward as an extenuating circumstance,> replied Eugénie. 

 These words were lost in the noise which the carriage made in rolling over the pavement of La Villette. M. Danglars no longer had a daughter. 