\chapter{The Will} 

 \lettrine{A}{s} soon as Barrois had left the room, Noirtier looked at Valentine with a malicious expression that said many things. The young girl perfectly understood the look, and so did Villefort, for his countenance became clouded, and he knitted his eyebrows angrily. He took a seat, and quietly awaited the arrival of the notary. Noirtier saw him seat himself with an appearance of perfect indifference, at the same time giving a side look at Valentine, which made her understand that she also was to remain in the room. Three-quarters of an hour after, Barrois returned, bringing the notary with him. 

 <Sir,> said Villefort, after the first salutations were over, <you were sent for by M. Noirtier, whom you see here. All his limbs have become completely paralysed, he has lost his voice also, and we ourselves find much trouble in endeavouring to catch some fragments of his meaning.> 

 Noirtier cast an appealing look on Valentine, which look was at once so earnest and imperative, that she answered immediately. 

 <Sir,> said she, <I perfectly understand my grandfather's meaning at all times.> 

 <That is quite true,> said Barrois; <and that is what I told the gentleman as we walked along.> 

 <Permit me,> said the notary, turning first to Villefort and then to Valentine—<permit me to state that the case in question is just one of those in which a public officer like myself cannot proceed to act without thereby incurring a dangerous responsibility. The first thing necessary to render an act valid is, that the notary should be thoroughly convinced that he has faithfully interpreted the will and wishes of the person dictating the act. Now I cannot be sure of the approbation or disapprobation of a client who cannot speak, and as the object of his desire or his repugnance cannot be clearly proved to me, on account of his want of speech, my services here would be quite useless, and cannot be legally exercised.> 

 The notary then prepared to retire. An imperceptible smile of triumph was expressed on the lips of the procureur. Noirtier looked at Valentine with an expression so full of grief, that she arrested the departure of the notary. 

 <Sir,> said she, <the language which I speak with my grandfather may be easily learnt, and I can teach you in a few minutes, to understand it almost as well as I can myself. Will you tell me what you require, in order to set your conscience quite at ease on the subject?> 

 <In order to render an act valid, I must be certain of the approbation or disapprobation of my client. Illness of body would not affect the validity of the deed, but sanity of mind is absolutely requisite.> 

 <Well, sir, by the help of two signs, with which I will acquaint you presently, you may ascertain with perfect certainty that my grandfather is still in the full possession of all his mental faculties. M. Noirtier, being deprived of voice and motion, is accustomed to convey his meaning by closing his eyes when he wishes to signify <yes,> and to wink when he means <no.> You now know quite enough to enable you to converse with M. Noirtier;—try.> 

 Noirtier gave Valentine such a look of tenderness and gratitude that it was comprehended even by the notary himself. 

 <You have heard and understood what your granddaughter has been saying, sir, have you?> asked the notary. Noirtier closed his eyes. 

 <And you approve of what she said—that is to say, you declare that the signs which she mentioned are really those by means of which you are accustomed to convey your thoughts?> 

 <Yes.> 

 <It was you who sent for me?> 

 <Yes.> 

 <To make your will?> 

 <Yes.> 

 <And you do not wish me to go away without fulfilling your original intentions?> The old man winked violently. 

 <Well, sir,> said the young girl, <do you understand now, and is your conscience perfectly at rest on the subject?> 

 But before the notary could answer, Villefort had drawn him aside. 

 <Sir,> said he, <do you suppose for a moment that a man can sustain a physical shock, such as M. Noirtier has received, without any detriment to his mental faculties?> 

 <It is not exactly that, sir,> said the notary, <which makes me uneasy, but the difficulty will be in wording his thoughts and intentions, so as to be able to get his answers.> 

 <You must see that to be an utter impossibility,> said Villefort. Valentine and the old man heard this conversation, and Noirtier fixed his eye so earnestly on Valentine that she felt bound to answer to the look. 

 <Sir,> said she, <that need not make you uneasy, however difficult it may at first sight appear to be. I can discover and explain to you my grandfather's thoughts, so as to put an end to all your doubts and fears on the subject. I have now been six years with M. Noirtier, and let him tell you if ever once, during that time, he has entertained a thought which he was unable to make me understand.> 

 <No,> signed the old man. 

 <Let us try what we can do, then,> said the notary. <You accept this young lady as your interpreter, M. Noirtier?> 

 <Yes.> 

 <Well, sir, what do you require of me, and what document is it that you wish to be drawn up?> 

 Valentine named all the letters of the alphabet until she came to W. At this letter the eloquent eye of Noirtier gave her notice that she was to stop. 

 <It is very evident that it is the letter W which M. Noirtier wants,> said the notary. 

 <Wait,> said Valentine; and, turning to her grandfather, she repeated, <Wa—We—Wi\longdash> The old man stopped her at the last syllable. Valentine then took the dictionary, and the notary watched her while she turned over the pages. 

 She passed her finger slowly down the columns, and when she came to the word <Will,> M. Noirtier's eye bade her stop. 

 <Will,> said the notary; <it is very evident that M. Noirtier is desirous of making his will.> 

 <Yes, yes, yes,> motioned the invalid. 

 <Really, sir, you must allow that this is most extraordinary,> said the astonished notary, turning to M. de Villefort. 

 <Yes,> said the procureur, <and I think the will promises to be yet more extraordinary, for I cannot see how it is to be drawn up without the intervention of Valentine, and she may, perhaps, be considered as too much interested in its contents to allow of her being a suitable interpreter of the obscure and ill-defined wishes of her grandfather.> 

 <No, no, no,> replied the eye of the paralytic. 

 <What?> said Villefort, <do you mean to say that Valentine is not interested in your will?> 

 <No.> 

 <Sir,> said the notary, whose interest had been greatly excited, and who had resolved on publishing far and wide the account of this extraordinary and picturesque scene, <what appeared so impossible to me an hour ago, has now become quite easy and practicable, and this may be a perfectly valid will, provided it be read in the presence of seven witnesses, approved by the testator, and sealed by the notary in the presence of the witnesses. As to the time, it will not require very much more than the generality of wills. There are certain forms necessary to be gone through, and which are always the same. As to the details, the greater part will be furnished afterwards by the state in which we find the affairs of the testator, and by yourself, who, having had the management of them, can doubtless give full information on the subject. But besides all this, in order that the instrument may not be contested, I am anxious to give it the greatest possible authenticity, therefore, one of my colleagues will help me, and, contrary to custom, will assist in the dictation of the testament. Are you satisfied, sir?> continued the notary, addressing the old man.  <Yes,> looked the invalid, his eye beaming with delight at the ready interpretation of his meaning. 

 <What is he going to do?> thought Villefort, whose position demanded much reserve, but who was longing to know what his father's intentions were. He left the room to give orders for another notary to be sent, but Barrois, who had heard all that passed, had guessed his master's wishes, and had already gone to fetch one. The procureur then told his wife to come up. In the course of a quarter of an hour everyone had assembled in the chamber of the paralytic; the second notary had also arrived. 

 A few words sufficed for a mutual understanding between the two officers of the law. They read to Noirtier the formal copy of a will, in order to give him an idea of the terms in which such documents are generally couched; then, in order to test the capacity of the testator, the first notary said, turning towards him: 

 <When an individual makes his will, it is generally in favour or in prejudice of some person.> 

 <Yes.> 

 <Have you an exact idea of the amount of your fortune?> 

 <Yes.> 

 <I will name to you several sums which will increase by gradation; you will stop me when I reach the one representing the amount of your own possessions?> 

 <Yes.> 

 There was a kind of solemnity in this interrogation. Never had the struggle between mind and matter been more apparent than now, and if it was not a sublime, it was, at least, a curious spectacle. They had formed a circle round the invalid; the second notary was sitting at a table, prepared for writing, and his colleague was standing before the testator in the act of interrogating him on the subject to which we have alluded. 

 <Your fortune exceeds 300,000 francs, does it not?> asked he. Noirtier made a sign that it did. 

 <Do you possess 400,000 francs?> inquired the notary. Noirtier's eye remained immovable. 

 <500,000?> The same expression continued. 

 <600,000—700,000—800,000—900,000?> 

 Noirtier stopped him at the last-named sum. 

 <You are then in possession of 900,000 francs?> asked the notary. 

 <Yes.> 

 <In landed property?> 

 <No.> 

 <In stock?> 

 <Yes.> 

 <The stock is in your own hands?> 

 The look which M. Noirtier cast on Barrois showed that there was something wanting which he knew where to find. The old servant left the room, and presently returned, bringing with him a small casket. 

 <Do you permit us to open this casket?> asked the notary. Noirtier gave his assent. 

 They opened it, and found 900,000 francs in bank scrip. The first notary handed over each note, as he examined it, to his colleague. 

 The total amount was found to be as M. Noirtier had stated. 

 <It is all as he has said; it is very evident that the mind still retains its full force and vigour.> Then, turning towards the paralytic, he said, <You possess, then, 900,000 francs of capital, which, according to the manner in which you have invested it, ought to bring in an income of about 40,000 livres?> 

 <Yes.> 

 <To whom do you desire to leave this fortune?> 

 <Oh!> said Madame de Villefort, <there is not much doubt on that subject. M. Noirtier tenderly loves his granddaughter, Mademoiselle de Villefort; it is she who has nursed and tended him for six years, and has, by her devoted attention, fully secured the affection, I had almost said the gratitude, of her grandfather, and it is but just that she should reap the fruit of her devotion.> 

 The eye of Noirtier clearly showed by its expression that he was not deceived by the false assent given by Madame de Villefort's words and manner to the motives which she supposed him to entertain. 

 <Is it, then, to Mademoiselle Valentine de Villefort that you leave these 900,000 francs?> demanded the notary, thinking he had only to insert this clause, but waiting first for the assent of Noirtier, which it was necessary should be given before all the witnesses of this singular scene. 

 Valentine, when her name was made the subject of discussion, had stepped back, to escape unpleasant observation; her eyes were cast down, and she was crying. The old man looked at her for an instant with an expression of the deepest tenderness, then, turning towards the notary, he significantly winked his eye in token of dissent. 

 <What,> said the notary, <do you not intend making Mademoiselle Valentine de Villefort your residuary legatee?> 

 <No.> 

 <You are not making any mistake, are you?> said the notary; <you really mean to declare that such is not your intention?> 

 <No,> repeated Noirtier; <No.> 

 Valentine raised her head, struck dumb with astonishment. It was not so much the conviction that she was disinherited that caused her grief, but her total inability to account for the feelings which had provoked her grandfather to such an act. But Noirtier looked at her with so much affectionate tenderness that she exclaimed: 

 <Oh, grandpapa, I see now that it is only your fortune of which you deprive me; you still leave me the love which I have always enjoyed.> 

 <Ah, yes, most assuredly,> said the eyes of the paralytic, for he closed them with an expression which Valentine could not mistake. 

 <Thank you, thank you,> murmured she. The old man's declaration that Valentine was not the destined inheritor of his fortune had excited the hopes of Madame de Villefort; she gradually approached the invalid, and said: 

 <Then, doubtless, dear M. Noirtier, you intend leaving your fortune to your grandson, Edward de Villefort?> 

 The winking of the eyes which answered this speech was most decided and terrible, and expressed a feeling almost amounting to hatred. 

 <No?> said the notary; <then, perhaps, it is to your son, M. de Villefort?> 

 <No.> The two notaries looked at each other in mute astonishment and inquiry as to what were the real intentions of the testator. Villefort and his wife both grew red, one from shame, the other from anger. 

 <What have we all done, then, dear grandpapa?> said Valentine; <you no longer seem to love any of us?> 

 The old man's eyes passed rapidly from Villefort and his wife, and rested on Valentine with a look of unutterable fondness. 

 <Well,> said she; <if you love me, grandpapa, try and bring that love to bear upon your actions at this present moment. You know me well enough to be quite sure that I have never thought of your fortune; besides, they say I am already rich in right of my mother—too rich, even. Explain yourself, then.> 

 Noirtier fixed his intelligent eyes on Valentine's hand. 

 <My hand?> said she. 

 <Yes.> 

 <Her hand!> exclaimed everyone. 

 <Oh, gentlemen, you see it is all useless, and that my father's mind is really impaired,> said Villefort. 

 <Ah,> cried Valentine suddenly, <I understand. It is my marriage you mean, is it not, dear grandpapa?> 

 <Yes, yes, yes,> signed the paralytic, casting on Valentine a look of joyful gratitude for having guessed his meaning.  <You are angry with us all on account of this marriage, are you not?> 

 <Yes?> 

 <Really, this is too absurd,> said Villefort. 

 <Excuse me, sir,> replied the notary; <on the contrary, the meaning of M. Noirtier is quite evident to me, and I can quite easily connect the train of ideas passing in his mind.> 

 <You do not wish me to marry M. Franz d'Épinay?> observed Valentine. 

 <I do not wish it,> said the eye of her grandfather. 

 <And you disinherit your granddaughter,> continued the notary, <because she has contracted an engagement contrary to your wishes?> 

 <Yes.> 

 <So that, but for this marriage, she would have been your heir?> 

 <Yes.> 

 There was a profound silence. The two notaries were holding a consultation as to the best means of proceeding with the affair. Valentine was looking at her grandfather with a smile of intense gratitude, and Villefort was biting his lips with vexation, while Madame de Villefort could not succeed in repressing an inward feeling of joy, which, in spite of herself, appeared in her whole countenance. 

 <But,> said Villefort, who was the first to break the silence, <I consider that I am the best judge of the propriety of the marriage in question. I am the only person possessing the right to dispose of my daughter's hand. It is my wish that she should marry M. Franz d'Épinay—and she shall marry him.> 

 Valentine sank weeping into a chair. 

 <Sir,> said the notary, <how do you intend disposing of your fortune in case Mademoiselle de Villefort still determines on marrying M. Franz?> The old man gave no answer. 

 <You will, of course, dispose of it in some way or other?> 

 <Yes.> 

 <In favour of some member of your family?> 

 <No.> 

 <Do you intend devoting it to charitable purposes, then?> pursued the notary. 

 <Yes.> 

 <But,> said the notary, <you are aware that the law does not allow a son to be entirely deprived of his patrimony?> 

 <Yes.> 

 <You only intend, then, to dispose of that part of your fortune which the law allows you to subtract from the inheritance of your son?> Noirtier made no answer. 

 <Do you still wish to dispose of all?> 

 <Yes.> 

 <But they will contest the will after your death?> 

 <No.> 

 <My father knows me,> replied Villefort; <he is quite sure that his wishes will be held sacred by me; besides, he understands that in my position I cannot plead against the poor.> The eye of Noirtier beamed with triumph. 

 <What do you decide on, sir?> asked the notary of Villefort. 

 <Nothing, sir; it is a resolution which my father has taken and I know he never alters his mind. I am quite resigned. These 900,000 francs will go out of the family in order to enrich some hospital; but it is ridiculous thus to yield to the caprices of an old man, and I shall, therefore, act according to my conscience.> 

 Having said this, Villefort quitted the room with his wife, leaving his father at liberty to do as he pleased. The same day the will was made, the witnesses were brought, it was approved by the old man, sealed in the presence of all and given in charge to M. Deschamps, the family notary. 