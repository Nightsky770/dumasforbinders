\chapter{The Law} 

 \lettrine{W}{e} have seen how quietly Mademoiselle Danglars and Mademoiselle d'Armilly accomplished their transformation and flight; the fact being that everyone was too much occupied in his or her own affairs to think of theirs. 

 We will leave the banker contemplating the enormous magnitude of his debt before the phantom of bankruptcy, and follow the baroness, who after being momentarily crushed under the weight of the blow which had struck her, had gone to seek her usual adviser, Lucien Debray. The baroness had looked forward to this marriage as a means of ridding her of a guardianship which, over a girl of Eugénie's character, could not fail to be rather a troublesome undertaking; for in the tacit relations which maintain the bond of family union, the mother, to maintain her ascendancy over her daughter, must never fail to be a model of wisdom and a type of perfection. 

 Now, Madame Danglars feared Eugénie's sagacity and the influence of Mademoiselle d'Armilly; she had frequently observed the contemptuous expression with which her daughter looked upon Debray,—an expression which seemed to imply that she understood all her mother's amorous and pecuniary relationships with the intimate secretary; moreover, she saw that Eugénie detested Debray, not only because he was a source of dissension and scandal under the paternal roof, but because she had at once classed him in that catalogue of bipeds whom Plato endeavors to withdraw from the appellation of men, and whom Diogenes designated as animals upon two legs without feathers. 

 Unfortunately, in this world of ours, each person views things through a certain medium, and so is prevented from seeing in the same light as others, and Madame Danglars, therefore, very much regretted that the marriage of Eugénie had not taken place, not only because the match was good, and likely to insure the happiness of her child, but because it would also set her at liberty. She ran therefore to Debray, who, after having, like the rest of Paris, witnessed the contract scene and the scandal attending it, had retired in haste to his club, where he was chatting with some friends upon the events which served as a subject of conversation for three-fourths of that city known as the capital of the world. 

 At the precise time when Madame Danglars, dressed in black and concealed in a long veil, was ascending the stairs leading to Debray's apartments, notwithstanding the assurances of the concierge that the young man was not at home, Debray was occupied in repelling the insinuations of a friend, who tried to persuade him that after the terrible scene which had just taken place he ought, as a friend of the family, to marry Mademoiselle Danglars and her two millions. Debray did not defend himself very warmly, for the idea had sometimes crossed his mind; still, when he recollected the independent, proud spirit of Eugénie, he positively rejected it as utterly impossible, though the same thought again continually recurred and found a resting-place in his heart. Tea, play, and the conversation, which had become interesting during the discussion of such serious affairs, lasted till one o'clock in the morning. 

 Meanwhile Madame Danglars, veiled and uneasy, awaited the return of Debray in the little green room, seated between two baskets of flowers, which she had that morning sent, and which, it must be confessed, Debray had himself arranged and watered with so much care that his absence was half excused in the eyes of the poor woman. 

 At twenty minutes to twelve, Madame Danglars, tired of waiting, returned home. Women of a certain grade are like prosperous grisettes in one respect, they seldom return home after twelve o'clock. The baroness returned to the hotel with as much caution as Eugénie used in leaving it; she ran lightly upstairs, and with an aching heart entered her apartment, contiguous, as we know, to that of Eugénie. She was fearful of exciting any remark, and believed firmly in her daughter's innocence and fidelity to the paternal roof. She listened at Eugénie's door, and hearing no sound tried to enter, but the bolts were in place. Madame Danglars then concluded that the young girl had been overcome with the terrible excitement of the evening, and had gone to bed and to sleep. She called the maid and questioned her. 

 <Mademoiselle Eugénie,> said the maid, <retired to her apartment with Mademoiselle d'Armilly; they then took tea together, after which they desired me to leave, saying that they needed me no longer.> 

 Since then the maid had been below, and like everyone else she thought the young ladies were in their own room; Madame Danglars, therefore, went to bed without a shadow of suspicion, and began to muse over the recent events. In proportion as her memory became clearer, the occurrences of the evening were revealed in their true light; what she had taken for confusion was a tumult; what she had regarded as something distressing, was in reality a disgrace. And then the baroness remembered that she had felt no pity for poor Mercédès, who had been afflicted with as severe a blow through her husband and son. 

 <Eugénie,> she said to herself, <is lost, and so are we. The affair, as it will be reported, will cover us with shame; for in a society such as ours satire inflicts a painful and incurable wound. How fortunate that Eugénie is possessed of that strange character which has so often made me tremble!> 

 And her glance was turned towards heaven, where a mysterious Providence disposes all things, and out of a fault, nay, even a vice, sometimes produces a blessing. And then her thoughts, cleaving through space like a bird in the air, rested on Cavalcanti. This Andrea was a wretch, a robber, an assassin, and yet his manners showed the effects of a sort of education, if not a complete one; he had been presented to the world with the appearance of an immense fortune, supported by an honourable name. How could she extricate herself from this labyrinth? To whom would she apply to help her out of this painful situation? Debray, to whom she had run, with the first instinct of a woman towards the man she loves, and who yet betrays her,—Debray could but give her advice, she must apply to someone more powerful than he. 

 The baroness then thought of M. de Villefort. It was M. de Villefort who had remorselessly brought misfortune into her family, as though they had been strangers. But, no; on reflection, the procureur was not a merciless man; and it was not the magistrate, slave to his duties, but the friend, the loyal friend, who roughly but firmly cut into the very core of the corruption; it was not the executioner, but the surgeon, who wished to withdraw the honour of Danglars from ignominious association with the disgraced young man they had presented to the world as their son-in-law. And since Villefort, the friend of Danglars, had acted in this way, no one could suppose that he had been previously acquainted with, or had lent himself to, any of Andrea's intrigues. Villefort's conduct, therefore, upon reflection, appeared to the baroness as if shaped for their mutual advantage. But the inflexibility of the procureur should stop there; she would see him the next day, and if she could not make him fail in his duties as a magistrate, she would, at least, obtain all the indulgence he could allow. She would invoke the past, recall old recollections; she would supplicate him by the remembrance of guilty, yet happy days. M. de Villefort would stifle the affair; he had only to turn his eyes on one side, and allow Andrea to fly, and follow up the crime under that shadow of guilt called contempt of court. And after this reasoning she slept easily. 

 At nine o'clock next morning she arose, and without ringing for her maid or giving the least sign of her activity, she dressed herself in the same simple style as on the previous night; then running downstairs, she left the hotel, walked to the Rue de Provence, called a cab, and drove to M. de Villefort's house. 

 For the last month this wretched house had presented the gloomy appearance of a lazaretto infected with the plague. Some of the apartments were closed within and without; the shutters were only opened to admit a minute's air, showing the scared face of a footman, and immediately afterwards the window would be closed, like a gravestone falling on a sepulchre, and the neighbours would say to each other in a low voice, <Will there be another funeral today at the procureur's house?> 

 Madame Danglars involuntarily shuddered at the desolate aspect of the mansion; descending from the cab, she approached the door with trembling knees, and rang the bell. Three times did the bell ring with a dull, heavy sound, seeming to participate, in the general sadness, before the concierge appeared and peeped through the door, which he opened just wide enough to allow his words to be heard. He saw a lady, a fashionable, elegantly dressed lady, and yet the door remained almost closed. 

 <Do you intend opening the door?> said the baroness. 

 <First, madame, who are you?> 

 <Who am I? You know me well enough.> 

 <We no longer know anyone, madame.> 

 <You must be mad, my friend,> said the baroness. 

 <Where do you come from?> 

 <Oh, this is too much!> 

 <Madame, these are my orders; excuse me. Your name?> 

 <The baroness Danglars; you have seen me twenty times.> 

 <Possibly, madame. And now, what do you want?> 

 <Oh, how extraordinary! I shall complain to M. de Villefort of the impertinence of his servants.> 

 <Madame, this is precaution, not impertinence; no one enters here without an order from M. d'Avrigny, or without speaking to the procureur.> 

 <Well, I have business with the procureur.> 

 <Is it pressing business?> 

 <You can imagine so, since I have not even brought my carriage out yet. But enough of this—here is my card, take it to your master.> 

 <Madame will await my return?> 

 <Yes; go.> 

 The concierge closed the door, leaving Madame Danglars in the street. She had not long to wait; directly afterwards the door was opened wide enough to admit her, and when she had passed through, it was again shut. Without losing sight of her for an instant, the concierge took a whistle from his pocket as soon as they entered the court, and blew it. The valet de chambre appeared on the door-steps. 

 <You will excuse this poor fellow, madame,> he said, as he preceded the baroness, <but his orders are precise, and M. de Villefort begged me to tell you that he could not act otherwise.> 

 In the court showing his merchandise, was a tradesman who had been admitted with the same precautions. The baroness ascended the steps; she felt herself strongly infected with the sadness which seemed to magnify her own, and still guided by the valet de chambre, who never lost sight of her for an instant, she was introduced to the magistrate's study. 

 Preoccupied as Madame Danglars had been with the object of her visit, the treatment she had received from these underlings appeared to her so insulting, that she began by complaining of it. But Villefort, raising his head, bowed down by grief, looked up at her with so sad a smile that her complaints died upon her lips. 

 <Forgive my servants,> he said, <for a terror I cannot blame them for; from being suspected they have become suspicious.> 

 Madame Danglars had often heard of the terror to which the magistrate alluded, but without the evidence of her own eyesight she could never have believed that the sentiment had been carried so far. 

 <You too, then, are unhappy?> she said. 

 <Yes, madame,> replied the magistrate. 

 <Then you pity me!> 

 <Sincerely, madame.> 

 <And you understand what brings me here?> 

 <You wish to speak to me about the circumstance which has just happened?> 

 <Yes, sir,—a fearful misfortune.> 

 <You mean a mischance.> 

 <A mischance?> repeated the baroness. 

 <Alas, madame,> said the procureur with his imperturbable calmness of manner, <I consider those alone misfortunes which are irreparable.> 

 <And do you suppose this will be forgotten?> 

 <Everything will be forgotten, madame,> said Villefort. <Your daughter will be married tomorrow, if not today—in a week, if not tomorrow; and I do not think you can regret the intended husband of your daughter.> 

 Madame Danglars gazed on Villefort, stupefied to find him so almost insultingly calm. <Am I come to a friend?> she asked in a tone full of mournful dignity. 

 <You know that you are, madame,> said Villefort, whose pale cheeks became slightly flushed as he gave her the assurance. And truly this assurance carried him back to different events from those now occupying the baroness and him. 

 <Well, then, be more affectionate, my dear Villefort,> said the baroness. <Speak to me not as a magistrate, but as a friend; and when I am in bitter anguish of spirit, do not tell me that I ought to be gay.> Villefort bowed. 

 <When I hear misfortunes named, madame,> he said, <I have within the last few months contracted the bad habit of thinking of my own, and then I cannot help drawing up an egotistical parallel in my mind. That is the reason that by the side of my misfortunes yours appear to me mere mischances; that is why my dreadful position makes yours appear enviable. But this annoys you; let us change the subject. You were saying, madame\longdash> 

 <I came to ask you, my friend,> said the baroness, <what will be done with this impostor?> 

 <Impostor,> repeated Villefort; <certainly, madame, you appear to extenuate some cases, and exaggerate others. Impostor, indeed!—M. Andrea Cavalcanti, or rather M. Benedetto, is nothing more nor less than an assassin!> 

 <Sir, I do not deny the justice of your correction, but the more severely you arm yourself against that unfortunate man, the more deeply will you strike our family. Come, forget him for a moment, and instead of pursuing him, let him go.> 

 <You are too late, madame; the orders are issued.> 

 <Well, should he be arrested—do they think they will arrest him?> 

 <I hope so.> 

 <If they should arrest him (I know that sometimes prisons afford means of escape), will you leave him in prison?> 

 The procureur shook his head. 

 <At least keep him there till my daughter be married.> 

 <Impossible, madame; justice has its formalities.> 

 <What, even for me?> said the baroness, half jesting, half in earnest. 

 <For all, even for myself among the rest,> replied Villefort.  <Ah!> exclaimed the baroness, without expressing the ideas which the exclamation betrayed. Villefort looked at her with that piercing glance which reads the secrets of the heart. 

 <Yes, I know what you mean,> he said; <you refer to the terrible rumours spread abroad in the world, that the deaths which have kept me in mourning for the last three months, and from which Valentine has only escaped by a miracle, have not happened by natural means.> 

 <I was not thinking of that,> replied Madame Danglars quickly. 

 <Yes, you were thinking of it, and with justice. You could not help thinking of it, and saying to yourself, <you, who pursue crime so vindictively, answer now, why are there unpunished crimes in your dwelling?>> The baroness became pale. <You were saying this, were you not?> 

 <Well, I own it.> 

 <I will answer you.> 

 Villefort drew his armchair nearer to Madame Danglars; then resting both hands upon his desk he said in a voice more hollow than usual: 

 <There are crimes which remain unpunished because the criminals are unknown, and we might strike the innocent instead of the guilty; but when the culprits are discovered> (Villefort here extended his hand toward a large crucifix placed opposite to his desk)—<when they are discovered, I swear to you, by all I hold most sacred, that whoever they may be they shall die. Now, after the oath I have just taken, and which I will keep, madame, dare you ask for mercy for that wretch!> 

 <But, sir, are you sure he is as guilty as they say?> 

 <Listen; this is his description: <Benedetto, condemned, at the age of sixteen, for five years to the galleys for forgery.> He promised well, as you see—first a runaway, then an assassin.> 

 <And who is this wretch?> 

 <Who can tell?—a vagabond, a Corsican.> 

 <Has no one owned him?> 

 <No one; his parents are unknown.> 

 <But who was the man who brought him from Lucca?> 

 <Another rascal like himself, perhaps his accomplice.> The baroness clasped her hands. 

 <Villefort,> she exclaimed in her softest and most captivating manner. 

 <For Heaven's sake, madame,> said Villefort, with a firmness of expression not altogether free from harshness—“for Heaven's sake, do not ask pardon of me for a guilty wretch! What am I?—the law. Has the law any eyes to witness your grief? Has the law ears to be melted by your sweet voice? Has the law a memory for all those soft recollections you endeavour to recall? No, madame; the law has commanded, and when it commands it strikes. You will tell me that I am a living being, and not a code—a man, and not a volume. Look at me, madame—look around me. Has mankind treated me as a brother? Have men loved me? Have they spared me? Has anyone shown the mercy towards me that you now ask at my hands? No, madame, they struck me, always struck me!  <Woman, siren that you are, do you persist in fixing on me that fascinating eye, which reminds me that I ought to blush? Well, be it so; let me blush for the faults you know, and perhaps—perhaps for even more than those! But having sinned myself,—it may be more deeply than others,—I never rest till I have torn the disguises from my fellow-creatures, and found out their weaknesses. I have always found them; and more,—I repeat it with joy, with triumph,—I have always found some proof of human perversity or error. Every criminal I condemn seems to me living evidence that I am not a hideous exception to the rest. Alas, alas, alas; all the world is wicked; let us therefore strike at wickedness!> 

 Villefort pronounced these last words with a feverish rage, which gave a ferocious eloquence to his words. 

 <But>' said Madame Danglars, resolving to make a last effort, <this young man, though a murderer, is an orphan, abandoned by everybody.> 

 <So much the worse, or rather, so much the better; it has been so ordained that he may have none to weep his fate.> 

 <But this is trampling on the weak, sir.> 

 <The weakness of a murderer!> 

 <His dishonor reflects upon us.> 

 <Is not death in my house?> 

 <Oh, sir,> exclaimed the baroness, <you are without pity for others, well, then, I tell you they will have no mercy on you!> 

 <Be it so!> said Villefort, raising his arms to heaven with a threatening gesture. 

 <At least, delay the trial till the next assizes; we shall then have six months before us.> 

 <No, madame,> said Villefort; <instructions have been given. There are yet five days left; five days are more than I require. Do you not think that I also long for forgetfulness? While working night and day, I sometimes lose all recollection of the past, and then I experience the same sort of happiness I can imagine the dead feel; still, it is better than suffering.> 

 <But, sir, he has fled; let him escape—inaction is a pardonable offence.> 

 <I tell you it is too late; early this morning the telegraph was employed, and at this very minute\longdash> 

 <Sir,> said the valet de chambre, entering the room, <a dragoon has brought this despatch from the Minister of the Interior.> 

 Villefort seized the letter, and hastily broke the seal. Madame Danglars trembled with fear; Villefort started with joy. 

 <Arrested!> he exclaimed; <he was taken at Compiègne, and all is over.> 

 Madame Danglars rose from her seat, pale and cold. 

 <Adieu, sir,> she said. 

 <Adieu, madame,> replied the king's attorney, as in an almost joyful manner he conducted her to the door. Then, turning to his desk, he said, striking the letter with the back of his right hand: 

 <Come, I had a forgery, three robberies, and two cases of arson, I only wanted a murder, and here it is. It will be a splendid session!> 