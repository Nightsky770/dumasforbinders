\chapter{Progress of Cavalcanti the Younger} 

 \lettrine{M}{eanwhile} M. Cavalcanti the elder had returned to his service, not in the army of his majesty the Emperor of Austria, but at the gaming-table of the baths of Lucca, of which he was one of the most assiduous courtiers. He had spent every farthing that had been allowed for his journey as a reward for the majestic and solemn manner in which he had maintained his assumed character of father. 

 M. Andrea at his departure inherited all the papers which proved that he had indeed the honour of being the son of the Marquis Bartolomeo and the Marchioness Oliva Corsinari. He was now fairly launched in that Parisian society which gives such ready access to foreigners, and treats them, not as they really are, but as they wish to be considered. Besides, what is required of a young man in Paris? To speak its language tolerably, to make a good appearance, to be a good gamester, and to pay in cash. They are certainly less particular with a foreigner than with a Frenchman. Andrea had, then, in a fortnight, attained a very fair position. He was called count, he was said to possess 50,000 livres per annum; and his father's immense riches, buried in the quarries of Saravezza, were a constant theme. A learned man, before whom the last circumstance was mentioned as a fact, declared he had seen the quarries in question, which gave great weight to assertions hitherto somewhat doubtful, but which now assumed the garb of reality. 

 Such was the state of society in Paris at the period we bring before our readers, when Monte Cristo went one evening to pay M. Danglars a visit. M. Danglars was out, but the count was asked to go and see the baroness, and he accepted the invitation. It was never without a nervous shudder, since the dinner at Auteuil, and the events which followed it, that Madame Danglars heard Monte Cristo's name announced. If he did not come, the painful sensation became most intense; if, on the contrary, he appeared, his noble countenance, his brilliant eyes, his amiability, his polite attention even towards Madame Danglars, soon dispelled every impression of fear. It appeared impossible to the baroness that a man of such delightfully pleasing manners should entertain evil designs against her; besides, the most corrupt minds only suspect evil when it would answer some interested end—useless injury is repugnant to every mind.  When Monte Cristo entered the boudoir, to which we have already once introduced our readers, and where the baroness was examining some drawings, which her daughter passed to her after having looked at them with M. Cavalcanti, his presence soon produced its usual effect, and it was with smiles that the baroness received the count, although she had been a little disconcerted at the announcement of his name. The latter took in the whole scene at a glance. 

 The baroness was partially reclining on a sofa, Eugénie sat near her, and Cavalcanti was standing. Cavalcanti, dressed in black, like one of Goethe's heroes, with varnished shoes and white silk open-worked stockings, passed a white and tolerably nice-looking hand through his light hair, and so displayed a sparkling diamond, that in spite of Monte Cristo's advice the vain young man had been unable to resist putting on his little finger. This movement was accompanied by killing glances at Mademoiselle Danglars, and by sighs launched in the same direction. 

 Mademoiselle Danglars was still the same—cold, beautiful, and satirical. Not one of these glances, nor one sigh, was lost on her; they might have been said to fall on the shield of Minerva, which some philosophers assert protected sometimes the breast of Sappho. Eugénie bowed coldly to the count, and availed herself of the first moment when the conversation became earnest to escape to her study, whence very soon two cheerful and noisy voices being heard in connection with occasional notes of the piano assured Monte Cristo that Mademoiselle Danglars preferred to his society and to that of M. Cavalcanti the company of Mademoiselle Louise d'Armilly, her singing teacher. 

 It was then, especially while conversing with Madame Danglars, and apparently absorbed by the charm of the conversation, that the count noticed M. Andrea Cavalcanti's solicitude, his manner of listening to the music at the door he dared not pass, and of manifesting his admiration. 

 The banker soon returned. His first look was certainly directed towards Monte Cristo, but the second was for Andrea. As for his wife, he bowed to her, as some husbands do to their wives, but in a way that bachelors will never comprehend, until a very extensive code is published on conjugal life. 

 <Have not the ladies invited you to join them at the piano?> said Danglars to Andrea. 

 <Alas, no, sir,> replied Andrea with a sigh, still more remarkable than the former ones. Danglars immediately advanced towards the door and opened it.  The two young ladies were seen seated on the same chair, at the piano, accompanying themselves, each with one hand, a fancy to which they had accustomed themselves, and performed admirably. Mademoiselle d'Armilly, whom they then perceived through the open doorway, formed with Eugénie one of the \textit{tableaux vivants} of which the Germans are so fond. She was somewhat beautiful, and exquisitely formed—a little fairy-like figure, with large curls falling on her neck, which was rather too long, as Perugino sometimes makes his Virgins, and her eyes dull from fatigue. She was said to have a weak chest, and like Antonia in the \textit{Cremona Violin}, she would die one day while singing. 

 Monte Cristo cast one rapid and curious glance round this sanctum; it was the first time he had ever seen Mademoiselle d'Armilly, of whom he had heard much. 

 <Well,> said the banker to his daughter, <are we then all to be excluded?> 

 He then led the young man into the study, and either by chance or manœuvre the door was partially closed after Andrea, so that from the place where they sat neither the Count nor the baroness could see anything; but as the banker had accompanied Andrea, Madame Danglars appeared to take no notice of it. 

 The count soon heard Andrea's voice, singing a Corsican song, accompanied by the piano. While the count smiled at hearing this song, which made him lose sight of Andrea in the recollection of Benedetto, Madame Danglars was boasting to Monte Cristo of her husband's strength of mind, who that very morning had lost three or four hundred thousand francs by a failure at Milan. The praise was well deserved, for had not the count heard it from the baroness, or by one of those means by which he knew everything, the baron's countenance would not have led him to suspect it. 

 <Hem,> thought Monte Cristo, <he begins to conceal his losses; a month since he boasted of them.> 

 Then aloud,—<Oh, madame, M. Danglars is so skilful, he will soon regain at the Bourse what he loses elsewhere.> 

 <I see that you participate in a prevalent error,> said Madame Danglars. 

 <What is it?> said Monte Cristo. 

 <That M. Danglars speculates, whereas he never does.> 

 <Truly, madame, I recollect M. Debray told me—apropos, what has become of him? I have seen nothing of him the last three or four days.> 

 <Nor I,> said Madame Danglars; <but you began a sentence, sir, and did not finish.> 

 <Which?> 

 <M. Debray had told you\longdash> 

 <Ah, yes; he told me it was you who sacrificed to the demon of speculation.> 

 <I was once very fond of it, but I do not indulge now.> 

 <Then you are wrong, madame. Fortune is precarious; and if I were a woman and fate had made me a banker's wife, whatever might be my confidence in my husband's good fortune, still in speculation you know there is great risk. Well, I would secure for myself a fortune independent of him, even if I acquired it by placing my interests in hands unknown to him.> Madame Danglars blushed, in spite of all her efforts. 

 <Stay,> said Monte Cristo, as though he had not observed her confusion, <I have heard of a lucky hit that was made yesterday on the Neapolitan bonds.> 

 <I have none—nor have I ever possessed any; but really we have talked long enough of money, count, we are like two stockbrokers; have you heard how fate is persecuting the poor Villeforts?> 

 <What has happened?> said the count, simulating total ignorance. 

 <You know the Marquis of Saint-Méran died a few days after he had set out on his journey to Paris, and the marchioness a few days after her arrival?> 

 <Yes,> said Monte Cristo, <I have heard that; but, as Claudius said to Hamlet, <it is a law of nature; their fathers died before them, and they mourned their loss; they will die before their children, who will, in their turn, grieve for them.>> 

 <But that is not all.> 

 <Not all!> 

 <No; they were going to marry their daughter\longdash> 

 <To M. Franz d'Épinay. Is it broken off?> 

 <Yesterday morning, it appears, Franz declined the honour.> 

 <Indeed? And is the reason known?> 

 <No.> 

 <How extraordinary! And how does M. de Villefort bear it?> 

 <As usual. Like a philosopher.> 

 Danglars returned at this moment alone. 

 <Well,> said the baroness, <do you leave M. Cavalcanti with your daughter?> 

 <And Mademoiselle d'Armilly,> said the banker; <do you consider her no one?> Then, turning to Monte Cristo, he said, <Prince Cavalcanti is a charming young man, is he not? But is he really a prince?> 

 <I will not answer for it,> said Monte Cristo. <His father was introduced to me as a marquis, so he ought to be a count; but I do not think he has much claim to that title.> 

 <Why?> said the banker. <If he is a prince, he is wrong not to maintain his rank; I do not like anyone to deny his origin.> 

 <Oh, you are a thorough democrat,> said Monte Cristo, smiling. 

 <But do you see to what you are exposing yourself?> said the baroness. <If, perchance, M. de Morcerf came, he would find M. Cavalcanti in that room, where he, the betrothed of Eugénie, has never been admitted.> 

 <You may well say, perchance,> replied the banker; <for he comes so seldom, it would seem only chance that brings him.> 

 <But should he come and find that young man with your daughter, he might be displeased.> 

 <He? You are mistaken. M. Albert would not do us the honour to be jealous; he does not like Eugénie sufficiently. Besides, I care not for his displeasure.> 

 <Still, situated as we are\longdash> 

 <Yes, do you know how we are situated? At his mother's ball he danced once with Eugénie, and M. Cavalcanti three times, and he took no notice of it.> 

 The valet announced the Vicomte Albert de Morcerf. The baroness rose hastily, and was going into the study, when Danglars stopped her. 

 <Let her alone,> said he. 

 She looked at him in amazement. Monte Cristo appeared to be unconscious of what passed. Albert entered, looking very handsome and in high spirits. He bowed politely to the baroness, familiarly to Danglars, and affectionately to Monte Cristo. Then turning to the baroness: <May I ask how Mademoiselle Danglars is?> said he. 

 <She is quite well,> replied Danglars quickly; <she is at the piano with M. Cavalcanti.> 

 Albert retained his calm and indifferent manner; he might feel perhaps annoyed, but he knew Monte Cristo's eye was on him. <M. Cavalcanti has a fine tenor voice,> said he, <and Mademoiselle Eugénie a splendid soprano, and then she plays the piano like Thalberg. The concert must be a delightful one.> 

 <They suit each other remarkably well,> said Danglars. Albert appeared not to notice this remark, which was, however, so rude that Madame Danglars blushed. 

 <I, too,> said the young man, <am a musician—at least, my masters used to tell me so; but it is strange that my voice never would suit any other, and a soprano less than any.> 

 Danglars smiled, and seemed to say, <It is of no consequence.> Then, hoping doubtless to effect his purpose, he said,—<The prince and my daughter were universally admired yesterday. You were not of the party, M. de Morcerf?> 

 <What prince?> asked Albert. 

 <Prince Cavalcanti,> said Danglars, who persisted in giving the young man that title. 

 <Pardon me,> said Albert, <I was not aware that he was a prince. And Prince Cavalcanti sang with Mademoiselle Eugénie yesterday? It must have been charming, indeed. I regret not having heard them. But I was unable to accept your invitation, having promised to accompany my mother to a German concert given by the Baroness of Château-Renaud.> 

 This was followed by rather an awkward silence. 

 <May I also be allowed,> said Morcerf, <to pay my respects to Mademoiselle Danglars?> 

 <Wait a moment,> said the banker, stopping the young man; <do you hear that delightful cavatina? Ta, ta, ta, ti, ta, ti, ta, ta; it is charming, let them finish—one moment. Bravo, bravi, brava!> The banker was enthusiastic in his applause.  <Indeed,> said Albert, <it is exquisite; it is impossible to understand the music of his country better than Prince Cavalcanti does. You said prince, did you not? But he can easily become one, if he is not already; it is no uncommon thing in Italy. But to return to the charming musicians—you should give us a treat, Danglars, without telling them there is a stranger. Ask them to sing one more song; it is so delightful to hear music in the distance, when the musicians are unrestrained by observation.> 

 Danglars was quite annoyed by the young man's indifference. He took Monte Cristo aside. 

 <What do you think of our lover?> said he. 

 <He appears cool. But, then your word is given.> 

 <Yes, doubtless I have promised to give my daughter to a man who loves her, but not to one who does not. See him there, cold as marble and proud like his father. If he were rich, if he had Cavalcanti's fortune, that might be pardoned. \textit{Ma foi}, I haven't consulted my daughter; but if she has good taste\longdash> 

 <Oh,> said Monte Cristo, <my fondness may blind me, but I assure you I consider Morcerf a charming young man who will render your daughter happy and will sooner or later attain a certain amount of distinction, and his father's position is good.> 

 <Hem,> said Danglars. 

 <Why do you doubt?> 

 <The past—that obscurity on the past.> 

 <But that does not affect the son.> 

 <Very true.> 

 <Now, I beg of you, don't go off your head. It's a month now that you have been thinking of this marriage, and you must see that it throws some responsibility on me, for it was at my house you met this young Cavalcanti, whom I do not really know at all.> 

 <But I do.> 

 <Have you made inquiry?> 

 <Is there any need of that! Does not his appearance speak for him? And he is very rich.> 

 <I am not so sure of that.> 

 <And yet you said he had money.> 

 <Fifty thousand livres—a mere trifle.> 

 <He is well educated.> 

 <Hem,> said Monte Cristo in his turn. 

 <He is a musician.> 

 <So are all Italians.> 

 <Come, count, you do not do that young man justice.> 

 <Well, I acknowledge it annoys me, knowing your connection with the Morcerf family, to see him throw himself in the way.> Danglars burst out laughing. 

 <What a Puritan you are!> said he; <that happens every day.> 

 <But you cannot break it off in this way; the Morcerfs are depending on this union.> 

 <Indeed.> 

 <Positively.> 

 <Then let them explain themselves; you should give the father a hint, you are so intimate with the family.> 

 <I?—where the devil did you find out that?> 

 <At their ball; it was apparent enough. Why, did not the countess, the proud Mercédès, the disdainful Catalane, who will scarcely open her lips to her oldest acquaintances, take your arm, lead you into the garden, into the private walks, and remain there for half an hour?> 

 <Ah, baron, baron,> said Albert, <you are not listening—what barbarism in a megalomaniac like you!> 

 <Oh, don't worry about me, Sir Mocker,> said Danglars; then turning to Monte Cristo he said: 

 <But will you undertake to speak to the father?> 

 <Willingly, if you wish it.> 

 <But let it be done explicitly and positively. If he demands my daughter let him fix the day—declare his conditions; in short, let us either understand each other, or quarrel. You understand—no more delay.> 

 <Yes, sir, I will give my attention to the subject.> 

 <I do not say that I await with pleasure his decision, but I do await it. A banker must, you know, be a slave to his promise.> And Danglars sighed as M. Cavalcanti had done half an hour before. 

 <Bravi! bravo! brava!> cried Morcerf, parodying the banker, as the selection came to an end. Danglars began to look suspiciously at Morcerf, when someone came and whispered a few words to him. 

 <I shall soon return,> said the banker to Monte Cristo; <wait for me. I shall, perhaps, have something to say to you.> And he went out. 

 The baroness took advantage of her husband's absence to push open the door of her daughter's study, and M. Andrea, who was sitting before the piano with Mademoiselle Eugénie, started up like a jack-in-the-box. Albert bowed with a smile to Mademoiselle Danglars, who did not appear in the least disturbed, and returned his bow with her usual coolness. Cavalcanti was evidently embarrassed; he bowed to Morcerf, who replied with the most impertinent look possible. Then Albert launched out in praise of Mademoiselle Danglars' voice, and on his regret, after what he had just heard, that he had been unable to be present the previous evening. 

 Cavalcanti, being left alone, turned to Monte Cristo. 

 <Come,> said Madame Danglars, <leave music and compliments, and let us go and take tea.> 

 <Come, Louise,> said Mademoiselle Danglars to her friend. 

 They passed into the next drawing-room, where tea was prepared. Just as they were beginning, in the English fashion, to leave the spoons in their cups, the door again opened and Danglars entered, visibly agitated. Monte Cristo observed it particularly, and by a look asked the banker for an explanation. 

 <I have just received my courier from Greece,> said Danglars. 

 <Ah, yes,> said the count; <that was the reason of your running away from us.> 

 <Yes.> 

 <How is King Otho getting on?> asked Albert in the most sprightly tone. 

 Danglars cast another suspicious look towards him without answering, and Monte Cristo turned away to conceal the expression of pity which passed over his features, but which was gone in a moment. 

 <We shall go together, shall we not?> said Albert to the count. 

 <If you like,> replied the latter. 

 Albert could not understand the banker's look, and turning to Monte Cristo, who understood it perfectly,—<Did you see,> said he, <how he looked at me?> 

 <Yes,> said the count; <but did you think there was anything particular in his look?> 

 <Indeed, I did; and what does he mean by his news from Greece?> 

 <How can I tell you?> 

 <Because I imagine you have correspondents in that country.> 

 Monte Cristo smiled significantly. 

 <Stop,> said Albert, <here he comes. I shall compliment Mademoiselle Danglars on her cameo, while the father talks to you.> 

 <If you compliment her at all, let it be on her voice, at least,> said Monte Cristo. 

 <No, everyone would do that.> 

 <My dear viscount, you are dreadfully impertinent.> 

 Albert advanced towards Eugénie, smiling. 

 Meanwhile, Danglars, stooping to Monte Cristo's ear, <Your advice was excellent,> said he; <there is a whole history connected with the names Fernand and Yanina.> 

 <Indeed?> said Monte Cristo. 

 <Yes, I will tell you all; but take away the young man; I cannot endure his presence.> 

 <He is going with me. Shall I send the father to you?> 

 <Immediately.> 

 <Very well.> The count made a sign to Albert and they bowed to the ladies, and took their leave, Albert perfectly indifferent to Mademoiselle Danglars' contempt, Monte Cristo reiterating his advice to Madame Danglars on the prudence a banker's wife should exercise in providing for the future. 

 M. Cavalcanti remained master of the field. 