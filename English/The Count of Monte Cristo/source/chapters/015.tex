\chapter{Number 34 and Number 27} 

 \lettrine{D}{antès} passed through all the stages of torture natural to prisoners in suspense. He was sustained at first by that pride of conscious innocence which is the sequence to hope; then he began to doubt his own innocence, which justified in some measure the governor's belief in his mental alienation; and then, relaxing his sentiment of pride, he addressed his supplications, not to God, but to man. God is always the last resource. Unfortunates, who ought to begin with God, do not have any hope in him till they have exhausted all other means of deliverance. 

 Dantès asked to be removed from his present dungeon into another, even if it were darker and deeper, for a change, however disadvantageous, was still a change, and would afford him some amusement. He entreated to be allowed to walk about, to have fresh air, books, and writing materials. His requests were not granted, but he went on asking all the same. He accustomed himself to speaking to the new jailer, although the latter was, if possible, more taciturn than the old one; but still, to speak to a man, even though mute, was something. Dantès spoke for the sake of hearing his own voice; he had tried to speak when alone, but the sound of his voice terrified him. 

 Often, before his captivity, Dantès' mind had revolted at the idea of assemblages of prisoners, made up of thieves, vagabonds, and murderers. He now wished to be amongst them, in order to see some other face besides that of his jailer; he sighed for the galleys, with the infamous costume, the chain, and the brand on the shoulder. The galley-slaves breathed the fresh air of heaven, and saw each other. They were very happy. 

 He besought the jailer one day to let him have a companion, were it even the mad abbé. The jailer, though rough and hardened by the constant sight of so much suffering, was yet a man. At the bottom of his heart he had often had a feeling of pity for this unhappy young man who suffered so; and he laid the request of number 34 before the governor; but the latter sapiently imagined that Dantès wished to conspire or attempt an escape, and refused his request. Dantès had exhausted all human resources, and he then turned to God. 

 All the pious ideas that had been so long forgotten, returned; he recollected the prayers his mother had taught him, and discovered a new meaning in every word; for in prosperity prayers seem but a mere medley of words, until misfortune comes and the unhappy sufferer first understands the meaning of the sublime language in which he invokes the pity of heaven! He prayed, and prayed aloud, no longer terrified at the sound of his own voice, for he fell into a sort of ecstasy. He laid every action of his life before the Almighty, proposed tasks to accomplish, and at the end of every prayer introduced the entreaty oftener addressed to man than to God: <Forgive us our trespasses as we forgive them that trespass against us.> Yet in spite of his earnest prayers, Dantès remained a prisoner. 

 Then gloom settled heavily upon him. Dantès was a man of great simplicity of thought, and without education; he could not, therefore, in the solitude of his dungeon, traverse in mental vision the history of the ages, bring to life the nations that had perished, and rebuild the ancient cities so vast and stupendous in the light of the imagination, and that pass before the eye glowing with celestial colours in Martin's Babylonian pictures. He could not do this, he whose past life was so short, whose present so melancholy, and his future so doubtful. Nineteen years of light to reflect upon in eternal darkness! No distraction could come to his aid; his energetic spirit, that would have exalted in thus revisiting the past, was imprisoned like an eagle in a cage. He clung to one idea—that of his happiness, destroyed, without apparent cause, by an unheard-of fatality; he considered and reconsidered this idea, devoured it (so to speak), as the implacable Ugolino devours the skull of Archbishop Roger in the Inferno of Dante. 

 Rage supplanted religious fervour. Dantès uttered blasphemies that made his jailer recoil with horror, dashed himself furiously against the walls of his prison, wreaked his anger upon everything, and chiefly upon himself, so that the least thing,—a grain of sand, a straw, or a breath of air that annoyed him, led to paroxysms of fury. Then the letter that Villefort had showed to him recurred to his mind, and every line gleamed forth in fiery letters on the wall like the \textit{mene, mene, tekel upharsin} of Belshazzar. He told himself that it was the enmity of man, and not the vengeance of Heaven, that had thus plunged him into the deepest misery. He consigned his unknown persecutors to the most horrible tortures he could imagine, and found them all insufficient, because after torture came death, and after death, if not repose, at least the boon of unconsciousness. 

 By dint of constantly dwelling on the idea that tranquillity was death, and if punishment were the end in view other tortures than death must be invented, he began to reflect on suicide. Unhappy he, who, on the brink of misfortune, broods over ideas like these! 

 Before him is a dead sea that stretches in azure calm before the eye; but he who unwarily ventures within its embrace finds himself struggling with a monster that would drag him down to perdition. Once thus ensnared, unless the protecting hand of God snatch him thence, all is over, and his struggles but tend to hasten his destruction. This state of mental anguish is, however, less terrible than the sufferings that precede or the punishment that possibly will follow. There is a sort of consolation at the contemplation of the yawning abyss, at the bottom of which lie darkness and obscurity. 

 Edmond found some solace in these ideas. All his sorrows, all his sufferings, with their train of gloomy spectres, fled from his cell when the angel of death seemed about to enter. Dantès reviewed his past life with composure, and, looking forward with terror to his future existence, chose that middle line that seemed to afford him a refuge. 

 <Sometimes,> said he, <in my voyages, when I was a man and commanded other men, I have seen the heavens overcast, the sea rage and foam, the storm arise, and, like a monstrous bird, beating the two horizons with its wings. Then I felt that my vessel was a vain refuge, that trembled and shook before the tempest. Soon the fury of the waves and the sight of the sharp rocks announced the approach of death, and death then terrified me, and I used all my skill and intelligence as a man and a sailor to struggle against the wrath of God. But I did so because I was happy, because I had not courted death, because to be cast upon a bed of rocks and seaweed seemed terrible, because I was unwilling that I, a creature made for the service of God, should serve for food to the gulls and ravens. But now it is different; I have lost all that bound me to life, death smiles and invites me to repose; I die after my own manner, I die exhausted and broken-spirited, as I fall asleep when I have paced three thousand times round my cell,—that is thirty thousand steps, or about ten leagues.> 

 No sooner had this idea taken possession of him than he became more composed, arranged his couch to the best of his power, ate little and slept less, and found existence almost supportable, because he felt that he could throw it off at pleasure, like a worn-out garment. Two methods of self-destruction were at his disposal. He could hang himself with his handkerchief to the window bars, or refuse food and die of starvation. But the first was repugnant to him. Dantès had always entertained the greatest horror of pirates, who are hung up to the yard-arm; he would not die by what seemed an infamous death. He resolved to adopt the second, and began that day to carry out his resolve.  Nearly four years had passed away; at the end of the second he had ceased to mark the lapse of time. Dantès said, <I wish to die,> and had chosen the manner of his death, and fearful of changing his mind, he had taken an oath to die. <When my morning and evening meals are brought,> thought he, <I will cast them out of the window, and they will think that I have eaten them.> 

 He kept his word; twice a day he cast out, through the barred aperture, the provisions his jailer brought him—at first gayly, then with deliberation, and at last with regret. Nothing but the recollection of his oath gave him strength to proceed. Hunger made viands once repugnant, now acceptable; he held the plate in his hand for an hour at a time, and gazed thoughtfully at the morsel of bad meat, of tainted fish, of black and mouldy bread. It was the last yearning for life contending with the resolution of despair; then his dungeon seemed less sombre, his prospects less desperate. He was still young—he was only four or five-and-twenty—he had nearly fifty years to live. What unforseen events might not open his prison door, and restore him to liberty? Then he raised to his lips the repast that, like a voluntary Tantalus, he refused himself; but he thought of his oath, and he would not break it. He persisted until, at last, he had not sufficient strength to rise and cast his supper out of the loophole. The next morning he could not see or hear; the jailer feared he was dangerously ill. Edmond hoped he was dying. 

 Thus the day passed away. Edmond felt a sort of stupor creeping over him which brought with it a feeling almost of content; the gnawing pain at his stomach had ceased; his thirst had abated; when he closed his eyes he saw myriads of lights dancing before them like the will-o'-the-wisps that play about the marshes. It was the twilight of that mysterious country called Death! 

 Suddenly, about nine o'clock in the evening, Edmond heard a hollow sound in the wall against which he was lying. 

 So many loathsome animals inhabited the prison, that their noise did not, in general, awake him; but whether abstinence had quickened his faculties, or whether the noise was really louder than usual, Edmond raised his head and listened. It was a continual scratching, as if made by a huge claw, a powerful tooth, or some iron instrument attacking the stones. 

 Although weakened, the young man's brain instantly responded to the idea that haunts all prisoners—liberty! It seemed to him that heaven had at length taken pity on him, and had sent this noise to warn him on the very brink of the abyss. Perhaps one of those beloved ones he had so often thought of was thinking of him, and striving to diminish the distance that separated them. 

 No, no, doubtless he was deceived, and it was but one of those dreams that forerun death! 

 Edmond still heard the sound. It lasted nearly three hours; he then heard a noise of something falling, and all was silent. 

 Some hours afterwards it began again, nearer and more distinct. Edmond was intensely interested. Suddenly the jailer entered. 

 For a week since he had resolved to die, and during the four days that he had been carrying out his purpose, Edmond had not spoken to the attendant, had not answered him when he inquired what was the matter with him, and turned his face to the wall when he looked too curiously at him; but now the jailer might hear the noise and put an end to it, and so destroy a ray of something like hope that soothed his last moments. 

 The jailer brought him his breakfast. Dantès raised himself up and began to talk about everything; about the bad quality of the food, about the coldness of his dungeon, grumbling and complaining, in order to have an excuse for speaking louder, and wearying the patience of his jailer, who out of kindness of heart had brought broth and white bread for his prisoner. 

 Fortunately, he fancied that Dantès was delirious; and placing the food on the rickety table, he withdrew. Edmond listened, and the sound became more and more distinct. 

 <There can be no doubt about it,> thought he; <it is some prisoner who is striving to obtain his freedom. Oh, if I were only there to help him!> 

 Suddenly another idea took possession of his mind, so used to misfortune, that it was scarcely capable of hope—the idea that the noise was made by workmen the governor had ordered to repair the neighboring dungeon. 

 It was easy to ascertain this; but how could he risk the question? It was easy to call his jailer's attention to the noise, and watch his countenance as he listened; but might he not by this means destroy hopes far more important than the short-lived satisfaction of his own curiosity? Unfortunately, Edmond's brain was still so feeble that he could not bend his thoughts to anything in particular. He saw but one means of restoring lucidity and clearness to his judgment. He turned his eyes towards the soup which the jailer had brought, rose, staggered towards it, raised the vessel to his lips, and drank off the contents with a feeling of indescribable pleasure. 

 He had the resolution to stop with this. He had often heard that shipwrecked persons had died through having eagerly devoured too much food. Edmond replaced on the table the bread he was about to devour, and returned to his couch—he did not wish to die. He soon felt that his ideas became again collected—he could think, and strengthen his thoughts by reasoning. Then he said to himself: 

 <I must put this to the test, but without compromising anybody. If it is a workman, I need but knock against the wall, and he will cease to work, in order to find out who is knocking, and why he does so; but as his occupation is sanctioned by the governor, he will soon resume it. If, on the contrary, it is a prisoner, the noise I make will alarm him, he will cease, and not begin again until he thinks everyone is asleep.> 

 Edmond rose again, but this time his legs did not tremble, and his sight was clear; he went to a corner of his dungeon, detached a stone, and with it knocked against the wall where the sound came. He struck thrice. 

 At the first blow the sound ceased, as if by magic. 

 Edmond listened intently; an hour passed, two hours passed, and no sound was heard from the wall—all was silent there. 

 Full of hope, Edmond swallowed a few mouthfuls of bread and water, and, thanks to the vigour of his constitution, found himself well-nigh recovered. 

 The day passed away in utter silence—night came without recurrence of the noise. 

 <It is a prisoner,> said Edmond joyfully. His brain was on fire, and life and energy returned. 

 The night passed in perfect silence. Edmond did not close his eyes. 

 In the morning the jailer brought him fresh provisions—he had already devoured those of the previous day; he ate these listening anxiously for the sound, walking round and round his cell, shaking the iron bars of the loophole, restoring vigour and agility to his limbs by exercise, and so preparing himself for his future destiny. At intervals he listened to learn if the noise had not begun again, and grew impatient at the prudence of the prisoner, who did not guess he had been disturbed by a captive as anxious for liberty as himself. 

 Three days passed—seventy-two long tedious hours which he counted off by minutes! 

 At length one evening, as the jailer was visiting him for the last time that night, Dantès, with his ear for the hundredth time at the wall, fancied he heard an almost imperceptible movement among the stones. He moved away, walked up and down his cell to collect his thoughts, and then went back and listened. 

 The matter was no longer doubtful. Something was at work on the other side of the wall; the prisoner had discovered the danger, and had substituted a lever for a chisel. 

 Encouraged by this discovery, Edmond determined to assist the indefatigable laborer. He began by moving his bed, and looked around for anything with which he could pierce the wall, penetrate the moist cement, and displace a stone. 

 He saw nothing, he had no knife or sharp instrument, the window grating was of iron, but he had too often assured himself of its solidity. All his furniture consisted of a bed, a chair, a table, a pail, and a jug. The bed had iron clamps, but they were screwed to the wood, and it would have required a screw-driver to take them off. The table and chair had nothing, the pail had once possessed a handle, but that had been removed.  Dantès had but one resource, which was to break the jug, and with one of the sharp fragments attack the wall. He let the jug fall on the floor, and it broke in pieces. 

 Dantès concealed two or three of the sharpest fragments in his bed, leaving the rest on the floor. The breaking of his jug was too natural an accident to excite suspicion. Edmond had all the night to work in, but in the darkness he could not do much, and he soon felt that he was working against something very hard; he pushed back his bed, and waited for day. 

 All night he heard the subterranean workman, who continued to mine his way. Day came, the jailer entered. Dantès told him that the jug had fallen from his hands while he was drinking, and the jailer went grumblingly to fetch another, without giving himself the trouble to remove the fragments of the broken one. He returned speedily, advised the prisoner to be more careful, and departed. 

 Dantès heard joyfully the key grate in the lock; he listened until the sound of steps died away, and then, hastily displacing his bed, saw by the faint light that penetrated into his cell, that he had laboured uselessly the previous evening in attacking the stone instead of removing the plaster that surrounded it. 

 The damp had rendered it friable, and Dantès was able to break it off—in small morsels, it is true, but at the end of half an hour he had scraped off a handful; a mathematician might have calculated that in two years, supposing that the rock was not encountered, a passage twenty feet long and two feet broad, might be formed. 

 The prisoner reproached himself with not having thus employed the hours he had passed in vain hopes, prayer, and despondency. During the six years that he had been imprisoned, what might he not have accomplished? 

 This idea imparted new energy, and in three days he had succeeded, with the utmost precaution, in removing the cement, and exposing the stone-work. The wall was built of rough stones, among which, to give strength to the structure, blocks of hewn stone were at intervals imbedded. It was one of these he had uncovered, and which he must remove from its socket. 

 Dantès strove to do this with his nails, but they were too weak. The fragments of the jug broke, and after an hour of useless toil, Dantès paused with anguish on his brow. 

 Was he to be thus stopped at the beginning, and was he to wait inactive until his fellow workman had completed his task? Suddenly an idea occurred to him—he smiled, and the perspiration dried on his forehead. 

 The jailer always brought Dantès' soup in an iron saucepan; this saucepan contained soup for both prisoners, for Dantès had noticed that it was either quite full, or half empty, according as the turnkey gave it to him or to his companion first. 

 The handle of this saucepan was of iron; Dantès would have given ten years of his life in exchange for it.  The jailer was accustomed to pour the contents of the saucepan into Dantès' plate, and Dantès, after eating his soup with a wooden spoon, washed the plate, which thus served for every day. Now when evening came Dantès put his plate on the ground near the door; the jailer, as he entered, stepped on it and broke it. 

 This time he could not blame Dantès. He was wrong to leave it there, but the jailer was wrong not to have looked before him. The jailer, therefore, only grumbled. Then he looked about for something to pour the soup into; Dantès' entire dinner service consisted of one plate—there was no alternative. 

 <Leave the saucepan,> said Dantès; <you can take it away when you bring me my breakfast.> 

 This advice was to the jailer's taste, as it spared him the necessity of making another trip. He left the saucepan. 

 Dantès was beside himself with joy. He rapidly devoured his food, and after waiting an hour, lest the jailer should change his mind and return, he removed his bed, took the handle of the saucepan, inserted the point between the hewn stone and rough stones of the wall, and employed it as a lever. A slight oscillation showed Dantès that all went well. At the end of an hour the stone was extricated from the wall, leaving a cavity a foot and a half in diameter. 

 Dantès carefully collected the plaster, carried it into the corner of his cell, and covered it with earth. Then, wishing to make the best use of his time while he had the means of labour, he continued to work without ceasing. At the dawn of day he replaced the stone, pushed his bed against the wall, and lay down. The breakfast consisted of a piece of bread; the jailer entered and placed the bread on the table. 

 <Well, don't you intend to bring me another plate?> said Dantès. 

 <No,> replied the turnkey; <you destroy everything. First you break your jug, then you make me break your plate; if all the prisoners followed your example, the government would be ruined. I shall leave you the saucepan, and pour your soup into that. So for the future I hope you will not be so destructive.> 

 Dantès raised his eyes to heaven and clasped his hands beneath the coverlet. He felt more gratitude for the possession of this piece of iron than he had ever felt for anything. He had noticed, however, that the prisoner on the other side had ceased to labour; no matter, this was a greater reason for proceeding—if his neighbour would not come to him, he would go to his neighbour. All day he toiled on untiringly, and by the evening he had succeeded in extracting ten handfuls of plaster and fragments of stone. When the hour for his jailer's visit arrived, Dantès straightened the handle of the saucepan as well as he could, and placed it in its accustomed place. The turnkey poured his ration of soup into it, together with the fish—for thrice a week the prisoners were deprived of meat. This would have been a method of reckoning time, had not Dantès long ceased to do so. Having poured out the soup, the turnkey retired. 

 Dantès wished to ascertain whether his neighbour had really ceased to work. He listened—all was silent, as it had been for the last three days. Dantès sighed; it was evident that his neighbour distrusted him. However, he toiled on all the night without being discouraged; but after two or three hours he encountered an obstacle. The iron made no impression, but met with a smooth surface; Dantès touched it, and found that it was a beam. This beam crossed, or rather blocked up, the hole Dantès had made; it was necessary, therefore, to dig above or under it. The unhappy young man had not thought of this. 

 <Oh, my God, my God!> murmured he, <I have so earnestly prayed to you, that I hoped my prayers had been heard. After having deprived me of my liberty, after having deprived me of death, after having recalled me to existence, my God, have pity on me, and do not let me die in despair!>  <Who talks of God and despair at the same time?> said a voice that seemed to come from beneath the earth, and, deadened by the distance, sounded hollow and sepulchral in the young man's ears. Edmond's hair stood on end, and he rose to his knees. 

 <Ah,> said he, <I hear a human voice.> Edmond had not heard anyone speak save his jailer for four or five years; and a jailer is no man to a prisoner—he is a living door, a barrier of flesh and blood adding strength to restraints of oak and iron. 

 <In the name of Heaven,> cried Dantès, <speak again, though the sound of your voice terrifies me. Who are you?> 

 <Who are you?> said the voice. 

 <An unhappy prisoner,> replied Dantès, who made no hesitation in answering. 

 <Of what country?> 

 <A Frenchman.> 

 <Your name?> 

 <Edmond Dantès.> 

 <Your profession?> 

 <A sailor.> 

 <How long have you been here?> 

 <Since the 28th of February, 1815.> 

 <Your crime?> 

 <I am innocent.> 

 <But of what are you accused?> 

 <Of having conspired to aid the emperor's return.> 

 <What! For the emperor's return?—the emperor is no longer on the throne, then?> 

 <He abdicated at Fontainebleau in 1814, and was sent to the Island of Elba. But how long have you been here that you are ignorant of all this?> 

 <Since 1811.> 

 Dantès shuddered; this man had been four years longer than himself in prison. 

 <Do not dig any more,> said the voice; <only tell me how high up is your excavation?> 

 <On a level with the floor.> 

 <How is it concealed?> 

 <Behind my bed.> 

 <Has your bed been moved since you have been a prisoner?> 

 <No.> 

 <What does your chamber open on?> 

 <A corridor.> 

 <And the corridor?> 

 <On a court.> 

 <Alas!> murmured the voice. 

 <Oh, what is the matter?> cried Dantès. 

 <I have made a mistake owing to an error in my plans. I took the wrong angle, and have come out fifteen feet from where I intended. I took the wall you are mining for the outer wall of the fortress.> 

 <But then you would be close to the sea?> 

 <That is what I hoped.> 

 <And supposing you had succeeded?> 

 <I should have thrown myself into the sea, gained one of the islands near here—the Isle de Daume or the Isle de Tiboulen—and then I should have been safe.> 

 <Could you have swum so far?> 

 <Heaven would have given me strength; but now all is lost.> 

 <All?> 

 <Yes; stop up your excavation carefully, do not work any more, and wait until you hear from me.> 

 <Tell me, at least, who you are?> 

 <I am—I am № 27.> 

 <You mistrust me, then,> said Dantès. Edmond fancied he heard a bitter laugh resounding from the depths. 

 <Oh, I am a Christian,> cried Dantès, guessing instinctively that this man meant to abandon him. <I swear to you by him who died for us that naught shall induce me to breathe one syllable to my jailers; but I conjure you do not abandon me. If you do, I swear to you, for I have got to the end of my strength, that I will dash my brains out against the wall, and you will have my death to reproach yourself with.> 

 <How old are you? Your voice is that of a young man.> 

 <I do not know my age, for I have not counted the years I have been here. All I do know is, that I was just nineteen when I was arrested, the 28th of February, 1815.> 

 <Not quite twenty-six!> murmured the voice; <at that age he cannot be a traitor.> 

 <Oh, no, no,> cried Dantès. <I swear to you again, rather than betray you, I would allow myself to be hacked in pieces!> 

 <You have done well to speak to me, and ask for my assistance, for I was about to form another plan, and leave you; but your age reassures me. I will not forget you. Wait.> 

 <How long?> 

 <I must calculate our chances; I will give you the signal.> 

 <But you will not leave me; you will come to me, or you will let me come to you. We will escape, and if we cannot escape we will talk; you of those whom you love, and I of those whom I love. You must love somebody?> 

 <No, I am alone in the world.> 

 <Then you will love me. If you are young, I will be your comrade; if you are old, I will be your son. I have a father who is seventy if he yet lives; I only love him and a young girl called Mercédès. My father has not yet forgotten me, I am sure, but God alone knows if she loves me still; I shall love you as I loved my father.> 

 <It is well,> returned the voice; <tomorrow.> 

 These few words were uttered with an accent that left no doubt of his sincerity; Dantès rose, dispersed the fragments with the same precaution as before, and pushed his bed back against the wall. He then gave himself up to his happiness. He would no longer be alone. He was, perhaps, about to regain his liberty; at the worst, he would have a companion, and captivity that is shared is but half captivity. Plaints made in common are almost prayers, and prayers where two or three are gathered together invoke the mercy of heaven. 

 All day Dantès walked up and down his cell. He sat down occasionally on his bed, pressing his hand on his heart. At the slightest noise he bounded towards the door. Once or twice the thought crossed his mind that he might be separated from this unknown, whom he loved already; and then his mind was made up—when the jailer moved his bed and stooped to examine the opening, he would kill him with his water jug. He would be condemned to die, but he was about to die of grief and despair when this miraculous noise recalled him to life. 

 The jailer came in the evening. Dantès was on his bed. It seemed to him that thus he better guarded the unfinished opening. Doubtless there was a strange expression in his eyes, for the jailer said, <Come, are you going mad again?> 

 Dantès did not answer; he feared that the emotion of his voice would betray him. The jailer went away shaking his head. Night came; Dantès hoped that his neighbour would profit by the silence to address him, but he was mistaken. The next morning, however, just as he removed his bed from the wall, he heard three knocks; he threw himself on his knees. 

 <Is it you?> said he; <I am here.> 

 <Is your jailer gone?> 

 <Yes,> said Dantès; <he will not return until the evening; so that we have twelve hours before us.> 

 <I can work, then?> said the voice. 

 <Oh, yes, yes; this instant, I entreat you.> 

 In a moment that part of the floor on which Dantès was resting his two hands, as he knelt with his head in the opening, suddenly gave way; he drew back smartly, while a mass of stones and earth disappeared in a hole that opened beneath the aperture he himself had formed. Then from the bottom of this passage, the depth of which it was impossible to measure, he saw appear, first the head, then the shoulders, and lastly the body of a man, who sprang lightly into his cell. 