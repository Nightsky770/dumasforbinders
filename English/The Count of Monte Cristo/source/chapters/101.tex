\chapter{Locusta} 

 \lettrine{V}{alentine} was alone; two other clocks, slower than that of Saint-Philippe-du-Roule, struck the hour of midnight from different directions, and excepting the rumbling of a few carriages all was silent. Then Valentine's attention was engrossed by the clock in her room, which marked the seconds. She began counting them, remarking that they were much slower than the beatings of her heart; and still she doubted,—the inoffensive Valentine could not imagine that anyone should desire her death. Why should they? To what end? What had she done to excite the malice of an enemy? 

 There was no fear of her falling asleep. One terrible idea pressed upon her mind,—that someone existed in the world who had attempted to assassinate her, and who was about to endeavour to do so again. Supposing this person, wearied at the inefficacy of the poison, should, as Monte Cristo intimated, have recourse to steel!—What if the count should have no time to run to her rescue!—What if her last moments were approaching, and she should never again see Morrel! 

 When this terrible chain of ideas presented itself, Valentine was nearly persuaded to ring the bell, and call for help. But through the door she fancied she saw the luminous eye of the count—that eye which lived in her memory, and the recollection overwhelmed her with so much shame that she asked herself whether any amount of gratitude could ever repay his adventurous and devoted friendship. 

 Twenty minutes, twenty tedious minutes, passed thus, then ten more, and at last the clock struck the half-hour. 

 Just then the sound of finger-nails slightly grating against the door of the library informed Valentine that the count was still watching, and recommended her to do the same; at the same time, on the opposite side, that is towards Edward's room, Valentine fancied that she heard the creaking of the floor; she listened attentively, holding her breath till she was nearly suffocated; the lock turned, and the door slowly opened. Valentine had raised herself upon her elbow, and had scarcely time to throw herself down on the bed and shade her eyes with her arm; then, trembling, agitated, and her heart beating with indescribable terror, she awaited the event. 

 Someone approached the bed and drew back the curtains. Valentine summoned every effort, and breathed with that regular respiration which announces tranquil sleep. 

 <Valentine!> said a low voice. 

 The girl shuddered to the heart but did not reply. 

 <Valentine,> repeated the same voice. 

 Still silent: Valentine had promised not to wake. Then everything was still, excepting that Valentine heard the almost noiseless sound of some liquid being poured into the glass she had just emptied. Then she ventured to open her eyelids, and glance over her extended arm. She saw a woman in a white dressing-gown pouring a liquor from a phial into her glass. During this short time Valentine must have held her breath, or moved in some slight degree, for the woman, disturbed, stopped and leaned over the bed, in order the better to ascertain whether Valentine slept: it was Madame de Villefort. 

 On recognizing her step-mother, Valentine could not repress a shudder, which caused a vibration in the bed. Madame de Villefort instantly stepped back close to the wall, and there, shaded by the bed-curtains, she silently and attentively watched the slightest movement of Valentine. The latter recollected the terrible caution of Monte Cristo; she fancied that the hand not holding the phial clasped a long sharp knife. Then collecting all her remaining strength, she forced herself to close her eyes; but this simple operation upon the most delicate organs of our frame, generally so easy to accomplish, became almost impossible at this moment, so much did curiosity struggle to retain the eyelid open and learn the truth. Madame de Villefort, however, reassured by the silence, which was alone disturbed by the regular breathing of Valentine, again extended her hand, and half hidden by the curtains succeeded in emptying the contents of the phial into the glass. Then she retired so gently that Valentine did not know she had left the room. She only witnessed the withdrawal of the arm—the fair round arm of a woman but twenty-five years old, and who yet spread death around her.  It is impossible to describe the sensations experienced by Valentine during the minute and a half Madame de Villefort remained in the room. 

 The grating against the library-door aroused the young girl from the stupor in which she was plunged, and which almost amounted to insensibility. She raised her head with an effort. The noiseless door again turned on its hinges, and the Count of Monte Cristo reappeared. 

 <Well,> said he, <do you still doubt?> 

 <Oh,> murmured the young girl. 

 <Have you seen?> 

 <Alas!> 

 <Did you recognize?> Valentine groaned. 

 <Oh, yes;> she said, <I saw, but I cannot believe!> 

 <Would you rather die, then, and cause Maximilian's death?> 

 <Oh,> repeated the young girl, almost bewildered, <can I not leave the house?—can I not escape?> 

 <Valentine, the hand which now threatens you will pursue you everywhere; your servants will be seduced with gold, and death will be offered to you disguised in every shape. You will find it in the water you drink from the spring, in the fruit you pluck from the tree.> 

 <But did you not say that my kind grandfather's precaution had neutralized the poison?> 

 <Yes, but not against a strong dose; the poison will be changed, and the quantity increased.> He took the glass and raised it to his lips. <It is already done,> he said; <brucine is no longer employed, but a simple narcotic! I can recognize the flavour of the alcohol in which it has been dissolved. If you had taken what Madame de Villefort has poured into your glass, Valentine—Valentine—you would have been doomed!> 

 <But,> exclaimed the young girl, <why am I thus pursued?> 

 <Why?—are you so kind—so good—so unsuspicious of ill, that you cannot understand, Valentine?> 

 <No, I have never injured her.> 

 <But you are rich, Valentine; you have 200,000 livres a year, and you prevent her son from enjoying these 200,000 livres.> 

 <How so? The fortune is not her gift, but is inherited from my relations.> 

 <Certainly; and that is why M. and Madame de Saint-Méran have died; that is why M. Noirtier was sentenced the day he made you his heir; that is why you, in your turn, are to die—it is because your father would inherit your property, and your brother, his only son, succeed to his.> 

 <Edward? Poor child! Are all these crimes committed on his account?> 

 <Ah, then you at length understand?> 

 <Heaven grant that this may not be visited upon him!> 

 <Valentine, you are an angel!> 

 <But why is my grandfather allowed to live?> 

 <It was considered, that you dead, the fortune would naturally revert to your brother, unless he were disinherited; and besides, the crime appearing useless, it would be folly to commit it.> 

 <And is it possible that this frightful combination of crimes has been invented by a woman?> 

 <Do you recollect in the arbour of the Hôtel des Postes, at Perugia, seeing a man in a brown cloak, whom your stepmother was questioning upon \textit{aqua tofana}? Well, ever since then, the infernal project has been ripening in her brain.> 

 <Ah, then, indeed, sir,> said the sweet girl, bathed in tears, <I see that I am condemned to die!> 

 <No, Valentine, for I have foreseen all their plots; no, your enemy is conquered since we know her, and you will live, Valentine—live to be happy yourself, and to confer happiness upon a noble heart; but to insure this you must rely on me.> 

 <Command me, sir—what am I to do?> 

 <You must blindly take what I give you.> 

 <Alas, were it only for my own sake, I should prefer to die!> 

 <You must not confide in anyone—not even in your father.> 

 <My father is not engaged in this fearful plot, is he, sir?> asked Valentine, clasping her hands. 

 <No; and yet your father, a man accustomed to judicial accusations, ought to have known that all these deaths have not happened naturally; it is he who should have watched over you—he should have occupied my place—he should have emptied that glass—he should have risen against the assassin. Spectre against spectre!> he murmured in a low voice, as he concluded his sentence. 

 <Sir,> said Valentine, <I will do all I can to live, for there are two beings who love me and will die if I die—my grandfather and Maximilian.> 

 <I will watch over them as I have over you.> 

 <Well, sir, do as you will with me;> and then she added, in a low voice, <oh, heavens, what will befall me?> 

 <Whatever may happen, Valentine, do not be alarmed; though you suffer; though you lose sight, hearing, consciousness, fear nothing; though you should awake and be ignorant where you are, still do not fear; even though you should find yourself in a sepulchral vault or coffin. Reassure yourself, then, and say to yourself: <At this moment, a friend, a father, who lives for my happiness and that of Maximilian, watches over me!>> 

 <Alas, alas, what a fearful extremity!> 

 <Valentine, would you rather denounce your stepmother?> 

 <I would rather die a hundred times—oh, yes, die!> 

 <No, you will not die; but will you promise me, whatever happens, that you will not complain, but hope?> 

 <I will think of Maximilian!> 

 <You are my own darling child, Valentine! I alone can save you, and I will.> 

 Valentine in the extremity of her terror joined her hands,—for she felt that the moment had arrived to ask for courage,—and began to pray, and while uttering little more than incoherent words, she forgot that her white shoulders had no other covering than her long hair, and that the pulsations of her heart could be seen through the lace of her nightdress. Monte Cristo gently laid his hand on the young girl's arm, drew the velvet coverlet close to her throat, and said with a paternal smile: 

 <My child, believe in my devotion to you as you believe in the goodness of Providence and the love of Maximilian.> Valentine gave him a look full of gratitude, and remained as docile as a child. 

 Then he drew from his waistcoat-pocket the little emerald box, raised the golden lid, and took from it a pastille about the size of a pea, which he placed in her hand. She took it, and looked attentively on the count; there was an expression on the face of her intrepid protector which commanded her veneration. She evidently interrogated him by her look. 

 <Yes,> said he. 

 Valentine carried the pastille to her mouth, and swallowed it. 

 <And now, my dear child, adieu for the present. I will try and gain a little sleep, for you are saved.> 

 <Go,> said Valentine, <whatever happens, I promise you not to fear.> 

 Monte Cristo for some time kept his eyes fixed on the young girl, who gradually fell asleep, yielding to the effects of the narcotic the count had given her. Then he took the glass, emptied three parts of the contents in the fireplace, that it might be supposed Valentine had taken it, and replaced it on the table; then he disappeared, after throwing a farewell glance on Valentine, who slept with the confidence and innocence of an angel at the feet of the Lord. 