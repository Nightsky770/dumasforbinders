\chapter{The Marriage Feast} 

 \lettrine{T}{he} morning's sun rose clear and resplendent, touching the foamy waves into a network of ruby-tinted light. 

\zz
 The feast had been made ready on the second floor at La Réserve, with whose arbour the reader is already familiar. The apartment destined for the purpose was spacious and lighted by a number of windows, over each of which was written in golden letters for some inexplicable reason the name of one of the principal cities of France; beneath these windows a wooden balcony extended the entire length of the house. And although the entertainment was fixed for twelve o'clock, an hour previous to that time the balcony was filled with impatient and expectant guests, consisting of the favoured part of the crew of the \textit{Pharaon}, and other personal friends of the bridegroom, the whole of whom had arrayed themselves in their choicest costumes, in order to do greater honour to the occasion. 

 Various rumours were afloat to the effect that the owners of the \textit{Pharaon} had promised to attend the nuptial feast; but all seemed unanimous in doubting that an act of such rare and exceeding condescension could possibly be intended. 

 Danglars, however, who now made his appearance, accompanied by Caderousse, effectually confirmed the report, stating that he had recently conversed with M. Morrel, who had himself assured him of his intention to dine at La Réserve. 

 In fact, a moment later M. Morrel appeared and was saluted with an enthusiastic burst of applause from the crew of the \textit{Pharaon}, who hailed the visit of the shipowner as a sure indication that the man whose wedding feast he thus delighted to honour would ere long be first in command of the ship; and as Dantès was universally beloved on board his vessel, the sailors put no restraint on their tumultuous joy at finding that the opinion and choice of their superiors so exactly coincided with their own. 

 With the entrance of M. Morrel, Danglars and Caderousse were despatched in search of the bridegroom to convey to him the intelligence of the arrival of the important personage whose coming had created such a lively sensation, and to beseech him to make haste. 

 Danglars and Caderousse set off upon their errand at full speed; but ere they had gone many steps they perceived a group advancing towards them, composed of the betrothed pair, a party of young girls in attendance on the bride, by whose side walked Dantès' father; the whole brought up by Fernand, whose lips wore their usual sinister smile. 

 Neither Mercédès nor Edmond observed the strange expression of his countenance; they were so happy that they were conscious only of the sunshine and the presence of each other. 

 Having acquitted themselves of their errand, and exchanged a hearty shake of the hand with Edmond, Danglars and Caderousse took their places beside Fernand and old Dantès,—the latter of whom attracted universal notice. 

 The old man was attired in a suit of glistening watered silk, trimmed with steel buttons, beautifully cut and polished. His thin but wiry legs were arrayed in a pair of richly embroidered clocked stockings, evidently of English manufacture, while from his three-cornered hat depended a long streaming knot of white and blue ribbons. Thus he came along, supporting himself on a curiously carved stick, his aged countenance lit up with happiness, looking for all the world like one of the aged dandies of 1796, parading the newly opened gardens of the Luxembourg and Tuileries. 

 Beside him glided Caderousse, whose desire to partake of the good things provided for the wedding party had induced him to become reconciled to the Dantès, father and son, although there still lingered in his mind a faint and unperfect recollection of the events of the preceding night; just as the brain retains on waking in the morning the dim and misty outline of a dream.  As Danglars approached the disappointed lover, he cast on him a look of deep meaning, while Fernand, as he slowly paced behind the happy pair, who seemed, in their own unmixed content, to have entirely forgotten that such a being as himself existed, was pale and abstracted; occasionally, however, a deep flush would overspread his countenance, and a nervous contraction distort his features, while, with an agitated and restless gaze, he would glance in the direction of Marseilles, like one who either anticipated or foresaw some great and important event. 

 Dantès himself was simply, but becomingly, clad in the dress peculiar to the merchant service—a costume somewhat between a military and a civil garb; and with his fine countenance, radiant with joy and happiness, a more perfect specimen of manly beauty could scarcely be imagined. 

 Lovely as the Greek girls of Cyprus or Chios, Mercédès boasted the same bright flashing eyes of jet, and ripe, round, coral lips. She moved with the light, free step of an Arlesienne or an Andalusian. One more practised in the arts of great cities would have hid her blushes beneath a veil, or, at least, have cast down her thickly fringed lashes, so as to have concealed the liquid lustre of her animated eyes; but, on the contrary, the delighted girl looked around her with a smile that seemed to say: <If you are my friends, rejoice with me, for I am very happy.> 

 As soon as the bridal party came in sight of La Réserve, M. Morrel descended and came forth to meet it, followed by the soldiers and sailors there assembled, to whom he had repeated the promise already given, that Dantès should be the successor to the late Captain Leclere. Edmond, at the approach of his patron, respectfully placed the arm of his affianced bride within that of M. Morrel, who, forthwith conducting her up the flight of wooden steps leading to the chamber in which the feast was prepared, was gaily followed by the guests, beneath whose heavy tread the slight structure creaked and groaned for the space of several minutes. 

 <Father,> said Mercédès, stopping when she had reached the centre of the table, <sit, I pray you, on my right hand; on my left I will place him who has ever been as a brother to me,> pointing with a soft and gentle smile to Fernand; but her words and look seemed to inflict the direst torture on him, for his lips became ghastly pale, and even beneath the dark hue of his complexion the blood might be seen retreating as though some sudden pang drove it back to the heart. 

 During this time, Dantès, at the opposite side of the table, had been occupied in similarly placing his most honoured guests. M. Morrel was seated at his right hand, Danglars at his left; while, at a sign from Edmond, the rest of the company ranged themselves as they found it most agreeable. 

 Then they began to pass around the dusky, piquant, Arlesian sausages, and lobsters in their dazzling red cuirasses, prawns of large size and brilliant colour, the echinus with its prickly outside and dainty morsel within, the clovis, esteemed by the epicures of the South as more than rivalling the exquisite flavour of the oyster, North. All the delicacies, in fact, that are cast up by the wash of waters on the sandy beach, and styled by the grateful fishermen <fruits of the sea.> 

 <A pretty silence truly!> said the old father of the bridegroom, as he carried to his lips a glass of wine of the hue and brightness of the topaz, and which had just been placed before Mercédès herself. <Now, would anybody think that this room contained a happy, merry party, who desire nothing better than to laugh and dance the hours away?> 

 <Ah,> sighed Caderousse, <a man cannot always feel happy because he is about to be married.> 

 <The truth is,> replied Dantès, <that I am too happy for noisy mirth; if that is what you meant by your observation, my worthy friend, you are right; joy takes a strange effect at times, it seems to oppress us almost the same as sorrow.> 

 Danglars looked towards Fernand, whose excitable nature received and betrayed each fresh impression. 

 <Why, what ails you?> asked he of Edmond. <Do you fear any approaching evil? I should say that you were the happiest man alive at this instant.> 

 <And that is the very thing that alarms me,> returned Dantès. <Man does not appear to me to be intended to enjoy felicity so unmixed; happiness is like the enchanted palaces we read of in our childhood, where fierce, fiery dragons defend the entrance and approach; and monsters of all shapes and kinds, requiring to be overcome ere victory is ours. I own that I am lost in wonder to find myself promoted to an honour of which I feel myself unworthy—that of being the husband of Mercédès.> 

 <Nay, nay!> cried Caderousse, smiling, <you have not attained that honour yet. Mercédès is not yet your wife. Just assume the tone and manner of a husband, and see how she will remind you that your hour is not yet come!> 

 The bride blushed, while Fernand, restless and uneasy, seemed to start at every fresh sound, and from time to time wiped away the large drops of perspiration that gathered on his brow. 

 <Well, never mind that, neighbour Caderousse; it is not worthwhile to contradict me for such a trifle as that. 'Tis true that Mercédès is not actually my wife; but,> added he, drawing out his watch, <in an hour and a half she will be.> 

 A general exclamation of surprise ran round the table, with the exception of the elder Dantès, whose laugh displayed the still perfect beauty of his large white teeth. Mercédès looked pleased and gratified, while Fernand grasped the handle of his knife with a convulsive clutch. 

 <In an hour?> inquired Danglars, turning pale. <How is that, my friend?> 

 <Why, thus it is,> replied Dantès. <Thanks to the influence of M. Morrel, to whom, next to my father, I owe every blessing I enjoy, every difficulty has been removed. We have purchased permission to waive the usual delay; and at half-past two o'clock the Mayor of Marseilles will be waiting for us at the city hall. Now, as a quarter-past one has already struck, I do not consider I have asserted too much in saying, that, in another hour and thirty minutes Mercédès will have become Madame Dantès.>  Fernand closed his eyes, a burning sensation passed across his brow, and he was compelled to support himself by the table to prevent his falling from his chair; but in spite of all his efforts, he could not refrain from uttering a deep groan, which, however, was lost amid the noisy felicitations of the company. 

 <Upon my word,> cried the old man, <you make short work of this kind of affair. Arrived here only yesterday morning, and married today at three o'clock! Commend me to a sailor for going the quick way to work!> 

 <But,> asked Danglars, in a timid tone, <how did you manage about the other formalities—the contract—the settlement?> 

 <The contract,> answered Dantès, laughingly, <it didn't take long to fix that. Mercédès has no fortune; I have none to settle on her. So, you see, our papers were quickly written out, and certainly do not come very expensive.> This joke elicited a fresh burst of applause. 

 <So that what we presumed to be merely the betrothal feast turns out to be the actual wedding dinner!> said Danglars. 

 <No, no,> answered Dantès; <don't imagine I am going to put you off in that shabby manner. Tomorrow morning I start for Paris; four days to go, and the same to return, with one day to discharge the commission entrusted to me, is all the time I shall be absent. I shall be back here by the first of March, and on the second I give my real marriage feast.> 

 This prospect of fresh festivity redoubled the hilarity of the guests to such a degree, that the elder Dantès, who, at the commencement of the repast, had commented upon the silence that prevailed, now found it difficult, amid the general din of voices, to obtain a moment's tranquillity in which to drink to the health and prosperity of the bride and bridegroom. 

 Dantès, perceiving the affectionate eagerness of his father, responded by a look of grateful pleasure; while Mercédès glanced at the clock and made an expressive gesture to Edmond. 

 Around the table reigned that noisy hilarity which usually prevails at such a time among people sufficiently free from the demands of social position not to feel the trammels of etiquette. Such as at the commencement of the repast had not been able to seat themselves according to their inclination rose unceremoniously, and sought out more agreeable companions. Everybody talked at once, without waiting for a reply and each one seemed to be contented with expressing his or her own thoughts. 

 Fernand's paleness appeared to have communicated itself to Danglars. As for Fernand himself, he seemed to be enduring the tortures of the damned; unable to rest, he was among the first to quit the table, and, as though seeking to avoid the hilarious mirth that rose in such deafening sounds, he continued, in utter silence, to pace the farther end of the salon. 

 Caderousse approached him just as Danglars, whom Fernand seemed most anxious to avoid, had joined him in a corner of the room. 

 <Upon my word,> said Caderousse, from whose mind the friendly treatment of Dantès, united with the effect of the excellent wine he had partaken of, had effaced every feeling of envy or jealousy at Dantès' good fortune,—<upon my word, Dantès is a downright good fellow, and when I see him sitting there beside his pretty wife that is so soon to be. I cannot help thinking it would have been a great pity to have served him that trick you were planning yesterday.> 

 <Oh, there was no harm meant,> answered Danglars; <at first I certainly did feel somewhat uneasy as to what Fernand might be tempted to do; but when I saw how completely he had mastered his feelings, even so far as to become one of his rival's attendants, I knew there was no further cause for apprehension.> Caderousse looked full at Fernand—he was ghastly pale. 

 <Certainly,> continued Danglars, <the sacrifice was no trifling one, when the beauty of the bride is concerned. Upon my soul, that future captain of mine is a lucky dog! Gad! I only wish he would let me take his place.> 

 <Shall we not set forth?> asked the sweet, silvery voice of Mercédès; <two o'clock has just struck, and you know we are expected in a quarter of an hour.>  
 
 <To be sure!—to be sure!> cried Dantès, eagerly quitting the table; <let us go directly!> 

 His words were re-echoed by the whole party, with vociferous cheers. 

 At this moment Danglars, who had been incessantly observing every change in Fernand's look and manner, saw him stagger and fall back, with an almost convulsive spasm, against a seat placed near one of the open windows. At the same instant his ear caught a sort of indistinct sound on the stairs, followed by the measured tread of soldiery, with the clanking of swords and military accoutrements; then came a hum and buzz as of many voices, so as to deaden even the noisy mirth of the bridal party, among whom a vague feeling of curiosity and apprehension quelled every disposition to talk, and almost instantaneously the most deathlike stillness prevailed. 

 The sounds drew nearer. Three blows were struck upon the panel of the door. The company looked at each other in consternation. 

 <I demand admittance,> said a loud voice outside the room, <in the name of the law!> As no attempt was made to prevent it, the door was opened, and a magistrate, wearing his official scarf, presented himself, followed by four soldiers and a corporal. Uneasiness now yielded to the most extreme dread on the part of those present. 

 <May I venture to inquire the reason of this unexpected visit?> said M. Morrel, addressing the magistrate, whom he evidently knew; <there is doubtless some mistake easily explained.> 

 <If it be so,> replied the magistrate, <rely upon every reparation being made; meanwhile, I am the bearer of an order of arrest, and although I most reluctantly perform the task assigned me, it must, nevertheless, be fulfilled. Who among the persons here assembled answers to the name of Edmond Dantès?> 

 Every eye was turned towards the young man who, spite of the agitation he could not but feel, advanced with dignity, and said, in a firm voice: 

 <I am he; what is your pleasure with me?> 

 <Edmond Dantès,> replied the magistrate, <I arrest you in the name of the law!> 

 <Me!> repeated Edmond, slightly changing colour, <and wherefore, I pray?> 

 <I cannot inform you, but you will be duly acquainted with the reasons that have rendered such a step necessary at the preliminary examination.> 

 M. Morrel felt that further resistance or remonstrance was useless. He saw before him an officer delegated to enforce the law, and perfectly well knew that it would be as unavailing to seek pity from a magistrate decked with his official scarf, as to address a petition to some cold marble effigy. Old Dantès, however, sprang forward. There are situations which the heart of a father or a mother cannot be made to understand. He prayed and supplicated in terms so moving, that even the officer was touched, and, although firm in his duty, he kindly said, <My worthy friend, let me beg of you to calm your apprehensions. Your son has probably neglected some prescribed form or attention in registering his cargo, and it is more than probable he will be set at liberty directly he has given the information required, whether touching the health of his crew, or the value of his freight.> 

 <What is the meaning of all this?> inquired Caderousse, frowningly, of Danglars, who had assumed an air of utter surprise.  <How can I tell you?> replied he; <I am, like yourself, utterly bewildered at all that is going on, and cannot in the least make out what it is about.> Caderousse then looked around for Fernand, but he had disappeared. 

 The scene of the previous night now came back to his mind with startling clearness. The painful catastrophe he had just witnessed appeared effectually to have rent away the veil which the intoxication of the evening before had raised between himself and his memory. 

 <So, so,> said he, in a hoarse and choking voice, to Danglars, <this, then, I suppose, is a part of the trick you were concerting yesterday? All I can say is, that if it be so, 'tis an ill turn, and well deserves to bring double evil on those who have projected it.> 

 <Nonsense,> returned Danglars, <I tell you again I have nothing whatever to do with it; besides, you know very well that I tore the paper to pieces.> 

 <No, you did not!> answered Caderousse, <you merely threw it by—I saw it lying in a corner.> 

 <Hold your tongue, you fool!—what should you know about it?—why, you were drunk!> 

 <Where is Fernand?> inquired Caderousse. 

 <How do I know?> replied Danglars; <gone, as every prudent man ought to be, to look after his own affairs, most likely. Never mind where he is, let you and I go and see what is to be done for our poor friends.> 

 During this conversation, Dantès, after having exchanged a cheerful shake of the hand with all his sympathizing friends, had surrendered himself to the officer sent to arrest him, merely saying, <Make yourselves quite easy, my good fellows, there is some little mistake to clear up, that's all, depend upon it; and very likely I may not have to go so far as the prison to effect that.>  
 
 <Oh, to be sure!> responded Danglars, who had now approached the group, <nothing more than a mistake, I feel quite certain.> 

 Dantès descended the staircase, preceded by the magistrate, and followed by the soldiers. A carriage awaited him at the door; he got in, followed by two soldiers and the magistrate, and the vehicle drove off towards Marseilles. 

 <Adieu, adieu, dearest Edmond!> cried Mercédès, stretching out her arms to him from the balcony. 

 The prisoner heard the cry, which sounded like the sob of a broken heart, and leaning from the coach he called out, <Good-bye, Mercédès—we shall soon meet again!> Then the vehicle disappeared round one of the turnings of Fort Saint Nicholas. 

 <Wait for me here, all of you!> cried M. Morrel; <I will take the first conveyance I find, and hurry to Marseilles, whence I will bring you word how all is going on.> 

 <That's right!> exclaimed a multitude of voices, <go, and return as quickly as you can!> 

 This second departure was followed by a long and fearful state of terrified silence on the part of those who were left behind. The old father and Mercédès remained for some time apart, each absorbed in grief; but at length the two poor victims of the same blow raised their eyes, and with a simultaneous burst of feeling rushed into each other's arms. 

 Meanwhile Fernand made his appearance, poured out for himself a glass of water with a trembling hand; then hastily swallowing it, went to sit down at the first vacant place, and this was, by mere chance, placed next to the seat on which poor Mercédès had fallen half fainting, when released from the warm and affectionate embrace of old Dantès. Instinctively Fernand drew back his chair. 

 <He is the cause of all this misery—I am quite sure of it,> whispered Caderousse, who had never taken his eyes off Fernand, to Danglars. 

 <I don't think so,> answered the other; <he's too stupid to imagine such a scheme. I only hope the mischief will fall upon the head of whoever wrought it.> 

 <You don't mention those who aided and abetted the deed,> said Caderousse. 

 <Surely,> answered Danglars, <one cannot be held responsible for every chance arrow shot into the air.> 

 <You can, indeed, when the arrow lights point downward on somebody's head.> 

 Meantime the subject of the arrest was being canvassed in every different form. 

 <What think you, Danglars,> said one of the party, turning towards him, <of this event?> 

 <Why,> replied he, <I think it just possible Dantès may have been detected with some trifling article on board ship considered here as contraband.> 

 <But how could he have done so without your knowledge, Danglars, since you are the ship's supercargo?> 

 <Why, as for that, I could only know what I was told respecting the merchandise with which the vessel was laden. I know she was loaded with cotton, and that she took in her freight at Alexandria from Pastret's warehouse, and at Smyrna from Pascal's; that is all I was obliged to know, and I beg I may not be asked for any further particulars.> 

 <Now I recollect,> said the afflicted old father; <my poor boy told me yesterday he had got a small case of coffee, and another of tobacco for me!> 

 <There, you see,> exclaimed Danglars. <Now the mischief is out; depend upon it the custom-house people went rummaging about the ship in our absence, and discovered poor Dantès' hidden treasures.> 

 Mercédès, however, paid no heed to this explanation of her lover's arrest. Her grief, which she had hitherto tried to restrain, now burst out in a violent fit of hysterical sobbing. 

 <Come, come,> said the old man, <be comforted, my poor child; there is still hope!> 

 <Hope!> repeated Danglars. 

 <Hope!> faintly murmured Fernand, but the word seemed to die away on his pale agitated lips, and a convulsive spasm passed over his countenance. 

 <Good news! good news!> shouted forth one of the party stationed in the balcony on the lookout. <Here comes M. Morrel back. No doubt, now, we shall hear that our friend is released!> 

 Mercédès and the old man rushed to meet the shipowner and greeted him at the door. He was very pale. 

 <What news?> exclaimed a general burst of voices. 

 <Alas, my friends,> replied M. Morrel, with a mournful shake of his head, <the thing has assumed a more serious aspect than I expected.> 

 <Oh, indeed—indeed, sir, he is innocent!> sobbed forth Mercédès. 

 <That I believe!> answered M. Morrel; <but still he is charged\longdash> 

 <With what?> inquired the elder Dantès. 

 <With being an agent of the Bonapartist faction!> Many of our readers may be able to recollect how formidable such an accusation became in the period at which our story is dated. 

 A despairing cry escaped the pale lips of Mercédès; the old man sank into a chair. 

 <Ah, Danglars!> whispered Caderousse, <you have deceived me—the trick you spoke of last night has been played; but I cannot suffer a poor old man or an innocent girl to die of grief through your fault. I am determined to tell them all about it.> 

 <Be silent, you simpleton!> cried Danglars, grasping him by the arm, <or I will not answer even for your own safety. Who can tell whether Dantès be innocent or guilty? The vessel did touch at Elba, where he quitted it, and passed a whole day in the island. Now, should any letters or other documents of a compromising character be found upon him, will it not be taken for granted that all who uphold him are his accomplices?> 

 With the rapid instinct of selfishness, Caderousse readily perceived the solidity of this mode of reasoning; he gazed, doubtfully, wistfully, on Danglars, and then caution supplanted generosity. 

 <Suppose we wait a while, and see what comes of it,> said he, casting a bewildered look on his companion. 

 <To be sure!> answered Danglars. <Let us wait, by all means. If he be innocent, of course he will be set at liberty; if guilty, why, it is no use involving ourselves in a conspiracy.> 

 <Let us go, then. I cannot stay here any longer.> 

 <With all my heart!> replied Danglars, pleased to find the other so tractable. <Let us take ourselves out of the way, and leave things for the present to take their course.> 

 After their departure, Fernand, who had now again become the friend and protector of Mercédès, led the girl to her home, while some friends of Dantès conducted his father, nearly lifeless, to the Allées de Meilhan. 

 The rumour of Edmond's arrest as a Bonapartist agent was not slow in circulating throughout the city. 

 <Could you ever have credited such a thing, my dear Danglars?> asked M. Morrel, as, on his return to the port for the purpose of gleaning fresh tidings of Dantès, from M. de Villefort, the assistant procureur, he overtook his supercargo and Caderousse. <Could you have believed such a thing possible?> 

 <Why, you know I told you,> replied Danglars, <that I considered the circumstance of his having anchored at the Island of Elba as a very suspicious circumstance.> 

 <And did you mention these suspicions to any person beside myself?>  
 
 <Certainly not!> returned Danglars. Then added in a low whisper, <You understand that, on account of your uncle, M. Policar Morrel, who served under the \textit{other} government, and who does not altogether conceal what he thinks on the subject, you are strongly suspected of regretting the abdication of Napoleon. I should have feared to injure both Edmond and yourself, had I divulged my own apprehensions to a soul. I am too well aware that though a subordinate, like myself, is bound to acquaint the shipowner with everything that occurs, there are many things he ought most carefully to conceal from all else.> 

 <'Tis well, Danglars—'tis well!> replied M. Morrel. <You are a worthy fellow; and I had already thought of your interests in the event of poor Edmond having become captain of the \textit{Pharaon}.> 

 <Is it possible you were so kind?> 

 <Yes, indeed; I had previously inquired of Dantès what was his opinion of you, and if he should have any reluctance to continue you in your post, for somehow I have perceived a sort of coolness between you.> 

 <And what was his reply?> 

 <That he certainly did think he had given you offence in an affair which he merely referred to without entering into particulars, but that whoever possessed the good opinion and confidence of the ship's owners would have his preference also.> 

 <The hypocrite!> murmured Danglars. 

 <Poor Dantès!> said Caderousse. <No one can deny his being a noble-hearted young fellow.> 

 <But meanwhile,> continued M. Morrel, <here is the \textit{Pharaon} without a captain.> 

 <Oh,> replied Danglars, <since we cannot leave this port for the next three months, let us hope that ere the expiration of that period Dantès will be set at liberty.> 

 <No doubt; but in the meantime?> 

 <I am entirely at your service, M. Morrel,> answered Danglars. <You know that I am as capable of managing a ship as the most experienced captain in the service; and it will be so far advantageous to you to accept my services, that upon Edmond's release from prison no further change will be requisite on board the \textit{Pharaon} than for Dantès and myself each to resume our respective posts.> 

 <Thanks, Danglars—that will smooth over all difficulties. I fully authorize you at once to assume the command of the \textit{Pharaon}, and look carefully to the unloading of her freight. Private misfortunes must never be allowed to interfere with business.> 

 <Be easy on that score, M. Morrel; but do you think we shall be permitted to see our poor Edmond?> 

 <I will let you know that directly I have seen M. de Villefort, whom I shall endeavour to interest in Edmond's favour. I am aware he is a furious royalist; but, in spite of that, and of his being king's attorney, he is a man like ourselves, and I fancy not a bad sort of one.> 

 <Perhaps not,> replied Danglars; <but I hear that he is ambitious, and that's rather against him.> 

 <Well, well,> returned M. Morrel, <we shall see. But now hasten on board, I will join you there ere long.> 

 So saying, the worthy shipowner quitted the two allies, and proceeded in the direction of the Palais de Justice.  <You see,> said Danglars, addressing Caderousse, <the turn things have taken. Do you still feel any desire to stand up in his defence?> 

 <Not the slightest, but yet it seems to me a shocking thing that a mere joke should lead to such consequences.> 

 <But who perpetrated that joke, let me ask? neither you nor myself, but Fernand; you knew very well that I threw the paper into a corner of the room—indeed, I fancied I had destroyed it.> 

 <Oh, no,> replied Caderousse, <that I can answer for, you did not. I only wish I could see it now as plainly as I saw it lying all crushed and crumpled in a corner of the arbour.> 

 <Well, then, if you did, depend upon it, Fernand picked it up, and either copied it or caused it to be copied; perhaps, even, he did not take the trouble of recopying it. And now I think of it, by Heavens, he may have sent the letter itself! Fortunately, for me, the handwriting was disguised.> 

 <Then you were aware of Dantès being engaged in a conspiracy?> 

 <Not \textsc{i.} As I before said, I thought the whole thing was a joke, nothing more. It seems, however, that I have unconsciously stumbled upon the truth.> 

 <Still,> argued Caderousse, <I would give a great deal if nothing of the kind had happened; or, at least, that I had had no hand in it. You will see, Danglars, that it will turn out an unlucky job for both of us.> 

 <Nonsense! If any harm come of it, it should fall on the guilty person; and that, you know, is Fernand. How can we be implicated in any way? All we have got to do is, to keep our own counsel, and remain perfectly quiet, not breathing a word to any living soul; and you will see that the storm will pass away without in the least affecting us.> 

 <Amen!> responded Caderousse, waving his hand in token of adieu to Danglars, and bending his steps towards the Allées de Meilhan, moving his head to and fro, and muttering as he went, after the manner of one whose mind was overcharged with one absorbing idea. 

 <So far, then,> said Danglars, mentally, <all has gone as I would have it. I am, temporarily, commander of the \textit{Pharaon}, with the certainty of being permanently so, if that fool of a Caderousse can be persuaded to hold his tongue. My only fear is the chance of Dantès being released. But, there, he is in the hands of Justice; and,> added he with a smile, <she will take her own.> So saying, he leaped into a boat, desiring to be rowed on board the \textit{Pharaon}, where M. Morrel had agreed to meet him. 