\chapter{The Ball} 

 \lettrine{I}{t} was in the warmest days of July, when in due course of time the Saturday arrived upon which the ball was to take place at M. de Morcerf's. It was ten o'clock at night; the branches of the great trees in the garden of the count's house stood out boldly against the azure canopy of heaven, which was studded with golden stars, but where the last fleeting clouds of a vanishing storm yet lingered. 

 From the apartments on the ground floor might be heard the sound of music, with the whirl of the waltz and galop, while brilliant streams of light shone through the openings of the Venetian blinds. At this moment the garden was only occupied by about ten servants, who had just received orders from their mistress to prepare the supper, the serenity of the weather continuing to increase. Until now, it had been undecided whether the supper should take place in the dining-room, or under a long tent erected on the lawn, but the beautiful blue sky, studded with stars, had settled the question in favour of the lawn. 

 The gardens were illuminated with coloured lanterns, according to the Italian custom, and, as is usual in countries where the luxuries of the table—the rarest of all luxuries in their complete form—are well understood, the supper-table was loaded with wax-lights and flowers.  At the time the Countess of Morcerf returned to the rooms, after giving her orders, many guests were arriving, more attracted by the charming hospitality of the countess than by the distinguished position of the count; for, owing to the good taste of Mercédès, one was sure of finding some devices at her entertainment worthy of describing, or even copying in case of need. 

 Madame Danglars, in whom the events we have related had caused deep anxiety, had hesitated about going to Madame de Morcerf's, when during the morning her carriage happened to meet that of Villefort. The latter made a sign, and when the carriages had drawn close together, said: 

 <You are going to Madame de Morcerf's, are you not?> 

 <No,> replied Madame Danglars, <I am too ill.> 

 <You are wrong,> replied Villefort, significantly; <it is important that you should be seen there.> 

 <Do you think so?> asked the baroness. 

 <I do.> 

 <In that case I will go.> 

 And the two carriages passed on towards their different destinations. Madame Danglars therefore came, not only beautiful in person, but radiant with splendour; she entered by one door at the time when Mercédès appeared at the door. The countess took Albert to meet Madame Danglars. He approached, paid her some well merited compliments on her toilet, and offered his arm to conduct her to a seat. Albert looked around him. 

 <You are looking for my daughter?> said the baroness, smiling. 

 <I confess it,> replied Albert. <Could you have been so cruel as not to bring her?> 

 <Calm yourself. She has met Mademoiselle de Villefort, and has taken her arm; see, they are following us, both in white dresses, one with a bouquet of camellias, the other with one of myosotis. But tell me\longdash> 

 <Well, what do you wish to know?> 

 <Will not the Count of Monte Cristo be here tonight?> 

 <Seventeen!> replied Albert. 

 <What do you mean?> 

 <I only mean that the count seems the rage,> replied the viscount, smiling, <and that you are the seventeenth person that has asked me the same question. The count is in fashion; I congratulate him upon it.> 

 <And have you replied to everyone as you have to me?> 

 <Ah, to be sure, I have not answered you; be satisfied, we shall have this <lion>; we are among the privileged ones.> 

 <Were you at the Opera yesterday?> 

 <No.> 

 <He was there.> 

 <Ah, indeed? And did the eccentric person commit any new originality?> 

 <Can he be seen without doing so? Elssler was dancing in \textit{Le Diable boiteux}; the Greek princess was in ecstasies. After the cachucha he placed a magnificent ring on the stem of a bouquet, and threw it to the charming danseuse, who, in the third act, to do honour to the gift, reappeared with it on her finger. And the Greek princess,—will she be here?> 

 <No, you will be deprived of that pleasure; her position in the count's establishment is not sufficiently understood.> 

 <Wait; leave me here, and go and speak to Madame de Villefort, who is trying to attract your attention.> 

 Albert bowed to Madame Danglars, and advanced towards Madame de Villefort, whose lips opened as he approached. 

 <I wager anything,> said Albert, interrupting her, <that I know what you were about to say.> 

 <Well, what is it?> 

 <If I guess rightly, will you confess it?> 

 <Yes.> 

 <On your honour?> 

 <On my honour.> 

 <You were going to ask me if the Count of Monte Cristo had arrived, or was expected.> 

 <Not at all. It is not of him that I am now thinking. I was going to ask you if you had received any news of Monsieur Franz.> 

 <Yes,—yesterday.> 

 <What did he tell you?> 

 <That he was leaving at the same time as his letter.> 

 <Well, now then, the count?> 

 <The count will come, of that you may be satisfied.> 

 <You know that he has another name besides Monte Cristo?> 

 <No, I did not know it.> 

 <Monte Cristo is the name of an island, and he has a family name.> 

 <I never heard it.> 

 <Well, then, I am better informed than you; his name is Zaccone.> 

 <It is possible.> 

 <He is a Maltese.> 

 “That is also possible. 

 <The son of a shipowner.> 

 <Really, you should relate all this aloud, you would have the greatest success.> 

 <He served in India, discovered a mine in Thessaly, and comes to Paris to establish a mineral water-cure at Auteuil.> 

 <Well, I'm sure,> said Morcerf, <this is indeed news! Am I allowed to repeat it?> 

 <Yes, but cautiously, tell one thing at a time, and do not say I told you.> 

 <Why so?> 

 <Because it is a secret just discovered.> 

 <By whom?> 

 <The police.> 

 <Then the news originated\longdash> 

 <At the prefect's last night. Paris, you can understand, is astonished at the sight of such unusual splendour, and the police have made inquiries.> 

 <Well, well! Nothing more is wanting than to arrest the count as a vagabond, on the pretext of his being too rich.> 

 <Indeed, that doubtless would have happened if his credentials had not been so favourable.> 

 <Poor count! And is he aware of the danger he has been in?> 

 <I think not.> 

 <Then it will be but charitable to inform him. When he arrives, I will not fail to do so.> 

 Just then, a handsome young man, with bright eyes, black hair, and glossy moustache, respectfully bowed to Madame de Villefort. Albert extended his hand. 

 <Madame,> said Albert, <allow me to present to you M. Maximilian Morrel, captain of Spahis, one of our best, and, above all, of our bravest officers.> 

 <I have already had the pleasure of meeting this gentleman at Auteuil, at the house of the Count of Monte Cristo,> replied Madame de Villefort, turning away with marked coldness of manner. 

 This answer, and especially the tone in which it was uttered, chilled the heart of poor Morrel. But a recompense was in store for him; turning around, he saw near the door a beautiful fair face, whose large blue eyes were, without any marked expression, fixed upon him, while the bouquet of myosotis was gently raised to her lips. 

 The salutation was so well understood that Morrel, with the same expression in his eyes, placed his handkerchief to his mouth; and these two living statues, whose hearts beat so violently under their marble aspect, separated from each other by the whole length of the room, forgot themselves for a moment, or rather forgot the world in their mutual contemplation. They might have remained much longer lost in one another, without anyone noticing their abstraction. The Count of Monte Cristo had just entered. 

 We have already said that there was something in the count which attracted universal attention wherever he appeared. It was not the coat, unexceptional in its cut, though simple and unornamented; it was not the plain white waistcoat; it was not the trousers, that displayed the foot so perfectly formed—it was none of these things that attracted the attention,—it was his pale complexion, his waving black hair, his calm and serene expression, his dark and melancholy eye, his mouth, chiselled with such marvellous delicacy, which so easily expressed such high disdain,—these were what fixed the attention of all upon him. 

 Many men might have been handsomer, but certainly there could be none whose appearance was more \textit{significant}, if the expression may be used. Everything about the count seemed to have its meaning, for the constant habit of thought which he had acquired had given an ease and vigour to the expression of his face, and even to the most trifling gesture, scarcely to be understood. Yet the Parisian world is so strange, that even all this might not have won attention had there not been connected with it a mysterious story gilded by an immense fortune.  Meanwhile he advanced through the assemblage of guests under a battery of curious glances towards Madame de Morcerf, who, standing before a mantle-piece ornamented with flowers, had seen his entrance in a looking-glass placed opposite the door, and was prepared to receive him. She turned towards him with a serene smile just at the moment he was bowing to her. No doubt she fancied the count would speak to her, while on his side the count thought she was about to address him; but both remained silent, and after a mere bow, Monte Cristo directed his steps to Albert, who received him cordially. 

 <Have you seen my mother?> asked Albert. 

 <I have just had the pleasure,> replied the count; <but I have not seen your father.> 

 <See, he is down there, talking politics with that little group of great geniuses.> 

 <Indeed?> said Monte Cristo; <and so those gentlemen down there are men of great talent. I should not have guessed it. And for what kind of talent are they celebrated? You know there are different sorts.> 

 <That tall, harsh-looking man is very learned, he discovered, in the neighbourhood of Rome, a kind of lizard with a vertebra more than lizards usually have, and he immediately laid his discovery before the Institute. The thing was discussed for a long time, but finally decided in his favour. I can assure you the vertebra made a great noise in the learned world, and the gentleman, who was only a knight of the Legion of honour, was made an officer.> 

 <Come,> said Monte Cristo, <this cross seems to me to be wisely awarded. I suppose, had he found another additional vertebra, they would have made him a commander.> 

 <Very likely,> said Albert. 

 <And who can that person be who has taken it into his head to wrap himself up in a blue coat embroidered with green?> 

 <Oh, that coat is not his own idea; it is the Republic's, which deputed David\footnote{Jacques-Louis David, a famous French painter (1748-1825). } to devise a uniform for the Academicians.> 

 <Indeed?> said Monte Cristo; <so this gentleman is an Academician?> 

 <Within the last week he has been made one of the learned assembly.> 

 <And what is his especial talent?> 

 <His talent? I believe he thrusts pins through the heads of rabbits, he makes fowls eat madder, and punches the spinal marrow out of dogs with whalebone.> 

 <And he is made a member of the Academy of Sciences for this?> 

 <No; of the French Academy.> 

 <But what has the French Academy to do with all this?> 

 <I was going to tell you. It seems\longdash> 

 <That his experiments have very considerably advanced the cause of science, doubtless?> 

 <No; that his style of writing is very good.> 

 <This must be very flattering to the feelings of the rabbits into whose heads he has thrust pins, to the fowls whose bones he has dyed red, and to the dogs whose spinal marrow he has punched out?> 

 Albert laughed. 

 <And the other one?> demanded the count. 

 <That one?> 

 <Yes, the third.> 

 <The one in the dark blue coat?> 

 <Yes.> 

 <He is a colleague of the count, and one of the most active opponents to the idea of providing the Chamber of Peers with a uniform. He was very successful upon that question. He stood badly with the Liberal papers, but his noble opposition to the wishes of the court is now getting him into favour with the journalists. They talk of making him an ambassador.>  <And what are his claims to the peerage?> 

 <He has composed two or three comic operas, written four or five articles in the \textit{Siècle}, and voted five or six years on the ministerial side.> 

 <Bravo, viscount,> said Monte Cristo, smiling; <you are a delightful \textit{cicerone}. And now you will do me a favour, will you not?> 

 <What is it?> 

 <Do not introduce me to any of these gentlemen; and should they wish it, you will warn me.> Just then the count felt his arm pressed. He turned round; it was Danglars. 

 <Ah! is it you, baron?> said he. 

 <Why do you call me baron?> said Danglars; <you know that I care nothing for my title. I am not like you, viscount; you like your title, do you not?> 

 <Certainly,> replied Albert, <seeing that without my title I should be nothing; while you, sacrificing the baron, would still remain the millionaire.> 

 <Which seems to me the finest title under the royalty of July,> replied Danglars. 

 <Unfortunately,> said Monte Cristo, <one's title to a millionaire does not last for life, like that of baron, peer of France, or academician; for example, the millionaires Franck \& Poulmann, of Frankfurt, who have just become bankrupts.> 

 <Indeed?> said Danglars, becoming pale. 

 <Yes; I received the news this evening by a courier. I had about a million in their hands, but, warned in time, I withdrew it a month ago.> 

 <Ah, \textit{mon Dieu!}> exclaimed Danglars, <they have drawn on me for 200,000 francs!> 

 <Well, you can throw out the draft; their signature is worth five per cent.> 

 <Yes, but it is too late,> said Danglars, <I have honoured their bills.> 

 <Then,> said Monte Cristo, <here are 200,000 francs gone after\longdash> 

 <Hush, do not mention these things,> said Danglars; then, approaching Monte Cristo, he added, <especially before young M. Cavalcanti;> after which he smiled, and turned towards the young man in question. 

 Albert had left the count to speak to his mother, Danglars to converse with young Cavalcanti; Monte Cristo was for an instant alone. Meanwhile the heat became excessive. The footmen were hastening through the rooms with waiters loaded with ices. Monte Cristo wiped the perspiration from his forehead, but drew back when the waiter was presented to him; he took no refreshment. Madame de Morcerf did not lose sight of Monte Cristo; she saw that he took nothing, and even noticed his gesture of refusal. 

 <Albert,> she asked, <did you notice that?> 

 <What, mother?> 

 <That the count has never been willing to partake of food under the roof of M. de Morcerf.> 

 <Yes; but then he breakfasted with me—indeed, he made his first appearance in the world on that occasion.> 

 <But your house is not M. de Morcerf's,> murmured Mercédès; <and since he has been here I have watched him.> 

 <Well?> 

 <Well, he has taken nothing yet.> 

 <The count is very temperate.> 

 Mercédès smiled sadly. 

 <Approach him,> said she, <and when the next waiter passes, insist upon his taking something.> 

 <But why, mother?> 

 <Just to please me, Albert,> said Mercédès. Albert kissed his mother's hand, and drew near the count. Another salver passed, loaded like the preceding ones; she saw Albert attempt to persuade the count, but he obstinately refused. Albert rejoined his mother; she was very pale. 

 <Well,> said she, <you see he refuses?> 

 <Yes; but why need this annoy you?> 

 <You know, Albert, women are singular creatures. I should like to have seen the count take something in my house, if only an ice. Perhaps he cannot reconcile himself to the French style of living, and might prefer something else.> 

 <Oh, no; I have seen him eat of everything in Italy; no doubt he does not feel inclined this evening.> 

 <And besides,> said the countess, <accustomed as he is to burning climates, possibly he does not feel the heat as we do.> 

 <I do not think that, for he has complained of feeling almost suffocated, and asked why the Venetian blinds were not opened as well as the windows.> 

 <In a word,> said Mercédès, <it was a way of assuring me that his abstinence was intended.> 

 And she left the room. 

 A minute afterwards the blinds were thrown open, and through the jessamine and clematis that overhung the window one could see the garden ornamented with lanterns, and the supper laid under the tent. Dancers, players, talkers, all uttered an exclamation of joy—everyone inhaled with delight the breeze that floated in. At the same time Mercédès reappeared, paler than before, but with that imperturbable expression of countenance which she sometimes wore. She went straight to the group of which her husband formed the centre. 

 <Do not detain those gentlemen here, count,> she said; <they would prefer, I should think, to breathe in the garden rather than suffocate here, since they are not playing.> 

 <Ah,> said a gallant old general, who, in 1809, had sung \textit{Partant pour la Syrie},—<we will not go alone to the garden.> 

 <Then,> said Mercédès, <I will lead the way.> 

 Turning towards Monte Cristo, she added, <count, will you oblige me with your arm?> 

 The count almost staggered at these simple words; then he fixed his eyes on Mercédès. It was only a momentary glance, but it seemed to the countess to have lasted for a century, so much was expressed in that one look. He offered his arm to the countess; she took it, or rather just touched it with her little hand, and they together descended the steps, lined with rhododendrons and camellias. Behind them, by another outlet, a group of about twenty persons rushed into the garden with loud exclamations of delight. 