\chapter{Ideology} 

 \lettrine{I}{f} the Count of Monte Cristo had been for a long time familiar with the ways of Parisian society, he would have appreciated better the significance of the step which M. de Villefort had taken. Standing well at court, whether the king regnant was of the older or younger branch, whether the government was doctrinaire liberal, or conservative; looked upon by all as a man of talent, since those who have never experienced a political check are generally so regarded; hated by many, but warmly supported by others, without being really liked by anybody, M. de Villefort held a high position in the magistracy, and maintained his eminence like a Harlay or a Molé. His drawing-room, under the regenerating influence of a young wife and a daughter by his first marriage, scarcely eighteen, was still one of the well-regulated Paris salons where the worship of traditional customs and the observance of rigid etiquette were carefully maintained. A freezing politeness, a strict fidelity to government principles, a profound contempt for theories and theorists, a deep-seated hatred of ideality,—these were the elements of private and public life displayed by M. de Villefort.  M. de Villefort was not only a magistrate, he was almost a diplomatist. His relations with the former court, of which he always spoke with dignity and respect, made him respected by the new one, and he knew so many things, that not only was he always carefully considered, but sometimes consulted. Perhaps this would not have been so had it been possible to get rid of M. de Villefort; but, like the feudal barons who rebelled against their sovereign, he dwelt in an impregnable fortress. This fortress was his post as king's attorney, all the advantages of which he exploited with marvellous skill, and which he would not have resigned but to be made deputy, and thus to replace neutrality by opposition. 

 Ordinarily M. de Villefort made and returned very few visits. His wife visited for him, and this was the received thing in the world, where the weighty and multifarious occupations of the magistrate were accepted as an excuse for what was really only calculated pride, a manifestation of professed superiority—in fact, the application of the axiom, \textit{Pretend to think well of yourself, and the world will think well of you}, an axiom a hundred times more useful in society nowadays than that of the Greeks, <Know thyself,> a knowledge for which, in our days, we have substituted the less difficult and more advantageous science of \textit{knowing others}. 

 To his friends M. de Villefort was a powerful protector; to his enemies, he was a silent, but bitter opponent; for those who were neither the one nor the other, he was a statue of the law-made man. He had a haughty bearing, a look either steady and impenetrable or insolently piercing and inquisitorial. Four successive revolutions had built and cemented the pedestal upon which his fortune was based. 

 M. de Villefort had the reputation of being the least curious and the least wearisome man in France. He gave a ball every year, at which he appeared for a quarter of an hour only,—that is to say, five-and-forty minutes less than the king is visible at his balls. He was never seen at the theatres, at concerts, or in any place of public resort. Occasionally, but seldom, he played at whist, and then care was taken to select partners worthy of him—sometimes they were ambassadors, sometimes archbishops, or sometimes a prince, or a president, or some dowager duchess. 

 Such was the man whose carriage had just now stopped before the Count of Monte Cristo's door. The valet de chambre announced M. de Villefort at the moment when the count, leaning over a large table, was tracing on a map the route from St. Petersburg to China. 

 The procureur entered with the same grave and measured step he would have employed in entering a court of justice. He was the same man, or rather the development of the same man, whom we have heretofore seen as assistant attorney at Marseilles. Nature, according to her way, had made no deviation in the path he had marked out for himself. From being slender he had now become meagre; once pale, he was now yellow; his deep-set eyes were hollow, and the gold spectacles shielding his eyes seemed to be an integral portion of his face. He dressed entirely in black, with the exception of his white tie, and his funeral appearance was only mitigated by the slight line of red ribbon which passed almost imperceptibly through his button-hole, and appeared like a streak of blood traced with a delicate brush. 

 Although master of himself, Monte Cristo, scrutinized with irrepressible curiosity the magistrate whose salute he returned, and who, distrustful by habit, and especially incredulous as to social prodigies, was much more despised to look upon <the noble stranger,> as Monte Cristo was already called, as an adventurer in search of new fields, or an escaped criminal, rather than as a prince of the Holy See, or a sultan of the \textit{Thousand and One Nights}. 

 <Sir,> said Villefort, in the squeaky tone assumed by magistrates in their oratorical periods, and of which they cannot, or will not, divest themselves in society, <sir, the signal service which you yesterday rendered to my wife and son has made it a duty for me to offer you my thanks. I have come, therefore, to discharge this duty, and to express to you my overwhelming gratitude.> 

 And as he said this, the <eye severe> of the magistrate had lost nothing of its habitual arrogance. He spoke in a voice of the procureur-general, with the rigid inflexibility of neck and shoulders which caused his flatterers to say (as we have before observed) that he was the living statue of the law. 

 <Monsieur,> replied the count, with a chilling air, <I am very happy to have been the means of preserving a son to his mother, for they say that the sentiment of maternity is the most holy of all; and the good fortune which occurred to me, monsieur, might have enabled you to dispense with a duty which, in its discharge, confers an undoubtedly great honour; for I am aware that M. de Villefort is not usually lavish of the favour which he now bestows on me,—a favour which, however estimable, is unequal to the satisfaction which I have in my own consciousness.> 

 Villefort, astonished at this reply, which he by no means expected, started like a soldier who feels the blow levelled at him over the armor he wears, and a curl of his disdainful lip indicated that from that moment he noted in the tablets of his brain that the Count of Monte Cristo was by no means a highly bred gentleman. 

 He glanced around, in order to seize on something on which the conversation might turn, and seemed to fall easily on a topic. He saw the map which Monte Cristo had been examining when he entered, and said: 

 <You seem geographically engaged, sir? It is a rich study for you, who, as I learn, have seen as many lands as are delineated on this map.> 

 <Yes, sir,> replied the count; <I have sought to make of the human race, taken in the mass, what you practice every day on individuals—a physiological study. I have believed it was much easier to descend from the whole to a part than to ascend from a part to the whole. It is an algebraic axiom, which makes us proceed from a known to an unknown quantity, and not from an unknown to a known; but sit down, sir, I beg of you.> 

 Monte Cristo pointed to a chair, which the procureur was obliged to take the trouble to move forwards himself, while the count merely fell back into his own, on which he had been kneeling when M. Villefort entered. Thus the count was halfway turned towards his visitor, having his back towards the window, his elbow resting on the geographical chart which furnished the theme of conversation for the moment,—a conversation which assumed, as in the case of the interviews with Danglars and Morcerf, a turn analogous to the persons, if not to the situation. 

 <Ah, you philosophize,> replied Villefort, after a moment's silence, during which, like a wrestler who encounters a powerful opponent, he took breath; <well, sir, really, if, like you, I had nothing else to do, I should seek a more amusing occupation.> 

 <Why, in truth, sir,> was Monte Cristo's reply, <man is but an ugly caterpillar for him who studies him through a solar microscope; but you said, I think, that I had nothing else to do. Now, really, let me ask, sir, have you?—do you believe you have anything to do? or to speak in plain terms, do you really think that what you do deserves being called anything?> 

 Villefort's astonishment redoubled at this second thrust so forcibly made by his strange adversary. It was a long time since the magistrate had heard a paradox so strong, or rather, to say the truth more exactly, it was the first time he had ever heard of it. The procureur exerted himself to reply. 

 <Sir,> he responded, <you are a stranger, and I believe you say yourself that a portion of your life has been spent in Oriental countries, so you are not aware how human justice, so expeditious in barbarous countries, takes with us a prudent and well-studied course.> 

 <Oh, yes—yes, I do, sir; it is the \textit{pede claudo} of the ancients. I know all that, for it is with the justice of all countries especially that I have occupied myself—it is with the criminal procedure of all nations that I have compared natural justice, and I must say, sir, that it is the law of primitive nations, that is, the law of retaliation, that I have most frequently found to be according to the law of God.> 

 <If this law were adopted, sir,> said the procureur, <it would greatly simplify our legal codes, and in that case the magistrates would not (as you just observed) have much to do.> 

 <It may, perhaps, come to this in time,> observed Monte Cristo; <you know that human inventions march from the complex to the simple, and simplicity is always perfection.> 

 <In the meanwhile,> continued the magistrate, <our codes are in full force, with all their contradictory enactments derived from Gallic customs, Roman laws, and Frank usages; the knowledge of all which, you will agree, is not to be acquired without extended labour; it needs tedious study to acquire this knowledge, and, when acquired, a strong power of brain to retain it.> 

 <I agree with you entirely, sir; but all that even you know with respect to the French code, I know, not only in reference to that code, but as regards the codes of all nations. The English, Turkish, Japanese, Hindu laws, are as familiar to me as the French laws, and thus I was right, when I said to you, that relatively (you know that everything is relative, sir)—that relatively to what I have done, you have very little to do; but that relatively to all I have learned, you have yet a great deal to learn.> 

 <But with what motive have you learned all this?> inquired Villefort, in astonishment. 

 Monte Cristo smiled. 

 <Really, sir,> he observed, <I see that in spite of the reputation which you have acquired as a superior man, you look at everything from the material and vulgar view of society, beginning with man, and ending with man—that is to say, in the most restricted, most narrow view which it is possible for human understanding to embrace.> 

 <Pray, sir, explain yourself,> said Villefort, more and more astonished, <I really do—not—understand you—perfectly.> 

 <I say, sir, that with the eyes fixed on the social organization of nations, you see only the springs of the machine, and lose sight of the sublime workman who makes them act; I say that you do not recognize before you and around you any but those office-holders whose commissions have been signed by a minister or king; and that the men whom God has put above those office-holders, ministers, and kings, by giving them a mission to follow out, instead of a post to fill—I say that they escape your narrow, limited field of observation. It is thus that human weakness fails, from its debilitated and imperfect organs. Tobias took the angel who restored him to light for an ordinary young man. The nations took Attila, who was doomed to destroy them, for a conqueror similar to other conquerors, and it was necessary for both to reveal their missions, that they might be known and acknowledged; one was compelled to say, <I am the angel of the Lord>; and the other, <I am the hammer of God,> in order that the divine essence in both might be revealed.> 

 <Then,> said Villefort, more and more amazed, and really supposing he was speaking to a mystic or a madman, <you consider yourself as one of those extraordinary beings whom you have mentioned?> 

 <And why not?> said Monte Cristo coldly. 

 <Your pardon, sir,> replied Villefort, quite astounded, <but you will excuse me if, when I presented myself to you, I was unaware that I should meet with a person whose knowledge and understanding so far surpass the usual knowledge and understanding of men. It is not usual with us corrupted wretches of civilization to find gentlemen like yourself, possessors, as you are, of immense fortune—at least, so it is said—and I beg you to observe that I do not inquire, I merely repeat;—it is not usual, I say, for such privileged and wealthy beings to waste their time in speculations on the state of society, in philosophical reveries, intended at best to console those whom fate has disinherited from the goods of this world.> 

 <Really, sir,> retorted the count, <have you attained the eminent situation in which you are, without having admitted, or even without having met with exceptions? and do you never use your eyes, which must have acquired so much \textit{finesse} and certainty, to divine, at a glance, the kind of man by whom you are confronted? Should not a magistrate be not merely the best administrator of the law, but the most crafty expounder of the chicanery of his profession, a steel probe to search hearts, a touchstone to try the gold which in each soul is mingled with more or less of alloy?> 

 <Sir,> said Villefort, <upon my word, you overcome me. I really never heard a person speak as you do.> 

 <Because you remain eternally encircled in a round of general conditions, and have never dared to raise your wings into those upper spheres which God has peopled with invisible or exceptional beings.> 

 <And you allow then, sir, that spheres exist, and that these marked and invisible beings mingle amongst us?> 

 <Why should they not? Can you see the air you breathe, and yet without which you could not for a moment exist?> 

 <Then we do not see those beings to whom you allude?> 

 <Yes, we do; you see them whenever God pleases to allow them to assume a material form. You touch them, come in contact with them, speak to them, and they reply to you.> 

 <Ah,> said Villefort, smiling, <I confess I should like to be warned when one of these beings is in contact with me.> 

 <You have been served as you desire, monsieur, for you were warned just now, and I now again warn you.> 

 <Then you yourself are one of these marked beings?> 

 <Yes, monsieur, I believe so; for until now, no man has found himself in a position similar to mine. The dominions of kings are limited either by mountains or rivers, or a change of manners, or an alteration of language. My kingdom is bounded only by the world, for I am not an Italian, or a Frenchman, or a Hindu, or an American, or a Spaniard—I am a cosmopolite. No country can say it saw my birth. God alone knows what country will see me die. I adopt all customs, speak all languages. You believe me to be a Frenchman, for I speak French with the same facility and purity as yourself. Well, Ali, my Nubian, believes me to be an Arab; Bertuccio, my steward, takes me for a Roman; Haydée, my slave, thinks me a Greek. You may, therefore, comprehend, that being of no country, asking no protection from any government, acknowledging no man as my brother, not one of the scruples that arrest the powerful, or the obstacles which paralyse the weak, paralyses or arrests me. I have only two adversaries—I will not say two conquerors, for with perseverance I subdue even them,—they are time and distance. There is a third, and the most terrible—that is my condition as a mortal being. This alone can stop me in my onward career, before I have attained the goal at which I aim, for all the rest I have reduced to mathematical terms. What men call the chances of fate—namely, ruin, change, circumstances—I have fully anticipated, and if any of these should overtake me, yet it will not overwhelm me. Unless I die, I shall always be what I am, and therefore it is that I utter the things you have never heard, even from the mouths of kings—for kings have need, and other persons have fear of you. For who is there who does not say to himself, in a society as incongruously organized as ours, <Perhaps some day I shall have to do with the king's attorney>?> 

 <But can you not say that, sir? The moment you become an inhabitant of France, you are naturally subjected to the French law.> 

 <I know it sir,> replied Monte Cristo; <but when I visit a country I begin to study, by all the means which are available, the men from whom I may have anything to hope or to fear, till I know them as well as, perhaps better than, they know themselves. It follows from this, that the king's attorney, be he who he may, with whom I should have to deal, would assuredly be more embarrassed than I should.> 

 <That is to say,> replied Villefort with hesitation, <that human nature being weak, every man, according to your creed, has committed faults.> 

 <Faults or crimes,> responded Monte Cristo with a negligent air. 

 <And that you alone, amongst the men whom you do not recognize as your brothers—for you have said so,> observed Villefort in a tone that faltered somewhat—<you alone are perfect.> 

 <No, not perfect,> was the count's reply; <only impenetrable, that's all. But let us leave off this strain, sir, if the tone of it is displeasing to you; I am no more disturbed by your justice than are you by my second-sight.> 

 <No, no,—by no means,> said Villefort, who was afraid of seeming to abandon his ground. <No; by your brilliant and almost sublime conversation you have elevated me above the ordinary level; we no longer talk, we rise to dissertation. But you know how the theologians in their collegiate chairs, and philosophers in their controversies, occasionally say cruel truths; let us suppose for the moment that we are theologizing in a social way, or even philosophically, and I will say to you, rude as it may seem, <My brother, you sacrifice greatly to pride; you may be above others, but above you there is God.>>  <Above us all, sir,> was Monte Cristo's response, in a tone and with an emphasis so deep that Villefort involuntarily shuddered. <I have my pride for men—serpents always ready to threaten everyone who would pass without crushing them under foot. But I lay aside that pride before God, who has taken me from nothing to make me what I am.> 

 <Then, count, I admire you,> said Villefort, who, for the first time in this strange conversation, used the aristocratic form to the unknown personage, whom, until now, he had only called monsieur. <Yes, and I say to you, if you are really strong, really superior, really pious, or impenetrable, which you were right in saying amounts to the same thing—then be proud, sir, for that is the characteristic of predominance. Yet you have unquestionably some ambition.> 

 <I have, sir.> 

 <And what may it be?> 

 <I too, as happens to every man once in his life, have been taken by Satan into the highest mountain in the earth, and when there he showed me all the kingdoms of the world, and as he said before, so said he to me, <Child of earth, what wouldst thou have to make thee adore me?> I reflected long, for a gnawing ambition had long preyed upon me, and then I replied, <Listen,—I have always heard of Providence, and yet I have never seen him, or anything that resembles him, or which can make me believe that he exists. I wish to be Providence myself, for I feel that the most beautiful, noblest, most sublime thing in the world, is to recompense and punish.> Satan bowed his head, and groaned. <You mistake,> he said, <Providence does exist, only you have never seen him, because the child of God is as invisible as the parent. You have seen nothing that resembles him, because he works by secret springs, and moves by hidden ways. All I can do for you is to make you one of the agents of that Providence.> The bargain was concluded. I may sacrifice my soul, but what matters it?> added Monte Cristo. <If the thing were to do again, I would again do it.> 

 Villefort looked at Monte Cristo with extreme amazement. 

 <Count,> he inquired, <have you any relations?> 

 <No, sir, I am alone in the world.> 

 <So much the worse.> 

 <Why?> asked Monte Cristo. 

 <Because then you might witness a spectacle calculated to break down your pride. You say you fear nothing but death?> 

 <I did not say that I feared it; I only said that death alone could check the execution of my plans.> 

 <And old age?> 

 <My end will be achieved before I grow old.> 

 <And madness?> 

 <I have been nearly mad; and you know the axiom,—\textit{non bis in idem}. It is an axiom of criminal law, and, consequently, you understand its full application.>  
 
 <Sir,> continued Villefort, <there is something to fear besides death, old age, and madness. For instance, there is apoplexy—that lightning-stroke which strikes but does not destroy you, and yet which brings everything to an end. You are still yourself as now, and yet you are yourself no longer; you who, like Ariel, verge on the angelic, are but an inert mass, which, like Caliban, verges on the brutal; and this is called in human tongues, as I tell you, neither more nor less than apoplexy. Come, if so you will, count, and continue this conversation at my house, any day you may be willing to see an adversary capable of understanding and anxious to refute you, and I will show you my father, M. Noirtier de Villefort, one of the most fiery Jacobins of the French Revolution; that is to say, he had the most remarkable audacity, seconded by a most powerful organization—a man who has not, perhaps, like yourself seen all the kingdoms of the earth, but who has helped to overturn one of the greatest; in fact, a man who believed himself, like you, one of the envoys, not of God, but of a supreme being; not of Providence, but of fate. Well, sir, the rupture of a blood-vessel on the lobe of the brain has destroyed all this, not in a day, not in an hour, but in a second. M. Noirtier, who, on the previous night, was the old Jacobin, the old senator, the old Carbonaro, laughing at the guillotine, the cannon, and the dagger—M. Noirtier, playing with revolutions—M. Noirtier, for whom France was a vast chess-board, from which pawns, rooks, knights, and queens were to disappear, so that the king was checkmated—M. Noirtier, the redoubtable, was the next morning \textit{poor M. Noirtier}, the helpless old man, at the tender mercies of the weakest creature in the household, that is, his grandchild, Valentine; a dumb and frozen carcass, in fact, living painlessly on, that time may be given for his frame to decompose without his consciousness of its decay.> 

 <Alas, sir,> said Monte Cristo <this spectacle is neither strange to my eye nor my thought. I am something of a physician, and have, like my fellows, sought more than once for the soul in living and in dead matter; yet, like Providence, it has remained invisible to my eyes, although present to my heart. A hundred writers since Socrates, Seneca, St. Augustine, and Gall, have made, in verse and prose, the comparison you have made, and yet I can well understand that a father's sufferings may effect great changes in the mind of a son. I will call on you, sir, since you bid me contemplate, for the advantage of my pride, this terrible spectacle, which must have been so great a source of sorrow to your family.> 

 <It would have been so unquestionably, had not God given me so large a compensation. In contrast with the old man, who is dragging his way to the tomb, are two children just entering into life—Valentine, the daughter by my first wife—Mademoiselle Renée de Saint-Méran—and Edward, the boy whose life you have this day saved.> 

 <And what is your deduction from this compensation, sir?> inquired Monte Cristo. 

 <My deduction is,> replied Villefort, <that my father, led away by his passions, has committed some fault unknown to human justice, but marked by the justice of God. That God, desirous in his mercy to punish but one person, has visited this justice on him alone.> 

 Monte Cristo with a smile on his lips, uttered in the depths of his soul a groan which would have made Villefort fly had he but heard it. 

 <Adieu, sir,> said the magistrate, who had risen from his seat; <I leave you, bearing a remembrance of you—a remembrance of esteem, which I hope will not be disagreeable to you when you know me better; for I am not a man to bore my friends, as you will learn. Besides, you have made an eternal friend of Madame de Villefort.> 

 The count bowed, and contented himself with seeing Villefort to the door of his cabinet, the procureur being escorted to his carriage by two footmen, who, on a signal from their master, followed him with every mark of attention. When he had gone, Monte Cristo breathed a profound sigh, and said: 

 <Enough of this poison, let me now seek the antidote.> 

 Then sounding his bell, he said to Ali, who entered: 

 <I am going to madame's chamber—have the carriage ready at one o'clock.> 