\chapter{The Rain of Blood} 

 \lettrine[ante=']{A}{s} the jeweller returned to the apartment, he cast around him a scrutinizing glance—but there was nothing to excite suspicion, if it did not exist, or to confirm it, if it were already awakened. Caderousse's hands still grasped the gold and bank-notes, and La Carconte called up her sweetest smiles while welcoming the reappearance of their guest. 

 <Well, well,> said the jeweller, <you seem, my good friends, to have had some fears respecting the accuracy of your money, by counting it over so carefully directly I was gone.> 

 <Oh, no,> answered Caderousse, <that was not my reason, I can assure you; but the circumstances by which we have become possessed of this wealth are so unexpected, as to make us scarcely credit our good fortune, and it is only by placing the actual proof of our riches before our eyes that we can persuade ourselves that the whole affair is not a dream.> 

 The jeweller smiled. <Have you any other guests in your house?> inquired he. 

 <Nobody but ourselves,> replied Caderousse; <the fact is, we do not lodge travellers—indeed, our tavern is so near the town, that nobody would think of stopping here.> 

 <Then I am afraid I shall very much inconvenience you.> 

 <Inconvenience us? Not at all, my dear sir,> said La Carconte in her most gracious manner. <Not at all, I assure you.> 

 <But where will you manage to stow me?> 

 <In the chamber overhead.> 

<Surely that is where you yourselves sleep?> 

 <Never mind that; we have a second bed in the adjoining room.> 

 Caderousse stared at his wife with much astonishment. 

 The jeweller, meanwhile, was humming a song as he stood warming his back at the fire La Carconte had kindled to dry the wet garments of her guest; and this done, she next occupied herself in arranging his supper, by spreading a napkin at the end of the table, and placing on it the slender remains of their dinner, to which she added three or four fresh-laid eggs. Caderousse had once more parted with his treasure—the banknotes were replaced in the pocket-book, the gold put back into the bag, and the whole carefully locked in the cupboard. He then began pacing the room with a pensive and gloomy air, glancing from time to time at the jeweller, who stood reeking with the steam from his wet clothes, and merely changing his place on the warm hearth, to enable the whole of his garments to be dried. 

 <There,> said La Carconte, as she placed a bottle of wine on the table, <supper is ready whenever you are.> 

 <And you?> asked Joannes. 

 <I don't want any supper,> said Caderousse. 

 <We dined so very late,> hastily interposed La Carconte. 

 <Then it seems I am to eat alone,> remarked the jeweller. 

 <Oh, we shall have the pleasure of waiting upon you,> answered La Carconte, with an eager attention she was not accustomed to manifest even to guests who paid for what they took. 

 From time to time Caderousse darted on his wife keen, searching glances, but rapid as the lightning flash. The storm still continued. 

 <There, there,> said La Carconte; <do you hear that? upon my word, you did well to come back.> 

 <Nevertheless,> replied the jeweller, <if by the time I have finished my supper the tempest has at all abated, I shall make another start.> 

 <It's the mistral,> said Caderousse, <and it will be sure to last till tomorrow morning.> He sighed heavily. 

 <Well,> said the jeweller, as he placed himself at table, <all I can say is, so much the worse for those who are abroad.> 

 <Yes,> chimed in La Carconte, <they will have a wretched night of it.> 

 The jeweller began eating his supper, and the woman, who was ordinarily so querulous and indifferent to all who approached her, was suddenly transformed into the most smiling and attentive hostess. Had the unhappy man on whom she lavished her assiduities been previously acquainted with her, so sudden an alteration might well have excited suspicion in his mind, or at least have greatly astonished him. Caderousse, meanwhile, continued to pace the room in gloomy silence, sedulously avoiding the sight of his guest; but as soon as the stranger had completed his repast, the agitated innkeeper went eagerly to the door and opened it. 

 <I believe the storm is over,> said he. 

 But as if to contradict his statement, at that instant a violent clap of thunder seemed to shake the house to its very foundation, while a sudden gust of wind, mingled with rain, extinguished the lamp he held in his hand. 

 Trembling and awe-struck, Caderousse hastily shut the door and returned to his guest, while La Carconte lighted a candle by the smouldering ashes that glimmered on the hearth. 

 <You must be tired,> said she to the jeweller; <I have spread a pair of white sheets on your bed; go up when you are ready, and sleep well.> 

 Joannes stayed for a while to see whether the storm seemed to abate in its fury, but a brief space of time sufficed to assure him that, instead of diminishing, the violence of the rain and thunder momentarily increased; resigning himself, therefore, to what seemed inevitable, he bade his host good-night, and mounted the stairs. He passed over my head and I heard the flooring creak beneath his footsteps. The quick, eager glance of La Carconte followed him as he ascended, while Caderousse, on the contrary, turned his back, and seemed most anxiously to avoid even glancing at him. 

 All these circumstances did not strike me as painfully at the time as they have since done; in fact, all that had happened (with the exception of the story of the diamond, which certainly did wear an air of improbability), appeared natural enough, and called for neither apprehension nor mistrust; but, worn out as I was with fatigue, and fully purposing to proceed onwards directly the tempest abated, I determined to obtain a few hours' sleep. Overhead I could accurately distinguish every movement of the jeweller, who, after making the best arrangements in his power for passing a comfortable night, threw himself on his bed, and I could hear it creak and groan beneath his weight. 

 Insensibly my eyelids grew heavy, deep sleep stole over me, and having no suspicion of anything wrong, I sought not to shake it off. I looked into the kitchen once more and saw Caderousse sitting by the side of a long table upon one of the low wooden stools which in country places are frequently used instead of chairs; his back was turned towards me, so that I could not see the expression of his countenance—neither should I have been able to do so had he been placed differently, as his head was buried between his two hands. La Carconte continued to gaze on him for some time, then shrugging her shoulders, she took her seat immediately opposite to him. 

 At this moment the expiring embers threw up a fresh flame from the kindling of a piece of wood that lay near, and a bright light flashed over the room. La Carconte still kept her eyes fixed on her husband, but as he made no sign of changing his position, she extended her hard, bony hand, and touched him on the forehead.  
 
 Caderousse shuddered. The woman's lips seemed to move, as though she were talking; but because she merely spoke in an undertone, or my senses were dulled by sleep, I did not catch a word she uttered. Confused sights and sounds seemed to float before me, and gradually I fell into a deep, heavy slumber. How long I had been in this unconscious state I know not, when I was suddenly aroused by the report of a pistol, followed by a fearful cry. Weak and tottering footsteps resounded across the chamber above me, and the next instant a dull, heavy weight seemed to fall powerless on the staircase. I had not yet fully recovered consciousness, when again I heard groans, mingled with half-stifled cries, as if from persons engaged in a deadly struggle. A cry more prolonged than the others and ending in a series of groans effectually roused me from my drowsy lethargy. Hastily raising myself on one arm, I looked around, but all was dark; and it seemed to me as if the rain must have penetrated through the flooring of the room above, for some kind of moisture appeared to fall, drop by drop, upon my forehead, and when I passed my hand across my brow, I felt that it was wet and clammy. 

 To the fearful noises that had awakened me had succeeded the most perfect silence—unbroken, save by the footsteps of a man walking about in the chamber above. The staircase creaked, he descended into the room below, approached the fire and lit a candle. 

 The man was Caderousse—he was pale and his shirt was all bloody. Having obtained the light, he hurried upstairs again, and once more I heard his rapid and uneasy footsteps. 

 A moment later he came down again, holding in his hand the small shagreen case, which he opened, to assure himself it contained the diamond,—seemed to hesitate as to which pocket he should put it in, then, as if dissatisfied with the security of either pocket, he deposited it in his red handkerchief, which he carefully rolled round his head. 

 After this he took from his cupboard the bank-notes and gold he had put there, thrust the one into the pocket of his trousers, and the other into that of his waistcoat, hastily tied up a small bundle of linen, and rushing towards the door, disappeared in the darkness of the night. 

 Then all became clear and manifest to me, and I reproached myself with what had happened, as though I myself had done the guilty deed. I fancied that I still heard faint moans, and imagining that the unfortunate jeweller might not be quite dead, I determined to go to his relief, by way of atoning in some slight degree, not for the crime I had committed, but for that which I had not endeavoured to prevent. For this purpose I applied all the strength I possessed to force an entrance from the cramped spot in which I lay to the adjoining room. The poorly fastened boards which alone divided me from it yielded to my efforts, and I found myself in the house. Hastily snatching up the lighted candle, I hurried to the staircase; about midway a body was lying quite across the stairs. It was that of La Carconte. The pistol I had heard had doubtless been fired at her. The shot had frightfully lacerated her throat, leaving two gaping wounds from which, as well as the mouth, the blood was pouring in floods. She was stone dead. I strode past her, and ascended to the sleeping chamber, which presented an appearance of the wildest disorder. The furniture had been knocked over in the deadly struggle that had taken place there, and the sheets, to which the unfortunate jeweller had doubtless clung, were dragged across the room. The murdered man lay on the floor, his head leaning against the wall, and about him was a pool of blood which poured forth from three large wounds in his breast; there was a fourth gash, in which a long table knife was plunged up to the handle. 

I stumbled over some object; I stooped to examine—it was the second pistol, which had not gone off, probably from the powder being wet. I approached the jeweller, who was not quite dead, and at the sound of my footsteps and the creaking of the floor, he opened his eyes, fixed them on me with an anxious and inquiring gaze, moved his lips as though trying to speak, then, overcome by the effort, fell back and expired. 

This appalling sight almost bereft me of my senses, and finding that I could no longer be of service to anyone in the house, my only desire was to fly. I rushed towards the staircase, clutching my hair, and uttering a groan of horror. 

Upon reaching the room below, I found five or six custom-house officers, and two or three gendarmes—all heavily armed. They threw themselves upon me. I made no resistance; I was no longer master of my senses. When I strove to speak, a few inarticulate sounds alone escaped my lips. 

As I noticed the significant manner in which the whole party pointed to my blood-stained garments, I involuntarily surveyed myself, and then I discovered that the thick warm drops that had so bedewed me as I lay beneath the staircase must have been the blood of La Carconte. I pointed to the spot where I had concealed myself. 

<What does he mean?> asked a gendarme. 

One of the officers went to the place I directed. 

<He means,> replied the man upon his return, <that he got in that way;> and he showed the hole I had made when I broke through. 

Then I saw that they took me for the assassin. I recovered force and energy enough to free myself from the hands of those who held me, while I managed to stammer forth: 

<I did not do it! Indeed, indeed I did not!> 

A couple of gendarmes held the muzzles of their carbines against my breast. 

<Stir but a step,> said they, <and you are a dead man.> 

<Why should you threaten me with death,> cried I, <when I have already declared my innocence?> 

<Tush, tush,> cried the men; <keep your innocent stories to tell to the judge at Nîmes. Meanwhile, come along with us; and the best advice we can give you is to do so unresistingly.> 

Alas, resistance was far from my thoughts. I was utterly overpowered by surprise and terror; and without a word I suffered myself to be handcuffed and tied to a horse's tail, and thus they took me to Nîmes. 

I had been tracked by a customs-officer, who had lost sight of me near the tavern; feeling certain that I intended to pass the night there, he had returned to summon his comrades, who just arrived in time to hear the report of the pistol, and to take me in the midst of such circumstantial proofs of my guilt as rendered all hopes of proving my innocence utterly futile. One only chance was left me, that of beseeching the magistrate before whom I was taken to cause every inquiry to be made for the Abbé Busoni, who had stopped at the inn of the Pont du Gard on that morning. 

If Caderousse had invented the story relative to the diamond, and there existed no such person as the Abbé Busoni, then, indeed, I was lost past redemption, or, at least, my life hung upon the feeble chance of Caderousse himself being apprehended and confessing the whole truth. 

Two months passed away in hopeless expectation on my part, while I must do the magistrate the justice to say that he used every means to obtain information of the person I declared could exculpate me if he would. Caderousse still evaded all pursuit, and I had resigned myself to what seemed my inevitable fate. My trial was to come on at the approaching assizes; when, on the 8th of September—that is to say, precisely three months and five days after the events which had perilled my life—the Abbé Busoni, whom I never ventured to believe I should see, presented himself at the prison doors, saying he understood one of the prisoners wished to speak to him; he added, that having learned at Marseilles the particulars of my imprisonment, he hastened to comply with my desire. 

You may easily imagine with what eagerness I welcomed him, and how minutely I related the whole of what I had seen and heard. I felt some degree of nervousness as I entered upon the history of the diamond, but, to my inexpressible astonishment, he confirmed it in every particular, and to my equal surprise, he seemed to place entire belief in all I said. 

And then it was that, won by his mild charity, seeing that he was acquainted with all the habits and customs of my own country, and considering also that pardon for the only crime of which I was really guilty might come with a double power from lips so benevolent and kind, I besought him to receive my confession, under the seal of which I recounted the Auteuil affair in all its details, as well as every other transaction of my life. That which I had done by the impulse of my best feelings produced the same effect as though it had been the result of calculation. My voluntary confession of the assassination at Auteuil proved to him that I had not committed that of which I stood accused. When he quitted me, he bade me be of good courage, and to rely upon his doing all in his power to convince my judges of my innocence. 

I had speedy proofs that the excellent abbé was engaged in my behalf, for the rigors of my imprisonment were alleviated by many trifling though acceptable indulgences, and I was told that my trial was to be postponed to the assizes following those now being held. 

 In the interim it pleased Providence to cause the apprehension of Caderousse, who was discovered in some distant country, and brought back to France, where he made a full confession, refusing to make the fact of his wife's having suggested and arranged the murder any excuse for his own guilt. The wretched man was sentenced to the galleys for life, and I was immediately set at liberty.' 
 %manual close quote--opening quote is in the lettrine.

 <And then it was, I presume,> said Monte Cristo <that you came to me as the bearer of a letter from the Abbé Busoni?> 

\enquote{It was, your excellency; the benevolent abbé took an evident interest in all that concerned me. 

<Your mode of life as a smuggler,> said he to me one day, <will be the ruin of you; if you get out, don't take it up again.> 

<But how,> inquired I, <am I to maintain myself and my poor sister?> 

<A person, whose confessor I am,> replied he, <and who entertains a high regard for me, applied to me a short time since to procure him a confidential servant. Would you like such a post? If so, I will give you a letter of introduction to him.> 

<Oh, father,> I exclaimed, <you are very good.> 

<But you must swear solemnly that I shall never have reason to repent my recommendation.> 

I extended my hand, and was about to pledge myself by any promise he would dictate, but he stopped me. 

<It is unnecessary for you to bind yourself by any vow,> said he; <I know and admire the Corsican nature too well to fear you. Here, take this,> continued he, after rapidly writing the few lines I brought to your excellency, and upon receipt of which you deigned to receive me into your service, and proudly I ask whether your excellency has ever had cause to repent having done so?}

 <No,> replied the count; <I take pleasure in saying that you have served me faithfully, Bertuccio; but you might have shown more confidence in me.> 

 <I, your excellency?> 

 <Yes; you. How comes it, that having both a sister and an adopted son, you have never spoken to me of either?>  
 
 \enquote{Alas, I have still to recount the most distressing period of my life. Anxious as you may suppose I was to behold and comfort my dear sister, I lost no time in hastening to Corsica, but when I arrived at Rogliano I found a house of mourning, the consequences of a scene so horrible that the neighbours remember and speak of it to this day. Acting by my advice, my poor sister had refused to comply with the unreasonable demands of Benedetto, who was continually tormenting her for money, as long as he believed there was a sou left in her possession. One morning he threatened her with the severest consequences if she did not supply him with what he desired, and disappeared and remained away all day, leaving the kind-hearted Assunta, who loved him as if he were her own child, to weep over his conduct and bewail his absence. Evening came, and still, with all the patient solicitude of a mother, she watched for his return. 

As the eleventh hour struck, he entered with a swaggering air, attended by two of the most dissolute and reckless of his boon companions. She stretched out her arms to him, but they seized hold of her, and one of the three—none other than the accursed Benedetto exclaimed: 

<Put her to torture and she'll soon tell us where her money is.> 

It unfortunately happened that our neighbour, Wasilio, was at Bastia, leaving no person in his house but his wife; no human creature beside could hear or see anything that took place within our dwelling. Two held poor Assunta, who, unable to conceive that any harm was intended to her, smiled in the face of those who were soon to become her executioners. The third proceeded to barricade the doors and windows, then returned, and the three united in stifling the cries of terror incited by the sight of these preparations, and then dragged Assunta feet foremost towards the brazier, expecting to wring from her an avowal of where her supposed treasure was secreted. In the struggle her clothes caught fire, and they were obliged to let go their hold in order to preserve themselves from sharing the same fate. Covered with flames, Assunta rushed wildly to the door, but it was fastened; she flew to the windows, but they were also secured; then the neighbours heard frightful shrieks; it was Assunta calling for help. The cries died away in groans, and next morning, as soon as Wasilio's wife could muster up courage to venture abroad, she caused the door of our dwelling to be opened by the public authorities, when Assunta, although dreadfully burnt, was found still breathing; every drawer and closet in the house had been forced open, and the money stolen. Benedetto never again appeared at Rogliano, neither have I since that day either seen or heard anything concerning him. 

It was subsequently to these dreadful events that I waited on your excellency, to whom it would have been folly to have mentioned Benedetto, since all trace of him seemed entirely lost; or of my sister, since she was dead.}

 <And in what light did you view the occurrence?> inquired Monte Cristo. 

 <As a punishment for the crime I had committed,> answered Bertuccio. <Oh, those Villeforts are an accursed race!> 

 <Truly they are,> murmured the count in a lugubrious tone. 

 <And now,> resumed Bertuccio, <your excellency may, perhaps, be able to comprehend that this place, which I revisit for the first time—this garden, the actual scene of my crime—must have given rise to reflections of no very agreeable nature, and produced that gloom and depression of spirits which excited the notice of your excellency, who was pleased to express a desire to know the cause. At this instant a shudder passes over me as I reflect that possibly I am now standing on the very grave in which lies M. de Villefort, by whose hand the ground was dug to receive the corpse of his child.> 

 <Everything is possible,> said Monte Cristo, rising from the bench on which he had been sitting; <even,> he added in an inaudible voice, <even that the procureur be not dead. The Abbé Busoni did right to send you to me,> he went on in his ordinary tone, <and you have done well in relating to me the whole of your history, as it will prevent my forming any erroneous opinions concerning you in future. As for that Benedetto, who so grossly belied his name, have you never made any effort to trace out whither he has gone, or what has become of him?> 

 <No; far from wishing to learn whither he has betaken himself, I should shun the possibility of meeting him as I would a wild beast. Thank God, I have never heard his name mentioned by any person, and I hope and believe he is dead.> 

 <Do not think so, Bertuccio,> replied the count; <for the wicked are not so easily disposed of, for God seems to have them under his special watch-care to make of them instruments of his vengeance.> 

 <So be it,> responded Bertuccio, <all I ask of heaven is that I may never see him again. And now, your excellency,> he added, bowing his head, <you know everything—you are my judge on earth, as the Almighty is in heaven; have you for me no words of consolation?> 

 <My good friend, I can only repeat the words addressed to you by the Abbé Busoni. Villefort merited punishment for what he had done to you, and, perhaps, to others. Benedetto, if still living, will become the instrument of divine retribution in some way or other, and then be duly punished in his turn. As far as you yourself are concerned, I see but one point in which you are really guilty. Ask yourself, wherefore, after rescuing the infant from its living grave, you did not restore it to its mother? There was the crime, Bertuccio—that was where you became really culpable.> 

 <True, excellency, that was the crime, the real crime, for in that I acted like a coward. My first duty, directly I had succeeded in recalling the babe to life, was to restore it to its mother; but, in order to do so, I must have made close and careful inquiry, which would, in all probability, have led to my own apprehension; and I clung to life, partly on my sister's account, and partly from that feeling of pride inborn in our hearts of desiring to come off untouched and victorious in the execution of our vengeance. Perhaps, too, the natural and instinctive love of life made me wish to avoid endangering my own. And then, again, I am not as brave and courageous as was my poor brother.> 

 Bertuccio hid his face in his hands as he uttered these words, while Monte Cristo fixed on him a look of inscrutable meaning. After a brief silence, rendered still more solemn by the time and place, the count said, in a tone of melancholy wholly unlike his usual manner: 

 <In order to bring this conversation to a fitting termination (the last we shall ever hold upon this subject), I will repeat to you some words I have heard from the lips of the Abbé Busoni. For all evils there are two remedies—time and silence. And now leave me, Monsieur Bertuccio, to walk alone here in the garden. The very circumstances which inflict on you, as a principal in the tragic scene enacted here, such painful emotions, are to me, on the contrary, a source of something like contentment, and serve but to enhance the value of this dwelling in my estimation. The chief beauty of trees consists in the deep shadow of their umbrageous boughs, while fancy pictures a moving multitude of shapes and forms flitting and passing beneath that shade. Here I have a garden laid out in such a way as to afford the fullest scope for the imagination, and furnished with thickly grown trees, beneath whose leafy screen a visionary like myself may conjure up phantoms at will. This to me, who expected but to find a blank enclosure surrounded by a straight wall, is, I assure you, a most agreeable surprise. I have no fear of ghosts, and I have never heard it said that so much harm had been done by the dead during six thousand years as is wrought by the living in a single day. Retire within, Bertuccio, and tranquillize your mind. Should your confessor be less indulgent to you in your dying moments than you found the Abbé Busoni, send for me, if I am still on earth, and I will soothe your ears with words that shall effectually calm and soothe your parting soul ere it goes forth to traverse the ocean called eternity.> 

 Bertuccio bowed respectfully, and turned away, sighing heavily. Monte Cristo, left alone, took three or four steps onwards, and murmured: 

 <Here, beneath this plane-tree, must have been where the infant's grave was dug. There is the little door opening into the garden. At this corner is the private staircase communicating with the sleeping apartment. There will be no necessity for me to make a note of these particulars, for there, before my eyes, beneath my feet, all around me, I have the plan sketched with all the living reality of truth.> 

 After making the tour of the garden a second time, the count re-entered his carriage, while Bertuccio, who perceived the thoughtful expression of his master's features, took his seat beside the driver without uttering a word. The carriage proceeded rapidly towards Paris. 

 That same evening, upon reaching his abode in the Champs-Élysées, the Count of Monte Cristo went over the whole building with the air of one long acquainted with each nook or corner. Nor, although preceding the party, did he once mistake one door for another, or commit the smallest error when choosing any particular corridor or staircase to conduct him to a place or suite of rooms he desired to visit. Ali was his principal attendant during this nocturnal survey. Having given various orders to Bertuccio relative to the improvements and alterations he desired to make in the house, the Count, drawing out his watch, said to the attentive Nubian: 

 <It is half-past eleven o'clock; Haydée will soon be here. Have the French attendants been summoned to await her coming?> 

 Ali extended his hands towards the apartments destined for the fair Greek, which were so effectually concealed by means of a tapestried entrance, that it would have puzzled the most curious to have divined their existence. Ali, having pointed to the apartments, held up three fingers of his right hand, and then, placing it beneath his head, shut his eyes, and feigned to sleep. 

 <I understand,> said Monte Cristo, well acquainted with Ali's pantomime; <you mean to tell me that three female attendants await their new mistress in her sleeping-chamber.> 

 Ali, with considerable animation, made a sign in the affirmative. 

 <Madame will be tired tonight,> continued Monte Cristo, <and will, no doubt, wish to rest. Desire the French attendants not to weary her with questions, but merely to pay their respectful duty and retire. You will also see that the Greek servants hold no communication with those of this country.> 

 He bowed. Just at that moment voices were heard hailing the concierge. The gate opened, a carriage rolled down the avenue, and stopped at the steps. The count hastily descended, presented himself at the already opened carriage door, and held out his hand to a young woman, completely enveloped in a green silk mantle heavily embroidered with gold. She raised the hand extended towards her to her lips, and kissed it with a mixture of love and respect. Some few words passed between them in that sonorous language in which Homer makes his gods converse. The young woman spoke with an expression of deep tenderness, while the count replied with an air of gentle gravity. 

 Preceded by Ali, who carried a rose-coloured flambeau in his hand, the young lady, who was no other than the lovely Greek who had been Monte Cristo's companion in Italy, was conducted to her apartments, while the count retired to the pavilion reserved for himself. In another hour every light in the house was extinguished, and it might have been thought that all its inmates slept. 