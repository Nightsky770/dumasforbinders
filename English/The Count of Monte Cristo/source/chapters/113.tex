\chapter{The Past} 

 \lettrine{T}{he} count departed with a sad heart from the house in which he had left Mercédès, probably never to behold her again. Since the death of little Edward a great change had taken place in Monte Cristo. Having reached the summit of his vengeance by a long and tortuous path, he saw an abyss of doubt yawning before him. More than this, the conversation which had just taken place between Mercédès and himself had awakened so many recollections in his heart that he felt it necessary to combat with them. A man of the count's temperament could not long indulge in that melancholy which can exist in common minds, but which destroys superior ones. He thought he must have made an error in his calculations if he now found cause to blame himself. 

 <I cannot have deceived myself,> he said; <I must look upon the past in a false light. What!> he continued, <can I have been following a false path?—can the end which I proposed be a mistaken end?—can one hour have sufficed to prove to an architect that the work upon which he founded all his hopes was an impossible, if not a sacrilegious, undertaking? I cannot reconcile myself to this idea—it would madden me. The reason why I am now dissatisfied is that I have not a clear appreciation of the past. The past, like the country through which we walk, becomes indistinct as we advance. My position is like that of a person wounded in a dream; he feels the wound, though he cannot recollect when he received it. 

Come, then, thou regenerate man, thou extravagant prodigal, thou awakened sleeper, thou all-powerful visionary, thou invincible millionaire,—once again review thy past life of starvation and wretchedness, revisit the scenes where fate and misfortune conducted, and where despair received thee. Too many diamonds, too much gold and splendour, are now reflected by the mirror in which Monte Cristo seeks to behold Dantès. Hide thy diamonds, bury thy gold, shroud thy splendour, exchange riches for poverty, liberty for a prison, a living body for a corpse!> 

 As he thus reasoned, Monte Cristo walked down the Rue de la Caisserie. It was the same through which, twenty-four years ago, he had been conducted by a silent and nocturnal guard; the houses, today so smiling and animated, were on that night dark, mute, and closed. 

 <And yet they were the same,> murmured Monte Cristo, <only now it is broad daylight instead of night; it is the sun which brightens the place, and makes it appear so cheerful.> 

 He proceeded towards the quay by the Rue Saint-Laurent, and advanced to the Consigne; it was the point where he had embarked. A pleasure-boat with striped awning was going by. Monte Cristo called the owner, who immediately rowed up to him with the eagerness of a boatman hoping for a good fare. 

 The weather was magnificent, and the excursion a treat. The sun, red and flaming, was sinking into the embrace of the welcoming ocean. The sea, smooth as crystal, was now and then disturbed by the leaping of fish, which were pursued by some unseen enemy and sought for safety in another element; while on the extreme verge of the horizon might be seen the fishermen's boats, white and graceful as the sea-gull, or the merchant vessels bound for Corsica or Spain. 

 But notwithstanding the serene sky, the gracefully formed boats, and the golden light in which the whole scene was bathed, the Count of Monte Cristo, wrapped in his cloak, could think only of this terrible voyage, the details of which were one by one recalled to his memory. The solitary light burning at the Catalans; that first sight of the Château d'If, which told him whither they were leading him; the struggle with the gendarmes when he wished to throw himself overboard; his despair when he found himself vanquished, and the sensation when the muzzle of the carbine touched his forehead—all these were brought before him in vivid and frightful reality. 

 Like the streams which the heat of the summer has dried up, and which after the autumnal storms gradually begin oozing drop by drop, so did the count feel his heart gradually fill with the bitterness which formerly nearly overwhelmed Edmond Dantès. Clear sky, swift-flitting boats, and brilliant sunshine disappeared; the heavens were hung with black, and the gigantic structure of the Château d'If seemed like the phantom of a mortal enemy. As they reached the shore, the count instinctively shrunk to the extreme end of the boat, and the owner was obliged to call out, in his sweetest tone of voice: 

 <Sir, we are at the landing.> 

 Monte Cristo remembered that on that very spot, on the same rock, he had been violently dragged by the guards, who forced him to ascend the slope at the points of their bayonets. The journey had seemed very long to Dantès, but Monte Cristo found it equally short. Each stroke of the oar seemed to awaken a new throng of ideas, which sprang up with the flying spray of the sea.  There had been no prisoners confined in the Château d'If since the revolution of July; it was only inhabited by a guard, kept there for the prevention of smuggling. A concierge waited at the door to exhibit to visitors this monument of curiosity, once a scene of terror. 

 The count inquired whether any of the ancient jailers were still there; but they had all been pensioned, or had passed on to some other employment. The concierge who attended him had only been there since 1830. He visited his own dungeon. He again beheld the dull light vainly endeavouring to penetrate the narrow opening. His eyes rested upon the spot where had stood his bed, since then removed, and behind the bed the new stones indicated where the breach made by the Abbé Faria had been. Monte Cristo felt his limbs tremble; he seated himself upon a log of wood. 

 <Are there any stories connected with this prison besides the one relating to the poisoning of Mirabeau?> asked the count; <are there any traditions respecting these dismal abodes,—in which it is difficult to believe men can ever have imprisoned their fellow-creatures?> 

 <Yes, sir; indeed, the jailer Antoine told me one connected with this very dungeon.> 

 Monte Cristo shuddered; Antoine had been his jailer. He had almost forgotten his name and face, but at the mention of the name he recalled his person as he used to see it, the face encircled by a beard, wearing the brown jacket, the bunch of keys, the jingling of which he still seemed to hear. The count turned around, and fancied he saw him in the corridor, rendered still darker by the torch carried by the concierge. 

 <Would you like to hear the story, sir?> 

 <Yes; relate it,> said Monte Cristo, pressing his hand to his heart to still its violent beatings; he felt afraid of hearing his own history. 

 <This dungeon,> said the concierge, <was, it appears, some time ago occupied by a very dangerous prisoner, the more so since he was full of industry. Another person was confined in the Château at the same time, but he was not wicked, he was only a poor mad priest.> 

 <Ah, indeed?—mad!> repeated Monte Cristo; <and what was his mania?> 

 <He offered millions to anyone who would set him at liberty.> 

 Monte Cristo raised his eyes, but he could not see the heavens; there was a stone veil between him and the firmament. He thought that there had been no less thick a veil before the eyes of those to whom Faria offered the treasures. 

 <Could the prisoners see each other?> he asked. 

 <Oh, no, sir, it was expressly forbidden; but they eluded the vigilance of the guards, and made a passage from one dungeon to the other.> 

 <And which of them made this passage?> 

 <Oh, it must have been the young man, certainly, for he was strong and industrious, while the abbé was aged and weak; besides, his mind was too vacillating to allow him to carry out an idea.> 

 <Blind fools!> murmured the count. 

 <However, be that as it may, the young man made a tunnel, how or by what means no one knows; but he made it, and there is the evidence yet remaining of his work. Do you see it?> and the man held the torch to the wall.  <Ah, yes; I see,> said the count, in a voice hoarse from emotion. 

 <The result was that the two men communicated with one another; how long they did so, nobody knows. One day the old man fell ill and died. Now guess what the young one did?> 

 <Tell me.> 

 <He carried off the corpse, which he placed in his own bed with its face to the wall; then he entered the empty dungeon, closed the entrance, and slipped into the sack which had contained the dead body. Did you ever hear of such an idea?> 

 Monte Cristo closed his eyes, and seemed again to experience all the sensations he had felt when the coarse canvas, yet moist with the cold dews of death, had touched his face. 

 The jailer continued: 

 <Now this was his project. He fancied that they buried the dead at the Château d'If, and imagining they would not expend much labour on the grave of a prisoner, he calculated on raising the earth with his shoulders, but unfortunately their arrangements at the Château frustrated his projects. They never buried the dead; they merely attached a heavy cannon-ball to the feet, and then threw them into the sea. This is what was done. The young man was thrown from the top of the rock; the corpse was found on the bed next day, and the whole truth was guessed, for the men who performed the office then mentioned what they had not dared to speak of before, that at the moment the corpse was thrown into the deep, they heard a shriek, which was almost immediately stifled by the water in which it disappeared.> 

 The count breathed with difficulty; the cold drops ran down his forehead, and his heart was full of anguish. 

 <No,> he muttered, <the doubt I felt was but the commencement of forgetfulness; but here the wound reopens, and the heart again thirsts for vengeance. And the prisoner,> he continued aloud, <was he ever heard of afterwards?> 

 <Oh, no; of course not. You can understand that one of two things must have happened; he must either have fallen flat, in which case the blow, from a height of ninety feet, must have killed him instantly, or he must have fallen upright, and then the weight would have dragged him to the bottom, where he remained—poor fellow!> 

 <Then you pity him?> said the count. 

 <\textit{Ma foi}, yes; though he was in his own element.> 

 <What do you mean?> 

 <The report was that he had been a naval officer, who had been confined for plotting with the Bonapartists.> 

 <Great is truth,> muttered the count, <fire cannot burn, nor water drown it! Thus the poor sailor lives in the recollection of those who narrate his history; his terrible story is recited in the chimney-corner, and a shudder is felt at the description of his transit through the air to be swallowed by the deep.> Then, the count added aloud, <Was his name ever known?> 

 <Oh, yes; but only as № 34.> 

 <Oh, Villefort, Villefort,> murmured the count, <this scene must often have haunted thy sleepless hours!> 

 <Do you wish to see anything more, sir?> said the concierge. 

 <Yes, especially if you will show me the poor abbé's room.> 

 <Ah! № 27.> 

 <Yes; № 27.> repeated the count, who seemed to hear the voice of the abbé answering him in those very words through the wall when asked his name. 

 <Come, sir.> 

 <Wait,> said Monte Cristo, <I wish to take one final glance around this room.> 

 <This is fortunate,> said the guide; <I have forgotten the other key.> 

 <Go and fetch it.> 

 <I will leave you the torch, sir.> 

 <No, take it away; I can see in the dark.> 

 <Why, you are like № 34. They said he was so accustomed to darkness that he could see a pin in the darkest corner of his dungeon.> 

 <He spent fourteen years to arrive at that,> muttered the count. 

 The guide carried away the torch. The count had spoken correctly. Scarcely had a few seconds elapsed, ere he saw everything as distinctly as by daylight. Then he looked around him, and really recognized his dungeon. 

 <Yes,> he said, <there is the stone upon which I used to sit; there is the impression made by my shoulders on the wall; there is the mark of my blood made when one day I dashed my head against the wall. Oh, those figures, how well I remember them! I made them one day to calculate the age of my father, that I might know whether I should find him still living, and that of Mercédès, to know if I should find her still free. After finishing that calculation, I had a minute's hope. I did not reckon upon hunger and infidelity!> and a bitter laugh escaped the count. 

 He saw in fancy the burial of his father, and the marriage of Mercédès. On the other side of the dungeon he perceived an inscription, the white letters of which were still visible on the green wall: 

 <<\textit{Oh, God!}>> he read, <<\textit{preserve my memory!}>> 

 <Oh, yes,> he cried, <that was my only prayer at last; I no longer begged for liberty, but memory; I dreaded to become mad and forgetful. Oh, God, thou hast preserved my memory; I thank thee, I thank thee!> 

 At this moment the light of the torch was reflected on the wall; the guide was coming; Monte Cristo went to meet him. 

 <Follow me, sir;> and without ascending the stairs the guide conducted him by a subterraneous passage to another entrance. There, again, Monte Cristo was assailed by a multitude of thoughts. The first thing that met his eye was the meridian, drawn by the abbé on the wall, by which he calculated the time; then he saw the remains of the bed on which the poor prisoner had died. The sight of this, instead of exciting the anguish experienced by the count in the dungeon, filled his heart with a soft and grateful sentiment, and tears fell from his eyes. 

 <This is where the mad abbé was kept, sir, and that is where the young man entered;> and the guide pointed to the opening, which had remained unclosed. <From the appearance of the stone,> he continued, <a learned gentleman discovered that the prisoners might have communicated together for ten years. Poor things! Those must have been ten weary years.> 

 Dantès took some louis from his pocket, and gave them to the man who had twice unconsciously pitied him. The guide took them, thinking them merely a few pieces of little value; but the light of the torch revealed their true worth. 

 <Sir,> he said, <you have made a mistake; you have given me gold.> 

 <I know it.> 

 The concierge looked upon the count with surprise. 

 <Sir,> he cried, scarcely able to believe his good fortune—<sir, I cannot understand your generosity!> 

 <Oh, it is very simple, my good fellow; I have been a sailor, and your story touched me more than it would others.> 

 <Then, sir, since you are so liberal, I ought to offer you something.>

<What have you to offer to me, my friend? Shells? Straw-work? Thank you!> 

 <No, sir, neither of those; something connected with this story.> 

 <Really? What is it?> 

 <Listen,> said the guide; <I said to myself, <Something is always left in a cell inhabited by one prisoner for fifteen years,> so I began to sound the wall.> 

 <Ah,> cried Monte Cristo, remembering the abbé's two hiding-places. 

 <After some search, I found that the floor gave a hollow sound near the head of the bed, and at the hearth.> 

 <Yes,> said the count, <yes.> 

 <I raised the stones, and found\longdash> 

 <A rope-ladder and some tools?> 

 <How do you know that?> asked the guide in astonishment. 

 <I do not know—I only guess it, because that sort of thing is generally found in prisoners' cells.> 

 <Yes, sir, a rope-ladder and tools.> 

 <And have you them yet?> 

 <No, sir; I sold them to visitors, who considered them great curiosities; but I have still something left.> 

 <What is it?> asked the count, impatiently. 

 <A sort of book, written upon strips of cloth.> 

 <Go and fetch it, my good fellow; and if it be what I hope, you will do well.> 

 <I will run for it, sir;> and the guide went out. 

 Then the count knelt down by the side of the bed, which death had converted into an altar. 

 <Oh, second father,> he exclaimed, <thou who hast given me liberty, knowledge, riches; thou who, like beings of a superior order to ourselves, couldst understand the science of good and evil; if in the depths of the tomb there still remain something within us which can respond to the voice of those who are left on earth; if after death the soul ever revisit the places where we have lived and suffered,—then, noble heart, sublime soul, then I conjure thee by the paternal love thou didst bear me, by the filial obedience I vowed to thee, grant me some sign, some revelation! Remove from me the remains of doubt, which, if it change not to conviction, must become remorse!> The count bowed his head, and clasped his hands together. 

 <Here, sir,> said a voice behind him. 

 Monte Cristo shuddered, and arose. The concierge held out the strips of cloth upon which the Abbé Faria had spread the riches of his mind. The manuscript was the great work by the Abbé Faria upon the kingdoms of Italy. The count seized it hastily, his eyes immediately fell upon the epigraph, and he read: 

 <Thou shalt tear out the dragons' teeth, and shall trample the lions under foot, saith the Lord.> 

 <Ah,> he exclaimed, <here is my answer. Thanks, father, thanks.> And feeling in his pocket, he took thence a small pocket-book, which contained ten bank-notes, each of 1,000 francs. 

 <Here,> he said, <take this pocket-book.> 

 <Do you give it to me?> 

 <Yes; but only on condition that you will not open it till I am gone;> and placing in his breast the treasure he had just found, which was more valuable to him than the richest jewel, he rushed out of the corridor, and reaching his boat, cried, <To Marseilles!> 

 Then, as he departed, he fixed his eyes upon the gloomy prison. 

 <Woe,> he cried, <to those who confined me in that wretched prison; and woe to those who forgot that I was there!>  As he repassed the Catalans, the count turned around and burying his head in his cloak murmured the name of a woman. The victory was complete; twice he had overcome his doubts. The name he pronounced, in a voice of tenderness, amounting almost to love, was that of Haydée. 

 On landing, the count turned towards the cemetery, where he felt sure of finding Morrel. He, too, ten years ago, had piously sought out a tomb, and sought it vainly. He, who returned to France with millions, had been unable to find the grave of his father, who had perished from hunger. Morrel had indeed placed a cross over the spot, but it had fallen down and the grave-digger had burnt it, as he did all the old wood in the churchyard. 

 The worthy merchant had been more fortunate. Dying in the arms of his children, he had been by them laid by the side of his wife, who had preceded him in eternity by two years. Two large slabs of marble, on which were inscribed their names, were placed on either side of a little enclosure, railed in, and shaded by four cypress-trees. Morrel was leaning against one of these, mechanically fixing his eyes on the graves. His grief was so profound that he was nearly unconscious.  <Maximilian,> said the count, <you should not look on the graves, but there;> and he pointed upwards. 

 <The dead are everywhere,> said Morrel; <did you not yourself tell me so as we left Paris?> 

 <Maximilian,> said the count, <you asked me during the journey to allow you to remain some days at Marseilles. Do you still wish to do so?> 

 <I have no wishes, count; only I fancy I could pass the time less painfully here than anywhere else.> 

 <So much the better, for I must leave you; but I carry your word with me, do I not?> 

 <Ah, count, I shall forget it.> 

 <No, you will not forget it, because you are a man of honour, Morrel, because you have taken an oath, and are about to do so again.> 

 <Oh, count, have pity upon me. I am so unhappy.> 

 <I have known a man much more unfortunate than you, Morrel.> 

 <Impossible!> 

 <Alas,> said Monte Cristo, <it is the infirmity of our nature always to believe ourselves much more unhappy than those who groan by our sides!> 

 <What can be more wretched than the man who has lost all he loved and desired in the world?> 

 <Listen, Morrel, and pay attention to what I am about to tell you. I knew a man who like you had fixed all his hopes of happiness upon a woman. He was young, he had an old father whom he loved, a betrothed bride whom he adored. He was about to marry her, when one of the caprices of fate,—which would almost make us doubt the goodness of Providence, if that Providence did not afterwards reveal itself by proving that all is but a means of conducting to an end,—one of those caprices deprived him of his mistress, of the future of which he had dreamed (for in his blindness he forgot he could only read the present), and cast him into a dungeon.> 

 <Ah,> said Morrel, <one quits a dungeon in a week, a month, or a year.> 

 <He remained there fourteen years, Morrel,> said the count, placing his hand on the young man's shoulder. Maximilian shuddered. 

 <Fourteen years!> he muttered. 

 <Fourteen years!> repeated the count. <During that time he had many moments of despair. He also, Morrel, like you, considered himself the unhappiest of men.> 

 <Well?> asked Morrel. 

 <Well, at the height of his despair God assisted him through human means. At first, perhaps, he did not recognize the infinite mercy of the Lord, but at last he took patience and waited. One day he miraculously left the prison, transformed, rich, powerful. His first cry was for his father; but that father was dead.> 

 <My father, too, is dead,> said Morrel. 

 <Yes; but your father died in your arms, happy, respected, rich, and full of years; his father died poor, despairing, almost doubtful of Providence; and when his son sought his grave ten years afterwards, his tomb had disappeared, and no one could say, <There sleeps the father you so well loved.>> 

 <Oh!> exclaimed Morrel. 

 <He was, then, a more unhappy son than you, Morrel, for he could not even find his father's grave.> 

 <But then he had the woman he loved still remaining?> 

 <You are deceived, Morrel, that woman\longdash> 

 <She was dead?> 

 <Worse than that, she was faithless, and had married one of the persecutors of her betrothed. You see, then, Morrel, that he was a more unhappy lover than you.> 

 <And has he found consolation?> 

 <He has at least found peace.> 

 <And does he ever expect to be happy?> 

 <He hopes so, Maximilian.> 

 The young man's head fell on his breast. 

 <You have my promise,> he said, after a minute's pause, extending his hand to Monte Cristo. <Only remember\longdash> 

 <On the 5th of October, Morrel, I shall expect you at the Island of Monte Cristo. On the 4th a yacht will wait for you in the port of Bastia, it will be called the \textit{Eurus}. You will give your name to the captain, who will bring you to me. It is understood—is it not?> 

 <But, count, do you remember that the 5th of October\longdash> 

 <Child,> replied the count, <not to know the value of a man's word! I have told you twenty times that if you wish to die on that day, I will assist you. Morrel, farewell!> 

 <Do you leave me?> 

 <Yes; I have business in Italy. I leave you alone in your struggle with misfortune—alone with that strong-winged eagle which God sends to bear aloft the elect to his feet. The story of Ganymede, Maximilian, is not a fable, but an allegory.> 

 <When do you leave?> 

 <Immediately; the steamer waits, and in an hour I shall be far from you. Will you accompany me to the harbour, Maximilian?>

<I am entirely yours, count.> 

 Morrel accompanied the count to the harbour. The white steam was ascending like a plume of feathers from the black chimney. The steamer soon disappeared, and in an hour afterwards, as the count had said, was scarcely distinguishable in the horizon amidst the fogs of the night. 